\chapter{Машини на Тюринг}

\newcommand{\tape}[1]{\dots\blank\blank\blank{#1}\blank\blank\blank\dots}

Машина на Тюринг ще наричаме седморка от вида $\M = \TM$, където:
\marginpar{Sipser дава малко по-различна дефиниция - с едно финално приемащо и едно финално отхвърлящо.}
\begin{itemize}
\item 
  $Q$ - състояния;
\item
  $\Sigma$ - азбука за входа;
\item
  $\Gamma$ - азбука за лентата, $\Sigma \subseteq \Gamma$;
\item
  $\delta:Q\times\Gamma \to Q\times \Gamma \times \{L,R,N\}$ - (частична) функция на преходите;
\item
  $s$ - начално състояние, $s \in Q$;
\item
  $\blank$ - празен символ,  $\blank \in \Gamma \setminus \Sigma$;
\item
  $F$ - финални състояния, $F \subseteq Q$.
\item 
  Първоначално, лентата съдържа входа, който е обграден от безкрайно много символи $\blank$ и в двете посоки.
\item
  МТ се намира в началното състояние $s$ и главата е върху най-левия символ от входа.
\end{itemize}

{\em Моментно описание} на едно изчисление на МТ е тройка от вида $(\alpha, q, \beta) \in \Gamma^\star\times Q \times \Gamma^\star$. Това означава, че
машината се намира в състояние $q$ и лентата има вида
\[\tape{\alpha\beta},\]
като главата на мишната е поставена върху първия символ на $\beta$.

Както за автомати, удобно е да дефинираме бинарна релация $\vdash_\M$,
която ще казва как моментното състояние на машината $\M$ се променя при 
изпълнение на една стъпка от изчислението.
\begin{itemize}
\item
  Ако $\delta_\M(q,Z) = (p,Y,R)$, то пишем
  $(\alpha, q, Z\beta) \vdash_\M (\alpha Y, p, \beta)$.
  При $Z = \blank$, също така можем да запишем 
  $(\alpha, q, \varepsilon) \vdash_\M (\alpha Y, p, \varepsilon)$
\item 
  Ако $\delta_\M(q,Z) = (p,Y,L)$, то пишем
  $(\alpha X, q, Z\beta) \vdash_\M (\alpha , p, XY\beta)$.
  При $X = \blank$, $(\varepsilon, q, Z\beta) \vdash_\M (\varepsilon, p, \blank Y\beta)$.
\end{itemize}
С $\vdash^\star_\M$ ще означаваме рефлексивното и транзитивно затваряне на $\vdash_\M$.

Езикът, който се {\bf разпознава чрез финални състояния} от машината $M$ е:
\[\L_F(\M) = \{\alpha\in\Sigma^\star \mid (\varepsilon, s, \alpha) \vdash^\star (\beta, q, \gamma)\ \&\ q\in F\ \&\ \beta,\gamma\in\Gamma^\star\}.\]

\marginpar{Това трябва ли ми ?}
Езикът, който се {\bf разпознава чрез спиране} от $M$ е:
\[\L_H(\M) = \{\alpha \in \Sigma^\star \mid (\varepsilon, s, \alpha) \vdash^\star (\beta, q, X\gamma)\ \&\ \neg !\delta(q,X)\}\]

Може да се докаже, че се разпознават едни и същи езици.

\index{език!полуразрешим}
Езиците, които се разпознават от МТ се наричат {\bf полуразрешими езици}.
Един език $L$ се нарича разрешим, ако за него съществува $\M$, $L = \L_F(\M)$
и освен това $\M$ завършва върху всички входни думи.

\section{Примери}

\subsection*{Удвояване на броя на единиците}

\[\stackrel{1}{1}11\ \Rightarrow\ \stackrel{2}{\blank} 111\ \Rightarrow\ \stackrel{3}{\blank}\blank 111\ \Rightarrow\ \stackrel{4}{\blank}1\blank 111\ \Rightarrow\ 1\stackrel{5}{1}\blank 111\ \Rightarrow\  11\stackrel{5}{\blank}111\ \Rightarrow\  11 \blank \stackrel{6}{1}11\]
\[\dots \Rightarrow\ 11\blank 111\stackrel{6}{\blank}\ \Rightarrow\ 11\blank 11\stackrel{7}{1}\blank\ \Rightarrow\ 11\blank 1\stackrel{8}{1}\blank\blank\ \Rightarrow\ 11\blank \stackrel{9}{1}1\blank\blank\ \Rightarrow\  1\stackrel{10}{1}\blank 11\blank\blank\]
\[\dots \Rightarrow\ \stackrel{10}{\blank}11\blank 11\blank\blank\ \Rightarrow\ \stackrel{2}{1}1\blank 11\blank\blank\ \Rightarrow\ \cdots\]

\begin{figure}[H]
  \begin{center}
    \begin{tikzpicture}[->,>=stealth,thick,node distance=45pt]
      \tikzstyle{every state}=[circle,minimum size=10pt,auto,scale=.7]
      
      \node[state]   (1) {$1$};
      \node[state]            (2) [right of=1]{$2$};
      \node[state]            (3) [right of=2]{$3$};
      \node[state]            (4) [right of=3]{$4$};
      \node[state]            (5) [right of=4]{$5$};
      \node[state]            (6) [right of=5]{$6$};
      \node[state]            (7) [right of=6]{$7$};
      \node[state]            (8) [right of=7]{$8$};
      \node[state]            (9) [right of=8]{$9$};
      \node[state]            (10) [right of=9]{$10$};
      \node[state]            (11) [below of=8]{$11$};
      \node[state,accepting]  (12) [below of=11]{$12$};
      
      \begin{scope}[every node/.style={scale=.8}]
      \path
      (1) edge [bend left=15] node [above] {$1;L$} (2)
      (2) edge [bend left=15] node [above] {$1;L$} (3)
      (2) edge [bend right=15] node [below] {$\blank;L$} (3)
      % (3) edge [bend left=15] node [above] {$1/1;L$} (4)
      (3) edge [bend left=15] node [above] {$\blank/1;L$} (4)
      (4) edge [bend left=15] node [above] {$\blank/1;R$} (5)
      % (4) edge [bend right=15] node [below] {$1/1;R$} (5)
      (5) edge [loop below] node [below] {$1;R$} (5)
      (5) edge [bend left=15] node [above] {$\blank;R$} (6)
      (6) edge [loop below] node [below] {$1;R$} (6)
      (6) edge [bend left=15] node [above] {$\blank;L$} (7)
      (7) edge [bend left=15] node [above] {$1/\blank;L$} (8)
      % (7) edge [bend right=15] node [below] {$\blank;L$} (8)
      (8) edge [bend left=15] node [above] {$1;L$} (9)
      (9) edge [loop below] node [below] {$1;L$} (9)
      (9) edge [bend right=15] node [below] {$\blank;L$} (10)
      (10) edge [loop below] node [below] {$1;L$} (10)
      (10) edge [bend right=30] node [above] {$\blank;R$} (2)
      (8) edge [] node [right] {$\blank;L$} (11)
      (11) edge [loop left] node [left] {$1;L$} (11)
      (11) edge [] node [right] {$\blank;R$} (12);
      \end{scope}
    \end{tikzpicture}
  \end{center}
\end{figure}    

\begin{problem}
  За произволно естествено число $n$, дефинирайте МТ $\M_n$ с $n+11$ състояние, за която,
  ако главата е на най-лявата $1$-ца върху блок от $1$-ци, то $\M_n$
  завършва като записва $2n$ единици на лентата и завършва в стандартна конфигурация.
\end{problem}


\subsection*{Канонична подредба на $\Sigma^\star$}

Нека $\Sigma = \{a_0,a_1,\dots,a_{k-1}\}$.
Подреждаме думите по ред на тяхната дължина.
Думите с еднаква дължина подреждаме по техния числов ред, т.е.
гледаме на буквите $a_i$ като числото $i$ в $k$-ична бройна система.
Тогава думите с дължина $n$ са числата от $0$ до $k^n-1$ записани в $k$-ична бройна система.
Ще означаваме с $\omega_i$ $i$-тата дума в $\Sigma^\star$ при тази подредба.

\begin{example}
  Ако $\Sigma = \{0,1\}$, то наредбата започва така:
  \[\varepsilon, 0, 1, \underbrace{00, 01, 10, 11}_{\text{от $0$ до $3$}}, \underbrace{000, 001, 010, 011, 100, 101, 110, 111}_{\text{от $0$ до $7$}}, 0000, 0001, \dots\]
  В този случай, $\omega_0 = \varepsilon$, $\omega_7 = 000$, $\omega_{13} = 110$.
\end{example}

\subsection*{Многолентови машини на Тюринг}

Това е просто като имаш shift.
Използват се при недет. машини

Машина на Тюринг с $k$ ленти има същата дефиниция като еднолентова МТ
с единствената разлика, че
\[\delta: Q^k \times \Gamma^k\to Q^k \times \Gamma^k \times \{L,R\}^k.\]

\begin{prop}
  За всяка $k$-лентова МТ $\M$ съществува еднолентова МТ $\M'$,
  такава че $\L(\M) = \L(\M')$.
\end{prop}
\begin{proof}
  $\Gamma' = \{\sharp\} \cup \Gamma \cup \{\hat{X} \mid X \in \Gamma\}$.
\end{proof}



\subsection*{Недетерминистични машини на Тюринг}

\begin{thm}
  Ако $L$ се разпознава от НМТ $\N$, до $L$
  също се разпознава и от ДМТ $\D$.
\end{thm}
\begin{proof}
  Понеже функцията $\Delta_\D$ е крайна, нека с $r$ да означим 
  максималния брой избори за следваща стъпка в произволно изчисление на $\D$.
  $\D$ има три ленти.
  \begin{itemize}
  \item 
    На първата лента съхраняваме входящия низ и тя никога не се променя.
  \item
    На втората лента помним мястото на $\D$ в дървото на недетерминистичните изчисления на $\N$.  
  \item
    На третата лента съхраняваме лентата на $\N$ за детерминистичното изчисление на $\N$, 
    определено от втората лента. Например, ако съдържанието на втората лента е $3,1,2$,
    това означава, че симулираме изчисление от три стъпки като на първата стъпка избираме третия 
    клон, на втората стъпка избираме първия клон, на третата стъпка избираме втория клон.
  \end{itemize}
\end{proof}


\subsection*{Полуразрешими и разрешими езици}

\begin{thm}
  Ако $L$ и $\Sigma^\star \setminus L$ са полуразрешими езици, то $L$ е разрешим език.
\end{thm}

\subsection*{Busy Beaver}

Тук разглеждаме азбука $\Sigma = \{1,\blank\}$.
\marginpar{\cite{tibor-rado}}


\subsection*{Тезис на Чърч-Тюринг}

\section{Универсална машина на Тюринг}
За простота, нека $\Sigma = \{0,1\}$ и $\Gamma = \{0,1,\blank\}$.
\begin{itemize}
\item 
  $X_1 = 0$, $X_2 = 1$, $X_3 = \blank$;
\item
  $D_1 = L$, $D_2 = R$
\end{itemize}

\subsection*{Кодиране на преход}
Да разгледаме прехода $\delta(q_i,X_j) = (q_k,X_l,D_m)$.
Кодираме този преход по следния начин:
\[0^i10^j10^k10^l10^m\]
Да обърнем внимание, че в този двоичен код няма последователни единици и той 
започва и завършва с нула.
\subsection*{Кодиране на машина на Тюринг}
За да кодираме една машина на Тюринг $\M$ е достатъчно да кодираме функцията на преходите $\delta$.
Понеже $\delta$ е крайна функция, нека с числото $r$ да означим броя на всички възможни преходи.
По описания по-горе начин, нека $code_i$ е числото в двоичен запис, получено за $i$-тия преход на $\delta$.
Тогава кодът на $\M$ е следното число в двоичен запис:
\[\pair{\M} = 111\ code_1\ 11\ code_2\ 11\ \cdots\ 11\ code_r\ 111.\]
\begin{itemize}
\item
  Лесно се съобразява, че за две МТ $\M$ и $\M'$ с различни функции на преходите, имаме $\pair{\M} \neq \pair{\M'}$.
\item
  Ще казваме, че числото $r$ е {\bf код на } $\M$, ако числото $r$, записано в двоичен запис представлява думата $\pair{\M}$.
  Оттук нататък, когато пишем $\M_r$, ще имаме предвид машината на Тюринг с код $r$.
\item
  С $\pair{\M,w}$ ще означаваме кода на МТ $\M$ при вход $w$ е числото с двоичен запис описанието на $\M$ и след това прикрепена думата $w$.
  При едно число $r = \pair{M,w}$, лесно се намира кода на $\M$.
  Просто започваме да четем двоичния запис на $r$ докато не срещнем за втори път $111$.
  След това започва думата $w$.
\end{itemize}


\section{Изчислими функции}

Нека е дадена функцията $f:\Nat^k \to \Nat$.
Ще казваме, че $f$ е изчислима с машината на Тюринг $\M$,
ако за всяко $n_1,\dots,n_k$ е изпълнено:
\begin{itemize}
\item 
  Представяме всяко от числата $n_1,\dots,n_k$ в монадична бройна система
  като лентата на $\M$ има вида:  
  \[\dots \blank \blank \underbrace{1111\dots 11}_{n} \blank\blank\dots,\]
  като изискваме главата на $\M$ да е позиционирана върху най-лявата единица.
  Такава конфигурация ще наричаме {\bf стандартна начална конфигурация}.
\item
  Ако $f(n_1,\dots,n_k) = m$, то $\M$ завършва с резултат върху лентата
  \[\dots \blank \blank \underbrace{1111\dots 11}_{m} \blank\blank\dots,\]
  като главата на $\M$ е върху най-лявата 1-ца.
  Такава конфигурация се нарича {\bf стандартна финална конфигурация}.
\item
  Ако $f(n_1,\dots,n_k)$ е недефенирана, то $\M$ няма да завърши в стандартна конфигурация, т.е.
  или $\M$ ще работи безкрайно време, или ще завърши в конфигурация, която не е стандартна.
\end{itemize}

% За фунция от вида $f:\Nat^k\to \Nat$, дефинираме аналогично какво означава $f$
% да е изчислима с машина на Тюринг $\M$, като тук за да изчислим $f(n_1,\dots,n_k)$, входната лента има вида:
% \[\dots \blank \blank \underbrace{11\dots 1}_{n_1}\blank  \underbrace{11\dots 11}_{n_2} \blank \dots \blank  \underbrace{11\dots 11}_{n_k} \blank\blank\dots\]

\begin{thm}
  За всяко $k$, съществуват функции от вида $f:\Nat^k\to\Nat$, които не са изчислими с МТ.
\end{thm}
\begin{proof}
  Знаем, че всяка МТ може да се кодира с естествено число.
  Това означава, че съществуват изброимо безкрайно много различни машини на Тюринг.
  Също така, ние знаем, че съществуват неизброимо много различни функции от вида $f:\Nat^k\to\Nat$.
  Заключаваме, че със сигурност съществуват функции, които не са изчислими с МТ.
\end{proof}




\section{Пример за език, който не се разпознава от МТ}

Да разгледаме безкрайната таблица $\{a_{ij} \mid i,j \in \Nat\}$, където:
\begin{align*}
  a_{ij} = 
  \begin{cases}
    1, & \text{ ако } w_i \in L(\M_j), \\
    0, & \text{ ако } w_i \not\in L(\M_j).
  \end{cases}
\end{align*}
Идеята е да вземем диагонала на тази таблица.

\begin{framed}
  Езикът 
  $L_d = \{w_i \mid w_i \not\in L(\M_i)\}$ не се разпознава от MT.
\end{framed}
Да допуснем, че $L_d$ се разпознава от МТ, т.е. $L_d = L(\M_i)$, за някоя МТ с код $i$.
Тогава:
\begin{itemize}
\item 
  Ако
  $w_i \in L_d\ \rightarrow\ w_i \in L(\M_i)\ \rightarrow\ w_i \not\in L_d$;
\item
  Ако 
  $w_i \not\in L_d\ \rightarrow\ w_i \not\in L(\M_i)\ \rightarrow\ w_i \in L_d$.
\end{itemize}

\section{Универсалният език $L_u$}

Да разгледаме езика $L_u = \{\pair{\M,w} \mid w\in L(\M)\}$.

\section{Проблемът за съответствие на Пост (PCP)}

\subsection*{MPCP}

\section{Разрешими и полуразрешими езици}




\section{Теорема на Райс-Успенски}

\section{Проблеми за безконтекстни езици}

\begin{lemma}
  Нека е дадена $\M = \TM$.
  Тогава езикът 
  \[L = \{\alpha\sharp\beta^R \mid \alpha,\beta \in \Gamma^\star Q \Gamma^\star\ \&\  \alpha \vdash_\M \beta\}\]
  е безконтекстен.
\end{lemma}
\begin{proof}
  Ще покажем, че съществува стеков автомат $P$, за който $\L_S(P) = L$.
  Четем буквата $X$. Тогава:
  \begin{itemize}
  \item 
    ако $\delta_\M(q,X) =(p,Y,R)$, то слагаме $Yp$ на върха на стека;
  \item
    ако $\delta_\M(q,X) =(p,Y,L)$, то ако $Z$ е върха на стека, заменяме $Z$ с $pZY$;
  \end{itemize}
\end{proof}

\begin{lemma}
  Нека е дадена $\M = \TM$.
  Тогава езикът 
  \[L = \{\alpha\sharp\beta^R \mid \alpha,\beta \in \Gamma^\star Q \Gamma^\star\ \&\  \alpha \not\vdash_\M \beta\}\]
  е безконтекстен.
\end{lemma}


\begin{thm}
  Неразрешим е проблемът за проверка дали при дадени две произволни безконтекстни граматики $G_1$ и $G_2$,
  $\L(G_1) \cap \L(G_2) = \emptyset$.  
\end{thm}

\begin{thm}
  Неразрешим е проблемът за проверка дали при дадена произволна безконтекстна граматика $G$,
  $\L(G) = \Sigma^\star$.  
\end{thm}


\section{Въпроси}

Вярно ли е, че следният проблем е {\em разрешим}:
\begin{itemize}
\item
  за произволна безконтекстна граматика $G$, проверява дали $\L(G) = \emptyset$?
\item
  за произволна безконтекстна граматика $G$, проверява дали $\L(G) = \Sigma^\star$?
\item
  за произволни безконтекстни граматики $G_1$ и $G_2$, проверява дали $\L(G_1) \cap \L(G_2) = \emptyset$?
\item
  за произволни безконтекстни граматики $G_1$ и $G_2$, проверява дали $\L(G_1) \cap \L(G_2) = \Sigma^\star$?
\item
  за произволни безконтекстни граматики $G_1$ и $G_2$, проверява дали $\L(G_1) = \L(G_2)$?
\item
  за произволни безконтекстни граматики $G_1$ и $G_2$, проверява дали $\L(G_1) \subseteq \L(G_2)$?
\item
  за произволна безконтекстна граматика $G$ и произволен регулярен израз $r$,
  проверява дали $\L(G) = \L(r)$?
\item
  за произволна безконтекстна граматика $G$ и произволен регулярен израз $r$,
  проверява дали $\L(G) \subseteq \L(r)$?
\item
  за произволна безконтекстна граматика $G$ и произволен регулярен израз $r$,
  проверява дали $\L(r) \subseteq \L(G)$?
% \item
%   за произволни безконтекстни граматики $G_1$ и $G_2$, проверява дали $\L(G_1) \subseteq \L(G_2)$ 
%   е безконтекстен език ?
% \item
%   за произволна безконтекстна граматика $G$, проверява дали $\Sigma^\star \setminus \L(G)$
%   е безконтекстен език ?
% \item
%   за произволна безконтекстна граматика $G$, проверява дали $\L(G)$ е регулярен език?
\end{itemize}


%%% Local Variables: 
%%% mode: latex
%%% TeX-master: "EAI"
%%% End: 
