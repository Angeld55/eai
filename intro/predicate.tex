\section{Предикати и квантори}

\subsection*{Квантори}

Свойствата или отношенията на елементите в произволно множество се наричат {\bf предикати}.
Нека да разгледаме един едноместен предикат $P(\cdot)$.

\bigskip
\begin{tabular}{|l|p{4.2cm}|p{4.5cm}|}
  \hline
  твърдение & Кога е истина? & Кога е неистина?\\
  \hline
  $\forall x P(x)$ & $P(x)$ е вярно за всяко $x$ & съществува $x$, за което $P(x)$ {\bf не} е вярно \\
  \hline
  $\exists x P(x)$ & съществува $x$, за което $P(x)$ е вярно & $P(x)$ {\bf не} е вярно за всяко $x$\\
  \hline
\end{tabular}  
\bigskip

\begin{enumerate}[(I)]
\item 
  {\bf Квантор за общност} $\forall x$.
  Записът $(\forall x \in A) P(x)$ означава, че за всеки елемент $a$ в $A$, 
  твърдението $P(a)$ има стойност истина.
  Например, $(\forall x\in\Real)[(x+1)^2 = x^2+2x+1]$.
\item
  {\bf Квантор за съществуване} $\exists x$.
  Записът $(\exists x \in A) P(x)$ означава, че съществува елемент $a$ в $A$, 
  за който твърдението $P(a)$ има стойност истина.
  Например, $(\exists x \in\mathbb{C})[x^2 = -1]$, но $(\forall x\in\Real)[x^2 \neq -1]$.
\end{enumerate}

% \begin{example}
%   \begin{itemize}
%   \item
%     За всяко естествено число, съществува по-голямо от него:
%     \[(\forall x\in\Nat)(\exists z\in\Nat)[x < z].\]
%   \item
%     Съществува естествено число, от което няма по-малко:
%     \[(\exists x\in\Nat)(\forall y\in\Nat)[x < y \vee x = y].\]
%     Нека да означим с $Zero(x)$ предиката, който казва, че $x$ е най-малкото число, т.е.
%     \[Zero(x) \equiv (\forall y)[x < y \vee x =y].\]
%   \item
%     Нека $S(x,y)$ да бъде предиката, който казва, че $y = x+1$ в естествените числа:
%     \[S(x,y) \equiv (x < y\ \wedge\ (\forall z\in\Nat)[x < z\ \rightarrow (z = y\ \vee\ y < z)].\]
%   \item
%     $One(x)$ - $x$ е числото $1$:
%     \[One(x) \equiv (\exists y)[Z(y)\ \wedge\ S(y,x)].\]
%   \item
%     $Div(x,y)$ - $x$ се дели на $y$:
%     \[Div(x,y) \equiv (\exists z)[x = y.z].\]
%   \item
%     $Prime(x)$ - $x$ е просто число:
%     \[Prime(x) \equiv x \geq 2\ \wedge\ (\forall y\in\Nat)[\neg (O(y)\ \wedge Z(y))\ \rightarrow\ \neg Div(x,y)].\]
%   \end{itemize}
% \end{example}


\subsection*{Закони на предикатното смятане}

\begin{enumerate}[(I)]
  \item
    $\neg\forall x P(x) \iff \exists x \neg P(x)$
  \item
    $\neg\exists x P(x) \iff \forall x \neg P(x)$
  \item
    $\forall x P(x) \iff \neg\exists x \neg P(x)$
  \item
    $\exists x P(x) \iff \neg\forall x \neg P(x)$
  \item
    $\forall x \forall y P(x,y) \iff \forall y\forall x P(x,y)$
  \item
    $\exists x\exists y P(x,y) \iff \exists y \exists x P(x,y)$  
  \item
    $\exists x\forall y P(x,y) \rightarrow \forall y \exists x P(x,y)$
\end{enumerate}

\bigskip
\begin{tabular}{|l|p{2.5cm}|p{3.2cm}|p{3cm}|}
  \hline
  \multicolumn{4}{|c|}{{\bf Закони на Де Морган за квантори}}\\
  \hline
  Твърдение & Еквивалентно твърдение & Кога е истина? & Кога е неистина?\\
  \hline
  $\neg \exists x P(x)$ & $\forall x \neg P(x)$ & за всяко $x$ $P(x)$ {\bf не} е вярно & съществува $x$, за което $P(x)$ е вярно \\
  \hline
  $\neg \forall x P(x)$ & $\exists x \neg P(x)$ & съществува $x$, за което $P(x)$ {\bf не} е вярно & $P(x)$ е вярно за всяко $x$\\
  \hline
\end{tabular}  
\bigskip

\begin{problem}
  Да означим с $K(x,y)$ твърдението ``$x$ познава $y$''.
  Изразете като предикатна формула следните твърдения.
  \begin{enumerate}[1)]
  \item
    \mynote{$\forall x \exists y K(x,y)$}
    Всеки познава някого.
  \item
    \mynote{$\exists x \forall y K(x,y)$}
    Някой познава всеки.
  \item
    \mynote{$\exists x\forall y K(y,x)$}
    Някой е познаван от всички.
  \item
    \mynote{$\forall x \exists y(K(x,y)\wedge \neg K(y,x)) $}
    Всеки знае някой, който не го познава.
  \item
    \mynote{$\exists x \forall y(K(y,x)\ \rightarrow K(x,y))$}
    Има такъв, който знае всеки, който го познава.
  \item
    \mynote{$(\forall x,y)(K(x,y)\ \&\ K(y,x) \to \exists z(K(x,z)\ \&\ K(y,z))$}
    Всеки двама познати имат общ познат.
  \end{enumerate}
\end{problem}

\begin{example}
  Нека $D \subseteq \Real$.
  Казваме, че $f:D \to \Real$ е {\em непрекъсната} в точката $x_0 \in D$, ако 
  \[(\forall \varepsilon > 0)(\exists \delta >0)(\forall x\in D)(\ |x_0 - x| < \delta\ \to\ |f(x_0) - f(x)| < \varepsilon\ ).\]
  Да видим какво означава $f$ да бъде {\em прекъсната} в точката $x_0 \in D$:
  \mynote{$f$ е прекъсната в $x_0$ точно тогава, когато $f$ не е непрекъсната в $x_0$}
  \begin{align*}
    & \neg (\forall \varepsilon > 0)(\exists \delta >0)(\forall x\in D)(\ |x_0 - x| < \delta\ \to\ |f(x_0) - f(x)| < \varepsilon\ ) \equiv \\
    & (\exists \varepsilon > 0) \neg (\exists \delta >0)(\forall x\in D)(\ |x_0 - x| < \delta\ \to\ |f(x_0) - f(x)| < \varepsilon\ ) \equiv \\
    & (\exists \varepsilon > 0)(\forall \delta >0)\neg(\forall x\in D)(\ |x_0 - x| < \delta\ \to\ |f(x_0) - f(x)| < \varepsilon\ ) \equiv \\
    & (\exists \varepsilon > 0)(\forall \delta >0)(\exists x\in D)\neg(\ |x_0 - x| < \delta\ \to\ |f(x_0) - f(x)| < \varepsilon\ ) \equiv \\
    & (\exists \varepsilon > 0)(\forall \delta >0)(\exists x\in D)\neg(\ \neg (|x_0 - x| <\delta) \vee |f(x_0) - f(x)| < \varepsilon\ ) \equiv \\
    & (\exists \varepsilon > 0)(\forall \delta >0)(\exists x\in D)(\ \neg\neg (|x_0 - x| <\delta) \land \neg (|f(x_0) - f(x)| < \varepsilon)\ ) \equiv \\
    & (\exists \varepsilon > 0)(\forall \delta >0)(\exists x\in D)(\ |x_0 - x| < \delta\ \land\ |f(x_0) - f(x)| \geq \varepsilon\ ).
  \end{align*}
\end{example}


%%% Local Variables:
%%% mode: latex
%%% TeX-master: "../eai"
%%% End:
