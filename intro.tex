\chapter{Увод}

\section{Съждително смятане}
\label{sect:propositional}
\marginpar{На англ. Propositional calculus}

Съждителното смятане наподобява аритметичното смятане, като вместо аритметичните операции $+,-,\cdot,/$, 
имаме съждителни операции като $\neg, \wedge, \vee$.
Например, $(p\vee q) \wedge \neg  r$ е съждителен израз.
Освен това, докато аритметичните променливи приемат стойности произволни числа, то
съждителните променливи приемат само стойности {\bf истина (1)} или {\bf неистина (0)}.

{\bf Съждителен израз} наричаме съвкупността от съждителни променливи $p,q,r,\dots$, свързани със знаците за логически операции
$\neg, \vee, \wedge, \rightarrow, \leftrightarrow$ и скоби, определящи реда на операциите.

\subsection*{Съждителни операции}

\begin{itemize}
\item
  Отрицание $\neg$
\item 
  Дизюнкция $\vee$
\item
  Конюнкция $\wedge$
\item
  Импликация $\rightarrow$
\item
  Еквивалентност $\iff$
\end{itemize}

Ще използваме таблица за истинност за да определим стойностите на основните съждителни операции
при всички възможни набори на стойностите на променливите.

\[
\begin{array}{|c|c|c|c|c|c|c|c|c|}
  \hline
  p & q & \neg p & p \vee q & p \wedge q & p \rightarrow q & \neg p \vee q & p \iff q & (\neg{p}\wedge q)\ \vee\ (p\wedge \neg q) \\
  \hline
  0 & 0 & 1 & 0 & 0 & 1 & 1 & 1 & 1\\
  \hline
  0 & 1 & 1 & 1 & 0 & 1 & 1 & 0 & 0\\
  \hline
  1 & 0 & 0 & 1 & 0 & 0 & 0 & 0 & 0\\
  \hline
  1 & 1 & 0 & 1 & 1 & 1 & 1 & 1 & 1\\
  \hline
\end{array}
\]


{\bf Съждително верен} (валиден) е този логически израз, който има верностна стойност {\bf 1} при всички възможни набори на
стойностите на съждителните променливи в израза, т.е. стълбът на израза в таблицата за истинност трябва да съдържа само 
стойности {\bf 1}. 

Два съждителни израза $\varphi$ и $\psi$ са {\bf еквивалентни}, което означаваме $\varphi \equiv \psi$, ако са съставени от 
едни и същи съждителни променливи и двата израза имат едни и същи верностни стойности при всички комбинации от верностни 
стойности на променливите. С други думи, колоните на двата израза в съответните им таблици за истинност трябва да съвпадат.
Така например, от горната таблица се вижда, че 
$p\to q \equiv \neg p \vee q$ и $p \iff q \equiv (\neg{p}\wedge q)\ \vee\ (p\wedge \neg q)$.

\subsection*{Съждителни закони}

\begin{enumerate}[I)]
  \item
    {\bf Комутативен закон}
    \[p\vee q \equiv q\vee p\] 
    \[p \wedge q \equiv q \wedge p\]
  \item
    {\bf Асоциативен закон}
    \[(p\vee q)\vee r \equiv p\vee(q\vee r)\]
    \[(p\ \wedge\ q)\ \wedge\ r \equiv p\ \wedge\ (q\ \wedge\ r)\]
  \item
    {\bf Дистрибутивен закон}
    \[p\ \wedge\ (q \vee r) \equiv (p\ \wedge q)\vee (p\ \wedge\ r)\]
    \[p\vee (q\ \wedge\ r) \equiv (p\vee q)\ \wedge\ (p\vee r)\]
  \item
    {\bf Закони на де Морган}
    \[\neg(p \wedge q) \equiv (\neg p \vee \neg q)\]
    \[\neg(p\vee q) \equiv (\neg p \wedge \neg q)\]
  \item
    {\bf Закон за контрапозицията}
    \[p\rightarrow q \equiv \neg q \rightarrow \neg p\]
  \item
    {\bf Обобщен закон за контрапозицията}
    \[(p \wedge q)\rightarrow r \equiv (p \wedge \neg r) \rightarrow \neg q\]
  \item
    {\bf Закон за изключеното трето}
    \[p\vee \neg p \equiv {\mathbf 1}\]
  \item
    {\bf Закон за силогизма (транзитивност)}
    \[[(p\rightarrow q)\ \wedge\ (q\rightarrow r)] \rightarrow (p\rightarrow r) \equiv {\mathbf 1}\]
\end{enumerate}

Лесно се проверява с таблиците за истинност, че законите са валидни.


\section{Предикатно смятане}

\section{Индукция}

\section{Азбуки, думи, езици}

\subsection*{Основни понятия}

\begin{itemize}
\item 
  \index{азбука}
  {\bf Азбука} ще наричаме всяко крайно множество,
  като обикновено ще я означаваме със $\Sigma$.
  \marginpar{Често ще използваме буквите $a$, $b$, $c$ за да означаваме букви.}
  Елементите на азбуката $\Sigma$ ще наричаме {\bf букви} или символи.
\item
  \index{дума}
  {\bf Дума} над азбуката $\Sigma$ е произволна крайна редица от елементи на $\Sigma$.
  Например, за $\Sigma = \{a,b\}$, $aababba$ е дума над $\Sigma$ с дължина $7$.
  С $\abs{\alpha}$ ще означаваме дължината на думата $\alpha$.
  \marginpar{Обикновено ще означаваме думите с $\alpha$, $\beta$, $\gamma$, $\omega$.}
\item
  Обърнете внимание, че имаме единствена дума с дължина $0$.
  Тази дума ще означаваме с $\varepsilon$ и ще я наричаме {\bf празната дума},
  т.е. $\abs{\varepsilon} = 0$.
\item
  С $a^n$ ще означаваме думата съставена от $n$ $a$-та.
  Формалната индуктивна дефиниция е следната:
  \begin{align*}
    a^0 & = \varepsilon,\\
    a^{n+1} & = a^na.
  \end{align*}
\item
  Множеството от всички думи над азбуката $\Sigma$ ще означаваме със $\Sigma^\star$.
  Например, за $\Sigma = \{a,b\}$,
  \[\Sigma^\star = \{\varepsilon,a,b,aa,ab,ba,bb,aaa,aab,\dots\}.\]
  Обърнете внимание, че $\emptyset^\star = \{\varepsilon\}$.
\end{itemize}

\subsection*{Операции върху думи}

\begin{itemize}
\item 
  \index{конкатенация}
  Операцията {\bf конкатенация} взима две думи $\alpha$ и $\beta$ и образува 
  новата дума $\alpha\cdot\beta$ като слепва двете думи.
  Например $aba\cdot bb = ababb$.
  Обърнете внимание, че в общия 
  случай $\alpha\cdot\beta \neq \beta\cdot\alpha$. 
  \marginpar{Често ще пишем $\alpha\beta$ вместо $\alpha\cdot\beta$}
  Можем да дадем формална индуктивна дефиниция на операцията конкатенация по
  дължината на думата $\beta$.
  \begin{itemize}
  \item 
    Ако $\abs{\beta} = 0$, то $\beta = \varepsilon$.
    Тогава $\alpha\cdot \varepsilon \df \alpha$.
  \item
    Ако $\abs{\beta} = n+1$, то $\beta = \gamma b$, $\abs{\gamma} = n$.
    Тогава $\alpha\cdot\beta \df (\alpha\cdot\gamma)b$.
  \end{itemize}
\item
  \marginpar{Обърнете внимание, че $\emptyset\cdot A = A\cdot\emptyset = \emptyset$}
  \marginpar{Също така, $\{\varepsilon\}\cdot A = A\cdot\{\varepsilon\} = A$}
  Нека $A$ и $B$ са множества от думи.
  Дефинираме конкатенацията на $A$ и $B$ като
  \[A\cdot B \df \{\alpha\cdot\beta \mid \alpha\in A\ \&\ \beta \in B\}.\]
\item
  Сега за едно множество от думи $A$, дефинираме $A^n$ индуктивно:
  \begin{align*}
    A^0 & \df \{\varepsilon\},\\
    A^{n+1} & \df A^n \cdot A.
  \end{align*}
  Ако $A = \{ab, ba\}$, то
  $A^0 = \{\varepsilon\}$, $A^1 = A$, $A^2 = \{abab, abba, baba, baab\}$.
  Ако $A = \{a,b\}$, то $A^n = \{\alpha \in \{a,b\}^\star \mid \abs{\alpha} = n\}$.
\item
  За едно множеството от думи $A$, дефинираме:
  \begin{align*}
    A^\star & \df \bigcup_{n\geq 0} A^n\\
    & = A^0 \cup A^1 \cup A^2 \cup A^3 \cup \dots\\
    A^+ & \df A\cdot A^\star.
  \end{align*}
\end{itemize}

\begin{problem}
  Докажете, че:
  \begin{enumerate}[a)]
  \item 
    операцията конкатенация е {\em асоциативна}, т.е. за всеки три думи $\alpha$, $\beta$, $\gamma$,
    \[(\alpha\cdot\beta)\cdot\gamma = \alpha\cdot(\beta\cdot\gamma);\]
  \item
    за множествата от думи $A$, $B$ и $C$,
    \[(A\cdot B)\cdot C = A\cdot (B\cdot C);\]
  \item
    за произволно множество от думи $A$,
    $A^\star = A^\star\cdot A^\star$ и $(A^\star)^\star = A^\star$.
  \end{enumerate}
\end{problem}


\section*{Библиография}

\begin{itemize}
\item 
  Глава 1 от \cite{rosen}
\item
  На практика следваме Глава 2 от \cite{kozen} в описанието на думи и азбуки.
\end{itemize}

%%% Local Variables: 
%%% mode: latex
%%% TeX-master: "EAI"
%%% End: 
