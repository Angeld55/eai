\chapter{Увод}
\label{ch:intro}

\section{Съждително смятане}
\label{sect:propositional}
\marginpar{На англ. Propositional calculus}

Съждителното смятане наподобява аритметичното смятане, като вместо аритметичните операции $+,-,\cdot,/$, 
имаме съждителни операции като $\neg, \wedge, \vee$.
Например, $(p\vee q) \wedge \neg  r$ е съждителна формула.
Освен това, докато аритметичните променливи приемат стойности числа, то
съждителните променливи приемат само стойности {\bf истина (1)} или {\bf неистина (0)}.

{\bf Съждителна формула} наричаме съвкупността от съждителни променливи $p,q,r,\dots$, свързани със знаците за логически операции
$\neg, \vee, \wedge, \rightarrow, \leftrightarrow$ и скоби, определящи реда на операциите.

\subsection*{Съждителни операции}

\begin{itemize}
\item
  Отрицание $\neg$
\item 
  Дизюнкция $\vee$
\item
  Конюнкция $\wedge$
\item
  Импликация $\rightarrow$
\item
  Еквивалентност $\iff$
\end{itemize}

Ще използваме таблица за истинност за да определим стойностите на основните съждителни операции
при всички възможни набори на стойностите на променливите.

\[
\begin{array}{|c|c|c|c|c|c|c|c|c|}
  \hline
  p & q & \neg p & p \vee q & p \wedge q & p \rightarrow q & \neg p \vee q & p \iff q & (\neg{p}\wedge q)\ \vee\ (p\wedge \neg q) \\
  \hline
  0 & 0 & 1 & 0 & 0 & 1 & 1 & 1 & 1\\
  \hline
  0 & 1 & 1 & 1 & 0 & 1 & 1 & 0 & 0\\
  \hline
  1 & 0 & 0 & 1 & 0 & 0 & 0 & 0 & 0\\
  \hline
  1 & 1 & 0 & 1 & 1 & 1 & 1 & 1 & 1\\
  \hline
\end{array}
\]


{\bf Съждително верен} (валиден) е този логически израз, който има верностна стойност {\bf 1} при всички възможни набори на
стойностите на съждителните променливи в израза, т.е. стълбът на израза в таблицата за истинност трябва да съдържа само 
стойности {\bf 1}. 

Два съждителни израза $\varphi$ и $\psi$ са {\bf еквивалентни}, което означаваме $\varphi \equiv \psi$, ако са съставени от 
едни и същи съждителни променливи и двата израза имат едни и същи верностни стойности при всички комбинации от верностни 
стойности на променливите. С други думи, колоните на двата израза в съответните им таблици за истинност трябва да съвпадат.
Така например, от горната таблица се вижда, че 
$p\to q \equiv \neg p \vee q$ и $p \iff q \equiv (\neg{p}\wedge q)\ \vee\ (p\wedge \neg q)$.

\subsection*{Съждителни закони}

\begin{enumerate}[I)]
  \item
    {\bf Комутативен закон}
    \[p\vee q \equiv q\vee p\] 
    \[p \wedge q \equiv q \wedge p\]
  \item
    {\bf Асоциативен закон}
    \[(p\vee q)\vee r \equiv p\vee(q\vee r)\]
    \[(p\ \wedge\ q)\ \wedge\ r \equiv p\ \wedge\ (q\ \wedge\ r)\]
  \item
    {\bf Дистрибутивен закон}
    \[p\ \wedge\ (q \vee r) \equiv (p\ \wedge q)\vee (p\ \wedge\ r)\]
    \[p\vee (q\ \wedge\ r) \equiv (p\vee q)\ \wedge\ (p\vee r)\]
  \item
    {\bf Закони на де Морган}
    \[\neg(p \wedge q) \equiv (\neg p \vee \neg q)\]
    \[\neg(p\vee q) \equiv (\neg p \wedge \neg q)\]
  \item
    {\bf Закон за контрапозицията}
    \[p\rightarrow q \equiv \neg q \rightarrow \neg p\]
  \item
    {\bf Обобщен закон за контрапозицията}
    \[(p \wedge q)\rightarrow r \equiv (p \wedge \neg r) \rightarrow \neg q\]
  \item
    {\bf Закон за изключеното трето}
    \[p\vee \neg p \equiv {\mathbf 1}\]
  \item
    {\bf Закон за силогизма (транзитивност)}
    \[[(p\rightarrow q)\ \wedge\ (q\rightarrow r)] \rightarrow (p\rightarrow r) \equiv {\mathbf 1}\]
\end{enumerate}

Лесно се проверява с таблиците за истинност, че законите са валидни.

\section{Предикати и квантори}

\subsection*{Квантори}

Свойствата или отношенията на елементите в произволно множество се наричат {\bf предикати}.
Нека да разгледаме един едноместен предикат $P(\cdot)$.

\bigskip
\begin{tabular}{|l|p{4.2cm}|p{4.5cm}|}
  \hline
  твърдение & Кога е истина? & Кога е неистина?\\
  \hline
  $\forall x P(x)$ & $P(x)$ е вярно за всяко $x$ & съществува $x$, за което $P(x)$ {\bf не} е вярно \\
  \hline
  $\exists x P(x)$ & съществува $x$, за което $P(x)$ е вярно & $P(x)$ {\bf не} е вярно за всяко $x$\\
  \hline
\end{tabular}  
\bigskip

\begin{enumerate}[(I)]
\item 
  {\bf Квантор за общност} $\forall x$.
  Записът $(\forall x \in A) P(x)$ означава, че за всеки елемент $a$ в $A$, 
  твърдението $P(a)$ има стойност истина.
  Например, $(\forall x\in\Real)[(x+1)^2 = x^2+2x+1]$.
\item
  {\bf Квантор за съществуване} $\exists x$.
  Записът $(\exists x \in A) P(x)$ означава, че съществува елемент $a$ в $A$, 
  за който твърдението $P(a)$ има стойност истина.
  Например, $(\exists x \in\mathbb{C})[x^2 = -1]$, но $(\forall x\in\Real)[x^2 \neq -1]$.
\end{enumerate}

% \begin{example}
%   \begin{itemize}
%   \item
%     За всяко естествено число, съществува по-голямо от него:
%     \[(\forall x\in\Nat)(\exists z\in\Nat)[x < z].\]
%   \item
%     Съществува естествено число, от което няма по-малко:
%     \[(\exists x\in\Nat)(\forall y\in\Nat)[x < y \vee x = y].\]
%     Нека да означим с $Zero(x)$ предиката, който казва, че $x$ е най-малкото число, т.е.
%     \[Zero(x) \equiv (\forall y)[x < y \vee x =y].\]
%   \item
%     Нека $S(x,y)$ да бъде предиката, който казва, че $y = x+1$ в естествените числа:
%     \[S(x,y) \equiv (x < y\ \wedge\ (\forall z\in\Nat)[x < z\ \rightarrow (z = y\ \vee\ y < z)].\]
%   \item
%     $One(x)$ - $x$ е числото $1$:
%     \[One(x) \equiv (\exists y)[Z(y)\ \wedge\ S(y,x)].\]
%   \item
%     $Div(x,y)$ - $x$ се дели на $y$:
%     \[Div(x,y) \equiv (\exists z)[x = y.z].\]
%   \item
%     $Prime(x)$ - $x$ е просто число:
%     \[Prime(x) \equiv x \geq 2\ \wedge\ (\forall y\in\Nat)[\neg (O(y)\ \wedge Z(y))\ \rightarrow\ \neg Div(x,y)].\]
%   \end{itemize}
% \end{example}


\subsection*{Закони на предикатното смятане}

\begin{enumerate}[(I)]
  \item
    $\neg\forall x P(x) \iff \exists x \neg P(x)$
  \item
    $\neg\exists x P(x) \iff \forall x \neg P(x)$
  \item
    $\forall x P(x) \iff \neg\exists x \neg P(x)$
  \item
    $\exists x P(x) \iff \neg\forall x \neg P(x)$
  \item
    $\forall x \forall y P(x) \iff \forall y\forall x P(x)$
  \item
    $\exists x\exists y P(x,y) \iff \exists y \exists x P(x)$  
  \item
    $\exists x\forall y P(x,y) \rightarrow \forall y \exists x P(x,y)$
\end{enumerate}

\bigskip
\begin{tabular}{|l|p{2.5cm}|p{3.2cm}|p{3cm}|}
  \hline
  \multicolumn{4}{|c|}{{\bf Закони на Де Морган за квантори}}\\
  \hline
  твърдение & Еквивалентно твърдение & Кога е истина? & Кога е неистина?\\
  \hline
  $\neg \exists x P(x)$ & $\forall x \neg P(x)$ & за всяко $x$ $P(x)$ {\bf не} е вярно & съществува $x$, за което $P(x)$ е вярно \\
  \hline
  $\neg \forall x P(x)$ & $\exists x \neg P(x)$ & съществува $x$, за което $P(x)$ {\bf не} е вярно & $P(x)$ е вярно за всяко $x$\\
  \hline
\end{tabular}  
\bigskip

\begin{problem}
  Да означим с $K(x,y)$ твърдението ``$x$ познава $y$''.
  Изразете като предикатна формула следните твърдения.
  \begin{enumerate}[1)]
  \item
    \marginpar{$\forall x \exists y K(x,y)$}
    Всеки познава някого.
  \item
    \marginpar{$\exists x \forall y K(x,y)$}
    Някой познава всеки.
  \item
    \marginpar{$\exists x\forall y K(y,x)$}
    Някой е познаван от всички.
  \item
    \marginpar{$\forall x \exists y(K(x,y)\wedge \neg K(y,x)) $}
    Всеки знае някой, който не го познава.
  \item
    \marginpar{$\exists x \forall y(K(y,x)\ \rightarrow K(x,y))$}
    Има такъв, който знае всеки, който го познава.
  \item
    \marginpar{$(\forall x,y)(K(x,y)\ \&\ K(y,x) \to \exists z(K(x,z)\ \&\ K(y,z))$}
    Всеки двама познати имат общ познат.
  \end{enumerate}
\end{problem}

\begin{example}
  Нека $D \subseteq \Real$.
  Казваме, че $f:D \to \Real$ е {\em непрекъсната} в точката $x_0 \in D$, ако 
  \[(\forall \varepsilon > 0)(\exists \delta >0)(\forall x\in D)[|x_0 - x| < \delta \implies |f(x_0) - f(x)| < \varepsilon].\]
  Да видим какво означава $f$ да бъде {\em прекъсната} в точката $x_0 \in D$:
  \marginpar{$f$ е прекъсната в $x_0$ ако $f$ не е непрекъсната в $x_0$}
  \begin{align*}
    & \neg (\forall \varepsilon > 0)(\exists \delta >0)(\forall x\in D)[|x_0 - x| < \delta \implies |f(x_0) - f(x)| < \varepsilon] \iff\\
    & (\exists \varepsilon > 0) \neg (\exists \delta >0)(\forall x\in D)[|x_0 - x| < \delta \implies |f(x_0) - f(x)| < \varepsilon] \iff \\
    & (\exists \varepsilon > 0)(\forall \delta >0)\neg(\forall x\in D)[|x_0 - x| < \delta \implies |f(x_0) - f(x)| < \varepsilon] \iff \\
    & (\exists \varepsilon > 0)(\forall \delta >0)(\exists x\in D)\neg[|x_0 - x| < \delta \implies |f(x_0) - f(x)| < \varepsilon] \iff \\
    & (\exists \varepsilon > 0)(\forall \delta >0)(\exists x\in D)\neg[\neg (|x_0 - x| <\delta) \vee |f(x_0) - f(x)| < \varepsilon] \iff \\
    & (\exists \varepsilon > 0)(\forall \delta >0)(\exists x\in D)[\neg\neg (|x_0 - x| <\delta) \land \neg (|f(x_0) - f(x)| < \varepsilon)] \iff \\
    & (\exists \varepsilon > 0)(\forall \delta >0)(\exists x\in D)[|x_0 - x| < \delta\ \land\ |f(x_0) - f(x)| \geq \varepsilon].
  \end{align*}
\end{example}

\section{Доказателства на твърдения}

\subsection*{Допускане на противното}

Да приемем, че искаме да докажем, че свойството $P(x)$
е вярно за всяко естествено число.
Един начин да направим това е следният:
\begin{itemize}
\item 
  Допускаме, че съществува елемент $n$, за който $\neg P(n)$.
\item
  Използвайки, че $\neg P(n)$ правим извод, от който следва факт, за който знаем, че винаги е лъжа.
  Това означава, че доказваме следното твърдение
  \[\exists x \neg P(x) \rightarrow \mathbf{0}.\]
\item
  Тогава можем да заключим, че $\forall x P(x)$, защото имаме следния извод:
  \begin{prooftree}
    \AxiomC{$\exists x \neg P(x) \rightarrow \mathbf{0}$}
    \UnaryInfC{$\mathbf{1} \rightarrow \neg \exists x \neg P(x)$}
    \UnaryInfC{$\neg \exists x \neg P(x)$}
    \UnaryInfC{$\forall x P(x)$}
  \end{prooftree}
\end{itemize}

Ще илюстрираме този метод като решим няколко прости задачи.

\begin{problem}
  \label{prob:even-number-square}
  За всяко $a \in \Int$, ако $a^2$ е четно, то $a$ е четно.
\end{problem}
\begin{proof}
  Ние искаме да докажем твърдението $P$, където:
  \[P \equiv (\forall a\in\Z)[a^2\mbox{ е четно}\ \rightarrow\ a\mbox{ е четно}].\]
  \marginpar{$\neg (\forall x)(A(x) \rightarrow B(x))$ е еквивалентно на $(\exists x)(A(x) \wedge \neg B(x))$}
  Да допуснем противното, т.е. изпълнено е $\neg P$. Лесно се вижда, че
  \[\neg P \iff (\exists a\in\Z)[a^2\mbox{ е четно}\ \wedge\ a\mbox{ не е четно}].\]
  Да вземем едно такова нечетно $a$, за което $a^2$ е четно.
  Това означава, че $a = 2k+1$, за някое $k \in \Z$,
  и \[a^2 = (2k+1)^2 = 4k^2 + 4k + 1,\]
  което очевидно е нечетно число.
  Но ние допуснахме, че $a^2$ е четно.
  Така достигаме до противоречие, следователно нашето допускане е грешно 
  и 
  \[(\forall a\in\Z)[a^2\mbox{ е четно}\ \rightarrow\ a\mbox{ е четно}].\]
\end{proof}

\begin{problem}
  Докажете, $\sqrt{2}$ {\bf не} е рационално число.
\end{problem}
\begin{proof}
  Да допуснем, че $\sqrt{2}$ е рационално число. Тогава  съществуват $a,b \in \Z$, такива че
  \[\sqrt{2} = \frac{a}{b}.\]
  Без ограничение, можем да приемем, че $a$ и $b$ са естествени числа,
  които нямат общи делители, т.е. не можем да съкратим дробта $\frac{a}{b}$.
  Получаваме, че \[2b^2 = a^2.\]
  Тогава $a^2$ е четно число и от Задача \ref{prob:even-number-square}, $a$ е четно число.
  Нека $a = 2k$. Получаваме, че
  \[2b^2 = 4k^2,\]
  от което следва, че
  \[b^2 = 2k^2.\]
  Това означава, че $b$ също е четно число, $b = 2n$, за някое $n \in \Z$.
  Следователно, $a$ и $b$ са четни числа и имат общ делител $2$,
  което е противоречие с нашето допускане, че $a$ и $b$ нямат общи делители.
  Така достигаме до противоречие.
  Накрая заключаваме, че $\sqrt{2}$ не е рационално число.
\end{proof}


\subsection*{Индукция върху естествените числа}

\marginpar{Да напомним, че естествените числа са $\Nat = \{0,1,2,\dots\}$}
Доказателството с индукция по $\Nat$ представлява следната схема:
\begin{prooftree}
  \AxiomC{$P(0)$}
  \AxiomC{$(\forall x\in\Nat)[P(x)\rightarrow P(x+1)]$}
  \BinaryInfC{$(\forall x\in\Nat) P(x)$}
\end{prooftree}

Това означава, че ако искаме да докажем, че свойството $P(x)$ е вярно за всяко естествено число $x$,
то трябва да докажем първо, че е изпълнено $P(0)$ и след това, за произволно естествено число $x$, ако $P(x)$ вярно, то също така е вярно $P(x+1)$.

\begin{problem}
  \label{prob:number-prod-prime}  
  Всяко естествено число $n \geq 2$ може да се запише като произведение на прости числа.
\end{problem}
\begin{proof}
  Доказателството протича с индукция по $n \geq 2$.
  \begin{enumerate}[a)]
  \item 
    За $n = 2$  е ясно.
  \item
    Ако $n+1$ е просто число, то всичко е ясно.
    Ако $n+1$ е съставно, то \[n + 1 = n_1\cdot n_2.\]
    Тогава $n_1 = p^{n_1}_1\cdots p^{n_k}_k$ и $n_2 = q^{m_1}_1\cdots q^{m_r}_r$,
    където $p_1,\dots,p_k$ и $q_1,\dots,q_r$ са прости числа.
    Тогава е ясно, че $n+1$ също е произведение на прости числа.
  \end{enumerate}
\end{proof}

\begin{problem}
  Докажете, че за всяко $n$, 
  \[\sum^n_{i=0} 2^i = 2^{n+1} - 1.\]
\end{problem}
\begin{proof}
  Доказателството протича с индукция по $n$.
  \begin{itemize}
  \item 
    За $n = 0$, $\sum^0_{i=0}2^i = 1 = 2^{1} - 1$.
  \item
    Нека твърдението е вярно за $n$.
    Ще докажем, че твърдението е вярно за $n+1$.
    \begin{align*}
      \sum^{n+1}_{i=0} 2^i & = \sum^{n}_{i=0}2^i + 2^{n+1}\\
      & = 2^{n+1} - 1 + 2^{n+1} & (\text{от И.П.})\\
      & = 2.2^{n+1} - 1 \\
      & = 2^{(n+1)+1} - 1.
    \end{align*}
  \end{itemize}
\end{proof}

%\subsection*{Пълна индукция върху естествените числа}

\section{Множества, релации, функции}

\subsection*{Основни операции върху множества}

Ще разгледаме няколко операции върху произволни множества $A$ и $B$.
\begin{itemize}
\item
  {\bf Сечение}
  \[A\cap B = \{x\ \mid\ x\in A\ \wedge\ x\in B\}.\]
  % Казано по-формално, $A\cap B$ е множеството, за което е изпълнена формулата
  % \[(\forall x)[x \in A\cap B \iff (x\in A\ \wedge\ x \in B)].\]
  % Примери:
  % \begin{itemize}
  % \item
  %   $A \cap A = A$, за всяко множество $A$.
  % \item
  %   $A \cap \emptyset = \emptyset$, за всяко множество $A$.
  % \item
  %   $\{1,\emptyset,\{\emptyset\}\} \cap \{\emptyset\} = \{\emptyset\}$.
  %   \item
  %     $\{1,2,\{1,2\}\} \cap \{1,\{1\}\} = \{1\}$.
  %   \end{itemize}
  \item
    {\bf Обединение}
    \[A\cup B = \{x\ \mid x\in A\ \vee\ x\in B\}.\]
    % $A\cup B$ е множеството, за което е изпълнена формулата
    % \[(\forall x)[x \in A\cup B \iff (x\in A\ \vee\ x \in B)].\]
    % Примери:
    % \begin{itemize}
    % \item
    %   $A \cup A = A$, за всяко множество $A$.
    % \item 
    %   $A \cup \emptyset = A$, за всяко множество $A$.
    % \item
    %   $\{1,2,\emptyset\} \cup \{1,2,\{\emptyset\}\} = \{1,2,\emptyset,\{\emptyset\}\}$.
    % \item
    %   $\{1,2,\{1,2\}\} \cup \{1,\{1\}\} = \{1,2,\{1\},\{1,2\}\}$.
    % \end{itemize}
  \item
    {\bf Разлика}
    \[A\setminus B = \{x\ \mid\ x\in A\ \wedge\ x\not\in B\}.\]
    % $A\setminus B$ е множеството, за което е изпълнена формулата
    % \[(\forall x)[x \in A\setminus B \iff (x\in A\ \wedge\ x \not\in B)].\]
    % Примери:
    % \begin{itemize}
    % \item
    %   $A \setminus A = \emptyset$, за всяко множество $A$.
    % \item 
    %   $A \setminus \emptyset = A$, за всяко множество $A$.
    % \item 
    %   $\emptyset \setminus A = \emptyset$, за всяко множество $A$.
    % \item
    %   $\{1,2,\emptyset\} \setminus \{1,2,\{\emptyset\}\} = \{\emptyset\}$.
    % \item
    %   $\{1,2,\{1,2\}\} \setminus \{1,\{1\}\} = \{2,\{1,2\}\}$.
    % \end{itemize}
  % \item
  %   {\bf Симетрична разлика}
  %   \[A\triangle B = (A\backslash B)\cup (B\backslash A).\]
  %   % $A\triangle B$ е множеството, за което е изпълнена формулата
  %   % \[(\forall x)[x \in A\triangle B \iff [(x\in A\ \wedge\ x \not\in B) \vee (x \in B\ \wedge\ x\not\in A)]].\]
  %   Примери:
  %   \begin{itemize}
  %   \item 
  %     $A \triangle \emptyset = A$, за всяко множество $A$.
  %   \item
  %     $A \triangle A = \emptyset$, за всяко множество $A$.
  %   \item
  %     $A\triangle B = B \triangle A$, за всеки две множества $A$ и $B$.
  %   \item
  %     $\{1,2,\emptyset\} \triangle \{1,2,\{\emptyset\}\} = \{\emptyset\} \cup \{\{\emptyset\}\} = \{\emptyset,\{\emptyset\}\}$.
  %   \item
  %     $\{1,2,\{1,2\}\} \triangle \{1,\{1\}\} = \{2,\{1,2\}\} \cup \{\{1\}\} = \{2,\{1\},\{1,2\}\}$.
  %   \end{itemize}
  \item
    {\bf Степенно множество}
    \[\Ps(A) = \{x\mid x\subseteq A\}.\]
    % $\Ps(A)$ е множеството, за което е изпълнена формулата
    % \[(\forall x)[x \in \Ps(A) \iff (\forall y)[y\in x\rightarrow y \in A]].\]
    Примери:
    \begin{itemize}
    \item 
      $\Ps(\emptyset) = \{\emptyset\}$.
    \item
      $\Ps(\{\emptyset\}) = \{\emptyset,\{\emptyset\}\}$.
    \item
      $\Ps(\{\emptyset,\{\emptyset\}\}) = \{\emptyset,\{\emptyset\},\{\{\emptyset\}\},\{\emptyset,\{\emptyset\}\}\}$.
    \item
      $\Ps(\{1,2\}) = \{\emptyset,\{1\},\{2\},\{1,2\}\}$.
    \end{itemize}
  \end{itemize}
  Нека имаме редица от множества $\{A_1,A_2,\dots,A_n\}$.
  Тогава имаме следните операции:
  \begin{itemize}
  \item
    {\bf Обединение на редица от множества}
    \[\bigcup^{n}_{i=1} A_i = \{x \mid \exists i (1\leq i\leq n\ \&\ x\in A_i)\}.\]
    % \[(\forall x)[x \in \bigcup^n_{i=1}A_i \iff (\exists i)[1 \leq i \leq n\ \wedge\ x \in A_i]].\]
  \item
    {\bf Сечение на редица от множества}
    \[\bigcap^{n}_{i=1} A_i = \{x \mid \forall i (1\leq i\leq n \rightarrow x\in A_i)\}.\]
    % \[(\forall x)[x \in \bigcap^n_{i=1}A_i \iff (\forall i)[1 \leq i \leq n\ \rightarrow\ x \in A_i]].\]
  \end{itemize}

% \begin{example}
%   Нека $A = \{x\in\Nat\mid x > 1\}$ и $B = \{x\in\Nat\mid x>3\}$. Тогава :
%   \begin{itemize}
%     \item
%       $A\cap B = \{x\in\Nat\mid x > 3\}$,
%     \item
%       $A\cup B = \{x\in\Nat\mid x > 1\}$,
%     \item
%       $A\setminus B = \{x\in\Nat\mid 1<x\leq 3\}$,
%     \item
%       $B\setminus A = \emptyset$,
%     % \item
%     %   $A\triangle B = \{x\in\Nat\mid 1<x\leq 3\}$
%     \end{itemize}
% \end{example}


\begin{problem}
  Проверете верни ли са свойствата:
  \begin{enumerate}[a)]
  \item
    $A\subseteq B \iff A\setminus B = \emptyset \iff A\cup B = B \iff A\cap B = A$;
  \item
    $A\setminus \emptyset = A$, $\emptyset\setminus A=\emptyset$, $A\setminus B = B\setminus A$.
  \item
    $A\cap (B\cup A) = A \cap B$;
  \item
    $A\cup(B\cap C) = (A\cup B)\cap(A\cup C)$ и $A \cap (B \cup C) = (A \cup B) \cap (A \cup C)$;
  % \item
  %   $C\subseteq A\ \&\ C\subseteq B \rightarrow C\subseteq A\cap B$;
  % \item
  %   $A\subseteq C\ \&\ B\subseteq C \rightarrow A\cup B\subseteq C$;
  \item
    $A\backslash B = A \iff A\cap B = \emptyset$;
  \item
    $A\backslash B = A\backslash (A\cap B)$ и $A\backslash B = A\backslash (A\cup B)$;
  \item
    $(A\cup B)\setminus C = (A\setminus C) \cup (B\setminus C)$;
  % \item
  %   \marginpar{Не е вярно!}
  %   $A\setminus (B\setminus C) = (A\setminus B)\setminus C$;
  \item
    \marginpar{Закони на Де Морган}
    $C\setminus (A\cup B) = (C\backslash A)\cap(C\backslash B)$ и $C \backslash (A\cap B) = (C\backslash A)\cup(C\backslash B)$
  \item
    $C\backslash(\bigcup^{n}_{i=1} A_i) = \bigcap^{n}_{i=1} (C\backslash A_i)$ и $C \backslash(\bigcap^{n}_{i=1} A_i) = \bigcup^{n}_{i=1} (C\backslash A_i)$;
  \item
    $(A\backslash B)\backslash C = (A\backslash C)\backslash(B \backslash C)$ и $A\backslash (B\backslash C) = (A\backslash B) \cup (A\cap C)$;
  \item
    $A\subseteq B \Rightarrow \Ps(A) \subseteq \Ps(B)$;
  \item
    \marginpar{$X \subseteq A\cup B \stackrel{?}{\Rightarrow} X\subseteq A \vee X \subseteq B$}
    $\Ps(A\cap B) = \Ps(A) \cap \Ps(B)$ и $\Ps(A\cup B) = \Ps(A) \cup \Ps(B)$;
  \end{enumerate}
\end{problem}

За да дадем определение на понятието релация, трябва първо 
да въведем понятието декартово произведение на множества,
което пък от своя страна се основава на понятието наредена двойка.

\subsection*{Наредена двойка}
\index{наредена двойка}
За два елемента $a$ и $b$ въвеждаме опрецията {\bf наредена двойка} $\pair{a,b}$.
Наредената двойка $\pair{a,b}$ има следното характеристичното свойство:
\[a_1 = a_2\ \wedge\ b_1 = b_2\ \iff\ \pair{a_1,b_1} = \pair{a_2,b_2}.\]
Понятието наредена двойка може да се дефинира по много начини, стига да изпълнява харектеристичното свойство.
Ето примери как това може да стане:
\begin{enumerate}[1)]
\item
  \marginpar{Norbert Wiener (1914)}
  Първото теоретико-множествено определение на понятието наредена двойка е
  дадено от Норберт Винер:
  \[\pair{a,b} \df \{\{\{a\},\emptyset\},\{\{b\}\}\}.\]
\item
  \marginpar{Kazimierz Kuratowski (1921)}
  Определението на Куратовски се приема за ,,стандартно'' в наши дни:
  \[\pair{a,b} \df \{\{a\},\{a,b\}\}.\]
\end{enumerate}

\begin{problem}
  Докажете, че горните дефиниции наистина изпълняват харектеристичното свойство за наредени двойки.
\end{problem}

\begin{dfn}
  \marginpar{Пример за индуктивна (рекурсивна) дефиниция}
  Сега можем, за всяко естествено число $n \geq 1$,
  да въведем понятието наредена $n$-орка $\pair{a_1,\dots,a_n}$:
  \begin{align*}
    & \pair{a_1} \df a_1,\\
    & \pair{a_1,a_2,\dots,a_n} \df \pair{a_1,\pair{a_2,\dots,a_n}}
  \end{align*}
\end{dfn}

Оттук нататък ще считаме, че имаме операцията наредена $n$-орка, без да се интересуваме от нейната формална дефиниция.
 
\subsection*{Декартово произведение}
\marginpar{На англ. cartesian product}
\index{декартово произведение}
\marginpar{Считаме, че $(A\times B)\times C = A\times (B\times C) = A\times B \times C$}

За две множества $A$ и $B$, определяме тяхното декартово произведение като
\[A\times B = \{\pair{a,b}\mid a\in A\ \&\ b\in B\}.\]
За краен брой множества $A_1,A_2,\dots,A_n$, определяме
\[A_1\times A_2\times\cdots\times A_n = \{\pair{a_1,a_2,\dots,a_n}\mid a_1 \in A_1\ \&\ a_2\in A_2\ \&\ \dots\ \&\ a_n \in A_n\}.\]

\begin{problem}
  Проверете, че:
  \begin{enumerate}[a)]
  \item
    $A\times(B\cup C) = (A\times B) \cup (A\times C)$.
  \item
    $(A\cup B)\times C = (A\times C)\cup (B\times C)$.
  \item 
    $A\times(B\cap C) = (A\times B) \cap (A\times C)$.
  \item
    $(A \cap B)\times C = (A \times C)\cap(B\times C)$.
  \item 
    $A\times(B\setminus C) = (A\times B) \setminus (A\times C)$.
  \item
    $(A\setminus B)\times C = (A\times C)\setminus (B\times C)$.
  \end{enumerate}
\end{problem}

\subsection*{Основни видове бинарни релации}
% \marginpar{Бинарни релации}

Подмножествата $R$ от вида $R \subseteq A\times A\times\cdots\times A$ се наричат релации.
Релациите от вида $R\ \subseteq\ A\times A$ са важен клас, който ще срещаме често.
Да разгледаме няколко основни видове релации от този клас:
\begin{enumerate}[I)]
\item
  {\bf рефликсивна}, ако
  \[(\forall x\in A)[\pair{x,x}\in R].\]
  Например, релацията $\leq\ \subseteq\ \Nat\times\Nat$ е рефлексивна, защото
  \[(\forall x\in \Nat)[x \leq x].\]
% \item
%   {\bf антирефлексивна}, ако
%   \[(\forall x\in A)[\pair{x,x}\not\in R].\]
%   Например, релацията $<\ \subseteq\ \Nat\times\Nat$ е антирефлексивна, защото
%   \[(\forall x\in \Nat)[x \not< x].\]
\item
  {\bf транзитивна}, ако
  \[(\forall x,y,z\in A)[\pair{x,y}\in R\ \&\ \pair{y,z}\in R \rightarrow \pair{x,z}\in R].\]
  Например, релацията $\leq\ \subseteq\ \Nat\times\Nat$ е транзитивна, защото
  \[(\forall x,y,z\in A)[x \leq y\ \&\ y \leq z\ \rightarrow\ x\leq z].\]
\item
  {\bf симетрична}, ако
  \[(\forall x,y\in A)[\pair{x,y}\in R \rightarrow \pair{y,x}\in R].\]
  Например, релацията $=\ \subseteq\ \Nat\times\Nat$ е рефлексивна, защото
  \[(\forall x,y\in \Nat)[x = y\ \rightarrow\ y = x].\]
\item
  {\bf антисиметрична}, ако
  \[(\forall x,y\in A)[\pair{x,y}\in R\ \&\ \pair{y,x}\in R \rightarrow x = y].\]
  Например, релацията $\leq\ \subseteq\ \Nat\times\Nat$ е антисиметрична, защото
  \[(\forall x,y,z\in A)[x \leq y\ \&\ y \leq x\ \rightarrow\ x = y].\]
% \item
%   {\bf асиметрична}, ако
%   \[(\forall x,y)[\pair{x,y}\in R \rightarrow \pair{y,x}\not\in R].\]
%   Например, релацията $\leq\ \subseteq\ \Nat\times\Nat$ е асиметрична, защото
%   \[(\forall x,y\in \Nat)[x < y\ \rightarrow\ y \not< x].\]
\end{enumerate}

% \begin{remark}
%   Добре е да запомните как се наричат тези основни видове релации,
%   защото ще ги използваме често.
% \end{remark}

% \begin{example}
%   Да обобщим примерите от по-горе.
%   \begin{enumerate}[a)]
%   \item
%     Релацията $\leq\ \subseteq\ \Nat\times\Nat$ е рефлексивна, транзитивна и антисиметрична.
%   \item
%     Релацията $<\ \subseteq\ \Nat\times\Nat$ е антирефлексивна, транзитивна и асиметрична.
%   \item
%     Релацията $=\ \subseteq\ \Nat\times\Nat$ е рефлексивна, транзитивна и симетрична.
%   \end{enumerate}
% \end{example}

\begin{itemize}
\item
  Една бинарна релация $R$ над множеството $A$ се нарича {\bf релация на еквивалентност}, 
  ако $R$ е рефлексивна, транзитивна и симетрична.
\item 
  За всеки елемент $a \in A$, определяме неговия 
  {\bf клас на еквивалентност} относно релацията на еквивалентност $R$ по следния начин:
  \[[a]_R \df \{b\in A \mid \pair{a,b} \in R\}.\]
\end{itemize}

\begin{remark}
  Лесно се съобразява, че за всеки два елемента $a, b\in A$,
  \[\pair{a,b} \in R \iff [a]_R = [b]_R.\]
\end{remark}

\begin{example}
  За всяко естествено число $n\geq 2$, дефинираме релацията $R_n$ като
  \[\pair{x,y}\in R_n \iff x \equiv y\ (\bmod\ n).\]
  Ясно е, че $R_n$ са релации на еквивалентност.
\end{example}


\subsection*{Операции върху бинарни релации}

\begin{enumerate}[I)]
\item
  {\bf Композиция} на две релации $R \subseteq B\times C$ и $P \subseteq A\times B$ е релацията $R\circ P \subseteq A\times C$,
  определена като:
  \[R\circ P \df \{\pair{a,c} \in A\times C \mid (\exists b \in B)[\pair{a,b}\in P\ \&\ \pair{b,c} \in R]\}.\]
\item
  {\bf Обръщане} на релацията $R \subseteq A\times B$ е релацията $R^{-1}\subseteq B\times A$, 
  определена като:
  \[R^{-1} \df \{\pair{x,y} \in B\times A \mid \pair{y,x} \in R\}.\]
\item
  \marginpar{Очевидно е, че $P$ е рефлексивна релация, дори ако $R$ не е.}
  {\bf Рефлексивно затваряне} на релацията $R \subseteq A\times A$ е релацията
  \[P \df R \cup \{\pair{a,a}\mid a \in A\}.\]
\item
  {\bf Итерация} на релацията $R \subseteq A\times A$ дефинираме като за всяко естествено число $n$,
  дефинираме релацията $R^n$ по следния начин:
  \marginpar{Лесно се вижда, че  $R^1 = R$}
  \begin{align*}
    R^0 & \df \{\pair{a,a} \mid a \in A\}\\
    R^{n+1} & \df R^n \circ R.
  \end{align*}
\item
  \marginpar{\ding{45} Проверете, че $R^+$ е транзитивна релация!}
  {\bf Транзитивно затваряне} на $R \subseteq A\times A$ е релацията
  \[R^+ \df \bigcup_{n\geq 1} R^n.\]
\end{enumerate}

\index{$R^\star$}
За дадена релация $R$, с $R^\star$ ще означаваме нейното {\em рефлексивно и транзитивно затваряне}.
От дефинициите е ясно, че \[R^\star = \bigcup_{n\geq 0} R^n.\]

\subsection*{Видове функции}

Функцията $f:A \to B$ е:
\begin{itemize}
\item
  \marginpar{(или $f$ е {\bf обратима})}
  {\bf инекция}\index{функция!инекция}, ако 
  \[(\forall a_1,a_2\in A)[a_1\neq a_2 \rightarrow f(a_1)\neq f(a_2)],\]
  или еквивалентно,
  \[(\forall a_1,a_2\in A)[f(a_1) = f(a_2) \rightarrow a_1 = a_2].\]
\item
  \marginpar{(или $f$ е {\bf върху} $B$)}
  {\bf сюрекция}\index{функция!сюрекция}, ако 
  \[(\forall b\in B)(\exists a\in A)[f(a) = b].\]
\item
  {\bf биекция}\index{функция!биекция}, ако е инекция и сюрекция.
\end{itemize}

\begin{problem}
  \marginpar{Канторово кодиране. Най-добре се вижда като се нарисува таблица}
  Докажете, че $f: \Nat \times \Nat\rightarrow \Nat$ е биекция, където
  \[f(x, y) = \frac{(x+y)(x+y+1)}{2} + x.\]
\end{problem}

\section{Азбуки, думи, езици}

\subsection*{Основни понятия}

\begin{itemize}
\item 
  \index{азбука}
  {\bf Азбука} ще наричаме всяко крайно множество,
  като обикновено ще я означаваме със $\Sigma$.
  \marginpar{Често ще използваме буквите $a$, $b$, $c$ за да означаваме букви.}
  Елементите на азбуката $\Sigma$ ще наричаме {\bf букви} или символи.
\item
  \index{дума}
  {\bf Дума} над азбуката $\Sigma$ е произволна крайна редица от елементи на $\Sigma$.
  Например, за $\Sigma = \{a,b\}$, $aababba$ е дума над $\Sigma$ с дължина $7$.
  С $\abs{\alpha}$ ще означаваме дължината на думата $\alpha$.
  \marginpar{Обикновено ще означаваме думите с $\alpha$, $\beta$, $\gamma$, $\omega$.}
\item
  Обърнете внимание, че имаме единствена дума с дължина $0$.
  Тази дума ще означаваме с $\varepsilon$ и ще я наричаме {\bf празната дума},
  т.е. $\abs{\varepsilon} = 0$.
\item
  С $a^n$ ще означаваме думата съставена от $n$ $a$-та.
  Формалната индуктивна дефиниция е следната:
  \begin{align*}
    a^0 & \df \varepsilon,\\
    a^{n+1} & \df a^na.
  \end{align*}
\item
  Множеството от всички думи над азбуката $\Sigma$ ще означаваме със $\Sigma^\star$.
  Например, за $\Sigma = \{a,b\}$,
  \[\Sigma^\star = \{\varepsilon,a,b,aa,ab,ba,bb,aaa,aab,\dots\}.\]
  Обърнете внимание, че $\emptyset^\star = \{\varepsilon\}$.
% \item
%   {\bf Лексикографска наредба}
\end{itemize}

\subsection*{Операции върху думи}

\begin{itemize}
\item 
  \index{конкатенация}
  Операцията {\bf конкатенация} взима две думи $\alpha$ и $\beta$ и образува 
  новата дума $\alpha\cdot\beta$ като слепва двете думи.
  Например $aba\cdot bb = ababb$.
  Обърнете внимание, че в общия 
  случай $\alpha\cdot\beta \neq \beta\cdot\alpha$. 
  \marginpar{Често ще пишем $\alpha\beta$ вместо $\alpha\cdot\beta$}
  Можем да дадем формална индуктивна дефиниция на операцията конкатенация по
  дължината на думата $\beta$.
  \begin{itemize}
  \item 
    Ако $\abs{\beta} = 0$, то $\beta = \varepsilon$.
    Тогава $\alpha\cdot \varepsilon \df \alpha$.
  \item
    Ако $\abs{\beta} = n+1$, то $\beta = \gamma b$, $\abs{\gamma} = n$.
    Тогава $\alpha\cdot\beta \df (\alpha\cdot\gamma)b$.
  \end{itemize}
\item
  Друга често срещана операция върху думи е {\bf обръщането} на дума.
  Дефинираме думата $\alpha^R$ като обръщането на $\alpha$ по следния начин.
  \begin{itemize}
  \item 
    Ако $\abs{\alpha} = 0$, то $\alpha = \varepsilon$ и $\alpha^R \df \varepsilon$.
  \item
    Ако $\abs{\alpha} = n+1$, то $\alpha = a\beta$, където $\abs{\beta} = n$.
    Тогава $\alpha^R \df (\beta^R)a$.
  \end{itemize}
  Например, $reverse^R = esrever$.
\item
  \index{дума!префикс}
  \index{дума!суфикс}
  Казваме, че думата $\alpha$ е {\bf префикс} на думата $\beta$,
  ако съществува дума $\gamma$, такава че $\beta = \alpha\cdot\gamma$.
  $\alpha$ е {\bf суфикс} на $\beta$, ако $\beta = \gamma\cdot\alpha$, за някоя дума $\gamma$.
\item
  \marginpar{Обърнете внимание, че $\emptyset\cdot A = A\cdot\emptyset = \emptyset$}
  \marginpar{Също така, $\{\varepsilon\}\cdot A = A\cdot\{\varepsilon\} = A$}
  Нека $A$ и $B$ са множества от думи.
  Дефинираме конкатенацията на $A$ и $B$ като
  \[A\cdot B \df \{\alpha\cdot\beta \mid \alpha\in A\ \&\ \beta \in B\}.\]
\item
  Сега за едно множество от думи $A$, дефинираме $A^n$ индуктивно:
  \begin{align*}
    A^0 & \df \{\varepsilon\},\\
    A^{n+1} & \df A^n \cdot A.
  \end{align*}
  Ако $A = \{ab, ba\}$, то
  $A^0 = \{\varepsilon\}$, $A^1 = A$, $A^2 = \{abab, abba, baba, baab\}$.
  Ако $A = \{a,b\}$, то $A^n = \{\alpha \in \{a,b\}^\star \mid \abs{\alpha} = n\}$.
\item
  За едно множеството от думи $A$, дефинираме:
  \marginpar{Операцията $\star$ е известна като звезда на Клини}
  \marginpar{Обърнете внимание, че $\emptyset^\star = \{\varepsilon\}$}
  \begin{align*}
    A^\star & \df \bigcup_{n\geq 0} A^n\\
    & = A^0 \cup A^1 \cup A^2 \cup A^3 \cup \dots\\
    A^+ & \df A\cdot A^\star.
  \end{align*}
\end{itemize}

\begin{problem}
  Проверете:
  \begin{enumerate}[a)]
  \item 
    операцията конкатенация е {\em асоциативна}, т.е. за всеки три думи $\alpha$, $\beta$, $\gamma$,
    \[(\alpha\cdot\beta)\cdot\gamma = \alpha\cdot(\beta\cdot\gamma);\]
  \item
    за множествата от думи $A$, $B$ и $C$,
    \[(A\cdot B)\cdot C = A\cdot (B\cdot C);\]
  \item
    $\{\varepsilon\}^\star = \varepsilon$;
  \item
    за произволно множество от думи $A$,
    $A^\star = A^\star\cdot A^\star$ и $(A^\star)^\star = A^\star$;
  \item
    за произволни букви $a$ и $b$,
    $\{a,b\}^\star = \{a\}^\star\cdot(\{b\}\cdot\{a\}^\star)^\star$.
  \end{enumerate}
\end{problem}


\begin{problem}
  Докажете, че за всеки две думи $\alpha$ и $\beta$ е изпълено:
  \begin{enumerate}[a)]
  \item 
    $(\alpha\cdot\beta)^R = \beta^R\cdot\alpha^R$;
  \item
    $\alpha$ е префикс на $\beta$ точно тогава, когато $\alpha^R$ е суфикс на $\beta^R$;
  \item
    $(\alpha^R)^R = \alpha$;
  \item
    $(\alpha^n)^R = (\alpha^R)^n$, за всяко $n \geq 0$.
  \end{enumerate}
\end{problem}

\begin{problem}
  \marginpar{С други думи, ако $\varepsilon \not\in X$, то $Z = X^\star Y$ е най-малкото решение на уравнението $Z = XZ \cup Y$.}
  Нека $X, Y, Z \subseteq \Sigma^\star$ със свойството, че $Z = XZ \cup Y$.
  \begin{enumerate}[a)]
  \item 
    Докажете, че за всяко $n \in \Nat$, $X^nY \subseteq Z$.
    Заключете, че $X^\star Y \subseteq Z$.
  \item
    Да предположим, че $\varepsilon \not\in X$.
    Докажете, че за всяка дума $\omega \in Z$ е изпълнено, че $\omega \in X^\star Y$.
  \end{enumerate}
\end{problem}

\section*{Библиография}

Повечето книги в тази област започват с уводна глава, в която въвеждат понятията множества, релации и езици.
\begin{itemize}
\item 
  Глава 1 от \cite{rosen}.
\item
  Глава 1 от \cite{papadimitriou}.
\item
  За описанието на думи и азбуки следваме \cite[Глава 2]{kozen}.
\end{itemize}

%%% Local Variables: 
%%% mode: latex
%%% TeX-master: "EAI"
%%% End: 
