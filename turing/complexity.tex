\section{Сложност}

\begin{itemize}
\item
  \index{машина на Тюринг!детерминистично полиномиално ограничена}
  Казваме, че детерминистичната машина на Тюринг $\M$ е {\bf полиномиално ограничена}, ако 
  същестува полином $p(x)$, такъв че няма дума $\omega$,
  за която $\M$ да извършва при вход $\omega$ повече от $p(|\omega|)$ стъпки.
\item
  \index{език!детерминистично полиномиално разрешим}
  Езикът $L$ се нарича {\bf детерминистично полиномиално разрешим},
  ако съществува полиномиално ограниченен детерминистичен разрешител $\M$, за който $L = \L(\M)$.
  Нека
  \[\mathcal{P} \df \{L \subseteq \Sigma^\star \mid L\text{ е полиномиално разрешим с ДМТ}\}.\]
\item
  \index{машина на Тюринг!полиномиално ограничена}
  Казваме, че детерминистичната машина на Тюринг $\M$ е {\bf експоненциално ограничена}, ако 
  същестува полином $p(x)$, такъв че няма дума $\omega$,
  за която $\M$ да извършва при вход $\omega$ повече от $2^{p(|\omega|)}$ стъпки.
\item
  \index{език!детерминистично експоненциално разрешим}
  Езикът $L$ се нарича {\bf детерминистично експоненциално разрешим},
  ако съществува експоненциално ограниченен детерминистичен разрешител $\M$, за който $L = \L(\M)$.
  Нека
  \[\mathcal{EXP} \df \{L \subseteq \Sigma^\star \mid L\text{ е експоненциално разрешим с ДМТ}\}.\]
\item
  \index{машина на Тюринг!недетерминистично полиномиално ограничена}
  Казваме, че недетерминистичната машина на Тюринг $\N$ е {\bf полиномиално ограничена}, ако 
  същестува полином $p(x)$, такъв че няма дума $\omega$,
  за която $\N$ да извършва при вход $\omega$ повече от $p(|\omega|)$ стъпки.
\item
  \index{език!недетерминистично полиномиално разрешим}
  Езикът $L$ се нарича {\bf недетерминистично полиномиално разрешим},
  ако съществува полиномиално ограничена недетерминистичен разрешител $\N$,
  за който $L = \L(\N)$. Нека
  \[\mathcal{NP} \df \{L \subseteq \Sigma^\star \mid L\text{ е полиномиално разрешим с НМТ}\}.\]
\end{itemize}

% \begin{framed}
%   \begin{dfn}
%     \begin{align*}
%       & \mathcal{P} \df \{L \subseteq \Sigma^\star \mid L\text{ е полиномиално разрешим с ДМТ}\};\\
%       & \mathcal{EXP} \df \{L \subseteq \Sigma^\star \mid L\text{ е експоненциално разрешим с ДМТ}\};\\
%       & \mathcal{NP} \df \{L \subseteq \Sigma^\star \mid L\text{ е полиномиално разрешим с НМТ}\}.
%     \end{align*}
%   \end{dfn}
% \end{framed}

\begin{problem}
  Докажете, че класът $\mathcal{P}$ е затворен относно допълнение, обединение, сечение и конкатенация.
\end{problem}

\begin{problem}
  Докажете, че класът $\mathcal{P}$ е затворен относно операцията звезда на Клини.
\end{problem}




\begin{thm}
  $\mathcal{NP} \subseteq \mathcal{EXP}$.
\end{thm}

\begin{proposition}
  За азбука $\Sigma$ от поне две букви, можем да обобщим някои от резултатите от предишните глави:
  \[\texttt{REG} \subsetneqq \texttt{CFG} \subsetneqq \mathcal{P}.\]
\end{proposition}
\begin{hint}
  Езикът $\{a^nb^nc^n \mid n \in \Nat\} \in \mathcal{P}$,
  но не е безконтекстен.
\end{hint}


%%% Local Variables:
%%% mode: latex
%%% TeX-master: "../eai"
%%% End:
