\subsection*{Многолентови машини на Тюринг}
\index{машина на Тюринг!многолентова}
%Това е просто като имаш shift.
%Използват се при недет. машини

Машина на Тюринг с $k$ ленти има същата дефиниция като еднолентова машина на Тюринг
с единствената разлика, че
\[\delta: Q \times \Gamma^k\to Q \times \Gamma^k \times \{L,R,S\}^k.\]
Тук добавяме и възможността главата върху някои от лентите да стои на място.
\marginpar{В \cite[стр. 177]{sipser3} конструкцията е малко по-различна. Там съдържанието на всяка лента се поставя последователно върху една лента, като се разделят със специален символ}
\begin{prop}
  За всяка $k$-лентова машина на Тюринг $\M$ съществува еднолентова машина на Тюринг $\M'$,
  такава че $\L(\M) = \L(\M')$.
\end{prop}
\begin{proof}
  \marginpar{Тук на практика следваме \cite[стр. 162]{hopcroft1}}
  Нека $\M$ е $k$-лентова машина на Тюринг.
  Ще построим еднолентова машина на Тюринг $\M'$, за която $\L(\M) = \L(\M')$.
  Да означим $\hat\Gamma = \{\hat X \mid X \in \Gamma\}$.
  Тогава азбуката на лентата на $\M'$ ще бъде $\Gamma' = (\hat\Gamma \cup \Gamma)^{k}$.
  Сега вместо да имаме $k$ ленти ще имаме една лента, която представлява $k$-орка.
  За да симулираме $\M$, използваме символите $\hat X$ за да маркират позицията на главите на $\M$,
  като във всяка координата на лентата има точно по един символ от вида $\hat X$.
  % С $\$$ ще отблезяваме границите на всяка лента, в която можем да търсим маркера.
  За да определим следващия ход на машината $\M'$, ние трябва да сканираме лентата докато не 
  открием разположението на всичките $k$ на брой маркирани клетки. Тогава симулираме ход на $\M$
  и отново трябва да променим маркираните клетки.
\end{proof}

%%% Local Variables:
%%% mode: latex
%%% TeX-master: "../eai"
%%% End:
