\section{Проблемът за съответствието на Пост}\label{sect:turing:pcp}

\mynote{На англ. \emph{Post's correspondence problem} \cite[стр. 392]{hopcroft2}, но по-добре е обяснено в \cite[стр. 227]{sipser3}. Тези двойки от думи се наричат домино.}

Пример за проблема за съответствието на Пост се нарича всяка крайна редица от елементи на $\Sigma^\star \times \Sigma^\star$,
които ние ще означаваме така:
\[\begin{bmatrix} \alpha_1\\ \beta_1\end{bmatrix},\begin{bmatrix} \alpha_2\\ \beta_2\end{bmatrix},\dots,\begin{bmatrix} \alpha_n\\ \beta_n\end{bmatrix}.\]
Всяка една редица от този вид се нарича {\em пример} за \PCP.
\mynote{Ако $|\alpha_i| = |\beta_i|$ за всяко $i$, то задачата е тривиална.}
Една непразна редица от индекси $i_1,i_2,\dots,i_n$ се нарича {\em решение} на \PCP примера, ако е изпълнено, че:
\[\alpha_{i_1}\alpha_{i_2}\cdots\alpha_{i_n} = \beta_{i_1}\beta_{i_2}\cdots\beta_{i_n}.\]

\begin{problem}
  \mynote{\cite[стр. 239]{sipser3}}
  Намерете решение на следния пример за \PCP:
  \[\begin{bmatrix}ab\\ abab\end{bmatrix},\begin{bmatrix} b\\ a\end{bmatrix},\begin{bmatrix} aba\\ b\end{bmatrix},\begin{bmatrix} aa\\ a\end{bmatrix}.\]
\end{problem}
\begin{solution}
  \[\begin{bmatrix}ab\cdot ab \cdot aba \cdot b \cdot b \cdot aa \cdot aa\\abab \cdot abab \cdot b \cdot a \cdot a \cdot a \cdot a\end{bmatrix}\]
\end{solution}

\subsubsection*{Модифициран проблем за съответствието }

\mynote{Тук искаме винаги да започваме с първото домино.}
Казваме, че \MPCP има решение, ако съществува произволна редица от индекси $i_1,\dots,i_n$ (може и празна), такава че:
\[\alpha_1\alpha_{i_1}\cdots\alpha_{i_n} = \beta_1\beta_{i_1}\cdots\beta_{i_n}.\]

\begin{lemma}
  Съществува алгоритъм, който свежда \MPCP към \PCP.
\end{lemma}
\begin{hint}
  Нека имаме пример за \MPCP:
  \[\begin{bmatrix} \alpha_1\\ \beta_1\end{bmatrix},\begin{bmatrix} \alpha_2\\ \beta_2\end{bmatrix},\dots,\begin{bmatrix} \alpha_k\\ \beta_k\end{bmatrix} .\]
  Нека символите $\star,\$$ не са от $\Sigma$.
  Нека за думата $\alpha = a_1\cdots a_n$ да дефинираме следните операции:
  \begin{align*}
    & \star\alpha = \star a_1 \star a_2\cdots \star a_n\\
    & \alpha\star = a_1\star a_2\star\cdots a_n \star\\
    & \star\alpha\star = \star a_1\star a_2 \star \cdots a_n\star.
  \end{align*}
  Тогава на базата на горния пример за \MPCP, строим пример за \PCP:
  \[\begin{bmatrix*}[l] \star\alpha_1\star\\ \star\beta_1\end{bmatrix*},\begin{bmatrix} \alpha_1\star\\ \star \beta_1\end{bmatrix},\dots,\begin{bmatrix} \alpha_k\star\\ \star\beta_k\end{bmatrix},\begin{bmatrix*}[r] \$\\ \star\$\end{bmatrix*}.\]
  Така ние показахме, че
  \[\MPCP \leq_m \PCP.\]
\end{hint}

\begin{corollary}
  Ако $\PCP$ е разрешим, то $\MPCP$ също е разрешим.
\end{corollary}

Ясно е, че проблемът на Пост е полуразрешим. Сега ще видим, че той не е разрешим.

\begin{framed}
  \begin{theorem}[Е. Пост \cite{pcp}]\index{Пост}
    Проблемът за съответствието на Пост е неразрешим при азбука $\Sigma$ с поне два символа.
  \end{theorem}
\end{framed}
\begin{hint}
  \mynote{Лесно се съобразява, че за азбука $\Sigma$ само с една буква проблемът е разрешим.}
  Нека приемем, че работим с машини на Тюринг, които не движат главата си наляво от левия край на лентата.
  Ще докажем, че $\Luniv \leq_m \MPCP$. Вече знаем, че \MPCP се свежда алгоритмично към \PCP, т.е. $\MPCP \leq_m \PCP$.
  Това означава, че ще опишем работата на тотална изчислима функция $f$, за която
  \[\gamma \in \Luniv \iff f(\gamma) \in \MPCP.\]
  Сега неформално ще опишем работата на функцията $f$.
  Нека фиксираме символа $\sharp \not \in \Gamma$.
  \begin{enumerate}[1)]
  \item
    Нека имаме като вход дума $\gamma = \code{\M}\sharp \alpha$.
  \item
    \mynote{Горната част на доминото се опитва да настигне долната част.}
    Започваме като добавяме за думата $\alpha = a_1\cdots a_n$ над азбуката $\Sigma$ следната двойка
    $\begin{bmatrix*}[l] \sharp\\ \sharp qa_1\cdots a_n\sharp\end{bmatrix*}$.
  \item
    Ако $\delta(q,a) = (p,b,\goright)$, то добавяме двойката
    $\begin{bmatrix*}[l] qa\\ bp\end{bmatrix*}$.
  \item
    Ако $\delta(q,\blank) = (p,b,\goright)$, то добавяме и двойката $\begin{bmatrix*}[l] q\sharp\\ bp\sharp\end{bmatrix*}$.
  \item
    \mynote{Тук е важно, че не позволяваме четящата глава да се мести по-наляво от първата клетка върху която е четящата глава при стартиране на изчислението.}
    Ако $\delta(q,a) = (p,b,\goleft)$, то добавяме двойките
    $\begin{bmatrix*}[l] xqa\\ pxb\end{bmatrix*}$.
  \item
    Ако $\delta(q,\blank) = (p,b,\goleft)$, то добавяме двойките
    $\begin{bmatrix*}[l] xq\sharp\\ pxb\sharp\end{bmatrix*}$.
  \item
    Ако $\delta(q,a) = (p,b,\stay)$, то добавяме двойката
    $\begin{bmatrix*}[l] qa\\ pb\end{bmatrix*}$.
  \item
    за всеки $x \in \Gamma$, добавяме $\begin{bmatrix} x\\ x\end{bmatrix}$.
    Освен това, добавяме и двойката $\begin{bmatrix} \sharp\\ \sharp\end{bmatrix}$.
  \item
    \mynote{Когато достигнем до приемащо състояние, то започваме да трием съдържанието на доминото за да можем да изравним двете части на доминото.}
    За всеки $x \in \Gamma$, добавяме двойката
    $\begin{bmatrix*}[l] x\qaccept\\ \qaccept\end{bmatrix*}$ и $\begin{bmatrix*}[l] \qaccept x\\ \qaccept\end{bmatrix*}$.
  \item
    За да завършим, добавяме двойката
    $\begin{bmatrix*}[r] \qaccept\sharp\sharp\\ \sharp\end{bmatrix*}$.
  \end{enumerate}
\end{hint}

\begin{corollary}
  \label{cor:turing:pcp:ambiguous}
  \mynote{Ясно е, че $\texttt{AMBIG}$ е полуразрешим език, нали?}
  Езикът 
  \[\texttt{AMBIG} = \{\code{G} \mid G \text{ е нееднозначна безконтекстна граматика}\}\]
  е неразрешим.
\end{corollary}
\begin{hint}
  Да разгледаме един пример за $\PCP$ над азбуката $\Sigma$:
  \[\begin{bmatrix} \alpha_1\\ \beta_1\end{bmatrix},\begin{bmatrix} \alpha_2\\ \beta_2\end{bmatrix},\dots,\begin{bmatrix} \alpha_n\\ \beta_n\end{bmatrix}.\]
  По него можем ефективно да построим следната безконтекстна граматика:
  \begin{align*}
    & S \to A\ |\ B\\
    & A \to \alpha_1A c_1\ |\ \alpha_2 A c_2\ |\ \cdots\ |\ \alpha_n A c_n\ |\ \alpha_1c_1\ |\ \alpha_2c_2\ |\ \cdots\ |\ \alpha_nc_n\\
    & B \to \beta_1B c_1\ |\ \beta_2 B c_2\ |\ \cdots\ |\ \beta_n B c_n\ |\ \beta_1c_1\ |\ \beta_2c_2\ |\ \cdots\ |\ \beta_nc_n,
  \end{align*}
  където $c_1,\dots,c_n \not \in \Sigma$.
  Лесно се съобразява, че горният пример за $\PCP$ има решение точно тогава, когато безконтекстната граматика е нееднозначна.
  С други думи, показахме, че
  \[\PCP \leq_m \texttt{AMBIG}.\]
\end{hint}

\begin{corollary}\label{cor:pcp:grammar-intersect}
  \mynote{Ясно е, че $\INTERSECT$ е полуразрешим език, нали? За момента не е ясно дали допълнението на $\INTERSECT$ е полуразрешим език. Това ще разберем по-късно в \Theorem{computations:intersect}.}
  Езикът 
  \[\INTERSECT = \{\code{G_1}\sharp\code{G_2} \mid \L(G_1) \cap \L(G_2) \neq \emptyset \}\]
  е неразрешим.
\end{corollary}
\begin{hint}
  Да разгледаме един пример за $\PCP$ над азбуката $\Sigma$:
  \[\begin{bmatrix} \alpha_1\\ \beta_1\end{bmatrix},\begin{bmatrix} \alpha_2\\ \beta_2\end{bmatrix},\dots,\begin{bmatrix} \alpha_n\\ \beta_n\end{bmatrix}.\]
  По него можем ефективно да построим следните две безконтекстни граматика с правила:
  \begin{align*}
    & S_1 \to \alpha_1S_1 c_1\ |\ \alpha_2 S_1 c_2\ |\ \cdots\ |\ \alpha_n S_1 c_n\ |\ \alpha_1c_1\ |\ \alpha_2c_2\ |\ \cdots\ |\ \alpha_nc_n\\
    & S_2 \to \beta_1S_2 c_1\ |\ \beta_2 S_2 c_2\ |\ \cdots\ |\ \beta_n S_2 c_n\ |\ \beta_1c_1\ |\ \beta_2c_2\ |\ \cdots\ |\ \beta_nc_n,
  \end{align*}
  където $c_1,\dots,c_n \not \in \Sigma$.
  Така показахме, че
  \[\PCP \leq_m \INTERSECT.\]
\end{hint}



%%% Local Variables:
%%% mode: latex
%%% TeX-master: "../eai"
%%% End:
