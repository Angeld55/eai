\section{Примери за разрешими езици}

\begin{example}
  \marginpar{Знаем, че $L$ не е безконтекстен}
  Да разгледаме езика $L = \{a^nb^nc^n \mid n\in\Nat\}$.
 
  Нека да въведем нов символ $d$, с който ще маркираме обработените символи $a$, $b$, $c$.
  Идеята на алгоритъма, който ще разгледаме е да маркира на всяка итерация по едно $a$, $b$, и $c$.
  Той завършва успешно ако всички символи на думата са маркирани.
  Нека първоначално думата е копирана върху лентата и четящата глава е върху първия символ на думата.
  \begin{enumerate}[(1)]
  \item 
    Чете $d$-та надясно по лентата докато срещне първото $a$ и го замества с $d$. Отива на стъпка (2).
    Ако символите свършат (т.е. достигне се $\blank$) преди да се достигне $a$,
    то алгоритъмът завършва успешно.
  \item
    Чете $d$-та надясно по лентата докато срещне първото $b$ и го замества с $d$.
    Отива на стъпка (3).
  \item
    Чете $d$-та надясно по лентата докато срещне първото $c$ и го замества с $d$.
  \item
    Връща четящата глава в началото на лентата, т.е. чете наляво докато не срещне символа $\blank$.
    Връща се в стъпка (1). 
  \end{enumerate}

  Нека сега да видим, че този алгоритъм може да се опише съвсем формално с машина на Тюринг.
  Ще построим машина на Тюринг $\M$, за която $L = \L(\M)$, където
  \begin{itemize}
  \item 
    $\Sigma = \{a,b,c\}$;
  \item
    $\Gamma = \{a,b,c,d,\blank\}$, за някой нов символ $d$;
  \item
    $Q = \{q_1,q_2,\dots,q_5\}$;
  \item
    $q_{accept} = q_5$;
  \item
    частичната функция на преходите $\delta:Q\times\Gamma \to Q\times\Gamma\times\{\goleft,\goright,\stay\}$
    е описана на схемата отдолу.
  \end{itemize}

  \begin{framed}
  \begin{figure}[H]
    \begin{center}
      \begin{tikzpicture}[->,>=stealth,thick,node distance=50pt]
        \tikzstyle{every state}=[circle,minimum size=10pt,auto]
        
        \node[state,initial]    (1) {$q_1$};
        \node[state]            (2) [right of=1]{$q_2$};
        \node[state]            (3) [right of=2]{$q_3$};
        \node[state]            (4) [right of=3]{$q_4$};
        \node[state,accepting]  (5) [below of=1]{$q_5$};
        % \node[state,accepting]  (6) [right of=5]{$6$};
        
        \begin{scope}[every node/.style={scale=.8}]
          \path
          (1) edge [loop above] node [above] {$d;\goright$} (1)
          (1) edge [bend right=15] node [left] {$\blank;\goright$} (5)
          % (1) edge [bend left=15] node [left] {$\{b,c\}$} (6)
          (1) edge [bend left=15] node [above] {$a/d;\goright$} (2)
          (2) edge [bend left=15] node [above] {$b/d;\goright$} (3)
          (2) edge [loop above] node   [above] {$\{a,d\};\goright$} (2)
          (3) edge [bend left=15] node [above] {$c/d;\goleft$} (4)
          (3) edge [loop above] node   [above] {$\{b,d\};\goright$} (3)
          (4) edge [loop right] node   [below right] {$\{a,b,d\};\goleft$} (4)
          (4) edge [in=65,out=115,above] node [above] {$\blank;\goright$} (1);
        \end{scope}
      \end{tikzpicture}
    \end{center}
    \caption{детерминистична частична машина на Тюринг $\M$, за която $\L(\M) = \{a^nb^nc^n \mid n \in \Nat\}$}
  \end{figure}
  \end{framed}

  Например,
  \begin{align*}
    & \delta(q_1, a) = (q_2, d, \goright)\\
    & \delta(q_4, \blank) = (q_1, \blank, \goright)\\
    & \delta(q_2, a) = (q_2, a, \goright).
  \end{align*}

  % Да проследим изчислението на думата $aabbcc$:
  
  % \[_1aabbcc \vdash d_2abbcc \vdash da_2bbcc \vdash dad_3bcc \vdash dadb_3cc \vdash dad_4bdc \vdash da_4dbdc \vdash \cdots \vdash\]
  % \[_4dadbdc \vdash\ _4\blank dadbdc \vdash\ _1dadbdc \vdash d_1adbdc \vdash dd_2dbdc \vdash ddd_2bdc \vdash dddd_3dc \vdash \]
  % \[ ddddd_3c \vdash dddddd_4 \vdash \cdots \vdash\ _4\blank dddddd \vdash\ _1dddddd \vdash \cdots \vdash dddddd_1\blank \vdash dddddd_5\blank.\]

  Съобразете, че тази машина на Тюринг може да се направи тотална като се добави ново състояние $\qreject$
  и за всяка двойка $(q,x)$, за която функцията на преходите не е дефинирана, да сочи към $\qreject = q_6$.
  Така можем да получим {\em тотална} машина на Тюринг за езика $L$, което означава, че 
  $L$ е не само полуразрешим, но {\em разрешим} език.

\begin{framed}
  \begin{figure}[H]
    \begin{center}
      \begin{tikzpicture}[->,>=stealth,thick,node distance=50pt]
        \tikzstyle{every state}=[circle,minimum size=10pt,auto]
        
        \node[state,initial]    (1) {$q_1$};
        \node[state]            (2) [right of=1]{$q_2$};
        \node[state]            (3) [right of=2]{$q_3$};
        \node[state]            (4) [right of=3]{$q_4$};
        \node[state,accepting]  (5) [below of=1]{$q_5$};
        \node[state]            (6) [below of=3]{$q_6$};
        
        \begin{scope}[every node/.style={scale=.8}]
          \path
          (1) edge [loop above] node [above] {$d;\goright$} (1)
          (1) edge [bend right=15] node [left] {$\blank;\stay$} (5)
          % (1) edge [bend left=15] node [left] {$\{b,c\}$} (6)
          (1) edge [bend left=15] node [above] {$a/d;\goright$} (2)
          (2) edge [bend left=15] node [above] {$b/d;\goright$} (3)
          (2) edge [loop above] node [above] {$\{a,d\};\goright$} (2)
          (3) edge [bend left=15] node [above] {$c/d;\goleft$} (4)
          (3) edge [loop above] node [above] {$\{b,d\};\goright$} (3)
          (4) edge [loop right] node [below right] {$\{a,b,d\};\goleft$} (4)
          (4) edge [in=65,out=115,above] node [above] {$\blank;\goright$} (1);

          \path
          (1) edge [dashed, bend right=15] node [left] {$\{b,c\};\stay$} (6)
          (2) edge [dashed, bend left=30] node [left] {$\{c,\blank\};\stay$} (6)
          (3) edge [dashed, bend left=25] node [right] {$\{a,\blank\};\stay$} (6)
          (5) edge [dashed, loop below] node [below] {$\{a,b,c,d,\blank\};\stay$} (5)
          (6) edge [dashed, loop right] node [right] {$\{a,b,c,d,\blank\};\stay$} (6);
        \end{scope}
      \end{tikzpicture}
    \end{center}
  \end{figure}
\end{framed}

\end{example}

\begin{example}
  \marginpar{Да напомним, че този език не е безконтекстен}
  \marginpar{В \cite[стр. 155]{hopcroft1} е дадено по-различно решение. Тук следваме \cite[стр. 173]{sipser3}. Там има малка грешка}
  Да разгледаме езика $L = \{\omega \sharp \omega \mid \omega\in\{a,b\}^\star\}$.
  Нека първо да видим, че можем неформално да опишем алгоритъм, който да разпознава думите на езика $L$.
  Нека една дума е копирана върху лентата и четящата глава е поставена върху първия символ от думата.
  \begin{enumerate}[(1)]
  \item 
    Чете $x$-ове надясно по лентата докато не срещне $a$ или $b$ и го замества с $x$.
    Запомня дали сме срещнали $a$ или $b$.
    Ако вместо $a$ или $b$ срещне $\sharp$, то отива на стъпка $(6)$.
  \item
    Чете $a$-та и $b$-та надясно по лентата докато не стигне $\sharp$. 
  \item
    Чете $c$-то надясно по лентата и всички следващи $x$-ове докато не срещне символа $a$ или $b$.
    Той трябва да е същия символ, който сме запаметили на стъпка $(1)$.
    Заместваме този символ с $x$.
  \item
    Чете $x$-ове наляво по лентата докато не стигне $\sharp$.
  \item
    Чете $a$-та и $b$-та по лентата докато не стигне $x$.
    Поставя четящата глава върху символа точно след първия $x$.
    Отива на стъпка $(1)$.
  \item
    Прочита $\sharp$ надясно по лентата и чете надясно $x$-ове докато не срещне $\blank$.
    Алгоритъмът завършва успешно.
  \end{enumerate}

  Ще построим машина на Тюринг $\M$, за която $L = \L(\M)$.
  \begin{itemize}
  \item 
    $\Sigma = \{a,b,\sharp\}$;
  \item
    $\Gamma = \{a,b,\sharp,x,\blank\}$;
  \item
    $Q = \{q_1,q_2,\dots,q_9\}$;
  \item
    $\qaccept = q_9$;
  \end{itemize}

  \begin{framed}
  \begin{figure}[H]
    \begin{center}
      \begin{tikzpicture}[->,>=stealth,thick,node distance=50pt]
        \tikzstyle{every state}=[circle,minimum size=10pt,auto,scale=.9]
        
        \node[state,initial]    (1) {$q_1$};
        \node[state]            (2) [above right of=1]{$q_2$};
        \node[state]            (3) [below right of=1]{$q_3$};
        \node[state]            (4) [right of=2]{$q_4$};
        \node[state]            (5) [right of=3]{$q_5$};
        \node[state]            (6) [below right of=4]{$q_6$};
        \node[state]            (7) [above of=6]{$q_7$};
        \node[state]            (8) [left of=3]{$q_8$};
        \node[state,accepting]  (9) [below left of=3]{$q_9$};
        
        \begin{scope}[every node/.style={scale=.8}]
          \path
          (1) edge [bend left=15] node [below right] {$a/x;\goright$} (2)
              edge [bend right=15] node [above right] {$b/x;\goright$} (3)
              edge [bend right=15] node [left] {$\sharp;\goright$} (8)
          (2) edge [loop above] node [above] {$\{a,b\};\goright$} (2)
              edge [bend left=15] node [above] {$\sharp;\goright$} (4)
          (3) edge [loop below] node [below] {$\{a,b\};\goright$} (3)
              edge [bend right=15] node [below] {$\sharp;\goright$} (5)
          (4) edge [loop above] node [above] {$x;\goright$} (4)
              edge [bend left=15] node [below left] {$a/x;\goleft$} (6)
          (5) edge [loop below] node [below] {$x;\goright$} (5)
              edge [bend right=15] node [above left] {$b/x;\goleft$} (6)
          (6) edge [loop right] node [right] {$x;\goleft$} (6)
              edge [bend right=15] node [right] {$\sharp;\goleft$} (7)
          (7) edge [loop right] node [right] {$\{a,b\};\goleft$} (7)
              edge [out=130,in=120,above,distance=2.5cm] node [above] {$x;\goright$} (1)
          (8) edge [loop left] node [left] {$x;\goright$} (8)
              edge [bend right=15] node [left] {$\blank;\stay$} (9);
        \end{scope}
      \end{tikzpicture}
    \end{center}
    \caption{детерминистична частична машина на г-н Тюринг $\M$, за която $\L(\M) = \{\omega\sharp\omega \mid \omega \in \{a,b\}^\star\}$}
  \end{figure}
\end{framed}

  Да проследим изчислението на думата $ab\sharp ab$.
  
  \begin{align*}
    (q_1, \underline{a}b\sharp ab) & \vdash (q_2, x\underline{b}\sharp ab) \vdash (q_2,xb\underline{\sharp}ab) \vdash (q_4, xb\sharp\underline{a}b) \vdash (q_6, xb\underline{\sharp}xb)\\
    & \vdash (q_7, x\underline{b}\sharp xb) \vdash (q_7, \underline{x}b\sharp xb) \vdash (q_1, x\underline{b}\sharp xb) \vdash (q_3, xx\underline{\sharp}xb)\\
    & \vdash (q_5, xx\sharp\underline{x}b) \vdash (q_5, xx\sharp x\underline{b}) \vdash (q_6, xx\sharp\underline{x}x) \vdash (q_6, xx\underline{\sharp} xx)\\
    & \vdash (q_7, x\underline{x}\sharp xx) \vdash (q_1, xx\underline{\sharp} xx) \vdash (q_8, xx\sharp\underline{x}x) \vdash (q_8, xx\sharp x\underline{x})\\
    & \vdash (q_8, xx\sharp xx\underline{\blank}) \vdash (q_9,xx\sharp xx\underline{\blank}).
  \end{align*}

  Може лесно да се съобрази, че тази машина на Тюринг може да се допълни до {\em тотална}.
  
\end{example}


Една {\em тотална} функция $f:\Sigma^\star \to \Sigma^\star$ се нарича изчислима с машина на Тюринг $\M$, ако 
за всяка дума $\alpha \in \Sigma^\star$,
\[(\varepsilon, s, \alpha) \vdash^\star_\M (\varepsilon, \qaccept, f(\alpha)).\]
Това означава, че машината на Тюринг $\M$ е тотална.

\[Graph(f) = \{\alpha\sharp f(\alpha) \mid \alpha \in \Sigma^\star\}.\]

\begin{example}
  Нека $\Sigma = \{a,b\}$.
  Да видим защо функцията $f(\alpha) = \alpha\alpha$ е изчислима с машина на Тюринг.
  
  \begin{itemize}
  \item
    $\Sigma = \{a,b\}$;
  \item 
    $\Gamma = \{a,b,x,A,B\}$;
  \end{itemize}
  
  \begin{framed}
  \begin{figure}[H]
    \begin{center}
      \begin{tikzpicture}[->,>=stealth,thick,node distance=70pt]
        \tikzstyle{every state}=[circle,minimum size=10pt,auto,scale=.9]
        
        \node[state]            (1) {$s$};
        \node[state]            (2) [above of=1]{$q_1$};
        \node[state]            (3) [right of=2]{$q_2$};
        \node[state]            (4) [below of=1]{$q_3$};
        \node[state]            (5) [right of=4]{$q_4$};
        \node[state]            (6) [right of=1]{$q_5$};
        \node[state,accepting]  (7) [right of=6]{$q_6$};
        
        \begin{scope}[every node/.style={scale=.8}]
          \path
          (1) edge [bend left=15] node [left] {$a/x;\goright$} (2)
          (2) edge [loop above] node [above] {$\{a,b,A,B\};\goright$} (2)
          (2) edge [bend left=15] node [above] {$\blank/A;\goleft$} (3)
          (3) edge [loop right] node [right] {$\{a,b,A,B\};\goleft$} (3)
          (3) edge [bend right=15] node [right] {$x/a;\goright$} (1)
          (1) edge [bend right=15] node [left] {$b/x;\goright$} (4)
          (4) edge [loop below] node [below] {$\{a,b,A,B\};\goright$} (4)
          (4) edge [bend right=15] node [below] {$\blank/B;\goleft$} (5)
          (5) edge [loop right] node [right] {$\{a,b,A,B\};\goleft$} (5)
          (5) edge [bend left=15] node [right] {$x/b;\goright$} (1)
          (1) edge [loop left] node [left] {$A/a,B/b;\goright$} (1)
          (1) edge [bend left=15] node [above] {$\blank;\goleft$} (6)
          (6) edge [loop below] node [right] {$\{a,b\};\goleft$} (6)
          (6) edge [bend left=15] node [above] {$\blank;\goright$} (7);
        \end{scope}
      \end{tikzpicture}
    \end{center}
  \end{figure}
\end{framed}

Да проследим работата на $\M$ върху думата $ab$:

\begin{align*}
  (s, \underline{a}b) & \vdash (q_1, x\underline{b}) \vdash (q_1,xb\underline{\blank}) \vdash (q_2, x\underline{b}A) \vdash (q_2, \underline{x}bA)\\
                      & \vdash (s, a\underline{b}A) \vdash (q_3, ax\underline{A}) \vdash (q_3, axA\underline{\blank}) \vdash (q_4, ax\underline{A}B)\\
                      & \vdash (q_4, a\underline{x}AB) \vdash (s, ab\underline{A}B) \vdash (s,aba\underline{B}) \vdash (s, abab\underline{\blank})\\
                      & \vdash (q_5, aba\underline{b}) \vdash (q_5, ab\underline{a}b) \vdash (q_5, a\underline{b}ab) \vdash (q_5, \underline{a}bab)\\
                      & \vdash (q_5, \underline{\blank}abab) \vdash (q_6, \underline{a}bab).
\end{align*}

  
\end{example}


{\bf Да се каже нещо за графиката}

%%% Local Variables:
%%% mode: latex
%%% TeX-master: "../eai"
%%% End:
