\subsection*{Примери}
\label{sect:turing:examples}

\begin{extra}
\begin{example}
  \mynote{Вече сме срещали езикът $L$ много пъти. Знаем много добре, че $L$ не е безконтекстен, но е контекстен.}
  Да разгледаме езика $L = \{a^nb^nc^n \mid n\in\Nat\}$.
  Нека да видим защо този език е разрешим.
  Идеята на алгоритъма, който ще разгледаме е да маркира на всяка итерация по един символ $a$, $b$ или $c$.
  Той завършва успешно, ако всички символи на думата са маркирани.
  Да въведем нов символ $x$, с който ще маркираме обработените от входната дума символи $a$, $b$, $c$.
  Нека първоначално думата е копирана върху лентата и четящата глава е върху първия символ на думата.
  \mynote{Алгоритъмът има времева сложност $\mathcal{O}(n^2)$. Ако разгледаме машина на Тюринг с три ленти, то ще получим времева сложност $\mathcal{O}(n)$.}
  \begin{enumerate}[(1)]
  \item 
    Чете $x$-ове надясно по лентата докато срещне първото $a$ и го замества с $x$. Отива на стъпка (2).
    Ако символите свършат (т.е. достигне се $\blank$) преди да се достигне $a$,
    то алгоритъмът завършва успешно.
  \item
    Чете $x$-ове надясно по лентата докато срещне първото $b$ и го замества с $x$.
    Отива на стъпка (3).
  \item
    Чете $x$-ове надясно по лентата докато срещне първото $c$ и го замества с $x$.
  \item
    Връща четящата глава в началото на лентата, т.е. чете наляво докато не срещне символа $\blank$.
    Връща се в стъпка (1). 
  \end{enumerate}

  Нека сега да видим, че този алгоритъм може да се опише съвсем формално с машина на Тюринг.
  Ще построим машина на Тюринг $\M$, за която $L = \L(\M)$, където
  \begin{itemize}
  \item 
    $\Sigma = \{a,b,c\}$;
  \item
    $\Gamma = \{a,b,c,x,\blank\}$, за някой нов символ $x$;
  \item
    $Q = \{q_0,q_1,q_2,q_3,q_4,q_5\}$, като $\qstart = q_0$ и $\qaccept = q_5$;
  \item
    Частичната функция на преходите $\delta:(Q\setminus\{q_5\})\times\Gamma \to Q\times\Gamma\times\{\goleft,\goright,\stay\}$
    е описана на схемата отдолу. Остава да добавим състоянието $\qreject$.
  \end{itemize}

  \begin{framed}
  \begin{figure}[H]
    \begin{center}
      \begin{tikzpicture}[->,>=stealth,thick,node distance=60pt]
        \tikzstyle{every state}=[circle,minimum size=10pt,auto,initial text=начало]
        
        \node[state,initial]    (1) {$q_0$};
        \node[state]            (2) [right of=1]{$q_1$};
        \node[state]            (3) [right of=2]{$q_2$};
        \node[state]            (4) [right of=3]{$q_3$};
        \node[state]            (5) [right of=4]{$q_4$};
        \node[state,accepting]  (6) [above of=1]{$q_5$};
        
        \begin{scope}[every node/.style={scale=.8}]
          \path
          (1) edge [loop below]    node [below] {$x;\goright$} (1)
          (1) edge [bend left=15]  node [left]  {$\blank;\stay$} (6)
          (1) edge [bend left=15]  node [above] {$a/x;\goright$} (2)
          (2) edge [bend left=15]  node [above] {$b/x;\goright$} (3)
          (2) edge [loop above]    node [above] {$\{a,x\};\goright$} (2)
          (3) edge [bend left=15]  node [above] {$c/x;\goright$} (4)
          (4) edge [bend left=15]  node [above] {$\{c,\blank\};\goleft$} (5)
          (3) edge [loop above]    node [above] {$\{b,x\};\goright$} (3)
          (5) edge [loop right]    node [below right] {$\{a,b,x\};\goleft$} (5)
          (5) edge [in=65,out=115,above] node [above] {$\blank;\goright$} (1);
        \end{scope}
      \end{tikzpicture}
    \end{center}
    \caption{Част от детерминистична машина на Тюринг $\M$, за която разрешава езика $\{a^nb^nc^n \mid n \in \Nat\}$. Остава да се добави $\qreject$ и да се попълни $\delta$ функцията до тотална.}
  \end{figure}
  \end{framed}

  Горната схема определя точно функцията на преходите $\delta$. Например,
  \begin{align*}
    & \delta(q_0, a) = (q_1, x, \goright)\\
    & \delta(q_4, \blank) = (q_0, \blank, \goright)\\
    & \delta(q_1, a) = (q_1, a, \goright)\\
    & \delta(q_1, x) = (q_1, x, \goright).
  \end{align*}

  Съобразете, че тази машина на Тюринг може да се направи тотална като се добави ново състояние $q_6 = \qreject$
  и за всяка двойка $(q,z)$, за която функцията на преходите не е дефинирана, да сочи към $\qreject$.
  Така получаваме пълното описание на детерминистична машина на Тюринг $\M$, за която $\L(\M) = L$.
  Лесно се съобразява, че тази машина на Тюринг е {\em тотална}, т.е. за всеки вход $\M$ завършва в $\qaccept$ или $\qreject$.
  Заключаваме, че $L$ е не само полуразрешим, но {\em разрешим} език.

\begin{framed}
  \begin{figure}[H]
    \begin{center}
      \begin{tikzpicture}[->,>=stealth,thick,node distance=60pt]
        \tikzstyle{every state}=[circle,minimum size=10pt,auto,initial text=начало]
        
        \node[state,initial]    (1) {$q_0$};
        \node[state]            (2) [right of=1]{$q_1$};
        \node[state]            (3) [right of=2]{$q_2$};
        \node[state]            (4) [right of=3]{$q_3$};
        \node[state]            (5) [right of=4]{$q_4$};
        \node[state,accepting]  (6) [above of=1]{$q_5$};
        \node[state]            (7) [below of=3]{$q_6$};
        
        \begin{scope}[every node/.style={scale=.8}]
          \path
          (1) edge [loop below]    node [below] {$x;\goright$} (1)
          (1) edge [bend left=15] node [left]  {$\blank;\stay$} (6)
          (1) edge [bend left=15]  node [above] {$a/x;\goright$} (2)
          (2) edge [bend left=15]  node [above] {$b/x;\goright$} (3)
          (2) edge [loop above]    node [above] {$\{a,x\};\goright$} (2)
          (3) edge [bend left=15]  node [above] {$c/x;\goright$} (4)
          (4) edge [bend left=15]  node [above] {$\{c,\blank\};\goleft$} (5)
          (3) edge [loop above]    node [above] {$\{b,x\};\goright$} (3)
          (5) edge [loop right]    node [below right] {$\{a,b,x\};\goleft$} (5)
          (5) edge [in=65,out=115,above] node [above] {$\blank;\goright$} (1)
          (1) edge [dashed, bend right=25] node [left] {$\{b,c\};\stay$} (7)
          (2) edge [dashed, bend left=30] node [left] {$\{c,\blank\};\stay$} (7)
          (3) edge [dashed, bend left=25] node [right] {$\{a,\blank\};\stay$} (7)
          (4) edge [dashed, bend left=45] node [right] {$\{a,b,x\};\stay$} (7)
          (5) edge [dashed, bend left=55] node [right] {$c;\stay$} (7);
        \end{scope}
      \end{tikzpicture}
    \end{center}
    \caption{Детерминистична машина на Тюринг $\M$, която \emph{разрешава} езика $\{a^nb^nc^n \mid n \in \Nat\}$. Отхвърлящото състояние е $q_6$, т.е. $\qreject \df q_6$.}
  \end{figure}
\end{framed}
\end{example}

\begin{example}
  \mynote{Да напомним, че този език не е безконтекстен. В \cite[стр. 155]{hopcroft1} е дадено по-различно решение. Тук следваме \cite[стр. 173]{sipser3}. Там има малка грешка.}
  Да разгледаме езика $L = \{\omega \sharp \omega \mid \omega\in\{a,b\}^\star\}$.
  Първо неформално ще опишем алгоритъм, който да разпознава думите на езика $L$.
  Нека една дума е копирана върху лентата и четящата глава е поставена върху първия символ от думата.
  \begin{enumerate}[(1)]
  \item
    Чете $x$-ове надясно по лентата докато не срещне $a$ или $b$ и го замества с $x$.
    Запомня дали сме срещнали $a$ или $b$.
    Ако вместо $a$ или $b$ срещне $\sharp$, то отива на стъпка $(6)$.
    \mynote{Това запаметяване става в състоянията.}
  \item
    Чете $a$-та и $b$-та надясно по лентата докато не стигне $\sharp$. 
  \item
    Чете символа $\sharp$ надясно по лентата и всички следващи $x$-ове докато не срещне символа $a$ или $b$.
    Той трябва да е същия символ, който сме запаметили на стъпка $(1)$.
    Заместваме този символ с $x$.
  \item
    Чете $x$-ове наляво по лентата докато не стигне $\sharp$.
  \item
    Чете $a$-та и $b$-та по лентата докато не стигне $x$.
    Поставя четящата глава върху символа точно след първия $x$.
    Отива на стъпка $(1)$.
  \item
    Прочита $\sharp$ надясно по лентата и чете надясно $x$-ове докато не срещне $\blank$.
    Алгоритъмът завършва успешно.
  \end{enumerate}
  \mynote{Обърнете внимание, че този алгоритъм има времева сложност $\mathcal{O}(n^2)$. Ще видим в Пример~\ref{ex:multitape:omega:sharp:omega}, че ако разгледаме двулентова машина на Тюринг, то имаме алгоритъм със сложност $\mathcal{O}(n)$.}
  Ще построим машина на Тюринг $\M$, за която $L = \L(\M)$.
  \begin{itemize}
  \item 
    $\Sigma \df \{a,b,\sharp\}$;
  \item
    $\Gamma \df \{a,b,\sharp,x,\blank\}$;
  \item
    $Q \df \{q_0,q_1,\dots,q_8\}$, като $\qstart \df q_0$ и $\qaccept \df q_8$;
  \end{itemize}

  \begin{framed}
    \begin{figure}[H]
      \begin{center}
        \begin{tikzpicture}[->,>=stealth,thick,node distance=60pt]
          \tikzstyle{every state}=[circle,minimum size=10pt,auto,scale=.9,initial text=начало]
          
          \node[state,initial]    (1) {$q_0$};
          \node[state]            (2) [above right of=1]{$q_1$};
          \node[state]            (3) [below right of=1]{$q_2$};
          \node[state]            (4) [right of=2]{$q_3$};
          \node[state]            (5) [right of=3]{$q_4$};
          \node[state]            (6) [below right of=4]{$q_5$};
          \node[state]            (7) [above of=6]{$q_6$};
          \node[state]            (8) [left of=3]{$q_7$};
          \node[state,accepting]  (9) [below left of=3]{$q_8$};
          
          \begin{scope}[every node/.style={scale=.8}]
            \path
            (1) edge [bend left=15]  node [below right] {$a/x;\goright$} (2)
                edge [bend right=15] node [above right] {$b/x;\goright$} (3)
                edge [bend right=15] node [left] {$\sharp;\goright$} (8)
            (2) edge [loop above]    node [above] {$\{a,b\};\goright$} (2)
                edge [bend left=15]  node [above] {$\sharp;\goright$} (4)
            (3) edge [loop below]    node [below] {$\{a,b\};\goright$} (3)
                edge [bend right=15] node [below] {$\sharp;\goright$} (5)
            (4) edge [loop above]    node [above] {$x;\goright$} (4)
                edge [bend left=15]  node [below left] {$a/x;\goleft$} (6)
            (5) edge [loop below]    node [below] {$x;\goright$} (5)
                edge [bend right=15] node [above left] {$b/x;\goleft$} (6)
            (6) edge [loop right]    node [right] {$x;\goleft$} (6)
                edge [bend right=15] node [right] {$\sharp;\goleft$} (7)
            (7) edge [loop right]    node [right] {$\{a,b\};\goleft$} (7)
                edge [out=130,in=120,above,distance=2.5cm] node [above] {$x;\goright$} (1)
            (8) edge [loop left]     node [left] {$x;\goright$} (8)
                edge [bend right=15] node [left] {$\blank;\stay$} (9);
          \end{scope}
        \end{tikzpicture}
      \end{center}
      \caption{Част от машина на Тюринг $\M$, която разрешава езика $\{\omega\sharp\omega \mid \omega \in \{a,b\}^\star\}$.}
    \end{figure}
  \end{framed}
  
  Да проследим изчислението на думата $ab\sharp ab$.  
  \begin{align*}
    (q_0, \underline{a}b\sharp ab\blank) & \vdash_\M (q_1, x\underline{b}\sharp ab\blank) \vdash_\M (q_1,xb\underline{\sharp}ab\blank) \vdash_\M (q_3, xb\sharp\underline{a}b\blank)\\
                                   & \vdash_\M (q_5, xb\underline{\sharp}xb\blank) \vdash_\M (q_6, x\underline{b}\sharp xb\blank) \vdash_\M (q_6, \underline{x}b\sharp xb\blank)\\
                                   & \vdash_\M (q_0, x\underline{b}\sharp xb\blank) \vdash_\M (q_2, xx\underline{\sharp}xb\blank) \vdash_\M (q_4, xx\sharp\underline{x}b\blank)\\
                                   & \vdash_\M (q_4, xx\sharp x\underline{b}\blank) \vdash_\M (q_5, xx\sharp\underline{x}x\blank) \vdash_\M (q_5, xx\underline{\sharp} xx\blank)\\
                                   & \vdash_\M (q_6, x\underline{x}\sharp xx\blank) \vdash_\M (q_0, xx\underline{\sharp} xx\blank) \vdash_\M (q_7, xx\sharp\underline{x}x\blank)\\
                                   & \vdash_\M (q_7, xx\sharp x\underline{x}\blank) \vdash_\M (q_7, xx\sharp xx\underline{\blank}) \vdash_\M (q_8,xx\sharp xx\underline{\blank}).
  \end{align*}
  Може лесно да се съобрази, че тази машина на Тюринг може да се допълни до {\em тотална}.
\end{example}

\begin{problem}
  Докажете, че езикът $L = \{\ \omega \cdot \omega \mid \omega\in\{a,b\}^\star\ \}$ е разрешим.
\end{problem}

\end{extra}

%%% Local Variables:
%%% mode: latex
%%% TeX-master: "../eai"
%%% End:
