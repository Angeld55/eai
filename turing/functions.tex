\section{Изчислими функции}

\mynote{Възможно е да се дефинира и за двулентова машина на Тюринг, като $f(\alpha)$ ще бъде върху втората лента.}
Една {\em тотална} функция $f:\Sigma^\star \to \Sigma^\star$ се нарича изчислима с машина на Тюринг $\M$, ако 
за всяка дума $\alpha \in \Sigma^\star$,
\[(\varepsilon, \qstart, \alpha\blank) \vdash^\star_\M (\blank^m, \qaccept, f(\alpha)\blank^k), \text{ за някои }m,k \in \Nat.\]
Това означава, че машината на Тюринг $\M$ винаги завършва. Лесно се съобразява, че езикът
$\texttt{Graph}(f) = \{\ \alpha\sharp f(\alpha) \mid \alpha \in \Sigma^\star\ \}$ е разрешим.

\begin{problem}
  Докажете, че съществуват функции от вида $f:\Sigma^\star\to\Sigma^\star$, които не са изчислими с машина на Тюринг.
\end{problem}
\begin{hint}
  Всяка машина на Тюринг може да се кодира с естествено число.
  Това означава, че съществуват изброимо безкрайно много машини на Тюринг.
  От друга страна, съществуват неизброимо много функции от вида $f:\Sigma^\star \to \Sigma^\star$.
\end{hint}

\begin{extra}
\begin{example}
  \mynote{При двулентова машина на Тюринг тази задача е много по-лесна и сложността ще бъде $\mathcal{O}(n)$ вместо $\mathcal{O}(n^2)$.}
  Да разгледаме функцията $f:\{1\}^\star \to \{1\}^\star$
  дефинирана като
  \[f(1^n) = 1^{2n}.\]
  Да видим защо $f$ е изчислима с машина на Тюринг.
  \begin{framed}
    \begin{figure}[H]
      \begin{center}
        \begin{tikzpicture}[->,>=stealth,thick,node distance=50pt]
          \tikzstyle{every state}=[circle,minimum size=10pt,auto,scale=.7]
          
          \node[state,initial below]    (1) {$q_0$};
          \node[state]            (2) [right of=1]{$q_1$};
          \node[state]            (3) [right of=2]{$q_2$};
          \node[state]            (4) [right of=3]{$q_3$};
          \node[state]            (5) [right of=4]{$q_4$};
          \node[state]            (6) [right of=5]{$q_5$};
          \node[state]            (7) [right of=6]{$q_6$};
          \node[state]            (8) [right of=7]{$q_7$};
          \node[state]            (9) [right of=8]{$q_8$};
          \node[state]            (10) [right of=9]{$q_9$};
          \node[state]            (11) [below of=8]{$q_{10}$};
          \node[state,accepting]  (12) [below of=11]{$q_{11}$};
          
          \begin{scope}[every node/.style={scale=.8}]
            \path
            (1) edge [bend left=15] node [above] {$1;\goleft$} (2)
            (1) edge [bend right=30] node [above] {$\blank;\stay$} (12)
            % (2) edge [bend left=15] node [above] {$1;\goleft$} (3)
            (2) edge [bend right=15] node [below] {$\{1,\blank\};\goleft$} (3)
            (3) edge [bend left=15] node [above] {$\blank/1;\goleft$} (4)
            (4) edge [bend left=15] node [above] {$\blank/1;\goright$} (5)
            (5) edge [loop below] node [below] {$1;\goright$} (5)
            (5) edge [bend left=15] node [above] {$\blank;\goright$} (6)
            (6) edge [loop below] node [below] {$1;\goright$} (6)
            (6) edge [bend left=15] node [above] {$\blank;\goleft$} (7)
            (7) edge [bend left=15] node [above] {$1/\blank;\goleft$} (8)
            (8) edge [bend left=15] node [above] {$1;\goleft$} (9)
            (9) edge [loop below] node [below] {$1;\goleft$} (9)
            (9) edge [bend right=15] node [below] {$\blank;\goleft$} (10)
            (10) edge [loop below] node [below] {$1;\goleft$} (10)
            (10) edge [out=140,in=60, above] node [below] {$\blank;\goright$} (2)
            (8) edge [] node [right] {$\blank;\goleft$} (11)
            (11) edge [loop left] node [left] {$1;\goleft$} (11)
            (11) edge [] node [right] {$\blank;\goright$} (12);
          \end{scope}
        \end{tikzpicture}
        \caption{Машина на Тюринг за функцията $f(1^n) = 1^{2n}$.}
      \end{center}
    \end{figure}
  \end{framed}

% \begin{align*}
%   (q_0, \underline{1}1\blank) & \vdash (q_1, \underline{\blank}11\blank) \vdash  (q_2, \underline{\blank} \blank 11 \blank) \vdash  (q_3, \underline{\blank} 1 \blank 11\blank)\\
%                         & \vdash (q_4, 1\underline{1}\blank 11\blank) \vdash (q_4, 11 \underline{\blank} 11\blank) \vdash (q_5, 11\blank \underline{1}1\blank)\\
%                         & \vdash \cdots \vdash (q_7, 11 \blank \underline{1}\blank) \vdash \cdots
% \end{align*}

\end{example}

\begin{example}
  \mynote{Това пак става много по-лесно с двулентова машина на Тюринг и сложността пак ще бъде $\mathcal{O}(n)$ вместо $\mathcal{O}(n^2)$.}
  Да видим защо тоталната функция $f:\{a,b\}^\star \to \{a,b\}^\star$, дефинирана като
  $f(\alpha) = \alpha\cdot\alpha$ е изчислима с машина на Тюринг.
  
  \begin{itemize}
  \item
    $\Sigma \df \{a,b\}$;
  \item 
    $\Gamma \df \{a,b,x,A,B\}$;
  \item
    $\qstart \df q_0$ и $\qaccept \df q_6$
  \end{itemize}
  \begin{framed}
  \begin{figure}[H]
    \begin{center}
      \begin{tikzpicture}[->,>=stealth,thick,node distance=70pt]
        \tikzstyle{every state}=[circle,minimum size=10pt,auto,scale=.9]
        
        \node[state]            (1) {$q_0$};
        \node[state]            (2) [above of=1]{$q_1$};
        \node[state]            (3) [right of=2]{$q_2$};
        \node[state]            (4) [below of=1]{$q_3$};
        \node[state]            (5) [right of=4]{$q_4$};
        \node[state]            (6) [right of=1]{$q_5$};
        \node[state,accepting]  (7) [right of=6]{$q_6$};
        
        \begin{scope}[every node/.style={scale=.8}]
          \path
          (1) edge [bend left=15] node [left] {$a/x;\goright$} (2)
          (2) edge [loop above] node [above] {$\{a,b,A,B\};\goright$} (2)
          (2) edge [bend left=15] node [above] {$\blank/A;\goleft$} (3)
          (3) edge [loop right] node [right] {$\{a,b,A,B\};\goleft$} (3)
          (3) edge [bend right=15] node [right] {$x/a;\goright$} (1)
          (1) edge [bend right=15] node [left] {$b/x;\goright$} (4)
          (4) edge [loop below] node [below] {$\{a,b,A,B\};\goright$} (4)
          (4) edge [bend right=15] node [below] {$\blank/B;\goleft$} (5)
          (5) edge [loop right] node [right] {$\{a,b,A,B\};\goleft$} (5)
          (5) edge [bend left=15] node [right] {$x/b;\goright$} (1)
          (1) edge [loop left] node [left] {$A/a,B/b;\goright$} (1)
          (1) edge [bend left=15] node [above] {$\blank;\goleft$} (6)
          (6) edge [loop below] node [right] {$\{a,b\};\goleft$} (6)
          (6) edge [bend left=15] node [above] {$\blank;\goright$} (7);
        \end{scope}
      \end{tikzpicture}
      \caption{Машина на Тюринг за $f(\alpha) = \alpha\cdot \alpha$.}
    \end{center}
  \end{figure}
\end{framed}

Да проследим работата на $\M$ върху думата $ab$.
Първо копираме добавяме $AB$ и така лентата съдържа $abAB$. 
След това заменяме $A$ с $a$ и $B$ с $b$. Така най-накравя полуваме върху лентата думата $abab$.

\begin{align*}
  (q_0, \underline{a}b\blank) & \vdash_\M (q_1, x\underline{b}\blank) \vdash_\M (q_1,xb\underline{\blank}) \vdash_\M (q_2, x\underline{b}A) \vdash_\M (q_2, \underline{x}bA)\\
                        & \vdash_\M (q_0, a\underline{b}A) \vdash_\M (q_3, ax\underline{A}) \vdash_\M (q_3, axA\underline{\blank}) \vdash_\M (q_4, ax\underline{A}B)\\
                        & \vdash_\M (q_4, a\underline{x}AB) \vdash_\M (q_0, ab\underline{A}B) \vdash_\M (q_0,aba\underline{B}) \vdash_\M (q_0, abab\underline{\blank})\\
                        & \vdash_\M (q_5, aba\underline{b}) \vdash_\M (q_5, ab\underline{a}b) \vdash_\M (q_5, a\underline{b}ab) \vdash_\M (q_5, \underline{a}bab)\\
                        & \vdash_\M (q_5, \underline{\blank}abab) \vdash_\M (q_6, \underline{a}bab).
\end{align*}

Не можем директно да започнем да копираме $\alpha$, защото така няма да знаем къде е края на първото копие на $\alpha$.
Това можем да направим като първо запишем на лентата $\alpha \sharp \alpha$ и след това второто копие на $\alpha$ го изместим с една позиция наляво.
\end{example}

\begin{example}
  \mynote{Изискваме $f(\alpha)$ да започва с $1$ за да може $f$ да бъде функция, т.е. $f(\alpha)$ е най-късият двоичен запис на числото $\bin{\alpha} + 1$.}
  Да разгледаме тоталната функция $f:\{0,1\}^\star \to 1\cdot\{0,1\}^\star$, 
  където \[\bin{f(\alpha)} \df \bin{\alpha} + 1.\]
  Нека да видим, че тази функция е изчислима с машина на Тюринг.

  \begin{itemize}
  \item 
    $\Sigma \df \{0,1\}$;
  \item
    $\Gamma \df \{0,1,\blank\}$;
  \item
    $\qstart = q_0$ и $\qaccept \df q_4$.
  \end{itemize}

  \begin{framed}
  \begin{figure}[H]
    \begin{center}
      \begin{tikzpicture}[->,>=stealth,thick,node distance=70pt]
        \tikzstyle{every state}=[circle,minimum size=10pt,auto,scale=.9]
        
        \node[state,initial below]    (0) {$q_0$};
        \node[state]            (1) [right of=0]{$q_1$};
        \node[state]            (2) [right of=1]{$q_2$};
        \node[state]            (3) [right of=2]{$q_3$};
        \node[state,accepting]  (4) [right of=3]{$q_4$};
        
        \begin{scope}[every node/.style={scale=.8}]
          \path
          (0) edge [loop above] node [above] {$0/\blank;\goright$} (0)
          (0) edge [bend left=15] node [above] {$1;\goright$} (1)
          (0) edge [bend right=30] node [below] {$\blank;\stay$} (2)
          (1) edge [loop above] node [above] {$\{0,1\};\goright$} (1)
          (1) edge [bend left=15] node [above] {$\blank;\goleft$} (2)
          (2) edge [loop above] node [above] {$1/0;\goleft$} (2)
          (2) edge [bend left=15] node [above] {$0/1;\goleft$} (3)
          (2) edge [bend right=30] node [below] {$\blank/1;\stay$} (4)
          (3) edge [loop above] node [above] {$\{0,1\};\goleft$} (3)
          (3) edge [bend left=15] node [above] {$\blank;\goright$} (4);
        \end{scope}
      \end{tikzpicture}
    \end{center}
    \caption{Машина на Тюринг изчисляваща $f$, за която $\bin{f(\alpha)} = \bin{\alpha} + 1$.}
  \end{figure}
\end{framed}

Да проследим изчислението на $\M$ върху вход $01011$.

\begin{align*}
  (q_0, 0\underline{1}011\blank) & \vdash_\M (q_0,\blank\underline{1}011\blank) \vdash_\M (q_1, \blank 1\underline{0}11\blank) \vdash_\M (q_1, \blank 10\underline{1}1\blank)\\
                           & \vdash_\M (q_1, \blank 101\underline{1}\blank) \vdash_\M (q_1, \blank 1011\underline{\blank}) \vdash_\M (q_2, \blank 101\underline{1}\blank)\\
                           & \vdash_\M (q_2, \blank 10\underline{1}0\blank) \vdash_\M (q_2, \blank 1\underline{0}00\blank) \vdash_\M (q_3, \blank \underline{1}100\blank)\\
                           & \vdash_\M (q_3, \underline{\blank}1100\blank) \vdash_\M (q_4, \blank\underline{1}100\blank).
\end{align*}
\end{example}


\begin{problem}
  Да разгледаме азбуката $\Sigma = \{0,1,\dots,k-1\}$, където $k > 2$.
  Да разгледаме тоталната функция 
  \[f:\Sigma^\star \to (\Sigma\setminus\{0\})\cdot\Sigma^\star,\]
  дефинирана като
  \[\ov{f(\alpha)}_{(k)} = \ov{\alpha}_{(k)} + 1.\]
  Дефинирайте машина на Тюринг $\M$, която изчислява функцията $f$.
\end{problem}

\end{extra}

%%% Local Variables:
%%% mode: latex
%%% TeX-master: "../eai"
%%% End:
