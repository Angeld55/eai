\subsection*{Кодиране на машина на Тюринг}

\mynote{Тук е удобно да индексираме от $1$ вместо от $0$.}
Да приемем, че:
\begin{itemize}
\item
  $Q = \{q_1,q_2,\dots,q_n\}$, където $n \geq 2$;
\item
  $q_1 = \qstart$, $q_2 = \qaccept$ и $q_3 = \qreject$.
\item
  $\Sigma = \{x_1,\dots,x_\ell\}$;
\item
  $\Gamma = \{x_1,x_2,\dots,x_s\}$, където $\ell < s$ и $x_s = \blank$;
\item
  $D_1 = \stay$, $D_2 = \goleft$, $D_3 = \goright$;
\end{itemize}

Така можем да кодираме преходите на детерминистична машина на Тюринг като думи.
\mynote{Аналогично така можем да кодираме и преходите на недетерминистична машина на Тюринг.}
Да разгледаме прехода $\delta(q_i,x_j) = (q_k,x_t,D_m)$.
Кодираме този преход със следната дума:
\[0^i10^j10^k10^t10^m.\]
Да обърнем внимание, че в този двоичен код няма последователни единици и той 
започва и завършва с нула.

За да кодираме една машина на Тюринг $\M$ е достатъчно да кодираме функцията на преходите $\delta$.
Понеже $\delta$ е крайна функция, нека с числото $r$ да означим броя на всички възможни преходи.
По описания по-горе начин, нека $\texttt{code}_i$ е числото в двоичен запис, получено за $i$-тия преход на $\delta$.
Тогава кодът на $\M$ е следното число в двоичен запис:
\[\code{\M} \df 111 0^\ell 11\ \texttt{code}_1\ 11\ \texttt{code}_2\ 11\ \cdots\ 11\ \texttt{code}_r\ 111.\]

Лесно се съобразява, че за две машини на Тюринг $\M$ и $\M'$ с различни функции на преходите, имаме $\code{\M} \neq \code{\M'}$.
Ще означаваме с $\M_\omega$ машината на Тюринг, чийто код е $\omega$, т.е. $\code{\M_\omega} = \omega$.

\mynote{По същия начин можем да дефинираме и код на краен автомат и стеков автомат.}

\begin{problem}
  Докажете, че езикът 
  \[L_{\texttt{code}} = \{\ \omega \in \{0,1\}^\star \mid \omega \text{ е код на машина на Тюринг } \}\]
  е разрешим.
\end{problem}

%%% Local Variables:
%%% mode: latex
%%% TeX-master: "../eai"
%%% End:
