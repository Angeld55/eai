\subsection{Кодиране на машина на Тюринг}

\mynote{Тук е важно, че започваме да индексираме от $1$ вместо от $0$.}
Нека тук да приемем, че:
\begin{itemize}
\item
  $Q = \{q_1,q_2,\dots,q_n\}$;
\item
  $\Gamma = \{X_1,X_2,\dots,X_s\}$; 
\item
  $D_1 = \stay$, $D_2 = \goleft$, $D_3 = \goright$;
\item
  $\qstart = q_1$, $\qaccept = q_2$ и $\qreject = q_3$.
\end{itemize}

Така можем да кодираме преходите на машина на Тюринг като думи.
Да разгледаме прехода $\delta(q_i,X_j) = (q_k,X_\ell,D_m)$.
Кодираме този преход със следната дума:
\[0^i10^j10^k10^\ell10^m.\]
Да обърнем внимание, че в този двоичен код няма последователни единици и той 
започва и завършва с нула.

За да кодираме една машина на Тюринг $\M$ е достатъчно да кодираме функцията на преходите $\delta$.
Понеже $\delta$ е крайна функция, нека с числото $r$ да означим броя на всички възможни преходи.
По описания по-горе начин, нека $\texttt{code}_i$ е числото в двоичен запис, получено за $i$-тия преход на $\delta$.
Тогава кодът на $\M$ е следното число в двоичен запис:
\[\code{\M} \df 111\ \texttt{code}_1\ 11\ \texttt{code}_2\ 11\ \cdots\ 11\ \texttt{code}_r\ 111.\]
\begin{itemize}
\item
  Лесно се съобразява, че за две машини на Тюринг $\M$ и $\M'$ с различни функции на преходите, имаме $\code{\M} \neq \code{\M'}$.
% \item
%   Ще казваме, че числото $r\in\Nat$ е {\bf код на} $\M$, ако $r$, записано в двоичен запис представлява думата $\code{\M}$.
%   Оттук нататък, когато пишем $\M_r$, ще имаме предвид машината на Тюринг с код $r$.
% \item
%   Ясно е, че не всяко естествено число е код на машина на Тюринг, но по дадено число $n$
%   има ефективна процедура, която ни казва дали $n$ е код на машина на Тюринг или не.
% \item
  % С $\pair{\M,\omega}$ ще означаваме кода на $\M$ при вход $\omega$ е числото с двоичен запис описанието на $\M$ и след това прикрепена думата $\omega$.
  % При едно число $r = \pair{M,\omega}$, лесно се намира кода на $\M$.
  % Просто започваме да четем двоичния запис на $r$ докато не срещнем за втори път $111$.
  % След това започва думата $\omega$.
% \item
%   Да въведем означението $\M_i$ за произволно ествестено число $i$.
%   Ако $i$ е код на машина на Тюринг $\M$, то $\M_i \df \M$.
%   Ако $i$ не е код на машина на Тюринг, то $\M_i$ е машина на Тюринг с празна функция на преходите.
\end{itemize}

\begin{problem}
  Докажете, че следните езици са разрешими:
  \begin{itemize}
  \item 
    $L = \{\ \code{\M} \mid \M\text{ е машина на Тюринг } \}$;
  \item 
    $L = \{\ \code{\M} \mid \M\text{ е детерминистична машина на Тюринг }\}$.
  \end{itemize}
\end{problem}

% \begin{remark}
%   По-късно ще докажем, че следният език {\bf не} е разрешим:
%   \[L_{\texttt{tot}} = \{\ \code{\M} \mid \M\text{ е разрешител }\}.\]
% \end{remark}


%%% Local Variables:
%%% mode: latex
%%% TeX-master: "../eai"
%%% End:
