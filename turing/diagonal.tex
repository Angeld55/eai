\section{Диагоналният език}

\newcommand{\Luniv}{L_{\texttt{univ}}}
\newcommand{\Lhalt}{L_{\texttt{halt}}}
\newcommand{\Laccept}{L_{\texttt{accept}}}

\index{език!неполуразрешим}
\begin{important}
  \begin{theorem}\label{th:diagonal}
    Диагоналният език
    \[\Ldiag \df \{\ \omega \in \{0,1\}^\star \mid \omega \text{ е код на машина на Тюринг и }\omega \not\in L(\M_\omega)\ \}\]
    не се разпознава от машина на Тюринг, т.е. $\Ldiag$ {\bf не} е полуразрешим език.
  \end{theorem}
\end{important}
\begin{proof}
  \mynote{Това е версия на диагоналния метод на Кантор, с чиято помощ се доказва, че реалните числа са неизброимо много, т.е. има повече реални числа отколкото естествените.}
  Да допуснем, че $\Ldiag$ се разпознава от машина на Тюринг $\D$, т.е. $\Ldiag = \L(\D)$.
  Тогава да видим какво имаме за думата $\code{\D}$:
  \begin{align*}
    & \code{\D} \in \Ldiag \implies \code{\D} \in \L(\D) \implies \code{\D} \not\in \Ldiag,\\
    & \code{\D} \not\in \Ldiag \implies \code{\D} \not\in \L(\code{\D}) \implies \code{\D} \in \Ldiag.
  \end{align*}
  Достигаме до противоречие.
  \mynote{Тук е добре една безкрайна таблица да се нарисува.}
\end{proof}

\index{език!полуразрешим}
\begin{important}
  \begin{proposition}\label{pr:diagonal:accept}
    Езикът 
    \[\Laccept \df \{\ \omega \in \{0,1\}^\star \mid \text{$\omega$ е код на машина на Тюринг и }\omega \in \L(\M_\omega)\ \}\]
    е полуразрешим, но не е разрешим.
  \end{proposition}  
\end{important}
\begin{hint}
  Лесно се съобразява, че $\Laccept$ е полуразрешим.
  Дефинираме (многолентова) машина на Тюринг $\M'$, която работи по следния начин:
  \begin{itemize}
  \item
    вход дума $\alpha$;
  \item 
    $\M'$ проверява дали $\alpha$ има вида $\code{\M}$,
    за някоя машина на Тюринг $\M$;
  \item
    Ако $\alpha$ няма вида $\code{\M}$,
    то $\M'$ завършва като отхвърля думата $\alpha$.
  \item
    Ако $\alpha = \code{\M}$, 
    то $\M'$ симулира работата на $\M$ върху $\alpha$. Тогава:
    \begin{itemize}
    \item 
      Ако $\M$ завърши след краен брой стъпки като приема $\alpha$,
      то $\M'$ приема $\alpha$.
    \item
      Ако $\M$ завърши след краен брой стъпки като отхвърля $\alpha$,
      то $\M'$ отхвърля $\alpha$.
    \item
      Ако $\M$ никога не завършва върху $\alpha$,
      то $\M'$ също никога не завършва върху $\alpha$.
    \end{itemize}
  \end{itemize}
  Получаваме, че
  \[\alpha \in \Laccept \iff \alpha \in \L(\M'),\]
  откъдето следва, че $\Laccept$ е полуразрешим език.

  Ако допуснем, че $\Laccept$ е разрешим,
  то езикът $L_{\texttt{code}} \setminus \Laccept = \Ldiag$ би бил разрешим, 
  което е противоречие, защото $\Ldiag$ не е е дори полуразрешим.
\end{hint}

\begin{corollary}
  Съществуват полуразрешим език $L$, за който $\ov{L}$ не е полуразрешим.
\end{corollary}

\begin{problem}
  Докажете, че езикът
  \[\Lhalt = \{\omega \in \{0,1\}^\star \mid \omega\text{ е код на машина на Тюринг и }\M_\omega\text{ спира при вход }\omega\}\]
  е полуразрешим, но не е разрешим.
\end{problem}

%%% Local Variables:
%%% mode: latex
%%% TeX-master: "../eai"
%%% End:
