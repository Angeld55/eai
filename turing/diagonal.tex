\subsection*{Диагоналният език}


\index{език!неполуразрешим}
\begin{important}
  \begin{theorem}\label{th:diagonal}
    Диагоналният език
    \[\Ldiag \df \{\ \omega \in \{0,1\}^\star \mid \omega \text{ е код на машина на Тюринг и }\omega \not\in L(\M_\omega)\ \}\]
    не се разпознава от машина на Тюринг, т.е. $\Ldiag$ {\bf не} е полуразрешим език.
  \end{theorem}
\end{important}
\begin{proof}
  \mynote{Това е версия на диагоналния метод на Кантор, с чиято помощ се доказва, че реалните числа са неизброимо много, т.е. има повече реални числа отколкото естествените.}
  Да допуснем, че $\Ldiag$ се разпознава от машина на Тюринг $\M$, т.е. $\Ldiag = \L(\M)$.
  Тогава да видим какво имаме за думата $\code{\M}$:
  \begin{align*}
    & \code{\M} \in \Ldiag \implies \code{\M} \in \L(\M) \implies \code{\M} \not\in \Ldiag,\\
    & \code{\M} \not\in \Ldiag \implies \code{\M} \not\in \L(\code{\M}) \implies \code{\M} \in \Ldiag.
  \end{align*}
  Достигаме до противоречие.
  \mynote{Тук е добре една безкрайна таблица да се нарисува.}
\end{proof}

\index{език!полуразрешим}
\begin{important}
  \begin{proposition}\label{pr:diagonal:accept}
    Езикът 
    \[\Laccept \df \{\ \omega \in \{0,1\}^\star \mid \text{$\omega$ е код на машина на Тюринг и }\omega \in \L(\M_\omega)\ \}\]
    е полуразрешим, но не е разрешим.
  \end{proposition}  
\end{important}
\begin{hint}
  Лесно се съобразява, че $\Laccept$ е полуразрешим.
  Дефинираме машина на Тюринг $\M'$, която, при вход произволна дума $\omega$, работи по следния начин:
  \begin{enumerate}[(1)]
  \item
    \mynote{Това можем да го направим, защото знаем, че $\Lcode$ е разрешим.}
    $\M'$ проверява дали $\omega$ е код на машина на Тюринг $\M_\omega$.
  \item
    Ако $\omega$ е код на машина на Тюринг $\M_\omega$, 
    то $\M'$ симулира работата на $\M_\omega$ върху $\omega$.
  \item
    Ако след краен брой стъпки симулацията завърши с резултат, че $\M_\omega$ приема думата $\omega$,
    то $\M'$ също завършва като приеме $\omega$.
  \end{enumerate}
  Получаваме, че за всяка дума $\omega$ е изпълнена еквивалентността
  \[\omega \in \Laccept \iff \omega \in \L(\M'),\]
  откъдето следва, че $\Laccept$ е полуразрешим език.

  Ако допуснем, че $\Laccept$ е разрешим,
  то езикът $L_{\texttt{code}} \setminus \Laccept = \Ldiag$ би бил разрешим, 
  което е противоречие, защото $\Ldiag$ не е е дори полуразрешим.
\end{hint}

\begin{corollary}
  Съществува полуразрешим език $L$, за който $\ov{L}$ не е полуразрешим.
\end{corollary}
\begin{hint}
  Вземете езика $L = \Laccept$. Знаем, че $L$ е полуразрешим, но не е разрешим.
  Ако допуснем, че $\ov{L}$ е полуразрешим, то тогава от \hyperref[th:turing:kleene-post]{теоремата на Клини-Пост}
  би следвало, че $L$ е разрешим, което е противоречие.
\end{hint}

\begin{important}
  \begin{proposition}
    \label{pr:diagonal:halt}
    Докажете, че езикът
    \[\Lhalt = \{\omega \in \{0,1\}^\star \mid \omega\text{ е код на машина на Тюринг и }\M_\omega\text{ спира при вход }\omega\}\]
    е полуразрешим, но не е разрешим.
  \end{proposition}    
\end{important}
\begin{hint}
  Лесно се вижда, че $\Lhalt$ е полуразрешим.
  Да допуснем, че съществува машина на Тюринг $\M_{\texttt{halt}}$, която е разрешител за $\Lhalt$.
  Ще покажем, че тогава можем да построим разрешител $\M'$ за $\Laccept$, което би било противоречие.
  Машината на Тюринг $\M'$ би работила така върху произволен вход $\omega$:
  \begin{enumerate}[(1)]
  \item
    Ако $\omega$ не е код на машина на Тюринг, то $\M'$ директно отхвърляме $\omega$.
  \item
    Ако $\omega$ е код на машина на Тюринг $\M_\omega$, то $\M'$ симулира работата на $\M_{\texttt{halt}}$
    върху $\omega$.
  \item
    Ако симулацията завърши с резултат, че $\M_{\texttt{halt}}$  отхвътрля думата $\omega$, то
    $\M'$ завършва като отхвърля думата $\omega$.
  \item
    \mynote{Щом $\omega \in \Lhalt$, то знаем, че рано или късно тази симулация ще завърши в някое състояние.}
    Ако симулацията завърши с резултат, че $\M_{\texttt{halt}}$ приема думата $\omega$, то
    $\M'$ симулира работата на $\M_\omega$ върху $\omega$, докато тя завърши в състояние $q$.
    \begin{itemize}
    \item
      Ако $q = \qaccept$ на $\M_\omega$, то $\M'$ завършва като приеме думата $\omega$.
    \item
      В противен случай, $\M'$ завършва като отхвърля думата $\omega$.
    \end{itemize}
  \end{enumerate}
  
  
\end{hint}
%%% Local Variables:
%%% mode: latex
%%% TeX-master: "../eai"
%%% End:
