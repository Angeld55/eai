\subsection{Диагоналният език $L_{\texttt{diag}}$}

% Нека $\omega_0,\omega_1,\dots,\omega_n,\dots$ е каноничната подредба на всички думи над азбуката $\{0,1\}$.
% Да разгледаме безкрайната таблица $\{a_{ij} \mid i,j \in \Nat\}$, където:
% \begin{align*}
%   a_{ij} = 
%   \begin{cases}
%     1, & \text{ ако $j$ е код на М.Т. и }\omega_i \in L(\M_j), \\
%     0, & \text{ ако $j$ не е код на М.Т. или } \omega_i \not\in L(\M_j).
%   \end{cases}
% \end{align*}

% Идеята е да вземем $0$-ите по диагонала на тази таблица.

\begin{framed}
  \begin{thm}
    Езикът 
    \[L_{\texttt{diag}} \df \{\code{\M} \mid \text{$\M$ е М.Т. и }\code{\M} \not\in L(\M)\}\]
    не се разпознава от машина на Тюринг, т.е. $L_{\texttt{diag}}$ {\bf не} е полуразрешим език.
  \end{thm}
\end{framed}
\begin{proof}
  Да допуснем, че $L_{\texttt{diag}}$ се разпознава от машина на Тюринг, т.е. 
  \[L_{\texttt{diag}} = \L(\M),\] за някоя машина на Тюринг $\M$.
  Тогава:
  \begin{align*}
    & \code{\M} \in L_{\texttt{diag}} \implies \code{\M} \in \L(\M) \implies \code{\M} \not\in L_{\texttt{diag}},\\
    & \code{\M} \not\in L_{\texttt{diag}} \implies \code{\M} \not\in \L(\code{\M}) \implies \code{\M} \in L_{\texttt{diag}}.
  \end{align*}
  Достигаме до противоречие.
\end{proof}

\begin{prop}
  Езикът 
  \[L_{\texttt{halt}} \df \{\code{\M} \mid \text{$\M$ е М.Т. и }\code{\M} \in \L(\M)\}\]
  е полуразрешим и не е разрешим.
\end{prop}
\begin{hint}
  Лесно се съобразява, че $L_{\texttt{halt}}$ е полуразрешим.
  Ако допуснем, че $L_{\texttt{halt}}$ е разрешим,
  то 
  \[L_{\texttt{diag}} = (\{0,1\}^\star \setminus L_{\texttt{halt}}) \cap \{\code{\M} \mid \text{$\M$ е М.Т.}\}\]
  е разрешим език, което е противоречие.
\end{hint}


%%% Local Variables:
%%% mode: latex
%%% TeX-master: "../eai"
%%% End:
