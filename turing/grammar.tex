\section{Неограничени граматики}
\index{граматика!неограничена}

% \begin{definition}
%   \mynote{На англ. unrestricted grammar \cite[стр. 220]{hopcroft1}. Според йерархията на Чомски, това е граматика от тип 0.}
%   Граматиката $G = (V,\Sigma,R,S)$
%   се нарича неограничена граматика, 
%   ако правилата $R$ са от вида $\alpha \to \beta$,
%   където $\alpha,\beta \in (V\cup\Sigma)^\star$.
% \end{definition}

\begin{lemma}
  За всеки полуразрешим език $L$, $L = \L(G)$, за някоя неограничена граматика $G$.  
\end{lemma}
\begin{proof}
  Нека $L = \L(\M)$, където 
  \[\M = \TM\] е детерминистична машина на Тюринг,
  като искаме лентата да е безкрайна само отдясно и входната дума $\alpha$ е
  поставена в началото на лентата.
  Ще построим граматика $G = \CFG$, където 
  \[V = ((\Sigma\cup\{\varepsilon\})\times\Gamma) \cup \{A_1,A_2,A_3\}.\]
  Правилата на $G$ са следните:
  \begin{enumerate}[1)]
  \item 
    $A_1 \to sA_2$;
  \item
    $A_2 \to [a,a]A_2$, за всяка $a\in\Sigma$;
  \item
    $A_2 \to A_3$;
  \item
    $A_3 \to [\varepsilon,\blank]A_3$;
  \item
    $A_3 \to \varepsilon$;
  \item
    $q[a,X] \to [a,Y]p$, за всяка $a \in \Sigma\cup\{\varepsilon\}$, всяко $q\in Q$, $X,Y \in\Gamma$, 
    за които $\delta(q,X) = (p,Y,\goright)$;
  % \item
  %   $q[a,X] \to p[a,Y]$, за всяка $a \in \Sigma\cup\{\varepsilon\}$, всяко $q\in Q$, $X,Y \in\Gamma$, 
  %   за които $\delta(q,X) = (p,Y,N)$;
  \item
    $[b,Z]q[a,X] \to p[b,Z][z,Y]$, за всяко $X,Y,Z \in \Gamma$, $a,b\in\Sigma\cup\{\varepsilon\}$, $q\in Q$,
    за които $\delta(q,X) = (p,Y,\goleft)$;
  \item
    $[a,X]q \to qaq$, $q[a,X] \to qaq$, $q \to \varepsilon$, за всяко $a\in\Sigma\cup\{\varepsilon\}$, $X\in\Gamma$,
    и $q \in F$.
  \end{enumerate}
  
  Лесно се вижда, че, използвайки правилата 1) и 2), за всяко $n$, имаме
  \[A_1 \to^\star s[a_1,a_1]\cdots[a_n,a_n]A_2,\]
  където $a_i \in \Sigma$.

  Нека $\M$ приема думата $\alpha = a_1\cdots a_n$.
  Това означава, че за някое $m$, $\M$ използва не повече от $m$ клетки от лентата отдясно на входната дума.
  Ясно е, че имаме
  \[A_1 \to^\star s[a_1,a_1]\cdots[a_n,a_n][\varepsilon,\blank]^m.\]
  Оттук нататък, можем да използваме само правилата 6), 7), 8), докато не срещнем финално състояние.
  С индукция по броя на стъпки в $\M$, можем да докаже, че ако е изпълнено
  $(\varepsilon,s,a_1\cdots a_n) \vdash^\star_\M (X_1\cdots X_{r-1},q,X_r\cdots X_l)$, 
  то \[s[a_1,a_1]\dots[a_n,a_n][\varepsilon,\blank]^m \rightarrow^\star_G [a_1,X_1]\cdots[a_{r-1},X_{r-1}]q[a_r,X_r]\cdots[a_{n+m},X_{n+m}],\]
  където $a_1,\dots,a_n \in \Sigma$, $a_{n+1},\dots,a_{n+m} = \varepsilon$, $X_1,\dots,X_{n+m} \in \Gamma$ и
  $X_{l+1} = X_{l+2} = \dots = X_{n+m} = \blank$.
  
  Най-накрая, ако $q \in F$, то можем да използваме правилата от 9) и да докажем, че
  \[[a_1,X_1]\cdots[a_{t-1},X_{t-1}]q[a_t,X_t]\cdots[a_{n+m},X_{n+m}] \rightarrow^\star_G a_1\cdots a_n.\]
  
  Така доказахме, че ако $\alpha \in \L(\M)$, то $\alpha \in \L(G)$, т.е. $\L(\M) \subseteq \L(G)$.
  За да докажем обратната посока, трябва да направим подобни разсъждения.
\end{proof}

\begin{lemma}
  Ако $L = \L(G)$, където $G$ е неограничена граматика, то $L$ е полуразрешим език.
\end{lemma}
\mynote{Доказателствата в \cite{hopcroft1} и \cite{papadimitriou} са различни}
\begin{proof}
  $\M$ ще бъде недетерминистична машина с три ленти.
  \begin{enumerate}[1)]
  \item
    Записваме входната дума $\omega$ на първата лента на $\M$.
    Тя никога не се променя.
  \item
    На втората лента ще имаме думата $\gamma \in (V\cup\Sigma)^\star$.
    В началото $\gamma := S$.
  \item 
    Недетерминистично избираме правило $\alpha \to \beta$ от граматиката $G$.
  \item
    Недетерминистично избираме $\gamma_0,\gamma_1 \in (V\cup\Sigma)^\star$, за които 
    $\gamma = \gamma_0\alpha\gamma_1$.
    Тогава $\gamma := \gamma_0\beta\gamma_1$.
    Ако няма такива $\gamma_0$ и $\gamma_1$, то $\M$ ,,зацикля'' - текущият опит за извеждане на $\omega$ пропада.
  \item
    Сравняваме съдържанието на първите две ленти, т.е. проверяваме дали $\omega = \gamma$.
    Ако $\omega = \gamma$, то спираме и казваме, че $\M$ разпознава думата $\omega$.
    Ако $\omega \neq \gamma$, то се връщаме на стъпка 3).
  \end{enumerate}

  \begin{algorithm}[H]
  \caption{}
%  \label{alg:}
  \begin{algorithmic}[1]
    \State $\gamma:= S$
    \ForAll{$\alpha\to\beta \in R$}
    \If{$(\exists \gamma_0,\gamma_1\in (V\cup\Sigma)^\star)[\gamma = \gamma_0\alpha\gamma_1]$}
    \State $\gamma := \gamma_0\beta\gamma_1$
    \Else ...
    \EndIf
    \EndFor
  \end{algorithmic}
\end{algorithm}

\end{proof}

%%% Local Variables:
%%% mode: latex
%%% TeX-master: "../eai"
%%% End:
