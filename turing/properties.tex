\section{Основни свойства}

\begin{prop}
  Ако $L$ е разрешим език, то $\ov{L}$ е разрешим език.
\end{prop}
\begin{hint}
  Нека $L = \L(\M)$, където $\M$ е тотална машина на Тюринг.
  Нека $\M'$ е същата като $\M$, само със сменени $\qaccept$ и $\qreject$ състояния.
  Тогава $\ov{L} = \L(\M')$.
\end{hint}

\begin{prop}
  Ако $L_1$ и $L_2$ са разрешими езици, то $L_1 \cup L_2$ е разрешим език.
\end{prop}
\begin{hint}
  Нека $L_1 = \L(\M_1)$ и $L_2 = \L(\M_2)$.
  Симулираме двете изчисления едновременно.
  Ако едната машина достигне accept състоянието си, то връщаме accept.
\end{hint}

\begin{prop}
  Ако $L_1$ и $L_2$ са полуразрешими езици, то $L_1 \cup L_2$ е полуразрешим език.
\end{prop}

\begin{prop}
  Ако $L_1$ и $L_2$ са разрешими езици, то $L_1 \cap L_2$ е разрешим език.
\end{prop}
\begin{hint}
  Нека $L_1 = \L(\M_1)$ и $L_2 = \L(\M_2)$.
  Симулираме двете изчисления едновременно.
  Ако и двете машини достигнат accept състоянията си, то връщаме accept.
\end{hint}

\index{Клини-Пост}
\begin{framed}
  \begin{thm}[Клини-Пост]
    $L$ и $\ov{L}$ са полуразрешими езици точно тогава, когато $L$ е разрешим език.
  \end{thm}
\end{framed}
\begin{hint}
  Посоката $(\Rightarrow)$ е ясна.
  За посоката $(\Leftarrow)$, нека $L = \L(\M_1)$ и $\ov{L} = \L(\M_2)$.
  Симулираме едновременно и двете изчисления.
  Знаем със сигурност, че точно едно от тях ще завърши в accept състояние.
  Ако това е $\M_1$, връщаме accept.
  Ако това е $\M_2$, връщаме reject.
\end{hint}

%%% Local Variables:
%%% mode: latex
%%% TeX-master: "../eai"
%%% End:
