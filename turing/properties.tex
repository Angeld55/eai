\section{Основни свойства}

\begin{proposition}
  Ако $L$ е разрешим език над азбуката $\Sigma$.
  Тогава $\Sigma^\star \setminus L$ също е разрешим език.
\end{proposition}
\begin{hint}
  Нека $L = \L(\M)$, където $\M$ е тотална машина на Тюринг.
  Нека $\M'$ е същата като $\M$, само със сменени $\qaccept$ и $\qreject$ състояния.
  Тогава $\M'$ също е тотална машина на Тюринг и $\ov{L} = \L(\M')$.
\end{hint}

\begin{proposition}
  Ако $L_1$ и $L_2$ са разрешими езици, то $L_1 \cup L_2$ е разрешим език.
\end{proposition}
\begin{hint}
  Нека $L_1 = \L(\M_1)$ и $L_2 = \L(\M_2)$.
  Строим нова машина на Тюринг $\M$, която при вход думата $\alpha$
  симулира едновременно изчисленията на $\M_1$ и $\M_2$ върху $\alpha$.
  Това можем да направим като приемем, че $\M$ има две ленти - една за лентата на $\M_1$ и една за лентата на $\M_2$,
  като състоянията на $\M$ ще бъдат елементи на $Q_1 \times Q_2$.
  Ако една от двете машини достигне своето приемащо състояние, то $\M$ приема думата $\alpha$.
  Ако и двете машини достигнат своите отхвърлящи състояния, то $\M$ отхвърля думата $\alpha$.
\end{hint}

\begin{cor}
  Всяко крайно обединение на разрешими езици е разрешим език.
\end{cor}

\begin{proposition}
  Ако $L_1$ и $L_2$ са полуразрешими езици, то $L_1 \cup L_2$ е полуразрешим език.
\end{proposition}

\begin{proposition}
  Ако $L_1$ и $L_2$ са разрешими езици, то $L_1 \cap L_2$ е разрешим език.
\end{proposition}
\begin{hint}
  Нека $L_1 = \L(\M_1)$ и $L_2 = \L(\M_2)$.
  Строим нова машина на Тюринг $\M$, която при вход думата $\alpha$
  симулира едновременно изчисленията на $\M_1$ и $\M_2$ върху $\alpha$.
  Ако и двете машини достигнат до приемащите си състояния, то $\M$ приема думата $\alpha$.
  Ако поне една от двете машини достигне до отхвърлящо състояние, то $\M$ отхвърля думата $\alpha$.
\end{hint}

\begin{cor}
  Всяко крайно сечение на разрешими езици е разрешим език.
\end{cor}

\begin{framed}
  \begin{thm}
    Разрешимите езици са затворени относно операциите обединение, сечение, допълнение.
  \end{thm}
\end{framed}


\index{Клини-Пост}
\begin{framed}
  \begin{thm}[Клини-Пост]
    $L$ и $\ov{L}$ са полуразрешими езици точно тогава, когато $L$ е разрешим език.
  \end{thm}
\end{framed}
\begin{hint}
  Посоката $(\Leftarrow)$ е ясна.
  За посоката $(\Rightarrow)$, нека $L = \L(\M_1)$ и $\ov{L} = \L(\M_2)$.
  Строим нова машина на Тюринг $\M$, която при вход думата $\alpha$ симулира едновременно изчисленията на $\M_1$ и $\M_2$ върху $\alpha$.
  Знаем със сигурност, че точно едно от двете симулирани изчисления ще завърши в приемащо състояние.
  Ако това е $\M_1$, то $\M$ приема $\alpha$.
  Ако това е $\M_2$, то $\M$ отхвърля $\alpha$.
\end{hint}

%%% Local Variables:
%%% mode: latex
%%% TeX-master: "../eai"
%%% End:
