\begin{theorem}[Грейбах 1963]
  \index{Грейбах}
  \marginpar{\cite[стр. 205]{hopcroft1}}
  \marginpar{на англ. Sheila Greibach}
  Нека $\mathcal{C}$ е клас от езици, за който съществува ефективно кодиране $\code{L}$ на езиците в $\mathcal{C}$ и който е:
  \marginpar{По дадена дума $\omega$ можем ефективно да проверим дали тя кодира език от $\mathcal{C}$ или не.}
  \begin{itemize}
  \item 
    ефективно затворен относно обединение;
  \item
    ефективно затворен относно конкатенация с регулярен език;
  \item
    "$= \Sigma^\star$" е неразрешим за достатъчно голяма $\Sigma$.
  \end{itemize}
  \marginpar{Съществуват езици от $\mathcal{C}$, които не притежават свойството $P$ и такива, които го притежават.}
  Нека $P$ е нетривиално свойство на $\mathcal{C}$, което е изпълнено за всеки регулярен език и ако $L \in P$,
  то $L/_a \in P$, където
  \[L/_a = \{\omega \mid \omega a \in L\}.\]
  Тогава езикът $\{\code{L} \mid P(L)\ \&\ L \in \mathcal{C}\}$ е неразрешим.
\end{theorem}
\begin{hint}
  Да фиксираме език $L_0 \in \Cs$, за който не е изпълнено свойството $P$.
  Нека да приемем, че $L_0 \subseteq \Sigma^\star$, която е достатъчно голяма азбука, за която
  въпроса ``$= \Sigma^\star$'' е неразрешим.
  За произволен език $L \subseteq \Sigma^\star$, да разгледаме езика
  \[\hat{L} \df L_0 \sharp \Sigma^\star\ \cup\ \Sigma^\star \sharp L.\]
  Ясно е, че $\hat{L}\in \mathcal{C}$, защото $\mathcal{C}$ е ефективно затворен относно конкатенация с регулярен език и относно обединение. 
  Първо ще докажем, че: 
  \begin{equation}
    \label{eq:2}
    L = \Sigma^\star\ \iff\ \code{\hat{L}} \in P.
  \end{equation}

  \begin{itemize}
  \item 
    Ако $L = \Sigma^\star$, то $\hat{L}$ е регулярен, защото тогава
    $\hat{L} = \Sigma^\star \sharp \Sigma^\star$ е очевидно регулярен и от избора на $P$, $\code{\hat{L}} \in \mathcal{C}$.
  \item
    \marginpar{Ако $\code{L} \in P$ , то за $L/_\beta \df \{\alpha \mid \alpha\beta \in L\}$ е изпълнено $P$.}  
    Ако $L \neq \Sigma^\star$, то нека да фиксираме дума $\omega \not\in L$.
    Ако допуснем, че $\code{\hat{L}} \in P$, то езикът
    за езикът $\hat{L}/_{\sharp\omega} = L_0$ също ще е изпълнено свойството $P$, което е противоречие с избора на $L_0$.
  \end{itemize}

  От (\ref{eq:2}) следва, че $P$ е разрешимо свойство точно тогава, когато въпросът ''$=\Sigma^\star$'' за езиците от $\mathcal{C}$ е разрешим, което е противоречие.
\end{hint}

\begin{corollary}
  Езикът
  \[\texttt{Reg} = \{\code{G} \mid G\text{ е безконтекстна граматика и }\L(G)\text{ е регулярен език}\}\]
  е неразрешим.
\end{corollary}
\begin{proof}
  Ясно е, че имаме ефективно кодиране на безконтекстните граматики $\code{G}$ и освен това те са
  ефективно затворени относно конкатенация с регулярен език и относно обединение.
  Вече знаем от ..., че $= \Sigma^\star$ за безконтекстни граматики е неразрешим за достатъчно голяма азбука $\Sigma$.
  Тогава от теоремата на Грейбах следва, че $\texttt{Reg}$ е неразрешим език.
\end{proof}


%%% Local Variables:
%%% mode: latex
%%% TeX-master: "../eai"
%%% End:
