\section{Критерии за полуразрешимост}

Вече знаем, че на практика всички интересни въпроси за машини на Тюринг не са разрешими.
Сега да видим какво можем да кажем за позитивната част на въпросите за машини на Тюринг.

\begin{lemma}\label{lem:rice-shapiro:finite}
  \mynote{Това означава, че ако $\texttt{Code}(\Ss)$ е полуразрешим език, то всеки език $L_0 \in \Ss$ притежава краен подезик, който също принадлежи на $\Ss$.}
  Нека $\Ss$ е свойство на полуразрешимите езици.
  Ако съществува безкраен език $L_0 \in \Ss$, който няма крайно подмножество в $\Ss$,
  то $\texttt{Code}(\Ss)$ {\em не е} полуразрешим език.  
\end{lemma}
\begin{hint}
  Нека $L_0 = \L(\M_0)$ като $L_0 \in \Ss$, но всеки краен подезик на $L_0$ не принадлежи на $\Ss$.
  Ще докажем, че:
  \[\Ldiag \leq_m \texttt{Code}(\Ss).\]
  Ще дефинираме тотална изчислима функция $f$, която при вход думата $\omega \in \{0,1\}^\star$ работи по следния начин:
  \begin{itemize}
  \item
    Ако $\omega$ не е код на машина на Тюринг, то $f(\omega) \df \omega$.
  \item
    Ако $\omega$ е код на машината на Тюринг $\M_\omega$, то тогава $f(\omega) \df \code{\M'}$,
    където $\M'$ работи така:
    \begin{itemize}
    \item 
      вход думата $\alpha$;
    \item
      $\M'$ симулира работата на $\M_\omega$ върху думата $\omega$:
      \begin{itemize}
      \item
        ако $\M_\omega$ завърши за по-малко от $|\alpha|$ на брой стъпки като \emph{приеме} $\omega$, 
        то $\M'$ завършва веднага като \emph{отхвърля} $\alpha$;
      \item
        в противен случай, $\M'$ симулира работата на $\M_0$ върху $\alpha$.
      \end{itemize}
    \end{itemize}
    Така получаваме, че 
    \begin{align*}
      \L(\M') = 
      \begin{cases}
        \{\alpha \in L_0 \mid \abs{\alpha} < s_0\}, & \text{ако } \M_\omega \text{ приема }\omega\\
        L_0, & \text{ако }\M_\omega \text{ не приема }\omega,
      \end{cases}
    \end{align*}
    където $s_0$ е минималният брой стъпки необходими на $\M_\omega$ за да приеме думата $\omega$.
  \end{itemize}
  
  Заключаваме, че 
  \begin{align*}
    & \M_\omega \text{ не приема }\omega \implies \omega \in \Ldiag \implies \code{\M'} \in \texttt{Code}(\Ss)\\
    & \M_\omega \text{ приема }\omega \implies \omega \not\in \Ldiag \implies \code{\M'} \not\in \texttt{Code}(\Ss).
  \end{align*}
  Понеже
  \[\omega \in \Ldiag \iff f(\omega) \in \texttt{Code}(\Ss),\]
  то $\texttt{Code}(\Ss)$ не е полуразрешим, защото от \Theorem{diagonal} ние знаем, че $\Ldiag$ не е полуразрешим.
\end{hint}

\begin{corollary}
  За всяко от следните свойства $\Ss$ на полуразрешимите езици, 
  $\texttt{Code}(\Ss)$ {\bf не} е полуразрешим език, където:
  \mynote{Защо не можем да използваме \Lemma{rice-shapiro:finite} за да докажем, че свойството празнота не е полуразрешимо, както и свойството регулярност, разрешимост, полуразрешимост?}
  \begin{itemize}
  \item
    $\Ss$ е свойството безкрайност, т.е. езикът
    \[\texttt{Code}(\Ss) = \{\code{\M} \mid \text{$\M$ е машина на Тюринг и }|\L(\M)| = \infty\}\]
    не е полуразрешим;
  \item
    $\Ss$ е свойството за пълнота, т.е. езикът
    \[\texttt{Code}(\Ss) = \{\code{\M} \mid \text{$\M$ е машина на Тюринг и } \L(\M) = \Sigma^\star\}\]
    не е полуразрешим;
  \item
    $\Ss$ е свойството неразрешимост, т.е. езикът
    \[\texttt{Code}(\Ss) = \{\code{\M} \mid \text{$\M$ е машина на Тюринг и $\L(\M)$ не е разрешим}\}\]
    не е полуразрешим;
  \item
    $\Ss$ е свойството неполуразрешимост, т.е. езикът
    \[\texttt{Code}(\Ss) = \{\code{\M} \mid \text{$\M$ е машина на Тюринг и $\L(\M)$ не е полуразрешим}\}\]
    не е полуразрешим;
  \item
    $\Ss$ е свойството нерегулярност, т.е. езикът
    \[\texttt{Code}(\Ss) = \{\code{\M} \mid \text{$\M$ е машина на Тюринг и $\L(\M)$ не е регулярен}\}\]
    не е полуразрешим.
  \end{itemize}
\end{corollary}

\begin{lemma}\label{lem:rice-shapiro:extension}
  \mynote{Това означава, че ако $\texttt{Code}(\Ss)$ е полуразрешим език, ако $L_0 \in \Ss$ и $L_0 \subseteq L_1$, като $L_1$ е полуразрешим, то $L_1 \in \Ss$.}
  Нека $L_1$ е език в $\Ss$ и нека $L_2$ е полуразрешим език, като $L_1 \subset L_2$ и $L_2 \not\in\Ss$.
  Тогава $\texttt{Code}(\Ss)$ не е полуразрешим език.
\end{lemma}
\begin{hint}
  Нека $L_1 = \L(\M_1)$ и $L_2 = \L(\M_2)$. Ще докажем, че
  \[\Ldiag \leq_m \texttt{Code}(\Ss).\]
  Ще дефинираме тотална изчислима функция $f$, която при вход думата $\omega \in \{0,1\}^\star$ работи по следния начин:
  \begin{itemize}
  \item
    Ако $\omega$ не е код на машина на Тюринг, то $f(\omega) = \omega$.
  \item
    Ако $\omega$ е код на машината на Тюринг $\M_\omega$, тогава $f(\omega)$ ще бъде кода на машината на Тюринг $\hat{\M}$,
    която работи по следния начин:
    \begin{itemize}
    \item 
      вход думата $\alpha$;
    \item
      $\hat{\M}$ симулира едновременно две изчисления - $\M_1$ върху $\alpha$ и $\M_\omega$ върху $\omega$
      докато намери стъпка $s$, такава че:    
      \begin{itemize}
      \item 
        ако $\M_1$ завършва за $s$ на брой стъпки като приема думата $\alpha$, то $\hat{\M}$ завършва като приема думата $\alpha$;
      \item
        ако $\M_\omega$ завършва за $s$ на брой стъпки като приема думата $\omega$, 
        то $\hat{\M}$ симулира работата $\M_2$ върху $\alpha$.
      \item
        ако $\hat{\M}$ не намери такава стъпка, то е ясно, че $\hat{\M}$ никога не завършва върху $\alpha$.
      \end{itemize}
    \end{itemize}
  \end{itemize}
  Съобразете сами, че получаваме следното:
  \begin{align*}
    \L(\hat{\M}) = 
    \begin{cases}
      L_2, & \text{ако $\M_\omega$ приема }\omega\\
      L_1, & \text{ако $\M_\omega$ не приема }\omega.
    \end{cases}
  \end{align*}
  Заключаваме, че:
  \[\omega \in \Ldiag \iff f(\omega) \in \texttt{Code}(\Ss),\]
  защото $L_2 \not\in \Ss$, а $L_1 \in \Ss$.
  Това означава, че ефективно можем да сведем въпрос за принадлежност в $\Ldiag$
  към въпрос за принадлежност в $\texttt{Code}(\Ss)$.
  Следователно, ако $\texttt{Code}(\Ss)$ е полуразрешим език, то $\Ldiag$ е полуразрешим език, което е противоречие.  
\end{hint}

\begin{corollary}
  За всяко от следните свойства $\Ss$ на полуразрешимите езици, 
  $\texttt{Code}(\Ss)$ {\bf не} е полуразрешим език, където:
  \begin{itemize}
  \item
    $\Ss$ е свойството крайност, т.е. езикът
    \[\texttt{Code}(\Ss) = \{\code{\M} \mid \text{$\M$ е машина на Тюринг и }|\L(\M)| < \infty\}\]
    не е полуразрешим;
  \item
    $\Ss$ е свойството празнота, т.е. езикът
    \[\texttt{Code}(\Ss) = \{\code{\M} \mid \text{$\M$ е машина на Тюринг и }\L(\M) = \emptyset\}\]
    не е полуразрешим;
  \item
    $\Ss$ е свойството разрешимост, т.е. езикът
    \[\texttt{Code}(\Ss) = \{\code{\M} \mid \text{$\M$ е машина на Тюринг и }\L(\M) \text{ е разрешим} \}\]
    не е полуразрешим;
  \item
    $\Ss$ е свойството безконтекстност, т.е. езикът
    \[\texttt{Code}(\Ss) = \{\code{\M} \mid \text{$\M$ е машина на Тюринг и }\L(\M) \text{ е безконтекстен} \}\]
    не е полуразрешим;
  \item
    $\Ss$ е свойството регулярност, т.е. езикът
    \[\texttt{Code}(\Ss) = \{\code{\M} \mid \text{$\M$ е машина на Тюринг и }\L(\M) \text{ е регулярен} \}\]
    не е полуразрешим.
  \end{itemize}
\end{corollary}


\begin{framed}
  \begin{theorem}[Райс-Шапиро]
    Нека $\texttt{Code}(\Ss)$ е полуразрешим език. Тогава е изпълнено, че:
    \[L \in \Ss \iff (\exists L_0 \subseteq \Sigma^\star )[L_0\text{ е краен и }L_0 \subseteq L \implies L_0 \in \Ss].\]
  \end{theorem}
\end{framed}
\begin{proof}
  Посоката $(\Rightarrow)$ следва от \Lemma{rice-shapiro:finite},
  докато посоката $(\Leftarrow)$ следва от \Lemma{rice-shapiro:extension}.
\end{proof}


% % \section{Проблеми за безконтекстни езици}

% % \begin{lemma}
% %   Нека е дадена $\M = \TM$.
% %   Тогава езикът 
% %   \[L = \{\alpha\sharp\beta^R \mid \alpha,\beta \in \Gamma^\star Q \Gamma^\star\ \&\  \alpha \vdash_\M \beta\}\]
% %   е безконтекстен.
% % \end{lemma}
% % \begin{proof}
% %   Ще покажем, че съществува стеков автомат $P$, за който $\L_S(P) = L$.
% %   Четем буквата $X$. Тогава:
% %   \begin{itemize}
% %   \item 
% %     ако $\delta_\M(q,X) =(p,Y,R)$, то слагаме $Yp$ на върха на стека;
% %   \item
% %     ако $\delta_\M(q,X) =(p,Y,L)$, то ако $Z$ е върха на стека, заменяме $Z$ с $pZY$;
% %   \end{itemize}
% % \end{proof}



% % \begin{thm}
% %   Неразрешим е проблемът за проверка дали при дадени две произволни безконтекстни граматики $G_1$ и $G_2$,
% %   $\L(G_1) \cap \L(G_2) = \emptyset$.  
% % \end{thm}

% % \begin{thm}
% %   Неразрешим е проблемът за проверка дали при дадена произволна безконтекстна граматика $G$,
% %   $\L(G) = \Sigma^\star$.  
% % \end{thm}


% % \section{Въпроси}

% % Вярно ли е, че следният проблем е {\em разрешим}:
% % \begin{itemize}
% % \item
% %   за произволна безконтекстна граматика $G$, проверява дали $\L(G) = \emptyset$?
% % \item
% %   за произволна безконтекстна граматика $G$, проверява дали $\L(G) = \Sigma^\star$?
% % \item
% %   за произволни безконтекстни граматики $G_1$ и $G_2$, проверява дали $\L(G_1) \cap \L(G_2) = \emptyset$?
% % \item
% %   за произволни безконтекстни граматики $G_1$ и $G_2$, проверява дали $\L(G_1) \cap \L(G_2) = \Sigma^\star$?
% % \item
% %   за произволни безконтекстни граматики $G_1$ и $G_2$, проверява дали $\L(G_1) = \L(G_2)$?
% % \item
% %   за произволни безконтекстни граматики $G_1$ и $G_2$, проверява дали $\L(G_1) \subseteq \L(G_2)$?
% % \item
% %   за произволна безконтекстна граматика $G$ и произволен регулярен израз $r$,
% %   проверява дали $\L(G) = \L(r)$?
% % \item
% %   за произволна безконтекстна граматика $G$ и произволен регулярен израз $r$,
% %   проверява дали $\L(G) \subseteq \L(r)$?
% % \item
% %   за произволна безконтекстна граматика $G$ и произволен регулярен израз $r$,
% %   проверява дали $\L(r) \subseteq \L(G)$?
% % \item
% %   за произволни безконтекстни граматики $G_1$ и $G_2$, проверява дали $\L(G_1) \subseteq \L(G_2)$ 
% %   е безконтекстен език ?
% % \item
% %   за произволна безконтекстна граматика $G$, проверява дали $\Sigma^\star \setminus \L(G)$
% %   е безконтекстен език ?
% % \item
% %   за произволна безконтекстна граматика $G$, проверява дали $\L(G)$ е регулярен език?
% % \end{itemize}


%%% Local Variables:
%%% mode: latex
%%% TeX-master: "../eai"
%%% End:
