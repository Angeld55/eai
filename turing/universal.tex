\subsection{Универсалният език $L_{\texttt{univ}}$}

\setlength{\epigraphwidth}{0.65\textwidth}\epigraph{A man provided with paper, pencil, and rubber, and subject to strict discipline, is in effect a universal machine. (Turing 1948: 416)}


\mynote{Можем за простота да считаме, че всички разглеждани машини на Тюринг са дефинирани над азбуката $\{0,1\}$.}
\index{език!неразрешим}
\begin{framed}
  \begin{thm}
    Езикът 
    \[\Luniv \df \{\ \code{\M} \sharp \omega \mid \text{$\M$ е машина на Тюринг и }\omega\in \L(\M)\ \}\]
    е полуразрешим, но {\bf не} е разрешим.
  \end{thm}
\end{framed}
\begin{hint}
  \mynote{Разсъждението е много сходно с това защо $\Laccept$ полуразрешим.}
  Първо да съобразим защо $\Luniv$ е полуразрешим език.
  Дефинираме (многолентова) машина на Тюринг $\M'$, която работи по следния начин:
  \begin{itemize}
  \item
    вход дума $\alpha$;
  \item 
    $\M'$ проверява дали $\alpha$ има вида $\code{\M} \cdot \omega$,
    за някоя машина на Тюринг $\M$ и дума $\omega$. Това става лесно, защото $\omega$
    започва веднага след второ срещане на $111$ в $\alpha$.
  \item
    Ако $\alpha = \code{\M} \sharp \omega$, 
    то $\M'$ симулира работата на $\M$ върху $\omega$.
    \begin{itemize}
    \item 
      Ако $\M$ завърши след краен брой стъпки като приеме $\omega$,
      то $\M'$ приема $\alpha$.
    \item
      Ако $\M$ завърши след краен брой стъпки като отхвърли $\omega$,
      то $\M'$ отхвърля $\alpha$.
    \item
      Ако $\M$ никога не завършва върху $\omega$,
      то очевидно $\M'$ също никога не завършва върху $\alpha$.
    \end{itemize}
  \item
    Ако $\alpha$ няма вида $\code{\M} \cdot \omega$,
    то $\M'$ завършва веднага като отхвърля думата $\alpha$.
  \end{itemize}
  Получаваме, че
  \[\alpha \in \Luniv \iff \alpha \in \L(\M').\]
  
  Сега да съобразим защо $\Luniv$ не е разрешим език.
  Имаме, че за произволна дума $\omega$,
  \[\omega \in \Laccept \iff \omega\sharp\omega \in \Luniv.\]
  Ако допуснем, че $\Luniv$ е разрешим, то тогава $\Laccept$ е разрешим език, което е противоречие.
\end{hint}

\begin{cor}
  Езикът
  \[\ov{\Luniv} \df \{\code{\M} \cdot \omega \mid \code{\M} \text{ е машина на Тюринг и }\omega\not\in \L(\M)\}\]
  {\bf не} е полуразрешим.
\end{cor}





%%% Local Variables:
%%% mode: latex
%%% TeX-master: "../eai"
%%% End:
