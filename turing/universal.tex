\subsection{Универсална машина на Тюринг}

\setlength{\epigraphwidth}{0.65\textwidth}\epigraph{A man provided with paper, pencil, and rubber, and subject to strict discipline, is in effect a universal machine. (Turing 1948: 416)}


\mynote{Можем за простота да считаме, че всички разглеждани машини на Тюринг са дефинирани над азбуката $\{0,1\}$.}
\index{език!неразрешим}
\begin{important}
  \begin{theorem}\label{th:universal}
    Универсалният език
    \[\Luniv \df \{\ \code{\M} \sharp \omega \mid \text{$\M$ е машина на Тюринг и }\omega\in \L(\M)\ \}\]
    е полуразрешим, но {\bf не} е разрешим.
  \end{theorem}
\end{important}
\begin{hint}
  \mynote{Разсъждението е много сходно с това защо $\Laccept$ полуразрешим.

    Ще наричаме $\mathcal{U}$ \emph{универсална машина на Тюринг}.}
  Първо да съобразим защо $\Luniv$ е полуразрешим език.
  Дефинираме (многолентова) машина на Тюринг $\mathcal{U}$, която работи по следния начин:
  \begin{itemize}
  \item
    вход дума $\alpha$;
  \item 
    $\mathcal{U}$ проверява дали $\alpha$ има вида $\code{\M} \sharp \omega$,
    за някоя машина на Тюринг $\M$ и дума $\omega$. Това става лесно, защото $\omega$
    започва веднага след второ срещане на $111$ в $\alpha$.
  \item
    Ако $\alpha$ е от вида $\code{\M} \sharp \omega$, 
    то $\mathcal{U}$ симулира работата на $\M$ върху $\omega$.
    \begin{itemize}
    \item 
      Ако $\M$ завърши след краен брой стъпки като приеме $\omega$,
      то $\mathcal{U}$ приема $\alpha$.
    \item
      Ако $\M$ завърши след краен брой стъпки като отхвърли $\omega$,
      то $\mathcal{U}$ отхвърля $\alpha$.
    \item
      Ако $\M$ никога не завършва върху $\omega$,
      то очевидно $\mathcal{U}$ също никога не завършва върху $\alpha$.
    \end{itemize}
  \item
    Ако $\alpha$ няма вида $\code{\M} \sharp \omega$,
    то $\mathcal{U}$ завършва веднага като отхвърля думата $\alpha$.
  \end{itemize}
  Получаваме, че
  \[\alpha \in \Luniv \iff \alpha \in \L(\mathcal{U}).\]
  Сега да съобразим защо $\Luniv$ не е разрешим език.
  За произволна дума $\omega$ имаме:
  \[\omega \in \Laccept \iff \omega\sharp\omega \in \Luniv.\]
  Ако допуснем, че $\Luniv$ е разрешим, то тогава $\Laccept$ е разрешим език, което е противоречие.
\end{hint}

\begin{corollary}
  Езикът
  \[\Luniv' \df \{\code{\M} \sharp \omega \mid \code{\M} \text{ е машина на Тюринг и }\omega\not\in \L(\M)\}\]
  {\bf не} е полуразрешим.
\end{corollary}



%%% Local Variables:
%%% mode: latex
%%% TeX-master: "../eai"
%%% End:
