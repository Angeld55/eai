\subsection{Недетерминистични машини на Тюринг}
\index{машина на Тюринг!недетерминистична}
Една машина на Тюринг $\N$ се нарича недетерминистична, ако функцията на преходите има вида
\[\Delta_\N: Q\times \Gamma \to \Ps(Q \times \Gamma\times \{L,R,S\}). \]

Отново можем да дефинираме бинарна релация $\vdash_\N$ над $\Gamma^\star \times Q \times \Gamma^\star$,
която ще казва как моментното описание на машината $\N$ се променя при изпълнение на една стъпка.
\begin{itemize}
\item
  Ако $\Delta_\N(q,Z) \ni (p,Y,R)$, то дефинираме $(\alpha, q, Z\beta) \vdash_\N (\alpha Y, p, \beta)$.
\item 
  Ако $\Delta_\N(q,Z) \ni (p,Y,L)$, то дефинираме $(\alpha X, q, Z\beta) \vdash_\N (\alpha , p, XY\beta)$.
\end{itemize}
С $\vdash^\star_\N$ ще означаваме рефлексивното и транзитивно затваряне на $\vdash_\N$.

Тогава за недетерминистична машина на Тюринг $\N$, 
\[\L(\N) = \{\alpha\in\Sigma^\star \mid (\varepsilon, s, \alpha) \vdash^\star_\N (\beta, q_{accept}, \gamma), \text{ за някои }\beta,\gamma\in\Gamma^\star\}.\]

\begin{remark}
  Върху дадена дума $\omega$, недетерминистичната машина на Тюринг $\N$ може да има много различни изчисления.
  Думата $\omega$ принадлежи на $\L(\N)$ ако съществува {\em поне едно} изчисление, което завършва в състоянието $q_{accept}$.
  Възможно е много други изчисления за $\omega$ да завършват в $q_{reject}$ или да зациклят.
\end{remark}

Аналогично, дефинираме една недетерминистична машина на Тюринг $\N$ да бъде {\bf тотална}, ако за всяка дума и 
всяко изчисление на $\N$ върху $\omega$ завършва в $q_{accept}$ или $q_{reject}$.

\begin{problem}
  \marginpar{\cite{hopcroft2}}
  $\N = (\{q_0,q_1,q_2,q_f\}, \{0,1\}, \{0,1,\blank\}, \Delta, q_0, \{q_f\})$,
  \begin{itemize}
  \item 
    $\Delta(q_0,0) = \{(q_0,1,R),(q_1,1,R)\}$;
  \item
    $\Delta(q_1,1) = \{(q_2,0,L)\}$;
  \item
    $\Delta(q_2,1) = \{(q_0,1,R)\}$;
  \item
    $\Delta(q_1,\blank) = \{(q_f,\blank,R)\}$.
  \end{itemize}
  \marginpar{$\{0^{n+1}1^k \mid n,k\in\Nat\}$}
  Опишете $\L(\N)$.
\end{problem}


\begin{framed}
  \begin{thm}
    Ако $L$ се разпознава от {\em недетерминистична} машина на Тюринг $\N$, то $L$
    е разпознава и от {\em детерминистична} машина на Тюринг $\D$.
  \end{thm}
\end{framed}
\begin{proof}
  \marginpar{В \cite[стр. 164]{hopcroft1} не е добре обяснено.}
  Нека имаме недетерминистичната машина на Тюринг $\N$, за която $L = \L(\N)$.
  Една дума $\alpha$ принадлежи на $\L(\N)$ точно тогава, когато съществува изчисление,
  което започва с думата $\alpha$ върху лентата и след краен брой стъпки, следвайки функцията на преходите $\Delta_\N$,
  достига до състоянието $q_{accept}$.
  Сложността идва от факта, че за думата $\alpha$ може да имаме много различни изчисления, 
  като само някои от тях завършват в $q_{accept}$. Ще построим детерминистична машина на Тюринг,
  която последователно ще симулира всички възможни {\em крайни} изчисления за думата $\alpha$, докато 
  намери такова, което завършва в състоянието $q_{accept}$.
  \marginpar{На практика това, което е правим е да представим всички възможни изчисления на $\N$ като $r$-разклонено дърво и да го обходим в широчина, докато не достигнем до $q_{accept}$}
  
  Лесно се съобразява, че всяко изчисление на $\N$ може да се представи като 
  крайна редица от елементи на $Q \times \Gamma \times \{L,R\}$.
  Понеже това множество е крайно, то можем на всяка такава тройка да
  съпоставим число в интервала $[1,r]$, където 
  \[r = 2 \cdot |Q| \cdot |\Gamma|.\]
  Оттук следва, че всяко изчисление на $\N$ може да се представи като крайна 
  редица от числа, всяко принадлежащо на интервала $[1,r]$.

  Детерминистичната машина на Тюринг $\D$ има три ленти.
  
  \begin{itemize}
  \item 
    На първата лента съхраняваме входящия низ и {\em тя никога не се променя}.
  \item
    На втората лента ще записваме последователно низове следвайки каноничната подредба на 
    думите над азбуката $\{1,2,\dots,r\}$.
  \item
    На третата лента симулираме изчислението на $\N$ върху думата от първата лента, използвайки изчислението, 
    което е описано на втората лента. Например, ако съдържанието на втората лента е $4,1,2$,
    това означава, че симулираме изчисление от три стъпки като на първата стъпка избираме четвъртата
    възможна тройка, на втората стъпка избираме първата възможна тройка, на третата стъпка избираме втората възможна тройка.
    
    Ако симулацията завърши в състоянието $q_{accept}$ на $\N$, то машината $\D$ завършва успешно.
    В противен случай, на втората лента записваме следващия низ; изтриваме третата лента и започваме нова симулация.
  \end{itemize}
\end{proof}

\begin{prop}[Лема на Кьониг]
  Ако $T$ е безкрайно дърво с крайно разклонение, то $T$ съдържа безкраен път.
\end{prop}

\begin{cor}
  Ако $L$ се разпознава от {\em тотална недетерминистична} машина на Тюринг $\N$, то $L$
  също се разпознава и от {\em тотална детерминистична} машина на Тюринг $\D$.
\end{cor}
\begin{proof}
  Да разгледаме дървото $T$, което представя всички изчисления на тоталната $\N$ при вход думата $\omega$.
  От лемата на Кьониг следва, че $T$ е крайно дърво, защото ако допуснем, че $T$ е безкрайно, то ще има безкрайно дълго изчисление на $\N$,
  което е невъзможно, понеже $\N$ винаги достига до финално състояние.
  \begin{itemize}
  \item 
    Ако $\N$ приема дадена дума $\omega$, то детерминистичната ни симулация на $\N$ ще достигне до изчисление, кодирано като път в $T$, 
    което завършва в състояние $q_{accept}$.
  \item
    Ако $\N$ не приема дадена дума $\omega$, то детерминистичната ни симулация на $\N$ ще покаже, че всяко изчисление, кодирано като път в $T$, завършва в състояние $q_{reject}$.
  \end{itemize}
\end{proof}

%%% Local Variables:
%%% mode: latex
%%% TeX-master: "../eai"
%%% End:
