\section{Недетерминистични машини на Тюринг}
\index{машина на Тюринг!недетерминистична}

Една машина на Тюринг $\N$ се нарича недетерминистична, ако функцията на преходите има вида
\[\Delta: Q'\times \Gamma \to \Ps(Q \times \Gamma\times \{\goleft,\goright,\stay\}), \]
където да напомним, че $Q' = Q \setminus \{\qaccept,\qreject\}$.

Отново можем да дефинираме бинарна релация $\vdash$ над $\Gamma^\star \times Q \times \Gamma^+$,
която ще казва как моментното описание на машината $\N$ се променя при изпълнение на една стъпка.

\begin{figure}[H]
  \begin{subfigure}[b]{0.5\textwidth}
    \begin{prooftree}
      \AxiomC{$\Delta(q,x) \ni (p,y,\goright)$}
      \RightLabel{\scriptsize{(right-1)}}
      \UnaryInfC{$(\alpha, q, xz\beta) \vdash (\alpha y, p, z\beta)$}
    \end{prooftree}
    \vspace*{2mm}
  \end{subfigure}
  ~
\begin{subfigure}[b]{0.5\textwidth}
\begin{prooftree}
  \AxiomC{$\Delta(q,x) \ni (p,y,\goleft)$}
  \RightLabel{\scriptsize{(left-1)}}
  \UnaryInfC{$(\alpha z, q, x\beta) \vdash (\alpha,p,zy\beta)$}
\end{prooftree}
\vspace*{2mm}
\end{subfigure}

\begin{subfigure}[b]{0.5\textwidth}
\begin{prooftree}
  \AxiomC{$\Delta(q,x) \ni (p,y,\goright)$}
  \RightLabel{\scriptsize{(right-2)}}
  \UnaryInfC{$(\alpha, q, x) \vdash (\alpha y, p, \blank)$}
\end{prooftree}
\vspace*{2mm}
\end{subfigure}
~
\begin{subfigure}[b]{0.5\textwidth}
  \begin{prooftree}
    \AxiomC{$\Delta(q,x) \ni (p,y,\goleft)$}
    \RightLabel{\scriptsize{(left-2)}}
    \UnaryInfC{$(\varepsilon,q,x\beta) \vdash (\varepsilon,p,\blank y\beta)$}
  \end{prooftree}
  \vspace*{2mm}
\end{subfigure}

\begin{prooftree}
  \AxiomC{$\Delta(q, z) \ni (p, y, \stay)$}
  \RightLabel{\scriptsize{(stay)}}
  \UnaryInfC{$(\alpha, q, z\beta) \vdash (\alpha , p, y\beta)$}
\end{prooftree}
\end{figure}

С $\vdash^\star$ ще означаваме рефлексивното и транзитивно затваряне на $\vdash$.
Тогава за недетерминистична машина на Тюринг $\N$, 
\[\L(\N) = \{\ \omega\in\Sigma^\star \mid (\exists \gamma_1,\gamma_2 \in \Gamma^\star)[(\varepsilon, \qstart, \omega\blank) \vdash^\star_\N (\gamma_1, \qaccept, \gamma_2) ]\ \}.\]

\begin{remark}
  Върху дадена дума $\omega$, недетерминистичната машина на Тюринг $\N$ може да има много различни изчисления.
  Думата $\omega$ принадлежи на $\L(\N)$ ако съществува {\em поне едно} изчисление, което завършва в състоянието $\qaccept$.
  Възможно е много други изчисления при вход $\omega$ да завършват в $\qreject$ или никога да не завършват.
\end{remark}

Аналогично, дефинираме една недетерминистична машина на Тюринг $\N$ да бъде {\bf разрешител}, ако за всяка дума $\omega$ и 
\emph{всяко изчисление} на $\N$ върху $\omega$ завършва в $\qaccept$ или $\qreject$.

% \begin{problem}
%   \mynote{\cite{hopcroft2}}
%   Нека
%   \[\N = (\{q_0,q_1,q_2,q_f\}, \{0,1\}, \{0,1,\blank\}, \Delta, q_0, \{q_f\}),\]
%   \begin{itemize}
%   \item 
%     $\Delta(q_0,0) = \{(q_0,1,\goright),(q_1,1,\goright)\}$;
%   \item
%     $\Delta(q_1,1) = \{(q_2,0,\goleft)\}$;
%   \item
%     $\Delta(q_2,1) = \{(q_0,1,\goright)\}$;
%   \item
%     $\Delta(q_1,\blank) = \{(q_f,\blank,\goright)\}$.
%   \end{itemize}
%   \mynote{$\{0^{n+1}1^k \mid n,k\in\Nat\}$}
%   Опишете $\L(\N)$.
% \end{problem}

\begin{extra}
\begin{example}
  \mynote{Не е обяснено защо е разрешим.}
  Нека да видим, че $L = \{\alpha\sharp\beta \mid \alpha,\beta \in \{a,b\}^\star\ \&\ \alpha\text{ е подниз на }\beta\}$
  е разрешим език като построим недетерминистична машина на Тюринг $\N$,
  която разрешава този език.
  \begin{framed}
    \begin{figure}[H]
      \begin{center}
        \begin{tikzpicture}[->,>=stealth,thick,node distance=65pt]
          \tikzstyle{every state}=[circle,minimum size=10pt,scale=.9]
          
          \node[state,initial below]    (1) {$q_0$};
          \node[state]            (2) [right of=1]{$q_1$};
          \node[state]            (3) [right of=2,node distance=80pt]{$q_2$};
          \node[state]            (4) [below of=3]{$q_3$};
          \node[state]            (5) [below right of=4,node distance=80pt]{$q_4$};
          \node[state]            (6) [right of=4]{$q_5$};
          \node[state]            (7) [above of=6]{$q_6$};
          \node[state]            (8) [right of=6,node distance=90pt]{$q_7$};
          \node[state]            (9) [right of=7,node distance=90pt]{$q_8$};
          \node[state,accepting]  (10)[below right of=5]{$q_{9}$};
          
          \begin{scope}[every node/.style={scale=.8}]
            \path
            (1) edge [loop above] node [above] {$\{a,b\};\goright$} (1)
            (1) edge [bend left=15] node [above] {$\sharp;\goright$} (2)
            (2) edge [loop above] node [above] {$a/\blank,b/\blank;\goright$} (2)
            (2) edge [bend left=15] node [above] {$\{a,b,\blank\};\goleft$} (3)
            (3) edge [loop above] node [above] {$\blank;\goleft$} (3)
            (3) edge [bend right=15] node [left] {$\sharp;\goleft$} (4)
            (4) edge [loop left] node [left] {$\{a,b\};\goleft$} (4)
            (4) edge [bend right=30] node [left] {$\blank;\goright$} (5)
            (5) edge [bend right=15] node [right] {$a/\blank;\goright$} (6)
            (6) edge [loop right] node [right] {$\{a,b\};\goright$} (6)
            (6) edge [bend right=15] node [right] {$\sharp;\goright$} (7)
            (7) edge [loop right] node [right] {$\blank;\goright$} (7)
            (7) edge [bend left=15] node [below] {$a/\blank;\goleft$} (3)
            (8) edge [loop right] node [right] {$\{a,b\};\goright$} (8)
            (5) edge [bend right=30] node [right] {$b/\blank;\goright$} (8)
            (8) edge [bend right=15] node [right] {$\sharp;\goright$} (9)
            (9) edge [loop right] node [above] {$\blank;\goright$} (9)
            (9) edge [bend right=45] node [above] {$b/\blank;\goleft$} (3)
            (5) edge [bend right=15] node [left] {$\sharp;\stay$} (10);
          \end{scope}
        \end{tikzpicture}
      \end{center}
    \end{figure}
  \end{framed}
  Да видим, че $\M$ успешно разпознава думата $ab\sharp aabb$, която принадлежи на $L$.
  \begin{align*}
    (q_0, \underline{a}b\sharp aabb\blank) & \vdash (q_0, a\underline{b}\sharp aabb\blank) \vdash (q_0, ab\underline{\sharp} aabb\blank) \vdash (q_1, ab\sharp\underline{a}abb\blank) \\
                                           & \vdash (q_1, ab\sharp\blank\underline{a}bb\blank) \vdash (q_2, ab\sharp\underline{\blank}abb\blank) \vdash (q_2, ab\underline{\sharp}\blank abb\blank)\\
                                           & \vdash (q_3, a\underline{b}\sharp\blank abb\blank) \vdash (q_3, \underline{a}b\sharp\blank abb\blank) \vdash (q_3, \underline{\blank}ab\sharp\blank abb\blank)\\
                                           & \vdash (q_4, \underline{a}b\sharp\blank abb\blank) \vdash (q_5, \blank\underline{b}\sharp \blank abb\blank) \vdash (q_5, \blank b\underline{\sharp} \blank abb\blank)\\
                                           & \vdash (q_6, \blank b \sharp \underline{\blank} abb\blank) \vdash (q_6, \blank b \sharp \blank \underline{a}bb\blank) \vdash (q_2, \blank b \sharp \underline{\blank}\blank bb\blank)\\
                                           & \vdash (q_2, \blank b \underline{\sharp} \blank\blank bb\blank) \vdash (q_3, \blank \underline{b} \sharp \blank\blank bb\blank) \vdash (q_3, \underline{\blank} b \sharp \blank\blank bb\blank)\\
                                           & \vdash (q_4,  \blank \underline{b} \sharp \blank\blank bb\blank) \vdash (q_7, \blank \blank \underline{\sharp} \blank\blank bb\blank) \\
                                           & \vdash (q_8, \blank \blank \sharp \underline{\blank}\blank bb\blank) \vdash (q_8, \blank \blank \sharp \blank \underline{\blank} bb\blank) \\
                                           & \vdash (q_8, \blank \blank \sharp \blank \blank \underline{b}b\blank) \vdash (q_2, \blank \blank \sharp \blank \underline{\blank} \blank b\blank)\\
                                           & \vdash \cdots \vdash (q_4, \blank\blank\underline{\sharp}\blank\blank\blank b\blank) \vdash (q_9, \blank\blank\underline{\sharp}\blank\blank\blank b\blank)
  \end{align*}
\end{example}
\end{extra}

\subsection*{Канонична наредба на $\Sigma^\star$}
\index{наредба!канонична}

\mynote{За доказателството, че всяка НМТ е еквивалентна на ДМТ, е необходимо да фиксираме канонична подредба на думите над дадена азбука}
Нека $\Sigma = \{a_0,a_1,\dots,a_{k-1}\}$.
Подреждаме думите по ред на тяхната дължина.
Думите с еднаква дължина подреждаме по техния числов ред, т.е.
гледаме на буквите $a_i$ като числото $i$ в $k$-ична бройна система.
Тогава думите с дължина $n$ са числата от $0$ до $k^n-1$ записани в $k$-ична бройна система.
Ще означаваме с $\omega_i$ $i$-тата дума в $\Sigma^\star$ при тази подредба.

Ако $\Sigma = \{0,1\}$, то наредбата започва така:
\[\varepsilon, 0, 1, \underbrace{00, 01, 10, 11}_{\text{от $0$ до $3$}}, \underbrace{000, 001, 010, 011, 100, 101, 110, 111}_{\text{от $0$ до $7$}}, 0000, 0001, \dots\]
В този случай, $\omega_0 = \varepsilon$, $\omega_7 = 000$, $\omega_{13} = 110$.
Обърнете внимание, че тази наредба отговаря на обхождане в широчина на едно пълно наредено двоично дърво.
Можем да дефинираме и релацията $<_{\text{can}}$ по следния начин:
\[\alpha <_{\text{can}} \beta \dff |\alpha| < |\beta| \lor ( |\alpha| = |\beta|\ \&\ \alpha <_{\text{lex}} \beta). \]

\begin{extra}
\begin{problem}
  \label{prob:canonical:function}
  Нека $\Sigma = \{a_0,\dots,a_{k-1}\}$.
  Да разгледаме функцията $f:\Sigma^\star \to \Sigma^\star$, за която 
  $f(\alpha)$ е думата веднага след $\alpha$ в каноничната подредба на $\Sigma^\star$.
  Докажете, че $f$ е изчислима с еднолетнова детерминистична машина на Тюринг.
\end{problem}
\begin{hint}
  Ако $\Sigma = \{0,1\}$, то машината на Тюринг има следния вид:
  \begin{framed}
    \begin{figure}[H]
      \begin{center}
        \begin{tikzpicture}[->,>=stealth,thick,node distance=70pt]
          \tikzstyle{every state}=[circle,minimum size=10pt,auto,scale=.9]
          
          \node[state,initial]    (1) [right of=0]{$q_1$};
          \node[state]            (2) [right of=1]{$q_2$};
          \node[state]            (3) [right of=2]{$q_3$};
          \node[state,accepting]  (4) [right of=3]{$q_4$};
          
          \begin{scope}[every node/.style={scale=.8}]
            \path
            (1) edge [loop above] node [above] {$\{0,1\};\goright$} (1)
            (1) edge [bend left=15] node [above] {$\blank;\goleft$} (2)
            (2) edge [loop above] node [above] {$1/0;\goleft$} (2)
            (2) edge [bend left=15] node [above] {$0/1;\goleft$} (3)
            (2) edge [bend right=30] node [below] {$\blank/0;\stay$} (4)
            (3) edge [loop above] node [above] {$\{0,1\};\goleft$} (3)
            (3) edge [bend left=15] node [above] {$\blank;\goright$} (4);
          \end{scope}
        \end{tikzpicture}
        \caption{Генериране на следващата дума в каноничната наредба.}
      \end{center}
    \end{figure}
  \end{framed}
\end{hint}
\end{extra}

\begin{important}
  \begin{theorem}
    Ако $L$ се разпознава от {\em недетерминистична} машина на Тюринг $\N$, то $L$
    е разпознава и от {\em детерминистична} машина на Тюринг $\D$.
  \end{theorem}
\end{important}
\begin{proof}
  \mynote{В \cite[стр. 164]{hopcroft1} не е добре обяснено.}
  Нека имаме недетерминистичната машина на Тюринг $\N$, за която $L = \L(\N)$.
  Една дума $\alpha$ принадлежи на $\L(\N)$ точно тогава, когато съществува изчисление,
  което започва с думата $\alpha$ върху лентата и след краен брой стъпки, следвайки функцията на преходите $\Delta_\N$,
  достига до състоянието $\qaccept$.
  Сложността идва от факта, че за думата $\alpha$ може да имаме много различни изчисления, 
  като само някои от тях завършват в $\qaccept$. Ще построим детерминистична машина на Тюринг $\D$,
  която последователно ще симулира всички възможни {\em крайни} изчисления за думата $\alpha$, докато 
  намери такова, което завършва в състоянието $\qaccept$.
  \mynote{На практика това, което правим е да представим всички възможни изчисления на $\N$ като $r$-разклонено дърво и да го обходим в широчина, докато не достигнем до $\qaccept$}
  
  Лесно се съобразява, че всяко изчисление на $\N$ може да се представи като 
  крайна редица от елементи на $Q \times \Gamma \times \{\goleft,\goright,\stay\}$.
  Понеже това множество е крайно, то можем на всяка такава тройка да
  съпоставим естествено число $ < r$, където 
  \[r = |Q| \cdot |\Gamma| \cdot 3.\]
  Например, нека $Q = \{q_0,q_1\}$, $\Gamma = \{a,b\}$. Тогава можем да направим следната съпоставка:
  \begin{align*}
    & (q_0,a,\stay) \to 0,\ (q_0,a,\goleft) \to 1,\ (q_0,a,\goright) \to 2,\\
    & (q_0,b,\stay) \to 3,\ (q_0,b,\goleft) \to 4, \ (q_0,b,\goright) \to 5,\\
    & (q_1,a,\stay) \to 6, (q_1,a,\goleft) \to 7, (q_1,a,\goright) \to 8,\\
    & (q_1,b,\stay) \to 9, (q_1,b,\goleft) \to 10,\ (q_1,b,\goright) \to 11.
  \end{align*}
  Ясно е, че всяко изчисление на $\N$ може да се представи като дума над азбуката $\Sigma = \{x_0,x_1,\dots,x_{r-1}\}$.
  Например, изчислението от три стъпки
  \[(\blank,q_0,aba) \vdash_N (b,q_1,ba) \vdash_\N (b,q_1,aa) \vdash_\N (ba,q_0,a)\]
  може да се опише като думата $x_{11}x_6x_2$ над азбуката $\Sigma = \{x_0,x_1,\dots,x_{11}\}$.
  
  Детерминистичната машина на Тюринг $\D$ има три ленти.
  \begin{itemize}
  \item 
    На първата лента съхраняваме входящия низ и {\em тя никога не се променя}.
  \item
    На втората лента ще записваме последователно думи следвайки каноничната наредба на 
    думите над азбуката $\{x_0,x_1,\dots,x_{r-1}\}$.
    От \Problem{canonical:function} знаем как последователно да генерираме тези думи върху една лента.
  \item
    На третата лента симулираме изчислението на $\N$ върху думата от първата лента, използвайки изчислението, 
    което е описано на втората лента. Например, ако съдържанието на втората лента е $x_{11}x_6x_2$,
    това означава, че симулираме изчисление от три стъпки като на първата стъпка избираме дванайсетата
    възможна тройка, на втората стъпка избираме седмата възможна тройка, на третата стъпка избираме третата възможна тройка.
    
    Ако симулацията завърши в състоянието $\qaccept$ на $\N$, то машината $\D$ завършва успешно.
    В противен случай, на втората лента генерираме чрез функцията от \Problem{canonical:function} следващия низ относно каноничната наредба на $\{x_0,x_1\dots,x_{r-1}\}$;
    изтриваме третата лента, копираме първата лента на третата и започваме нова детерминистична симулация като думата върху втората лента ни ръководи какъв преход да правим на всяка стъпка.
  \end{itemize}
\end{proof}


\begin{corollary}
  Ако $L$ се разпознава от {\em недетерминистичен} разрешител $\N$, то $L$
  също се разпознава от {\em детерминистичен} разрешител $\D$.
\end{corollary}
\begin{proof}
  Да разгледаме дървото $T$ с крайно разклонение $r$, което представя всички изчисления на разрешителя $\N$ при вход думата $\omega$.
  От \Lemma{konig} следва, че $T$ е крайно дърво, да кажем с височина $h$, защото ако допуснем, че $T$ е безкрайно, то ще има безкрайно дълго изчисление на $\N$,
  което е невъзможно, понеже $\N$ винаги достига до заключително състояние ($\qaccept$ или $\qreject$).
  \begin{itemize}
  \item 
    Ако $\N$ приема дадена дума $\omega$, то детерминистичната ни симулация на $\N$ ще достигне до изчисление, кодирано като път в $T$, 
    което завършва в състояние $\qaccept$.
  \item
    Ако $\N$ не приема дадена дума $\omega$, то детерминистичната ни симулация на $\N$ ще покаже, че всяко изчисление, кодирано като път в $T$, завършва в състояние $\qreject$.
    Един начин да направим това е да имаме една допълнителна лента, която използваме за брояч колко от възможните изчисления на $\N$ са завършили.
    Спираме, когато този брояч достигне $r^h$, където $h$ е дължината на думата на втората лента, т.е. дълбочината на дървото на изчисленията на $\N$.
  \end{itemize}
\end{proof}


%%% Local Variables:
%%% mode: latex
%%% TeX-master: "../eai"
%%% End:
