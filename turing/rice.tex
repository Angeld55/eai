\section{Критерий за разрешимост}

\begin{prop}
  Докажете, че езикът
  \[L_{\texttt{All}} \df \{\code{\M} \mid \M\text{ е М.Т. и }\L(\M) = \Sigma^\star\}\]
  не е разрешим.
\end{prop}
\begin{proof}
  Ще дефинираме алгоритъм, за който по вход думата $\code{\M}\cdot\omega$ 
  връща код на машината на Тюринг $\M'_\omega$, която работи по следния начин:
  \marginpar{За различни $\M$ и $\omega$, получаваме различни $\M'_\omega$}
  \begin{itemize}
  \item 
    Вход дума $\alpha$;
  \item
    Първоначално $\M'_\omega$ не обръща внимание на $\alpha$.
  \item
    $\M'_\omega$ симулира $\M$ върху думата $\omega$;
    \begin{itemize}
    \item 
      Ако след краен брой стъпки $\M$ завърши като приеме думата $\omega$,
      то $\M'_\omega$ приема думата $\alpha$, т.е. $\M'_\omega$ завършва в състоянието $q_{\texttt{accept}}$.
    \item
      Ако след краен брой стъпки $\M$ завърши като отхрърли думата $\omega$,
      то $\M'_\omega$ отхвърля думата $\alpha$, т.е. $\M'_\omega$ завършва в състоянието $q_{\texttt{reject}}$.
    \item
      В противен случай, $\M$ никога не завършва върху $\omega$.
      Това означава, че $\M'_\omega$ никога не завършва върху входа $\alpha$
      и следователно $\M'_\omega$ не приема думата $\alpha$.
    \end{itemize}
  \end{itemize}
  Получаваме, че:
  \begin{align*}
    & \code{\M}\cdot\omega \in L_{\texttt{univ}} \implies \L(\M'_\omega) = \Sigma^\star \implies \code{\M'_\omega} \in L_{\texttt{All}},\\
    & \code{\M}\cdot\omega \not\in L_{\texttt{univ}} \implies \L(\M'_\omega) = \emptyset \implies \code{\M'_\omega} \not\in L_{\texttt{All}}.
  \end{align*}
  Ако допуснем, че $L_{\texttt{All}}$ е разрешим език, то $L_{\texttt{univ}}$ също ще е разрешим, което е противоречие.
\end{proof}

\begin{cor}
  Езикът
  \[\ov{L}_{\texttt{Empty}} \df \{\code{\M} \mid \M\text{ е М.Т. и }\L(\M) \neq \emptyset\}\]
  не е разрешим.
\end{cor}
\begin{hint}
  Същата конструкция както горе.
\end{hint}

\begin{cor}
  Езикът
  \[L_{\texttt{Empty}} \df \{\code{\M} \mid \M\text{ е М.Т. и }\L(\M) = \emptyset\}\]
  не е разрешим.
\end{cor}
\begin{hint}
  Ако $L_{\texttt{Empty}}$ беше разрешим, то
  \[\ov{L}_{\texttt{Empty}} = \{0,1\}^\star \setminus L_{\texttt{Empty}} \cap \{\code{\M} \mid \M\text{ е М.Т.}\}\]
  щеше да е разрешим език, което е противоречие.
\end{hint}

\begin{prop}
  Докажете, че езикът
  \[L_{\texttt{reg}} \df \{\code{\M} \mid \M\text{ е М.Т. и }\L(\M) \text{ е регулярен език}\}\]
  не е разрешим.
\end{prop}
\begin{proof}
  Да фиксираме един език, за който знаем, че не е регулярен, например, 
  $\{0^n1^n \mid n \in \Nat\}$.

  Дефинираме алгоритъм, за който по вход $\code{\M}\cdot\omega$ 
  връща код на машината на Тюринг $\M'_\omega$.
  Сега ще опишем как работи $\M'_\omega$.
  \begin{itemize}
  \item 
    Вход думата $\alpha$;
  \item
    Ако $\alpha = 0^n1^n$, за някое $n$, то $\M'_\omega$ приема думата $\alpha$.
  \item
    Ако $\alpha$ не е от вида $0^n1^n$, тогава $\M'_\omega$ симулира $\M$ върху думата $\omega$.
    \begin{itemize}
    \item 
      Ако след краен брой стъпки $\M$ завърши като приеме думата $\omega$, то $\M'_\omega$ приема $\alpha$.
    \item
      Ако след краен брой стъпки $\M$ завърши като отхвърли думата $\omega$, то $\M'_\omega$ отхвърля думата $\alpha$.
    \item
      В противен случай, $\M$ никога не завършва върху $\omega$.
      Това означава, че $\M'_\omega$ никога не завършва върху входа $\alpha$
      и следователно $\M'_\omega$ не приема думата $\alpha$.
    \end{itemize}
  \end{itemize}
  Получаваме, че:
  \begin{align*}
    & \code{\M}\cdot\omega \in L_{\texttt{univ}} \implies \L(\M'_\omega) = \Sigma^\star \implies \code{\M'_\omega} \in L_{\texttt{reg}},\\
    & \code{\M}\cdot\omega \not\in L_{\texttt{univ}} \implies \L(\M'_\omega) = \{0^n1^n \mid n \in \Nat\} \implies \code{\M'_\omega} \not\in L_{\texttt{reg}}.
  \end{align*}
  Ако допуснем, че $L_{\texttt{reg}}$ е разрешим език, то $L_{\texttt{univ}}$ също ще е разрешим, което е противоречие.
\end{proof}

\begin{prop}
  Докажете, че езикът
  \[\ov{L}_{\texttt{reg}} \df \{\code{\M} \mid \M\text{ е М.Т. и }\L(\M) \text{ не е регулярен език}\}\]
  не е разрешим.
\end{prop}
\begin{proof}
  Да фиксираме машина на Тюринг $\M_L$, за което $\L(\M_L) = L$ не е регулярен език.

  Дефинираме алгоритъм, за който по вход $\code{\M}\cdot\omega$ 
  връща код на машината на Тюринг $\M'_\omega$.
  Сега ще опишем как работи $\M'_\omega$.
  \marginpar{За различни $\M$ и $\omega$, получаваме различни $\M'_\omega$}
  \begin{itemize}
  \item 
    Вход думата $\alpha$;
  \item
    Първоначално $\M'_\omega$ не обръща внимание на $\alpha$.
  \item
    $\M'_\omega$ симулира $\M$ върху думата $\omega$;
    \begin{itemize}
    \item 
      Ако след краен брой стъпки $\M$ завърши като приеме думата $\omega$, то $\M'_\omega$ симулира $\M_L$ върху $\alpha$.
      \begin{itemize}
      \item 
        Ако след краен брой стъпки $\M_L$ завърши като приеме $\alpha$, то $\M'_\omega$ приема $\alpha$;
      \item
        Ако след краен брой стъпки $\M_L$ завърши като отхрърли $\alpha$, то $\M'_\omega$ отхвърля $\alpha$;
      \item
        Ако $\M_L$ никога не свършва върху $\alpha$, то $\M'_\omega$ никога няма да свирши върху $\alpha$,
        което означава, че $\M'_\omega$ не приема $\alpha$.
      \end{itemize}
    \item
      Ако след краен брой стъпки $\M$ завърши като отхвърли думата $\omega$,
      то $\M'_\omega$ отхвърля думата $\alpha$.
    \item
      В противен случай, $\M$ работи безкрайно много стъпки върху $\omega$ без да завърши.
      Това означава, че $\M'_\omega$ ще работи безкрайно много стъпки върху думата $\alpha$
      и следователно $\M'_\omega$ не приема думата $\alpha$.
    \end{itemize}
  \end{itemize}
  Получаваме, че:
  \begin{align*}
    & \code{\M}\cdot\omega \in L_{\texttt{univ}} \implies \L(\M'_\omega) = L \implies \code{\M'_\omega} \in \ov{L}_{\texttt{reg}},\\
    & \code{\M}\cdot\omega \not\in L_{\texttt{univ}} \implies \L(\M'_\omega) = \emptyset \implies \code{\M'_\omega} \not\in \ov{L}_{\texttt{reg}}.
  \end{align*}
  Ако допуснем, че $\ov{L}_{\texttt{reg}}$ е разрешим език, то $L_{\texttt{univ}}$ също ще е разрешим, което е противоречие.
\end{proof}

Сега ще видим, че идеята, която следваме в горните доказателства може да се обобщи.
Нека $\Ss$ е множество от полуразрешими езици над фиксирана азбука $\Sigma$.
Например, 
\[\Ss = \{L \subseteq \Sigma^\star \mid L\text{ е регулярен език}\}.\]
Ще казваме, че $\Ss$ е свойство на полуразрешимите езици.
$\Ss$ е тривиално свойство, ако $\Ss = \emptyset$ или $\Ss$ съдържа точно всички полуразрешими езици.
Нека 
\[L_\Ss \df \{\code{\M} \mid \text{$\M$ е машина на Тюринг и } \L(\M) \in \Ss\}.\]

\begin{framed}
  \begin{thm}[Райс]
    \index{Райс}
    Всяко нетривиално свойство $\Ss$ на полуразрешимите езици е неразрешимо.
  \end{thm}  
\end{framed}
\begin{proof}
  \marginpar{\cite[стр. 188]{hopcroft1}}
  \marginpar{Цел: да сведем ефективно $L_{\texttt{univ}}$ към $L_\Ss$}
  Без ограничение на общността, нека $\emptyset \not\in \Ss$.
  Понеже $\Ss$ е нетривиално свойство, да разгледаме езика $L \in \Ss$,
  като $\M_L$ е машина на Тюринг, за която $\L(\M_L) = L$.
  Да разгледаме алгоритъм, който по дадена дума $\code{\M}\cdot\omega$
  връща код на машина на Тюринг $\M'_\omega$, която зависи от $\M$, $\omega$ и от $\M_L$.
  Тя работи по следния начин:
  \begin{itemize}
  \item
    \marginpar{Неформално описваме функцията $\delta$ за $\M'$}
    вход думата $\alpha$;
  \item
    първоначално $\M'_\omega$ не обръщаме внимание на $\alpha$;
  \item
    $\M'_\omega$ симулира $\M$ върху думата $\omega$.
    \begin{itemize}
    \item
      \marginpar{в този случай ще получим, че $\L(\M') = L$}
      ако след краен брой стъпки $\M$ завърши като приеме думата $\omega$, то 
      $\M'_\omega$ симулира $\M_L$ върху входната дума $\alpha$;
      \begin{itemize}
      \item
        ако след краен брой стъпки $\M_L$ завърши като приеме думата $\alpha$, то 
        $\M'_\omega$ приема $\alpha$;
      \item 
        ако след краен брой стъпки $\M_L$ завърши като отхвърли думата $\alpha$, то 
        $\M'_\omega$ отхвърля $\alpha$;
      \item
        ако $\M_L$ никога не завършва върху $\alpha$, то 
        $\M'_\omega$ никога няма да завърши върху $\alpha$ и следователно $\M'_\omega$
        не приема $\alpha$.
      \end{itemize}
    \item
      \marginpar{при тези два случая ще получим, че $\L(\M') = \emptyset$}
      ако след краен брой стъпки $\M$ завърши като отхвърли думата $\omega$, то 
      $\M'_\omega$ отхвърля $\alpha$;
    \item
      Ако $\M$ никога не свършва върху $\omega$, то $\M'_\omega$ никога няма да свирши върху $\alpha$,
      което означава, че $\M'_\omega$ не приема $\alpha$.      
    \end{itemize}
  \end{itemize}
  От всичко това следва, че така описаната машина на Тюринг $\M'$ има свойствата:
  \begin{align*}
    & \code{\M}\cdot\omega \in L_{\texttt{univ}} \implies \L(\M') = L \implies \L(\M') \in \Ss,\\
    & \code{\M}\cdot\omega \not\in L_{\texttt{univ}} \implies \L(\M') = \emptyset \implies \L(\M') \not\in \Ss.
  \end{align*}
  Да допуснем, че $L_\Ss$ е разрешимо множество от полуразрешими езици.
  Тогава от еквивалентността,
  \[\code{\M}\cdot\omega \in L_{\texttt{univ}} \iff \code{\M'} \in L_\Ss,\]
  получаваме, че $L_{\texttt{univ}}$ е разрешимо множество, което е противоречие.

  Ако $\emptyset \in \Ss$, то правим горните разсъждения за класа 
  \[\ov{\Ss} = \{\L(\M) \mid \M\text{ е М.Т. и }\L(\M) \not\in\Ss\}.\]
  По аналогичен начин доказваме, че $L_{\ov{\Ss}}$ не е разрешим език.
  Понеже 
  \[L_{\ov{\Ss}} = (\{0,1\}^\star \setminus L_{\Ss}) \cap \{\code{\M} \mid \M \text{ е машина на Тюринг}\},\]
  то $L_{\Ss}$ също не е разрешим език.
\end{proof}

\begin{cor}
  За всяко от следните свойства $\Ss$ на полуразрешимите множества, 
  $L_\Ss$ {\bf не} е разрешимо множество, където:
  \begin{enumerate}[a)]
  \item 
    свойството празнота, т.е. 
    \[L_\Ss = \{\code{\M} \mid \text{$\M$ е М.Т. и } \L(\M) = \emptyset\};\]
  \item 
    свойството за пълнота, т.е. 
    \[L_\Ss = \{\code{\M} \mid \text{$\M$ е М.Т. и } \L(\M) = \{0,1\}^\star\};\]
  \item
    свойството крайност, т.е. 
    \[L_\Ss = \{\code{\M} \mid \text{$\M$ е М.Т. и }|\L(\M)| < \infty\};\]
  \item
    свойството безкрайност, т.е. 
    \[L_\Ss = \{\code{\M} \mid \text{$\M$ е М.Т. и }|\L(\M)| = \infty\};\]
  \item
    свойството регулярност, т.е. 
    \[L_\Ss = \{\code{\M} \mid \text{$\M$ е М.Т. и $\L(\M)$ е регулярен език}\};\]
  \item
    свойството безконтекстност, т.е. 
    \[L_\Ss = \{\code{\M} \mid \text{$\M$ е М.Т. и $\L(\M)$ е безконтекстен}\}.\]
  \item
    свойството разрешимост, т.е. 
    \[L_\Ss = \{\code{\M} \mid \text{$\M$ е М.Т. и $\L(\M)$ е разрешим}\}.\]
  \end{enumerate}
\end{cor}

%%% Local Variables:
%%% mode: latex
%%% TeX-master: "../eai"
%%% End:
