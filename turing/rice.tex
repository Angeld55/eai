\section{Критерий за разрешимост}

\mynote{Сипсър нарича $\leq_m$ \emph{mapping reducibility} \cite[235]{sipser3}.}

\begin{important}
  Доказателството, че $\Luniv$ не е разрешим е пример за една обща схема, с която можем да докажем, че даден език не е разрешим:
  \begin{itemize}
  \item 
    Нека имаме езика $K$, за който вече знаем, че не е разрешим.
    В нашия пример, $K = \Laccept$.
  \item
    Питаме се дали някой друг език $L$ е разрешим.
  \item
    Намираме изчислима тотална функция $f$, за която е изпълнено, че:
    \[\omega \in K \iff f(\omega) \in L.\]
    В \Theorem{universal}, това е функцията $f(\omega) = \omega \sharp \omega$.
  \item
    В този случай ще означаваме $K \leq_m L$.
  \item
    Тогава, ако $L$ е разрешим ще следва, че $K$ е разрешим, което е противоречие.
  \end{itemize}
\end{important}

Сега искаме да разгледаме един критерий, който ще ни казва кога един език съставен от кодове на машини на Тюринг е разрешим. С негова помощ ще можем директно да решаваме наглед трудни задачи. Например,
в момента не е очевидно защо следния език не е разрешим:
\begin{align*}
  L_{\texttt{palin}} \df \{\omega \in \{0,1\}^\star \mid & \ \omega\text{ е код на машина на Тюринг и }\L(\M_\omega)\\
                                                         & \text{ съдържа само думи палиндроми}\}.
\end{align*}
След малко ще видим, че според критерия, който ще разгледаме, директно ще можем да заключим, че $L_{\texttt{palin}}$ не е разрешим. Да започнем с няколко примера.

\begin{important}
  \begin{proposition}
    \label{pr:rice:sigma-star}
    Езикът
    \[L_{\Sigma^\star} \df \{\omega \in \{0,1\}^\star \mid \omega\text{ е код на машина на Тюринг и }\L(\M_\omega) = \Sigma^\star\}\]
    не е разрешим.
  \end{proposition}  
\end{important}
\begin{proof}
  \mynote{$L_{\Sigma^\star}$ не е дори полуразрешим, но за момента не знаем как да докажем това.}
  Ще дефинираме тотална изчислима функция $f$, която при вход думата $\omega \in \{0,1\}^\star$ работи по следния начин:
  \begin{itemize}
  \item
    Ако $\omega$ не е код на машина на Тюринг, то $f(\omega) \df \omega$.
  \item
    Ако $\omega$ е код на машина на Тюринг $\M_\omega$, то
    $f(\omega) \df \code{\M'}$, където $\M'$, при вход произволна дума $\alpha$, работи по следния начин:
    \begin{enumerate}[(1)]
    \item
      Първоначално $\M'$ не обръща внимание на $\alpha$, а $\M'$ симулира работата на $\M_\omega$ върху думата $\omega$.
    \item 
      Ако след краен брой стъпки симулацията завърши с резултат, че $\M_\omega$ приема думата $\omega$,
      то $\M'$ завършва като приема думата $\alpha$.
    \end{enumerate}    
  \end{itemize}
  Получаваме, че:
  \[\L(\M') =
    \begin{cases}
      \Sigma^\star, & \text{ако } \omega \in \L(\M_\omega)\\
      \emptyset, & \text{ако } \omega \not\in \L(\M_\omega).
    \end{cases}
  \]
  Можем да заключим, че за произволна дума $\omega$ са изпълнени импликациите:
  \begin{align*}
    \omega \in \Laccept & \implies \L(\M_{f(\omega)}) = \Sigma^\star \implies f(\omega) \in L_{\Sigma^\star},\\
    \omega \in \Lcode \setminus \Laccept & \implies \L(\M_{f(\omega)}) = \emptyset \implies f(\omega) \not\in L_{\Sigma^\star}\\
    \omega \not\in\Lcode & \implies f(\omega) = \omega \implies f(\omega) \not\in L_{\Sigma^\star},
  \end{align*}
  които можем да обобщим в следната еквивалентност:
  \[\omega \in \Laccept \iff f(\omega) \in L_{\Sigma^\star}\]
  Ако допуснем, че $L_{\Sigma^\star}$ е разрешим език, то $\Laccept$ също ще е разрешим, което е противоречие.
\end{proof}

\begin{corollary}
  Езикът
  \[\ov{L}_{\texttt{empty}} \df \{\omega \in \{0,1\}^\star \mid \omega \text{ е код на машина на Тюринг и }\L(\M_\omega) \neq \emptyset\}\]
  е полуразрешим, но не е разрешим.
\end{corollary}
\begin{hint}
  Съобразете, че в може да използвате функцията $f$ от доказателството на \Proposition{rice:sigma-star} за да получите, че:
  \[\omega \in \Laccept \iff f(\omega) \in \ov{L}_{\texttt{empty}}.\]
\end{hint}

\begin{corollary}
  Езикът
  \[\Lempty \df \{\omega \in \{0,1\}^\star \mid \omega\text{ е код на машина на Тюринг и }\L(\M_\omega) = \emptyset\}\]
  не е полуразрешим.
\end{corollary}
\begin{hint}
  Ако $\Lempty$ беше разрешим, то неговото допълнение
  \[\ov{L}_{\texttt{empty}} = \Lcode \setminus \Lempty\]
  щеше да е разрешим език, което е противоречие.

  Ако $\Lempty$ беше полуразрешим, тогава, използвайки, че $\ov{L}_{\texttt{empty}}$ е полуразрешим, от теоремата на Клини-Пост щеше да следва, че
  $\Lempty$ е разрешим, което е противоречие
\end{hint}


\begin{important}
  \begin{proposition}
    Езикът
    \[\Lreg \df \{\ \omega \mid \omega\text{ е код на машина на Тюринг и }\L(\M_\omega) \text{ е регулярен език}\ \}\]
    не е разрешим.
  \end{proposition}
\end{important}
\begin{proof}
  \mynote{\cite[стр. 219]{sipser3}}
  Да фиксираме един език, за който знаем, че не е регулярен, например, 
  $\{0^n1^n \mid n \in \Nat\}$.
  Ще дефинираме тотална изчислима функция $f$, която при вход думата $\omega \in \{0,1\}^\star$ работи по следния начин:
  \begin{itemize}
  \item
    Ако $\omega$ не е код на машина на Тюринг, то $f(\omega) \df \omega$.
  \item
    Ако $\omega$ е код на машината на Тюринг $\M_\omega$, то тогава $f(\omega) \df \code{\M'}$,
    където $\M'$, при вход произволна дума $\alpha$, работи така:
    \begin{enumerate}[(1)]
    \item
      Ако $\alpha = 0^n1^n$, за някое $n$, то $\M'$ приема думата $\alpha$.
    \item
      Ако $\alpha$ не е от вида $0^n1^n$, тогава $\M'$ симулира работата на $\M_\omega$ върху думата $\omega$.
    \item
      Ако след краен брой стъпки симулацията завърши с резултат, че $\M_\omega$ приема думата $\omega$, то $\M'$ завършва като приема $\alpha$.
    \end{enumerate}
  \end{itemize}
  \mynote{Използваме наготово, че $\{0,1\}^\star$ е регулярен език.}
  Получаваме, че:
  \[\L(\M') =
    \begin{cases}
      \{0,1\}^\star & \text{ако } \omega \in \L(\M_\omega)\\
      \{0^n1^n \mid n \in \Nat\} & \text{ако } \omega \not\in \L(\M_\omega).
    \end{cases}
  \]
  Сега можем да заключим, че за произволна дума $\omega$ са изполнени импликациите:
  \begin{align*}
    \omega \in \Laccept & \implies \L(\M_{f(\omega)}) = \{0,1\}^\star \implies f(\omega) \in \Lreg,\\
    \omega \in \Lcode \setminus \Laccept & \implies \L(\M_{f(\omega)}) = \{0^n1^n \mid n \in \Nat\} \implies f(\omega) \not\in \Lreg,\\
    \omega \not\in \Lcode & \implies f(\omega) = \omega \implies f(\omega) \not\in \Lreg,
  \end{align*}
  което можем да обединим в еквивалентността:
  \[\omega \in \Laccept \iff f(\omega) \in \Lreg\]
  и ако допуснем, че $\Lreg$ е разрешим език, то $\Laccept$ също ще е разрешим, което е противоречие.  
\end{proof}

Сега ще видим, че идеята, която следвахме в горните доказателства може да се обобщи.
Нека $\Ss$ е множество от полуразрешими езици над фиксирана азбука $\Sigma$.
Ще казваме, че $\Ss$ е свойство на полуразрешимите езици.
Например, 
\[\Ss = \{L \subseteq \Sigma^\star \mid L\text{ е регулярен език}\}.\]
$\Ss$ е {\bf тривиално свойство}, ако $\Ss = \emptyset$ или $\Ss$ съдържа точно всички полуразрешими езици.
Нека разгледаме изброимото множество от всички машини на Тюринг, които разпознават езиците от $\Ss$.
Ще представим това множество като език от кодовете на тези машини на Тюринг, т.е.
\index{$\texttt{Code}(\Ss)$}
\[\texttt{Code}(\Ss) \df \{\omega \mid \text{$\omega$ е код на машина на Тюринг и } \L(\M_\omega) \in \Ss\}.\]
\index{$\texttt{Code}(L)$}
\mynote{Можем да дефинираме и $\texttt{Code}(L)$, което е безкрайно изброимо множество, ако $L$ е полуразрешим език.}

\begin{problem}
  Докажете, че езикът
  \[L_{\texttt{Dec}} = \{\omega \in \{0,1\}^\star \mid \omega \text{ е код на машина на Тюринг и }\L(\M_\omega)\text{ е разрешим}\}\]
  не е разрешим.
\end{problem}

\begin{problem}
  Докажете, че езикът
  \begin{align*}
    L_{\texttt{palin}} \df \{\omega \in \{0,1\}^\star \mid & \ \omega\text{ е код на машина на Тюринг и }\L(\M_\omega)\\
                                                           & \text{ съдържа само думи палиндроми}\}.
  \end{align*}
  не е разрешим.
\end{problem}

Сега вече имаме достатъчно опит за да видим точно кои проблеми са разрешими.

\begin{important}
  \begin{theorem}[Райс 1953 \cite{rice}]
    \index{Райс}
    \mynote{\cite[стр. 188]{hopcroft1}}
    За всяко нетривиално свойство $\Ss$ на полуразрешимите езици,
    $\texttt{Code}(\Ss)$ е неразрешим.
  \end{theorem}
\end{important}
\begin{proof}
  \mynote{Цел: да сведем ефективно $\Laccept$ към $L_\Ss$}
  Без ограничение на общността, нека $\emptyset \not\in \Ss$.
  Понеже $\Ss$ е нетривиално свойство, да разгледаме езика $L \in \Ss$,
  като $\M_L$ е машина на Тюринг, за която $\L(\M_L) = L$.
  Ще дефинираме тотална изчислима функция $f$, която при вход думата $\omega \in \{0,1\}^\star$ работи по следния начин:
  \begin{itemize}
  \item
    Ако $\omega$ не е код на машина на Тюринг, то $f(\omega) \df \omega$.
  \item
    Ако $\omega$ е код на машината на Тюринг $\M_\omega$, то тогава $f(\omega) \df \code{\M'}$,
    където $\M'$, при вход произволна дума $\alpha$, работи така:
    \begin{enumerate}[(1)]
    \item
      първоначално $\M'$ не обръща внимание на входната дума $\alpha$, а започва да симулира работата на $\M_\omega$ върху $\omega$.
    \item
      % \mynote{в този случай ще получим, че $\L(\M') = L$}
      ако след краен брой стъпки симулацията завърши с резултат, че $\M_\omega$ приема думата $\omega$, то 
      $\M'$ започва да симулира работата на $\M_L$ върху входната дума $\alpha$;
    \item
      ако след краен брой стъпки симулацията завърши с резултат, че $\M_L$ приема думата $\alpha$, то 
      $\M'$ приема входната дума $\alpha$;
    \end{enumerate}
  \end{itemize}
  Така получаваме, че:
  \[\L(\M') =
    \begin{cases}
      L, & \text{ако }\omega \in \L(\M_\omega)\\
      \emptyset, & \text{ако }\omega \not\in \L(\M_\omega).
    \end{cases}
  \]
  Оттук заключаваме, че за произволна дума $\omega$ са изпълнени импликациите:
  \begin{align*}
    \omega \in \Laccept & \implies \L(\M_{f(\omega)}) = L \implies f(\omega) \in \texttt{Code}(\Ss),\\
    \omega \in \Lcode \setminus \Laccept & \implies \L(\M_{f(\omega)}) = \emptyset \implies f(\omega) \not\in\texttt{Code}(\Ss),\\
    \omega \not\in \Lcode & \implies f(\omega) = \omega \implies f(\omega) \not\in\texttt{Code}(\Ss),
  \end{align*}
  които можем да обобщим в следната еквивалентност:
  \[\omega \in \Laccept \iff f(\omega) \in \texttt{Code}(\Ss).\]
  Aко допуснем, че $\texttt{Code}(\Ss)$ е разрешимо множество, то ще следва, че $\Laccept$ е разрешимо, което е противоречие.

  Ако $\emptyset \in \Ss$, то правим горните разсъждения за класа от езици
  \[\ov{\Ss} = \{ L \subseteq \Sigma^\star \mid L\text{ е полуразрешим език и } L \not\in\Ss\ \}.\]
  По аналогичен начин доказваме, че $\texttt{Code}(\ov{\Ss})$ не е разрешим език.
  Понеже 
  \[\texttt{Code}(\ov{\Ss}) = \Lcode \setminus \texttt{Code}(\Ss),\]
  то $\texttt{Code}(\Ss)$ също не е разрешим език.
\end{proof}

\begin{corollary}
  За всяко от следните свойства $\Ss$ на полуразрешимите езици, 
  $\texttt{Code}(\Ss)$ {\bf не} е разрешим език, където:
  \mynote{Тук няма нужда нищо да доказваме. Просто съобразяваме, че всяко от тези свойства на полуразрешимите езици е нетривиално.}
  \begin{enumerate}[a)]
  \item 
    $\Ss$ е свойството празнота, т.е. езикът
    \[\texttt{Code}(\Ss) = \{\omega \mid \text{$\omega$ е код на машина на Тюринг и } \L(\M_\omega) = \emptyset\}\]
    не е разрешим;
  \item 
    $\Ss$ е свойството за пълнота, т.е. езикът
    \[\texttt{Code}(\Ss) = \{\omega \mid \text{$\omega$ е код на машина на Тюринг и } \L(\M_\omega) = \Sigma^\star\}\]
    не е разрешим;
  \item
    $\Ss$ е свойството крайност, т.е. езикът
    \[\texttt{Code}(\Ss) = \{\omega \mid \text{$\omega$ е код на машина на Тюринг и }|\L(\M_\omega)| < \infty\}\]
    не е разрешим;
  \item
    $\Ss$ е свойството безкрайност, т.е. езикът
    \[\texttt{Code}(\Ss) = \{\omega \mid \text{$\omega$ е код на машина на Тюринг и }|\L(\M_\omega)| = \infty\}\]
    не е разрешим;
  \item
    $\Ss$ е свойството регулярност, т.е. езикът
    \[\texttt{Code}(\Ss) = \{\omega \mid \text{$\omega$ е код на машина на Тюринг и $\L(\M_\omega)$ е регулярен език}\}\]
    не е разрешим;
  \item
    \mynote{Това свойство е нетривиално, защото вече показахме, че $\{a^nb^nc^n \mid n \in \Nat\}$ е полуразрешим (дори разрешим) език, а знаем отдавна, че този език не е безконтекстен.}
    $\Ss$ е свойството безконтекстност, т.е. езикът
    \[\texttt{Code}(\Ss) = \{\omega \mid \text{$\omega$ е код на машина на Тюринг и $\L(\M_\omega)$ е безконтекстен}\}\]
    не е разрешим;
  \item
    \mynote{Тук също - вече сме разгледали примери за полуразрешими езици, които не са разрешими.}
    $\Ss$ е свойството разрешимост, т.е. езикът
    \[\texttt{Code}(\Ss) = \{\omega \mid \text{$\omega$ е код на машина на Тюринг и $\L(\M_\omega)$ е разрешим}\}\]
    не е разрешим.
  \end{enumerate}
\end{corollary}


%%% Local Variables:
%%% mode: latex
%%% TeX-master: "../eai"
%%% End:
