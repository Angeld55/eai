\subsection{Безконтекстен език за преходите в машина на Тюринг}

% \subsection*{Валидни и невалидни изчисления на машини на Тюринг}
\mynote{\cite[стр. 201]{hopcroft1}}
Да разгледаме машината на Тюринг $\M$.
Една дума $\omega$ описва конфигурация на машина на Тюринг,
ако $\omega \in \Gamma^\star Q \Gamma^\star$.

\begin{framed}
  \begin{proposition}
    Да фиксираме една детерминистична машина на Тюринг $\M$. 
    Тогава следните езици за безконтекстни:
    \begin{align*}
      & \texttt{Valid}(\M) \df \{\ \alpha\sharp\beta^{\rev} \mid \alpha,\beta \in \Gamma^\star Q \Gamma^\star\ \&\ \alpha \vdash_\M \beta\ \} \\
      & \texttt{Valid}'(\M)\df \{\ \alpha^{\rev}\sharp\beta \mid \alpha,\beta \in \Gamma^\star Q \Gamma^\star\ \&\ \alpha \vdash_\M \beta\ \} \\
      & \texttt{Invalid}(\M) \df \{\ \alpha\sharp\beta^{\rev} \mid \alpha,\beta \in \Gamma^\star Q \Gamma^\star\ \&\  \alpha \not\vdash_\M \beta\ \}\\
      & \texttt{Invalid}'(\M) \df \{\ \alpha^{\rev}\sharp\beta \mid \alpha,\beta \in \Gamma^\star Q \Gamma^\star\ \&\ \alpha \not\vdash_\M \beta\ \}.
    \end{align*}
  \end{proposition}  
\end{framed}

\begin{hint}
  Да напомним първо как дефинираме релацията $\vdash_\M$:
  \begin{align*}
    & (\alpha_1, q, x\alpha_2) \vdash_\M  (\alpha_1 y, p, \alpha_2) & \comment{\text{ ако } q \overset{x/y;\goright}{\longrightarrow} p} \\
    & (\alpha_1, q, \varepsilon) \vdash_\M  (\alpha_1 y, p, \varepsilon) & \comment{\text{ ако } q \overset{\blank/y;\goright}{\longrightarrow} p} \\
    & (\alpha_1z, q, x\alpha_2) \vdash_\M (\alpha_1, p ,zy\alpha_2) & \comment{\text{ ако } q \overset{x/y;\goleft}{\longrightarrow} p} \\
    & (\alpha_1z, q, \varepsilon) \vdash_\M (\alpha_1, p ,z) & \comment{\text{ ако } q \overset{\blank/y;\goleft}{\longrightarrow} p} \\
    & (\varepsilon, q, x\alpha_2) \vdash_\M (\varepsilon, p ,\blank y\alpha_2) & \comment{\text{ ако } q \overset{x/y;\goleft}{\longrightarrow} p} \\
    & (\alpha_1, q, x\alpha_2) \vdash_\M (\alpha_1, p, y\alpha_2) & \comment{\text{ ако } q \overset{x/y;\stay}{\longrightarrow} p}.
  \end{align*}

  Думите в езика $\texttt{Valid}(\M)$ кодират релацията $\vdash_\M$. Това означава, че всяка дума на 
  $\texttt{Valid}(\M)$ има някое от следните представяния:
  \mynote{Последните два случая не са нужни, ако машината на Тюринг не може да чете по-наляво от най-лявата клетка на входната дума.}    
  \begin{align*}
    & \alpha_1qx\alpha_2 \sharp \alpha^{\rev}_2 p y \alpha^{\rev}_1 & \comment{\text{ ако } q \overset{x/y;\goright}{\longrightarrow} p} \\
    & \alpha_1q \sharp p y \alpha^{\rev}_1 & \comment{\text{ ако } q \overset{\blank/y;\goright}{\longrightarrow} p} \\
    & \alpha_1zqx\alpha_2 \sharp \alpha^{\rev}_2 y z p \alpha^{\rev}_1 & \comment{\text{ ако } q \overset{x/y;\goleft}{\longrightarrow} p} \\
    & \alpha_1zq\sharp y z p \alpha^{\rev}_1 & \comment{\text{ ако } q \overset{\blank/y;\goleft}{\longrightarrow} p} \\
    & \alpha_1qx\alpha_2 \sharp \alpha^{\rev}_2 y p \alpha^{\rev}_1 & \comment{\text{ ако } q \overset{x/y;\stay}{\longrightarrow} p}\\
    & qx\alpha_2\sharp\alpha_2^{\rev}y\blank p & \comment{\text{ ако } q \overset{x/y;\goleft}{\longrightarrow} p} \\
    & q\sharp y\blank p & \comment{\text{ ако } q \overset{\blank/y;\goleft}{\longrightarrow} p}
  \end{align*}

  Ще опишем неформално стеков автомат $P$ за езика $\texttt{Valid}(\M)$.
  Нека 
  \[Q^{P} \df \{r_q \mid q \in Q^\M\} \cup \{r, \hat{r}\}.\]

  \begin{itemize}
  \item
    Първо четем $\alpha_1$ и я записваме в стека като $\alpha^{\rev}_1$.
    Това правим като дефинираме функцията на преходите като 
    \[(\forall a\in\Sigma)(z \in \Gamma)[\ \Delta_{P}(r,a,z) \df \{(r,az)\}\ ].\]
  \item 
    Правим това докато не срещнем някое $q \in Q^\M$. Тогава трябва да направим преход на $\M$.
    Тук трябва да внимаваме, защото за да направим преход, трябва да знаем състоянието $q$ и да прочетем следващия символ.
    Един начин да разрешим този проблем е като запомним кое състояние сме прочели на машината на Тюринг в състоянията на стековия автомат:
    \mynote{$Q^\M$ са букви от азбуката на стековия автомат $P$.}
    \[(\forall q \in Q^\M)(\forall z \in \Gamma)[\ \Delta_{P}(r,q,z) = \{(r_q,z)\}\ ].\]
    \begin{itemize}
    \item 
      \mynote{Стекът представлява $\alpha^{\rev}_1$. Ако $\alpha_1 = \varepsilon$, то $z$ е символа за дъно на стека.}
      ако $q \overset{x/y;\goright}{\longrightarrow} p$, то слагаме $yp$ на върха на стека, т.е.
      \[\Delta_{P}(r_q,x,z) \ni (\hat{r}, pyz).\]
    \item
      Трябва да внимаваме, когато имаме $q\sharp$, защото това се интерпретира като четящата глава да е върху $\blank$.
      Това означава, че ако $q \overset{\blank/y;\goright}{\longrightarrow} p$, то трябва да имаме и следното:
      \[\Delta_{P}(r_q,\sharp,z) \ni (r_\sharp, pyz).\]
    \item
      ако $q \overset{x/y;\goleft}{\longrightarrow} p$, то ако $z$ е върха на стека, заменяме $z$ с $pzy$, т.е.
      \[\Delta_{P}(r_q,x,z) \ni (\hat{r}, yzp).\]
    \item
      Отново трябва да внимаваме, когато имаме $q\sharp$, защото това се интерпретира като четящата глава да е върху $\blank$.
      Това означава, че ако $q \overset{\blank/y;\goleft}{\longrightarrow} p$, то трябва да имаме и следното:
      \[\Delta_{P}(r_q,\sharp,z) \ni (r_\sharp, yzp).\]
    \item
      ако $q \overset{x/y;\stay}{\longrightarrow} p$, то ако $z$ е върха на стека, заменяме $z$ с $ypz$, т.е.
      \[\Delta_{P}(r_q,x,z) \ni (\hat{r}, ypz).\]
    \item
      Ако позволяваме машината на Тюринг да се движи по-наляво от най-лявата клетка на лентата, върху която е записана входната дума, то трябва да
      разгледаме и случая, когато $\alpha_1 = \varepsilon$, т.е. $z$ е символа за дъно на стека. Тогава:
      \begin{itemize}
      \item
        Ако $q \overset{x/y;\goleft}{\longrightarrow} p$, то
        \[\Delta_{P}(r_q,x,z) \ni (\hat{r},y\blank pz).\]
      \item
        Ако $q \overset{\blank/y;\goleft}{\longrightarrow} p$, то
        \[\Delta_{P}(r_q,\sharp,z) \ni (r_\sharp,y\blank pz).\]
      \end{itemize}
    \end{itemize}
  \item
    Сега вече сме в състояние $\hat{r}$ и остава да прочетем $\alpha_2$ и да я запишем в стека като $\alpha^{\rev}_2$:
    \[\Delta_{P}(\hat{r},x,z) = \{(\hat{r}, xz)\}.\]
  \item
    \mynote{За да разпознаем $\texttt{Invalid}(\M)$ трябва само да разменим условията за приемане и отхвърляне на думата.}
    Разбираме кога сме свършили с $\alpha_2$ когато стигнем до $\sharp$. Преминаваме в състояние $r_\sharp$ и
    започваме да четем думата след $\sharp$ и сравняваме с това, което имаме в стека.
    \begin{itemize}
    \item
      Ако намерим разлика, то отхвърляме думата.
    \item
      Ако достигнем до дъното на стека, то приемаме думата.
    \end{itemize}
  \end{itemize}
\end{hint}

\begin{remark}
  Да обърнем внимание, че горната конструкция на стековия автомат $P$ е {\bf ефективна}, т.е.
  съществува алгоритъм, който при вход код на машина на Тюринг $\M$ връща като изход код на стеков автомат $P$ за езика $\texttt{Valid}(\M)$.
  % С други думи, езикът 
  % \[\{\code{\M} \cdot \code{P} \mid \L(P) = \texttt{Valid}(\M)\}\]
  % е разрешим.
\end{remark}

\subsection*{История на машина на Тюринг}
\index{история на приемащо изчисление}

Дума от вида $\omega_1 \sharp \omega_2 \sharp \omega_3 \sharp \omega_4\sharp\cdots\omega_n\sharp$ се нарича {\bf история на приемащо изчисление} на машината на Тюринг $\M$, ако
\begin{itemize}
\item
  $\omega_i \in \Gamma^\star Q \Gamma^\star$, т.е. $\omega_i$ описва моментна конфигурация
  и $\omega_i$ не започва и не завършва на $\blank$.
\item
  $\omega_1 \in \qstart\Sigma^\star$ описва начална конфигурация.
\item
  $\omega_n \in \Gamma^\star \cdot\{\qaccept\} \cdot \Gamma^\star$ описва приемаща конфигурация.
\item
  За четно $i < n$, $\omega^{\rev}_i \vdash_\M \omega_{i+1}$;
\item
  За нечетно $i < n$, $\omega_i \vdash_\M \omega^{\rev}_{i+1}$;
\end{itemize}
Езикът съставен от всички такива думи ще означаваме с $\texttt{History}(\M)$.

\begin{lemma}
  \mynote{\cite[стр. 201]{hopcroft1}}
  За всяка машина на Тюринг $\M$ е изпълнено, че:
  \[\texttt{History}(\M) = L_1 \cap L_2,\]
  където $L_1$ и $L_2$ са безконтекстни езици.
  Освен това, граматиките на $L_1$ и $L_2$ могат ефективно да бъдат построени по кода на $\M$.
\end{lemma}
\begin{hint}
  Разгледайте следните безконтекстни езици:
  \begin{align*}
    & L_1 \df (\texttt{Valid}(\M) \cdot \{\sharp\})^\star \cdot (\{\varepsilon\}\cup \Gamma^\star \cdot \{\qaccept\} \cdot \Gamma^\star \cdot \{\sharp\})\\
    & L_2 \df \{\qstart\}\cdot\Sigma^\star \cdot \{\sharp\} \cdot (\texttt{Valid'}(\M) \cdot \{\sharp\})^\star \cdot (\{\varepsilon\}\cup \Gamma^\star \cdot \{\qaccept\} \cdot \Gamma^\star\cdot\{\sharp\}),
  \end{align*}
\end{hint}

\begin{problem}
  Обяснете как може ефективно да се кодира всяка безконтекстна граматика $G$ като дума $\code{G}$ над азбуката $\{0,1\}$.
\end{problem}

\begin{important}
\begin{theorem}\label{th:computations:intersect}
  Езикът
  \[\overline{\INTERSECT} = \{\code{G_1}\cdot\code{G_2} \mid \text{$G_1$ и $G_2$ са безконт. грам. и }\L(G_1) \cap \L(G_2) = \emptyset\}\]
  не е полуразрешим.
\end{theorem}  
\end{important}
\mynote{Да напомним, че от \Corollary{pcp:grammar-intersect} знаем, че $\INTERSECT$ не е разрешим.}
\begin{hint}
  Лесно се вижда, че $\LEmpty \leq_m \overline{\INTERSECT}$.
  По дадена дума $\code{\M}$, можем ефективно да намерим $G_1$ и $G_2$, за които
  $\L(G_1) \cap \L(G_2) = \texttt{Accept}(\M)$, т.е. съществува тотална изчислима функция $f$, за която
  \[f(\code{\M}) = \code{G_1} \cdot \code{G_2}.\]
  Тогава ако $L$ е полуразрешим език, то $\LEmpty$ е полуразрешим език, което е противоречие, защото
  \[\code{\M} \in L_{\texttt{Empty}} \iff f(\code{\M}) \in \overline{\INTERSECT}.\]
\end{hint}

\begin{lemma}
  За всяка машина на Тюринг $\M$,о допълнението на $\texttt{History}(\M)$, което означаваме като $\overline{\texttt{History}(\M)}$, е безконтекстен език.
  Освен това, по кода на $\M$ можем ефективно да намерим код на безконтекстната граматика за $\overline{\texttt{History}}(\M)$.
\end{lemma}
\begin{hint}
  Една дума $\alpha$ не е история на приемащо изчисление, ако е изпълнено някое от следните условия:
  \begin{itemize}
  \item 
    \mynote{Можем да опишем това свойство с регулярен език}
    $\alpha$ не е от вида $\omega_1 \sharp \omega_2 \sharp \cdots \sharp \omega_n\sharp$,
    където $\omega_i \in \Gamma^\star Q \Gamma^\star$, или
  \item
    ако $\alpha$ е от вида $\omega_1 \sharp \omega_2 \sharp \cdots \sharp \omega_n\sharp$,
    където $\omega_i \in \Gamma^\star Q \Gamma^\star$, тогава:
    \begin{itemize}
    \item 
      $\omega_1 \not\in \{\qstart\} \cdot \Gamma^\star$, или
    \item
      $\omega_n \not\in \Gamma^\star \cdot \{\qaccept\} \cdot \Gamma^\star$, или
    \item
      $\omega_i \not\vdash_\M \omega^{\rev}_{i+1}$, т.е. $\omega_i \sharp \omega^{\rev}_{i+1} \in \texttt{Invalid}(\M)$, за някое нечетно $i < n$, или
    \item
      $\omega^{\rev}_i \not\vdash_\M \omega_{i+1}$, т.е. $\omega^{\rev}_i \sharp \omega_{i+1} \in \texttt{Invalid'}(\M)$, за някое четно $i < n$.
    \end{itemize}
  \end{itemize}
  Думите притежаващи някое от тези свойства могат да се опишат като обединение на три регулярни езика, което е лесно защото регулярните езици са затворени относно допълнение, и двата безконтекстни езика
  $\texttt{Invalid}(\M)$ и $\texttt{Invalid'}(\M)$.
\end{hint}

\begin{important}
  \begin{theorem}\label{th:computations:all-cfg}
    За дадена азбука $\Sigma$, езикът 
    \[\texttt{All}_{\texttt{CFG}} = \{\code{G} \mid G\text{ е безконтекстна граматика и }\L(G) = \Sigma^\star\}\]
    не е полуразрешим.
  \end{theorem}
\end{important}
\begin{hint}
  Ще видим, че имаме следното свеждане:
  \[\LEmpty \leq_m \texttt{All}_{\texttt{CFG}}.\]
  Тук ще използваме, че ако $\L(\M) = \emptyset$, то $\overline{\texttt{History}(\M)} = \hat{\Sigma}^\star$,
  където $\hat{\Sigma} = \Gamma \cup Q \cup \{\sharp\}$.
  По даден код на машината на Тюринг $\M$, можем ефективно да намерим код на безконтекстната граматика $G$, за която
  $\L(G)$ са точно невалидните изчисления на $\M$, т.е. съществува тотална изчислима $f$, за която
  \[f(\code{\M}) = \code{G}\text{ и }\L(G) = \overline{\texttt{History}(\M)}.\]
  Тогава, ако допуснем, че $\texttt{All}_{\texttt{CFG}}$ е полуразрешим език, то $\LEmpty$ е полуразрешим, защото
  \[\code{\M} \in L_{\texttt{Empty}} \iff f(\code{\M}) \in \texttt{All}_{\texttt{CFG}}.\]
\end{hint}

\begin{corollary}
  Следните езици не са полуразрешим:
  \begin{enumerate}[a)]
  \item
    $\{\code{G_1}\cdot\code{G_2} \mid \text{$G_1$ и $G_2$ са безконт. грам. и }\L(G_1) = \L(G_2)\}$;
  \item
    $\{\code{G_1}\cdot\code{G_2} \mid \text{$G_1$ и $G_2$ са безконт. грам. и }\L(G_1) \subseteq \L(G_2)\}$;
  \item 
    $\{\code{G}\cdot r \mid \text{$G$ е безконт. грам. и $r$ е рег. израз и }\L(G) = \L(r)\}$;
  \item
    $\{\code{G}\cdot \code{\A} \mid \text{$G$ е безконт. грам. и $\A$ е ДКА и }\L(G) = \L(\A)\}$;
  \item 
    $\{\code{G}\cdot r \mid \text{$G$ е безконт. грам. и $r$ е рег. израз и }\L(r) \subseteq \L(G)\}$;
  \item
    $\{\code{G}\cdot \code{\A} \mid \text{$G$ е безконт. грам. и $\A$ е ДКА и }\L(\A) \subseteq \L(G)\}$.
  \end{enumerate}
\end{corollary}

\begin{remark}
  Добре е да обърнем внимание, че езикът 
  \[L = \{\code{G}\cdot \code{\A} \mid \text{$G$ е безконт. грам. и $\A$ е ДКА и }\L(G) \subseteq \L(\A)\}\]
  е разрешим.
  Това е така, защото $\L(G) \subseteq \L(\A) \iff \L(G) \cap \L(\ov{\A}) = \emptyset$,
  защото сечението на безконтекстен и регулярен език е безконтекстен език.
\end{remark}


%%% Local Variables:
%%% mode: latex
%%% TeX-master: "../eai"
%%% End:
