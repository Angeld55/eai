\subsection*{Валидни и невалидни изчисления на машини на Тюринг}
\marginpar{\cite{hopcroft1} стр. 201}
Да разгледаме машината на Тюринг $\M$.
Дума от вида 
\[\omega_1 \sharp \omega^{rev}_2 \sharp \omega_3 \sharp \omega^{rev}_4\sharp\omega_5\cdots\]
ще наричаме валидно изчисление на $\M$, ако 
\begin{itemize}
\item
  всяко $\omega_i \in \Gamma^\star Q \Gamma^\star$ описва моментна конфигурация
  и $\omega_i$ не завършва на $\blank$.
\item
  $\omega_1 \in s\Sigma^\star$ описва начална конфигурация.
\item
  $\omega_n \in \Gamma^\star Q \Gamma^\star$ описва финална конфигурация.
\item 
  $\omega_i \vdash_\M \omega_{i+1}$ за $i = 1,\dots,n-1$.
\end{itemize}

% \begin{prop}
%   Езикът
%   $L = \{\omega_1 \sharp \omega^{rev}_2 \mid \omega_1, \omega_2 \in \Gamma^\star Q \Gamma^\star\ \&\ \omega_1 \vdash_\M \omega_2\}$
%   е безконтекстен.
% \end{prop}


\begin{lemma}
  \marginpar{\cite{hopcroft1}, стр. 201}
  Множеството от валидни изчисления на машина на Тюринг $\M$ е сечението на два безконтекстни езика $L_1$ и $L_2$.
  Освен това, граматиките на $L_1$ и $L_2$ могат ефективно да бъдат построени от $\M$.
\end{lemma}
\begin{hint}
  Да разгледаме езика
  \[L_3 = \{\alpha\#\beta^{rev} \mid \alpha \vdash_\M \beta\}.\]
  Лесно е да построим стеков автомат $P_3$, който разпознава езика $L_3$.
  Четем буквата $X$. Тогава:
  \begin{itemize}
  \item 
    ако $\delta_\M(q,X) =(p,Y,R)$, то слагаме $Yp$ на върха на стека;
  \item
    ако $\delta_\M(q,X) =(p,Y,L)$, то ако $Z$ е върха на стека, заменяме $Z$ с $pZY$;
  \end{itemize}
  Аналогично разглеждаме безконтекстния език
  \[L_4 = \{\alpha^{rev}\#\beta \mid \alpha \vdash_\M \beta\}.\]
  Сега можем да дефинираме езиците
  \begin{align*}
    & L_1 = (L_3\#)^\star(\{\varepsilon\}\cup \Gamma^\star q_{accept} \Gamma^\star\#)\\
    & L_2 = q_0\Sigma^\star \# (L_4\#)^\star(\{\varepsilon\}\cup \Gamma^\star q_{accept} \Gamma^\star\#),
  \end{align*}
  за които е ясно, че са безконтекстни.
  Тогава езикът от валидните изчисления на $\M$
  е $L_1 \cap L_2$.
\end{hint}

\begin{thm}
  Езикът
  \[L = \{\code{G_1}\cdot\code{G_2} \mid \text{$G_1$ и $G_2$ са безконтекстни граматики и }\L(G_1) \cap \L(G_2) = \emptyset\}\]
  е неразрешим.
\end{thm}
\begin{proof}
  По дадена дума $\code{\M}$, можем ефективно да намерим $G_1$ и $G_2$, за които
  $\L(G_1) \cap \L(G_2)$ са точно валитните изчисления на $\M$.
  Тогава ако $L$ е разрешим език, то $L_{\texttt{Empty}}$ е разрешим език, което е противоречие.
\end{proof}

\begin{lemma}
  Нека е дадена $\M = \TM$.
  Тогава езикът 
  \[L = \{\alpha\sharp\beta^R \mid \alpha,\beta \in \Gamma^\star Q \Gamma^\star\ \&\  \alpha \not\vdash_\M \beta\}\]
  е безконтекстен.
\end{lemma}

\begin{lemma}
  Множеството от невалидни изчисления на машина на Тюринг е безконтекстен език.
\end{lemma}
\begin{proof}
  
\end{proof}

\begin{framed}
  \begin{thm}
    За дадена азбука $\Sigma$, 
    езикът 
    \[L = \{\code{G} \mid G\text{ е безконтекстна граматика и }\L(G) = \Sigma^\star\}\]
    е неразрешим.
  \end{thm}
\end{framed}
\begin{proof}
  По дадена дума $\code{\M}$, можем ефективно да намерим $G$, за която
  $\L(G)$ са точно невалитните изчисления на $\M$.
  Тогава ако допуснем, че $L$ е разрешим език, то $L_{\texttt{Empty}}$ е разрешим, което е противоречие.
\end{proof}

\begin{cor}
  Следните езици не са разрешими:
  \begin{enumerate}[a)]
  \item
    $L = \{\code{G_1}\cdot\code{G_2} \mid \text{$G_1$ и $G_2$ са безконт. грам. и }\L(G_1) = \L(G_2)\}$;
  \item
    $L = \{\code{G_1}\cdot\code{G_2} \mid \text{$G_1$ и $G_2$ са безконт. грам. и }\L(G_1) \subseteq \L(G_2)\}$;
  \item 
    $L = \{\code{G}\cdot r \mid \text{$G$ е безконт. грам. и $r$ е рег. израз и }\L(G) = \L(r)\}$;
  \item
    $L = \{\code{G}\cdot \code{\A} \mid \text{$G$ е безконт. грам. и $\A$ е ДКА и }\L(G) = \L(\A)\}$;
  \item 
    $L = \{\code{G}\cdot r \mid \text{$G$ е безконт. грам. и $r$ е рег. израз и }\L(r) \subseteq \L(G)\}$;
  \item
    $L = \{\code{G}\cdot \code{\A} \mid \text{$G$ е безконт. грам. и $\A$ е ДКА и }\L(\A) \subseteq \L(G)\}$;
  \end{enumerate}
\end{cor}
% \begin{hint}
%   \begin{enumerate}
%   \item 
%     Фикс. $\L(G_1) = \Sigma^\star$.
%   \item
%     Фикс. $\L(G_1) = \Sigma^\star$.
%   \end{enumerate}
% \end{hint}


\begin{remark}
  Добре е да обърнем внимание, че езикът 
  \[L = \{\code{G}\cdot \code{\A} \mid \text{$G$ е безконт. грам. и $\A$ е ДКА и }\L(G) \subseteq \L(\A)\}\]
  е разрешим.
  Това е така, защото $\L(G) \subseteq \L(\A) \iff \L(G) \cap \L(\ov{\A}) = \emptyset$,
  което сечението на безконтекстен и регулярен език е безконтекстен език.
\end{remark}


% \begin{cor}
%   Нека $G_1$ и $G_2$ са произволни безконтекстни граматики, а $r$ е произволен регулярен израз.
%   Следните проблеми са неразрешими:
%   \begin{enumerate}
%   \item 
%     $\L(G_1) = \L(G_2)$;
%   \item
%     $\L(G_2) \subseteq \L(G_1)$;
%   \item
%     $\L(G_1) = \L(r)$;
%   \item
%     $\L(r) \subseteq \L(G_1)$.
%   \end{enumerate}
% \end{cor}

%%% Local Variables:
%%% mode: latex
%%% TeX-master: "../eai"
%%% End:
