\documentclass[a4paper, 12pt, oneside]{book}
% \usepackage[mathjax,lwarpmk]{lwarp}

% \usepackage[left=0.0cm, right=0.5cm]{geometry}
% \special{papersize=9cm,12cm}

\usepackage{cmap}
\usepackage{gitinfo2}


%%%%%%%%%%%%%%%%%%%%%%%%%%%%%%%%%%%%%%%%%%%%%%%%%%%%%%%%%%%%%%%%%%%%%%%%%%%
% ИНДЕКСИ
%%%%%%%%%%%%%%%%%%%%%%%%%%%%%%%%%%%%%%%%%%%%%%%%%%%%%%%%%%%%%%%%%%%%%%%%%%%

% работи само с memoir:
%\xindyindex\makeindex\makeindex[names]

%\usepackage{index}\makeindex%[program=truexindy,options=-M texindy -L bulgarian -C utf8]
%\usepackage{makeidx}\makeindex%\makeatletter
%\def\imki@progdefault{xindy} \makeatother
%\renewcommand{\indexname}{New name}

% вж. 'How to move from makeindex to xindy?.html'
\usepackage[xindy]{imakeidx}
\def\xindylangopt{-M lang/bulgarian/utf8-lang.xdy}
\makeindex[options = \xindylangopt]
% \makeindex[name=names, title = Именен указател, options = \xindylangopt]
% \makeindex[name=proglangs, title = Изчислителни машини и програмни езици, options = \xindylangopt]
% \makeindex[name=organisations, title = Организации и фирми, options = \xindylangopt]

% \makeindex[options = \xindylangopt]
\makeatletter
\let\original@index\index
\renewcommand{\index}[2][\imki@jobname]{%
  \original@index[#1]{\detokenize{#2}}%
}



%%%%%%%%%%%%%
%% MARGINS %%
%%%%%%%%%%%%%

% \reversemarginpar
% \usepackage{marginnote}
\setlength{\marginparsep}{1cm}
\setlength{\oddsidemargin}{0.3cm}
\setlength{\hoffset}{-0.75in}
\setlength{\voffset}{-0.75in}
\setlength{\marginparwidth}{110pt}
\setlength{\textwidth}{420pt}
\setlength{\textheight}{670pt}

% \setlength{\leftmargin}{0.2cm}

\let\oldmarginpar\marginpar
\renewcommand\marginpar[1]{\leavevmode\oldmarginpar{\raggedright\scriptsize #1}}
% \renewcommand\marginpar[1]{}

  
% \renewcommand\marginpar[1]{\-\oldmarginpar[\raggedleft\scriptsize #1]%
% {\raggedright\scriptsize #1}}
%\renewcommand\marginpar[1]{\oldmarginpar{\scriptsize #1}}

%%%%%%%%%%%%%%

\usepackage[T2A]{fontenc}
\usepackage[bulgarian]{babel}
% \usepackage[OT2,T1]{fontenc}
\usepackage[utf8]{inputenc}

\usepackage[pdfencoding=unicode, colorlinks=true, linkcolor=blue, pdfstartview=FitV, citecolor=green, urlcolor=blue]{hyperref}
\usepackage{amssymb, amsmath, amsthm, mathrsfs, latexsym, bm, mathtools}
\hypersetup{pdfauthor={Стефан Вътев},pdftitle={Езици, Автомати и Изчислимост},pdftex,unicode}

\usepackage{epigraph}
\usepackage{makeidx}
\usepackage{layout, framed}
\usepackage{qtree, bussproofs, algorithm}
\usepackage[noend]{algpseudocode}
\usepackage{float}

\usepackage{paralist}
\usepackage[shortlabels]{enumitem}
\setlist{leftmargin=*}

\usepackage{tikz, pgf}
\usetikzlibrary{arrows, automata, positioning, backgrounds, decorations.pathmorphing, decorations.markings}


\usepackage{multicol}
\usepackage{color}
 
% \usepackage{comment}

\setlength{\columnsep}{2pc}
\setlength{\columnseprule}{.5pt}
\def\columnseprulecolor{\color{gray}}


\usepackage[font=small,format=plain,labelfont=bf,textfont=normal,justification=justified]{caption}
\usepackage[justification=justified]{subcaption}
\usepackage[activate={true,nocompatibility},final,tracking=true,kerning=true,spacing=true,factor=1100,stretch=10,shrink=10]{microtype}

\usepackage{mybgdef}
\usepackage{mysymbols}
\usepackage{eai}

\usepackage{pifont}
\newcommand{\writedown}{\ding{45}\ }

\newcommand{\qstart}{q_{\texttt{start}}}
\newcommand{\qaccept}{q_{\texttt{accept}}}
\newcommand{\qreject}{q_{\texttt{reject}}}

\newcommand{\FA}{\langle{\Sigma,Q,\qstart,\delta,F}\rangle}
\newcommand{\FAn}[1]{\langle{\Sigma,Q_#1,s_#1,\delta_#1,F_#1}\rangle}
\newcommand{\NFA}{\langle{\Sigma,Q,\qstart,\Delta,F}\rangle}
\newcommand{\NFAn}[1]{\langle{\Sigma,Q_#1,s_#1,\Delta_#1,F_#1}\rangle}
\newcommand{\PDA}{\langle{Q,\Sigma,\Gamma,\sharp,\Delta,\qstart,\qaccept}\rangle}
\newcommand{\PDAn}[1]{\langle{Q_#1,\Sigma,\Gamma,\#,\s_#1,\Delta_#1,F_#1}\rangle}
\newcommand{\CFG}{\langle{V,\Sigma,R,S}\rangle}
\newcommand{\TM}{\langle{Q,\Sigma,\Gamma,\delta,\blank,\qstart,\qaccept, \qreject}\rangle}

\newcommand{\Theorem}[1]{{\em Теорема~\ref{th:#1}}}
\newcommand{\Lemma}[1]{{\em Лема~\ref{lem:#1}}}
\newcommand{\Corollary}[1]{{\em Следствие~\ref{cor:#1}}}
\newcommand{\Problem}[1]{{\em Задача~\ref{prob:#1}}}
\newcommand{\Proposition}[1]{{\em Твърдение~\ref{pr:#1}}}
\newcommand{\Example}[1]{{\em Пример~\ref{ex:#1}}}
\newcommand{\Figure}[1]{{\em Фигура~\ref{fig:#1}}}
\newcommand{\Property}[1]{{\em Свойство~(\ref{#1})}}

\newif\ifhints
\newif\ifcode                   % Използва се само за един пример на си++
\hintstrue
% \codetrue

\usepackage[scaled=0.89]{PTSerif}
\usepackage[scaled=0.89]{PTSans}


\ifcode
\usepackage{minted}
\fi

% \setsecheadstyle{\large\usefont{T2A}{fag}{b}{r}} %\scshape
% \setsubsecheadstyle{\bfseries\sffamily}

% \renewcommand\familydefault{\sfdefault}

\title{Записки по ,,Езици, автомати, изчислимост''}
\author{Стефан Вътев\thanks{Това е чернова. Възможни са неточности и грешки, а също и несъответсвия с терминологията въведена на лекции. За забележки и коментари: \href{mailto:stefanv@fmi.uni-sofia.bg}{stefanv@fmi.uni-sofia.bg}.
    Конкретната версия на \LaTeX файловете, от които е компилиран този файл, може да бъде намерена
    \href{https://github.com/stefk0/eai/commit/\gitHash}{тук} от \gitAuthorDate.}
% Revision\gitRel: \gitAbbrevHash{} (\gitBranch)\\
% \gitAuthorDate}
  }

% \makeindex
\begin{document} 
\maketitle
  % \layout

\begin{tikzpicture}
\pgftransformscale{.8}

%%% HELP LINES - uncomment to design/extend
% \draw[step=1cm,gray,very thin] (-10,0) grid (10,12);
% \node at (0,0) {\textbf{(0,0)}};

%% Horizontal bar
\draw[very thick] (10,0) -- (-10,0);

% LOG TIME
% \draw (-1,0) parabola bend (0,2) (1,0) ;
% \node at (0,1) {
%   \begin{tabular}{c}
%     Крайни \\ езици
%   \end{tabular}
% };

% LOG SPACE
\draw (-3,0) parabola bend (0,3.5) (3,0);
\node at (0,2) {
	\begin{tabular}{c}
          Крайни \\ езици
	\end{tabular}
};

\draw (-5.5,0) parabola bend (0,6) (5.5,0);
\node at (0,5) {
  \begin{tabular}{c}
    Регулярни \\ езици
  \end{tabular}
};

\draw (-7.5,0) parabola bend (0,8) (7.5,0);
\node at (0,7) {  
  \begin{tabular}{c}
    Безконтекстни \\ езици
  \end{tabular}
};

\draw (-9.5,0) parabola bend (0,10) (9.5,0);
\node at (0,9) {  
  \begin{tabular}{c}
    Разрешими \\ езици
  \end{tabular}
};

\draw[very thick] (-9.5,0) parabola bend (0,12.5) (9.5,0);
\node at (0,11) {Полуразрешими езици};
\end{tikzpicture}

%%% Local Variables:
%%% mode: latex
%%% TeX-master: "eai"
%%% End:


\tableofcontents

\chapter{Увод}
\label{ch:intro}

\section{Съждително смятане}\label{sect:propositional}
\marginpar{На англ. Propositional calculus}

Както при езиците за програмиране, всяка логика има свой синтаксис и семантика.
Тук ще разгледаме класическата съждителна логика, при която те са сравнително прости.

Съждителното смятане наподобява аритметичното смятане, като вместо аритметичните операции $+,-,\cdot,/$, 
имаме съждителни операции като $\neg, \wedge, \vee$.
Например, $(p\vee q) \wedge \neg  r$ е съждителна формула.
Освен това, докато аритметичните променливи приемат стойности числа, то
съждителните променливи приемат само стойности {\bf истина (1)} или {\bf неистина (0)}.

\marginpar{Това не е формална дефиниция, но за момента е достатъчно.}
{\bf Съждителна формула} наричаме съвкупността от съждителни променливи $p,q,r,\dots$, свързани със знаците за логически операции
$\neg$, $\vee$, $\wedge$, $\rightarrow$, $\leftrightarrow$ и скоби, определящи реда на операциите.

\subsection*{Съждителни операции}

\begin{itemize}
\item
  Отрицание $\neg$
\item 
  Дизюнкция $\vee$
\item
  Конюнкция $\wedge$
\item
  Импликация $\rightarrow$
\item
  Еквивалентност $\iff$
\end{itemize}

Ще използваме таблица за истинност за да определим стойностите на основните съждителни операции
при всички възможни набори на стойностите на променливите.

\[
\begin{array}{|c|c|c|c|c|c|c|c|c|}
  \hline
  p & q & \neg p & p \vee q & p \wedge q & p \rightarrow q & \neg p \vee q & p \iff q & (p \wedge q)\ \vee\ (\neg p\wedge \neg q) \\
  \hline
  0 & 0 & 1 & 0 & 0 & 1 & 1 & 1 & 1\\
  \hline
  0 & 1 & 1 & 1 & 0 & 1 & 1 & 0 & 0\\
  \hline
  1 & 0 & 0 & 1 & 0 & 0 & 0 & 0 & 0\\
  \hline
  1 & 1 & 0 & 1 & 1 & 1 & 1 & 1 & 1\\
  \hline
\end{array}
\]


{\bf Съждително верен} (валиден) е този логически израз, който има верностна стойност {\bf 1} при всички възможни набори на
стойностите на съждителните променливи в израза, т.е. стълбът на израза в таблицата за истинност трябва да съдържа само 
стойности {\bf 1}. 

Два съждителни израза $\varphi$ и $\psi$ са {\bf еквивалентни}, което означаваме $\varphi \equiv \psi$, ако са съставени от 
едни и същи съждителни променливи и двата израза имат едни и същи верностни стойности при всички комбинации от верностни 
стойности на променливите. С други думи, колоните на двата израза в съответните им таблици за истинност трябва да съвпадат.
Така например, от горната таблица се вижда, че 
$p\to q \equiv \neg p \vee q$ и $p \iff q \equiv (\neg{p}\wedge q)\ \vee\ (p\wedge \neg q)$.

\subsection{Съждителни закони}

\begin{enumerate}[I)]
\item
  {\bf Закон за идемпотентността}
  \[p \land p \equiv p\]
  \[p \lor p \equiv p\]
\item
    {\bf Комутативен закон}
    \[p\vee q \equiv q\vee p\] 
    \[p \wedge q \equiv q \wedge p\]
  \item
    {\bf Асоциативен закон}
    \[(p\vee q)\vee r \equiv p\vee(q\vee r)\]
    \[(p\ \wedge\ q)\ \wedge\ r \equiv p\ \wedge\ (q\ \wedge\ r)\]
  \item
    {\bf Дистрибутивен закон}
    \[p\ \wedge\ (q \vee r) \equiv (p\ \wedge q)\vee (p\ \wedge\ r)\]
    \[p\vee (q\ \wedge\ r) \equiv (p\vee q)\ \wedge\ (p\vee r)\]
  \item
    {\bf Закони на де Морган}
    \[\neg(p \wedge q) \equiv (\neg p \vee \neg q)\]
    \[\neg(p\vee q) \equiv (\neg p \wedge \neg q)\]
  \item
    {\bf Закон за контрапозицията}
    \[p\rightarrow q \equiv \neg q \rightarrow \neg p\]
  \item
    {\bf Обобщен закон за контрапозицията}
    \[(p \wedge q)\rightarrow r \equiv (p \wedge \neg r) \rightarrow \neg q\]
  \item
    {\bf Закон за изключеното трето}
    \[p\vee \neg p \equiv {\mathbf 1}\]
  \item
    {\bf Закон за силогизма (транзитивност)}
    \[ ((p\rightarrow q)\ \wedge\ (q\rightarrow r)) \rightarrow (p\rightarrow r) \equiv {\mathbf 1}\]
\end{enumerate}

Лесно се проверява с таблиците за истинност, че законите са валидни.

\subsection{Нормални форми}

\begin{itemize}
\item
  Конюнктивна нормална форма
\item
  Дизюнктивна нормална форма
\end{itemize}


%%% Local Variables:
%%% mode: latex
%%% TeX-master: "../eai"
%%% End:


\section{Предикати и квантори}

\subsection*{Квантори}

Свойствата или отношенията на елементите в произволно множество се наричат {\bf предикати}.
Нека да разгледаме един едноместен предикат $P(\cdot)$.

\bigskip
\begin{tabular}{|l|p{4.2cm}|p{4.5cm}|}
  \hline
  твърдение & Кога е истина? & Кога е неистина?\\
  \hline
  $\forall x P(x)$ & $P(x)$ е вярно за всяко $x$ & съществува $x$, за което $P(x)$ {\bf не} е вярно \\
  \hline
  $\exists x P(x)$ & съществува $x$, за което $P(x)$ е вярно & $P(x)$ {\bf не} е вярно за всяко $x$\\
  \hline
\end{tabular}  
\bigskip

\begin{enumerate}[(I)]
\item 
  {\bf Квантор за общност} $\forall x$.
  Записът $(\forall x \in A) P(x)$ означава, че за всеки елемент $a$ в $A$, 
  твърдението $P(a)$ има стойност истина.
  Например, $(\forall x\in\Real)[(x+1)^2 = x^2+2x+1]$.
\item
  {\bf Квантор за съществуване} $\exists x$.
  Записът $(\exists x \in A) P(x)$ означава, че съществува елемент $a$ в $A$, 
  за който твърдението $P(a)$ има стойност истина.
  Например, $(\exists x \in\mathbb{C})[x^2 = -1]$, но $(\forall x\in\Real)[x^2 \neq -1]$.
\end{enumerate}

% \begin{example}
%   \begin{itemize}
%   \item
%     За всяко естествено число, съществува по-голямо от него:
%     \[(\forall x\in\Nat)(\exists z\in\Nat)[x < z].\]
%   \item
%     Съществува естествено число, от което няма по-малко:
%     \[(\exists x\in\Nat)(\forall y\in\Nat)[x < y \vee x = y].\]
%     Нека да означим с $Zero(x)$ предиката, който казва, че $x$ е най-малкото число, т.е.
%     \[Zero(x) \equiv (\forall y)[x < y \vee x =y].\]
%   \item
%     Нека $S(x,y)$ да бъде предиката, който казва, че $y = x+1$ в естествените числа:
%     \[S(x,y) \equiv (x < y\ \wedge\ (\forall z\in\Nat)[x < z\ \rightarrow (z = y\ \vee\ y < z)].\]
%   \item
%     $One(x)$ - $x$ е числото $1$:
%     \[One(x) \equiv (\exists y)[Z(y)\ \wedge\ S(y,x)].\]
%   \item
%     $Div(x,y)$ - $x$ се дели на $y$:
%     \[Div(x,y) \equiv (\exists z)[x = y.z].\]
%   \item
%     $Prime(x)$ - $x$ е просто число:
%     \[Prime(x) \equiv x \geq 2\ \wedge\ (\forall y\in\Nat)[\neg (O(y)\ \wedge Z(y))\ \rightarrow\ \neg Div(x,y)].\]
%   \end{itemize}
% \end{example}


\subsection*{Закони на предикатното смятане}

\begin{enumerate}[(I)]
  \item
    $\neg\forall x P(x) \iff \exists x \neg P(x)$
  \item
    $\neg\exists x P(x) \iff \forall x \neg P(x)$
  \item
    $\forall x P(x) \iff \neg\exists x \neg P(x)$
  \item
    $\exists x P(x) \iff \neg\forall x \neg P(x)$
  \item
    $\forall x \forall y P(x,y) \iff \forall y\forall x P(x,y)$
  \item
    $\exists x\exists y P(x,y) \iff \exists y \exists x P(x,y)$  
  \item
    $\exists x\forall y P(x,y) \rightarrow \forall y \exists x P(x,y)$
\end{enumerate}

\bigskip
\begin{tabular}{|l|p{2.5cm}|p{3.2cm}|p{3cm}|}
  \hline
  \multicolumn{4}{|c|}{{\bf Закони на Де Морган за квантори}}\\
  \hline
  Твърдение & Еквивалентно твърдение & Кога е истина? & Кога е неистина?\\
  \hline
  $\neg \exists x P(x)$ & $\forall x \neg P(x)$ & за всяко $x$ $P(x)$ {\bf не} е вярно & съществува $x$, за което $P(x)$ е вярно \\
  \hline
  $\neg \forall x P(x)$ & $\exists x \neg P(x)$ & съществува $x$, за което $P(x)$ {\bf не} е вярно & $P(x)$ е вярно за всяко $x$\\
  \hline
\end{tabular}  
\bigskip

\begin{problem}
  Да означим с $K(x,y)$ твърдението ``$x$ познава $y$''.
  Изразете като предикатна формула следните твърдения.
  \begin{enumerate}[1)]
  \item
    \marginpar{$\forall x \exists y K(x,y)$}
    Всеки познава някого.
  \item
    \marginpar{$\exists x \forall y K(x,y)$}
    Някой познава всеки.
  \item
    \marginpar{$\exists x\forall y K(y,x)$}
    Някой е познаван от всички.
  \item
    \marginpar{$\forall x \exists y(K(x,y)\wedge \neg K(y,x)) $}
    Всеки знае някой, който не го познава.
  \item
    \marginpar{$\exists x \forall y(K(y,x)\ \rightarrow K(x,y))$}
    Има такъв, който знае всеки, който го познава.
  \item
    \marginpar{$(\forall x,y)(K(x,y)\ \&\ K(y,x) \to \exists z(K(x,z)\ \&\ K(y,z))$}
    Всеки двама познати имат общ познат.
  \end{enumerate}
\end{problem}

\begin{example}
  Нека $D \subseteq \Real$.
  Казваме, че $f:D \to \Real$ е {\em непрекъсната} в точката $x_0 \in D$, ако 
  \[(\forall \varepsilon > 0)(\exists \delta >0)(\forall x\in D)(\ |x_0 - x| < \delta\ \to\ |f(x_0) - f(x)| < \varepsilon\ ).\]
  Да видим какво означава $f$ да бъде {\em прекъсната} в точката $x_0 \in D$:
  \marginpar{$f$ е прекъсната в $x_0$ точно тогава, когато $f$ не е непрекъсната в $x_0$}
  \begin{align*}
    & \neg (\forall \varepsilon > 0)(\exists \delta >0)(\forall x\in D)(\ |x_0 - x| < \delta\ \to\ |f(x_0) - f(x)| < \varepsilon\ ) \equiv \\
    & (\exists \varepsilon > 0) \neg (\exists \delta >0)(\forall x\in D)(\ |x_0 - x| < \delta\ \to\ |f(x_0) - f(x)| < \varepsilon\ ) \equiv \\
    & (\exists \varepsilon > 0)(\forall \delta >0)\neg(\forall x\in D)(\ |x_0 - x| < \delta\ \to\ |f(x_0) - f(x)| < \varepsilon\ ) \equiv \\
    & (\exists \varepsilon > 0)(\forall \delta >0)(\exists x\in D)\neg(\ |x_0 - x| < \delta\ \to\ |f(x_0) - f(x)| < \varepsilon\ ) \equiv \\
    & (\exists \varepsilon > 0)(\forall \delta >0)(\exists x\in D)\neg(\ \neg (|x_0 - x| <\delta) \vee |f(x_0) - f(x)| < \varepsilon\ ) \equiv \\
    & (\exists \varepsilon > 0)(\forall \delta >0)(\exists x\in D)(\ \neg\neg (|x_0 - x| <\delta) \land \neg (|f(x_0) - f(x)| < \varepsilon)\ ) \equiv \\
    & (\exists \varepsilon > 0)(\forall \delta >0)(\exists x\in D)(\ |x_0 - x| < \delta\ \land\ |f(x_0) - f(x)| \geq \varepsilon\ ).
  \end{align*}
\end{example}


%%% Local Variables:
%%% mode: latex
%%% TeX-master: "../eai"
%%% End:


\section{Множества, релации, функции}\label{sect:intro:sets}
\index{множества}

\subsection*{Основни отношения между множества}

За произволни множества $A$ и $B$, ще казваме, че:
\begin{itemize}
\item
  $A$ е подмножество на $B$, което ще означаваме като $A \subseteq B$, ако:
  \[(\forall x)[x \in A \implies x \in B].\]
\item
  $A$ е равно на $B$, което ще означаваме като $A = B$, ако:
  \[(\forall x)[x \in A \iff x \in B],\]
  или
  \[A = B\ \iff\ A \subseteq B\ \&\ B \subseteq A.\]
\end{itemize}

\subsection*{Основни операции върху множества}

Ще разгледаме няколко операции върху произволни множества $A$ и $B$.
\begin{itemize}
\item
  \index{множества!сечение}
  \mynote{На англ. \emph{intersection}}
  {\bf Сечение}
  \[A\cap B = \{x\ \mid\ x\in A\ \wedge\ x\in B\}.\]
  Казано по-формално, $A\cap B$ е множеството, за което е изпълнено, че:
  \[(\forall x)[x \in A\cap B \iff (x\in A\ \land\ x \in B)].\]
  Примери:
  \begin{itemize}
  \item
    $A \cap A = A$, за всяко множество $A$.
  \item
    $A \cap \emptyset = \emptyset$, за всяко множество $A$.
  \item
    \mynote{Макар и $\emptyset$, $\{\emptyset\}$ и $\{1,2\}$ да са множества, те може да са елементи на други множества.}
    $\{1,\emptyset,\{\emptyset\}\} \cap \{\emptyset\} = \{\emptyset\}$.
  \item
    $\{1,2,\{1,2\}\} \cap \{1,\{1\}\} = \{1\}$.
  \end{itemize}
\item
  \index{множества!обединение}
  \mynote{На англ. \emph{union}}
  {\bf Обединение}
  \[A\cup B = \{x\ \mid x\in A\ \vee\ x\in B\}.\]
  $A\cup B$ е множеството, за което е изпълнено, че:
  \[(\forall x)[x \in A\cup B \iff (x\in A\ \lor\ x \in B)].\]
  Примери:
  \begin{itemize}
  \item
    $A \cup A = A$, за всяко множество $A$.
  \item 
    $A \cup \emptyset = A$, за всяко множество $A$.
  \item
    $\{1,2,\emptyset\} \cup \{1,2,\{\emptyset\}\} = \{1,2,\emptyset,\{\emptyset\}\}$.
  \item
    $\{1,2,\{1,2\}\} \cup \{1,\{1\}\} = \{1,2,\{1\},\{1,2\}\}$.
  \end{itemize}
\item
  \index{множества!разлика}
  {\bf Разлика}
  \[A\setminus B = \{x\ \mid\ x\in A\ \wedge\ x\not\in B\}.\]
  $A\setminus B$ е множеството, за което е изпълнено, че:
  \[(\forall x)[x \in A\setminus B \iff (x\in A\ \wedge\ x \not\in B)].\]
  Примери:
  \begin{itemize}
  \item
    $A \setminus A = \emptyset$, за всяко множество $A$.
  \item 
    $A \setminus \emptyset = A$, за всяко множество $A$.
  \item 
    $\emptyset \setminus A = \emptyset$, за всяко множество $A$.
  \item
    $\{1,2,\emptyset\} \setminus \{1,2,\{\emptyset\}\} = \{\emptyset\}$.
  \item
    $\{1,2,\{1,2\}\} \setminus \{1,\{1\}\} = \{2,\{1,2\}\}$.
  \end{itemize}
\item
  \index{множества!степенно множество}
  {\bf Степенно множество}
  \[\Ps(A) = \{x\mid x\subseteq A\}.\]
  \mynote{На англ. \emph{power set} }
  $\Ps(A)$ е множеството, за което е изпълнено, че:
  \[(\forall x)[x \in \Ps(A) \iff (\forall y)[y\in x\rightarrow y \in A]].\]
  \mynote{В литературата се среща също така и означението $2^A$ за степенното множество на $A$.}
  Примери:
  \begin{itemize}
  \item 
    $\Ps(\emptyset) = \{\emptyset\}$.
  \item
    $\Ps(\{\emptyset\}) = \{\emptyset,\{\emptyset\}\}$.
  \item
    $\Ps(\{\emptyset,\{\emptyset\}\}) = \{\emptyset,\{\emptyset\},\{\{\emptyset\}\},\{\emptyset,\{\emptyset\}\}\}$.
  \item
    $\Ps(\{1,2\}) = \{\emptyset,\{1\},\{2\},\{1,2\}\}$.
  \end{itemize}
\end{itemize}

\begin{problem}
  Проверете верни ли са свойствата:
  \begin{enumerate}[a)]
  \item
    $A\subseteq B \iff A\setminus B = \emptyset \iff A\cup B = B \iff A\cap B = A$;
  \item
    $A\setminus \emptyset = A$, $\emptyset\setminus A=\emptyset$, $A\setminus B = B\setminus A$.
  \item
    $A\cap (B\cup A) = A \cap B$;
  \item
    $A\cup(B\cap C) = (A\cup B)\cap(A\cup C)$ и $A \cap (B \cup C) = (A \cup B) \cap (A \cup C)$;
  % \item
  %   $C\subseteq A\ \&\ C\subseteq B \rightarrow C\subseteq A\cap B$;
  % \item
  %   $A\subseteq C\ \&\ B\subseteq C \rightarrow A\cup B\subseteq C$;
  \item
    $A\backslash B = A \iff A\cap B = \emptyset$;
  \item
    $A\backslash B = A\backslash (A\cap B)$ и $A\backslash B = A\backslash (A\cup B)$;
  \item
    $(A\cup B)\setminus C = (A\setminus C) \cup (B\setminus C)$;
  % \item
  %   \mynote{Не е вярно!}
  %   $A\setminus (B\setminus C) = (A\setminus B)\setminus C$;
  \item
    \index{Де Морган}
    \mynote{Закони на Де Морган}
    $C\setminus (A\cup B) = (C\backslash A)\cap(C\backslash B)$ и $C \backslash (A\cap B) = (C\backslash A)\cup(C\backslash B)$
  % \item
  %   $C\backslash(\bigcup^{n}_{i=1} A_i) = \bigcap^{n}_{i=1} (C\backslash A_i)$ и $C \backslash(\bigcap^{n}_{i=1} A_i) = \bigcup^{n}_{i=1} (C\backslash A_i)$;
  \item
    $(A\backslash B)\backslash C = (A\backslash C)\backslash(B \backslash C)$ и $A\backslash (B\backslash C) = (A\backslash B) \cup (A\cap C)$;
  \item
    $A\subseteq B \Rightarrow \Ps(A) \subseteq \Ps(B)$;
  \item
    \mynote{$X \subseteq A\cup B \stackrel{?}{\Rightarrow} X\subseteq A \vee X \subseteq B$}
    $\Ps(A\cap B) = \Ps(A) \cap \Ps(B)$ и $\Ps(A\cup B) = \Ps(A) \cup \Ps(B)$;
  \end{enumerate}
\end{problem}

За да дадем определение на понятието релация, трябва първо 
да въведем понятието декартово произведение на множества,
което пък от своя страна се основава на понятието наредена двойка.

\subsection*{Наредена двойка}
\index{наредена двойка}
За два елемента $a$ и $b$ въвеждаме опрецията {\bf наредена двойка} $\pair{a,b}$.
Наредената двойка $\pair{a,b}$ има следното характеристичното свойство:
\[a_1 = a_2\ \wedge\ b_1 = b_2\ \iff\ \pair{a_1,b_1} = \pair{a_2,b_2}.\]
Понятието наредена двойка може да се дефинира по много начини, стига да изпълнява харектеристичното свойство.
Ето примери как това може да стане:
\begin{enumerate}[1)]
\item
  \mynote{Norbert Wiener (1914)}
  Първото теоретико-множествено определение на понятието наредена двойка е
  дадено от Норберт Винер:
  \index{Винер}
  \[\pair{a,b} \df \{\{\{a\},\emptyset\},\{\{b\}\}\}.\]
\item
  \mynote{Kazimierz Kuratowski (1921)}
  \index{Куратовски}
  Определението на Куратовски се приема за ,,стандартно'' в наши дни:
  \[\pair{a,b} \df \{\{a\},\{a,b\}\}.\]
\end{enumerate}

\begin{problem}
  Докажете, че горните дефиниции наистина изпълняват харектеристичното свойство за наредени двойки.
\end{problem}

\begin{definition}
  \mynote{Пример за индуктивна (рекурсивна) дефиниция}
  Сега можем, за всяко естествено число $n \geq 1$,
  да въведем понятието наредена $n$-орка $\pair{a_1,\dots,a_n}$:
  \begin{align*}
    & \pair{a_1} \df a_1,\\
    & \pair{a_1,a_2,\dots,a_n} \df \pair{a_1,\pair{a_2,\dots,a_n}}.
  \end{align*}
\end{definition}

Оттук нататък ще считаме, че имаме дадено понятието наредена $n$-орка, без да се интересуваме от нейната формална дефиниция.
 
\subsection*{Декартово произведение}
\mynote{На англ. cartesian product. Считаме, че $(A\times B)\times C = A\times (B\times C) = A\times B \times C$.}
\index{декартово произведение}

За две множества $A$ и $B$, определяме тяхното декартово произведение като
\[A\times B = \{\pair{a,b}\mid a\in A\ \&\ b\in B\}.\]
За краен брой множества $A_1,A_2,\dots,A_n$, определяме
\[A_1\times A_2\times\cdots\times A_n = \{\pair{a_1,a_2,\dots,a_n}\mid a_1 \in A_1\ \&\ \dots\ \&\ a_n \in A_n\}.\]

\begin{problem}
  Проверете, че:
  \begin{enumerate}[a)]
  \item
    $A\times(B\cup C) = (A\times B) \cup (A\times C)$.
  \item
    $(A\cup B)\times C = (A\times C)\cup (B\times C)$.
  \item 
    $A\times(B\cap C) = (A\times B) \cap (A\times C)$.
  \item
    $(A \cap B)\times C = (A \times C)\cap(B\times C)$.
  \item 
    $A\times(B\setminus C) = (A\times B) \setminus (A\times C)$.
  \item
    $(A\setminus B)\times C = (A\times C)\setminus (B\times C)$.
  \end{enumerate}
\end{problem}


\subsection*{Видове функции}

Функцията $f:A \to B$ е:
\begin{itemize}
\item
  \mynote{\comment{или $f$ е {\bf обратима}}}
  {\bf инекция}\index{функция!инекция}, ако е изпълнено свойството:
  \[(\forall a_1,a_2\in A)[\ a_1\neq a_2\ \to\ f(a_1)\neq f(a_2)\ ],\]
  или еквивалентно,
  \[(\forall a_1,a_2\in A)[\ f(a_1) = f(a_2)\ \to\ a_1 = a_2\ ].\]
\item
  \mynote{\comment{или $f$ е {\bf върху} $B$ }}
  {\bf сюрекция}\index{функция!сюрекция}, ако е изпълнено свойството:
  \[(\forall b\in B)(\exists a\in A)[\ f(a) = b\ ].\]
\item
  {\bf биекция}\index{функция!биекция}, ако е инекция и сюрекция.
\end{itemize}

\begin{problem}
  \index{Кантор}
  \mynote{Канторово кодиране. Най-добре се вижда като се нарисува таблица}
  Докажете, че $f: \Nat \times \Nat\rightarrow \Nat$ е биекция, където
  \[f(x, y) = \frac{(x+y)(x+y+1)}{2} + x.\]
\end{problem}

%%% Local Variables:
%%% mode: latex
%%% TeX-master: "../eai"
%%% End:


\section{Доказателства на твърдения}

\subsection*{Допускане на противното}

Да приемем, че искаме да докажем, че свойството $P(x)$
е вярно за всяко естествено число.
Един начин да направим това е следният:
\begin{itemize}
\item 
  Допускаме, че съществува елемент $n$, за който $\neg P(n)$.
\item
  Използвайки, че $\neg P(n)$ правим извод, от който следва факт, за който знаем, че винаги е лъжа.
  Това означава, че доказваме следното твърдение
  \[\exists x \neg P(x) \rightarrow \mathbf{0}.\]
\item
  Тогава можем да заключим, че $\forall x P(x)$, защото имаме следния извод:
  \begin{prooftree}
    \AxiomC{$\exists x \neg P(x) \rightarrow \mathbf{0}$}
    \UnaryInfC{$\mathbf{1} \rightarrow \neg \exists x \neg P(x)$}
    \UnaryInfC{$\neg \exists x \neg P(x)$}
    \UnaryInfC{$\forall x P(x)$}
  \end{prooftree}
\end{itemize}

Ще илюстрираме този метод като решим няколко прости задачи.

\begin{problem}
  \label{prob:even-number-square}
  За всяко $a \in \Int$, ако $a^2$ е четно, то $a$ е четно.
\end{problem}
\begin{proof}
  Ние искаме да докажем твърдението $P$, където:
  \[P \equiv (\forall a\in\Int)[a^2\mbox{ е четно}\ \rightarrow\ a\mbox{ е четно}].\]
  \mynote{$\neg (\forall x)(A(x) \rightarrow B(x))$ е еквивалентно на $(\exists x)(A(x) \wedge \neg B(x))$}
  Да допуснем противното, т.е. изпълнено е $\neg P$. Лесно се вижда, че
  \[\neg P \iff (\exists a\in\Int)[a^2\mbox{ е четно}\ \land\ a\mbox{ не е четно}].\]
  Да вземем едно такова нечетно $a$, за което $a^2$ е четно.
  Това означава, че $a = 2k+1$, за някое $k \in \Int$,
  и \[a^2 = (2k+1)^2 = 4k^2 + 4k + 1,\]
  което очевидно е нечетно число.
  Но ние допуснахме, че $a^2$ е четно.
  Така достигаме до противоречие, следователно нашето допускане е грешно 
  и 
  \[(\forall a\in\Int)[a^2\mbox{ е четно}\ \rightarrow\ a\mbox{ е четно}].\]
\end{proof}

\begin{problem}
  Докажете, $\sqrt{2}$ {\bf не} е рационално число.
\end{problem}
\begin{proof}
  Да допуснем, че $\sqrt{2}$ е рационално число. Тогава  съществуват $a,b \in \Int$, такива че
  \[\sqrt{2} = \frac{a}{b}.\]
  Без ограничение, можем да приемем, че $a$ и $b$ са естествени числа,
  които нямат общи делители, т.е. не можем да съкратим дробта $\frac{a}{b}$.
  Получаваме, че \[2b^2 = a^2.\]
  Тогава $a^2$ е четно число и от Задача \ref{prob:even-number-square}, $a$ е четно число.
  Нека $a = 2k$, за някое естествено число $k$. Получаваме, че
  \[2b^2 = 4k^2,\]
  от което следва, че
  \[b^2 = 2k^2.\]
  Това означава, че $b$ също е четно число, $b = 2n$, за някое естествено число $n$.
  Следователно, $a$ и $b$ са четни числа и имат общ делител $2$,
  което е противоречие с нашето допускане, че $a$ и $b$ нямат общи делители.
  Така достигаме до противоречие.
  Накрая заключаваме, че $\sqrt{2}$ не е рационално число.
\end{proof}


\subsection*{Индукция върху естествените числа}
\index{индукция}

\mynote{Да напомним, че естествените числа са $\Nat = \{0,1,2,\dots\}$}
Доказателството с индукция по $\Nat$ представлява следната схема:
\begin{prooftree}
  \AxiomC{$P(0)$}
  \AxiomC{$(\forall x\in\Nat)[P(x)\rightarrow P(x+1)]$}
  \BinaryInfC{$(\forall x\in\Nat) P(x)$}
\end{prooftree}

Това означава, че ако искаме да докажем, че свойството $P(x)$ е вярно за всяко естествено число $x$,
то трябва да докажем първо, че е изпълнено $P(0)$ и след това, за произволно естествено число $x$, ако $P(x)$ вярно, то също така е вярно $P(x+1)$.

\begin{problem}
  \label{prob:number-prod-prime}  
  Всяко естествено число $n \geq 2$ може да се запише като произведение на прости числа.
\end{problem}
\begin{proof}
  Искаме да докажем, че $(\forall n \geq 2)P(n)$, където $P(n)$ казва, че $n$ може да се запише като произведение на прости числа, т.е.
  \[n = p^{m_1}_1p^{m_2}_2\cdots p^{m_k}_k,\]
  за някои прости числа $p_1,p_2,\dots,p_k$ и естествени числа $m_1,m_2,\dots,m_k$.
  
  Доказателството протича с индукция по $n \geq 2$.
  \begin{enumerate}[a)]
  \item 
    За $n = 2$ е ясно, защото $2$ е просто число. В този случай $n = p^{m_1}_1$ и $p_1 = 2$ и $m_1 = 1$.
  \item
    Да приемем, че $P(n)$ е изпълнено за някое естествено число $n > 2$.
  \item
    Да разгледаме следващото естествено число $n+1$.
    Ако $n+1$ е просто число, то всичко е ясно.
    Ако $n+1$ е съставно, то съществуват естествени числа $n_1,n_2 \geq 2$, за които
    \[n + 1 = n_1\cdot n_2.\]
    Тогава, понеже $n_1,n_2 \leq n$, от от И.П. следва, че $P(n_1)$ и $P(n_2)$, т.е.
    \[n_1 = p^{\ell_1}_1\cdots p^{\ell_k}_k\text{ и }n_2 = q^{m_1}_1\cdots q^{m_r}_r,\]
    където $p_1,\dots,p_k$ и $q_1,\dots,q_r$ са прости числа, а $\ell_1,\dots,\ell_k$ и $m_1,\dots,m_r$ са естествени числа.
    Тогава е ясно, че $n+1$ също е произведение на прости числа.
  \end{enumerate}
\end{proof}

\begin{problem}
  Докажете, че за всяко естествено число $n$, 
  \[\sum^n_{i=0} 2^i = 2^{n+1} - 1.\]
\end{problem}
\begin{proof}
  Да разгледаме свойството
  \[P(n) \df \sum^n_{i=0} 2^i = 2^{n+1} - 1.\]
  Ще докажем с индукция по $n$, че $(\forall n)P(n)$, т.е. ще докажем следния извод:
  \begin{prooftree}
    \AxiomC{$P(0)$}
    \AxiomC{$(\forall n)[P(n) \implies P(n+1)]$}
    \BinaryInfC{$(\forall n)P(n)$}
  \end{prooftree}
  \begin{itemize}
  \item
    \mynote{Това е базата на индукцията.}
    Нека първо $n = 0$. Oчевидно е, че $P(0)$ е изпълнено, защото
    \[\sum^0_{i=0}2^i = 1 = 2^{1} - 1.\]
  \item
    \mynote{$P(n)$ се нарича индукционно предположение, а $P(n+1)$ се нарича индукционна стъпка.}
    Да разгледаме сега произволно естествено число $n$, като
    приемем, че свойството $P(n)$ е изпълнено.
    Ще докажем, че $P(n+1)$ също е изпълнено.
    Но това е лесно защото имаме следната верига от равенства:
    \begin{align*}
      \sum^{n+1}_{i=0} 2^i & = \sum^{n}_{i=0}2^i + 2^{n+1}\\
                           & = 2^{n+1} - 1 + 2^{n+1} & \comment\text{защото $P(n)$ е изпълнено}\\
                           & = 2.2^{n+1} - 1 \\
                           & = 2^{1+(n+1)} - 1\\
                           & = 2^{n+2} - 1.
    \end{align*}
  \end{itemize}
\end{proof}

%\subsection*{Пълна индукция върху естествените числа}


%%% Local Variables:
%%% mode: latex
%%% TeX-master: "../eai"
%%% End:


\section{Азбуки, думи, езици}

\subsection*{Основни понятия}

\begin{itemize}
\item 
  \index{азбука}
  {\bf Азбука} ще наричаме всяко крайно множество,
  като обикновено ще я означаваме със $\Sigma$.
  \marginpar{Често ще използваме буквите $a$, $b$, $c$ за да означаваме букви.}
  Елементите на азбуката $\Sigma$ ще наричаме {\bf букви} или символи.
\item
  \index{дума}
  {\bf Дума} над азбуката $\Sigma$ е произволна крайна редица от елементи на $\Sigma$.
  Например, за $\Sigma = \{a,b\}$, $aababba$ е дума над $\Sigma$ с дължина $7$.
  С $\abs{\alpha}$ ще означаваме дължината на думата $\alpha$.
  \marginpar{Обикновено ще означаваме думите с $\alpha$, $\beta$, $\gamma$, $\omega$.}
\item
  Обърнете внимание, че имаме единствена дума с дължина $0$.
  Тази дума ще означаваме с $\varepsilon$ и ще я наричаме {\bf празната дума},
  т.е. $\abs{\varepsilon} = 0$.
\item
  С $a^n$ ще означаваме думата съставена от $n$ $a$-та.
  Формалната индуктивна дефиниция е следната:
  \begin{align*}
    a^0 & \df \varepsilon,\\
    a^{n+1} & \df a^na.
  \end{align*}
\item
  Множеството от всички думи над азбуката $\Sigma$ ще означаваме със $\Sigma^\star$.
  Например, за $\Sigma = \{a,b\}$,
  \[\Sigma^\star = \{\varepsilon,a,b,aa,ab,ba,bb,aaa,aab,\dots\}.\]
  Обърнете внимание, че $\emptyset^\star = \{\varepsilon\}$.
\item
  {\bf Език} над азбуката $\Sigma$ ще наричаме всяко подмножество на $\Sigma^\star$.
  Например, за $\Sigma = \{a, b\}$,
  \[L = \{\alpha \in \{a, b\}^\star \mid \alpha\mbox{ започва с }a\}\]
  е език над $\Sigma$.
% \item
%   {\bf Лексикографска наредба}
\end{itemize}

\subsection*{Операции върху думи}

\begin{itemize}
\item 
  \index{конкатенация}
  Операцията {\bf конкатенация} взима две думи $\alpha$ и $\beta$ и образува 
  новата дума $\alpha\cdot\beta$ като слепва двете думи.
  Например $aba\cdot bb = ababb$.
  Обърнете внимание, че в общия 
  случай $\alpha\cdot\beta \neq \beta\cdot\alpha$. 
  \marginpar{Често ще пишем $\alpha\beta$ вместо $\alpha\cdot\beta$}
  Можем да дадем формална индуктивна дефиниция на операцията конкатенация по
  дължината на думата $\beta$.
  \begin{itemize}
  \item 
    Ако $\abs{\beta} = 0$, то $\beta = \varepsilon$.
    Тогава $\alpha\cdot \varepsilon \df \alpha$.
  \item
    Ако $\abs{\beta} = n+1$, то $\beta = \gamma b$, $\abs{\gamma} = n$.
    Тогава $\alpha\cdot\beta \df (\alpha\cdot\gamma)b$.
  \end{itemize}
\item
  Друга често срещана операция върху думи е {\bf обръщането} на дума.
  Дефинираме думата $\alpha^R$ като обръщането на $\alpha$ по следния начин.
  \begin{itemize}
  \item 
    Ако $\abs{\alpha} = 0$, то $\alpha = \varepsilon$ и $\alpha^R \df \varepsilon$.
  \item
    Ако $\abs{\alpha} = n+1$, то $\alpha = a\beta$, където $\abs{\beta} = n$.
    Тогава $\alpha^R \df (\beta^R)a$.
  \end{itemize}
  Например, $reverse^R = esrever$.
\item
  \index{дума!префикс}
  \index{дума!суфикс}
  Казваме, че думата $\alpha$ е {\bf префикс} на думата $\beta$,
  ако съществува дума $\gamma$, такава че $\beta = \alpha\cdot\gamma$.
  $\alpha$ е {\bf суфикс} на $\beta$, ако $\beta = \gamma\cdot\alpha$, за някоя дума $\gamma$.
\item
  \marginpar{Обърнете внимание, че $\emptyset\cdot A = A\cdot\emptyset = \emptyset$}
  \marginpar{Също така, $\{\varepsilon\}\cdot A = A\cdot\{\varepsilon\} = A$}
  Нека $A$ и $B$ са множества от думи.
  Дефинираме конкатенацията на $A$ и $B$ като
  \[A\cdot B \df \{\alpha\cdot\beta \mid \alpha\in A\ \&\ \beta \in B\}.\]
\item
  Сега за едно множество от думи $A$, дефинираме $A^n$ индуктивно:
  \begin{align*}
    A^0 & \df \{\varepsilon\},\\
    A^{n+1} & \df A^n \cdot A.
  \end{align*}
  Ако $A = \{ab, ba\}$, то
  $A^0 = \{\varepsilon\}$, $A^1 = A$, $A^2 = \{abab, abba, baba, baab\}$.
  Ако $A = \{a,b\}$, то $A^n = \{\alpha \in \{a,b\}^\star \mid \abs{\alpha} = n\}$.
\item
  За едно множеството от думи $A$, дефинираме:
  \marginpar{Операцията $\star$ е известна като звезда на Клини}
  \begin{align*}
    A^\star & \df \bigcup_{n\geq 0} A^n\\
    & = A^0 \cup A^1 \cup A^2 \cup A^3 \cup \dots\\
    A^+ & \df A\cdot A^\star.
  \end{align*}
\end{itemize}

\begin{problem}
  Проверете:
  \begin{enumerate}[a)]
  \item 
    операцията конкатенация е {\em асоциативна}, т.е. за всеки три думи $\alpha$, $\beta$, $\gamma$,
    \[(\alpha\cdot\beta)\cdot\gamma = \alpha\cdot(\beta\cdot\gamma);\]
  \item
    за множествата от думи $A$, $B$ и $C$,
    \[(A\cdot B)\cdot C = A\cdot (B\cdot C);\]
  \item
    $\{\varepsilon\}^\star = \{\varepsilon\}$;
  \item
    за произволно множество от думи $A$,
    $A^\star = A^\star\cdot A^\star$ и $(A^\star)^\star = A^\star$;
  \item
    за произволни букви $a$ и $b$,
    $\{a,b\}^\star = \{a\}^\star\cdot(\{b\}\cdot\{a\}^\star)^\star$.
  \end{enumerate}
\end{problem}


\begin{problem}
  Докажете, че за всеки две думи $\alpha$ и $\beta$ е изпълено:
  \begin{enumerate}[a)]
  \item 
    $(\alpha\cdot\beta)^R = \beta^R\cdot\alpha^R$;
  \item
    $\alpha$ е префикс на $\beta$ точно тогава, когато $\alpha^R$ е суфикс на $\beta^R$;
  \item
    $(\alpha^R)^R = \alpha$;
  \item
    $(\alpha^n)^R = (\alpha^R)^n$, за всяко $n \geq 0$.
  \end{enumerate}
\end{problem}

\begin{problem}
  \marginpar{С други думи, ако $\varepsilon \not\in X$, то $Z = X^\star Y$ е най-малкото решение на уравнението $Z = XZ \cup Y$.}
  Нека $X, Y, Z \subseteq \Sigma^\star$ със свойството, че $Z = XZ \cup Y$.
  \begin{enumerate}[a)]
  \item 
    Докажете, че за всяко $n \in \Nat$, $X^nY \subseteq Z$.
    Заключете, че $X^\star Y \subseteq Z$.
  \item
    Да предположим, че $\varepsilon \not\in X$.
    Докажете, че за всяка дума $\omega \in Z$ е изпълнено, че $\omega \in X^\star Y$.
  \end{enumerate}
\end{problem}


%%% Local Variables:
%%% mode: latex
%%% TeX-master: "../eai"
%%% End:


\section*{Бележки}

Повечето книги в тази област започват с уводна глава, в която въвеждат понятията множества, релации и езици.
\begin{itemize}
\item 
  Глава 1 от \cite{rosen}.
\item
  Глава 1 от \cite{papadimitriou}.
\item
  За описанието на думи и азбуки следваме \cite[Глава 2]{kozen}.
\end{itemize}



%%% Local Variables:
%%% mode: latex
%%% TeX-master: "../eai"
%%% End:


% Add Problem section

%%% Local Variables:
%%% mode: latex
%%% TeX-master: "../eai"
%%% End:


\chapter{Регулярни езици и автомати}\label{ch:regular}

\section{Автоматни езици}

\begin{definition}
  Детерниниран краен автомат е петорка $\A = \FA$, където
  \begin{itemize}
  \item
    $\Sigma$ е азбука;
  \item
    $Q$ е крайно множество от състояния;
  \item
    $\delta:Q\times\Sigma\to Q$ е тотална функция, която ще наричаме
    \emph{функция на преходите};
  \item
    $\qstart\in Q$ е начално състояние;
  \item
    $F\subseteq Q$ е множеството от финални състояния
  \end{itemize}
\end{definition}
\index{автомат!детерминиран}

Нека имаме една дума $\alpha \in \Sigma^\star$, $\alpha = a_0a_1\cdots a_{n-1}$.
Казваме, че $\alpha$ се {\bf разпознава} от автомата $\A$, ако
съществува редица от състояния $q_0,q_1,q_2,\dots,q_n$, такива че:
\begin{itemize}
\item
  $q_0 = \qstart$, началното състояние на автомата;
\item
  $\delta(q_i,a_{i}) = q_{i+1}$, за всяко $i = 0, \dots, n-1$;
\item
  $q_n \in F$.
\end{itemize}

Казваме, че $\A$ {\bf разпознава} езика $L$, ако $\A$ разпознава точно думите от $L$, т.е.
$L = \{\alpha \in \Sigma^\star \mid \A\mbox{ разпознава }\alpha\}$.
Обикновено означаваме езика, който се разпознава от даден автомат $\A$ с $\L(\A)$.
\index{език!автоматен}
В такъв случай ще казваме, че езикът $L$ е {\bf автоматен}.

При дадена функция на преходите $\delta$,
често е удобно да разглеждаме функция $\delta^\star:Q\times\Sigma^\star \to Q$, която е дефинирана по следния начин:
% \marginpar{Това е пример за индуктивна (рекурсивна) дефиниция по дължината на думата $\alpha$}
\begin{itemize}
\item 
  $\delta^\star(q,\varepsilon) = q$, за всяко $q\in Q$;
\item
  $\delta^\star(q,\beta a) = \delta(\delta^\star(q,\beta), a)$, за всяко $q\in Q$, всяко $a\in\Sigma$ и $\beta\in\Sigma^\star$.
\end{itemize}
Лесно се съобразява, че една дума $\alpha$ се {\em разпознава} от автомата $\A$ точно тогава, когато $\delta^\star(\qstart,\alpha) \in F$.
Оттук следва, че
\begin{framed}
\[\L(\A) \df \{\ \alpha\in\Sigma^\star \mid \delta^\star(\qstart,\alpha) \in F\ \}.\]
\end{framed}

\marginpar{Обърнете внимание, че $\delta(q,a) = \delta^\star(q,a)$ за $a\in\Sigma$}

\begin{proposition}
  \label{pr:delta-star}
  Нека $\A$ е ДКА. Тогава за всяко състояние $q$ и произволни думи $\alpha$ и $\beta$ е изпълнено, че
  % $(\forall q\in Q)(\forall\alpha,\beta\in\Sigma^\star)[\delta^\star(q,\alpha\beta) = \delta^\star(\delta^\star(q,\alpha),\beta)]$.
  \[\delta^\star(q,\alpha\beta) = \delta^\star(\delta^\star(q,\alpha),\beta).\]
\end{proposition}
\begin{hint}
  Индукция по дължината на $\beta$.

  \begin{itemize}
  \item
    $|\beta| = 0$, т.е. $\beta = \varepsilon$. Тогава за всяко състояние $q$ и произволна дума $\alpha$ имаме:
    \[\delta^\star(q, \alpha\varepsilon) = \delta^\star( \underbrace{\delta^\star(q, \alpha)}_{q}, \varepsilon).\]
  \item
    Да приемем, че твърдението е изпълнено за думи $\beta$ с дължина $n$.
  \item
    Нека $|\beta| = n+1$, т.е. $\beta = \gamma b$, където $|\gamma| = n$. Тогава за всяко състояние $q$ и произволна дума $\alpha$ имаме:
    \begin{align*}
      \delta^\star(q, \alpha\beta) & = \delta^\star(q, \alpha \gamma b)\\
                                   & = \delta( \delta^\star(q, \alpha\gamma),b) & \comment\text{от деф. на }\delta^\star\\
                                   & = \delta(\delta^\star(\underbrace{\delta^\star(q,\alpha)}_{p}, \gamma), b) & \comment{\text{от И.П. приложено за }\gamma}\\
                                   & = \delta( \delta^\star(p, \gamma), b) \\
                                   & = \delta^\star( p, \gamma b) & \comment\text{от деф. на }\delta^\star\\
                                   & = \delta^\star(\delta^\star(q, \alpha), \beta). & \comment{p = \delta^\star(q,\alpha)}
    \end{align*}
  \end{itemize}
\end{hint}

\index{моментно описание}
\marginpar{(На англ. {\em instantaneous description}). В случая въвеждането на това понятие не е напълно необходимо, но по-късно, когато въведем стекови автомати и машини на Тюринг, ще работим с такива моментни описания на изчисления и затова е добре още отначало да свикнем с него.}
{\bf Моментното описание} на изчисление с краен автомат представлява двойка от вида $(q,\alpha) \in Q\times\Sigma^\star$,
т.е. автоматът се намира в състояние $q$, а думата, която остава да се прочете е $\alpha$.
Удобно е да въведем бинарната релация $\vdash_\A$ над $Q\times\Sigma^\star$,
която ще ни казва как моментното описание на автомата $\A$ се променя след изпълнение на една стъпка:
\[(q,x\alpha) \vdash_\A (p,\alpha), \text{ ако } \delta(q,x) = p.\]
\marginpar{Рефл. и транз. затваряне на една релация е разгледано в Глава \ref{ch:intro}}
Рефлексивното и транзитивно затваряне на $\vdash_\A$ ще означаваме с $\vdash^\star_\A$.
За да дадем точна дефиниция на $\vdash^\star_\A$, първо ще дефинираме релацията $\vdash^n_\A$, която
определя работата на автомата $\A$ за $n$ стъпки.

\begin{prooftree}
  \AxiomC{}
  \UnaryInfC{$(q,\alpha) \vdash^0_\A (q,\alpha)$}
\end{prooftree}

\begin{prooftree}
  \AxiomC{$\delta(q,x) = q'$}
  \AxiomC{$(q',\alpha) \vdash^n_\A (p, \beta)$}
  \BinaryInfC{$(q,x\alpha) \vdash^{n+1}_\A (p,\beta)$}
\end{prooftree}


\begin{itemize}
\item 
  $(q,\alpha) \vdash^0_\A (q,\alpha)$, защото за $0$ стъпки се случва нищо.
\item
  Нека $\delta(q,x) = q'$ и $(q',\alpha) \vdash^n_\A (p, \beta)$. Тогава
  $(q,x\alpha) \vdash^{n+1}_\A (p,\beta)$, защото за $n+1$ стъпки първо правим една стъпка 
  и отиваме в моментното описание $(q',\alpha)$ и след това правим още $n$ стъпки.
\end{itemize}
Сега можем да дефинираме $\vdash^\star_\A$ като:
\[(q,\alpha) \vdash^\star_\A (p,\beta) \dff (\exists n\in\Nat)[(q,\alpha) \vdash^n_\A (p,\beta)].\]

\begin{problem}
  Докажете, че 
  \[(q,\alpha\beta) \vdash^\star_\A (p, \beta) \iff \delta^\star(q,\alpha) = p.\]  
\end{problem}

Получаваме, че 
\[\L(\A) = \{\alpha\in\Sigma^\star \mid (\exists q \in F)[(\qstart,\alpha) \vdash^\star_\A(q,\varepsilon)]\}.\]


%%% Local Variables:
%%% mode: latex
%%% TeX-master: "../eai"
%%% End:

\subsection{Примерни задачи}

\begin{example}
  \label{ex:automata-pictures}
  Да разгледаме няколко примера за автомати и езиците, които разпознават.
  Дефинирайте функцията на преходите $\delta$ за всеки автомат.

  \begin{figure}[H]
    \begin{subfigure}[t]{0.45\textwidth}
      \centering
      \begin{tikzpicture}[framed,->,>=stealth,thick,node distance=45pt]
        \tikzstyle{every state}=[circle,minimum size=20pt,auto]
        
        \node[initial below, state]   (0) {$q_0$};
        \node[state]                  (1) [right of=0]{$q_1$};
        \node[state]                  (2) [right of=1]{$q_2$};
        \node[state,accepting]        (3) [right of=2]{$q_3$};
        
        \path 
        (0) edge [loop above]   node [above] {$a$}    (0)
        (0) edge [bend left=15] node [above] {$b$}    (1)
        (1) edge [loop above]   node [above] {$b$}    (1)
        (1) edge [bend left=15] node [above] {$a$}    (2)
        (2) edge [bend left=45] node [below] {$a$}    (0)
        (2) edge [bend left=15] node [above] {$b$}    (3)
        (3) edge [loop above]   node [above] {$a,b$}  (3);
      \end{tikzpicture}
      \subcaption{$\{\omega \in \{a,b\}^\star \mid \omega\mbox{ съдържа }bab\}$}
    \end{subfigure}
    \qquad
    \begin{subfigure}[t]{0.45\textwidth}
      \centering
      \begin{tikzpicture}[framed,->,>=stealth,thick,node distance=45pt]
        \tikzstyle{every state}=[circle,minimum size=20pt,auto]
        
        \node[initial below, state]   (0) {$q_0$};
        \node[state]                  (1) [right of=0]{$q_1$};
        \node[state,accepting]        (2) [right of=1]{$q_2$};
        
        \path 
        (0) edge [loop above]   node [above] {$b$}    (0)
        (0) edge [bend left=15] node [above] {$a$}    (1)
        (1) edge [loop above]   node [above] {$b$}    (1)
        (1) edge [bend left=15] node [above] {$a$}    (2)
        (2) edge [loop above]   node [above] {$a,b$}  (2);
      \end{tikzpicture}
      \subcaption{$\{\omega \in \{a,b\}^\star \mid \card{\omega}{a} \geq 2\}$}
    \end{subfigure}
    \begin{subfigure}[t]{0.45\textwidth}
      \centering
      \begin{tikzpicture}[framed,->,>=stealth,thick,node distance=45pt]
        \tikzstyle{every state}=[circle,minimum size=20pt,auto]
        
        \node[initial below, accepting, state] (0) {$q_0$};
        \node[state]                           (1) [right of=0]{$q_1$};
        \node[state]                           (2) [right of=1]{$q_2$};
        
        \path 
        (0) edge [loop above]   node [above] {$b$}   (0)
        (0) edge [bend left=15] node [above] {$a$}   (1)
        (1) edge [bend left=15] node [below] {$b$}   (0)
        (1) edge [bend left=15] node [above] {$a$}   (2)
        (2) edge [loop above] node   [above] {$a,b$} (2);
      \end{tikzpicture}
      \subcaption{$\{\omega \in \{a,b\}^\star \mid $ всяко $a$ в $\omega$ веднага се следва от поне едно $b\}$ }
    \end{subfigure}
    \qquad
    \begin{subfigure}[t]{0.45\textwidth}
      \centering
      \begin{tikzpicture}[framed,->,>=stealth,thick,node distance=45pt]
        \tikzstyle{every state}=[circle,minimum size=20pt,auto]
        
        \node[initial below, state, accepting]   (0) {$q_0$};
        \node[state]                             (1) [right of=0]{$q_1$};
        \node[state]                             (2) [right of=1]{$q_2$};
        
        \path 
        (0) edge [loop above]   node   [above] {$b$}    (0)
        (0) edge [bend left=15] node   [above] {$a$}    (1)
        (1) edge [loop above]   node   [above] {$b$}    (1)
        (1) edge [bend left=15] node   [above] {$a$}    (2)
        (2) edge [loop above]   node   [above] {$b$}    (2)
        (2) edge [bend left=45] node   [below] {$a$}    (0);
      \end{tikzpicture}
      \subcaption{$\{\omega \in \{a,b\}^\star \mid \card{\omega}{a} \equiv 0 \bmod\ 3\}$}
    \end{subfigure}
  \end{figure}    
\end{example}

В повечето от горните примери може сравнително лесно да се съобрази, че построеният автомат разпознава желания език.
При по-сложни задачи обаче, ще се наложи да дадем доказателство, като обикновено се прилага 
{\em метода на математическата индукция} върху дължината на думите.
Ще разгледаме няколко такива примера.

\begin{problem}
  \mynote{Най-лесния начин да се сетим как да построим $\A$ е като първо построим автомат за езика, който разпознава тези думи, в които се съдържат две поредни срещания на $a$ и вземем неговото допълнение съгласно Твърдение \ref{pr:automata-complement}. По-късно ще разгледаме общ метод за строене на автомат по език.}
  Докажете, че езикът
  \[L = \{\alpha \in \{a,b\}^\star\ \mid\ \alpha\mbox{ не съдържа две поредни срещания на }a\}\]
  е автоматен.
\end{problem}
\begin{proof}
  Да разгледаме $\A = \FA$ с функция на преходите описана на \Figure{dfa:not-two-a}.
  \begin{figure}[H]
    \begin{center}
      \begin{tikzpicture}[framed,->,>=stealth,thick,node distance=55pt]
        \tikzstyle{every state}=[circle,minimum size=20pt,auto]
        
        \node[initial, accepting, state] (0) {$q_0$};
        \node[accepting, state]          (1) [right of=0]{$q_1$};
        \node[state]                     (2) [right of=1]{$q_2$};
        
        \path 
        (0) edge [loop above]   node [above] {$b$}   (0)
        (0) edge [bend left=15] node [above] {$a$}   (1)
        (1) edge [bend left=15] node [below] {$b$}   (0)
        (1) edge [bend left=15] node [above] {$a$}   (2)
        (2) edge [loop above]   node [above] {$a,b$} (2);
      \end{tikzpicture}
    \end{center}
    \caption{Автомат $\A$ разпознаващ думите, които не съдържат две поредни срещания на $a$}
    \label{fig:dfa:not-two-a}
  \end{figure}

  Ще докажем, че $L = \L(\A)$.
  Първо ще се концентрираме върху доказателството на $\L(\A) \subseteq L$.
  \mynote{$\abs{\alpha} \df $ дължината на $\alpha$.}
  Ще докажем, че за всяка дума $\alpha \in \Sigma^\star$ са изпълнени свойствата:
  \begin{enumerate}[(1)]
  \item 
    ако $\delta^\star(q_0,\alpha) = q_0$, то
    $\alpha$ не съдържа две поредни срещания на $a$
    и ако $\abs{\alpha} > 0$, то $\alpha$ завършва на $b$;
  \item
    ако $\delta^\star(q_0,\alpha) = q_1$, то
    $\alpha$ не съдържа две поредни срещания на $a$
    и завършва на $a$.
  \end{enumerate}
  Ще докажем (1) и (2) едновременно с индукция по дължината на думата $\alpha$.  
  \begin{itemize}
  \item
    За $\abs{\alpha} = 0$, т.е. $\alpha = \varepsilon$, твърденията (1) и (2) са ясни (Защо?).
  \item
    Да приемем, че твърденията $(1)$ и $(2)$ са верни за произволни думи $\alpha$ с дължина $n$.
  \item
    Нека $\abs{\alpha} = n+1$, т.е. $\alpha = \beta x$, където $\abs{\beta} = n$ и $x \in \Sigma$.
    Ще докажем (1) и (2) за $\alpha$.
    \begin{itemize}[-]
    \item 
      Нека $\delta^\star(q_0,\beta x) = q_0 = \delta(\delta^\star(q_0,\beta),x)$.
      Според дефиницията на функцията $\delta$, $x = b$ и $\delta^\star(q_0,\beta) \in \{q_0,q_1\}$.
      Тогава по {\bf И.П.} за (1) и (2), $\beta$ не съдържа две поредни срещания на $a$.
      Тогава е очевидно, че $\beta x$ също не съдържа две поредни срещания на $a$.
    \item
      Нека $\delta^\star(q_0,\beta x) = q_1 = \delta(\delta^\star(q_0,\beta),x)$.
      Според дефиницията на $\delta$, $x = a$ и $\delta^\star(q_0,\beta) = q_0$.
      Тогава по {\bf И.П.} за (1), $\beta$ не съдържа две поредни срещания на $a$ и ако $|\beta| > 0$, то $\beta$ завършва на $b$.
      Тогава е очевидно, че $\beta x$ също не съдържа две поредни срещания на $a$.
    \end{itemize}
  \end{itemize}
  
 Така доказахме с индукция по дължината на думата, че за всяка дума $\alpha$ са  изпълнени твърденията $(1)$ и $(2)$.
 По дефиниция, ако $\alpha \in \L(\A)$, то $\delta^\star(q_0,\alpha) \in F = \{q_0,q_1\}$ и от $(1)$ и $(2)$ следва, че и в двата случа
 $\alpha$ не съдържа две поредни срещания на буквата $a$, т.е. $\alpha \in L$.
 С други думи, доказахме, че 
 \[\L(\A) \subseteq L.\]
 Сега ще докажем другата посока, т.е. $L \subseteq \L(\A)$.
 Това означава да докажем, че
 \[(\forall \alpha \in \Sigma^\star)[\alpha \in L\ \Rightarrow\ \delta^\star(q_0,\alpha) \in F],\]
 \mynote{Да напомним, че от съждителното смятане знаем, че $p \rightarrow q \equiv \neg q \rightarrow \neg  p$.}
 което е еквивалентно на
 \begin{equation}
   \label{eq:case2}
   (\forall \alpha \in \Sigma^\star)[\delta^\star(q_0,\alpha) \not\in F \ \Rightarrow\ \alpha\not\in L].
 \end{equation}
 Това е лесно да се съобрази.
 Щом $\delta^\star(q_0,\alpha) \not\in F$, то 
 $\delta^\star(q_0,\alpha) = q_2$ и думата $\alpha$ може да се представи по следния начин:
 \[\alpha = \beta a \gamma\ \&\ \delta^\star(q_0,\beta) = q_1.\]
 
 Използвайки свойство (2) от по-горе, понеже $\delta^\star(q_0,\beta) = q_1$, то
 $\beta$ не съдържа две поредни срещания на $a$, но завършва на $a$.
 Сега е очевидно, че $\beta a$ съдържа две поредни срещания на $a$ и 
 щом $\beta a$ е префикс на $\alpha$, то думата $\alpha \not\in L$.
 С това доказахме Свойство \ref{eq:case2}, а следователно и посоката $L\subseteq \L(\A)$.
\end{proof}


\begin{problem}
  Докажете, че езикът
  \[L = \{\omega \in \{a,b\}^\star \mid a \text{ се среща четен брой, докато $b$ нечетен брой пъти в }\omega\}\]
  е автоматен.
\end{problem}
\begin{hint}
  Разгледайте автомата $\A$:
  \begin{figure}[H]
    \begin{center}
      \begin{tikzpicture}[framed,->,>=stealth,thick,node distance=55pt]
        \tikzstyle{every state}=[circle,minimum size=15pt,auto]
        
        \node[initial,state]      (00) {$q_{00}$};
        \node[state]              (10) [above right of=00]{$q_{10}$};
        \node[accepting, state]   (01) [below right of=00]{$q_{01}$};
        \node[state]              (11) [below right of=10]{$q_{11}$};
        
        \path 
        (00) edge  [bend left=15]  node [above]  {$a$} (10)
        (10) edge  [bend left=15]  node [below]  {$a$} (00)
        (01) edge  [bend right=15] node [above]  {$b$} (00)
        (00) edge  [bend right=15] node [below]  {$b$} (01)
        (10) edge  [bend left=15]  node [above]  {$b$} (11)
        (11) edge  [bend left=15]  node [below]  {$b$} (10)
        (01) edge  [bend right=15] node [below]  {$a$} (11)
        (11) edge  [bend right=15] node [above]  {$a$} (01);
      \end{tikzpicture}
    \end{center}
    \caption{$\L(\A) \stackrel{?}{=} L$}
    \label{fig:dfa:modulus-2}
  \end{figure}

  Докажете с индукция по дължината на думата $\omega$, че:
  \mynote{$\card{\omega}{a} \df $ броят на срещанията на буквата $a$ в думата $\omega$.}
  \begin{enumerate}[a)]
  \item 
    $\delta^\star(q_{00}, \omega) = q_{00} \implies \card{\omega}{a} \equiv 0 \bmod 2\ \&\ \card{\omega}{b} \equiv 0 \bmod 2$;
  \item 
    $\delta^\star(q_{00}, \omega) = q_{01} \implies \card{\omega}{a} \equiv 0 \bmod 2\ \&\ \card{\omega}{b} \equiv 1 \bmod 2$;
  \item 
    $\delta^\star(q_{00}, \omega) = q_{10} \implies \card{\omega}{a} \equiv 1 \bmod 2\ \&\ \card{\omega}{b} \equiv 0 \bmod 2$;
  \item 
    $\delta^\star(q_{00}, \omega) = q_{11} \implies \card{\omega}{a} \equiv 1 \bmod 2\ \&\ \card{\omega}{b} \equiv 1 \bmod 2$;
  \end{enumerate}
  Оттук направете извода, че за произволна дума $\omega$,
  \[(\forall i<2)(\forall j<2)[\ \delta^\star(q_{00},\omega) = q_{ij} \iff \card{\omega}{a} \equiv i \bmod 2\ \&\ \card{\omega}{b} \equiv j \bmod 2\ ].\]
\end{hint}

За една дума $\alpha \in \{0,1\}^\star$, 
нека с $\ov{\alpha}_{(k)}$ да означим числото, което се представя в $k$-ична бройна система като $\alpha$.
Например, 
\[\bin{1101} = 1 \cdot 2^3+1\cdot 2^2+0\cdot 2^1+1\cdot 2^0 = \ov{13}_{(10)} = 1.10^1 + 3.10^0.\]
\mynote{Да отбележим, че за всяко число $n$ има безкрайно много думи $\alpha$, за които $\bin{\alpha} = n$.
  Например,
  \begin{align*}
    \bin{10} & = \bin{010}\\
             & = \bin{0010}\\
             & = \cdots
  \end{align*}}
За $k = 2$, имаме следната дефиниция:
\begin{itemize}
\item
  $\bin{\varepsilon} = 0$,
\item
  $\bin{\alpha 0} = 2\cdot \bin{\alpha}$,
\item
  $\bin{\alpha 1} = 2\cdot \bin{\alpha} + 1$.
\end{itemize}

\begin{problem}\label{prob:regular:dfa:binary}
  Докажете, че езикът $L = \{\alpha \in \{0,1\}^\star \mid \bin{\alpha} \equiv 2 \bmod\ 3\}$
  е автоматен.
\end{problem}
\begin{proof}
  Нашият автомат ще има три състояния $\{q_0,q_1,q_2\}$, като началното състояние ще бъде $q_0$.
  Целта ни е да дефинираме така автомата, че да имаме следното свойство:
  \begin{equation}
    (\forall\alpha\in\Sigma^\star)(\forall i < 3)[\ \bin{\alpha} \equiv i \bmod\ 3\ \Leftrightarrow\ \delta^\star(q_0,\alpha) = q_i\ ],
  \end{equation}
  т.е. всяко състояние отговаря на определен остатък при деление на три.
  Понеже искаме нашия автомат да разпознава тези думи $\alpha$,
  за които $\alpha_{(2)} \equiv 2\mod 3$, финалното състояние ще бъде $q_2$.
  Дефинираме функцията $\delta$ по следния начин:
  \begin{align*}
    & \delta(q_0,0) = q_0 & \comment{\ \bin{\alpha} \equiv 0 \bmod 3 \implies \bin{\alpha 0} \equiv 0 \bmod 3 }\\
    & \delta(q_0,1) = q_1 & \comment{\ \bin\alpha \equiv 0 \bmod 3 \implies \bin{\alpha 1} \equiv 1 \bmod 3 }\\
    & \delta(q_1,0) = q_2 & \comment{\ \bin{\alpha} \equiv 1 \bmod 3 \implies \bin{\alpha 0} \equiv 2 \bmod 3 }\\
    & \delta(q_1,1) = q_0 & \comment{\ \bin{\alpha} \equiv 1 \bmod 3 \implies \bin{\alpha 1} \equiv 0 \bmod 3 }\\
    & \delta(q_2,0) = q_1 & \comment{\ \bin{\alpha} \equiv 2 \bmod 3 \implies \bin{\alpha 0} \equiv 1 \bmod 3 }\\
    & \delta(q_2,1) = q_2 & \comment{\ \bin{\alpha} \equiv 2 \bmod 3 \implies \bin{\alpha 1} \equiv 2 \bmod 3 }
  \end{align*}
  Ето и картинка на автомата $\A$:
  \begin{framed}
  \begin{figure}[H]
    \begin{center}
      \begin{tikzpicture}[->,>=stealth,thick,node distance=55pt]
        \tikzstyle{every state}=[circle,minimum size=15pt,auto]
        
        \node[initial,state]      (0) {$q_0$};
        \node[state]              (1) [right of=0]{$q_1$};
        \node[accepting, state]   (2) [right of=1]{$q_2$};
        
        \path 
        (0) edge  [loop above]    node [above]  {$0$} (0)
        (0) edge  [bend left=15]  node [above]  {$1$} (1)
        (2) edge  [bend left=15]  node [below]  {$0$} (1)
        (1) edge  [bend left=15]  node [below]  {$1$} (0)
        (1) edge  [bend left=15]  node [above]  {$0$} (2)
        (2) edge  [loop above]    node [above]  {$1$} (2);
      \end{tikzpicture}
      \end{center}
      \caption{$\L(\A) \stackrel{?}{=} \{\alpha\in\{0,1\}^\star \mid \bin{\alpha} \equiv 2\ (\bmod\ 3)\}$}
 \end{figure}
 \end{framed}
 \noindent Да разгледаме твърденията:
 \begin{enumerate}[(1)]
  \item 
    $\delta^\star(q_0,\alpha) = q_0 \implies \bin{\alpha} \equiv 0 \mod 3$;
  \item 
    $\delta^\star(q_0,\alpha) = q_1 \implies \bin{\alpha} \equiv 1 \mod 3$;
  \item 
    $\delta^\star(q_0,\alpha) = q_2 \implies \bin{\alpha} \equiv 2 \mod 3$.
  \end{enumerate}
  \mynote{Обърнете внимание, че в доказателството на (3) използваме И.П. не само за (3), но и за (2). Поради тази причина трябва да докажем трите свойства едновременно}
  Ще докажем (1), (2) и (3) {\em едновременно} с индукция по дължината на думата $\alpha$.
  За $\abs{\alpha} = 0$, всички условия са изпълнени. (Защо?)
  Да приемем, че (1), (2) и (3) са изпълнени за думи с дължина $n$.
  Нека $\abs{\alpha} = n+1$, т.е. $\alpha = \beta x$, $\abs{\beta} = n$.

  Ще докажем подробно само (3) понеже другите твърдения се доказват по сходен начин.
  Нека $\delta^\star(q_0,\beta x) = q_2$. 
  Имаме два случая:
  \begin{itemize}
  \item 
    $x = 0$. 
    Тогава, по дефиницията на $\delta$, 
    $\delta(q_1,0) = q_2$ и следователно, $\delta^\star(q_0,\beta) = q_1$.
    По {\bf И.П.} за (2) с $\beta$,
    \[\delta^\star(q_0,\beta) = q_1\ \Rightarrow\ \bin{\beta} \equiv 1 \bmod 3\]
    Тогава, $\bin{\beta0} \equiv 2 \mod 3$. Така доказахме, че
    \[\delta^\star(q_0,\beta 0) = q_2\ \Rightarrow\ \bin{\beta 0} \equiv 2 \bmod 3.\]
  \item
    $x = 1$.
    Тогава, по дефиницията на $\delta$, $\delta(q_2,1) = q_2$ и следователно,
    $\delta^\star(q_0,\beta) = q_2$.
    По {\bf И.П.} за (3) с $\beta$,
    \[\delta^\star(q_0,\beta) = q_2\ \Rightarrow\ \bin{\beta} \equiv 2 \bmod 3.\]
    Тогава, $\bin{\beta1} \equiv 2 \mod 3$. Така доказахме, че
    \[\delta^\star(q_0,\beta 1) = q_2\ \Rightarrow\ \bin{\beta 1} \equiv 2 \bmod 3.\]
  \end{itemize}
  \mynote{\writedown Довършете доказателствата на (1) и (2)}
  За да докажем (1), нека $\delta^\star(q_0,\beta x) = q_0$. 
  \begin{itemize}
  \item 
    $x = 0$. Разсъжденията са аналогични, като използваме {\bf И.П.} за (1).
  \item
    $x = 1$. Разсъжденията са аналогични, като използваме {\bf И.П.} за (2).
  \end{itemize}
  
  По същия начин доказваме и (2). Нека $\delta^\star(q_0,\beta x) = q_1$. 
  \begin{itemize}
  \item 
    При $x = 0$, използваме {\bf И.П.} за (3).
  \item
    При $x = 1$, използваме {\bf И.П.} за (1).
  \end{itemize}
  \mynote{\writedown Защо?}
  От (1), (2) и (3) следва директно, че
  \[(\forall\alpha\in\Sigma^\star)(\forall i < 3)[\ \bin{\alpha} \equiv i\ (\bmod\ 3)\ \Leftrightarrow\ \delta^\star(q_0,\alpha) = q_i\ ],\]
  откъдето получаваме, че $\L(\A) = L$.
\end{proof}


%%% Local Variables:
%%% mode: latex
%%% TeX-master: "../eai"
%%% End:

\subsection{Затвореност относно сечиение, обединение, разлика}

\begin{framed}
  \begin{proposition}
    \index{автоматни езици!сечение}
    \label{pr:automata-cap}
    Класът на автоматните езици е затворен относно операцията {\em сечение}.
    Това означава, че ако $L_1$ и $L_2$ са два произволни автоматни езика над азбуката $\Sigma$, то $L_1\cap L_2$
    също е автоматен език.
  \end{proposition}  
\end{framed}
\begin{proof}
  Да разгледаме два автомата \[\A' = \langle{\Sigma,Q',\qstart',\delta',F'}\rangle\text{ и } \A'' = \langle{\Sigma, Q'', \qstart'', \delta'', F''}\rangle.\]
  Определяме автомата $\A = \FA$, по следния начин:
  \marginpar{Изчислението на автомата $\A$ върху думата $\alpha$ едновременно симулира изчислението на $\A'$ и $\A''$ върху $\alpha$.}
  \begin{itemize}
  \item
    $Q \df Q'\times Q''$;
  \item
    \marginpar{Съобразете, че $\delta$ е тотална функция.}
    За всяко $\pair{r_1,r_2} \in Q$ и всяко $a \in \Sigma$,
    \[\delta(\pair{r_1,r_2},a) \df \pair{\delta_1(r_1,a),\delta_2(r_2,a)};\]
  \item
    $\qstart \df \pair{\qstart',\qstart''}$;
  \item
    $F \df \{\pair{r_1,r_2}\mid r_1\in F''\ \&\ r_2 \in F''\} = F' \times F''$
  \end{itemize}
  Трябва да докажем, че за всяка дума $\alpha \in \Sigma^\star$ е изпълнено, че:
  \begin{equation}
    \label{eq:7}
    (\forall p\in Q')(\forall q\in Q'')[\ \delta^\star(\pair{p,q},\alpha) = \pair{ \delta^\star_1(p,\alpha), \delta^\star_2(q,\alpha) }\ ].
  \end{equation}
  Ще докажем \Property{eq:7} с индукция по дължината на думата $\alpha$.
  \begin{itemize}
  \item
    Нека $|\alpha| = 0$, т.е. $\alpha = \varepsilon$. Тогава всичко е ясно, защото
    \begin{align*}
      \delta^\star( \pair{p,q}, \varepsilon) & =\pair{p,q} & \comment\text{деф. на }\delta^\star\\
                                             & = \pair{\delta^\star_1(p,\varepsilon), \delta^\star_2(q,\varepsilon)}. & \comment\text{деф. на $\delta^\star_1$ и $\delta^\star_2$} 
    \end{align*}
  \item
    Да приемем, че \Property{eq:7} е изпълнено за думи $\alpha$ с дължина $n$.
  \item
    Нека $|\alpha| = n+1$, т.е. $\alpha = \beta a$ и $|\beta| = n$. Тогава:
    \begin{align*}
      \delta^\star(\pair{p, q},\beta a) & = \delta( \delta^\star( \pair{p, q}, \beta ), a) & \comment{\text{деф. на }\delta^\star}\\
                                        & = \delta( \pair{ \delta^\star_1(p, \beta), \delta^\star_2(q, \beta)}, a ) & \comment{\text{ от И.П.}}\\
                                        & = \pair{ \delta_1( \delta^\star_1(p, \beta), a), \delta_2(\delta^\star_2(q,\beta), a)} & \comment\text{деф. на }\delta\\
                                        & = \pair{ \delta^\star_1(p, \beta a), \delta^\star_2(q, \beta a) } & \comment{\text{деф. на $\delta^\star_1$ и $\delta^\star_2$}}.
    \end{align*}
  \end{itemize}
  Използвайки \Property{eq:7} лесно можем да докажем, че
  \[\L(\A) = \L(\A') \cap \L(\A'').\]
  Имаме следните еквивалентности:
  \begin{align*}
    \omega \in \L(\A) & \iff \delta^\star(\qstart, \omega) \in F & \comment\text{деф. на }\L(\A)\\
                      & \iff \delta^\star(\pair{\qstart',\qstart''}, \omega) \in F' \times F'' & \comment\text{деф. на }\A\\
                      & \iff \pair{\delta^\star_1(\qstart',\omega), \delta^\star_2(\qstart'',\omega)} \in F' \times F'' & \comment{\text{от (\ref{eq:7})}}\\
                      & \iff \delta^\star_1(\qstart',\omega) \in F'\ \&\ \delta^\star_2(\qstart'',\omega) \in F''\\
                      & \iff \omega \in \L(\A')\ \&\ \omega\in\L(\A'') \\
                      & \iff \omega \in \L(\A') \cap \L(\A'').
  \end{align*}
  
\end{proof}

\begin{framed}
  \begin{proposition}
    \index{автоматни езици!допълнение}
    \label{pr:automata-complement}
    Класът на автоматните езици са затворени относно операцията допълнение, т.е.
    ако $L$ е автоматен език, то $\Sigma^\star\setminus L$ също е автоматен език.
  \end{proposition}  
\end{framed}
\begin{hint}
  \marginpar{\writedown Проверете, че $\Sigma^\star\setminus L = \L(\A')$. Съобразете, че тук е важно, че $\delta$ е тотална функция на преходите.}
  Нека $L = L(\A)$, където $\A = \FA$.
  Да вземем автомата
  \[\A' = \pair{Q,\Sigma,\qstart,\delta,Q\setminus F},\]
  т.е. $\A'$ е същия като $\A$, с единствената разлика, че финалните състояния на $\A'$
  са тези състояния, които {\bf не} са финални в $\A$.
\end{hint}

\begin{framed}
  \begin{proposition}
    \index{автоматни езици!обединение}
    \label{pr:automata-union}
    Класът на автоматните езици е затворен относно операцията {\bf обединение}.
    Това означава, че ако $L_1$ и $L_2$ са два произволни автоматни езика, то $L_1\cup L_2$
    също е автоматен език.
  \end{proposition}  
\end{framed}
\begin{hint}
  \marginpar{\writedown Докажете, че така построения автомат $\A$ разпознава $L_1\cup L_2$. Тук отново е важно, че $\delta_1$ и $\delta_2$ са тотални функции на преходите.}
  Първият подход е да използвайте конструкцията на автомата $\A$ от \Proposition{automata-cap},
  с единствената разлика, че тук избираме финалните състояния да бъдат елементите на множеството
  \begin{align*}
    F & \df \{\pair{q',q''} \in Q' \times Q'' \mid q' \in F'\ \lor\ q'' \in F''\}\\
      & = F'\times Q'' \cup Q'\times F''.
  \end{align*}
  Друг подход е да се използва правилото на Де Морган, а именно:
  \[L_1 \cup L_2 = \ov{\ov{L_1} \cap \ov{L_2}}.\]
\end{hint}

%%% Local Variables:
%%% mode: latex
%%% TeX-master: "../eai"
%%% End:

\section{Регулярни изрази и езици}

Да фиксираме една непразна азбука $\Sigma$.
\index{регулярен израз}
{\bf Регулярните изрази} $\mathbf{r}$ могат да се опишат със следната абстрактна граматика
\[\mathbf{r} ::= \bm{\emptyset}\ |\ \bm{\varepsilon}\ |\ \mathbf{a}\ |\ (\bm{r}_1 \cdot \bm{r}_2)\ |\ (\bm{r}_1 + \bm{r_2})\ |\ \bm{r}^\star_1.\]
Регулярните изрази могат да се опишат и по следния начин:
\marginpar{Това е пример за индуктивна дефиниция}
\begin{itemize}
\item 
  Символите $\bm{\emptyset}$, $\bm{\varepsilon}$ са регулярни изрази;
\item
  за всяка буква $a \in \Sigma$, символът $\bm{a}$ е регулярен израз;
\item
  \marginpar{В литературата също се среща записът $(\bm{r}_1\ |\ \bm{r}_2)$ вместо $(\bm{r}_1 + \bm{r}_2)$}
  ако $\mathbf{r_1}$ и $\mathbf{r_2}$ са регулярни изрази, то думите $(\bm{r}_1 \cdot \bm{r}_2)$, $(\bm{r}_1 + \bm{r}_2)$ и $\bm{r}^\star_1$
  също са регулярни изрази;
\item
  Всеки регулярен израз е получен по някое от горните правила.
\end{itemize}

\index{език!регулярен}
\marginpar{Това е друг пример за индуктивна (рекурсивна) дефиниция.}
Сега ще дефинираме езиците, които се описват с регулярни изрази.
Тези езици се наричат {\bf регулярни}.
Това ще направим следвайки индуктивната дефиниция на регулярните изрази,
т.е. за всеки регулярен израз $\mathbf{r}$ ще определим език $\L(\mathbf{r})$.
\begin{itemize}
\item
  $\emptyset$ е регулярен език,
  който се описва от регулярния израз $\bm{\emptyset}$. Означаваме $\L(\bm{\emptyset}) = \emptyset$;
\item
  $\{\varepsilon\}$ е регулярен език,
  който се описва от регулярния израз $\bm{\varepsilon}$.
  Означаваме $\L(\bm{\varepsilon}) = \{\varepsilon\}$;
\item
  за всяка буква $a \in \Sigma$, $\{a\}$ е регулярен език,
  който се описва от регулярния израз $\mathbf{a}$.
  Означаваме $\L(\mathbf{a}) = \{a\}$;
\item
  \marginpar{Понякога, когато приоритетът на операциите е ясен, ще изпускаме да пишем скоби.}
  Нека $L_1$ и $L_2$ са регулярни езици, т.е. съществуват регулярни изрази $\mathbf{r}_1$
  и $\mathbf{r}_2$, за които $\L(\mathbf{r}_1) = L_1$ и $\L(\mathbf{r}_2) = L_2$.
  Тогава:
  \begin{itemize}
  \item 
    \index{регулярни операции!обединение}
    $L_1 \cup L_2$ е регулярен език, който се описва с регулярния израз $(\mathbf{r}_1 + \mathbf{r}_2)$.
    Това означава, че $\L(\mathbf{r}_1) \cup \L(\mathbf{r}_2) = \L(\mathbf{r}_1 + \mathbf{r}_2)$.
  \item
    \index{регулярни операции!конкатенация}
    \marginpar{Тази операция се нарича конкатенация. Обикновено изпускаме знака $\cdot$}
    $L_1 \cdot L_2$ е регулярен език, който се описва с регулярния израз $(\mathbf{r}_1 \cdot \mathbf{r}_2)$.
    Това означава, че $\L(\mathbf{r}_1) \cdot \L(\mathbf{r}_2) = \L(\mathbf{r}_1 \cdot \mathbf{r}_2)$.
  \item
    \marginpar{Звезда на Клини}
    \index{регулярни операции!звезда на Клини}
    $L^\star_1$ е регулярен език, който се описва с регулярния израз $\mathbf{r}^\star_1$.
    Това означава, че $\L(\mathbf{r}_1)^\star = \L(\mathbf{r}^\star_1)$.
  \end{itemize}
\end{itemize}

\begin{remark}
  Ние знаем, че:
  \begin{itemize}
  \item
    Всеки регулярен израз представлява крайна дума над крайна азбука.
    Това означава, че множеството от всички регулярни изрази е изброимо безкрайно.
    Оттук следва, че всички регулярни езици образуват изброимо безкрайно множество.
  \item 
    Понеже $\Sigma$ е крайна азбука, то $\Sigma^\star$ е изброимо безкрайно множество;
  \item
    Един език над азбуката $\Sigma$ представлява елемент на $\Ps(\Sigma^\star)$.
    Това означава, че всички езици над азбуката $\Sigma$ представляват неизброимо безкрайно множество.
  \end{itemize}
  От всичко това следва, че има езици, които не са регулярни.
  По-нататък ще видим примери за такива езици.
\end{remark}

\begin{example}
  \marginpar{В \cite[стр. 70]{sipser3} е показан алгоритъм, за който по един автомат може да се получи регулярен израз описващ езика на автомата. Ние няма да разглеждаме този алгоритъм. }
  Нека да построим регулярни изрази за всеки от езиците от \Ex{automata-pictures}.
  \begin{enumerate}[a)]
  \item 
    Нека $\mathbf{r} \df \mathbf{(a+b)^\star bab(a+b)^\star}$. Тогава
    \[\L(\mathbf{r}) = \{\omega \in \{a,b\}^\star \mid \omega \text{ съдържа } bab\}.\]
  \item
    Нека $\mathbf{r} \df \mathbf{b^\star ab^\star a(a+b)^\star}$. Тогава
    \[\L(\mathbf{r}) = \{\omega \in \{a,b\}^\star \mid N_a(\omega) \geq 2\}.\]
  \item
    Нека $\mathbf{r} \df \mathbf{(b^\star abb^\star)^\star}$. Тогава
    \[\L(\mathbf{r}) = \{\omega \in \{a,b\}^\star \mid \text{ всяко $a$ в $\omega$ се следва от поне едно $b$}\}.\]
  \item
    Нека $\mathbf{r} \df \mathbf{(b^\star ab^\star ab^\star ab^\star)^\star}$. Тогава
    \[\L(\mathbf{r}) = \{\omega \in \{a,b\}^\star \mid N_a(\omega) \equiv 0 \bmod 3\}.\]
  \end{enumerate}
\end{example}


  
\begin{problem}
  За произволни регулярни изрази $\bm{r}$ и $\bm{s}$, ще пишем $\bm{r} \equiv \bm{s}$, ако $\L(\bf{r}) = \L(\bm{s})$.
  Проверете дали са изпълнени следните равенства:
  \begin{enumerate}[a)]
  \item 
    $\bm{r + s} \equiv \bm{s + r}$;
  \item
    $\bm{(\varepsilon + r)^\star} \equiv \bm{r^\star}$;
  \item
    $\bm{\emptyset^\star} \equiv \bm{\varepsilon}$;
  \item
    $\bm{(r^\star s^\star)^\star} \equiv \bm{(r+s)^\star}$;
  \item
    $\bm{(r^\star)^\star} \equiv \bm{r^\star}$;
  \item
    $\bm{(rs + r)^\star r} \equiv \bm{r(sr+r)^\star}$;
  \item
    $\bm{s(rs+s)^\star r} \equiv \bm{rr^\star s(rr^\star s)^\star}$;
  \item
    $\bm{(r+s)^\star} \equiv \bm{r^\star + s^\star}$;
  \item
    $\bm{(r+s)^\star s} \equiv \bm{(r^\star s)^\star}$;
  \item
    $\bm{(rs + r)^\star rs} \equiv \bm{(rr^\star s)^\star}$;
  \item
    $\bm{\emptyset^\star} \equiv \bm{\varepsilon^\star}$.
  \end{enumerate}
\end{problem}



%%% Local Variables: 
%%% mode: latex
%%% TeX-master: "../eai"
%%% End: 

\begin{framed}
\begin{thm}[Клини]
  \label{th:regular-kleene}
  \index{Клини}
  Всеки автоматен език се описва с регулярен израз.
\end{thm}
\end{framed}
\begin{proof}
  \marginpar{\cite[стр. 79]{papadimitriou}; \cite[стр. 33]{hopcroft1}}
  Нека  $L = \L(\A)$, за някой краен детерминиран автомат $\A$.
  Да фиксираме едно изброяване на състоянията $Q = \{q_1,\dots,q_n\}$,
  като началното състояние е $q_1$.
  Ще означаваме с $L(i,j,k)$ множеството от тези думи, които
  могат да се разпознаят от автомата по път, който започва от $q_i$,
  завършва в $q_j$, и междинните състояния имат индекси $\leq k$.
  Например, за думата $\alpha = a_1a_2\cdots a_n$ имаме, че $\alpha \in L(i,j,k)$
  точно тогава, когато съществуват състояния $q_{l_1},\dots,q_{l_{n-1}}$, като $l_1,\dots,l_{n-1} \leq k$ и
  \[q_i\stackrel{a_1}{\rightarrow} q_{l_1} \stackrel{a_2}{\rightarrow} q_{l_2} \stackrel{a_3}{\rightarrow} \dots \stackrel{a_{n-1}}{\rightarrow} q_{l_{n-1}}\stackrel{a_n}{\rightarrow} q_j.\]
  Тогава за $n = \abs{Q}$, 
  \[L(i,j,n) = \{\alpha\in\Sigma^\star\mid \delta^\star(q_i,\alpha) = q_j\}.\]
  Така получаваме, че 
  \[\L(\A) = \bigcup\{L(1,j,n)\mid q_j \in F\} = \bigcup_{q_j\in F}L(1,j,n).\]
  Ще докажем с {\em индукция по $k$}, че за всяко $i,j,k$, множествата от думи $L(i,j,k)$
  се описват с регулярен израз $\mathbf{r^k_{i,j}}$
  \begin{enumerate}[a)]
  \item
    Нека $k = 0$. Ще докажем, че за всяко $i,j$, $L(i,j,0)$ се описва с регулярен израз.
    Имаме да разгледаме два случая.
    
    Ако $i = j$, то 
    \begin{equation}
      \label{eq:kleene-equal}
      L(i, j, 0) = \{\varepsilon\}\cup\{a\in\Sigma \mid \delta(q_i,a) = q_j\}.
    \end{equation}
    Ако $i \neq j$, то
    \[L(i, j, 0) = \{a\in\Sigma \mid \delta(q_i, a) = q_j\}.\]
    И в двата случая, понеже $L(i,j,0)$ е краен език, то е ясно, че той се описва с регулярен израз.
  \item
    Да предположим, че $k > 0$, и за всяко $i,j \leq n$, имаме регулярните изрази $\mathbf{r}^{k-1}_{i,j}$, които
    описват езиците $L(i,j,k-1)$, т.е. имаме индукционното предположение, че
    \[(\forall i,j \leq n)[L(i,j,k-1) = \L(\mathbf{r}^{k-1}_{i,j})].\] 
    Ще докажем, че съществуват регулярни изрази $\mathbf{r}^k_{i,j}$, такива че
    \[(\forall i,j \leq n)[L(i,j,k) = \L(\mathbf{r}^{k}_{i,j})].\] 
    Можем да изразим езика $L(i,j,k)$ по следния начин:
    \[L(i,j,k) = \underbrace{L(i,j,k-1)}_{\L(\mathbf{r}^{k-1}_{i,j})}\ \cup\ \underbrace{L(i,k,k-1)}_{\L(\mathbf{r}^{k-1}_{i,k})}\cdot (\underbrace{L(k,k,k-1)}_{\L(\mathbf{r}^{k-1}_{k,k})})^\star \cdot \underbrace{L(k,j,k-1)}_{\L(\mathbf{r}^{k-1}_{k,j})}.\]

    Тогава по {\bf И.П.} следва, че $L(i,j,k)$ може да се опише с регулярния израз
    \begin{equation}
      \label{eq:kleene}
      \mathbf{r}^k_{i,j} = \mathbf{r}^{k-1}_{i,j} + \mathbf{r}^{k-1}_{i,k}\cdot (\mathbf{r}^{k-1}_{k,k})^\star\cdot \mathbf{r}^{k-1}_{k,j}.
    \end{equation}
  \end{enumerate}
  Заключаваме, че за всяко $i,j,k$, $L(i,j,k)$ може да се опише с регулярен израз $\mathbf{r}^{k}_{i,j}$.
  Тогава ако $F = \{q_{i_1},\dots,q_{i_k}\}$, то $\L(\A)$ се описва с регулярния израз
  \[\mathbf{r}^n_{1,i_1} + \mathbf{r}^n_{1,i_2} + \dots + \mathbf{r}^n_{1,i_k}.\]
\end{proof}

\begin{cor}
  Съществува алгоритъм, за който при вход краен детерминиран автомат $\A$,
  извежда като изход регулярен израз $\mathbf{r}$, такъв че $\L(\A) = \L(\mathbf{r})$.
\end{cor}

Доказателството на това следствие използва идеята от доказателството на \Th{regular-kleene},
но излиза извън нашия интерес. За подробно изложение на този въпрос, вижте \cite[стр. 69]{sipser3}.
Възможно е по-късно да използваме този резултат наготово.
Нека поне да разгледаме един пример, чиято цел е да ни убеди, 
че наистина може да се извлече алгоритъм от доказателството на \Th{regular-kleene}.

\begin{example}
  Да разгледаме следния автомат:

  \begin{framed}
    \begin{figure}[H]
      \begin{center}
        \begin{tikzpicture}[->,>=stealth,thick,node distance=45pt]
          \tikzstyle{every state}=[circle,minimum size=15pt,auto]
          
          \node[initial,state]      (1) {$q_1$};
          \node[accepting, state]   (2) [right of=1]{$q_2$};
          
          \path 
          (1) edge [loop above]  node [above] {$1$} (1)
          (1) edge  node [above] {$0$} (2)
          (2) edge [loop above] node [above] {$0,1$} (2);
        \end{tikzpicture}
      \end{center}
      \caption{Автомат разпознаващ $\L(\mathbf{1^\star 0 (0 + 1)^\star)}$}
      \label{fig:a1}
    \end{figure}
\end{framed}

Лесно се съобразява, че езикът на автомата от Фигура \ref{fig:a1} се описва с регулярния израз $\mathbf{1^\star 0 (0 + 1)^\star}$.
Следвайки конструкцията от доказателството на \Th{regular-kleene},
езикът на този автомат се описва с регулярния израз $\mathbf{r^2_{1,2}}$, защото началното състояние е $q_1$, финалното е $q_2$ и 
броят на състоянията в автомата е $2$.
\begin{align*}
  \mathbf{r}^2_{1,2} & = \underbrace{\mathbf{r}^{1}_{1,2}}_{\mathbf{1^\star 0}} + \underbrace{\mathbf{r}^{1}_{1,2}}_{\mathbf{1^\star 0}}\cdot \underbrace{\mathbf{(r^1_{2,2})^\star}}_{\mathbf{(\varepsilon+0+1)^\star}} \cdot \underbrace{\mathbf{r}^1_{2,2}}_{\mathbf{\varepsilon+0+1}} & (\text{според (\ref{eq:kleene}})) \\
                     &  = \mathbf{1^\star0 + 1^\star 0 (\varepsilon + 0 + 1)^\star (\varepsilon + 0 + 1)}\\
                     & =  \mathbf{1^\star0 + 1^\star 0 (\varepsilon + 0 + 1)^+} & (\mathbf{r^+ = r^\star r})\\
                     & =  \mathbf{1^\star0 + 1^\star 0 (0 + 1)^\star} & (\mathbf{r^\star = (\varepsilon + r)^+})\\
                     & = \mathbf{1^\star 0 (\varepsilon + (0 + 1)^\star)} & (\mathbf{r + rq = r(\varepsilon + q)})\\
                     & = \mathbf{1^\star 0 (0 + 1)^\star} & (\mathbf{r^\star = \varepsilon + r^\star})
\end{align*}

Тук използвахме, че:
\begin{align*}
  \mathbf{r^1_{1,2}} & = \underbrace{\mathbf{r^0_{1,2}}}_{\mathbf{0}} + \underbrace{\mathbf{r^0_{1,1}}}_{\mathbf{\varepsilon + 1}}\cdot\underbrace{\mathbf{(r^0_{1,1})^\star}}_{\mathbf{(\varepsilon+1)^\star}} \cdot \underbrace{\mathbf{r^0_{1,2}}}_{\mathbf{0}}\\
                     & = \mathbf{0 + (\varepsilon + 1)(\varepsilon + 1)^\star0} \\
                     & = \mathbf{0 + 1^\star 0}\\
                     & = \mathbf{1^\star0},\\
  \mathbf{r^1_{2,2}} & = \underbrace{\mathbf{r^0_{2,2}}}_{\mathbf{\varepsilon+0+1}} + \underbrace{\mathbf{r^0_{2,1}}}_{\mathbf{\emptyset}} \cdot \underbrace{\mathbf{(r^0_{1,1})^\star}}_{\mathbf{\varepsilon+1}}\cdot \underbrace{\mathbf{r^0_{1,2}}}_{\mathbf{0}}\\
                     & = \mathbf{\varepsilon + 0 + 1 + \emptyset(\varepsilon + 1)^\star0}\\
                     & = \varepsilon + 0 + 1 & (\text{защото }\mathbf{\emptyset \cdot r = \emptyset})
\end{align*}
\end{example}

Следващата ни цел е да видим, че имаме и обратната посока на горната лема.
Ще докажем, че всеки регулярен език е автоматен. За тази цел първо ще 
въведем едно обобщение на понятието краен детерминиран автомат.


%%% Local Variables:
%%% mode: latex
%%% TeX-master: "../eai"
%%% End:

\subsection{Примерни задачи}

\begin{problem}
  Приложете конструкцията от \hyperref[th:regular:kleenef]{теоремата на Клини} за да получите регулярен израз, който описва езика на автомата $\A$ изобразен на Фигура~\ref{fig:a1}.
    \begin{figure}[H]
      \begin{center}
        \begin{tikzpicture}[framed,->,>=stealth,thick,node distance=45pt]
          \tikzstyle{every state}=[circle,minimum size=15pt,auto]
          
          \node[initial,state]      (1) {$q_0$};
          \node[accepting, state]   (2) [right of=1]{$q_1$};
          
          \path 
          (1) edge [loop above]  node [above] {$b$} (1)
          (1) edge  node [above] {$a$} (2)
          (2) edge [loop above] node [above] {$a,b$} (2);
        \end{tikzpicture}
      \end{center}
      \caption{$\L(\A) \stackrel{?}{=} \L(\mathbf{b^\star a (a + b)^\star)}$}
      \label{fig:a1}
    \end{figure}
\end{problem}
\ExtraMaterial{
\begin{solution}
  Лесно се съобразява, че езикът на автомата от \Figure{a1} се описва с регулярния израз $\mathbf{b^\star a (a + b)^\star}$.
  Следвайки означенията и конструкцията от доказателството на \hyperref[th:regular:kleene]{теоремата на Клини},
  езикът на този автомат се описва с регулярния израз $\mathbf{r}^2_{0,1}$, защото началното състояние е $q_0$, финалното е $q_1$ и 
  броят на състоянията в автомата е $2$. Подробните сметки са следните:
  \begin{align*}
    \bm{r}^2_{0,1} & = \underbrace{\bm{r}^{1}_{0,1}}_{\bm{b^\star a}} + \underbrace{\bm{r}^{1}_{0,1}}_{\bm{b^\star a}}\cdot \underbrace{\bm{(r^1_{1,1})^\star}}_{\mathbf{(\bm{\varepsilon}+a+b)^\star}} \cdot \underbrace{\mathbf{r}^1_{1,1}}_{\mathbf{\bm{\varepsilon}+a+b}} & \comment \text{според (\ref{eq:kleene}}) \\
                   &  = \mathbf{b^\star a + b^\star a (\bm{\varepsilon} + a + b)^\star (\bm{\varepsilon} + a + b)} & \comment{\text{просто заместваме}}\\
                   & = \mathbf{b^\star a + b^\star a (\bm{\varepsilon} + a + b)^+} & \comment \mathbf{r^+ \df r^\star r}\\
                   & = \mathbf{b^\star a + b^\star a (a + b)^\star} & \comment \mathbf{r^\star = (\bm{\varepsilon} + r)^+}\\
                   & = \mathbf{b^\star a (\bm{\varepsilon} + (a + b)^\star)} & \comment \mathbf{r + rq = r(\bm{\varepsilon} + q)}\\
                   & = \mathbf{b^\star a (a + b)^\star}. & \comment \mathbf{r^\star = \bm{\varepsilon} + r^\star}
\end{align*}
Тук използвахме, че:
\begin{align*}
  \mathbf{r^1_{0,1}} & = \underbrace{\mathbf{r}^0_{0,1}}_{\mathbf{a}} + \underbrace{\mathbf{r}^0_{0,0}}_{\bm{\varepsilon + b}}\cdot\underbrace{\mathbf{(r^0_{0,0})^\star}}_{\bm{(\varepsilon+b)^\star}} \cdot \underbrace{\mathbf{r^0_{0,1}}}_{\mathbf{b}} & \comment \text{според (\ref{eq:kleene}})\\
                     & = \bm{a + (\varepsilon + b)(\varepsilon + b)^\star a} & \comment{\text{просто заместваме}}\\
                     & = \mathbf{a + b^\star a}  & \comment \mathbf{r}^\star = \bm{\varepsilon} + \mathbf{r}^\star \\
                     & = \mathbf{b^\star a} \\
  \mathbf{r}^b_{b,b} & = \underbrace{\mathbf{r}^a_{b,b}}_{\bm{\varepsilon+a+b}} + \underbrace{\mathbf{r}^a_{b,a}}_{\bm{\emptyset}} \cdot
                       \underbrace{(\mathbf{r}^a_{a,a})^\star}_{\bm{\varepsilon+b}} \cdot \underbrace{\mathbf{r}^a_{a,b}}_{\bm{a}}& \comment \text{отново според (\ref{eq:kleene}})\\
                     & = \bm{\varepsilon + a + b + \emptyset(\varepsilon + b)^\star a} & \comment{\text{просто заместваме}}\\
                     & = \bm{\varepsilon + a + b}. & \comment \text{защото }\bm{\emptyset \cdot r = \emptyset}
\end{align*}  
\end{solution}
}

\begin{problem}
  Приложете конструкцията от \hyperref[th:regular:kleenef]{теоремата на Клини} за да получите регулярен израз, който описва езика на автомата $\A$ изобразен на Фигура~\ref{fig:a2}.
  \begin{figure}[H]
      \begin{center}
        \begin{tikzpicture}[framed,->,>=stealth,thick,node distance=45pt]
          \tikzstyle{every state}=[circle,minimum size=15pt,auto]
          
          \node[initial,state]      (1) {$q_0$};
          \node[state]              (2) [right of=1]{$q_1$};
          \node[accepting, state]   (3) [right of=2]{$q_2$};
          
          \path 
          (2) edge [loop above]    node [above] {$a$} (2)
          (1) edge [bend left=15]  node [above] {$a$} (2)
          (2) edge [bend left=15]  node [above] {$b$} (3)
          (1) edge [bend right=45] node [below] {$b$} (3)
          (3) edge [loop above]    node [above] {$a,b$} (3);
        \end{tikzpicture}
      \end{center}
      \caption{$\L(\A) \stackrel{?}{=} \L(\mathbf{a^\star b(a+b)^\star})$.}
      \label{fig:a2}
    \end{figure}
  \end{problem}
\ExtraMaterial{
  \begin{solution}
    От \hyperref[th:regular:kleene]{теоремата на Клини} знаем, че $\L(\A) = \L(\mathbf{r}^3_{0,2})$, където:
  \begin{align*}
    \mathbf{r}^3_{0,2} & = \underbrace{\mathbf{r}^2_{0,2}}_{\bm{a^\star b}} + \underbrace{\mathbf{r}^2_{0,2}}_{\bm{a^\star b}} \cdot \underbrace{(\mathbf{r}^2_{2,2})^\star}_{\bm{(\varepsilon+a+b)^\star}} \cdot \underbrace{\mathbf{r}^2_{2,2}}_{\bm{\varepsilon+a+b}} & \comment \text{според (\ref{eq:kleene}})\\
                       & = \bm{a^\star b + a^\star b \cdot (a+b)^\star \cdot (\varepsilon+a+b)} & \comment{\text{просто заместваме}}\\
                       & = \bm{a^\star b + a^\star b \cdot (a+b)^\star} & \comment \bm{r^\star} = \bm{r^\star \cdot (\varepsilon + r)}\\
                       & = \bm{a^\star b (\varepsilon + (a+b)^\star)} & \comment\bm{r_1 + r_1\cdot r_2 = r_1 \cdot (\varepsilon+r_2)}\\
                       & = \bm{a^\star b (a+b)^\star}. & \comment \bm{r^\star} = \bm{\varepsilon + r^\star}
  \end{align*}
  Тук използвахме, че:
  \begin{align*}
    \mathbf{r}^2_{0,2} & = \underbrace{\mathbf{r}^1_{0,2}}_{\bm{b}} + \underbrace{\mathbf{r}^1_{0,1}}_{\bm{a}} \cdot \underbrace{(\mathbf{r}^1_{1,1})^\star}_{\bm{a}} \cdot \underbrace{\mathbf{r}^1_{1,2}}_{\bm{b}} & \comment \text{според (\ref{eq:kleene}}) \\
                       & = \bm{b + a \cdot a^\star \cdot b} & \comment{\text{просто заместваме}}\\
                       & = \bm{(\varepsilon + a^+)\cdot b} & \comment \mathbf{r}^+ \df \mathbf{r} \cdot \mathbf{r}^\star\\
                       & = \bm{a^\star b}. & \comment \mathbf{r}^\star = \bm{\varepsilon + r^+.г}
  \end{align*}
  Заключаваме, че $\L(\A) = \L(\mathbf{a^\star b(a+b)^\star})$.
\end{solution}
}

% \todo{Може да се сложи още един пример, но без да има решение.}

%%% Local Variables:
%%% mode: latex
%%% TeX-master: "../eai"
%%% End:

\section{Недетерминирани крайни автомати}
\index{автомат!недетерминиран}
\begin{dfn}
  \marginpar{Въведени от Рабин и Скот \cite{rabin-scott}}
  \marginpar{За по-голяма яснота, често ще означаваме с $\N$ недетерминирани автомати}
  \index{Рабин}
  \index{Скот}
  Недетерминиран краен автомат представлява
  \[\N = \NFA,\]
  \begin{itemize}
  \item
    $Q$ е крайно множество от състояния;
  \item
    $\Sigma$ е крайна азбука;
  \item
    $\Delta: Q\times\Sigma \to \Ps(Q)$ е функцията на преходите.
    \marginpar{Да напомним, че $\Ps(Q) \df \{R\mid R\subseteq Q\}$, $\abs{\Ps(Q)} = 2^{\abs{Q}}$}
    Да обърнем внимание, че е възможно за някоя двойка $(q,a)$ да няма нито един преход в автомата.
    Това е възможно, когато $\Delta(q,a) = \emptyset$;
  \item
    $\qstart \in Q$ е началното състояние;
  \item
    $F\subseteq Q$ е множеството от финални състояния.
  \end{itemize}
\end{dfn}

Удобно е да разширим функцията на преходите $\Delta: Q\times\Sigma \to \Ps(Q)$ 
до функцията $\Delta^\star: \Ps(Q)\times\Sigma^\star \to \Ps(Q)$,
която дефинираме по следния начин:
\marginpar{Обърнете внимание, че $\Delta^\star(R,a) = \bigcup_{r\in R}\Delta(r,a)$.}
\begin{itemize}
\item 
  $\Delta^\star(R, \varepsilon) = R$, за произволно $R \subseteq Q$;
\item
  $\Delta^\star(R, \alpha b) = \bigcup_{p \in \Delta^\star(R,\alpha)} \Delta(p, b)$, за произволни $b \in \Sigma$, $\alpha \in \Sigma^\star$, $q\in Q$.
\end{itemize}

\begin{framed}
  \[\L(\N) \df \{\omega \in \Sigma^\star \mid \Delta^\star(\{\qstart\},\omega) \cap F \neq \emptyset \}.\]
\end{framed}

\begin{prop}
  За всеки две думи $\alpha,\beta \in \Sigma^\star$ и всяко $R \subseteq Q$,
  \[ \Delta^\star(R, \alpha\beta) = \Delta^\star( \Delta^\star(R,\alpha),\beta).\]
\end{prop}
\begin{proof}
  \marginpar{Да напомним, че $\bigcup\{\{0,1\}, \{1,2,3\}\} = \{0,1\} \cup \{1,2,3\}$.}
  Отново индукция по дължината на $\beta$.
  \begin{itemize}
  \item
    Нека $|\beta| = 0$, т.е. $\beta = \varepsilon$. Тогава:
    \begin{align*}
      \Delta^\star(R,\alpha\varepsilon) & = \Delta^\star(R,\alpha) \\
                                        & = \Delta^\star( \Delta^\star(R,\alpha), \varepsilon). & \comment\text{деф. на }\Delta^\star
    \end{align*}
  \item
    Да приемем, че твърдението е вярно за думи $\beta$ с дължина $n$.
  \item
    Нека $|\beta| = n+1$, т.е. $\beta = \gamma b$, където $|\gamma| = n$.
    \begin{align*}
      \Delta^\star(R, \alpha\gamma b)  & = \bigcup_{p \in \Delta^\star(R,\alpha\gamma)} \Delta(p,b) & \comment\text{от деф. на }\Delta^\star\\
                                       & = \bigcup_{p \in \Delta^\star(\Delta^\star(R,\alpha),\gamma))} \Delta(p,b) & \comment\text{от И.П. за }\gamma\\
                                       & = \bigcup_{p \in \Delta^\star(U,\gamma)} \Delta(p,b) & \comment{\text{нека }U \df \Delta^\star(R,\alpha)}\\
                                       & = \Delta^\star(U,\gamma b) & \comment\text{от деф. на }\Delta^\star\\
                                       & = \Delta^\star(U,\beta) & \comment{\beta = \gamma b}\\
                                       & = \Delta^\star( \Delta^\star(R,\alpha), \beta). & \comment{\text{защото }U = \Delta^\star(R,\alpha)}
    \end{align*}
  \end{itemize}
\end{proof}

% \begin{problem}
%   Докажете, че за произволни $R_i \subseteq Q$, където $i < k$, е изпълнено, че:
%   \[\Delta^\star( \bigcup_{i<k} R_i, \alpha) = \bigcup_{i<k} \Delta^\star( R_i, \alpha).\]
% \end{problem}

И тук е удобно да въведем бинарната релация $\vdash_\N$ над $Q\times\Sigma^\star$,
която ще ни казва как моментното описание на автомата $\N$ се променя след изпълнение на една стъпка:
\[(q,a\beta) \vdash_\N (p,\beta), \text{ ако } p \in \Delta(q,a).\]
\marginpar{Рефл. и транз. затваряне на една релация е разгледано в Глава \ref{ch:intro}}
Рефлексивното и транзитивно затваряне на $\vdash_\N$ ще означаваме с $\vdash^\star_\N$.
За да дадем по-ясна дефиниция на $\vdash^\star_\N$, първо ще дефинираме релацията $\vdash^n_\N$, която
определя работата на автомата $\N$ за $n$ на брой стъпки.
\begin{itemize}
\item 
  $(q,\alpha) \vdash^0_\N (q,\alpha)$, защото за $0$ стъпки се случва нищо.
\item
  Нека $\Delta(q,x) \ni q'$ и $(q',\alpha) \vdash^n_\N (p, \beta)$. Тогава
  $(q,x\alpha) \vdash^{n+1}_\N (p,\beta)$, защото за $n+1$ на брой стъпки първо правим една стъпка 
  и отиваме в моментното описание $(q',\alpha)$ и след това правим още $n$ на брой стъпки.
\end{itemize}
Сега можем да дефинираме $\vdash^\star_\N$ като:
\[(q,\alpha) \vdash^\star_\N (p,\beta) \dff (\exists n\in\Nat)[(q,\alpha) \vdash^n_\N (p,\beta)].\]
Друг начин да дефинираме релацията $\vdash^\star_\N$ е следния:
\[(q,\alpha\beta) \vdash^\star_\N (p, \beta) \iff p \in \Delta^\star(\{q\},\alpha).\]
Получаваме, че 
\[\L(\N) = \{\alpha\in\Sigma^\star \mid (\exists q \in F)[(\qstart,\alpha) \vdash^\star_\N (q,\varepsilon)]\}.\]


\begin{framed}
\begin{thm}[Рабин-Скот \cite{rabin-scott}]
  За всеки недетерминиран краен автомат $\N$ съществува еквивалентен на него детерминиран краен автомат $\D$, т.е. $\L(\N) = \L(\D)$.
\end{thm}
\end{framed}
\begin{hint}
  Нека $\N = \NFA$. Ще построим детерминиран автомат
  \[\D = (Q',\Sigma,\delta,\qstart',F'),\]
  за който $\L(\N) = \L(\D)$.
  Конструкцията е следната:
  \marginpar{Да отбележим, че детерминираният автомат $\D$ има не повече от $2^{\abs{Q}}$ на брой състояния $Q'$}
  \begin{itemize}
  \item
    $Q' = \Ps(Q)$. Някои от тези състояния може да са недостижими и следователно да са излишни, но в общия случай трябва да имаме
    всички подмножества на $Q$.
  \item
    За произволна буква $a\in\Sigma$ и произволно $R \subseteq Q$,
    \begin{align*}
      \delta(R,a) & = \{q\in Q\mid (\exists r\in R)[q\in\Delta(r,a)]\}\\
                  & = \bigcup_{r\in R}\Delta(r,a).
    \end{align*}
  \item
    $\qstart' = \{\qstart\}$;
  \item
    $F' = \{R \subseteq Q \mid R\cap F \neq \emptyset\}$.
  \end{itemize}
  Ще докажем с индукция по дължината на думата $\alpha$, че
  \begin{equation}
    \label{eq:6}
    (\forall \alpha\in\Sigma^\star)(\forall R \subseteq Q)[\ \Delta^\star(R,\alpha) = \delta^\star(R,\alpha)\ ].
  \end{equation}

  \begin{itemize}
  \item
    Ако $|\alpha| = 0$, т.е. $\alpha = \varepsilon$, то е ясно от дефиницията на $\Delta^\star$ и $\delta^\star$.
  \item
    Да приемем, че (\ref{eq:6}) е изпълнено за думи $\alpha$ с дължина $n$.
  \item
    Нека сега $\alpha$ има дължина $n+1$, т.е. $\alpha = \beta a$, където $|\beta| =n$ и $a \in \Sigma$.
    Тогава:
    \begin{align*}
      \Delta^\star(R, \beta a) & = \bigcup_{q\in\Delta^\star(R,\beta)}\Delta(q, a) & \comment\text{деф. на }\Delta^\star\\
                               & = \bigcup_{q\in\delta^\star(R,\beta)}\Delta(q, a) & \comment\text{от И.П.}\\
                               & = \delta( \delta^\star(R, \beta), a) & \comment\text{деф. на }\delta\\
                               & = \delta^\star( R, \beta a) & \comment\text{деф. на }\delta^\star
    \end{align*}
  \end{itemize}
  Сега вече е лесно да съобразим, че
  \begin{align*}
    \omega \in \L(\D) & \iff \delta^\star(\{\qstart\},\omega) \in F' & \comment\text{деф. на }\L(\D)\\
                      & \iff \delta^\star(\{\qstart\},\omega) \cap F \neq \emptyset & \comment\text{деф. на }F'\\
                      & \iff \Delta^\star(\{\qstart\},\omega) \cap F \neq \emptyset & \comment\text{от (\ref{eq:6})}\\
                      & \iff \omega \in \L(\N) & \comment\text{деф. на }\L(\N).
  \end{align*}
\end{hint}

\begin{problem}
  За всеки краен недетерминиран автомат $\N$ съществува краен недетерминиран автомат $\N'$ с едно финално състояние, 
  за който $\L(\N) = \L(\N')$.
\end{problem}
\begin{hint}
  Вместо формална конструкция, да разгледаме един пример, който илюстрира идеята.
  \begin{figure}[H]
    \begin{subfigure}[b]{0.2\textwidth}
      \begin{tikzpicture}[framed,->,>=stealth,thick,node distance=45pt]
        \tikzstyle{every state}=[circle,minimum size=20pt,auto]
        \node[initial below,state]      (1) {$q_0$};
        \node[state,accepting]     [above right of=1] (2) {$q_1$};
        \node[state,accepting]     [below right of=1] (3) {$q_2$};
        \path
        (1) edge [bend left=15] node  [above] {$a$} (2)
        (2) edge [bend left=15] node  [right] {$b$} (1)
        (2) edge [bend left=15] node  [right] {$a$} (3)
        (3) edge [bend left=15] node  [below] {$a$} (1)
        (3) edge [loop below] node  [right] {$b$} (3);
      \end{tikzpicture}
      \caption{автомат $\N$}
    \end{subfigure}
    \hspace*{1.4cm}
    \begin{subfigure}[b]{0.5\textwidth}
      \begin{tikzpicture}[framed,->,>=stealth,thick,node distance=45pt]
        \tikzstyle{every state}=[circle,minimum size=20pt,auto]
        \node[initial below,state]                    (1) {$q_0$};
        \node[state]               [above right of=1] (2) {$q_1$};
        \node[state]               [below right of=1] (3) {$q_2$};
        \node[state,accepting]     [right=5cm of 1]   (4) {$f$};

        \path
        (1) edge [bend left=15] node  [above] {$a$} (2)
        (2) edge [bend left=15] node  [right] {$b$} (1)
        (2) edge [bend left=15] node  [right] {$a$} (3)
        (3) edge [loop below]   node  [right] {$b$} (3)
        (3) edge [bend left=15] node  [below] {$a$} (1)
        (1) edge [dashed,bend left=15] node  [above] {$a$} (4)
        (2) edge [dashed,bend left=15] node  [above] {$a$} (4)
        (3) edge [dashed,bend right=15] node  [below] {$b$} (4);
      \end{tikzpicture}
    \caption{автомат $\N'$, за който $\L(\N') = \L(\N)$}
  \end{subfigure}
\end{figure}  
За произволен автомат $\N$, формулирайте точно конструкцията на $\N'$ с едно финално състояние и докажете, че наистина $\L(\N) = \L(\N')$.
Обърнете внимание, че примерът показва, че е възможно $\N$ да е детерминиран автомат, но полученият $\N'$ да бъде недетерминиран.
\end{hint}

\begin{problem}
  \marginpar{По-късно ще видим, че можем да дадем и друго доказателство на това твърдение, като направим индукция по построението на регулярните езици.}
  Докажете, че автоматните езици са затворени относно операцията $\texttt{rev}$.
  С други думи, докажете, че ако $L$ е автоматен език, то
  \[L^{\texttt{rev}} = \{\omega^{\texttt{rev}} \mid \omega \in L\}\]
  също е автоматен.
\end{problem}

\begin{example}
  \begin{figure}[H]
    \begin{tikzpicture}[framed,->,>=stealth,thick,node distance=45pt]
      \tikzstyle{every state}=[circle,minimum size=20pt,auto]
      \node[initial below,state]              (0) {$q_0$};
      \node[state]                [right of=0] (1) {$q_1$};
      \node[state]                [right of=1] (2) {$q_2$};
      \node[state]                [right of=2] (3) {$q_3$};
      \coordinate[right of=3] (4);
      \coordinate[right of=4] (5);
      \node[state,accepting]      [right of=5] (6) {$q_n$};
      \path
      (0) edge [loop above] node [above] {$a,b$} (0)
      (0) edge [bend left=15] node  [above] {$a$} (1)
      (1) edge [bend left=15] node  [above] {$a,b$} (2)
      (2) edge [bend left=15] node  [above] {$a,b$} (3);

      \draw [dashed,->,shorten >=0pt] (3) to[bend left=15] node[above] {$a,b$} (4);
      \draw [dashed,->,shorten >=0pt] (5) to[bend left=15] node[above] {$a,b$} (6);
    \end{tikzpicture}
    \caption{автомат $\N$}
  \end{figure}

  Лесно се съобразява, че за $n > 0$, крайният недетерминиран автомат $\N$ разпознава езика
  \[L = \{\alpha \in \{a,b\}^\star \mid \alpha = \beta \cdot a \cdot \gamma\ \&\ |\gamma| = n-1\}.\]
  Нека да съобразим, че не е възможно да има краен детерминиран автомат $\A = \DFA$ разпознаващ езика $L$ с по-малко от $2^n$ състояния.

  Да допуснем, че $|Q| < 2^n$. От принципа на Дирихле имаме, че съществуват две различни думи $\alpha$ и $\beta$ с дължина $n$,
  за които
  \[\delta^\star(\qstart,\alpha) = q = \delta^\star(\qstart,\beta).\]

  Да видим къде е първата разлика на позиция $i$ в тези две думи.
  \begin{itemize}
  \item
    Ако $i = 0$, то нека без ограничение на общността да имаме, че $\alpha[0] = a$ и $\beta[0] = b$.
    Това означава, че $\alpha \in L$, но $\beta \not\in L$.
    Следователно състоянието $q$ е едновременно финално и нефинално. Това е противоречие.
  \item
    Ако $i > 0$, то нека без ограничение на общността да имаме, че $\alpha[i] = a$ и $\beta[i] = b$.
    Да разгледаме следните думи:
    \begin{align*}
      & \alpha_0 = \alpha \cdot a^{i}\\
      & \beta_0 = \beta \cdot a^{i}.
    \end{align*}
    Така отново получаваме, че $\alpha_0 \in L$, но $\beta_0 \not\in L$.
    Получаваме противоречие, защото
    \[\delta^\star(\qstart,\alpha_0) = p = \delta^\star(\qstart,\beta_0)\]
    и състоянието $p$ трябва да е едновременно финално и нефинално.
  \end{itemize}  
\end{example}


\begin{lemma}
  \label{lem:automata-basic}
  Съществува краен детерминиран автомат $\A = \FA$, който разпознава езика $L$, където
  \begin{itemize}
  \item
    $L = \emptyset$,
  \item
    $L = \{\varepsilon\}$, или
  \item
    $L = \{a\}$, за произволна буква $a\in\Sigma$.
  \end{itemize}
\end{lemma}
\begin{hint}
  \begin{figure}[H]
    \begin{subfigure}[b]{0.2\textwidth}
      \label{subf:a1}
      \begin{tikzpicture}[->,>=stealth,thick,node distance=35pt]
        \tikzstyle{every state}=[circle,minimum size=15pt,auto]
        \node[initial below,state]      (1) {$q_0$};
      \end{tikzpicture}
      \caption{$L(\A) = \emptyset$}
    \end{subfigure}
    \qquad
    \begin{subfigure}[b]{0.2\textwidth}
      \begin{tikzpicture}[->,>=stealth,thick,node distance=35pt]
        \tikzstyle{every state}=[circle,minimum size=15pt,auto]
        \node[initial below,state,accepting]      (1) {$q_0$};
      \end{tikzpicture}
      \caption{$\L(\A) = \{\varepsilon\}$}
    \end{subfigure}
    \qquad
    \begin{subfigure}[b]{0.2\textwidth}
      \begin{tikzpicture}[->,>=stealth,thick,node distance=45pt]
        \tikzstyle{every state}=[circle,minimum size=15pt,auto]
        \node[initial below,state]      (1)              {$q_0$};
        \node[accepting,state]    (2) [right of=1] {$q_1$};
        \path 
        (1) edge  node [above] {$a$} (2);
      \end{tikzpicture}
      \caption{$\L(\A) = \{a\}$}
    \end{subfigure}
  \end{figure}
\end{hint}

\begin{framed}
  \begin{lemma}
    \label{lem:concat}
    Класът на автоматните езици е затворен относно операцията {\em конкатенация}.
    Това означава, че ако $L_1$ и $L_2$ са два произволни автоматни езика, то $L_1\cdot L_2$
    също е автоматен език.
  \end{lemma}  
\end{framed}
\begin{proof}
  \marginpar{Тук отново приемаме, че $Q_1 \cap Q_2 = \emptyset$.}
  Нека са дадени крайните детерминирани автомати:
  \begin{itemize}
  \item
    $\A_1 = \pair{\Sigma,Q_1,\delta_1,\qstart',F_1}$, където $\L(\A_1) = L_1$;
  \item
    $\A_2 = \pair{\Sigma,Q_2,\delta_2,\qstart'', F_2}$, където $\L(\A_2) = L_2$.
  \end{itemize}
  Ще дефинираме автомата $\N = \NFA$ по такъв начин, че
  \[\L(\N) = L_1\cdot L_2 = \L(\A_1)\cdot\L(\A_2).\]
  \begin{itemize}
  \item
    $Q \df Q_1 \cup Q_2$;
  \item
    $\qstart \df \qstart'$;
  \item
    $F \df \begin{cases}
      F_1 \cup F_2, & \text{ ако } \qstart'' \in F_2\\
      F_2,          & \text{ иначе}.
    \end{cases}$
  \item 
    $\Delta(q,a) \df
    \begin{cases}
      \{\delta_1(q,a)\},                      & \text{ ако }q\in Q_1\setminus F_1\ \&\ a\in\Sigma\\
      \{\delta_1(q,a), \delta_2(\qstart'',a)\}, & \text{ ако }q \in F_1\ \&\ a\in\Sigma\\
      \{\delta_2(q,a)\},                      & \text{ ако }q\in Q_2\ \&\ a\in\Sigma.
    \end{cases}$
  \end{itemize}
  Първо ще докажем, че
  \[\L(\A_1)\cdot\L(\A_2) \subseteq \L(\N).\]
  За целта, нека разгледаме думата $\alpha \in \L(\A_1)$ и $\beta \in \L(\A_2)$. Това означава, че имаме следните изчисления:
  \begin{align*}
    & (\qstart',\alpha) \vdash^\star_{\A_1} (q_1,\varepsilon), \text{за някое }q_1 \in F_1\\
    & (\qstart'',\beta) \vdash^\star_{\A_2} (q_2,\varepsilon), \text{за някое }q_2 \in F_2.
  \end{align*}
  Според дефиницията на крайния недетерминиран автомат $\N$ е ясно, че:
  \[(\qstart',\alpha) \vdash^\star_{\N} (q_1,\varepsilon), \text{за някое }q_1 \in F_1.\]
  % \begin{align*}
  %   & (\qstart',\alpha) \vdash^\star_{\N} (q_1,\varepsilon), \text{за някое }q_1 \in F_1\\
  %   & (\qstart'',\beta) \vdash^\star_{\N} (q_2,\varepsilon), \text{за някое }q_2 \in F_2.
  % \end{align*}
  
  Ако $\beta = \varepsilon$, то това означава, че $\qstart'' \in F_2$ и следователно $F_1 \subseteq F$.
  Тогава получаваме, че $\alpha \cdot \beta = \alpha \in \L(\N)$, защото
  \[(\qstart',\alpha) \vdash^\star_{\N} (q_1,\varepsilon), \text{за някое }q_1 \in F_1 \subseteq F.\]

  Ако $\beta = b\gamma$, за някоя дума $\gamma \in \Sigma^\star$, то тогава можем да разбием изчислението на $\beta$ в $\A_2$ по следния начин:
  \[(\qstart'',b\gamma) \vdash_{\A_2} (q,\gamma) \vdash^\star_{\A_2} (q_2,\varepsilon), \text{за някое }q_2 \in F_2,\]
  където $q = \delta_2(\qstart'',b)$.
  
  Според дефиницията на крайния недетерминиран автомат $\N$ е ясно, че:
  \[(q,\gamma) \vdash^\star_{\N} (q_2,\varepsilon), \text{за някое }q_2 \in F_2.\]
  Освен това, имаме, че $q \in \Delta(q_1,b)$, защото $q_1 \in F_1$. Това означава, че
  \[(q_1,c\gamma) \vdash_\N (q,\gamma).\]
  Съединявайки последните две изчисления, получаваме, че:
  \[(q_1,\beta) \vdash^\star_\N (q_2,\varepsilon),\text{ за някое }q_2 \in F_2.\]
  Сега съединяваме и изчислението за $\alpha$ и получаваме, че:
  \[(\qstart',\alpha\beta) \vdash^\star_\N (q_1,\beta) \vdash^\star_\N (q_2,\varepsilon),\text{ за някое }q_2 \in F_2.\]
  Оттук заключаваме, че във всички случаи за $\beta$, $\alpha \cdot \beta \in \L(\N)$.


  % където $\alpha = a_0a_1\cdots a_{n-1}$, и нека редицата $(q_i)^n_{i=0}$ описва приемащото изчисление на $\A_1$ върху думата $\alpha$.
  % Това означава, че:
  % \begin{itemize}
  % \item
  %   $q_0 = \qstart'$;
  % \item
  %   $q_{i+1} = \delta_1(q_i,a_i)$ за $i < n$;
  % \item
  %   $q_n \in F_1$.
  % \end{itemize}  
  % Също така, нека разгледаме думата $\beta \in \L(\A_2)$, където $\beta = b_0b_1\cdots b_{m-1}$, и нека редицата $(p_i)^m_{i=0}$ описва приемащото изчисление на $\A_2$ върху думата $\beta$.
  % Това означава, че:
  % \begin{itemize}
  % \item
  %   $p_0 = \qstart''$;
  % \item
  %   $p_{i+1} = \delta_2(p_i,b_i)$ за $i < m$;
  % \item
  %   $p_m \in F_2$.
  % \end{itemize}  
  % От конструкцията на $\N$ се вижда лесно, че $(q_i)^n_{i=0}$ описва изчисление на $\N$ върху $\alpha$ и
  % $(p_i)^{m}_{i=0}$ описва изчисление на $\N$ върху $\beta$.
  % Това означава, че:
  % \begin{itemize}
  % \item
  %   $q_{i+1} \in \Delta(q_i,a_i)$ за $i < n$;
  % \item
  %   $p_{i+1} \in \Delta(p_i,b_i)$ за $i < m$.
  % \end{itemize}
  % Тогава:
  % \[(q_0,\alpha\beta) \vdash^\star_\N (q_n,\beta)\text{ и } (p_0, \underbrace{b_0b_1\cdots b_{m-1}}_{\beta}) \vdash_\N (p_1,b_1\cdots b_{m-1}) \vdash^\star_{\N} (p_m,\varepsilon).\]
  % Понеже $q_n \in F_1$, а $p_0 = \qstart''$, то от конструкцията на $\N$ следва, че
  % \[(q_n, \underbrace{b_0b_1\cdots b_{m-1}}_{\beta}) \vdash_\N (p_1,b_1\cdots b_{m-1}),\]
  % защото $\delta_2(\qstart'',b_0) \in \Delta(q_n,b_0)$.
  
  % Обединявайки всичко това, получаваме, че:
  % \begin{align*}
  %   (\qstart, \alpha\beta) & \vdash^\star_\N (q_n,\beta)\\
  %                          & \vdash_\N (p_1,b_1\cdots b_{m-1})\\
  %                          & \vdash^\star_\N(p_m,\varepsilon).
  % \end{align*}
  % Понеже $p_m \in F_2$, то $\alpha\beta \in \L(\N)$.
  

  Сега ще докажем, че
  \[\L(\N) \subseteq \L(\A_1) \cdot \L(\A_2).\]
  За целта, нека разгледаме думата $\omega \in \L(\N)$, където $\omega = a_0a_1\cdots a_{n-1}$.
  Да разгледаме редицата от състояния $(q_i)^{n}_{i=0}$, която описва едно приемащо изчисление на $\N$ върху $\omega$.
  \marginpar{Възможно е да има и други редици от състояния $(p_i)^{n}_{i=0}$, които да описват приемащи изчисления на $\N$ върху $\omega$.}
  Това означава, че:
  \begin{itemize}
  \item
    $q_0 = \qstart$;
  \item
    $q_{i+1} \in \Delta(q_i,a_i)$ за $i < n$;
  \item
    $q_n \in F$.
  \end{itemize}
  
  Ако $q_n \in F_1$, то според конструкцията на $\N$, $\varepsilon \in \L(\A_2)$ и всяко състояние от $(q_i)^{n}_{i=0}$ принадлежи на $Q_1$ и оттам $\omega \in \L(\A_1)$.
  Интересният случай е когато $q_n \in F_2$.
  Според конструкцията на $\N$, не можем да преминем от състояние от $Q_2$ в състояние от $Q_1$.
  Това означава, че можем да разбием редицата от състояния $(q_i)^n_{i=0}$ на две непразни подредици:
  \begin{itemize}
  \item
    $(q_{i})^{l}_{i=0}$ - тези които са от $Q_1$,
  \item
    $(q_i)^{n}_{i=l+1}$ - тези, които са от $Q_2$.
  \end{itemize}
  Нека $\alpha = a_0a_1\cdots a_{l-1}$ и $\beta = a_la_{l+1}\cdots a_{n-1}$.
  Ясно е, че:
  \[(q_0,\alpha\beta) \vdash^\star_\N (q_{l}, \underbrace{a_{l}a_{l+1}\cdots a_{n-1}}_{\beta}) \vdash_\N (q_{l+1},a_{l+1}\cdots a_{n-1}) \vdash^\star_\N (q_n,\varepsilon).\]
  От конструкцията на $\N$ следва, че редицата от състояния $(q_i)^{l}_{i=0}$ описва изчислението на $\A_1$ върху $\alpha$.
  \marginpar{Това е единственият начин да направим преход от състояние на $Q_1$ към състояние на $Q_2$.}
  Също така от конструкцията следва, че щом $q_{l+1} \in \Delta(q_l,a_l)$, то $q_l \in F_1$ и $\delta_2(\qstart'',a_l) = q_{l+1}$. Заключаваме, че:
  \begin{itemize}
  \item
    $(q_0, \alpha) \vdash^\star_{\A_1} (q_l,\varepsilon)$.
    Понеже $q_0 = \qstart'$ и $q_l \in F_1$, то $\alpha \in \L(\A_1)$.
  \item
    $(\qstart'', \beta) \vdash^\star_{\A_2} (q_n,\varepsilon)$.
    Понеже $q_n \in F_2$, то $\beta \in \L(\A_2)$.
  \end{itemize}
\end{proof}

\begin{figure}[H]
  \center
  \begin{subfigure}[b]{0.3\textwidth}
    \label{subf:a1}
    \begin{tikzpicture}[framed,->,>=stealth,thick,node distance=45pt]
      \tikzstyle{every state}=[circle,minimum size=15pt,auto]
      \node[initial,state,accepting]      (1) {$q'_0$};
      \node[state]                        (2) [right of=1] {$q_1$};
      \node[state]                        (3) [above right of=2] {$q_2$};
      \node[state,accepting]              (4) [below right of=2] {$q_3$};
      \path
      (1) edge [loop above] node [above] {$b$} (1)
      (1) edge node [above] {$a$} (2)
      (2) edge node [above] {$a$} (3)
      (3) edge [loop right] node [right] {$b$} (3)
      (2) edge node [below] {$b$} (4)
      (3) edge [bend right=30] node [above] {$a$} (1)
      (4) edge [bend right=15] node [right] {$a$} (3)
      (4) edge [bend left=30] node [below] {$b$} (1);
    \end{tikzpicture}
    \caption{автомат $\A_1$}
  \end{subfigure}
  \qquad
  \qquad
  \qquad
  \begin{subfigure}[b]{0.3\textwidth}
    \begin{tikzpicture}[framed,->,>=stealth,thick,node distance=45pt]
      \tikzstyle{every state}=[circle,minimum size=15pt,auto]
      \node[initial,state]                (1) {$q''_0$};
      \node[state]     [above right of=1] (2) {$q_4$};
      \node[state,accepting]     [below right of=1] (3) {$q_5$};
      \path
      (1) edge [bend left=15] node  [above] {$a$} (2)
      (2) edge [loop above] node [above] {$b$} (2)
      (2) edge [bend left=15] node  [right] {$a$} (3)
      (3) edge [loop right]  node [right] {$a,b$} (3)
      (1) edge [bend right=15] node [below] {$b$} (3);
    \end{tikzpicture}
    \caption{автомат $\A_2$}
  \end{subfigure}
\end{figure}

\begin{example}
    За да построим автомат, който разпознава конкатенацията на $\L(\A_1)$ и $\L(\A_2)$,
    трябва да свържем финалните състояния на $\A_1$ с изходящите от $s_2$ състояния на $\A_2$.
    
    \begin{figure}[H]
      \center
      % \begin{subfigure}[b]{0.3\textwidth}
      \begin{tikzpicture}[framed,->,>=stealth,thick,node distance=2cm]
        \tikzstyle{every state}=[circle,minimum size=15pt,auto]
        \node[initial,state]                      (1) {$q'_0$};
        \node[state] [right of=1]                 (2) {$q_1$};
        \node[state] [above right of=2]           (3) {$q_2$};
        \node[state] [below right of=2]           (4) {$q_3$};
        \node[state] [right=4cm of 1]             (5) {$q''_0$};
        \node[state] [above right of=5]           (6) {$q_4$};
        \node[state,accepting] [below right of=5] (7) {$q_5$};
        \path
        (1) edge [loop above] node [above] {$b$} (1)
        (1) edge node [above]                         {$a$} (2)
        (2) edge node [above]                         {$a$} (3)
        (2) edge node [below]                         {$b$} (4)
        (3) edge [loop right] node [right]            {$b$} (3)
        (6) edge [loop above] node [above]            {$b$} (6)
        (7) edge [loop right] node [right]            {$a,b$} (7)
        (3) edge [bend right=15] node [above]         {$a$} (1)
        (4) edge [bend left=15] node [below]          {$b$} (1)
        (4) edge [bend left=15] node [left]          {$a$} (3)
        (5) edge [bend left=15] node [below]          {$a$} (6)
        (6) edge [bend left=15] node [right]          {$a$} (7)
        (5) edge [bend right=15] node [above]         {$b$} (7)
        (1) edge [dashed, bend left=45] node [above]  {$a$} (6)
        (1) edge [dashed, bend right=45] node [below] {$b$} (7)
        (4) edge [dashed, bend left=30] node [above]  {$a$} (6)
        (4) edge [dashed, bend left=10] node [above]  {$b$} (7);
      \end{tikzpicture}
      \caption{$\L(\N) = \L(\A_1)\cdot\L(\A_2)$}
  \end{figure}  
  Обърнете внимание, че $\A_1$ и $\A_2$ са детерминирани автомати, но $\N$ е недетерминиран.
  Също така, в този пример се оказва, че вече $q''_0$ е недостижимо състояние, но в общия случай не можем да 
  го премахнем, защото може да има преходи влизащи в $q''_0$.
\end{example}

\begin{lemma}
  \label{lem:union}
  \marginpar{Второ доказателство на това твърдение.}
  Класът от автоматните езици е затворен относно операцията {\em обединение}.
\end{lemma}
\begin{hint}
  Нека са дадени детерминистичните автомати:
  \begin{itemize}
  \item 
    $\A_1 = \pair{\Sigma,Q_1,\delta_1,\qstart',F_1}$, като $L(\A_1) = L_1$;
  \item
    $\A_2=\pair{\Sigma,Q_2,\delta_2,\qstart'',F_2}$, като $L(\A_2) = L_2$.
  \end{itemize}
  Ще дефинираме автомата $\N=\NFA$, така че
  \[L(\N) = L(\A_1) \cup L(\A_2).\]
  \begin{itemize}
  \item 
    $Q \df Q_1 \cup Q_2 \cup \{\qstart\}$, където $\qstart\not\in Q_1\cup Q_2$;
  \item
    $F \df 
    \begin{cases}
      F_1 \cup F_2 \cup \{\qstart\}, & \text{ ако } \qstart' \in F_1 \vee \qstart'' \in F_2\\
      F_1 \cup F_2,            & \text{ иначе } 
    \end{cases}$
  \item
    $\Delta(q,a) \df
    \begin{cases}
      \{\delta_1(q,a)\},                       & \text{ ако } q\in Q_1\ \&\ a\in\Sigma\\
      \{\delta_2(q,a)\},                       & \text{ ако } q\in Q_2\ \&\  a\in\Sigma\\
      \{\delta_1(\qstart',a), \delta_2(\qstart'',a)\}, & \text{ ако } q = \qstart\ \&\ a \in\Sigma.
    \end{cases}$
  \end{itemize}
\end{hint}

\begin{example}
    За да построим автомат, който разпознава обединението на $\L(\A_1)$ и $\L(\A_2)$,
    трябва да добавим ново начално състояние, което да свържем с наследниците на началните състояния на $\A_1$ и $\A_2$.
    
    \begin{figure}[H]
      \center
      % \begin{subfigure}[b]{0.3\textwidth}
      \begin{tikzpicture}[framed,->,>=stealth,thick,node distance=2cm]
        \tikzstyle{every state}=[circle,minimum size=20pt,auto]
        \node[initial,state,accepting]      (0) {$q_0$};
        \node[state,accepting] [above right of=0]        (1) {$q'_0$};
        \node[state]    [right of=1]        (2) {$q_1$};
        \node[state]                        (3) [above right of=2] {$q_2$};
        \node[state,accepting]                        (4) [below right of=2] {$q_3$};
        \node[state]    [below right=2cm of 0] (5) {$q''_0$};
        \node[state]     [above right of=5] (6) {$q_4$};
        \node[state,accepting]     [below right of=5] (7) {$q_5$};
        \path
        (1) edge [loop above] node [above] {$b$} (1)
        (1) edge node [above]                  {$a$} (2)
        (2) edge node [above]                  {$a$} (3)
        (2) edge node [below]                  {$b$} (4)
        (3) edge [bend right=15] node [above]  {$a,b$} (1)
        (4) edge [bend left=15]  node [below]  {$b$} (1)
        (4) edge [bend right=15]  node [right]  {$a$} (3)
        (5) edge [bend left=15] node [below]   {$a$} (6)
        (6) edge [bend left=15] node  [right] {$a,b$} (7)
        (5) edge [bend right=15] node [above]  {$b$} (7)
        (7) edge [loop right] node [right] {$a,b$} (7)
        (0) edge [dashed, bend right=15] node [below]  {$a$} (2)
        (0) edge [dashed, bend left=15] node [above]  {$b$} (1)
        (0) edge [dashed, bend right=15] node [below]  {$a$} (6)
        (0) edge [dashed, bend right=45] node [below]  {$b$} (7);
      \end{tikzpicture}
      \caption{$\L(\N) = \L(\A_1)\cup\L(\A_2)$}
  \end{figure}  
  Обърнете внимание, че $\A_1$ и $\A_2$ са детерминирани автомати, но $\N$ е недетерминиран.
  Освен това, новото състояние $q_0$ трябва да бъде маркирано като финално, защото $q'_0$ е финално.
\end{example}

\begin{framed}
  \begin{lemma}
    \label{lem:kleene-star}
    Класът от автоматните езици е затворен относно операцията {\em звезда на Клини}, т.е.
    за всеки автоматен език $L$, езикът $L^\star$ също е автоматен.
  \end{lemma}  
\end{framed}
\begin{proof}
  Да разгледаме крайния детерминиран автомат
  \[\A = \pair{\Sigma,Q,\qstart,\delta,F}.\]
  Първо ще построим краен недетерминиран автомат
  \[\N = \pair{\Sigma, Q, \qstart, \Delta, F},\] такъв че
  \[\L(\N) = (\L(\A))^+.\]
  После ще построим краен недетерминиран автомат $\N'$, за който
  \[\L(\N') = \L(\N) \cup \{\varepsilon\} = (\L(\A))^\star.\]

  Дефинираме функцията на преходите $\Delta$ на $\N$ като за $q \in Q$ и $a \in \Sigma$ определяме функцията на преходите $\Delta$ по следния начин:
  \begin{align*}
    \Delta(q,a) \df
    \begin{cases}
      \{\delta(q,a)\}, & \text{ако } q \not\in F\\
      \{\delta(q,a), \delta(\qstart,a) \}, & \text{ако } q \in F.
    \end{cases}
  \end{align*}
    
  Нека $\alpha = a_0a_1\cdots a_{n-1} \in \L(\N)$.
  Това означава, че $(\qstart,\alpha) \vdash^\star_\N (f,\varepsilon)$ за някое $f \in F$.
  Нека редицата от състояния $(q_i)^n_{i=0}$ описва едно приемащо изчисление на $\N$ върху $\alpha$, т.е.
  \begin{itemize}
  \item
    $q_0 = \qstart$;
  \item
    $q_{i+1} \in \Delta(q_i,a_i)$;
  \item
    $q_n \in F$.
  \end{itemize}
  Да разгледаме максималната подпоследователност от състояния $(q_{i_j})^{l+1}_{j = 0}$ на $(q_i)^{n}_{i=0}$ съставена от тези състояния, за които
  \marginpar{Тук е малко по-сложно, защото правим разбиване на изчислението не на база всички финални състояния, а на тези финални състояния, от които изчислението продължава в насленик на $\qstart$.}
  \begin{itemize}
  \item
    $q_{i_0} = \qstart$;
  \item
    $q_{i_j} \in F\ \&\ \delta(\qstart,a_{i_j}) = q_{i_j+1}$, за $j = 1,\dots,l$;
  \item
    $q_{i_{l+1}} = q_n$.
  \end{itemize}
  Да разбием думата $\alpha$ като $\alpha = \alpha_0\alpha_1\cdots\alpha_l$, където:
  \begin{align*}
    & \alpha_0 \df a_{i_0} \alpha'_0\\
    & \alpha_1 \df a_{i_1}\alpha'_1\\
    & \cdots\\
    & \alpha_l \df a_{i_l}\alpha'_l.
  \end{align*}
  Сега можем да разбием изчислението на $\N$ върху $\alpha$ по следния начин:
  \begin{align*}
    (q_{i_0},\alpha_{0}\alpha_{1}\cdots \alpha_{l}) & \vdash^\star_\N (q_{i_1}, \alpha_{1}\cdots \alpha_{l}) & \comment{ q_{i_0} = \qstart}\\
                                                    & \vdash^\star_\N (q_{i_2},\alpha_{2}\cdots \alpha_{l})\\
                                                    & \cdots\\
                                                    & \vdash^\star_\N (q_{i_{l}}, \alpha_{l})\\
                                                    & \vdash^\star_\N(q_{i_{l+1}},\varepsilon). & \comment{ q_{i_{l+1}} = q_n \in F}
  \end{align*}
  Да разгледаме само първата част на изчислението:
  \[\underbrace{(q_{i_0},\alpha_0) \vdash^\star_\N (q_{i_1}, \varepsilon)}_{\text{само преходи от }\A}.\]
  Понеже $q_{i_0} = \qstart$ и $q_{i_1} \in F$, то е ясно, че $\alpha_0 \in \L(\A)$.
  
  За $j = 0,\dots,l$, изчислението
  \[(q_{i_j},\alpha_j) \vdash^\star_\N (q_{i_{j+1}},\varepsilon)\]
  може по-подробно да се запише и така:
  \[(q_{i_j}, a_{i_{j}}\alpha'_j) \vdash_\N \underbrace{(q_{i_j+1}, \alpha'_j)  \vdash^\star_\N (q_{i_{j+1}},\varepsilon)}_{\text{само преходи от }\A}.\]
  Понеже имаме, че $\delta(\qstart,a_{i_j}) = q_{i_j+1}$, то оттук следва, че:
  \[\underbrace{(\qstart,a_{i_j}\alpha'_{j}) \vdash_\N (q_{i_j+1}, \alpha'_{j})}_{\text{преход от }\A}  \vdash^\star_\A (q_{i_{j+1}},\varepsilon).\]
  Заключаваме, че
  \[(\qstart,\alpha_{j}) \vdash^\star_\A (q_{i_{j+1}},\varepsilon).\]
  Понеже $q_{i_{j+1}} \in F$, веднага следва, че $\alpha_{j} \in \L(\A)$.
  От всичко дотук заключваме, че $\alpha \in (\L(\A))^+$.

  За другата посока, нека $\alpha \in (\L(\A))^+$.
  Това означава, че думата $\alpha$ може да се представи като
  $\alpha = \alpha_0 \cdot \alpha_1 \cdots \alpha_l$, където $\alpha_j \in \L(\A)$ и $\alpha_j \neq \varepsilon$, за $j = 0,\dots, l$.
  Нека за $j=0,\dots,l$ да положим
  \[\alpha_j \df a_j \cdot \alpha'_j\text{ и } q_{j} \df \delta(\qstart, a_j).\]
  Оттук получаваме за $j = 0,\dots,l$ следните изчисления:
  \[(\qstart, \alpha_{j}) \vdash_\A (q_{j}, \alpha'_j) \vdash^\star_\A (f_{j}, \varepsilon), \text{ за някои }f_j \in F.\]
  Понеже функцията на преходите на $\N$ разширява функцията на преходите на $\A$, то е ясно е, че имаме също така и следното:
  \[(\qstart, \alpha_{j}) \vdash_\N (q_{j}, \alpha'_j) \vdash^\star_\N (f_{j}, \varepsilon), \text{ за някои }f_j \in F.\]

  За $1 \leq j < l$,
  понеже $\delta(\qstart,a_{j}) = q_{j}$ и $f_{j-1} \in F$, то според конструкцията на крайния недетерминиран автомат $\N$,
  \[q_{j} \in \Delta(f_{j-1}, a_{j}).\]
  Оттук следва, че имаме следното изчисление на $\N$ върху $\alpha$:
  \begin{align*}
    &(\qstart, \alpha_{0}) \vdash^\star_\N (f_{0}, \varepsilon) & \comment\text{за някое }f_0 \in F\\
    &(f_0, \alpha_{1}) \vdash_\N (q_{1}, \alpha'_1) \vdash^\star_\N (f_{1}, \varepsilon) & \comment\text{за някое }f_1 \in F\\
    &(f_1, \alpha_{2}) \vdash_\N (q_{2}, \alpha'_2) \vdash^\star_\N (f_{2}, \varepsilon) & \comment\text{за някое }f_2 \in F\\
    &\dots\\
    &(f_{l-1}, \alpha_{l}) \vdash_\N (q_{l}, \alpha'_l) \vdash^\star_\N (f_{l}, \varepsilon). & \comment\text{за някое }f_l \in F
  \end{align*}
  Обединявайки всичко това, заключаваме, че  $\alpha \in \L(\N)$.

  Така доказахме, че $\L(\N) = (\L(\A))^+$.
  Сега ще построим недетерминистичен автомат
  \[\N' = \pair{\Sigma,Q',\qstart',\Delta',F'},\]
  такъв че
  \[\L(\N') = \L(\N)\cup\{\varepsilon\} = (\L(\A))^+ \cup \{\varepsilon\} = (\L(\A))^\star.\]
  \begin{itemize}
  \item
    $Q' = Q \cup \{\qstart'\}$;
  \item
    $F' = F \cup \{\qstart'\}$;
  \item
    $\Delta'(\qstart',a) = \Delta(\qstart,a)$, за всяко $a \in \Sigma$;
  \item
    $\Delta'(q,a) = \Delta(q,a)$, за всяко $a \in \Sigma$ и $q\in Q$.
  \end{itemize}
  Лесно се съобразява, че $\L(\N') = \L(\N)\cup\{\varepsilon\}$.
\end{proof}

\begin{example}
  Нека да приложим конструкцията за да намерим автомат разпознаващ $\L(\A)^\star$.
  
  \begin{figure}[H]
    % \center
    \begin{subfigure}[b]{0.3\textwidth}
      \begin{tikzpicture}[framed,->,>=stealth,thick,node distance=55pt]
        \tikzstyle{every state}=[circle,minimum size=20pt,auto]
        \node[initial below,state] (1) {$q_0$};
        \node[state]               (2) [right of=1] {$q_1$};
        \node[state,accepting]     (3) [right of=2] {$q_2$};
        \node[state,accepting]     (4) [above of=2] {$q_3$};
        \node[state]               (5) [above of=3] {$q_4$};
        \path
        (1) edge [bend right=15] node [below] {$a$} (2)
        (1) edge [bend left=15]  node [above] {$b$} (4)
        (2) edge [bend left=15]  node [left] {$a$} (4)
        (2) edge [bend right=15] node [below] {$b$} (3)
        (5) edge [loop above]    node [above] {$a,b$} (5)
        (4) edge [bend left=15]  node [above] {$a,b$} (5)
        (3) edge [bend right=15] node [right] {$b$} (5)
        (3) edge [bend left=45]  node [below] {$a$} (1);
      \end{tikzpicture}
      \caption{автомат $\A$}
    \end{subfigure}
    \hspace{2cm}
    \begin{subfigure}[b]{0.5\textwidth}
      \begin{tikzpicture}[framed,->,>=stealth,thick,node distance=55pt]
        \tikzstyle{every state}=[circle,minimum size=20pt,auto]
        
        \node[initial below, state] (1) [below right of=0] {$q_0$};
        \node[state]                (2) [right of=1] {$q_1$};
        \node[state,accepting]      (3) [right of=2] {$q_2$};
        \node[state,accepting]      (4) [above of=2] {$q_3$};
        \node[state]                (5) [above of=3] {$q_4$};
        \path
        (1) edge [bend left=15] node [above] {$b$} (4)
        (1) edge [bend right=15] node [below] {$a$} (2)
        (2) edge [bend right=15] node [below] {$b$} (3)
        (2) edge [bend left=15]  node [left] {$a$} (4)
        (3) edge [bend left=45]  node [below] {$a$} (1)
        (3) edge [dashed, bend right=20] node [above] {$b$} (4)
        (4) edge [dashed,bend left=15] node [right] {$a$} (2)
        (4) edge [dashed, loop above] node {$b$} (4)
        (5) edge [loop above] node [above] {$a,b$} (5)
        (4) edge [bend left=15] node [above] {$a,b$} (5)
        (3) edge [bend right=15] node [right] {$b$} (5)
        (3) edge [dashed, bend right=15] node [above] {$a$} (2);        
      \end{tikzpicture}
      \caption{$\L(\N) = \L(\A)^+$}
    \end{subfigure}
  \end{figure}
    
  \marginpar{Лесно се вижда, че $\L(\A) = \{b\} \cup \{ab\}\cdot\{aba\}^\star$}
  След като построим автомат за езика $\L(\A)^+$, трябва да приложим
  конструкцията за обединение на автомата за езика $\L(\A)^+$ с автомата за езика $\{\varepsilon\}$.
  Защо трябва да добавим ново начално състояние $q_0'$?
  Да допуснем, че вместо това сме направили $q_0$ финално.
  Тогава има опасност да разпознаем повече думи. Например, думата $aba$ би се разпознала от този автомат,
  но $aba \not\in\L(\A)^\star$.
  
  \begin{figure}[H]
    \centering
      \begin{tikzpicture}[framed,->,>=stealth,thick,node distance=55pt]
        \tikzstyle{every state}=[circle,minimum size=20pt,auto]
        \node[initial, state, accepting] (0) [above left of=1] {$q'_0$};
        \node[state] (1) [below right of=0] {$q_0$};
        \node[state]                (2) [right of=1] {$q_1$};
        \node[state,accepting]      (3) [right of=2] {$q_2$};
        \node[state,accepting]      (4) [above of=2] {$q_3$};
        \node[state]                (5) [above of=3] {$q_4$};
        \path
        (0) edge [dashed, bend left=15] node [above] {$b$} (4)
        (0) edge [dashed, bend left=15] node [above] {$a$} (2)
        (1) edge [bend left=15] node [above] {$b$} (4)
        (1) edge [bend right=15] node [below] {$a$} (2)
        (2) edge [bend right=15] node [below] {$b$} (3)
        (2) edge [bend left=15]  node [left] {$a$} (4)
        (3) edge [bend left=45]  node [below] {$a$} (1)
        (3) edge [dashed, bend right=20] node [above] {$b$} (4)
        (4) edge [dashed,bend left=15] node [right] {$a$} (2)
        (4) edge [dashed, loop above] node {$b$} (4)
        (5) edge [loop above] node [above] {$a,b$} (5)
        (4) edge [bend left=15] node [above] {$a,b$} (5)
        (3) edge [bend right=15] node [right] {$b$} (5)
        (3) edge [dashed, bend right=15] node [above] {$a$} (2);        
      \end{tikzpicture}
      \caption{$\L(\N') = \L(\N) \cup \{\varepsilon\} = \L(\A)^\star$}    
  \end{figure}

\end{example}



%%% Local Variables:
%%% mode: latex
%%% TeX-master: "../eai"
%%% End:

\subsection{Експоненциална експлозия}\label{sect:regular:nfa:exponential-blowup}
\mynote{Тук следваме \cite[стр. 66]{khoussainov-nerode} и \cite[стр. 164]{compilers2}. В \cite[стр. 80]{shallit} има друг пример за НКА с $n$ състояния вместо $n+1$, но доказателството изглежда по-сложно.}
\begin{proposition}
  Съществува НКА $\N$ с $n+1$ състояния, за който не съществува ДКА $\A$ с по-малко от $2^n$ състояния.
\end{proposition}
\begin{hint}
  Да разгледаме следния недетерминиран автомат $\N$.
  \begin{figure}[H]
    \centering
    \begin{tikzpicture}[framed,->,>=stealth,thick,node distance=45pt]
      \tikzstyle{every state}=[circle,minimum size=20pt,auto]
      \node[initial below,state]              (0) {$q_0$};
      \node[state]                [right of=0] (1) {$q_1$};
      \node[state]                [right of=1] (2) {$q_2$};
      \node[state]                [right of=2] (3) {$q_3$};
      \coordinate[right of=3] (4);
      \coordinate[right of=4] (5);
      \node[state,accepting]      [right of=5] (6) {$q_n$};
      \path
      (0) edge [loop above] node [above] {$a,b$} (0)
      (0) edge [bend left=15] node  [above] {$a$} (1)
      (1) edge [bend left=15] node  [above] {$a,b$} (2)
      (2) edge [bend left=15] node  [above] {$a,b$} (3);

      \draw [dashed,->,shorten >=0pt] (3) to[bend left=15] node[above] {$a,b$} (4);
      \draw [dashed,->,shorten >=0pt] (5) to[bend left=15] node[above] {$a,b$} (6);
    \end{tikzpicture}
    \caption{Недетерминиран автомат $\N$ за езика $\{a,b\}^\star \cdot a\cdot \{a,b\}^{n-1}$.}
    \label{fig:nfa:exp}
  \end{figure}
  Лесно се съобразява, че за $n > 0$, недетерминирният краен автомат $\N$ на \Figure{nfa:exp} с $n+1$ на брой състояния разпознава езика
  \[L = \{\alpha \in \{a,b\}^\star \mid \alpha = \beta \cdot a \cdot \gamma\ \&\ |\gamma| = n-1\}.\]
  Нека да съобразим, че не е възможно да съществува краен детерминиран автомат $\A = \FA$ разпознаващ същия език $L$ с по-малко от $2^n$ състояния.

  Да допуснем, че $|Q| < 2^n$. От принципа на Дирихле имаме, че съществуват две различни думи $\alpha$ и $\beta$ с дължина $n$,
  за които съществува $q \in Q$ и
  \mynote{Да напомним, че броят на всички думи с дължина $n$ над азбука с $k$ букви е $k^n$. В нашия случай, $k = 2$.}
  \[\delta^\star(\qstart,\alpha) = q = \delta^\star(\qstart,\beta).\]

  Нека първата разлика в тези две думи е на позиция $i < n$.
  \begin{itemize}
  \item
    Ако $i = 0$, то нека без ограничение на общността да имаме, че $\alpha[0] = a$ и $\beta[0] = b$.
    Това означава, че $\alpha \in L$, но $\beta \not\in L$.
    Следователно състоянието $q$ е едновременно финално и нефинално. Това е противоречие.
  \item
    Ако $i > 0$, то нека без ограничение на общността да имаме, че $\alpha[i] = a$ и $\beta[i] = b$.
    Да разгледаме следните думи:
    \begin{align*}
      & \alpha_0 = \alpha \cdot a^{i}\\
      & \beta_0 = \beta \cdot a^{i}.
    \end{align*}
    Така отново получаваме, че $\alpha_0 \in L$, но $\beta_0 \not\in L$.
    И в този случай получаваме противоречие, защото
    \[\delta^\star(\qstart,\alpha_0) = q = \delta^\star(\qstart,\beta_0)\]
    и състоянието $q$ трябва да е едновременно финално и нефинално.
  \end{itemize}  
\end{hint}

%%% Local Variables:
%%% mode: latex
%%% TeX-master: "../eai"
%%% End:

\newpage
\section{Лема за покачването}

\begin{lemma}[за покачването]
  \index{лема за покачването!регулярни езици}
  \label{lem:pumping-reg}
  \marginpar{На англ. се нарича \\ Pumping Lemma}
  \marginpar{Има подобна лема и за безконтекстни езици}
  \marginpar{Обърнете внимание, че $0 \in \Nat$ и $xy^0z =  xz$}
  Нека $L$ да бъде {\em безкраен} регулярен език.
  Съществува число $p\geq 1$, зависещо само от $L$, 
  за което за всяка дума $\alpha\in L, \abs{\alpha}\geq p$ може да 
  бъде записана във вида $\alpha = xyz$ и 
  \begin{enumerate}[1)]
  \item
    $|y|\geq 1$;
  \item
    $|xy|\leq p$;
  \item
    $(\forall i\in\Nat)[xy^iz \in L]$.
  \end{enumerate}
\end{lemma}
\begin{hint}
  \marginpar{\cite[стр. 88]{papadimitriou}, \cite[стр. 78]{sipser1}}
  Понеже $L$ е регулярен, то $L$ е и автоматен език. Нека $\A = \FA$ е краен детерминиран 
  автомат, за който $L = \L(\A)$.
  Да положим $p = \abs{Q}$ и нека $\alpha = a_1a_2\cdots a_k$ е дума, за която $k \geq p$.
  Да разгледаме първите $p$ стъпки от изпълнението на $\alpha$ върху $\A$:
  \[s\stackrel{a_1}{\rightarrow} q_1 \stackrel{a_2}{\rightarrow}q_2 \dots \stackrel{a_p}{\rightarrow} q_p.\]
  Тъй като $\abs{Q} = p$, а по този път участват $p+1$ състояния $q_0,q_1,\dots,q_p$,
  то съществуват числа $i, j$, за които $0\leq i < j\leq p$ и $q_i = q_j$.
  Нека разделим думата $\alpha$ на три части по следния начин:
  \[x = a_1\cdots a_i,\quad y = a_{i+1}\cdots a_j,\quad z = a_{j+1}\cdots a_k.\]
  Ясно е, че $\abs{y} \geq 1$ и $\abs{xy} = j \leq p$.
  \marginpar{\ding{45} Докажете!}
  Освен това, лесно се съобразява, че за всяко $i \in\Nat$,
  $xy^iz \in L$. Да разгледаме случая за $i = 0$.
  Думата $xy^0z = xz \in L$, защото имаме следното изчисление:
  \[s\underbrace{\stackrel{a_1}{\rightarrow}q_1 \cdots \stackrel{a_i}{\rightarrow}}_{x} q_i\underbrace{\stackrel{a_{j+1}}{\rightarrow}q_{j+1}\cdots\stackrel{a_{k}}{\rightarrow}}_{z}q_k\in F,\]
  защото $q_i = q_j$.
  Да разгледаме и случая $i = 2$. Тогава думата $xy^2z \in L$, защото имаме следното изчисление:
  \[s\underbrace{\stackrel{a_1}{\rightarrow}q_1 \cdots \stackrel{a_i}{\rightarrow}}_{x} q_i\underbrace{\stackrel{a_{i+1}}{\rightarrow}q_{i+1}\cdots\stackrel{a_{j}}{\rightarrow}}_{y}q_j\underbrace{\stackrel{a_{i+1}}{\rightarrow}q_{i+1}\cdots\stackrel{a_{j}}{\rightarrow}}_{y}q_j\underbrace{\stackrel{a_{j+1}}{\rightarrow}\cdots\stackrel{a_{k}}{\rightarrow}}_{z}q_k\in F.\]
\end{hint}

Практически е по-полезно да разглеждаме следната еквивалентна формулировка на лемата за покачването.
\begin{cor}[Контрапозиция на лемата за покачването]
  \label{cor:pumping-reg}
  \marginpar{Ясно е, че всеки краен език е регулярен. Нали?}
  Нека $L$ е произволен {\em безкраен} език. Нека също така е изпълнено, че за всяко естествено число $p \geq 1$ можем да намерим дума $\alpha \in L$, $\abs{\alpha}\geq p$, такава че за всяко разбиване на думата на три части, $\alpha = xyz$,
  със свойствата $\abs{y} \geq 1$ и $\abs{xy} \leq p$, е изпълнено, че $(\exists i)[xy^iz \not\in L]$.
  Тогава $L$ {\bf не} е регулярен език.
\end{cor}
\begin{proof}
  Да означим с $(P)$ следната формула:
  {\scriptsize
    \[(\exists p \geq 1)(\forall \alpha \in L)[\abs{\alpha} \geq p \Rightarrow (\exists x,y,z\in\Sigma^\star)[\alpha = xyz\ \wedge\ \abs{y} \geq 1\ \wedge\ \abs{xy} \leq p\ \wedge\ (\forall i\in\Nat)[xy^iz \in L]]].\]}
  \hyperref[lem:pumping-reg]{Лемата за покачването} представлява твърдението:
  
  \begin{center}
  {\em ,,Aко $L$ е регулярен език, то е изпълнено свойството $(P)$.''}
  \end{center}
  \marginpar{Контрапозиция на твърдението $p \to q$ е твърдението $\neg q \to \neg p$}
  \noindent
  Лемата може да се запише по следния еквивалентен начин:
  
  \begin{center}
    {\em ,,Ако свойството $(P)$ не е изпълнено, то $L$ не е регулярен език.''}
  \end{center}

  \marginpar{Използваме, че $\neg \exists \forall \exists \forall (\dots) \equiv \forall \exists \forall \exists \neg(\dots)$}

  \noindent Отрицанието на свойството $(P)$ може да се запише по следния начин:
  {\scriptsize  \[(\forall p \geq 1)(\exists \alpha \in L)[\abs{\alpha} \geq p\ \wedge (\forall x,y,z\in\Sigma^\star)[\alpha \neq xyz\ \vee\ \abs{y} \not\geq 1\ \vee\ \abs{xy} \not\leq p\ \vee\ (\exists i\in\Nat)[xy^iz \not\in L]]].\]}
  Горната формула е еквивалентна на:
  \marginpar{Използваме, че $\neg p \vee \neg q \vee r \equiv (p \wedge q) \to r$}
  {\scriptsize
    \[(\forall p \geq 1)(\exists \alpha \in L)[\abs{\alpha} \geq p\ \wedge\ (\forall x,y,z\in\Sigma^\star)[(\alpha = xyz \wedge \abs{y} \geq 1\wedge \abs{xy} \leq p) \Rightarrow (\exists i\in\Nat)[xy^iz \not\in L]]].\]}
  Това означава, че ако
  \begin{description}
  \item[$(\forall)$]
    вземем произволна константа $p \geq 1$,
  \item[$(\exists)$]
    за нея намерим дума $\alpha \in L$, такава че $\abs{\alpha} \geq p$ и 
  \item[$(\forall)$]
    докажем, че за всяко нейно разбиване на три части $x,y,z$, със свойствата
    $\abs{y} \geq 1$ и $\abs{xy} \leq p$,
  \item[$(\exists)$]
    можем да намерим $i$, за което $xy^iz \not\in L$,
  \end{description}
  то можем да заключим, че езикът $L$ не е регулярен.
\end{proof}

\subsection*{Приложения на лемата за покачването}

\begin{problem}
  Докажете, че езикът $L = \{a^nb^n \mid n\in \Nat\}$ не е регулярен.
\end{problem}
\begin{proof}
  \marginpar{Това е важен пример. По-късно ще видим, че този език е безконтекстен}
  Ще докажем, че
  {\scriptsize
    \[(\forall p \geq 1)(\exists \alpha \in L)[\abs{\alpha} \geq p\ \wedge\ (\forall x,y,z\in\Sigma^\star)[(\alpha = xyz \wedge \abs{y} \geq 1\wedge \abs{xy} \leq p) \Rightarrow (\exists i\in\Nat)[xy^iz \not\in L]].\]}
  Доказателството следва стъпките:
  \begin{description}
  \item[$(\forall)$]
    \marginpar{нямаме власт над избора на числото $p$}
    Разглеждаме произволно число $p \geq 1$.
  \item[$(\exists)$]
    \marginpar{Няма общо правило, което да ни казва как избираме думата $\alpha$}
    Избираме дума $\alpha \in L$, за която $\abs{\alpha} \geq p$. Имаме свободата да изберем каквато дума $\alpha$
    си харесаме, стига тя да принадлежи на $L$ и да има дължина поне $p$.
    \marginpar{Обърнете внимание, че думата $\alpha$ зависи от константата $p$}
    Щом имаме тази свобода, нека да изберем думата $\alpha = a^pb^p \in L$.
    Очевидно е, че $\abs{\alpha} \geq p$.
  \item[$(\forall)$]
    \marginpar{не знаем нищо друго за $x$, $y$ и $z$ освен тези две свойства}
    Разглеждаме произволно разбиване на $\alpha$ на три части, $\alpha = xyz$,
    за които изискваме свойствата $\abs{xy} \leq p$ и $\abs{y} \geq 1$.
  \item[$(\exists)$]
    Ще намерим $i\in\Nat$, за което $xy^iz \not\in L$.
    Понеже $\abs{xy} \leq p$, то $y = a^k$, за  $1\leq k \leq p$.
    Тогава ако вземем $i = 0$, получаваме $xy^0z = a^{p-k}b^p$.
    Ясно е, че $xz \not\in L$, защото $p-k < p$.
  \end{description}  
  Тогава от \Cor{pumping-reg} следва, че $L$ не е регулярен език.
\end{proof}

\begin{remark}
  Много често студентите правят следното разсъждение:
  \[(\forall L,L' \subseteq \Sigma^\star)[L \text{ е регулярен}\ \&\ L' \subseteq L \implies L'\text{ е регулярен}].\]
  Съобразете, че в общия случай това твърдение е невярно.
  За да видите това, достатъчно е да посочите регулярен език $L$, който има като
  подмножество нерегулярен език $L'$.
  Също лесно се вижда, че твърдението
  \[(\forall L,L' \subseteq \Sigma^\star)[L \text{ е регулярен}\ \&\ L \subseteq L' \implies L'\text{ е регулярен}]\]
  е невярно.
\end{remark}

\begin{problem}
  Докажете, че езикът $L = \{a^mb^n \mid m,n\in \Nat\ \&\ m < n\}$ не е регулярен.
\end{problem}
\begin{proof}
  Доказателството следва стъпките:
  \begin{description}
  \item[$(\forall)$]
    Разглеждаме произволно число $p \geq 1$.
  \item[$(\exists)$]
    Избираме дума $\alpha \in L$, за която $\abs{\alpha} \geq p$. Имаме свободата да изберем каквато дума $\alpha$
    си харесаме, стига тя да принадлежи на $L$ и да има дължина поне $p$.
    Щом имаме тази свобода, нека да изберем думата $\alpha = a^{p}b^{p+1} \in L$. Очевидно е, че $\abs{\alpha} \geq p$.
  \item[$(\forall)$]
    Разглеждаме произволно разбиване на $\alpha$ на три части, $\alpha = xyz$,
    за които изискваме свойствата $\abs{xy} \leq p$ и $\abs{y} \geq 1$.
  \item[$(\exists)$]
    Ще намерим $i\in\Nat$, за което $xy^iz \not\in L$.
    Понеже $\abs{xy} \leq p$, то $y = a^k$, за  $1\leq k \leq p$.
    Тогава ако вземем $i = 2$, получаваме 
    \[xy^2z = a^{p-k}a^{2k}b^{p+1} = a^{p+k}b^{p+1}.\]
    Ясно е, че $xy^2z \not\in L$, защото $p+k \geq p+1$.
  \end{description}
  Тогава от \Cor{pumping-reg} следва, че $L$ не е регулярен език.
\end{proof}

\begin{problem}
  Докажете, че езикът $L = \{a^n\ \mid\ n\mbox{ е просто число}\}$ не е регулярен.
\end{problem}
\begin{proof}
  Доказателството следва стъпките:
  \begin{description}
  \item[$(\forall)$] 
    Разглеждаме произволно число $p \geq 1$.
  \item[$(\exists)$]
    Избираме дума $w \in L$, за която $\abs{w} \geq p$. Можем да изберем каквото $w$ 
    си харесаме, стига то да принадлежи на $L$ и да има дължина поне $p$.
    Нека да изберем думата $w \in L$, такава че $\abs{w} > p+1$.
    Знаем, че такава дума съществува, защото $L$ е безкраен език. По-долу ще видим защо този избор е важен за нашите разсъждения.
  \item[$(\forall)$]
    Разглеждаме произволно разбиване на $w$ на три части, $w = xyz$,
    за които изискваме свойствата $\abs{xy} \leq p$ и $\abs{y} \geq 1$.
  \item[$(\exists)$]
    Ще намерим $i$, за което $xy^iz \not\in L$,
    т.е. ще намерим $i$, за което 
    $\abs{xy^iz} = \abs{xz} + i\cdot\abs{y}$ е {\em съставно число}.
    Понеже $\abs{xy} \leq p$ и $\abs{xyz} > p+1$, то $\abs{z} > 1$.
    Да изберем $i = \abs{xz} > 1$. Тогава:
    \[\abs{xy^iz} = \abs{xz} + i.\abs{y} = \abs{xz} + \abs{xz}.\abs{y} = (1 + \abs{y})\abs{xz}\] е съставно число, следователно 
    $xy^iz \not\in L$.
  \end{description}
  Тогава от \Cor{pumping-reg} следва, че $L$ не е регулярен език.
\end{proof}

\begin{problem}
  Докажете, че езикът $L = \{a^{n^2}\ \mid\ n\in\Nat\}$ не е регулярен.  
\end{problem}
\begin{proof}
  В тази задача ще използваме следното свойство:
  \[n\text{ не е точен квадрат} \iff (\exists p\in \Nat)[p^2 < n < (p+1)^2].\]
  Доказателството следва стъпките:
  \begin{description}
  \item[$(\forall)$]
    Разглеждаме произволно число $p \geq 1$.
  \item[$(\exists)$]
    Избираме достатъчно дълга дума, която принадлежи на езика $L$.
    Например, нека $w = a^{p^2}$.
  \item[$(\forall)$]
    Разглеждаме произволно разбиване на $w$ на три части, $w = xyz$, 
    като $\abs{xy} \leq p$ и $\abs{y} \geq 1$.
  \item[$(\exists)$]
    Ще намерим $i$, за което $xy^iz \not\in L$.
    В нашия случай това означава, че $\abs{xz} + i\cdot\abs{y}$ не е точен квадрат.
    Тогава за $i = 2$,
    \[p^2 = \abs{xyz} < \abs{xy^2z} = \abs{xz} + 2\abs{y} \leq p^2 + 2p < p^2 + 2p + 1 = (p+1)^2 .\]
    Получаваме, че $p^2 < \abs{xy^2z} < (p+1)^2$,
    откъдето следва, че $\abs{xy^2z}$ не е точен квадрат.
    Следователно, $xy^2z \not\in L$.
  \end{description}
  Тогава от \Cor{pumping-reg} следва, че $L$ не е регулярен език.  
\end{proof}

\begin{problem}
  Докажете, че езикът $L = \{a^{n!}\ \mid\ n\in\Nat\}$ не е регулярен.  
\end{problem}
\begin{proof}
  Доказателството следва стъпките:
  \begin{description}
  \item[$(\forall)$]
    Разглеждаме произволно число $p \geq 1$.
  \item[$(\exists)$]
    Избираме достатъчно дълга дума, която принадлежи на езика $L$. Например, нека $\omega = a^{(p+2)!}$.
  \item[$(\forall)$]
    Разглеждаме произволно разбиване на $\omega$ на три части, $\omega = xyz$, 
    като $\abs{xy} \leq p$ и $\abs{y} \geq 1$.
    Да обърнем внимание, че $1 \leq \abs{y} \leq p$
  \item[$(\exists)$]
    Ще намерим $i$, за което $xy^iz \not\in L$.
    В нашия случай това означава, че $\abs{xz} + i\cdot\abs{y}$ не е от вида $n!$.
    Възможно ли е $xy^0z \in L$?
    Понеже $\abs{xyz} = (p+2)!$, това означава, че $\abs{xz} = k!$, за някое $k \leq p+1$.
    Тогава 
    \[\abs{y} = \abs{xyz} - \abs{xz} = (p+2)! - k! \geq (p+2)! - (p+1)! = (p+1).(p+1)! > p.\]
    Достигнахме до противоречие с условието, че $\abs{y} \leq p$.
  \end{description}
  Тогава от \Cor{pumping-reg} следва, че $L$ не е регулярен език.  
\end{proof}

\subsection*{Следствия от лемата за покачването}

\begin{prop}
  Нека е даден автомата $\A = \FA$.
  Езикът $\L(\A)$ е {\em непразен} е точно тогава, когато съдържа дума $\alpha, \abs{\alpha} < \abs{Q}$.
\end{prop}
\begin{proof}
  Ще разгледаме двете посоки на твърдението.
  \begin{description}
  \item[$(\Rightarrow)$]
    Нека $L$ е непразен език и нека $m = \min\{\abs{\alpha} \mid \alpha \in L\}$.
    Ще докажем, че $m < \abs{Q}$.    
    За целта, да допуснем, че $m \geq \abs{Q}$ и да изберем $\alpha \in L$, за която $\abs{\alpha} = m$.
    Според \Lem{pumping-reg}, съществува разбиване $xyz = \alpha$, 
    такова че $xz \in L$.
    При положение, че $\abs{y} \geq 1$, то $\abs{xz} < m$, което 
    е противоречие с минималността на $m$.
    Заключаваме, че нашето допускане е грешно. Тогава $m < \abs{Q}$, откъдето следва, че 
    съществува дума $\alpha \in L$ с $\abs{\alpha} < \abs{Q}$.
  \item[$(\Leftarrow)$]
    Тази посока е тривиална.
    Ако $L$ съдържа дума $\alpha$, за която $\abs{\alpha} < \abs{Q}$,
    то е очевидно, че $L$ е непразен език.
  \end{description}
\end{proof}

\begin{cor}
  Съществува алгоритъм, който проверява дали даден регулярен език е празен или не.
\end{cor}


\begin{cor}
  \marginpar{$(L_1\setminus L_2) \cup (L_2 \setminus L_1) = \emptyset$?}
  Съществува алгоритъм, който определя дали два автомата $\A_1$ и $\A_2$ разпознават един и същ език.
\end{cor}

\begin{prop}
  Регулярният език $L$, 
  разпознаван от КДА $\A$, е {\em безкраен} точно тогава, когато съдържа дума $\alpha, \abs{Q} \leq \abs{\alpha} < 2\abs{Q}$.
\end{prop}
\begin{proof}
  Да разгледаме двете посоки на твърдението.
  \begin{description}
  \item[$(\Leftarrow)$]
    Нека $L$ е регулярен език, за който съществува дума $\alpha$, такава че $\abs{Q} \leq \abs{\alpha} < 2\abs{Q}$.
    Тогава от \Lem{pumping-reg} следва, че съществува разбиване $\alpha = xyz$ със свойството, че
    за всяко $i \in \Nat$, $xy^iz \in L$. Следователно, $L$ е безкраен, защото $\abs{y} \geq 1$.
  \item[$(\Rightarrow)$]
    Нека $L$ е безкраен език и % да приемем, че няма думи $\alpha$ със
    % свойството $\abs{Q} \leq \abs{\alpha} <  2\abs{Q}$.
    да вземем {\em най-късата} дума $\alpha \in L$, за която $\abs{\alpha} \geq 2\abs{Q}$.
    Понеже $L$ е безкраен, знаем, че такава дума съществува.
    Тогава отново по \Lem{pumping-reg}, имаме следното разбиване на $\alpha$:
    \[\alpha = xyz,\ \abs{xy} \leq \abs{Q},\ 1\leq \abs{y},\ xz \in L.\]
    Но понеже $\abs{xyz} \geq 2\abs{Q}$, а $1 \leq \abs{y} \leq \abs{Q}$, то $\abs{xyz} > \abs{xz} \geq \abs{Q}$ и понеже избрахме $\alpha = xyz$
    да бъде най-късата дума с дължина поне $2\abs{Q}$, заключаваме, че $\abs{Q} \leq \abs{xz} < 2\abs{Q}$ и $xz \in L$.
  \end{description}
\end{proof}

\begin{cor}
  Съществува алгоритъм, който проверява дали даден регулярен език е безкраен.
\end{cor}


\subsection*{Примери, за които лемата не е  приложима}

\begin{problem}
  \marginpar{Например, $\{c\}^+\cdot\{a^nb^n\mid n\in\Nat\}\cup \{a,b\}^\star$}
  Да се даде пример за език $L$, който {\bf не} е регулярен, но удовлетворява
  условието на \Lem{pumping-reg}.
\end{problem}

\begin{example}
  Езикът $L = \{c^ka^nb^m\mid k,n,m \in \Nat\ \&\ k = 1\implies m = n\}$
  {\bf не} е регулярен, но условието за покачване от \Lem{pumping-reg} е изпълнено за него.
\end{example}
\begin{proof}
  Да допуснем, че $L$ е регулярен.
  Тогава ще следва, че 
  \[L_1 = L\cap ca^\star b^\star = \{ca^nb^n \mid n\in\Nat\}\]
  е регулярен,
  но с лемата за разрастването лесно се вижда, че $L_1$ не е.

  Сега да проверим, че условието за покачване от \Lem{pumping-reg} е изпълнено за $L$.
  Да изберем константа $p = 2$.
  Сега трябва да разгледаме всички думи $\alpha \in L$, $\abs{\alpha} \geq 2$
  и за всяка $\alpha$ да посочим разбиване $xyz = \alpha$, за което са изпълнени трите свойства от лемата.
  \marginpar{Условията за $x,y,z$ са:
    \begin{align*}
      & \abs{xy} \leq 2\\
      & \abs{y} \geq 1\\
      & (\forall i\in\Nat)(xy^iz \in L)
    \end{align*}}

  \begin{itemize}
  \item
    Ако $\alpha = a^n$ или $\alpha = b^n$, $n\geq 2$, то е  очевидно, че можем да
    намерим такова разбиване.
  \item
    $\alpha = a^nb^m$ и $n+m \geq 2$, $n \geq 1$.
    Избираме $x = \varepsilon$, $y = a$, $z = a^{n-1}b^m$.
  \item
    $\alpha = ca^nb^n$, $n\geq 1$.
    Избираме $x = \varepsilon$, $y = c$, $z = a^nb^n$.
  \item
    $\alpha = c^2a^nb^m$. 
    Избираме $x = \varepsilon$, $y = c^2$, $z = a^nb^m$.
  \item
    $\alpha = c^ka^nb^m$, $k \geq 3$.
    Избираме $x = \varepsilon$, $y = c$, $z = c^{k-1}a^nb^m$.
  \end{itemize}
\end{proof}


%%% Local Variables:
%%% mode: latex
%%% TeX-master: "../eai"
%%% End:

\subsection{Следствия}

\begin{proposition}
  Езикът на автомата $\A = \FA$ е {\em непразен} точно тогава, когато съдържа дума $\alpha$,
  за която $\abs{\alpha} < \abs{Q}$.
\end{proposition}
\begin{proof}
  Ще разгледаме двете посоки на твърдението.
  \begin{description}
  \item[$(\Rightarrow)$]
    Нека $L$ е непразен език и нека $m = \min\{\abs{\alpha} \mid \alpha \in L\}$.
    Ще докажем, че $m < \abs{Q}$.    
    За целта, да допуснем, че $m \geq \abs{Q}$ и да изберем $\alpha \in L$, за която $\abs{\alpha} = m$.
    Според доказателството на \Lemma{pumping-reg}, съществува разбиване $xyz = \alpha$, 
    такова че $xz \in L$.
    При положение, че $\abs{y} \geq 1$, то $\abs{xz} < m$, което 
    е противоречие с минималността на думата $\alpha$.
    Заключаваме, че нашето допускане е грешно. Тогава $m < \abs{Q}$, откъдето следва, че 
    съществува дума $\alpha \in L$ с $\abs{\alpha} < \abs{Q}$.
  \item[$(\Leftarrow)$]
    Тази посока е тривиална.
    Ако $L$ съдържа дума $\alpha$, за която $\abs{\alpha} < \abs{Q}$,
    то е очевидно, че $L$ е непразен език.
  \end{description}
\end{proof}

\begin{cor}
  \marginpar{\writedown Обяснете!}
  Съществува алгоритъм, който проверява дали даден регулярен език е празен или не.
\end{cor}


\begin{cor}
  \marginpar{$(L_1\setminus L_2) \cup (L_2 \setminus L_1) = \emptyset$?}
  Съществува алгоритъм, който определя дали два автомата $\A_1$ и $\A_2$ разпознават един и същ език.
\end{cor}

\begin{proposition}
  Регулярният език $L$, 
  разпознаван от КДА $\A$, е {\em безкраен} точно тогава, когато съдържа дума $\alpha, \abs{Q} \leq \abs{\alpha} < 2\abs{Q}$.
\end{proposition}
\begin{proof}
  Да разгледаме двете посоки на твърдението.
  \begin{description}
  \item[$(\Leftarrow)$]
    Нека $L$ е регулярен език, за който съществува дума $\alpha$, такава че $\abs{Q} \leq \abs{\alpha} < 2\abs{Q}$.
    Тогава от \Lemma{pumping-reg} следва, че съществува разбиване $\alpha = xyz$ със свойството, че
    за всяко $i \in \Nat$, $xy^iz \in L$. Следователно, $L$ е безкраен, защото $\abs{y} \geq 1$.
  \item[$(\Rightarrow)$]
    Нека $L$ е безкраен език и % да приемем, че няма думи $\alpha$ със
    % свойството $\abs{Q} \leq \abs{\alpha} <  2\abs{Q}$.
    да вземем {\em най-късата} дума $\alpha \in L$, за която $\abs{\alpha} \geq 2\abs{Q}$.
    Понеже $L$ е безкраен, знаем, че такава дума съществува.
    Тогава отново по \Lemma{pumping-reg}, имаме следното разбиване на $\alpha$:
    \[\alpha = xyz,\ \abs{xy} \leq \abs{Q},\ 1\leq \abs{y},\ xz \in L.\]
    Но понеже $\abs{xyz} \geq 2\abs{Q}$, а $1 \leq \abs{y} \leq \abs{Q}$, то $\abs{xyz} > \abs{xz} \geq \abs{Q}$ и понеже избрахме $\alpha = xyz$
    да бъде най-късата дума с дължина поне $2\abs{Q}$, заключаваме, че $\abs{Q} \leq \abs{xz} < 2\abs{Q}$ и $xz \in L$.
  \end{description}
\end{proof}

\begin{cor}
  \marginpar{\writedown Обяснете!}
  Съществува алгоритъм, който проверява дали даден регулярен език е безкраен.
\end{cor}

\begin{cor}
  \marginpar{\writedown Обяснете!}
  Съществува алгоритъм, който проверява дали симетричната разлика на два регулярни езика е
  крайна.
\end{cor}


%%% Local Variables:
%%% mode: latex
%%% TeX-master: "../eai"
%%% End:

\subsection{Примерни задачи}

\begin{extra}
\begin{problem}
  Докажете, че езикът $L = \{a^mb^n \mid m,n\in \Nat\ \&\ m < n\}$ не е регулярен.
\end{problem}
\begin{proof}
  \mynote{На стъпка от вида $(\forall)$ нямаме власт над това как избираме съответния елемент.
    
  На стъпка от вида $(\exists)$ имаме тази власт. Тогава трябва да посочим конкретен елемент.}
  Доказателството следва стъпките:
  \begin{description}
  \item[$(\forall)$]
    Разглеждаме произволно число $p \geq 1$.
  \item[$(\exists)$]
    Избираме дума $\alpha \in L$, за която $\abs{\alpha} \geq p$. Имаме свободата да изберем каквато дума $\alpha$
    си харесаме, стига тя да принадлежи на $L$ и да има дължина поне $p$.
    Щом имаме тази свобода, нека да изберем думата $\alpha = a^{p}b^{p+1} \in L$. Очевидно е, че $\abs{\alpha} \geq p$.
  \item[$(\forall)$]
    Разглеждаме произволно разбиване на $\alpha$ на три части, $\alpha = xyz$,
    за които изискваме свойствата $\abs{xy} \leq p$ и $\abs{y} \geq 1$.
  \item[$(\exists)$]
    \mynote{Тук изборът на $i$ не зависи от изборите, които сме направили на предишните стъпки.}
    Ще намерим конкретно $i\in\Nat$, за което $xy^iz \not\in L$.
    Понеже $\abs{xy} \leq p$, то $y = a^k$, за  $1\leq k \leq p$.
    Тогава ако вземем $i = 2$, получаваме 
    \[xy^2z = a^{p-k}a^{2k}b^{p+1} = a^{p+k}b^{p+1}.\]
    Ясно е, че $xy^2z \not\in L$, защото $p+k \geq p+1$.
  \end{description}
  Тогава от \hyperref[cor:regular:pumping]{контрапозицията на лемата за покачването} следва, че $L$ не е регулярен език.
\end{proof}

\begin{problem}
  Докажете, че езикът $L = \{a^n\ \mid\ n\mbox{ е просто число}\}$ не е регулярен.
\end{problem}
\begin{proof}
  Доказателството следва стъпките:
  \begin{description}
  \item[$(\forall)$] 
    Разглеждаме произволно число $p \geq 1$.
  \item[$(\exists)$]
    Избираме дума $w \in L$, за която $\abs{w} \geq p$. Можем да изберем каквото $w$ 
    си харесаме, стига то да принадлежи на $L$ и да има дължина поне $p$.
    Нека да изберем една конкретна дума $w \in L$, такава че $\abs{w} > p+1$.
    Знаем, че такава дума съществува, защото $L$ е безкраен език. По-долу ще видим защо този избор е важен за нашите разсъждения.
  \item[$(\forall)$]
    \mynote{Обърнете внимание, че тук е по-интересно. Изборът на $i$ зависи от предишната стъпка, на която сме разбили думата $w$ на три части.}
    Разглеждаме произволно разбиване на $w$ на три части, $w = xyz$,
    за които изискваме свойствата $\abs{xy} \leq p$ и $\abs{y} \geq 1$.
  \item[$(\exists)$]
    Ще намерим конкретно $i$, за което $xy^iz \not\in L$,
    т.е. ще намерим $i$, за което 
    $\abs{xy^iz} = \abs{xz} + i\cdot\abs{y}$ е {\em съставно число}.
    Понеже $\abs{xy} \leq p$ и $\abs{xyz} > p+1$, то $\abs{z} > 1$.
    Да изберем $i = \abs{xz} > 1$. Тогава:
    \[\abs{xy^iz} = \abs{xz} + i.\abs{y} = \abs{xz} + \abs{xz}.\abs{y} = (1 + \abs{y})\abs{xz}\] е съставно число, следователно 
    $xy^iz \not\in L$.
    \mynote{Изискваме $|w| > p+1$, защото искаме да гарантираме, че $|xz| > 1$.}
  \end{description}
  Тогава от \hyperref[cor:regular:pumping]{контрапозицията на лемата за покачването} следва, че $L$ не е регулярен език.
\end{proof}

\begin{problem}
  Докажете, че езикът $L = \{a^{n^2}\ \mid\ n\in\Nat\}$ не е регулярен.  
\end{problem}
\begin{proof}
  В тази задача ще използваме следното свойство:
  \[n\text{ не е точен квадрат} \iff (\exists p\in \Nat)[p^2 < n < (p+1)^2].\]
  Доказателството следва стъпките:
  \begin{description}
  \item[$(\forall)$]
    Разглеждаме произволно число $p \geq 1$.
  \item[$(\exists)$]
    Избираме достатъчно дълга дума, която принадлежи на езика $L$.
    За да бъдем конкретни, нека $w = a^{p^2}$.
  \item[$(\forall)$]
    Разглеждаме произволно разбиване на $w$ на три части, $w = xyz$, 
    като $\abs{xy} \leq p$ и $\abs{y} \geq 1$.
  \item[$(\exists)$]
    Ще намерим конкретно $i$, за което $xy^iz \not\in L$.
    В нашия случай това означава, че $\abs{xz} + i\cdot\abs{y}$ не е точен квадрат.
    Тогава за $i = 2$,
    \[p^2 = \abs{xyz} < \abs{xy^2z} = \abs{xyz} + \abs{y} \leq p^2 + p < p^2 + 2p + 1 = (p+1)^2 .\]
    Получаваме, че $p^2 < \abs{xy^2z} < (p+1)^2$,
    откъдето следва, че $\abs{xy^2z}$ не е точен квадрат.
    Следователно, $xy^2z \not\in L$.
  \end{description}
  Тогава от \hyperref[cor:regular:pumping]{контрапозицията на лемата за покачването} следва, че $L$ не е регулярен език.  
\end{proof}

\begin{problem}
  Докажете, че езикът $L = \{a^{n!}\ \mid\ n\in\Nat\}$ не е регулярен.  
\end{problem}
\begin{proof}
  Доказателството следва стъпките:
  \begin{description}
  \item[$(\forall)$]
    Разглеждаме произволно число $p \geq 1$.
  \item[$(\exists)$]
    Избираме достатъчно дълга дума, която принадлежи на езика $L$. 
    За да бъдем конкретни, нека $\omega = a^{(p+1)!}$.
  \item[$(\forall)$]
    Разглеждаме произволно разбиване на $\omega$ на три части, $\omega = xyz$, 
    като $\abs{xy} \leq p$ и $\abs{y} \geq 1$.
    Да обърнем внимание, че $1 \leq \abs{y} \leq p$.
  \item[$(\exists)$]
    Ще намерим конкретно $i$, за което $xy^iz \not\in L$.
    Това означава да съществува $n$, за което
    \[n! < |xy^iz| < (n+1)!\]
    Да разгледаме $i = 2$. Тогава:
    \mynote{Възможно е да вземем $w = a^{(p+2)!}$. Тогава възможно ли е $xy^0z \not\in L$?
      Понеже $\abs{xyz} = (p+2)!$, това означава, че би трябвало $\abs{xz} = k!$, за някое $k \leq p+1$.
      Тогава
      \begin{align*}
        \abs{y} & = \abs{xyz} - \abs{xz}\\
                & = (p+2)! - k!\\
                & \geq (p+2)! - (p+1)!\\
                & = (p+1).(p+1)!\\
                & > p.
      \end{align*}
      Достигнахме до противоречие с условието, че $\abs{y} \leq p$.}
    \begin{align*}
      (p+1)! & < |xy^2z| \\
             & = (p+1)! + |y|\\
             & \leq (p+1)! + p \\
             & < (p+1)! + (p+1)!(p+1) \\
             & = (p+2)!
    \end{align*}
  \end{description}
  Тогава от \hyperref[cor:regular:pumping]{контрапозицията на лемата за покачването} следва, че $L$ не е регулярен език.  
\end{proof}

\begin{problem}
  Докажете, че езикът $L = \{\alpha\beta \in \{a,b\}^\star \mid |\alpha| = |\beta|\ \&\ \alpha \neq \beta\}$ не е регулярен.
\end{problem}
\begin{hint}
  Да допуснем, че $L$ е регулярен.
  Тогава езикът $\ov{L} = \{a,b\}^\star \setminus L$ също е регулярен.
  Ясно е, че
  \[\ov{L} = \{ \alpha\beta \in \{a,b\}^\star\ \mid\ \alpha = \beta\}\ \cup\ \{\omega \in \{a,b\}^\star \mid |\omega| \text{ е нечетно число}\}.\]
  Тогава езикът $L_1 = \ov{L} \cap \{ \omega \in \{a,b\}^\star \mid |\omega| \text{ е четно число}\}$ също е регулярен.
  Ясно е, че $L_1 = \{\alpha\beta \in \{a,b\}^\star\ \mid\ \alpha = \beta\}$.
  Сега можем да разгледаме регулярния език
  \[L_2 = L_1 \cap \L(\mathbf{a^\star b a^\star b}) = \{a^n b a^n b \mid n \in \Nat\}.\]
  За него вече лесно можем да приложим \hyperref[lem:regular:pumping]{Лемата за покачването} и да получим, че $L_2$ не е регулярен.
  Така достигаме до противоречие с допускането, че $L$ е регулярен.
\end{hint}
\end{extra}

\subsection*{Пример, за който лемата не е  приложима}

% Добре е да отбележим, че \hyperref[lem:pumping-reg]{Лемата за покачването} не задава пълен критерий за проверка за регулярност на език.
Да напомним, че условието на \hyperref[lem:regular:pumping]{Лемата за покачването} представлява твърдението:
\begin{center}
  {\em ,,Aко $L$ е регулярен език, то е изпълнено $P_{\text{reg}}(L)$.''}
\end{center}
Сега ще видим, че можем да посочим език $L$, който не е регулярен, но въпреки това условието $P_{\text{reg}}(L)$ е изпълнено.
Това означава, че нямаме обратната посока на горната импликация и може да срещнем примери за езици, които макар и нерегулярни, не можем да докажем тяхната нерегулярност с помощта на \hyperref[cor:regular:pumping]{контрапозицията на лемата за покачването}.
По-късно ще видим един пълен критерий за проверка за регулярност на език.

\begin{example}
  Езикът
  \[L = \{c\}^+\cdot\{a^nb^n\mid n\in\Nat\}\cup \{a\}^\star \cdot \{b\}^\star\]
  {\bf не} е регулярен, но условието $P_{\text{reg}}(L)$ е изпълнено.
\end{example}
\begin{hint}
  \mynote{
    За да покажем, че $P_{\text{reg}}(L)$ е изпълнено, трябва да следваме стъпките:
    \begin{description}
    \item[$(\exists)$]
      Избираме конкретно число $p \geq 1$.
    \item[$(\forall)$]
      Разглеждаме произволна дума $\alpha \in L$ и $\abs{\alpha} \geq p$.
    \item[$(\exists)$]
      Посочваме конкретно разбиване на думата $\alpha$ като $\alpha = xyz$ със свойството $\abs{xy} \leq p$ и $\abs{y} \geq 1$.
    \item[$(\forall)$]
      За всяко $i$ трябва да покажем, че $xy^iz \in L$.
    \end{description}
  }
  Ако допуснем, че $L$ е регулярен, то тогава ще следва, че 
  \[L_1 = L\cap \L(\mathbf{ca^\star b^\star}) = \{ca^nb^n \mid n\in\Nat\}\]
  е регулярен, но с \hyperref[cor:regular:pumping]{контрапозицията на лемата за покачването} лесно се вижда, че $L_1$ не е. Сега да проверим, че $P_{\text{reg}}(L)$ е изпълнено. 
  \begin{description}
  \item[$(\exists)$]
    Нека изберем $p = 2$.
  \item[$(\forall)$]
    Сега трябва да разгледаме всички думи $\alpha \in L$, $\abs{\alpha} \geq 2$.
  \item[$(\exists)$]
    Нека разбием думата $\alpha$ на три части по следния начин:
    \[x = \varepsilon,\ y = \alpha\slice{0},\ z = \alpha\slice{1:}.\]
  \item[$(\forall)$]
    Съобразете, че за всяко $i \in \Nat$ имаме, че $xy^iz \in L$.
  \end{description}
\end{hint}


%%% Local Variables:
%%% mode: latex
%%% TeX-master: "../eai"
%%% End:

\newpage
\section{Изоморфни автомати}
\label{sect:isomorphic}

\index{изоморфизъм}
Нека са дадени автоматите
$\A' = (\Sigma,Q',\qstart',\delta',F')$ и $\A'' = (\Sigma, Q'', \qstart'', \delta'', F'')$.
Казваме, че $\A'$ и $\A''$ са {\bf изоморфни}, което означаваме с $\A' \cong \A''$, ако
съществува функция $f: Q'\to Q''$, за която:
\begin{enumerate}[(1)]
\item
  $f$ е биекция;
\item
  $f(\qstart') = \qstart''$;
\item
  $q \in F' \iff f(q) \in F''$;
\item
  $(\forall a\in\Sigma)(\forall q\in Q')[f(\delta'(q,a)) = \delta''(f(q),a)]$.
\end{enumerate}
Ще казваме, че $f$ задава изоморфизъм на $\A'$ върху $\A''$.

\begin{framed}
  \begin{thm}
    Ако $\A' \cong_f \A''$, то $\L(\A') = \L(\A'')$.
  \end{thm}  
\end{framed}
\begin{hint}
  Нека $\A' \cong_f \A''$. Първо с индукция по дължината на думата $\alpha$ ще докажем, че за произволно състояние $q \in Q'$,
  \begin{equation}
    \label{eq:3}
    f(\delta^\star_{\A'}(q,\alpha)) = \delta^\star_{\A''}(f(q), \alpha).
  \end{equation}
  \begin{itemize}
  \item 
    Нека $|\alpha| = 0$, т.е. $\alpha = \varepsilon$. Тогава:
    \begin{align*}
      f(\delta^\star_{\A'}(q,\varepsilon)) & = f(q) & \comment{\text{от деф. на }\delta^\star_{\A'}}\\
                                           & = \delta^\star_{\A''}(f(q), \varepsilon).
    \end{align*}
  \item
    Да приемем, че (\ref{eq:3}) е изпълнено за думи с дължина $n$.
    Да разгледаме произволна дума $\alpha$ с дължина $n+1$, т.е. $\alpha = \beta x$. Тогава:
    \begin{align*}
      f(\delta^\star_{\A'}(q,\beta x)) & = f(\delta_{\A'}(\delta^\star_{\A'}(q,\beta), x)) & \comment{\text{от деф. на }\delta^\star_{\A'}}\\
                                       & = \delta_{\A''}( f(\delta^\star_{\A'}(q,\beta), x)) & \comment{\text{от деф. на }f}\\
                                       & = \delta_{\A''}( \delta^\star_{\A''}(f(q),\beta), x)) & \comment{\text{от И.П.}}\\
                                       & = \delta^\star_{\A''}(f(q), \beta x) & \comment{\text{от деф. на }\delta^\star_{\A''}}.
    \end{align*}
  \end{itemize}
  Сега вече е лесно:
  \begin{align*}
    \alpha \in \L(\A') & \dff \delta^\star_{\A'}(\qstart',\alpha) \in F' \\
                       & \iff f(\delta^\star_{\A'}(\qstart',\alpha)) \in F'' & \comment{\text{от деф. на изоморфизъм}}\\
                       & \iff \delta^\star_{\A''}(f(\qstart'),\alpha) \in F'' & \comment{\text{от (\ref{eq:3})}}\\
                       & \iff \delta^\star_{\A''}(\qstart'',\alpha) \in F'' & \comment{f(\qstart') \df \qstart''}\\
                       & \dff \alpha \in \L(\A'').
  \end{align*}
\end{hint}

\index{Бжозовски}
Нека е даден регулярен език $L$ над азбуката $\Sigma$.
Конструкцията на автомата $\B$ по метода на Бжозовски е следната:
\begin{itemize}
\item 
  $Q^\B \df \{\alpha^{-1}(L) \mid \alpha \in \Sigma^\star\}$;
\item
  $\qstart^\B \df L$;
\item
  $F^\B \df \{N \in Q^\B \mid \varepsilon \in N\}$;
\item
  $\delta_\B(N,x) \df x^{-1}(N)$, за произволни $x \in \Sigma$ и $N \in Q^\B$.
\end{itemize}

Да припомним конструкцията на минималния автомат $\M$ според \hyperref[th:myhill-nerode]{Теоремата на Майхил-Нероуд}.
\begin{itemize}
\item 
  $Q^\M \df \{[\alpha]_L \mid \alpha \in \Sigma^\star\}$;
\item
  $\qstart^\M \df [\varepsilon]_L$;
\item
  $F^\M \df \{[\alpha]_L \mid [\alpha]_L \subseteq L\}$;
\item
  $\delta_\M([\alpha]_L,x) \df [\alpha\cdot x]_L$, за произволни $x \in \Sigma$ и $\alpha \in \Sigma^\star$.
\end{itemize}

\begin{framed}
  \begin{thm}
    $\B \cong \M$.
  \end{thm}  
\end{framed}
\begin{hint}
  Нека да дефинираме $f:Q^\M \to Q^\B$ по следния начин:
  \[f([\alpha]_L) \df \alpha^{-1}(L).\] 
  Ще докажем, че $f$ изпълнява свойствата за изоморфизъм.

  \begin{enumerate}[(1)]
  \item 
    Трябва първо да проверим, че $f$ е биектвна, т.е. $f$
    е инективна и сюрективна.
    \begin{itemize}
    \item
      Дефиницията на $f$ е зададена спрямо представител $\alpha$ на класа на еквивалентност на релацията $\approx_L$.
      \marginpar{Това е важно да се провери, защото дясната страна е дефинирана спрямо произволен представител на класа $[\alpha]_L$}
      Първо да проверим, че $f$ е функция.
      Нека $[\alpha]_L = [\beta]_L$, т.е. $\alpha^{-1}(L) = \beta^{-1}(L)$. Тогава 
      \begin{align*}
        f([\alpha]_L) & \df \alpha^{-1}(L)\\
                      & = \beta^{-1}(L) & \comment{\text{от } [\alpha]_L = [\beta]_L}\\
                      & \df f([\beta]_L).
      \end{align*}
    \item 
      Нека $[\alpha]_L \neq [\beta]_L$.
      Тогава:
      \begin{align*}
        f([\alpha]_L) & \df \alpha^{-1}(L)\\
                      & \neq \beta^{-1}(L) & \comment{\text{от }[\alpha]_L \neq [\beta]_L}\\
                      & \df f([\beta]_L).
      \end{align*}
      Оттук следва, че $f$ е {\em инективна}.
    \item
      Да разгледаме произволен елемент $N \in Q^\B$, т.е. $N$ е множество от думи и $N = \alpha^{-1}(L)$, за някоя дума $\alpha \in \Sigma^\star$.
      Понеже $f([\alpha]_L) \df \alpha^{-1}(L)$, то това означава, че $f$ е {\em сюрективна}.      
    \end{itemize}
  \item
    Лесно се съобразява, че
    \begin{align*}
      f(\qstart^\M) & = f([\varepsilon]_L) & \comment{\qstart^\M \df [\varepsilon]_L}\\
                    & \df \varepsilon^{-1}(L)\\
                    & = L \\
                    & \df \qstart^\B. & \comment{\qstart^\B \df L}
    \end{align*}
  \item
    Също не е трудно да се съобрази, че
    \begin{align*}
      [\alpha]_L \in F^\M & \dff [\alpha]_L \subseteq L\\
                          & \iff \varepsilon \in \alpha^{-1}(L)\\
                          & \iff f([\alpha]_L) \df \alpha^{-1}(L) \in F^\B.
    \end{align*}
  \item
    Имаме и свойството за произволна дума $\alpha \in \Sigma^\star$ и произволна буква $x \in \Sigma$:
    \begin{align*}
      f(\delta_\M([\alpha]_L,x)) & = f([\alpha\cdot x]_L) & \comment{\text{деф. на }\delta_\M}\\
                                 & = (\alpha\cdot x)^{-1}(L) & \comment{\text{деф. на }f}\\
                                 & = x^{-1}(\alpha^{-1}(L)) & \comment{\text{от \Prob{pullback}}}\\
                                 & = \delta_\B(\alpha^{-1}(L), x) & \comment{\text{деф. на }\delta_\B}\\
                                 & = \delta_\B(f([\alpha]_L), x), & \comment{\text{деф. на }f}
    \end{align*}
    от което следва, че $f$ е {\em изоморфизъм}.
  \end{enumerate}
\end{hint}

\begin{cor}
  Автоматът $\B$, построен по метода на Бжожовски, за регулярния език $L$ е минимален.
\end{cor}


\begin{framed}
  \begin{thm}
    \label{th:regular:isomorphic:minimal}
    Нека е даден регулярния език $L$.
    Нека $\A = \FA$ е произволен тотален автомат, за който $\L(\A) = L$ и $\abs{Q} = \abs{\Sigma^\star/_{\approx_L}}$.
    Тогава $\A \cong \M$, където $\M$ е автоматът построен според \hyperref[th:myhill-nerode]{Теоремата на Майхил-Нероуд} за езика $L$.
  \end{thm}  
\end{framed}
\begin{proof}
  Съобразете, че $\A$ е {\em свързан}, т.е. всяко състояние на $\A$ е достижимо от началното.
  Искаме да докажем, че $\A \cong \M$.
  Понеже $\A$ е свързан, за всяко състояние $q$ можем да намерим дума $\omega_q$,
  за която $\delta^\star(\qstart,\omega_q) = q$.
  Да дефинираме изображението $f:Q^\A\to Q^\M$ като 
  \[f(q) \df [\omega_q]_L.\]
  Ще докажем, че $f$ задава изоморфизъм на $\A$ върху $\M$. 
  \begin{enumerate}[(1)]
  \item
    Първо да съобразим, че $f:Q^\A \to Q^\M$ е биекция.
    \begin{itemize}
    \item
      Първо да съобразим, че ако $\delta^\star_\A(\qstart,\alpha) = q$, то $[\omega_q]_L = [\alpha]_L$.
      Понеже $\delta^\star_\A(\qstart,\alpha) = q = \delta^\star_\A(\qstart,\omega_q)$, то $\omega_q \sim_\A \alpha$.
      От \Prop{rel-finer} имаме, че
      \[\omega_q \sim_\A \alpha \implies \omega_q \approx_L \alpha.\]
      Това означава, че $[\omega_q]_L = [\alpha]_L$ и следователно $f$ е определена коректно, т.е. $f$ е {\bf функция}.
    \item
      Ще проверим, че $f$ е {\bf инективна}, т.е.
      \[(\forall q_1,q_2 \in Q)[q_1\neq q_2 \implies f(q_1) \neq f(q_2)].\]
      Да допуснем, че има състояния $q_1 \neq q_2$, за които 
      \[f(q_1) = [\omega_{q_1}]_L = [\omega_{q_2}]_L = f(q_2).\]
      Тогава $\omega_{q_1} \not\sim_\A \omega_{q_2}$ и $\omega_{q_1} \approx_L \omega_{q_2}$.
      \marginpar{\writedown Обяснете!}
      Но тогава от \Cor{upper-bound} получаваме, че $\abs{Q} = \abs{\Sigma^\star/_{\sim_\A}} > \abs{\Sigma^\star/_{\approx_L}}$,
      което противоречи с минималността на $\A$.
    \item
      За да бъде $f$ {\bf сюрективна} трябва за всеки клас $[\beta]_L$ да съществува състояние $q$, за което $f(q) = [\beta]_L$.
      Понеже $\A$ е свързан, съществува състояние $q$, за което $\delta^\star_\A(\qstart,\beta) = q$.
      Вече се убедихме, че в този случай $\beta \approx_L \omega_q$, защото $\beta \sim_\A \omega_q$.
      Тогава $f(q) = [\omega_q]_L = [\beta]_L$.
    \end{itemize}
  \item
    Понеже $\delta^\star_\A(\qstart,\varepsilon) = \qstart$,
    то е ясно, че $f(\qstart) = [\varepsilon]_L$.
  \item
    Също лесно се съобразява, че
    $q \in F^\A \iff f(q) \in F^\M$.
  \item
    За последно оставихме проверката, че $f$ наистина е {\bf изоморфизъм}:
    \begin{align*}
      f(\delta_\A(q,a)) & = f(\delta_\A(\delta^\star_\A(\qstart,\omega_q),a)) & \comment{\text{от избора на }\omega_q}\\
      & = f(\delta^\star_\A(\qstart,\omega_qa)) & \comment{\text{от деф. на }\delta^\star_\A}\\
      & = [\omega_qa]_L & \comment{\text{от деф. на }f}\\
      & = \delta^\star_{\M}([\varepsilon]_L, \omega_qa) & \comment{\text{от деф. на }\M}\\ 
      & = \delta_{\M}(\delta^\star_{\M}([\varepsilon]_L, \omega_q),a) & \comment{\text{от деф. на }\delta^\star_{\M}}\\
      & = \delta_{\M}([\omega_q]_L, a) & \comment{\text{свойство на }\delta^\star_{\M}}\\
      & = \delta_{\M}(f(q), a) & \comment{f(q) \df [\omega_q]_L}.
    \end{align*}
  \end{enumerate}
\end{proof}




%%% Local Variables:
%%% mode: latex
%%% TeX-master: "../eai"
%%% End:

\section{Метод на Бжозовски}\label{sect:regular:brzozowski}
\index{Бжозовски}

Имаме следната операция за произволна буква $a$,
\[a^{-1}(L) \df \{\omega \in \Sigma^\star \mid a\omega \in L\}.\]
Аналогично, за произволна дума $\alpha$,
\[\alpha^{-1}(L) \df \{\omega \in \Sigma^\star \mid \alpha\omega \in L\}.\]

\begin{problem}
  Докажете, че:
  \begin{enumerate}[(1)]
  \item
    $a^{-1}(L_1 \cup L_2) = a^{-1}(L_1) \cup a^{-1}(L_2)$;
  \item
    $a^{-1}(L_1 \cap L_2) = a^{-1}(L_1) \cap a^{-1}(L_2)$;
  \item
    $a^{-1}(L_1 \setminus L_2) = a^{-1}(L_1) \setminus a^{-1}(L_2)$;
  \item
    $a^{-1}(L_1 \cdot L_2) =
    \begin{cases}
      a^{-1}(L_1) \cdot L_2, & \text{ ако }\varepsilon\not\in L_1\\
      a^{-1}(L_1) \cdot L_2 \cup L_2, & \text{ ако }\varepsilon\in L_1
    \end{cases}$
  \item
    $a^{-1}(L^\star) = a^{-1}(L) \cdot L^\star$.
  \end{enumerate}
\end{problem}


\marginpar{Бжозовски \cite{brzozowski-derivatives} описва алгоритъм за строене на автомат по регулярен израз.}

Нека е даден езикът $L$. Ще покажем конструкция на детерминиран автомат $\B = \FA$,
който разпознава $L$. Ако $L$ е регулярен, то $\B$ ще бъде детерминиран краен автомат,
но ако $L$ не е регулярен, то $\B$ ще бъде детерминиран \emph{безкраен} автомат.
Конструкцията на автомата $\B$ е следната:
\marginpar{Да напомним, че имаме свойството
  \[\alpha \in L \iff \varepsilon \in \alpha^{-1}(L).\]
  Все още не ясно, че ако $L$ е регулярен, то $Q$ е крайно множество. Това ще видим след малко.
  В \Example{regular:brzozowski:an-bn} ще видим един детерминиран безкраен автомат за език, който не е регулярен.}
\begin{itemize}
\item
  Състоянията $Q$ ще бъдат от вида $q_M$, за $M \subseteq \Sigma^\star$, където:
  \[Q \df \{q_M \mid (\exists \alpha\in\Sigma^\star)[M = \alpha^{-1}(L)].\]
\item
  $\qstart \df q_L$.
\item
  За произволни езици $M$ и $N$ и буква $a$,
  \[\delta(q_M,a) \df q_N \stackrel{\text{деф}}{\iff} N = a^{-1}(M).\]
\item
  $F \df \{ q_M \in Q\mid \varepsilon \in M\}$.
\end{itemize}

\begin{proposition}\label{pr:regular:brzozowski:delta}
  За всяка дума $\alpha$ е изпълнено, че:
  \[N = \alpha^{-1}(L) \iff \delta^\star(q_L,\alpha) = q_N.\]
\end{proposition}
\begin{hint}
  Индукция по дължината на думата $\alpha$, като използвате, че
  \[(\alpha b)^{-1}(L) = b^{-1}(\alpha^{-1}(L)).\]
\end{hint}

\begin{proposition}
  За даден език $L$, нека $\B$ е детерменираният автомат построен по метода на Бжозовски.
  Тогава $L = \L(\B)$.
\end{proposition}
\begin{hint}
  Съобразете, че имаме следните еквивалентности:
  \begin{align*}
    \alpha \in L & \iff \varepsilon\in\alpha^{-1}(L) & \comment\text{нека }M \df \alpha^{-1}(L)\\
                 & \iff \varepsilon\in M\ \&\ q_M = \delta^\star(q_L,\alpha) & \comment\text{от \Proposition{regular:brzozowski:delta}}\\
                 & \iff \delta^\star(\qstart,\alpha) \in F. & \comment \qstart \df q_L
  \end{align*}
\end{hint}

%%% Local Variables:
%%% mode: latex
%%% TeX-master: "../eai"
%%% End:

\newpage
\subsection{Примерни задачи}

\begin{problem}
  Постройте автомат $\B$ по метода на Бжозовски за регулярния език
  \[L = \L(\mathbf{((a+b)^+\cdot a)^\star}).\]
\end{problem}
\ExtraMaterial{
\begin{solution}
  \begin{multicols}{2}
  \begin{itemize}
\item
  Ясно е, че ще започнем от началното състояние $\hat{L}$.
  Удобно е да имаме предвид следното представяне
  \begin{align*}
    L & = \{\varepsilon\} \cup \{a,b\}^+ a L\\
      & = \{\varepsilon\} \cup \{a,b\} \cdot \{a,b\}^\star a L\\
      & = \{\varepsilon\} \cup a\{a,b\}^\star aL \cup b\{a,b\}^\star aL
  \end{align*}
\item
  Сега като имаме това представяне на $L$, лесно се съобразява, че
  \[a^{-1}(L) = b^{-1}(L) = \{a,b\}^\star aL.\]
  Нека положим $M \df \{a,b\}^\star aL$.
  Лесно се съобразява, че $M \neq L$, защото $\varepsilon \in L$, но $\varepsilon \not\in M$.
  Това означава, че имаме ново състояние $\hat{M}$ и
  \begin{align*}
    & \delta(\hat{L},a) \df \hat{M} & \comment\text{ защото }a^{-1}(L) = M\\
    & \delta(\hat{L},b) \df \hat{M}. & \comment\text{ защото }b^{-1}(L) = M
  \end{align*}
\item
  За следващата стъпка е удобно да представим езика $M$ по следния начин:
  \begin{align*}
    M & = \{a,b\}^\star aL\\
      & = aL \cup \{a,b\}^+aL\\
      & = aL \cup \{a,b\}\cdot \{a,b\}^\star aL\\
      & = aL \cup \{a,b\}\cdot M\\
      & = aL \cup aM \cup bM
  \end{align*}
  От това представяне на $M$ веднага се съобразява, че
  \begin{align*}
    & a^{-1}(M) = L \cup M\\
    & b^{-1}(M) = M.
  \end{align*}
  Нека да положим $N \df L \cup M$.
  Имаме, че $N \neq L$, защото $a\in N$, но $a \not\in L$.
  Освен това, $N \neq M$, защото $\varepsilon \in N$, но $\varepsilon \not\in M$.
  Това означава, че имаме ново състояние $\hat{N}$ и тогава
  \begin{align*}
    & \delta(\hat{M},a) \df \hat{N} & \comment\text{ защото }a^{-1}(M) = N\\
    & \delta(\hat{M},b) \df \hat{N} & \comment\text{ защото }b^{-1}(M) = M.
  \end{align*}
\item
  Да разгледаме следното представяне:
  \begin{align*}
    N & = L \cup M \\
      & = \{\varepsilon\} \cup aM \cup bM \cup M\\
      & = \{\varepsilon\} \cup aM \cup bM \cup aL\\
      & = \{\varepsilon\} \cup aN \cup bM.
  \end{align*}
  Веднага можем да съобразим, че
  \begin{align*}
    & a^{-1}(N) = N\\
    & b^{-1}(N) = M.
  \end{align*}
  Сега полагаме:
  \begin{align*}
    & \delta(\hat{N},a) \df \hat{N} & \comment\text{ защото }a^{-1}(N) = N\\
    & \delta(\hat{N},b) \df \hat{M} & \comment\text{ защото }b^{-1}(N) = M.
  \end{align*}
\item
  Нямаме повече нови състояния. Следователно,
  \[Q \df \{\hat{L}, \hat{M}, \hat{N}\}.\]
\item
  Понеже $\varepsilon \in L$ и $\varepsilon \in N$ е ясно, че
  \[F = \{\hat{L},\hat{N}\}.\]
  Сега вече сме готови да нарисуваме картинка на автомата.
\end{itemize}


% \begin{framed}
\begin{figure}[H]
  % \begin{subfigure}[b]{.45\textwidth}
  %   \centering
    \begin{tikzpicture}[framed,->,>=stealth,thick,node distance=70pt,scale=0.8, every node/.style={scale=0.8}]
      \tikzstyle{every state}=[circle,font=\small]
      \node[initial above, state,accepting]   (L) {$\hat{L}$};
      \node[state]                            (M) [right of=L]{$\hat{M}$};
      \node[state,accepting]                  (N) [right of=M]{$\hat{N}$};
      
      \path 
      (L) edge [bend left=15] node [above] {$a,b$} (M)
      (M) edge [loop above] node [above]   {$b$}   (M)
      (M) edge [bend left=15] node [above] {$a$}   (N)
      (N) edge [bend left=15] node [below] {$b$}   (M)
      (N) edge [loop above] node [above]   {$a$}   (N);
    \end{tikzpicture}
    \caption{\scriptsize{Автомат за езика $L$ по метода на Бжозовски.}}
    % \end{subfigure}
    \end{figure}
  % \quad
%   ~
%   \quad
%   \begin{subfigure}[b]{.45\textwidth}
%     \centering
%     \begin{tikzpicture}[framed,->,>=stealth,thick,node distance=55pt]
%       \tikzstyle{every state}=[circle,minimum size=20pt,auto]
      
%       \node[initial above, state,accepting]   (L) {$[\varepsilon]_L$};
%       \node[state]                            (M) [right of=L]{$[a]_L$};
%       \node[state,accepting]                  (N) [right of=M]{$[aa]_L$};
      
%       \path 
%       (L) edge [bend left=15] node [above] {$a,b$} (M)
%       (M) edge [loop above] node [above]   {$b$}   (M)
%       (M) edge [bend left=15] node [above] {$a$}   (N)
%       (N) edge [bend left=15] node [below] {$b$}   (M)
%       (N) edge [loop above] node [above]   {$a$}   (N);
%     \end{tikzpicture}
%     \caption{Минимален автомат $\M$ за езика $L$ по метода на Майхил-Нероуд.}
%   \end{subfigure}
% \end{figure}
\end{multicols}
  \end{solution}
}

\begin{problem}
  Постройте автомат $\B$ по метода на Бжозовски, който рапознава регулярния език
  \[L = \L(\mathbf{a\cdot(a+b)^\star\cdot b}).\]
\end{problem}
\ExtraMaterial{
\begin{solution}
  \begin{multicols}{2}
  \begin{itemize}
  \item 
    $a^{-1}(L) = \{a,b\}^\star b \df M$.
    Имаме, че $M \neq L$, защото $b \in M$, но $b \not\in L$.
    Тогава
    \begin{align*}
      & \delta(\hat{L}, a) \df \hat{M} & \comment\text{ защото }a^{-1}(L) = M\\
      & \delta(\hat{L},b) \df \hat{\emptyset} & \comment\text{ защото }b^{-1}(L) = \emptyset\\
    \end{align*}
  \item    
    За по-нататък ще е удобно да представим $M$ по следния начин:
    \begin{align*}
      M & = a\cdot \{a,b\}^\star \cdot b \cup b\cdot \{a,b\}^\star \cdot b \cup \{b\}\\
        & = aM \cup bM \cup \{b\}.
    \end{align*}
    Сега е ясно, че $a^{-1}(M) = M$, а $b^{-1}(M) = \{\varepsilon\} \cup M$.
    Нека да положим $N \df \{\varepsilon\} \cup M$.
    Имаме, че $N \neq L$ и $N \neq M$, защото $\varepsilon \in N$, но $\varepsilon \not\in L$ и $\varepsilon \not\in M$.
    Тогава
    \begin{align*}
      & \delta(\hat{M},a) \df \hat{M} & \comment\text{ защото }a^{-1}(M) = M\\
      & \delta(\hat{M},b) \df \hat{N} & \comment\text{ защото }b^{-1}(M) = N
    \end{align*}
  \item
    Можем да представим езика $N$ по следния начин:
    \[N = \{\varepsilon\} \cup aM \cup bM \cup \{b\}.\]
    Тогава имаме, че:
    \begin{align*}
      & \delta(\hat{N},a) \df \hat{M} & \comment\text{ защото } a^{-1}(N) = M\\
      & \delta(\hat{N},b) \df \hat{N} & \comment\text{ защото } b^{-1}(N) = M.
    \end{align*}
  \item
    Завършваме с дефиницията на функцията на преходите като:
    \begin{align*}
      & \delta(\hat{\emptyset},a) \df \hat{\emptyset} & \comment\text{ защото }a^{-1}(\emptyset) = \emptyset\\
      & \delta(\hat{\emptyset},b) \df \hat{\emptyset} & \comment\text{ защото }b^{-1}(\emptyset) = \emptyset.
    \end{align*}
  \item
    Съобразете сами, че $F = \{\hat{N}\}$.
  \end{itemize}    
  \begin{figure}[H]
    \centering
    \begin{tikzpicture}[framed,->,>=stealth,thick,node distance=55pt,scale=0.8, every node/.style={scale=0.8}]
      \tikzstyle{every state}=[circle]
      
      \node[initial, state]                   (L) {$\hat{L}$};
      \node[state]                            (M) [above right of=L]{$\hat{M}$};
      \node[state]                            (E) [below right of=L]{$\hat{\emptyset}$};
      \node[state,accepting]                  (N) [right of=M]{$\hat{N}$};
      
      \path 
      (L) edge [bend left=15]  node [left] {$a$} (M)
      (L) edge [bend right=15] node [left] {$b$} (E)
      (E) edge [loop right]    node [right] {$a,b$} (E) 
      (M) edge [bend right=15] node [below] {$b$} (N)
      (M) edge [loop above]    node [above] {$a$} (M)
      (N) edge [bend right=30] node [above] {$a$} (M)
      (N) edge [loop above]    node [above] {$b$} (N);
    \end{tikzpicture}
    \caption{\scriptsize{Автомат за езика $\L(\mathbf{a\cdot (a+b)^\star\cdot b})$ по метода на Бжозовски.}}
  \end{figure}
  \end{multicols}
\end{solution}
}

\begin{problem}
  Постройте автомат по метода на Бжозовски за регулярния език
  \[L = \{\omega \in \{a,b\}^\star \mid \card{\omega}{a}\text{ е четно и }\card{\omega}{b} = 1\}.\]
\end{problem}
\ExtraMaterial{
  \begin{solution}
    \begin{multicols}{2}
    \begin{align*}
        a^{-1}(L) & = \{\omega \in \{a,b\}^\star \mid \card{\omega}{a} \text{ е нечетно и }\card{\omega}{b} = 1\}\\
                  & \df M.
      \end{align*}
      Ясно е, че $M \neq L$, защото например $aab \in L$, но $aab \not\in M$.
      \begin{align*}
        b^{-1}(L) & = \{\omega \in \{a,b\}^\star \mid \card{\omega}{a} \text{ е четно и } \card{\omega}{b} = 0\}\\
                  & \df N.
      \end{align*}
      Ясно е, че $N \neq M$ и $N \neq L$, защото $aa \in N$, но $aa \notin M$ и $aa \notin L$.
      \begin{align*}
        a^{-1}(M) & = L;\\
        b^{-1}(M) & = \{\omega \in \{a,b\}^\star \mid \card{\omega}{a} \text{ е нечетно и } \card{\omega}{b} = 0\}\\
                  & \df P.
      \end{align*}
      Ясно е, че $P \neq M,L,N$, защото $a \in P$, но $a \notin M,L,N$.
      \begin{align*}
        a^{-1}(N) & = P\\
        b^{-1}(N) & = \emptyset\\
        a^{-1}(P) & = N\\
        b^{-1}(P) & = \emptyset.
      \end{align*}

      \begin{figure}[H]
    \centering
    \begin{tikzpicture}[framed,->,>=stealth,thick,node distance=65pt,scale=0.8, every node/.style={scale=0.8}]
      \tikzstyle{every state}=[circle]
      
      \node[initial below, state]  (L) {$\hat{L}$};
      \node[state]                 (M) [above of=L]{$\hat{M}$};
      \node[state]                 (P) [right of=M]{$\hat{P}$};
      \node[state, accepting]      (N) [below of=P]{$\hat{N}$};
      \node[state]                 (E) [above right of=N]{$\hat{\emptyset}$};
            
      \path 
      (L) edge [bend right=15]  node [right] {$a$} (M)
      (M) edge [bend right=15]  node [left]  {$a$} (L)
      (L) edge [bend right=15]  node [below] {$b$} (N)
      (M) edge [bend left=15]   node [above] {$b$} (P)
      (N) edge [bend left=15]   node [left]  {$a$} (P)
      (P) edge [bend left=15]   node [right] {$a$} (N)
      (P) edge [bend left=15]   node [above] {$b$} (E)
      (N) edge [bend right=15]  node [below] {$b$} (E)
      (E) edge [loop above]     node [above] {$a,b$} (E);      
    \end{tikzpicture}
    \caption{\scriptsize{Автомат, който приема думи с четен брой $a$ и точно едно $b$, получен чрез метода на Бжозовски.}}
  \end{figure}
  \end{multicols}
\end{solution}
}

\begin{problem}
  Да припомним, че в \Problem{regular:dfa:binary} се искаше да се докаже, че езикът 
  \[L = \{\alpha \in \{0,1\}^\star \mid \bin{\alpha} \equiv 2 \bmod 3\}\]
  е регулярен.
  Ние направихме това като построихме автомат за $L$ и доказахме, че той разпознава $L$.
  Сега пък постройте автомат за $L$ по метода на Бжозовски.
\end{problem}
\ExtraMaterial{
  \begin{solution}
    \begin{multicols}{2}
      За целта ще ни трябва алтернативна дефиниция на $\bin{\alpha}$.
  За една дума $\alpha \in \{0,1\}^\star$, можем да дадем следната дефиниция на $\bin{\alpha}$:
  \begin{itemize}
  \item
    $\bin\varepsilon = 0$,
  \item
    $\bin{0\alpha} = \bin{\alpha}$,
  \item
    $\bin{1\alpha} = 2^{|\alpha|} + \bin{\alpha}$.
  \end{itemize}
  Тогава имаме, че:
    \begin{align*}
      0^{-1}(L) & = \{\alpha \in \{0,1\}^\star \mid 0\alpha \in L\}\\
                & = \{\alpha \in \{0,1\}^\star \mid \bin{0\alpha} \equiv 2 \bmod 3\}\\
                & = \{\alpha \in \{0,1\}^\star \mid \bin{\alpha} \equiv 2 \bmod 3\}\\
                & = L.
    \end{align*}    
    % % \columnbreak
    \begin{align*}
      1^{-1}(L) & = \{\alpha \in \{0,1\}^\star \mid 1\alpha \in L\}\\
                & = \{\alpha \in \{0,1\}^\star \mid \bin{1\alpha} \equiv 2 \bmod 3\}\\
                & = \{\alpha \in \{0,1\}^\star \mid 2^{|\alpha|} + \bin{\alpha} \equiv 2 \bmod 3\}\\
                & \df M.
    \end{align*}
  % \end{multicols}
Лесно се съобразява, че $L \neq M$, защото например за думата $\alpha = 10$
имаме, че $\alpha \in L$, но $\alpha \not\in M$.
Продължаваме нататък:
% \begin{multicols}{2}
\begin{align*}
  0^{-1}(M) & = \{\alpha \in \{0,1\}^\star \mid 0\alpha \in M\}\\
              & = \{\alpha \in \{0,1\}^\star \mid 2^{|0\alpha|} + \bin{0\alpha} \equiv 2 \bmod 3\}\\
              & = \{\alpha \in \{0,1\}^\star \mid 2\cdot 2^{|\alpha|} + \bin{\alpha} \equiv 2 \bmod 3\}\\
              & \df N.
\end{align*}
% \vfill
% \columnbreak
\begin{align*}
  1^{-1}(M) & = \{\alpha \in \{0,1\}^\star \mid 1\alpha \in M\}\\
            & = \{\alpha \in \{0,1\}^\star \mid 2^{|1\alpha|} + \bin{1\alpha} \equiv 2 \bmod 3\}\\
            & = \{\alpha \in \{0,1\}^\star \mid 2\cdot 2^{|\alpha|} + 2^{|\alpha|} + \bin{\alpha} \equiv 2 \bmod 3\}\\
            & = \{\alpha \in \{0,1\}^\star \mid 3\cdot 2^{|\alpha|} + \bin{\alpha} \equiv 2 \bmod 3\}\\
            & = \{\alpha \in \{0,1\}^\star \mid \bin{\alpha} \equiv 2 \bmod 3\}\\
            & = L.
\end{align*}
% \end{multicols}
Да проверим, че $N \neq L$ и $N \neq M$.
Това отново е лесно. Нека например да разгледаме $\alpha = 11$.
Непосредствено се проверява, че $\alpha \in N$, $\alpha \not\in L$, $\alpha \not\in M$.
Продължаваме нататък:
% \begin{multicols}{2}
\begin{align*}
  0^{-1}(N) & = \{\alpha \in \{0,1\}^\star \mid 0\alpha \in N\}\\
            & = \{\alpha \in \{0,1\}^\star \mid 2\cdot 2^{|0\alpha|} + \bin{0\alpha} \equiv 2 \bmod 3\}\\
            & = \{\alpha \in \{0,1\}^\star \mid 4\cdot 2^{|\alpha|} + \bin{\alpha} \equiv 2 \bmod 3\}\\
            & = \{\alpha \in \{0,1\}^\star \mid 3\cdot 2^{|\alpha|} + 2^{|\alpha|} + \bin{\alpha} \equiv 2 \bmod 3\}\\
            & = \{\alpha \in \{0,1\}^\star \mid 2^{|\alpha|} + \bin{\alpha} \equiv 2 \bmod 3\}\\
            & = M\\
% \end{align*}
% \begin{align*}
  1^{-1}(N) & = \{\alpha \in \{0,1\}^\star \mid 1\alpha \in N\}\\
            & = \{\alpha \in \{0,1\}^\star \mid 2\cdot 2^{|1\alpha|} + \bin{1\alpha} \equiv 2 \bmod 3\}\\
            & = \{\alpha \in \{0,1\}^\star \mid 4\cdot 2^{|\alpha|} + 2^{|\alpha|} + \bin{\alpha} \equiv 2 \bmod 3\}\\
            & = \{\alpha \in \{0,1\}^\star \mid 3\cdot 2^{|\alpha|} + 2\cdot 2^{|\alpha|} + \bin{\alpha} \equiv 2 \bmod 3\}\\
            & = \{\alpha \in \{0,1\}^\star \mid 2\cdot 2^{|\alpha|} + \bin{\alpha} \equiv 2 \bmod 3\}\\
            & = N.
\end{align*}
% Така можем да получим автомат за езика $L$, където всяко състояние е свързано с език.

\begin{figure}[H]
  \begin{center}
    \begin{tikzpicture}[framed,->,>=stealth,thick,node distance=55pt,scale=0.8, every node/.style={scale=0.8}]
      \tikzstyle{every state}=[circle]
      
      \node[initial below,state] (0) {$\hat{L}$};
      \node[state]               (1) [right of=0]{$\hat{M}$};
      \node[accepting, state]    (2) [right of=1]{$\hat{N}$};
      
      \path 
      (0) edge  [loop above]    node [above]  {$0$} (0)
      (0) edge  [bend left=15]  node [above]  {$1$} (1)
      (2) edge  [bend left=15]  node [below]  {$0$} (1)
      (1) edge  [bend left=15]  node [below]  {$1$} (0)
      (1) edge  [bend left=15]  node [above]  {$0$} (2)
      (2) edge  [loop above]    node [above]  {$1$} (2);
    \end{tikzpicture}
  \end{center}
  \caption{\scriptsize{$\L(\A) = \{\alpha\in\{0,1\}^\star \mid \ov{\alpha}_{(2)} \equiv 2\ (\bmod\ 3)\}$.}}
\end{figure}
\end{multicols}
\end{solution}
}

\newpage
\begin{example}\label{ex:regular:brzozowski:an-bn}
  Да разгледаме езика $L = \{a^nb^n\mid n \in \Nat\}$.
  Ние вече знаем от \Problem{regular:pumping:an-bn}, че $L$ не е регулярен език.
  Да се опитаме да построим автомат, който го разпознава.
  \ExtraMaterial{
    \begin{multicols}{2}
      Нека положим
      \[L_k \df \{a^nb^{n+k}\mid n \in \Nat\}.\]
      Да видим какво се получава като приложим процедурата за строене 
      на минимален автомат.
      \begin{itemize}
    \item 
      $a^{-1}(L) = L_1$;
    \item
      $b^{-1}(L) = \emptyset$;
    \item
      $a^{-1}(L_1) = L_2$;
    \item
      $b^{-1}(L_1) = \{\varepsilon\}$;
    \item
      $a^{-1}(\{\varepsilon\}) = b^{-1}(\{\varepsilon\}) = \emptyset$;
    \item
      Лесно можем да докажем, че за всяко $k$ е изпълнено, че $a^{-1}(L_k) = L_{k+1}$.
    \item
      Лесно се вижда, че $b^{-1}(L_{k+1}) = \{b^k\}$, за всяко $k$.
    \item
      Ясно е, че $b^{-1}(\{b^{k}\}) = \{b^{k-1}\}$, за всяко $k \geq 1$.
    \end{itemize}    
    Получаваме, че езикът $L$ се разпознава от автомат с {\em безкрайно много състояния}.
  
% \begin{framed}  
  \begin{figure}[H]
    \centering
    \begin{tikzpicture}[framed,->,>=stealth,thick,node distance=60pt,scale=0.8, every node/.style={scale=0.8}]
      % \pgftransformscale{.4}
      \tikzstyle{every state}=[circle]% ,minimum size=15pt,auto]
      
      \node[state, initial above]             (0) {$L$};
      \node[state]                            (1) [right of=0]{$L_1$};
      \node[state]                            (2) [right of=1]{$L_2$};
      \node[state]                            (3) [right of=2]{$L_3$};
      \node[state,accepting]                  (A) [below of=1]{$\{\varepsilon\}$};
      \node[state]                            (B) [below right of=1]{$\{b\}$};
      \node[state]                            (BB) [below right of=2]{$\{bb\}$};
      \node[state]                            (E) [below of=A]{$\emptyset$};
      
      \coordinate[right of=3] (4);
      \coordinate[below right of=3] (BBB);
      \coordinate[below of=4] (BBBA);

      \path 
      (0) edge [bend left=15]   node [above] {$a$} (1)
      (1) edge [bend left=15]   node [above] {$a$} (2)
      (2) edge [bend left=15]   node [above] {$a$} (3)
      (0) edge [bend right=30]  node [left] {$b$} (E)
      (E) edge [loop left]      node [left] {$a,b$} (E)
      (1) edge [bend right=30]  node [left] {$b$} (A)
      (2) edge [bend right=15]  node [left] {$b$} (B)
      (3) edge [bend right=15]  node [left] {$b$} (BB)
      (B) edge [bend right=15]  node [above] {$b$} (A)
      (B) edge [bend left=15]  node [right] {$a$} (E)
      (A) edge [bend right=15]   node [right] {$a,b$} (E)
      (BB) edge [bend right=15] node [above] {$b$} (B)
      (BB) edge [bend left=15]  node [below] {$a$} (E);
      
      \draw [dashed,->,shorten >=0pt] (3) to[bend left=15] node[auto] {$a$} (4);
      \draw [dashed,->,shorten >=0pt] (BBB) to[bend right=15] node[above] {$b$} (BB);
      \draw [dashed,->,shorten >=0pt] (BBBA) to[bend left=30] node[below] {$a$} (E);
    \end{tikzpicture}
    \caption{\scriptsize{{\em Безкраен} автомат за $\{a^nb^n \mid n \in \Nat\}$ построен по метода на Бжозовски.}}
  \end{figure}
    \end{multicols}  
}

% \end{framed}
\end{example}

%%% Local Variables:
%%% mode: latex
%%% TeX-master: "../eai"
%%% End:

\section{Минимален автомат}
\label{sect:regular:brzozowski-minimal}

Сега ще видим, че автоматът на Бжозовски $\B$ за даден регулярен език $L$ е в известен смисъл най-добрият възможен.
Накратко, $\B$ има най-малкия възможен брой състояния измежду всички детерминирани крайни автомати, които разпознават $L$. За да успеем да видим това, първо трябва да се подготвим.

Нека $\A = \FA$ е ДКА. За всяко състояние $q$ на $\A$ да разгледаме езика
\[\L_\A(q) \df \{\omega \in \Sigma^\star \mid \delta^\star(q,\alpha) \in F\}.\]
В частност имаме, че $\L(\A) = \L_\A(\qstart)$.
Без ограничение на общността, нека приемем, че винаги разглеждаме само свързани ДКА $\A$, т.е.
всяко състояние е достижимо от началното.
\mynote{Тук използваме, че $\delta$ е тотална функция. За някои състояния $p$ може да съществуват безкрайно много думи $\alpha$, за които $q_\alpha = p$.}
Нека за всяка дума $\alpha$ да положим $q_\alpha \df \delta^\star(\qstart,\alpha)$.
Понеже $\A$ е свързан, то всяко състояние на $\A$ може да се разглежда като $q_\alpha$ за някоя дума $\alpha$.

\begin{proposition}\label{pr:well-defined-pullback}
  Нека $L = \L(\A)$. Тогава за всяка дума $\alpha$ е изпълнено, че:
  \[\L_\A(q_\alpha) = \alpha^{-1}(L).\]
\end{proposition}
\begin{proof}
  За произволна дума $\omega$ имаме следните еквивалентности:
  \begin{align*}
    \omega \in \L_\A(q_\alpha) & \iff \delta^\star(q_\alpha,\omega) \in F & \comment \text{от деф. на }\L_\A(q_\alpha)\\
                               & \iff \delta^\star(\delta^\star(\qstart,\alpha),\omega) \in F & \comment q_\alpha \df \delta^\star(\qstart,\alpha)\\
                               & \iff \delta^\star(\qstart,\alpha\omega) \in F & \comment\text{\Proposition{dfa:delta-star}}\\
                               & \iff \alpha\omega \in \L(\A) & \comment \text{деф. на }\L(\A)\\
                               & \iff \alpha\omega \in L & \comment{L = \L(\A)}\\
                               & \iff \omega \in \alpha^{-1}(L). & \comment\text{деф. на }\alpha^{-1}(L)
  \end{align*}
\end{proof}

\begin{important}
  \begin{proposition}
    Нека $\B$ е автомат на Бжозовски за езика $L$. Тогава за произволно състояние $M$ на $\B$ имаме равенството:
    \[\L_\B(\smallunderbrace{M}_{\text{\clap{състояние}}}) = \smallunderbrace{M}_{\text{\clap{език}}}.\]
  \end{proposition}  
\end{important}
\begin{proof}
  Понеже за произволна дума $\alpha$ имаме еквивалентностите
  \begin{align*}
    \alpha \in \L_\B(M) & \iff \delta^\star_\B(M,\alpha) \in F^\B\\
                        & \iff \varepsilon \in \alpha^{-1}(M) & \comment\text{деф. на }F^\B\\
                        & \iff \alpha \in M, & \comment\text{от \Problem{language-pullback}}
  \end{align*}
  заключаваме, че $\L_\B(M) = M$.
\end{proof}


\begin{proposition}\label{pr:surjective-cardinality}
  Нека $A$ и $B$ са множества, като $A$ е крайно, за които съществува \emph{сюрективна} функция $f: A \to B$.
  Тогава $|B| \leq |A|$.
\end{proposition}
\begin{hint}
  Понеже $A$ е крайно множество, можем да изброим елементите му в редица.
  Нека $A = \{a_0,a_1, \dots, a_{n-1}\}$. Разгледайте $g:B \to A$, където
  \mynote{Вярно ли е това твърдение, ако $A$ е безкрайно ?}
  \[g(b) \df a_m\text{ за }m = \min\{i < n \mid f(a_i) = b\}.\]
  Да отбележим, че дефиницията на $g$ е коректна, защото множеството $\{i < n \mid f(a_i) = b\}$ е непразно, понеже $f$ е сюректвина.
  Докажете, че $g$ е инективна.
\end{hint}

\begin{lemma}
  \label{lem:brzozowski:surjective}
  Нека $L$ е регулярен език и $\A$ е ДКА, който разпознава $L$,
  а $\B$ е автоматът на Бжозовски за $L$. Тогава $|Q^\B| \leq |Q^\A|$.
\end{lemma}
\mynote{Тази лема ни казва, че автоматът на Бжозовски има възможно най-малкия брой състояния измежду всички ДКА разпознаващи $L$.}
\begin{proof}
  Да разгледаме функцията $f:Q^\A \to Q^\B$ зададена по следния начин:
  \begin{equation}
    \label{eq:reg:brzozowski-minimal:f}
    f(q) \df \L_\A(q).
  \end{equation}
  Първо, да видим защо за всяко състояние $q \in Q^\A$, то $f(q) \in Q^\B$.
  Да напомним, че приехме още в началото на раздела приехме, че $\A$ е свързан автомат.
  Това означава, че за всяко състояние $p$, съществува дума $\alpha$, за която $p = \delta^\star_\A(\qstart^\A,\alpha)$,
  т.е. $p = q_\alpha$ според означението, което въведохме в началото на \Section{regular:brzozowski-minimal}.
  Тогава от \Proposition{well-defined-pullback} следва, че $f(q_\alpha) = \alpha^{-1}(L) \in Q^\B$, защото $\alpha^{-1}(L) = \L_\A(q_\alpha)$.
  
  % нека първо да съобразим защо $f$ е добре дефинирана.% , т.е. защо дефиницията (\ref{eq:reg:brzozowski-minimal:f}) задава функция.
  % Да напомним, че приехме, че $\A$ е свързан автомат. Тогава за произволно състояние $p$,
  % нека $\alpha$ е една дума, за която $p = \delta^\star(\qstart,\alpha)$, т.е. $p = q_\alpha$
  % според означението, което въведохме в началото на \Section{regular:brzozowski-minimal}.
  % Тогава от \Proposition{well-defined-pullback} следва, че $f(q_\alpha) = \alpha^{-1}(L)$, защото $\alpha^{-1}(L) = \L_\A(q_\alpha)$.

  Второ, да видим, че $f$ е сюрективна функция. За тази цел, да разгледаме произволно състояние $M \in Q^\B$, което означава, че има дума $\alpha$, за която $M = \alpha^{-1}(L)$.
  Отново според \Proposition{well-defined-pullback}, $f(q_\alpha) = \L_\A(q_\alpha) = \alpha^{-1} = M$.
  Сега от \Proposition{surjective-cardinality} можем да заключим, че
  \[|Q^\B| \leq |Q^\A|.\]
\end{proof}

% Така получаваме следния критерий за проверка дали даден автомат е минимален.


От \Lemma{brzozowski:surjective} получаваме следния критерий за проверка дали даден език е регулярен.

\begin{framed}
  \begin{corollary}[Критерий за регулярност на език]
    \label{cor:brzozowski:finite}
    Един език $L$ е регулярен точно тогава, когато автоматът $\B$ на Бжозовски за $L$, има крайно много състояния.
  \end{corollary}
\end{framed}
\begin{proof}
  Нека $L$ е регулярен. Тогава $L = \L(\A)$ за някой ДКА $\A$. Да напомним, че от \Proposition{brzozowski:language} също имаме и $\L(\B) = L$.
  Понеже $|Q^\B| \leq |Q^\A|$, то $\B$ има краен брой състояния.
  Обратно, ако $\B$ има крайно много състояния, то тогава $\B$ е ДКА. Понеже $\L(\B) = L$, то $L$ е автоматен и следователно регулярен.
\end{proof}


\begin{important}
  \begin{corollary}
    Нека $L$ е регулярен език. Тогава автоматът $\B$, построен по метода на Бжозовски за $L$, има минималния възможен брой състояния
    измежду всички детерминирани крайни автомати разпознаващи $L$.
  \end{corollary}  
\end{important}

\begin{proposition}\label{pr:surjection-bijection}
  Нека $A$ и $B$ са крайни равномощни множества.
  Докажете, че ако $g:A \to B$ е сюрекция, то $g$ е биекция.
\end{proposition}
\mynote{Ясно е, че щом $A$ и $B$ са равномощни, то има биекция между тях. Тук доказваме, че всяка сюрекция между тях е също така и биекция.}
\begin{proof}
  Нека $A = \{a_0,\dots,a_{n-1}\}$ и $B = \{b_0,\dots,b_{n-1}\}$.
  Нека, за всеки индекс $i < n$ да положим
  \[A_i \df \{a_j \in A \mid g(a_j) = b_i\}.\]
  Щом $g$ е сюрекция, то $A_i \neq \emptyset$ за всеки индекс $i < n$.
  Понеже $g$ е функция, то $A_i \cap A_j = \emptyset$ за всеки два различни индекса $i$ и $j$.
  \mynote{Да напомним, че от курса по Дискретна математика имаме формулата $|X\cup Y| = |X| + |Y| + |X \cap Y|$. Просто в нашия случай $|X \cap Y| = 0$.}
  Това означава, че
  \[n = |A| = |\bigcup_{i<n} A_i| = \sum_{i<n}|A_i|.\]
  Оттук следва, че щом за всяко $i$ имаме, че $|A_i| \neq 0$, то $|A_i| = 1$.
  Заключаваме, че $g$ е инекция, защото в противен случай щяхме да имаме някое $i$, за което $|A_i| > 1$.
\end{proof}

\begin{framed}
  \begin{theorem}\label{th:brzozowski-minimal:unique}
    За всеки регулярен език $L$ съществува единствен минимален ДКА с точност до изоморфизъм.
  \end{theorem}  
\end{framed}
\mynote{С други думи, ако имаме два минимални ДКА за $L$, то можем да получим единия автомат от другия чрез внимателно преименуване на състоянията.}
\begin{proof}
  Вече знаем, че автоматът $\B$, построен по метода на Бжозовски за $L$, има минималния възможен брой състояния.
  Нека $\A$ е друг ДКА разпознаващ $L$ и $|Q^\A| = |Q^\B|$. Трябва да докажем, че $\A \cong \B$.
  Ясно е, че $\A$ е свързан автомат, т.е. всяко състояние $p$ на $\A$ е от вида $p = q_\alpha$.
  В противен случай, $\A$ нямаше да бъде минимален.
  Да разгледаме функцията $f:Q^\A \to Q^\B$, където
  \begin{equation}
    \label{eq:reg:brzozowski-minimal:f:2}
    f(p) \df \L_\A(p).
  \end{equation}
  От \Lemma{brzozowski:surjective} знаем, че $f$ е сюрективна.
  % От \Proposition{well-defined-pullback} знаем, че $f(q_\alpha) = \L_\A(q_\alpha) = \alpha^{-1}(L)$ следва, че $f$ е сюрективна.
  Понеже $|Q^\A| = |Q^\B|$, то от \Proposition{surjection-bijection} имаме, че $f$ е всъщност биекция.
  Остава да видим защо $\A \cong_f \B$.
  Да напомним, че състоянията на $Q^\A$ можем да ги разглеждаме като $q_\alpha$ и тогава $\L_\A(q_\alpha) = \alpha^{-1}(L)$ от \Proposition{well-defined-pullback}.
  \begin{itemize}
  \item
    За началното състояние имаме, че:
    \begin{align*}
      f(\qstart^\A) & = f(q_\varepsilon) & \comment{\qstart^\A = q_\varepsilon}\\
                    & = \L_\A(q_\varepsilon) & \comment\text{от деф. на }f\\
                    & = \varepsilon^{-1}(L) & \comment\text{\Proposition{well-defined-pullback}}\\
                    & = L.
    \end{align*}    
  \item
    За финалните състояния имаме, че:
    \begin{align*}
      q_\alpha \in F^\A & \iff \delta^\star_\A(\qstart,\alpha) \in F^\A & \comment q_\alpha = \delta^\star_\A(\qstart,\alpha)\\
                        & \iff \alpha \in \L(\A) & \comment\text{деф. на }\L(\A)\\
                        & \iff \varepsilon \in \alpha^{-1}(L) & \comment{L = \L(\A)}\\
                        & \iff \alpha^{-1}(L) \in F^\B. & \comment\text{деф. на }F^\B
    \end{align*}
  \item
    \mynote{
     \begin{tikzpicture}[->,>=stealth,thick,node distance=45pt,initial text=начало,scale=0.8,every node/.style={scale=0.8}]
        \tikzstyle{every state}=[circle,minimum size=20pt,auto]
        
        \node[state,initial above]          (1)              {$q_\varepsilon$};
        \node[state]                  (2) [right of=1] {$q_\alpha$};
        \node[state, inner sep=2pt]                  (3) [right of=2] {$q_{\alpha b}$};

        \node[state, initial below, inner sep=2pt]   (4) [below of=1] {$L$};
        \node[state, inner sep=2pt]                  (5) [below of=2] {$L_\alpha$};
        \node[state, inner sep=2pt]                  (6) [below of=3] {$L_{\alpha b}$};
        
        \path
        (2) edge [bend left=15]  node [above] {$b$} (3)
        (5) edge [bend right=15] node [below] {$b$} (6)
        (1) edge [dashed, bend right=15] node [left]  {$f$} (4)
        (2) edge [dashed, bend right=15]  node [left] {$f$} (5)
        (3) edge [dashed, bend left=15]  node [right] {$f$} (6);

        \draw [photon] (1) -- node [above] {$\alpha$} (2);
        \draw [photon] (4) -- node [above] {$\alpha$} (5);
      \end{tikzpicture}

      Тук сме означили
      \begin{align*}
        & L_{\alpha} = \alpha^{-1}(L)\\
        & L_{\alpha b} = (\alpha b)^{-1}(L).
      \end{align*}
    }
    Остава да докажем, че за произволна буква $b$ и произволни състояния $q,p \in Q_1$ е изпълнено, че:
    \begin{equation}
      \label{eq:brzozowski-minimal:eq}
      \delta_\A(q,b) = p\ \iff\ \delta_\B(f(q),b) = f(p).
    \end{equation}
    Понеже всяко състояние на $\A$ може да се запише като $q_\alpha$ за някоя дума $\alpha$, то можем да запишем горния ред и така:
    \[\delta_\A(q_\alpha,b) = p\ \iff\ \delta_\B(f(q_{\alpha}),b) = f(p).\]
    % Нека положим
    % \begin{align*}
    %   M & \df \alpha^{-1}(L)\\
    %   N & \df (\alpha b)^{-1}(L) = b^{-1}(\alpha^{-1}(L)) = b^{-1}(M).    
    % \end{align*}
    От \Proposition{well-defined-pullback} знаем, че $M = f(q_\alpha)$, защото
    $\L_\A(q_\alpha) = \alpha^{-1}(L)$. 

    За посоката $(\Rightarrow)$, нека $p$ е състояние, за което имаме лявата страна на (\ref{eq:brzozowski-minimal:eq}), т.е.
    \[\delta_\A(q_\alpha,b) = p.\]
    Оттук веднага е ясно, че според нашите означения, $p = q_{\alpha b}$.
    Според дефиницията на $f$ имаме следното:
    \begin{align*}
      & f(q_{\alpha}) = \L_\A(q_\alpha) = \alpha^{-1}(L)\\
      & f(q_{\alpha b}) = \L_\A(q_{\alpha b}) = (\alpha b)^{-1}(L) = b^{-1}(\alpha^{-1}(L)).
    \end{align*}

    % $f(q_{\alpha b}) = \L_\A(q_{\alpha b}) = (\alpha b)^{-1}(L)$



    % разгледаме състоянието $p$, за което $\delta_\A(q_\alpha,b) = p$.
    % Ясно е, че $p = q_{\alpha b}$, защото
    % \[p = \delta_\A(q_\alpha,b) = \delta_\A(\delta^\star_\A(\qstart^\A,\alpha),b) = \delta^\star_\A(\qstart^\A,\alpha b).\]
    % Тогава имаме, че $f(p) = f(q_{\alpha b}) = N$, защото $(\alpha b)^{-1}(L) = \L_\A(q_{\alpha b})$ отново според \Proposition{well-defined-pullback}.
    Остава единствено да отбележим, че от конструкцията на автомата $\B$ знаем, че:
    \[\delta_\B(\underbrace{\alpha^{-1}(L)}_{f(q_{\alpha})},b) = \underbrace{(\alpha b)^{-1}(L)}_{f(p)}.\]
    Така получихме дясната страна на еквивалентността в Свойство~\ref{eq:brzozowski-minimal:eq}.
    
    За посоката $(\Leftarrow)$, нека $p$ е състояние, за което имаме дясната страна на (\ref{eq:brzozowski-minimal:eq}), т.е.
    \[\delta_\B(f(q_{\alpha}),b) = f(p).\]
    Според конструкцията на автомата $\B$,
    \[\delta_\B(\underbrace{\alpha^{-1}(L)}_{f(q_\alpha)},b) = \underbrace{(\alpha b)^{-1}(L)}_{f(p)}.\]
    Сега на помощ ни идва фактът, че $f$ е биекция.
    Понеже $f(p) = (\alpha b)^{-1}(L)$ и $f(q_{\alpha b}) = (\alpha b)^{-1}(L)$ и $f$ е биекция, то
    $p = q_{\alpha b}$ и следователно, $\delta_\A(q_\alpha, b) = p$.
    Така получихме дясната страна на еквивалентността (\ref{eq:brzozowski-minimal:eq}).
  \end{itemize}
\end{proof}


\begin{framed}
  \begin{corollary}[Критерий за минималност]
    \label{cr:regular:brzozowski-minimal:criterion}
    Нека $\A$ е свързан детерминиран краен автомат за езика $L$.
    Тогава $\A$ е минимален автомат за езика $L$ точно тогава, когато е изпълнена импликацията:
    \begin{equation}
      \label{eq:regular:brzozowski-minimal:criterion}
      (\forall p\in Q)(\forall q\in Q)[p \neq q \implies \L_\A(p) \neq \L_\A(q)].
    \end{equation}
  \end{corollary}
\end{framed}
\begin{proof}
  Нека $\B$ е автоматът на Бжозовски за $L$
  и да разгледаме функцията $f : Q^\A \to Q^\B$ дефинирана като $f(q) = \L_\A(q)$.
  
  Първо, нека $\A$ е минимален автомат за $L$. Тогава знаем от \Theorem{brzozowski-minimal:unique}, че
  $f$ е биекция. От инективността на $f$ следва, че имаме импликцията (\ref{eq:regular:brzozowski-minimal:criterion}).
  Второ, нека импликацията (\ref{eq:regular:brzozowski-minimal:criterion}) е изпълнена.
  Това означава, че $f$ е инективна функция, откъдето имаме, че $|Q^\A| \leq |Q^\B|$.
  Понеже $\B$ е минимален автомат, то $\A$ също е минимален автомат.  
\end{proof}



%%% Local Variables:
%%% mode: latex
%%% TeX-master: "../eai"
%%% End:

\subsection*{Примерни задачи}

За да докажем, че един език $L$ не е регулярен можем да приложим \Corollary{brzozowski:finite}
като докажем, че автоматът на Бжозовски за $L$ има безкрайно много състояния.
Обърнете внимание, че не е нужно да намерим всички състояния на автомата $\B$, а само това, че са безкрайно много.

\begin{example}
  \mynote{Сравнете с \Example{regular:brzozowski:an-bn}.}
  За езика $L = \{a^nb^n\mid n \in \Nat\}$ имаме, че за произволни естествени числа $k$ и $m$ е изпълнено следното свойство:
  \[k < m \implies (a^k)^{-1}(L) \neq (a^m)^{-1}(L).\]
  Проверете, че $(a^k)^{-1}(L) = \{a^{n}b^{n+k} \mid n\in\Nat\}$, за всяко $k \in \Nat$.
  Така получаваме, че автоматът на Бжозовски за $L$ ще има безкрайно много състояния.
  Заключаваме, че този език {\bf не} е регулярен.
\end{example}

\begin{example}
  За езика $L = \{a^{n^2} \mid n \in \Nat\}$ имаме, че за произволни естествени числа $k$ и $m$ е изпълнено следното свойство:
  \[k < m \implies (a^{k^2})^{-1}(L) \neq (a^{m^2})^{-1}(L).\]
  За да покажем, че $(a^{k^2})^{-1}(L) \neq (a^{m^2})^{-1}(L)$ е достатъчно да посочим дума $\gamma$, за която $ \gamma \in (a^{k^2})^{-1}(L)$, но $\gamma \not\in (a^{m^2})^{-1}(L)$.
  Да разгледаме думата $\gamma = a^{2k+1}$. Ясно е, че $\gamma \in (a^{k^2})^{-1}(L)$, защото $a^{k^2}\gamma = a^{(k+1)^2} \in L$, но понеже $k < m$, то
  \[m^2 < m^2 + 2k + 1 < m^2 + 2m + 1 = (m+1)^2\]
  и следователно $\gamma \not\in (a^{m^2})(L)$, защото $a^{m^2}\gamma = a^{m^2+2k+1}\not\in L$.
  Заключаваме, че този език {\bf не} е регулярен.
\end{example}

\begin{example}
  За езика $L = \{a^{n!} \mid n \in \Nat\}$ имаме, че за произволни естествени числа $k$ и $m$ е изпълнено следното свойство:
  \[k < m \implies (a^{k!})^{-1}(L) \neq (a^{m!})^{-1}(L).\]
  Да разгледаме $k < m$ и думата $\gamma = a^{(k!)k}$.
  Тогава $\gamma \in (a^{k!})^{-1}(L)$, защото $a^{k!}\gamma = a^{k! + (k!)k} = a^{(k+1)!} \in L$, но 
  \[m! < m! + (k!)k < m! + (m!)m = (m+1)!\]
  и следователно $\gamma \not\in (a^{m!})^{-1}(L)$, защото $a^{m!}\gamma = a^{m!+(k!)k}\not\in L$.
  Заключаваме, че този език {\bf не} е регулярен.
\end{example}

\begin{problem}
  Докажете, че езикът 
  \[L = \{a^{f_n} \mid f_0 = f_1 = 1\ \&\ f_{n+2} = f_{n+1} + f_{n}\}\]
  не е регулярен.
\end{problem}

%%% Local Variables:
%%% mode: latex
%%% TeX-master: "../eai"
%%% End:

% \section{Релация на Майхил-Нероуд}

\begin{itemize}
\item
  \index{Майхил-Нероуд!релация}
  \index{$\approx_L$}
  \marginpar{$\approx_L$ е известна като релация на Майхил-Нероуд}
  Нека $L \subseteq \Sigma^\star$ е език и нека $\alpha,\beta \in \Sigma^\star$.
  Казваме, че $\alpha$ и $\beta$ са {\bf еквивалентни относно} $L$, което записваме 
  като $\alpha \approx_L \beta$, когато:
  \[\alpha \approx_L \beta\ \dff\ \alpha^{-1}(L) = \beta^{-1}(L).\]
  С други думи, 
  \[\alpha \approx_L \beta \iff (\forall \omega \in \Sigma^\star)[\ \alpha\omega \in L \iff \beta\omega \in L\ ].\]
\item
  \marginpar{Трябва ли $\A$ да е тотален?}
  Нека $\A = \FA$ е краен детерминиран автомат.
  \index{$\sim_\A$}
  Казваме, че две думи $\alpha,\beta \in \Sigma^\star$ са {\bf еквивалентни относно $\A$},
  което означаваме с $\alpha \sim_\A \beta$, ако 
  \[\delta^\star(\qstart,\alpha) = \delta^\star(\qstart,\beta).\]
\item
  Проверете, че $\approx_L$ и $\sim_\A$ са {\bf релации на еквивалентност}, т.е.
  те са рефлексивни, транзитивни и симетрични.
\item
  Класът на еквивалентност на думата $\alpha$ относно релацията $\approx_L$ означаваме като
  \[[\alpha]_L \df \{\beta \in \Sigma^\star \mid \alpha \approx_L \beta\}.\]
  Означаваме 
  \[\Sigma^\star/_{\approx_L} \df \{[\alpha]_L \mid \alpha \in \Sigma^\star\}.\]
  Тогава с $\abs{\Sigma^\star/_{\approx_L}}$ ще означаваме броя на класовете на еквивалентност на релацията $\approx_L$.
\item
  Класът на еквивалентност на думата $\alpha$ относно релацията $\sim_\A$ означаваме като
  \[[\alpha]_\A \df \{\beta \in \Sigma^\star \mid \alpha \sim_\A \beta\}.\]
  Означаваме 
  \[\Sigma^\star/_{\sim_\A} \df \{[\alpha]_\A \mid \alpha \in \Sigma^\star\}.\]
  С $\abs{\Sigma^\star/_{\sim_\A}}$ ще означаваме броя на класовете на еквивалентност на релацията $\sim_\A$.
\item
  Съобразете, че всяко състояние на $\A$, което е достижимо от началното състояние, определя клас на еквивалентност относно 
  релацията $\sim_\A$. Това означава, че функцията $g:\Sigma^\star/_\A \to Q$, където
  \[g([\alpha]_\A) \df \delta^\star(\qstart,\alpha)\]
  е инекция. Следователно,
  \[|\Sigma^\star/_{\sim_\A}| \leq |Q|.\]
  Ако в автомата $\A$ няма недостижими от $\qstart$ състояния, то $g$ е биекция и съответно
  \[|\Sigma^\star/_{\sim_\A}| = |Q|.\]
\item
  Релациите $\approx_\L$ и $\sim_\A$ са дясно-инвариантни, т.е. за всеки две думи $\alpha$ и $\beta$
  е изпълнено:
  \marginpar{\writedown Проверете!}
  \begin{align*}
    \alpha \sim_\A \beta  &\implies (\forall \gamma\in\Sigma^\star)[\alpha\gamma \sim_\A \beta\gamma],\\
    \alpha \approx_\L \beta & \implies (\forall \gamma\in\Sigma^\star)[\alpha\gamma \approx_\L \beta\gamma].
  \end{align*}
\end{itemize}

\begin{problem}
  Докажете, че за всяка дума $\alpha \in \Sigma^\star$ е изпълнено, че:
  \[\alpha \in L \iff [\alpha]_L \subseteq L.\]
\end{problem}

\begin{proposition}
  \label{pr:rel-finer}
  \marginpar{С други думи,\\ $[\alpha]_\A \subseteq [\alpha]_L$.}
  Нека $\A = \FA$ е краен детерминиран автомат и $L = \L(\A)$. Тогава е изпълнено, че:
  \[(\forall \alpha,\beta \in \Sigma^\star)[\ \alpha\sim_\A\beta \implies \alpha\approx_{L}\beta\ ].\]
\end{proposition}
\begin{proof}
  Да вземем две произволни думи $\alpha$ и $\beta$, за които $\alpha \sim_\A \beta$, т.е. $\delta^\star(\qstart, \alpha) = \delta^\star(\qstart,\beta)$.
  Ще проверим, че  $\alpha \approx_{L} \beta$, т.е. $\alpha^{-1}(L) = \beta^{-1}(L)$.
  За произволна дума $\gamma$ имаме следното:
  \begin{align*}
    \gamma \in \alpha^{-1}(L) & \iff \alpha\gamma \in L\\
                              % & \iff \alpha\gamma \in \L(\A) & \comment{L = \L(\A)}\\
                              & \iff \delta^\star(\qstart,\alpha\gamma)\in F & \comment{L = \L(\A)}\\
                              & \iff \delta^\star(\delta^\star(\qstart,\alpha),\gamma) \in F & \comment{\text{деф. на }\delta^\star}\\
                              & \iff \delta^\star(\delta^\star(\qstart,\beta),\gamma) \in F & \comment{\text{защото }\alpha \sim_\A \beta}\\
                              & \iff \delta^\star(\qstart,\beta\gamma) \in F & \comment{\text{свойство на }\delta^\star}\\
                              & \iff \beta\gamma \in \L(\A) & \comment\text{деф. на }\L(\A)\\
                              % & \iff \beta\gamma \in L & \comment{L = \L(\A)} \\
                              & \iff \gamma \in \beta^{-1}(L).
  \end{align*}
  Заключаваме, че 
  \[(\forall \alpha,\beta \in \Sigma^\star)[\ \alpha\sim_\A\beta \implies \alpha\approx_{L}\beta\ ].\]
\end{proof}

\begin{problem}
  Докажете, че за ДКА $\A$ и $L = \L(\A)$,
  за всяка дума $\alpha$,
  \[[\alpha]_{L} = \bigcup_{\beta \in [\alpha]_{L}}[\beta]_\A.\]
\end{problem}
% \begin{hint}
%   Първо, нека $\gamma \in \bigcup_{\beta \in [\alpha]_{L}}[\beta]_\A$, т.е. за някое $\beta \in [\alpha]_L$ имаме, че $\gamma \in [\beta]_\A$.
%   Но щом $\beta \in [\alpha]_L$, то тогава $[\beta]_L = [\alpha]_L$, а от $\gamma \in [\beta]_\A$, по \Prop{rel-finer} следва, че $\gamma \in [\beta]_L = [\alpha]_L$.
%   Заключаваме, че $\bigcup_{\beta \in [\alpha]_{L}}[\beta]_\A \subseteq [\alpha]_L$.

%   Нека сега $\gamma \in [\alpha]_L$. Но тогава е ясно, че $\gamma \in [\gamma]_\A$ и следователно директно получаваме, че
%   $[\alpha]_L \subseteq \bigcup_{\beta \in [\alpha]_{L}}[\beta]_\A$.
% \end{hint}

\begin{problem}
  \marginpar{Тук говорим за тотални функции.}
  Нека $A$ и $B$ са крайни множества, за които съществува сюрективна функция $f: A \to B$.
  Тогава $|B| \leq |A|$.
\end{problem}

\begin{problem}
  Нека $A$ и $B$ са крайни множества, за които $|A| = |B|$.
  Ако $g:A \to B$ е сюрекция, то $g$ е биекция.
\end{problem}


\begin{proposition}
  \label{pr:approx-less-sim}
  Нека $\A$ е детерминиран краен автомат и $L = \L(\A)$.
  Тогава 
  \[\abs{\Sigma^\star/_{\approx_{L}}} \leq \abs{\Sigma^\star/_{\sim_\A}}.\]
\end{proposition}
\begin{hint}
  Да разгледаме функцията $f: \Sigma^\star/_{\sim_\A} \to \Sigma^\star/_{\approx_L}$, където
  \[f([\alpha]_\A) \df [\alpha]_L.\]
  Директно от \Proposition{rel-finer} се съобразява, че
  \[(\forall \alpha,\beta \in \Sigma^\star)[\ [\alpha]_\A = [\beta]_\A\ \implies f([\alpha]_\A) = f([\beta]_\A)\ ],\]
  откъдето следва, че $f$ е добре дефинирана.
  Ясно е, че $f$ е сюрективна функция.
  Оттук следва, че
  \[\abs{\Sigma^\star/_{\approx_{L}}} \leq \abs{\Sigma^\star/_{\sim_\A}}.\]
\end{hint}

\begin{framed}
  \begin{proposition}
    \label{pr:upper-bound}
    Нека $L$ е произволен регулярен език.
    Всеки краен детерминиран автомат $\A$, за който $L = \L(\A)$ има свойството
    \[\abs{\Sigma^\star/_{\approx_L}} \leq \abs{Q},\]
    т.е. броят на класовете на еквивалентност на релацията $\approx_L$
    не надвишава броя на състоянията на автомата $\A$.
  \end{proposition}  
\end{framed}
\begin{proof}
  Да изберем $\A$, който разпознава $L$. 
  Тъй като всяко достижимо състояние на $\A$ определя клас на еквивалентност относно $\sim_\A$,
  то получаваме, че $\abs{\Sigma^\star/_{\sim_\A}} \leq |Q|$.
  Комбинирайки със \Proposition{approx-less-sim},
  \[\abs{\Sigma^\star/_{\approx_L}} \leq \abs{\Sigma^\star/_{\sim_\A}} \leq \abs{Q}.\]
\end{proof}

Така получаваме {\em долна граница} за броя на състоянията в краен детерминистичен автомат разпознаващ езика $L$.
Този брой е не по-малък от броя на класовете на еквивалентност на $\approx_L$.
В следващия раздел ще видим, че тази долна граница може да бъде достигната.

\begin{framed}
  \begin{proposition}
    Езикът $L$ е регулярен точно тогава, когато релацията $\approx_L$ има {\em крайно много} класове на еквивалентност.
  \end{proposition}  
\end{framed}
\begin{proof}
  Ако $L$ е регулярен, то той се разпознава от някой ДКА $\A$, който има крайно много състояния 
  и следователно крайно много класове на еквивалентност относно $\sim_\A$.
  Релацията $\approx_L$ е по-груба от $\sim_\A$ и има по-малко класове на еквивалентност.
  Следователно, $\approx_L$ има крайно много класове на еквивалентност.
  
  За другата посока, ако $\approx_L$ има крайно много класове на еквивалентност, то можем да 
  построим ДКА $\A$ както в доказателството на \hyperref[th:myhill-nerode]{Теоремата на Майхил-Нероуд}, който разпознава $L$.
\end{proof}

Това следствие ни дава още един начин за проверка дали даден език е регулярен.
За разлика от \Lemma{pumping-reg}, сега имаме {\bf необходимо и достатъчно условие}.
При даден език $L$, ние разглеждаме неговата релация $\approx_L$.
Ако тя има крайно много класове, то езикът $L$ е регулярен.
В противен случай, езикът $L$ не е регулярен.

Ще завършим с едно твърдение, което ще ни бъде полезно по-нататък, когато искаме да докажем, че за всеки регулярен език съществува
единствен минимален краен детерминистичен автомат.
\begin{proposition}
  \label{pr:bijection-classes}
  Нека $L$ е произволен регулярен език и $\A$ е краен детерминиран автомат, за който $L = \L(\A)$.
  Ако $|Q| = |\Sigma^\star/_{\approx_L}|$, то функцията $h:\Sigma^\star/_{\approx_L} \to Q$, където
  \[h([\alpha]_L) \df \delta^\star(\qstart,\alpha),\]
  е биекция.
\end{proposition}
\begin{proof}
  Имаме, че:
  \[|Q| = |\Sigma^\star/_{\approx_L}| \leq |\Sigma^\star/_{\sim_\A}| \leq |Q|,\]
  откъдето следва, че
  \[|\Sigma^\star/_{\approx_L}| = |\Sigma^\star/_{\sim_\A}| = |Q|.\]
  Това означава, че функцията $g:\Sigma^\star/_{\sim_\A} \to Q$, където
  \[g([\alpha]_\A) \df \delta^\star(\qstart,\alpha)\] е биекция,
  защото $|\Sigma^\star/_{\sim_\A}| = |Q|$.
  От \Proposition{approx-less-sim} имаме, че функцията $f:\Sigma^\star/_{\sim_\A} \to \Sigma^\star/_{\approx_L}$, където
  \[f([\alpha]_\A) \df [\alpha]_L\] е биекция,
  защото знаем, че $f$ е сюрекция и $|\Sigma^\star/_{\approx_L}| = |\Sigma^\star/_{\sim_\A}|$.
  Оттук заключаваме, че $h = g \circ f^{-1}$ е биекция, защото е композиция на две биекции.
\end{proof}

%%% Local Variables:
%%% mode: latex
%%% TeX-master: "../eai"
%%% End:

\section{Теорема за съществуване на минимален автомат}
\begin{framed}
  \begin{thm}[Майхил-Нероуд]
    \label{th:myhill-nerode}
    \index{Майхил-Нероуд!теорема}
    Нека $L\subseteq \Sigma^\star$ е регулярен език.
    Тогава съществува ДКА $\M = \FA$, който разпознава $L$,
    с точно толкова състояния, колкото са класовете на еквивалентност на релацията $\approx_L$,
    т.е. $\abs{Q} = \abs{\Sigma^\star/_{\approx_L}}$.
  \end{thm}  
\end{framed}
\begin{proof}
  % \marginpar{стр. 96 от \cite{papadimitriou}}
  \marginpar{на англ. Myhill-Nerode}
  Да фиксираме регулярния език $L$.
  Ще дефинираме детерминистичен краен автомат $\M = \FA$, разпознаващ $L$, като:
  \begin{itemize}
  \item
    $Q \df \{\ [\alpha]_L\mid \alpha\in \Sigma^\star\ \}$;
  \item
    $\qstart \df [\varepsilon]_L$;
  \item
    $F \df \{\ [\alpha]_L\mid \alpha\in L\ \} = \{\ [\alpha]_L \mid [\alpha]_L \subseteq L\ \}$;
  \item
    Определяме изображението $\delta$ като 
    за всяка буква $x \in \Sigma$ и всяко състояние $[\alpha]_L\in Q$, 
    \[\delta([\alpha]_L,x) \df [\alpha x]_L.\]
  \end{itemize}
  
  Първо, трябва да се уверим, че множеството от състояния $Q$ е крайно, т.е.
  релацията $\approx_L$ има крайно много класове на еквивалентност.
  И така, тъй като $L$ е регулярен език, то той се разпознава от някой детерминистичен краен автомат $\A$.
  От \Prop{upper-bound} имаме, че $\abs{Q^{\A}} \geq \abs{\Sigma^\star/_{\approx_L}}$.
  Понеже $Q^{\A}$ е крайно множество, то $\approx_L$ има крайно много класове и 
  следователно $Q$ също е крайно множество.

  Второ, трябва да се уверим, че изображението $\delta$ задава функция, т.е. 
  да проверим, че за всеки две думи $\alpha$, $\beta$ и всяка буква $x$,
  \[[\alpha]_L = [\beta]_L \implies \delta([\alpha]_L,x) = \delta([\beta]_L,x).\]
  Но това се вижда веднага, защото от определението на релацията $\approx_L$ следва, че
  ако $\alpha \approx_L \beta$, то за всяка буква $x$, $\alpha x \approx_L \beta x$,
  т.е. $[\alpha x]_L = [\beta x]_L$ и 
  \marginpar{Функцията на преходите $\delta$ е дефинирана чрез произволен представител на всеки клас на еквивалентност относно $\approx_L$. Трябва да съобразим, че няма значение кой представител сме избрали.}
  \begin{align*}
    [\alpha]_L = [\beta]_L & \implies [\alpha x]_L = [\beta x]_L & \comment{\text{свойство на }\approx_L}\\
                           & \implies \delta([\alpha]_L,x) \df [\alpha x]_L = [\beta x]_L \df \delta([\beta]_L,x).
  \end{align*}
  
  Така вече сме показали, че $\M$ е коректно зададен детерминистичен краен автомат.
  Остава да покажем, че $\M$ разпознава езика $L$, т.е. $\L(\M) = L$.
  За целта, първо ще докажем едно помощно твърдение.
  \begin{prop}
    За всеки две думи $\alpha$ и $\beta$ е изпълнено, че:
    \begin{equation}
      \label{eq:4}
      \delta^\star([\alpha]_L,\beta) = [\alpha\beta]_L.
    \end{equation}
  \end{prop}
  \begin{proof}
    Ще докажем това свойство с индукция по дължината на думата $\beta$.
    \begin{itemize}
    \item
      Нека $|\beta| = 0$, т.е. $\beta = \varepsilon$. В този случай, свойството следва директно от дефиницията на $\delta^\star$, т.е.
      \[\delta^\star([\alpha]_L,\varepsilon) = [\alpha]_L.\]
    \item
      Да приемем, че свойството (\ref{eq:4}) е изпълнено за всички думи $\beta$ с дължина $n$.
    \item
      Нека $\abs{\beta} = n+1$. Тогава $\beta = \gamma a$, където $\abs{\gamma} = n$.
      В този случай, свойството (\ref{eq:4}) следва от следните равенства:
      \begin{align*}
        \delta^\star([\alpha]_L, \gamma a) & = \delta(\delta^\star([\alpha]_L,\gamma),a) & \comment{\text{деф. на }\delta^\star}\\
                                          & = \delta([\alpha\gamma]_L,a) & \comment{\text{от {\bf И.П.} за }\gamma}\\
                                          & = [\alpha\gamma a]_L & \comment{\text{от деф. на }\delta}\\
                                          & = [\alpha\beta]_L & \comment{\beta = \gamma a}.
      \end{align*}
    \end{itemize}
  \end{proof}
  \noindent За да се убедим, че $L = \L(\M)$ е достатъчно да проследим еквивалентностите:
  \begin{align*}
    \alpha\in \L(\M) & \iff \delta^\star(\qstart,\alpha) \in F & \comment{\text{деф. на }\L(\M)}\\
                     & \iff \delta^\star([\varepsilon]_L,\alpha) \in F & \comment{\qstart \df [\varepsilon]_L}\\
                     & \iff \delta^\star([\varepsilon]_L,\alpha) = [\varepsilon\alpha]_L\in F & \comment{\text{от (\ref{eq:4})}}\\
                     & \iff [\alpha]_L \in F \\
                     & \iff \alpha\in L & \comment{\text{деф. на }F}.
  \end{align*}
\end{proof}


%%% Local Variables:
%%% mode: latex
%%% TeX-master: "../eai"
%%% End:

% \subsection{Експоненциален алгоритъм за минимизация}



\begin{itemize}
\item
  Нека имаме ДКА $\A = \FA$. Да приемем, че $\A$ не е минимален автомат за езика $\L(\A)$.
  Това означава, че съществуват различни състояния $q_\alpha$ и $q_\beta$, за които $\L_\A(q_\alpha) = \L_\A(q_\beta)$.
  С други думи,
  \begin{equation}
    \label{eq:brzozowski-minimisation:equiv-states}
    (\forall \omega)[\delta^\star(q_\alpha,\omega) \in F \iff \delta^\star(q_\beta,\omega) \in F].
  \end{equation}
\item
  Нека $\N_1 = (\Sigma, Q, F, \Delta_1, \{\qstart\})$, където
  $\Delta_1(q,x) \df \{p \in Q\mid \delta(p,x) = q\}$
  и следователно
  \[\Delta^\star_1(U,\alpha) \df \bigcup\{p \in Q\mid \delta^\star(p,\alpha) \in U\}.\]
  \begin{equation}
    \label{eq:equiv-states:rev1}
    (\forall \omega)[\Delta_1^\star(F,\omega) \ni q_\alpha \iff \Delta_1^\star(F,\omega) \ni q_\beta].
  \end{equation}
\item
  Нека $\D_2 = (\Sigma, Q_2, \code{F}, \delta_2, F_2)$, където
  \begin{itemize}
  \item
    $Q_2 = \{\code{U} \mid (\exists \omega)[\Delta^\star_1(F,\omega) = U]\}$;
  \item
    $F_2 = \{\code{U} \in Q_2 \mid \qstart \in U\}$;
  \item
    $\delta_2(\code{U},x) = \code{\Delta^\star_1(U,x)}$.
  \end{itemize}
  Получаваме, че за едно състояние $\code{U}$,
  \begin{align*}
    q_\alpha \in U & \iff \delta^\star_2(\code{U},\alpha) \in F_2\\
                   & \iff \delta^\star_2(\code{U},\alpha) \ni \qstart.
  \end{align*}
\item
  Нека $\N_3 = (\Sigma, Q_2, F_2, \Delta_3, \{\code{F}\})$, където
  \[\Delta_3(\code{U},x) = \{\code{V} \mid \delta_2(\code{V},x) = \code{U}\}).\]
  Сега имаме, че
  \begin{align*}
    \Delta^\star_3(F_2,\alpha) & = \{\code{U} \mid \delta^\star_2(\code{U},\alpha) = \code{K}\ \&\ \code{K} \in F_2\}\\
                               & = \{\code{U} \mid \delta^\star_2(\code{U},\alpha) = \code{K}\ \&\ \qstart \in K\}\\
                               & = \{\code{U} \mid q_\alpha \in U\}\\
                               & = \{\code{U} \mid q_\beta \in U\}\\
                               & = \Delta^\star_3(F_2,\beta).
  \end{align*}
\end{itemize}

Дефинираме $\B = \rev(\A)$ да бъде ДКА по следния начин:
\begin{itemize}
\item
  Състоянията на $\B$ ще бъдат подмножества на състоянията на $\A$.

\item
  Да положим $Q_\alpha \df \{q \in Q^\A \mid \delta^\star_\A(q,\alpha) \in F\}$.
  Тогава
  $Q^\B \df \{Q_\alpha \mid \alpha \in \Sigma^{\star}\}$.
\item
  $\qstart^\B \df F^\A = Q_\varepsilon$.
\item
  За произволно $R \in Q^\B$,
  $\delta_\B(R,a) \df \{q \in Q^\A \mid \delta_\A(q,a) \in R\}$.
  С други думи,
  $\delta_\B(Q_\beta,a) = Q_{a\beta}$.
\item
  $F^\B = \{Q_\alpha \mid \qstart^\A \in Q_\alpha\}$.
\item
  Съобразете, че $\delta^\star_\B(Q_\alpha,\gamma) = Q_{\gamma^\rev\alpha}$.
  Тогава
  \begin{align*}
    \L(\B) & = \{\alpha \mid \delta^\star_\B(Q_\varepsilon,\alpha) = Q_{\alpha^\rev} \in F^\B\}\\
           & = \{\alpha \mid  \qstart^\A \in Q_{\alpha^\rev}\}\\
           & = \{\alpha \mid  \delta^\star(\qstart^\A,\alpha^\rev) \in F^\A\}\\
           & = \L(\A)^\rev.
  \end{align*}
\end{itemize}

\begin{problem}
  Докажете, че $\rev(\A)$ е минимален автомат за $\L(\A)^\rev$.
\end{problem}
\begin{hint}
  Достатъчно е да се докаже, че
  \[Q_\alpha = Q_\beta \iff \L_\B(Q_\alpha) = \L_\B(Q_\beta).\]
  Нека $\L_\B(Q_\alpha) = \L_\B(Q_\beta)$. Ще докажем, че $Q_\alpha \subseteq Q_\beta$.
  Нека $q \in Q_\alpha$. Това означава, че $\delta^\star_\A(q,\alpha) \in F^\A$.
  Следователно, съществува $\gamma$, за която $\delta^\star_\A(\qstart^\A,\gamma\alpha) \in F^\A$.
  От друга страна,
  \begin{align*}
    \gamma^\rev \in \L_\B(Q_\alpha) & \iff \delta^\star_\B(Q_\alpha,\gamma^\rev) \in F^\B\\
                                    & \iff Q_{\gamma\alpha} \in F^\B\\
                                    & \iff \delta^\star_\A(\qstart^\A,\gamma\alpha) \in F^\A.
  \end{align*}
\end{hint}


\begin{framed}
  \begin{theorem}[Бжозовски]
    Нека $\A$ е ДКА. Тогава $\rev(\rev(\A))$ е минимален автомат за $\L(\A)$.
  \end{theorem}
\end{framed}


%%% Local Variables:
%%% mode: latex
%%% TeX-master: "../eai"
%%% End:

\newpage
\section{Алгоритъм за строене на минимален автомат}
\begin{itemize}
\item
  Да фиксираме произволен детерминистичен краен автомат
  \[\A = \FA.\]
\item
  За състояние $p$ в автомата $\A$, да означим с $\L_\A(p)$ езикът, който се разпознава от автомата $\A$,
  ако приемем, че $p$ е началното състояние на автомата, т.е.
  \[\L_\A(p) \df \{\omega \in \Sigma^\star \mid \delta^\star(p,\omega) \in F\}.\]
  В частност, $\L(\A) = \L_\A(\qstart)$.
\item
  \marginpar{Може да е малко объркващо, но релацията $\equiv_\A$ няма нищо общо с релацията $\sim_\A$.
  Релацията $\sim_\A$ е върху думи, докато $\equiv_\A$ е върху състояния.}
  Сега дефинираме следната релация между състояния на автомата $\A$:
  \[p \equiv_\A q\ \dff\ \L_\A(p) = \L_\A(q).\]
  Това означава, че $p \equiv_\A q$ точно тогава, когато
  \begin{equation}
    \label{eq:1}
    (\forall \omega\in \Sigma^\star)[\ \delta^\star(p,\omega) \in F\ \iff\ \delta^\star(q,\omega) \in F\ ].
  \end{equation}
\item
  Релацията $\equiv_\A$ между състояния на автомата $\A$ е релация на еквивалентност. 
\item
  Да означим
  \[Q/_{\equiv_\A} \df \{[q]_{\equiv_\A} \mid q \in Q\}.\]
\end{itemize}

\begin{problem}
  Докажете, че
  \[[q]_{\equiv_\A} \cap F \neq \emptyset \iff [q]_{\equiv_\A} \subseteq F.\]
\end{problem}

\begin{framed}
  \begin{prop}
    \label{pr:equal-number}
    Нека $\A = \FA$ е детерминистичен краен автомат {\em без недостижими състояния от} $\qstart$.
    Тогава
    \[\abs{Q/_{\equiv_\A}} = \abs{\Sigma^\star/_{\approx_{\L(\A)}}}.\]  
  \end{prop}  
\end{framed}
\begin{hint}
  Понеже няма недостижими състояния от $\qstart$, нека 
  $q_\alpha$ е състоянието, което съответства на думата $\alpha$ в $\A$, т.е.
  $\delta^\star_\A(\qstart,\alpha) = q_\alpha$. Тогава 
  \[\L_\A(q_\alpha) = \alpha^{-1}(\L(\A)),\]
  защото за произволна дума $\beta$,
  \begin{align*}
    \beta \in \L_\A(q_\alpha) & \iff \delta^\star(q_\alpha,\beta) \in F\\
                              & \iff \delta^\star(\delta^\star(\qstart, \alpha),\beta) \in F\\
                              & \iff \alpha \cdot \beta \in \L(\A)\\
                              & \iff \beta \in \alpha^{-1}(\L(\A)).
  \end{align*}
  Оттук получаваме, че 
  \begin{align*}
    q_\alpha \equiv_\A q_\beta & \dff \L_\A(q_\alpha) = \L_\A(q_\beta)\\
                               & \iff \alpha^{-1}(\L(\A)) = \beta^{-1}(\L(\A))\\
                               & \dff \alpha \approx_{\L(\A)} \beta.
  \end{align*}
  Заключаваме, че $f:\Sigma^\star/_{\approx_L} \to Q/_{\equiv_\A}$, където
  \[f([\alpha]_L) \df [\underbrace{\delta^\star(\qstart,\alpha)}_{q_\alpha}]_{\equiv_\A}\]
  е биекция.  
\end{hint}

При даден език $L$ и тотален ДКА $\A = \FA$, който го разпознава, целта ни е да построим нов ДКА $\M$,
който има толкова състояния колкото са класовете на еквивалентност на релацията $\approx_L$.
Това ще направим като ,,слеем'' състоянията на $\A$, които са еквивалентни относно релацията $\equiv_\A$.
Според \Prop{equal-number}, това означава, че всяко състояние на $\M$ ще отговаря на един клас на еквивалентност на релацията $\equiv_\A$.
Проблемът с намирането на класовете на еквивалентност на релацията $\equiv_\A$ е кванторът $\forall \omega \in \Sigma^\star$
във нейната дефиниция (чрез Формула \ref{eq:1}), защото $\Sigma^\star$ е безкрайно множество от думи.

Да означим 
\[\L^n_\A(p) \df \{\omega \in \Sigma^\star \mid \abs{\omega} \leq n\ \&\ \delta^\star(p,\omega) \in F\}.\]
Лесно се съобразява, че
\marginpar{Можем ли да дадем горна граница на $n$?}
\[L(\A) = \bigcup_{n\geq 0} \L^n_\A(\qstart).\]

За всяко естествено число $n$, дефинираме бинарните релации $\equiv_n$ върху $Q$ по следния начин:
\[p \equiv_n q \dff \L^n_\A(p) = \L^n_\A(q).\]

Релациите $\equiv_n$ представляват апроксимации на релацията $\equiv_\A$.
Обърнете внимание, че за всяко $n$, $\equiv_n$ е {\em по-груба} релация от $\equiv_{n+1}$, 
която на свой ред е по-груба от $\equiv_\A$.
Алгоритъмът строи $\equiv_n$ докато не срещнем $n$, за което
\[\equiv_n\ =\ \equiv_{n+1}.\]
Тъй като броят на класовете на еквивалентност на $\equiv_\A$ е краен (той е $\leq \abs{Q}$), то 
със сигурност ще намерим такова $n$, за което $\equiv_n\ =\ \equiv_{n+1}$.
Тогава заключаваме, че за това $n$ имаме, че
\[\equiv_\A\ =\ \equiv_n.\]

Понеже единствената дума с дължина $0$ e $\varepsilon$ и по определение $\delta^\star(p,\varepsilon) = p$, 
лесно се съобразява, че $\equiv_0$ има два класа на еквивалентност.
Единият е $F$, а другият е $Q\setminus F$.

Вече имаме базовия случай за $n=0$.
Да видим сега как можем да намерим $\equiv_{n+1}$ при положение, че вече сме намерили $\equiv_n$.
\begin{framed}
  \begin{prop}
    \label{pr:one-letter-test}
    За всеки две състояния $p,q \in Q$, и всяко естествено число $n$, $p \equiv_{n+1} q$ точно тогава, когато
    \begin{enumerate}[a)]
    \item
      $p \equiv_{n} q$ и
    \item
      $(\forall a \in \Sigma)[\delta(q,a) \equiv_{n} \delta(p,a)]$.
    \end{enumerate}
  \end{prop}  
\end{framed}
\begin{hint}
  \marginpar{\cite[стр. 99]{papadimitriou}}
  \begin{align*}
    p \equiv_{n+1} q & \iff \L^{n+1}_\A(p) = \L^{n+1}_\A(q)\\
                     & \iff \L^n_\A(p) = \L^n_\A(q)\ \&\ (\forall a \in \Sigma)[\L^n_\A(\delta(p,a)) = \L^n_\A(\delta(q,a))]\\
                     & \iff p \equiv_n q\ \&\ (\forall a \in \Sigma)[\delta(p,a) \equiv_n \delta(q,a)].
  \end{align*}
\end{hint}

\begin{problem}
  Докажете, че ако $\equiv_n\ =\ \equiv_{n+1}$, то
  \[(\forall m \geq n)[\ \equiv_n\ =\ \equiv_m\ ].\]
\end{problem}

\begin{problem}
  Докажете, че ако $\equiv_n\ =\ \equiv_{n+1}$, то
  \[\equiv_n\ =\ \equiv_\A.\]
\end{problem}

\begin{problem}
  Докажете, че ако $n \geq |Q|$, то
  \[\equiv_n\ =\ \equiv_\A.\]
\end{problem}

\begin{problem}
  Докажете, че за произволни $p,q \in Q$ и произволно $a\in\Sigma$,
  \[p \equiv_\A q \implies \delta(p,a) \equiv_\A \delta(q,a).\]
\end{problem}

Нека е даден автомата $A = \FA$.
След като сме намерили релацията $\equiv_\A$ за $\A$, 
строим автомата $\A' = (Q',\Sigma,\qstart',\delta',F')$, където:
\begin{itemize}
\item
  $Q' \df \{[q]_{\equiv_\A} \mid q\in Q\}$;
\item
  $\qstart' \df [\qstart]_{\equiv_\A}$;
\item
  $\delta'([q]_{\equiv_\A}, a) \df [\delta(q,a)]_{\equiv_\A}$;
\item
  $F' \df \{[q]_{\equiv_\A}\mid [q]_{\equiv_\A} \subseteq F\}$;
\end{itemize}

\begin{framed}
  \begin{thm}
    $\A'$ е минимален автомат разпознаващ езика $\L(\A)$.
  \end{thm}
\end{framed}
\begin{hint}
  Лесно се съобразява, че $\L(\A) = \L(\A')$.
  От \Prop{equal-number} знаем, че $|Q'| = |\Sigma^\star/_{\approx_{\L(\A)}}|$.
  Тогава от \Prop{upper-bound} следва, че $\A'$ е минимален автомат.
\end{hint}

\begin{example}
  Да разгледаме следния краен детерминиран автомат $\A$.
  \marginpar{$\L(\A) = \{\alpha \in \{a,b\}^\star \mid \abs{\alpha} \geq 2\}$.}
  \begin{framed}
  \begin{figure}[H]
      \begin{subfigure}[b]{0.45\textwidth}
        \begin{tikzpicture}[->,>=stealth,thick,node distance=55pt]
          \tikzstyle{every state}=[circle,minimum size=20pt,auto]
          
          \node[initial above, state]   (0) {$0$};
          \node[state]            (1) [above right of=0]{$1$};
          \node[state]            (2) [below right of=0]{$2$};
          \node[state,accepting]  (3) [right of=1]{$3$};
          \node[state,accepting]  (4) [right of=2]{$4$};
          \node[state,accepting]  (5) [below right of=3]{$5$};
          
          
          \path 
          (0) edge  node [above] {$a$}   (1)
          (0) edge  node [below] {$b$}   (2)
          (1) edge node [above] {$a$}    (3)
          (1) edge [bend left=15] node [below] {$b$}    (4)
          (2) edge [bend left=15] node [left] {$b$}    (3)
          (2) edge node [below] {$a$}   (4)
          (4) edge  node [below] {$a,b$} (5)
          (3) edge  node [left] {$a,b$}  (5)
          (5) edge [loop above]   node [above] {$a,b$}  (5);
        \end{tikzpicture}
        \caption{Ще построим минимален автомат разпознаващ $\L(\A)$}
      \end{subfigure}
      \quad
      \quad
      \begin{subfigure}[b]{0.4\textwidth}
        \begin{tikzpicture}[->,>=stealth,thick,node distance=45pt]
        \tikzstyle{every state}=[circle,minimum size=20pt,auto,scale=.9]
        
        \node[initial above, state]   (0) {$B_0$};
        \node[state]            (1) [right of=0]{$B_1$};
        \node[state,accepting]  (2) [right of=1]{$B_2$};
        
        \path 
        (0) edge [bend left=15] node [above] {$a,b$}   (1)
        (1) edge [bend left=15] node [above] {$a,b$}   (2)
        (2) edge [loop above] node [above] {$a,b$}   (2);
      \end{tikzpicture}
      \caption{Получаваме минималния автомат $\M$, $\L(\M) = \L(\A)$}
      \label{sub:min1}
    \end{subfigure}
  \end{figure}
\end{framed}


Ще приложим алгоритъма за минимизация за да получим минималния автомат за езика $L$.
За всяко $n = 0,1,2,\dots$, ще намерим класовете на еквивалентност на $\equiv_n$,
докато не намерим $n$, за което $\equiv_n\ =\ \equiv_{n+1}$.

\begin{itemize}
\item 
  Класовете на еквивалентност на $\equiv_0$ са два.
  Те са $A_0 = Q\setminus F = \{0,1,2\}$ и $A_1 = F = \{3,4,5\}$.
\item
  Сега да видим дали можем да разбием някои от класовете на еквивалентност на $\equiv_0$.
  
  \begin{tabular}{|c|c|c|c|c|c|c|}
    \hline
    $Q$ & $0$ & $1$ & $2$ & $3^\star$ & $4^\star$ & $5^\star$ \\
    \hline
    \hline
    $\equiv_0$ & $A_0$ & $A_0$ & $A_0$ & $A_1$ & $A_1$ & $A_1$\\
    \hline
    $a$ & $A_0$& $A_1$ & $A_1$ & $A_1$ & $A_1$ & $A_1$\\
    \hline
    $b$ & $A_0$& $A_1$ & $A_1$ & $A_1$ & $A_1$ & $A_1$\\
    \hline
  \end{tabular}

  Виждаме, че $0 \not\equiv_1 1$ и $1 \equiv_1 2$.
  Класовете на еквивалентност на $\equiv_1$ са 
  $B_0 = \{0\}$, $B_1 = \{1,2\}$, $B_2 = \{3,4,5\}$.
\item
  Сега да видим дали можем да разбием някои от класовете на еквивалентност на $\equiv_1$.
  
  \begin{tabular}{|c|c|c|c|c|c|c|}
    \hline
    $Q$ & $0$ & $1$ & $2$ & $3^\star$ & $4^\star$ & $5^\star$ \\
    \hline
    \hline
    $\equiv_1$ & $B_0$ & $B_1$ & $B_1$ & $B_2$ & $B_2$ & $B_2$\\
    \hline
    $a$ & $B_1$ & $B_2$ & $B_2$ & $B_2$ & $B_2$ & $B_2$\\
    \hline
    $b$ & $B_1$ & $B_2$ & $B_2$ & $B_2$ & $B_2$ & $B_2$\\
    \hline
  \end{tabular}

  Виждаме, че $\equiv_1\ =\ \equiv_2$.
  \marginpar{Получаваме, че $\equiv_\A\ =\ \equiv_1$}
  Следователно, минималният автомат има три състояния.
  Той е изобразен на Фигура \ref{sub:min1}.  
  Минималният автомат може да се представи и таблично:
  
  \begin{tabular}{|c|c|c|c|c|c|c|}
    % \hline
    % $Q$ & $0$ & $1$ & $2$ & $3^\star$ & $4^\star$ & $5^\star$ \\
    % \hline
    \hline
    $\delta$ & $B_0$ & $B_1$ & $B_2$ \\
    \hline
    $a$ & $B_1$ & $B_2$ & $B_2$ \\
    \hline
    $b$ & $B_1$ & $B_2$ & $B_2$ \\
    \hline
  \end{tabular}
\end{itemize}
\end{example}

\begin{example}
  Да разгледаме следния краен детерминиран автомат $\A$.
  \begin{framed}
  \begin{figure}[H]
    % \begin{center}
    \begin{subfigure}[b]{0.4\textwidth}
      \begin{tikzpicture}[->,>=stealth,thick,node distance=55pt]
        \tikzstyle{every state}=[circle,minimum size=20pt,auto]
        
        \node[initial above, state]   (0) {$0$};
        \node[state,accepting]        (1) [above right of=0]{$1$};
        \node[state,accepting]        (2) [below right of=0]{$2$};
        \node[state]                  (3) [right of=1]{$3$};
        \node[state]                  (4) [right of=2]{$4$};
        \node[state,accepting]        (5) [below right of=3]{$5$};
        
        \path 
        (0) edge node [below] {$a$}   (1)
            edge node [below] {$b$}   (2)
        (1) edge node [above] {$a$}    (3)
            edge [bend left=15] node [below] {$b$}    (4)
        (2) edge [bend left=15] node [left] {$b$}    (3)
            edge node [below] {$a$}   (4)
        (4) edge node [below] {$a,b$} (5)
        (3) edge node [left] {$a,b$}  (5)
        (5) edge [loop above]   node [above] {$a,b$}  (5);
      \end{tikzpicture}
      \caption{Ще построим минимален автомат разпознаващ $\L(\A)$}
    \end{subfigure}
    \qquad
    \qquad
    \begin{subfigure}[b]{0.4\textwidth}
      \begin{tikzpicture}[->,>=stealth,thick,node distance=45pt]
        \tikzstyle{every state}=[circle,minimum size=20pt,auto,scale=.9]
        
        \node[initial above, state]   (0) {$C_0$};
        \node[state,accepting]  (1) [right of=0]{$C_1$};
        \node[state]            (2) [right of=1]{$C_2$};
        \node[state,accepting]  (3) [right of=2]{$C_3$};
                
        \path 
        (0) edge [bend left=15] node [above] {$a,b$}   (1)
        (1) edge [bend left=15] node [above] {$a,b$}   (2)
        (2) edge [bend left=15] node [above] {$a,b$}   (3)
        (3) edge [loop above]   node [above] {$a,b$}   (3);
      \end{tikzpicture}
      \caption{Получаваме минималния автомат $\M$, $\L(\M) = \L(\A)$}
      \label{sub:min2}
    \end{subfigure}
  \end{figure}
  \end{framed}

  \marginpar{Съобразете, че $\L(\A) = \{a,b\} \cup \{\alpha \in \{a,b\}^\star \mid \abs{\alpha} \geq 3\}$.}
  
  Отново следваме същата процедура за минимизация.
  Ще намерим класовете на еквивалентност на $\equiv_n$,
  докато не намерим $n$, за което $\equiv_n\ =\ \equiv_{n+1}$.
  \begin{itemize}
  \item
    Класовете на екиваленост на $\equiv_0$ са 
    $A_0 = Q\setminus F = \{0,3,4\}$ и $A_1 = F = \{1,2,5\}$.
  \item
    Разбиваме класовете на еквивалентност на $\equiv_0$ като използваме \Prop{one-letter-test}.
    
    \begin{tabular}{|c|c|c|c|c|c|c|}
      \hline
      $Q$ & 0 & $1^\star$ & $2^\star$ & 3 & 4 & $5^\star$ \\
      \hline
      \hline
      $\equiv_0$ & $A_0$ & $A_1$ & $A_1$ & $A_0$ & $A_0$ & $A_1$\\
      \hline
      $a$ & $A_1$& $A_0$ & $A_0$ & $A_1$ & $A_1$ & $A_1$\\
      \hline
      $b$ & $A_1$& $A_0$ & $A_0$ & $A_1$ & $A_1$ & $A_1$\\
      \hline
    \end{tabular}
    
    Виждаме, че $1 \not\equiv_1 5$ и $1 \equiv_0 5$.
    Следователно, $\equiv_0\ \neq\ \equiv_1$.
    Класовете на еквивалентност на $\equiv_1$ са 
    $B_0 = \{0,3,4\}$, $B_1 = \{1,2\}$, $B_2 = \{5\}$.
  \item
    Сега се опитваме да разбием класовете на еквивалентност на $\equiv_1$.

    \begin{tabular}{|c|c|c|c|c|c|c|}
      \hline
      $Q$ & 0 & $1^\star$ & $2^\star$ & 3 & 4 & $5^\star$ \\
      \hline
      \hline
      $\equiv_1$ & $B_0$ & $B_1$ & $B_1$ & $B_0$ & $B_0$ & $B_2$\\
      \hline
      $a$ & $B_1$ & $B_0$ & $B_0$ & $B_2$ & $B_2$ & $B_2$\\
      \hline
      $b$ & $B_1$ & $B_0$ & $B_0$ & $B_2$ & $B_2$ & $B_2$\\
      \hline
    \end{tabular}
    
    Имаме, че $0 \equiv_1 3$, но $0 \not\equiv_2 3$. Следователно $\equiv_1\ \neq\ \equiv_2$.
    Класовете на еквивалентност на $\equiv_2$ са 
    $C_0 = \{0\}$, $C_1 = \{1,2\}$, $C_2 = \{3,4\}$, $C_3 = \{5\}$.
  \item
    Отново опитваме да разбием класовете на $\equiv_2$.

      \begin{tabular}{|c|c|c|c|c|c|c|}
        \hline
        $Q$ & 0 & $1^\star$ & $2^\star$ & 3 & 4 & $5^\star$ \\
        \hline
        \hline
        $\equiv_2$ & $C_0$ & $C_1$ & $C_1$ & $C_2$ & $C_2$ & $C_3$\\
        \hline
        $a$ & $C_1$ & $C_2$ & $C_2$ & $C_3$ & $C_3$ & $C_3$\\
        \hline
        $b$ & $C_1$ & $C_2$ & $C_2$ & $C_3$ & $C_3$ & $C_3$\\
        \hline
      \end{tabular}
      
      Виждаме, че не можем да разбием $C_1$ или $C_2$.
      \marginpar{Получаваме, че $\equiv_\A\ =\ \equiv_2$}
      Следователно, $\equiv_2\ =\ \equiv_3$ и минималният автомат разпознаващ езика $L$
      има четири състояния. Вижте Фигура \ref{sub:min2} за преходите на минималния автомат.
      Минималният автомат може да се представи и таблично:

      \begin{tabular}{|c|c|c|c|c|}
        \hline
        $\delta$ & $C_0$ & $C_1$ & $C_2$ & $C_3$ \\
        \hline
        $a$ & $C_1$ & $C_2$ & $C_3$ & $C_3$ \\
        \hline
        $b$ & $C_1$ & $C_2$ & $C_3$ & $C_3$ \\
        \hline
      \end{tabular}
      
  \end{itemize}
\end{example}


%%% Local Variables:
%%% mode: latex
%%% TeX-master: "../eai"
%%% End:

\subsection{Кубичен алгоритъм за минимизация}

При даден език $L$ и детерминиран краен автомат $\A = \FA$, който го разпознава, целта ни е да построим нов детерминиран краен автомат $\A'$,
който има толкова състояния колкото са класовете на еквивалентност на релацията $\equiv_\A$.
Това ще направим като ,,слеем'' състоянията на $\A$, които са еквивалентни относно релацията $\equiv_\A$.
Проблемът с намирането на класовете на еквивалентност на релацията $\equiv_\A$ е кванторът $(\forall \omega \in \Sigma^\star)$
във нейната дефиниция (чрез Формула \ref{eq:1}), защото $\Sigma^\star$ е безкрайно множество от думи.
За да разрешим този проблем, ще разгледаме апроксимации на езиците $\L_\A(q)$.
За ествено число $n$, да означим 
\[\L^n_\A(p) \df \{\omega \in \Sigma^\star \mid \abs{\omega} \leq n\ \&\ \delta^\star(p,\omega) \in F\}.\]
Лесно се съобразява, че
\mynote{Можем ли да дадем горна граница на $n$?}
\[L(\A) = \bigcup_{n\geq 0} \L^n_\A(\qstart).\]

\index{$\equiv^n_\A$}
За всяко естествено число $n$, дефинираме бинарните релации $\equiv^n_\A$ върху $Q$ по следния начин:
\[p \equiv^n_\A q \dff \L^n_\A(p) = \L^n_\A(q).\]

Релациите $\equiv^n_\A$ представляват апроксимации на релацията $\equiv_\A$.
Обърнете внимание, че за всяко $n$, $\equiv^n_\A$ е {\em по-груба} релация от $\equiv^{n+1}_\A$, 
която на свой ред е по-груба от $\equiv_\A$.
Алгоритъмът строи $\equiv^n_\A$ докато не срещнем $n$, за което
\[\equiv^n_\A\ =\ \equiv^{n+1}_\A.\]
Тъй като броят на класовете на еквивалентност на $\equiv_\A$ е краен (той е $\leq \abs{Q}$), то 
със сигурност ще намерим такова $n$, за което $\equiv^n_\A\ =\ \equiv^{n+1}_\A$.
Тогава заключаваме, че за това $n$ имаме, че
\[\equiv_\A\ =\ \equiv^n_\A.\]

\mynote{Ако $q \in F$, то $\L^0_\A(q) = F$ и ако $q \not\in F$, то $\L^0_\A(q) = Q\setminus F$.}
Понеже единствената дума с дължина $0$ e $\varepsilon$ и по определение $\delta^\star(p,\varepsilon) = p$, 
лесно се съобразява, че $\equiv^0_\A$ има два класа на еквивалентност.
Единият е $F$, а другият е $Q\setminus F$.

Вече имаме базовия случай за $n=0$.
Да видим сега как можем да намерим $\equiv^{n+1}_\A$ при положение, че вече сме намерили $\equiv^n_\A$.
\begin{framed}
  \begin{proposition}\label{pr:one-letter-test}
    За всеки две състояния $p$ и $q$, и всяко естествено число $n$, $p \equiv^{n+1}_\A q$ точно тогава, когато
    \begin{enumerate}[a)]
    \item
      $p \equiv^n_\A q$ и
    \item
      $(\forall a \in \Sigma)[\delta(q,a) \equiv^n_\A \delta(p,a)]$.
    \end{enumerate}
  \end{proposition}  
\end{framed}
\begin{hint}
  \mynote{\cite[стр. 99]{papadimitriou}}
  \begin{align*}
    p \equiv^{n+1}_\A q & \iff \L^{n+1}_\A(p) = \L^{n+1}_\A(q)\\
                     & \iff \L^n_\A(p) = \L^n_\A(q)\ \&\ (\forall a \in \Sigma)[\ \L^n_\A(\delta(p,a)) = \L^n_\A(\delta(q,a))\ ]\\
                     & \iff p \equiv^n_\A q\ \&\ (\forall a \in \Sigma)[\ \delta(p,a) \equiv^n_\A \delta(q,a)\ ].
  \end{align*}
\end{hint}

\begin{proposition}\label{pr:minimisation-cubic:equiv-all}
  Ако $\equiv^n_\A\ =\ \equiv^{n+1}_\A$, то за всяко $m > n$ е изпълнено, че:
  \begin{equation}
    \label{eq:8}
    \equiv^n_\A\ =\ \equiv^m_\A.
  \end{equation}
\end{proposition}
\begin{proof}
  Ще докажем Свойство~(\ref{eq:8}) с индукция по $m > n$, като базата на индукцията е случаят $m = n + 1$, за който Свойство~(\ref{eq:8})
  е изпълнено по условие.
  Индукционното ни предположение е, че $\equiv^n_\A\ =\ \equiv^{m}_\A$ за някое $m > n+1$.
  Индукционната ни стъпка е да докажем, че $\equiv^n_\A\ =\ \equiv^{m+1}_\A$. За произволни състояния $p$ и $q$ имаме следните еквивалентности:
  \begin{align*}
    p \equiv^{m+1}_\A q & \iff p \equiv^m_\A q\ \&\ (\forall a \in \Sigma)[\delta(p,a) \equiv^m_\A \delta(q,a)] & \comment\text{от \Proposition{one-letter-test}}\\
                        & \iff p \equiv^n_\A q\ \&\ (\forall a \in \Sigma)[\delta(p,a) \equiv^n_\A \delta(q,a)] & \comment\text{от И.П.}\\
                        & \iff p \equiv^{n+1}_\A q & \comment\text{от \Proposition{one-letter-test}}\\
                        & \iff p \equiv^n_\A q. & \comment\text{от условието}
  \end{align*}
\end{proof}

\begin{proposition}\label{pr:minimisation-cubic:equiv-approx}
  Докажете, че за произволно естествено число $n$,
  \[\equiv^n_\A\ =\ \equiv^{n+1}_\A\ \implies\ \equiv^n_\A\ =\ \equiv_\A.\]
\end{proposition}
\begin{proof}
  Нека $\equiv^n_\A\ =\ \equiv^{n+1}_\A$.
  Ще докажем, че за произволни състояния $p$ и $q$ е изпълено, че:
  \[p \equiv_\A q \iff p \equiv^n_\A q.\]
  Ясно е, че $p \equiv_\A q \implies p \equiv^n_\A q$. 
  Да видим защо имаме и обратната посока, т.е. защо $p \equiv^n_\A q \implies p \equiv_\A q$.
  Ние ще докажем контрапозицията на импликацията, т.е. ще докажем следното:
  \[p \not\equiv_\A q \implies p \not\equiv^n_\A q.\]
  И така, нека $p \not\equiv_\A q$. Това означава, че същестува дума $\alpha$, за която:
  \[\neg(\delta^\star(p,\alpha) \in F \iff \delta^\star(q,\alpha) \in F).\]
  Това означава, че $p \not\equiv^{|\alpha|}_\A q$. Имаме два случая.
  \begin{itemize}
  \item
    Нека $|\alpha| \leq n$.
    Тогава $p \not\equiv^n_\A q$, защото от дефиницията следва, че
    \[p \not\equiv^{|\alpha|}_\A q \implies p \not\equiv^n_\A q.\]
  \item
    Ако $|\alpha| > n$, от \Proposition{minimisation-cubic:equiv-all} следва, че
    \[p \equiv^n_\A q \implies p \equiv^{|\alpha|}_\A q,\]
    чиято контрапозиция е:
    \[p \not\equiv^{|\alpha|}_\A q \implies p \not\equiv^{n}_\A q.\]
    Заключаваме, че $p \not\equiv^{n}_\A q$.
  \end{itemize}
\end{proof}

\begin{proposition}
  За всеки ДКА $\A$ е изпълнено, че 
  $\equiv^{|Q|}_\A\ =\ \equiv_\A$.
\end{proposition}
\begin{proof}
  Ако съществува $n < |Q|$, такова че $\equiv^n_\A\ =\ \equiv^{n+1}_\A$, то според предишните две твърдения,
  то $\equiv^n_\A\ =\ \equiv^{|Q|}_\A\ =\ \equiv_\A$.

  Нека сега приемем, че за всяко $n < |Q|$ е изпълнено, че $\equiv^n_\A\ \neq\ \equiv^{n+1}_\A$.
  Това означава, че всеки клас на еквивалентност на $\equiv^{|Q|}_\A$ съдържа точно едно състояние, защото
  всеки клас на еквивалентност съдържа поне едно състояние, а ние имаме точно $|Q|$ на брой състояния.
  Тогава със сигурност имаме, че $\equiv^{|Q|}_\A\ =\ \equiv^{|Q|+1}_\A$,
  защото
  \[p \equiv^{|Q|+1}_\A q \implies p \equiv^{|Q|}_\A q\]
  и няма как $\equiv^{|Q|+1}_\A$ да ,,раздроби'' някой клас на $\equiv^{|Q|}_\A$.
  Тогава от \Proposition{minimisation-cubic:equiv-approx} следва, че $\equiv^n_\A\ =\ \equiv_\A$.
\end{proof}

% \mynote{\cite{khoussainov-nerode}}



\begin{algorithm}[H]
  \caption{Кубичен алгоритъм за минимизация}
  \label{alg:minimisation-cube}
  \begin{algorithmic}[1]
    \ForAll{$p < |Q|$}\Comment{състоянията са индексирани от $0$ до $|Q|-1$}
    \ForAll{$q < p$} 
    \If{$(p \in F \iff q \not\in F)$}
    \State E[p][q] = false \Comment{Имаме, че $p \not\equiv q$}
    \Else
    \State E[p][q] = true \Comment{Имаме, че $p \equiv q$}
    \EndIf
    \EndFor
    \EndFor
    \Repeat
    \State ready = true
    \ForAll{$p < |Q|$}
    \ForAll{$q < p$}
    \If{E[p][q]}
    \ForAll{$a \in \Sigma$}
    \If{! E[$\delta(p,a)$][$\delta(q,a)$]}  \Comment{$p \equiv q$, но $\delta(p,a) \not\equiv \delta(q,a)$}
    \State{E[p][q] = false}
    \State{ready = false}
    \EndIf
    \EndFor
    \EndIf
    \EndFor
    \EndFor
    \Until{ready}
  \end{algorithmic}
\end{algorithm}

\begin{proposition}
  За всеки две състояния $q_i,q_j \in Q$, където $i > j$,
  \[q_i \equiv_\A q_j \iff \texttt{E[i][j] == true}.\]
\end{proposition}

\begin{problem}
  Съобразете, че \Algorithm{minimisation-cube} има времева сложност $\mathcal{O}(|\A|^3)$.
\end{problem}


%%% Local Variables:
%%% mode: latex
%%% TeX-master: "../eai"
%%% End:

\subsection{Примерни задачи}
\mynote{\cite[стр. 79]{kozen}}

\begin{problem}
  Постройте минимален автомат $\M$ разпознаващ езика на следния детерминиран краен автомат $\A$.
  \mynote{\writedown Съобразете, че \[\L(\A) = \{\omega \in \{a,b\}^\star \mid \abs{\omega} \geq 2\}.\]}
  \begin{framed}
    \begin{figure}[H]
      \begin{subfigure}[b]{0.45\textwidth}
        \begin{tikzpicture}[->,>=stealth,thick,node distance=55pt]
          \tikzstyle{every state}=[circle,minimum size=20pt,auto]
          
          \node[initial above, state]   (0) {$0$};
          \node[state]            (1) [above right of=0]{$1$};
          \node[state]            (2) [below right of=0]{$2$};
          \node[state,accepting]  (3) [right of=1]{$3$};
          \node[state,accepting]  (4) [right of=2]{$4$};
          \node[state,accepting]  (5) [below right of=3]{$5$};
          
          \path 
          (0) edge  node [above] {$a$}   (1)
          (0) edge  node [below] {$b$}   (2)
          (1) edge node [above] {$a$}    (3)
          (1) edge [bend left=15] node [below] {$b$}    (4)
          (2) edge [bend left=15] node [left] {$b$}    (3)
          (2) edge node [below] {$a$}   (4)
          (4) edge [bend right=15] node [right] {$a,b$} (5)
          (3) edge [bend left=15] node [right] {$a,b$}  (5)
          (5) edge [loop right]   node [above] {$a,b$}  (5);
        \end{tikzpicture}
        \caption{Ще построим минимален автомат разпознаващ езика на $\A$.}
      \end{subfigure}
      \quad
      \quad
      \begin{subfigure}[b]{0.4\textwidth}
        \begin{tikzpicture}[->,>=stealth,thick,node distance=45pt]
        \tikzstyle{every state}=[circle,minimum size=20pt,auto,scale=.9]
        
        \node[initial above, state]   (0) {$B_0$};
        \node[state]                  (1) [right of=0]{$B_1$};
        \node[state,accepting]        (2) [right of=1]{$B_2$};
        
        \path 
        (0) edge [bend left=15] node [above] {$a,b$}   (1)
        (1) edge [bend left=15] node [above] {$a,b$}   (2)
        (2) edge [loop above] node [above] {$a,b$}   (2);
      \end{tikzpicture}
      \caption{Получаваме минималния автомат $\M$, $\L(\M) = \L(\A)$}
      \label{sub:min1}
    \end{subfigure}
  \end{figure}
\end{framed}
\end{problem}

\ExtraMaterial{
  \begin{solution}
Ще приложим алгоритъма за минимизация за да получим минималния автомат за езика $L$.
За всяко $n = 0,1,2,\dots$, ще намерим класовете на еквивалентност на $\equiv^n_\A$,
докато не намерим $n$, за което $\equiv^n_\A\ =\ \equiv^{n+1}_\A$.

\begin{multicols}{2}
\begin{itemize}
\item 
  Класовете на еквивалентност на $\equiv^0_\A$ са два. Те са
  \[A_0 = Q\setminus F = \{0,1,2\}\text{ и }A_1 = F = \{3,4,5\}.\]
\item
  Сега да видим дали можем да разбием някои от класовете на еквивалентност на $\equiv^0_\A$.
  
  \begin{tabular}{|c|c|c|c|c|c|c|}
    \hline
    $Q$ & $0$ & $1$ & $2$ & $3^\star$ & $4^\star$ & $5^\star$ \\
    \hline
    \hline
    $\equiv^0_\A$ & $A_0$ & $A_0$ & $A_0$ & $A_1$ & $A_1$ & $A_1$\\
    \hline
    $a$ & $A_0$& $A_1$ & $A_1$ & $A_1$ & $A_1$ & $A_1$\\
    \hline
    $b$ & $A_0$& $A_1$ & $A_1$ & $A_1$ & $A_1$ & $A_1$\\
    \hline
  \end{tabular}

  Виждаме, че $0 \not\equiv^1_\A 1$ и $1 \equiv^1_\A 2$.
  Класовете на еквивалентност на $\equiv^1_\A$ са 
  $B_0 = \{0\}$, $B_1 = \{1,2\}$, $B_2 = \{3,4,5\}$.
\item
  Сега да видим дали можем да разбием някои от класовете на еквивалентност на $\equiv^1_\A$.
  
  \begin{tabular}{|c|c|c|c|c|c|c|}
    \hline
    $Q$ & $0$ & $1$ & $2$ & $3^\star$ & $4^\star$ & $5^\star$ \\
    \hline
    \hline
    $\equiv^1_\A$ & $B_0$ & $B_1$ & $B_1$ & $B_2$ & $B_2$ & $B_2$\\
    \hline
    $a$ & $B_1$ & $B_2$ & $B_2$ & $B_2$ & $B_2$ & $B_2$\\
    \hline
    $b$ & $B_1$ & $B_2$ & $B_2$ & $B_2$ & $B_2$ & $B_2$\\
    \hline
  \end{tabular}

  Виждаме, че $\equiv^1_\A\ =\ \equiv^2_\A$,
  което означава, че $\equiv_\A\ =\ \equiv^1_\A$.
  Следователно, минималният автомат има три състояния.
  Той е изобразен на Фигура \ref{sub:min1}.  
  Минималният автомат може да се представи и таблично:
  
  \begin{tabular}{|c|c|c|c|c|c|c|}
    \hline
    $\delta$ & $B_0$ & $B_1$ & $B_2$ \\
    \hline
    $a$ & $B_1$ & $B_2$ & $B_2$ \\
    \hline
    $b$ & $B_1$ & $B_2$ & $B_2$ \\
    \hline
  \end{tabular}
\end{itemize}
\end{multicols}

\end{solution}
}

\begin{problem}
  Постройте минимален автомат $\M$ разпознаващ езика на следния детерминиран краен автомат $\A$.
  \begin{framed}
  \begin{figure}[H]
    % \begin{center}
    \begin{subfigure}[b]{0.4\textwidth}
      \begin{tikzpicture}[->,>=stealth,thick,node distance=55pt]
        \tikzstyle{every state}=[circle,minimum size=20pt,auto]
        
        \node[initial above, state]   (0) {$0$};
        \node[state,accepting]        (1) [above right of=0]{$1$};
        \node[state,accepting]        (2) [below right of=0]{$2$};
        \node[state]                  (3) [right of=1]{$3$};
        \node[state]                  (4) [right of=2]{$4$};
        \node[state,accepting]        (5) [below right of=3]{$5$};
        
        \path 
        (0) edge [bend left=15] node [below] {$a$}   (1)
            edge [bend right=15] node [below] {$b$}   (2)
        (1) edge node [above] {$a$}    (3)
            edge [bend left=15] node [below] {$b$}    (4)
        (2) edge [bend left=15] node [left] {$b$}    (3)
            edge node [below] {$a$}   (4)
        (4) edge [bend right=15] node [below] {$a,b$} (5)
        (3) edge [bend left=15] node [right] {$a,b$}  (5)
        (5) edge [loop right]   node [above] {$a,b$}  (5);
      \end{tikzpicture}
      \caption{Ще построим минимален автомат разпознаващ $\L(\A)$}
    \end{subfigure}
    \qquad
    \qquad
    \begin{subfigure}[b]{0.4\textwidth}
      \begin{tikzpicture}[->,>=stealth,thick,node distance=45pt]
        \tikzstyle{every state}=[circle,minimum size=20pt,auto,scale=.9]
        
        \node[initial above, state]   (0) {$C_0$};
        \node[state,accepting]  (1) [right of=0]{$C_1$};
        \node[state]            (2) [right of=1]{$C_2$};
        \node[state,accepting]  (3) [right of=2]{$C_3$};
                
        \path 
        (0) edge [bend left=15] node [above] {$a,b$}   (1)
        (1) edge [bend left=15] node [above] {$a,b$}   (2)
        (2) edge [bend left=15] node [above] {$a,b$}   (3)
        (3) edge [loop above]   node [above] {$a,b$}   (3);
      \end{tikzpicture}
      \caption{Получаваме минималния автомат $\M$, $\L(\M) = \L(\A)$}
      \label{sub:min2}
    \end{subfigure}
  \end{figure}
  \end{framed}
\end{problem}

\ExtraMaterial{
  \begin{solution}
    \mynote{\writedown Съобразете, че $\L(\A) = \{\omega \in \{a,b\}^\star \mid \abs{\omega} = 1 \lor \abs{\omega} \geq 3\}$.}
    Отново следваме същата процедура за минимизация.
    Ще намерим класовете на еквивалентност на $\equiv^n_\A$,
    докато не намерим $n$, за което $\equiv^n_\A\ =\ \equiv^{n+1}_\A$.
    \begin{multicols}{2}
  \begin{itemize}
  \item
    Класовете на екиваленост на $\equiv^0_\A$ са следните:
    \[A_0 \df Q\setminus F = \{0,3,4\} \text{ и }A_1 \df F = \{1,2,5\}.\]
  \item
    Разбиваме класовете на еквивалентност на $\equiv^0_\A$ като използваме \Proposition{one-letter-test}.

    \begin{tabular}{|c|c|c|c|c|c|c|}
      \hline
      $Q$ & 0 & $1^\star$ & $2^\star$ & 3 & 4 & $5^\star$ \\
      \hline
      \hline
      $\equiv^0_\A$ & $A_0$ & $A_1$ & $A_1$ & $A_0$ & $A_0$ & $A_1$\\
      \hline
      $a$ & $A_1$& $A_0$ & $A_0$ & $A_1$ & $A_1$ & $A_1$\\
      \hline
      $b$ & $A_1$& $A_0$ & $A_0$ & $A_1$ & $A_1$ & $A_1$\\
      \hline
    \end{tabular}
    
    Виждаме, че $1 \not\equiv^1_\A 5$ и $1 \equiv^0_\A 5$.
    Следователно, $\equiv^0_\A\ \neq\ \equiv^1_\A$.
    Класовете на еквивалентност на $\equiv^1_\A$ са следните:
    \[B_0 \df \{0,3,4\},\ B_1 \df \{1,2\},\ B_2 \df \{5\}.\]
  \item
    Сега се опитваме да разбием класовете на еквивалентност на $\equiv^1_\A$.
    
    \begin{tabular}{|c|c|c|c|c|c|c|}
      \hline
      $Q$ & 0 & $1^\star$ & $2^\star$ & 3 & 4 & $5^\star$ \\
      \hline
      \hline
      $\equiv^1_\A$ & $B_0$ & $B_1$ & $B_1$ & $B_0$ & $B_0$ & $B_2$\\
      \hline
      $a$ & $B_1$ & $B_0$ & $B_0$ & $B_2$ & $B_2$ & $B_2$\\
      \hline
      $b$ & $B_1$ & $B_0$ & $B_0$ & $B_2$ & $B_2$ & $B_2$\\
      \hline
    \end{tabular}
    
    Имаме, че $0 \equiv^1_\A 3$, но $0 \not\equiv^2_\A 3$. Следователно $\equiv^1_\A\ \neq\ \equiv^2_\A$.
    Класовете на еквивалентност на $\equiv^2_\A$ са следните:
    \[C_0 \df \{0\},\ C_1 \df \{1,2\},\ C_2 \df \{3,4\},\ C_3 \df \{5\}.\]
  \item
    Отново опитваме да разбием класовете на $\equiv^2_\A$.

      \begin{tabular}{|c|c|c|c|c|c|c|}
        \hline
        $Q$ & 0 & $1^\star$ & $2^\star$ & 3 & 4 & $5^\star$ \\
        \hline
        \hline
        $\equiv^2_\A$ & $C_0$ & $C_1$ & $C_1$ & $C_2$ & $C_2$ & $C_3$\\
        \hline
        $a$ & $C_1$ & $C_2$ & $C_2$ & $C_3$ & $C_3$ & $C_3$\\
        \hline
        $b$ & $C_1$ & $C_2$ & $C_2$ & $C_3$ & $C_3$ & $C_3$\\
        \hline
      \end{tabular}
      
      Виждаме, че не можем да разбием $C_1$ или $C_2$.
      Следователно, $\equiv^2_\A\ =\ \equiv^3_\A$.
      Оттук следва, че $\equiv^2_\A\ =\ \equiv_\A$ и минималният автомат разпознаващ езика $L$
      има четири състояния. Вижте Фигура \ref{sub:min2} за преходите на минималния автомат,
      които могат да се представят и таблично чрез функцията на преходите:

      \begin{tabular}{|c|c|c|c|c|}
        \hline
        $\delta$ & $C_0$ & $C_1$ & $C_2$ & $C_3$ \\
        \hline
        $a$ & $C_1$ & $C_2$ & $C_3$ & $C_3$ \\
        \hline
        $b$ & $C_1$ & $C_2$ & $C_3$ & $C_3$ \\
        \hline
      \end{tabular}
      
  \end{itemize}      
    \end{multicols}
\end{solution}
}

%%% Local Variables:
%%% mode: latex
%%% TeX-master: "../eai"
%%% End:

\section{Неограничени граматики}
\index{граматика!неограничена}
\label{sect:regular-grammar}
\mynote{На англ. {\em unrestricted grammar}. Това е тип 0 граматики в йерархията на Чомски \cite[стр. 220]{hopcroft1}.}
{\bf Неограничена граматика} e наредена четворка от вида
\[G = (V, \Sigma, R, S),\]
където:
\begin{itemize}
\item
  $V$ е крайно множество от {\em променливи} (нетерминали);
\item
  $\Sigma$ е крайно множество от {\em букви} (терминали), $\Sigma \cap V = \emptyset$;
\item
  \mynote{В \cite{hopcroft1} правилата се наричат {\em productions} или {\em production rules}.}
  $R \subseteq (V\cup\Sigma)^+ \times (V \cup \Sigma)^\star$ е крайно множество от {\em правила}.
  За по-добра яснота, обикновено правилата $(\alpha, \beta) \in R$ ще означаваме като 
  $\alpha \to_G \beta$.
\item
  $S \in V$ е началната променлива (нетерминал). 
\end{itemize}

\index{граматика!извод}
Удобно е да дефинираме извод на думата $\beta$ от думата $\alpha$ в граматиката $G$ за $\ell$ стъпки, което ще означаваме като $\alpha \derive{\ell}_G \beta$,
с индукция по броя на стъпките $\ell$ по следния начин:
\mynote{Обърнете внимание, че имаме недетерминизъм в тази дефиниция на извод. Също така, понякога ,за удобство, ще пишем просто $\derive{\ell}$ вместо $\derive{\ell}_G$,
  когато се знае за коя граматика говорим.}
\begin{prooftree}
  \AxiomC{}
  \UnaryInfC{$\alpha \derive{0}_G \alpha$}
\end{prooftree}

\begin{prooftree}
  \AxiomC{$\alpha \to_G \gamma$}
  \AxiomC{$\gamma \derive{\ell}_G \beta$}
  \BinaryInfC{$\alpha \derive{\ell+1}_G \beta$}
\end{prooftree}

\begin{prooftree}
  \AxiomC{$\alpha_1 \derive{\ell_1}_G \beta_1$}
  \AxiomC{$\alpha_2 \derive{\ell_2}_G \beta_2$}
  \BinaryInfC{$\alpha_1\cdot\alpha_2 \derive{\ell_1+\ell_2}_G \beta_1\cdot \beta_2$}
\end{prooftree}

\mynote{В частност имаме следното:
\begin{prooftree}
  \AxiomC{$\alpha \to_G \beta$}
  \UnaryInfC{$\alpha \derive{1}_G \beta$}
\end{prooftree}
}

Нека $\derive{\star}_G$ е рефлексивното и транзитивно затваряне на релацията $\derive{1}_G$. С други думи,
\[ \alpha \derive{\star}_G \beta\ \iff\ (\exists \ell\in\Nat)[\ \alpha \derive{\ell}_G \beta\ ].\]
Езикът, който се поражда от граматиката $G$ е следния:
\[\L(G) \df \{\omega \in \Sigma^\star \mid S \derive{\star}_G \omega\}.\]

\begin{proposition}\label{pr:grammar:add}
  За произволно естествено число $\ell$ е изпълнено, че:
  \begin{prooftree}
    \AxiomC{$\alpha \derive{\ell}_G \beta$}
    \AxiomC{$\gamma,\rho \in (V \cup \Sigma)^\star$}
    \BinaryInfC{$\gamma\alpha\rho \derive{\ell} \gamma \beta \rho$}
  \end{prooftree}
\end{proposition}

\begin{proposition}\label{pr:grammar:concat}
  За всяко $k$ е изпълнено, че:
  \begin{prooftree}
    \AxiomC{$\alpha_1 \derive{\ell_1} \beta_1$}
    \AxiomC{$\dots$}
    \AxiomC{$\alpha_k \derive{\ell_k} \beta_k$}
    \RightLabel{\scriptsize{$(\ell = \sum^k_{i=1} \ell_i)$}}
    \TrinaryInfC{$\alpha_1\cdots\alpha_k \derive{\ell} \beta_1\cdots\beta_k$}
  \end{prooftree}
\end{proposition}
\begin{hint}
  Индукция по $k$.
\end{hint}

\begin{proposition}
  За произволни естествени числа $\ell_1$ и $\ell_2$ е изпълнено, че:
  \begin{prooftree}
    \AxiomC{$\alpha \derive{\ell_1} \beta$}
    \AxiomC{$\beta \derive{\ell_2} \gamma$}
    \BinaryInfC{$\alpha \derive{\ell_1+\ell_2} \gamma$}
  \end{prooftree}  
\end{proposition}
\begin{hint}
  Индукция по $\ell_1$.
\end{hint}



%%% Local Variables:
%%% mode: latex
%%% TeX-master: "../eai"
%%% End:

\section{Регулярни граматики}
\index{граматика!неограничена}
\label{sect:regular-grammar}
\mynote{На англ. {\em unrestricted grammar}. Това е тип 0 граматики в йерархията на Чомски \cite[стр. 220]{hopcroft1}.}
{\bf Неограничена граматика} e наредена четворка от вида
\[G = (V, \Sigma, R, S),\]
където:
\begin{itemize}
\item
  $V$ е крайно множество от {\em променливи} (нетерминали);
\item
  $\Sigma$ е крайно множество от {\em букви} (терминали), $\Sigma \cap V = \emptyset$;
\item
  \mynote{В \cite{hopcroft1} правилата се наричат {\em productions} или {\em production rules}.}
  $R \subseteq (V\cup\Sigma)^+ \times (V \cup \Sigma)^\star$ е крайно множество от {\em правила}.
  За по-добра яснота, обикновено правилата $(\alpha, \beta) \in R$ ще означаваме като 
  $\alpha \to_G \beta$.
\item
  $S \in V$ е началната променлива (нетерминал). 
\end{itemize}

\index{граматика!извод}
Удобно е да дефинираме извод на думата $\beta$ от думата $\alpha$ в граматиката $G$ за $\ell$ стъпки, което ще означаваме като $\alpha \derive{\ell} \beta$,
с индукция по броя на стъпките $\ell$ по следния начин:
\begin{prooftree}
  \AxiomC{}
  \UnaryInfC{$\alpha \derive{0} \alpha$}
\end{prooftree}

\begin{prooftree}
  \AxiomC{$(\beta,\beta') \in R$}
  \UnaryInfC{$\alpha\beta\gamma \derive{1} \alpha\beta'\gamma$}
\end{prooftree}

\begin{prooftree}
  \AxiomC{$\alpha \derive{\ell} \gamma$}
  \AxiomC{$\gamma \derive{1} \beta$}
  \BinaryInfC{$\alpha \derive{\ell+1} \beta$}
\end{prooftree}

Нека $\derive{\star}$ е рефлексивното и транзитивно затваряне на релацията $\derive{1}$. С други думи,
\[ \alpha \derive{\star} \beta\ \iff\ (\exists \ell\in\Nat)[\ \alpha \derive{\ell} \beta\ ].\]
Езикът, който се поражда от граматиката $G$ е
\[\L(G) \df \{\omega \in \Sigma^\star \mid S \derive{\star} \omega\}.\]

Граматиките се разделят на няколко вида в зависимост от това какви {\em ограничения} налагаме върху правилата $R$.
В следващите няколко глави ще разгледаме различни ограничения. Сега ще разгледаме граматики с такъв вид правила,
които пораждат точно регулярните (или еквивалентно автоматни) езици.

\index{граматика!регулярна}
\index{граматика!тип 3}
\mynote{Също така се наричат граматики от тип 3 в йерархията на Чомски \cite[стр. 217]{hopcroft1}.}
Граматиката $G = (V, \Sigma, R, S)$ се нарича {\bf регулярна граматика},
ако всички правила са от вида 
\begin{align*}
  & A \to aB,\\
  & A \to a,\\
  & A \to \varepsilon,
\end{align*}
за произволни $A, B \in V$ и $a \in \Sigma$.
% Ако началната променлива $S$ не се среща в дясна част на правило, то позволяваме и правилото $S \to \varepsilon$,
% ако искаме $\varepsilon \in \L(G)$.

\begin{lemma}
  За всяка регулярна граматика $G$ съществува НКА $\N$, такъв че $\L(G) = \L(\N)$.
\end{lemma}
\begin{hint}
  Нека $G = \CFG$ и $V = \{A_0,\dots,A_k\}$, където $S = A_0$. Тогава дефинираме $\N$ по следния начин:
  \begin{itemize}
  \item
    $Q \df \{q_0,\dots,q_k,f\}$;
  \item
    $Q_{\texttt{start}} \df \{q_0\}$;
  \item
    $F \df \{q_i \mid A_i \to \varepsilon\} \cup \{f\}$;
  \item
    Релацията на преходите $\Delta$ е дефинирана по следния начин:
    \begin{align*}
      \Delta(q_i,a) \df & \{ q_j\ \mid\ A_i \to aA_j \text{ е правило в граматиката}\}\ \cup\\
                      & \{ f\ \mid\ A_i \to a \text{ е правило в граматиката}\}.
    \end{align*}
  \end{itemize}
  Докажете, че $\L(\N) = \L(G)$.
\end{hint}

\begin{lemma}
  За всеки ДКА $\A$ съществува регулярна граматика $G$, такава че $\L(\A)~=~\L(G)$.
\end{lemma}
\begin{hint}
  Нека $\A = \FA$ и $Q = \{q_0,\dots,q_k\}$, където $\qstart = q_0$. Тогава дефинираме $G = \CFG$ по следния начин:
  \begin{itemize}
  \item 
    $V \df \{A_0,\dots,A_k\}$;
  \item
    $S \df A_0$;
  \item
    $A_i \to aA_j\ \dff\ \delta(q_i,a) = q_j$;
  \item
    $A_{i} \to \varepsilon\ \dff\ q_{i} \in F$.
  \end{itemize}
  Докажете, че $\L(\A) = \L(G)$.
\end{hint}

\begin{framed}
  \begin{theorem}
    Един език е регулярен точно тогава, когато се поражда от регулярна граматика.
  \end{theorem}
\end{framed}


%%% Local Variables:
%%% mode: latex
%%% TeX-master: "../eai"
%%% End:

\newpage
\section{Допълнителни задачи}

\subsection{Лесни задачи}

\ifcode
\begin{problem}
  \marginnote{Библиотеката {\bf regex} е част от си++11 стандарта на езика.}
  Да разгледаме следната програма на езика си++
  \begin{minted}[]{cpp}
    #include <iostream>
    #include <regex>
    #include <string>
    
    using namespace std;
    
    int main() {
      string input;
      regex reg("your regular expression");
      while (true) {
        cout << "Input:" << endl;
        cin >> input;
        if (!cin || input=="q") break;
        if (regex_match(input, reg)) {
          cout << "Valid input" << endl;
        }  
        else {
          cout << "Invalid input" << endl;
        }
      }
    }
  \end{minted}
  Разучете как се създават обекти от тип {\bf regex} и попълнете дефиницията на регулярния израз {\bf reg} в горната програма, така че програмата приема за валиден вход:
  \begin{enumerate}[a)]
  \item 
    само реални числа;
  \item
    факултетни номера във ФМИ;
  \item
    низове, които са съставени от поне 8 символа, измежду които се включват малки букви, големи букви,
    и цифри.    
  \end{enumerate}
\end{problem}
\fi

% \begin{problem}
%   Да фиксираме една дума $\alpha$ над дадена азбука $\Sigma$.
%   Опишете алгоритъм, който за вход произволен текстов файл, чието съдържание означаваме с $\beta$,
%   отговаря дали думата $\alpha$ се среща в $\beta$.
%   Каква е сложността на този алгоритъм относно дължините на $\alpha$ и $\beta$ ?
% \end{problem}

% \begin{problem}
%   Опишете алгоритъм, който при вход два регулярни израза $\mathbf{r}$ и $\mathbf{s}$,
%   проверява дали $\L(\mathbf{r}) \subseteq \L(\mathbf{s})$.
% \end{problem}

% \setlength{\marginparsep}{18pt}
% \setlength{\oddsidemargin}{5pt}
% \setlength{\evensidemargin}{100pt}
% \setlength{\hoffset}{-30pt}
% \setlength{\voffset}{-30pt}

% \newgeometry{marginparwidth=20pt, evensidemargin=20pt}
% \newgeometry{textwidth=600pt}

% \layout

\ExtraMaterial{
\begin{problem}
  За всеки от следните езици $L$, постройте минимален краен детерминиран автомат $\A$, който разпознава езика $L$, където:
  \marginnote{$\card{\omega}{a} \df $ броят на срещанията на буквата $a$ в думата $\omega$, $\abs{\omega} \df $ дължината на $\omega$.}
  \begin{enumerate}[a)]
  \item 
    $L = \{a^nb\mid n \geq 0\}$;
  \item
    $L = \{a,b\}^\star\setminus\{\varepsilon\}$;
  \item
    $L = \{\omega \in \{a,b\}^\star \mid \card{\omega}{a}\text{ и }\card{\omega}{b}\text{ са четни}\}$;
  \item
    $L = \{\omega \in \{a,b\}^\star \mid \card{\omega}{a}\text{ е четно}\ \&\ \card{\omega}{b}\text{ е нечетно}\}$;
  \item
    $L = \{a^nb^m\mid n,m \geq 0\}$;
  \item
    $L = \{a^nb^m\mid n,m \geq 1\}$;
  \item
    $L = \{a,b\}^\star \setminus \{a\}$;
  \item
    $L = \{\omega \in \{a,b\}^\star \mid \card{\omega}{a} \geq 2\ \lor\ \card{\omega}{b} \leq 3 \}$;
  \item
    $L = \{\omega \in \{a,b\}^\star \mid \card{\omega}{a} \geq 2\ \&\ \card{\omega}{b} \geq 1\}$;
  \item
    $L = \{\omega \in \{a,b\}^\star \mid \text{на всяка нечетна позиция на }\omega\text{ е буквата }a\}$;
  \item
    $L = \{\omega \in \{a,b\}^\star \mid \card{\omega}{a}\text{ е четно }\&\ \card{\omega}{b} \leq 1\}$;
  \item
    $L = \{\omega \in \{a,b\}^\star \mid \abs{\omega} \leq 3\}$;
  \item
    $L = \{\omega \in \{a,b\}^\star \mid \omega \text{ не започва с }ab\}$;
  \item
    $L = \{\omega \in \{a,b\}^\star \mid \omega \text{ завършва с }ab\text{ или ba}\}$;
  \item
    $L = \{\omega \in \{a,b\}^\star \mid \omega\text{ започва или завършва с } a\}$;
  \item
    $L = \{\omega \in \{a,b\}^\star \mid \omega\text{ започва с }a \iff \omega\text{ завършва с }b\}$;
  \item
    $L = \{\omega \in \{a,b\}^\star \mid \abs{\omega} \equiv 0\ (\bmod\ 2)\ \&\ \card{\omega}{a} = 1\}$;
  \item
    $L = \{\omega \in \{a,b\}^\star \mid \text{ всяко }a\text{ в }\omega\text{ се следва веднага от поне едно }b\}$;
  \item
    $L = \{\omega \in \{a,b\}^\star \mid \abs{\omega} \equiv 0 \bmod 3\}$;
  \item
    $L = \{\omega \in \{a,b\}^\star \mid \card{\omega}{a} \equiv 1 \bmod 3\}$;
  \item
    $L = \{\omega \in \{a,b\}^\star \mid \card{\omega}{a} \equiv 0 \bmod 3\ \&\ \card{\omega}{b} \equiv 1 \bmod 2\}$;
  \item
    $L = \{\omega \in \{a,b\}^\star \mid \card{\omega}{a} \equiv 0 \bmod 2\ \vee\ \card{\omega}{b} = 2\}$;
  \item
    $L = \{\omega \in \{a,b\}^\star \mid \omega \text{ съдържа равен брой срещания на }ab\text{ и на }ba\}$;
  \item
    $L = \{\omega_1 \sharp \omega_2 \sharp \omega_3 \mid |\omega_1| \geq 2\ \&\ |\omega_2| \geq 3\ \&\ |\omega_3| \geq 4\ \&\ \omega_i \in \{a,b\}^\star\text{
      за }i = 1,2,3\}$.
  \end{enumerate}
\end{problem}
}

\begin{extra2}
\begin{problem}
  Нека $\Sigma = \{a,b\}$.  Проверете дали $L$ е регулярен, където
  % \begin{multicols}{2}
  \begin{enumerate}[a)]
  % \item
  %   $L = \{\alpha^{\texttt{rev}} \mid \alpha \in L_0\}$, където $L_0$ е регулярен;
  \item
    % \marginnote{$\alpha = a^pb^p$}
    $L = \{a^ib^i\ \mid\ i\in\Nat\}$;
  \item
    $L = \{a^ib^i\ \mid\ i,j\in\Nat\ \&\ i\neq j\}$;
  \item
    % \marginnote{$\alpha = a^{p+1}b^p$.}
    $L = \{a^ib^j\ \mid\ i > j\}$;
  \item
    $L = \{a^nb^m \mid n\text{ дели }m\}$.
  \item
    $L = \{a^{2n}\ \mid\ n\geq 1\}$;
  \item
    $L = \{a^mb^na^{m+n}\ \mid\ m\geq 1\ \&\ n\geq 1\}$;
  \item
    $L = \{a^{n.m}\mid n,m\text{ са прости числа}\}$;
  \item
    $L = \{\omega\in\{a,b\}^\star \mid \card{\omega}{a} = \card{\omega}{b}\}$;
  \item
    % \marginnote{$\alpha = a^pba^pb$}
    $L = \{\omega\omega\mid \omega\in\{a,b\}^\star\}$;
  \item
    $L = \{\omega\omega^\rev\mid \omega\in\{a,b\}^\star\}$;
  \item
    $L = \{\alpha\beta\beta \in \{a,b\}^\star\mid \beta \neq \varepsilon\}$;
  \item
    $L = \{a^nb^nc^n\mid n\geq 0\}$;
  \item
    $L = \{\omega\omega\omega\mid \omega\in \Sigma^\star\}$;
  \item
    $L = \{a^{2^n}\mid n\geq 0\}$;
  \item
    $L = \{a^mb^n\mid n\neq m\}$;
  \item
    $L = \{a^{n!}b^{n!}\mid n\neq 1\}$;
  \item
    $L = \{a^{f_n} \mid f_0 = f_1 = 1\ \&\ f_{n+2} = f_{n+1} + f_{n}\}$;
  \item
    $L = \{\alpha \in \Sigma^\star \mid \abs{\ \card{\alpha}{a} - \card{\alpha}{b}\ } \leq 2\}$;
  \item
    $L = \{\alpha\beta\alpha \mid \alpha,\beta \in \Sigma^\star\ \&\ \abs{\beta} \leq \abs{\alpha}\}$;
  \item
    $L = \{\beta\gamma\gamma^\rev\mid \beta, \gamma \in \Sigma^\star\ \&\ \abs{\beta} \leq \abs{\gamma}\}$;
  \item
    $L = \{c^ka^nb^m \mid k,m,n > 0\ \&\ n \neq m\}$;
  \item
    $L = \{c^ka^nb^n \mid k > 0\ \&\ n \geq 0\}\cup\{a,b\}^\star$;
  \item
    $L = \{\omega \in \{a,b\}^\star \mid \card{\omega}{a}\text{ не дели }\card{\omega}{b}\}$;
  \item
    $L = \{\omega \in \{a,b\}^\star \mid \card{\omega}{a} < \card{\omega}{b}\}$;
  \item
    $L = \{\omega \in \{a,b\}^\star \mid \card{\omega}{a} = 2\card{\omega}{b}\}$;
  \item
    $L = \{\omega \in \{a,b\}^\star \mid \abs{\ \card{\omega}{a} - \card{\omega}{b}\ } \leq 3\}$.
  \end{enumerate}    
\end{problem}
\end{extra2}

% \restoregeometry

\begin{problem}
  Докажете, че следните езици са регулярни:
  \begin{enumerate}[a)]
  \item
    $L = \{\alpha \in \{a,b\}^\star \mid \text{ за всяка представка $\omega$ на $\alpha$ имаме }\abs{\ \card{\omega}{a} - \card{\omega}{b}\ } \leq 2 \}$;
  \item
    $L = \{\alpha \in \{a,b\}^\star \mid \text{ за някоя представка $\omega$ на $\alpha$ имаме }\abs{\ \card{\omega}{a} - \card{\omega}{b}\ } > 2 \}$;
  \item
    $L = \{\alpha \in \{a,b\}^\star \mid \text{ за някоя наставка $\omega$ на $\alpha$ имаме }\abs{\ \card{\omega}{a} - \card{\omega}{b}\ } > 2 \}$.
  \end{enumerate}
\end{problem}



% \begin{problem}
%   Нека $L$ е регулярен език над азбуката $\Sigma$. Докажете, че 
%   \[\text{Infix}(L) = \{\alpha \mid (\exists \beta,\gamma \in \Sigma^\star)[\beta\alpha\gamma \in L]\}\]
%   също е регулярен език.
% \end{problem}
% \begin{hint}
%   Най-лесно е да се построи автомат за $\text{Infix}(L)$ като се използва автомата за $L$.
% \end{hint}

\begin{problem}
  Нека $\Sigma = \{a,b,c,d\}$.
  Да се докаже, че езикът
  \[L = \{a_1a_2\cdots a_{2n} \in \Sigma^\star \mid (\forall j \in [1,n])[a_{2j-1} = a_{2j}]\ \&\ d\text{ се среща $\leq 3$ пъти}\}\]
  е регулярен.
\end{problem}

\begin{problem}
  Нека $L_1$ и $L_2$ са регулярни езици. Докажете, че $L$ също е регулярен език, където
  \[L = \{\alpha \mid (\exists \beta,\gamma)[\beta\alpha\gamma \in L_1]\ \&\ \alpha \in L_2 \vee \alpha^{\texttt{rev}} \in L_2\}.\]
\end{problem}

\begin{definition}
  \index{хомоморфизъм}
  Да фиксираме две азбуки $\Sigma_1$ и $\Sigma_2$.
  Хомоморфизъм е изображение $h:\Sigma^\star_1 \to \Sigma^\star_2$ със свойството, че
  за всеки две думи $\alpha,\beta\in\Sigma^\star_1$,
  \[h(\alpha\beta) = h(\alpha)\cdot h(\beta).\]
\end{definition}

Лесно се съобразява, че за всеки хомоморфизъм $h$, $h(\varepsilon) = \varepsilon$.

\begin{problem}
  Нека $L \subseteq \Sigma^\star_1$ е регулярен език и $h:\Sigma^\star_1\to\Sigma^\star_2$ е хомоморфизъм.
  Тогава
  $h(L) = \{h(\alpha) \in \Sigma^\star_2 \mid \alpha \in L\}$ е регулярен.
\end{problem}
\begin{hint}
  Индукция по построението на регулярни езици.
  % \begin{itemize}[-]
  % \item 
  %   За $L = \{a\}$, $h(L) = \{h(a)\}$.
  % \item
  %   $h(\emptyset) = \emptyset$.
  % \item
  %   Нека $L_1 = \L(r_1)$ и $L_2 = \L(r_2)$.
  %   Ще докажем, че $h(\L(r_1\cdot r_2))$ е регулярен.
  %   \begin{align*}
  %     h(\L(r_1\cdot r_2)) & = h(L_1\cdot L_2) & (\text{деф. на конкатенация})\\
  %     & = \{h(\gamma) \mid \gamma \in L_1 \cdot L_2\}\\
  %     & = \{h(\alpha\beta) \mid \alpha\in L_1\ \&\ \beta\in L_2\}\\
  %     & = \{h(\alpha)\cdot h(\beta) \mid \alpha \in L_1\ \&\ \beta \in L_2\} & (h\text{ е хомоморфизъм})\\
  %     & = \{\omega\gamma \mid \omega \in h(L_1)\ \&\ \gamma \in h(L_2)\}\\
  %     & = h(L_1)\cdot h(L_2).
  %   \end{align*}
  %   По И.П. имаме, че $h(L_1)$ и $h(L_2)$ са регулярни езици.
  %   Следователно, 
  %   \[h(\L(r_1\cdot r_2)) = h(L_1)\cdot h(L_2)\]
  %   е регулярен език.
  % \item
  %   От горното свойство имаме също, че за всяко $n$, $h(L^n) = h(L)^n$.
  % \item
  %   Освен това, 
  %   \begin{align*}
  %     h(\bigcup_n L_n) & = \{h(\alpha) \mid (\exists n)[\alpha \in L_n]\}\\
  %     & = \bigcup \{h(\alpha) \mid \alpha \in L_n\}\\
  %     & = \bigcup_n h(L_n).
  %   \end{align*}
  % \item
  %   Нека $L = \L(r^\star)$.
  %   Ще докажем, че $h(L^\star)$ е регулярен език.
  %   \begin{align*}
  %     h(L^\star) & = h(\bigcup_n L^n) & (\text{деф. на звезда на Клини})\\
  %     & = \bigcup_n h(L^n) & (\text{от горното свойство})\\
  %     & = \bigcup_n h(L)^n & (\text{от по-горното свойство})\\
  %     & = h(L)^\star & (\text{по деф.}).
  %   \end{align*}
  % \end{itemize}
\end{hint}

\begin{problem}
  Нека $L\subseteq \Sigma^\star_2$ е регулярен език и $h:\Sigma^\star_1\to\Sigma^\star_2$ е хомоморфизъм.
  Тогава езикът
  $h^{-1}(L) = \{\alpha \in \Sigma^\star_1 \mid h(\alpha) \in L\}$ е регулярен.  
\end{problem}
\begin{hint}
  Конструкция на автомат за $h^{-1}(L)$ при даден автомат за $L$.
  % Нека $\A$ е КДА разпознаващ езика $L$.
  % Ще построим $\A' = \pair{Q,\Sigma_1, \delta', s, F}$,
  % където дефинираме функцията на преходите $\delta'$ като $\delta'(q,a) = \delta^\star(q,h(a))$.
  % Понеже $h$ е хомоморфизъм, лесно се доказва с индукция по дължината на думата $\alpha \in \Sigma^\star_1$,
  % че $\delta'^\star(q,\alpha) = \delta^\star(q,h(\alpha))$.
  % Сега лесно се вижда, че $h^{-1}(\L(\A)) = \L(\A')$, защото:
  % \begin{align*}
  %   \alpha \in \L(\A') & \iff \delta'^\star(s,\alpha) \in F\\
  %   & \iff \delta^\star(s,h(\alpha)) \in F\\
  %   & \iff h(\alpha) \in \L(\A)\\
  %   & \iff \alpha \in h^{-1}(\L(\A)).
  % \end{align*}
\end{hint}

\begin{problem}
  Нека $\Sigma_1$ и $\Sigma_2$ са непресичащи се азбуки, а $L_1$ и $L_2$ са езици съответно над $\Sigma_1$ и $\Sigma_2$.
  За една дума $\omega \in (\Sigma_1 \cup \Sigma_2)^\star$, нека с $\omega_i \in \Sigma^\star_i$ да означим редицата от букви от $\Sigma_i$
  в реда, в който се срещат в $\omega$. Да разгледаме следния език
  \[L_1 \oplus L_2 = \{\omega \in (\Sigma_1 \cup \Sigma_2)^\star \mid \omega_1 \in L_1\ \&\ \omega_2 \in L_2\ \&\ |\omega_1| = |\omega_2|\}.\]
  \begin{enumerate}[a)]
  \item
    \marginnote{Да}
    Вярно ли е, че ако $L_1$ е краен, то $L_1 \oplus L_2$ е регулярен език?
  \item
    \marginnote{Не}
    Вярно ли е, че ако $L_1$ и $L_2$ са регулярни езици, то $L_1 \oplus L_2$ е регулярен език?
  \end{enumerate}
\end{problem}

\begin{problem}
  Нека $\Sigma_1$ и $\Sigma_2$ са непресичащи се азбуки, а $L_1$ и $L_2$ са езици съответно над $\Sigma_1$ и $\Sigma_2$.
  За една дума $\omega \in (\Sigma_1 \cup \Sigma_2)^\star$, нека с $\omega_i \in \Sigma^\star_i$ да означим редицата от букви от $\Sigma_i$
  в реда, в който се срещат в $\omega$. Да разгледаме следния език
  \[L_1 \oplus L_2 = \{\omega \in (\Sigma_1 \cup \Sigma_2)^\star \mid \omega_1 \in L_1\ \&\ \omega_2 \in L_2\}.\]
  \marginnote{Да}
  Вярно ли е, че ако $L_1$ и $L_2$ са регулярни езици, то $L_1 \oplus L_2$ е регулярен език?
\end{problem}

%%% Local Variables:
%%% mode: latex
%%% TeX-master: "../eai"
%%% End:

\section{Допълнителни задачи}

\section{Допълнителни задачи}

\subsection{Лесни задачи}

\ifcode
\begin{problem}
  \marginnote{Библиотеката {\bf regex} е част от си++11 стандарта на езика.}
  Да разгледаме следната програма на езика си++
  \begin{minted}[]{cpp}
    #include <iostream>
    #include <regex>
    #include <string>
    
    using namespace std;
    
    int main() {
      string input;
      regex reg("your regular expression");
      while (true) {
        cout << "Input:" << endl;
        cin >> input;
        if (!cin || input=="q") break;
        if (regex_match(input, reg)) {
          cout << "Valid input" << endl;
        }  
        else {
          cout << "Invalid input" << endl;
        }
      }
    }
  \end{minted}
  Разучете как се създават обекти от тип {\bf regex} и попълнете дефиницията на регулярния израз {\bf reg} в горната програма, така че програмата приема за валиден вход:
  \begin{enumerate}[a)]
  \item 
    само реални числа;
  \item
    факултетни номера във ФМИ;
  \item
    низове, които са съставени от поне 8 символа, измежду които се включват малки букви, големи букви,
    и цифри.    
  \end{enumerate}
\end{problem}
\fi

% \begin{problem}
%   Да фиксираме една дума $\alpha$ над дадена азбука $\Sigma$.
%   Опишете алгоритъм, който за вход произволен текстов файл, чието съдържание означаваме с $\beta$,
%   отговаря дали думата $\alpha$ се среща в $\beta$.
%   Каква е сложността на този алгоритъм относно дължините на $\alpha$ и $\beta$ ?
% \end{problem}

% \begin{problem}
%   Опишете алгоритъм, който при вход два регулярни израза $\mathbf{r}$ и $\mathbf{s}$,
%   проверява дали $\L(\mathbf{r}) \subseteq \L(\mathbf{s})$.
% \end{problem}

% \setlength{\marginparsep}{18pt}
% \setlength{\oddsidemargin}{5pt}
% \setlength{\evensidemargin}{100pt}
% \setlength{\hoffset}{-30pt}
% \setlength{\voffset}{-30pt}

% \newgeometry{marginparwidth=20pt, evensidemargin=20pt}
% \newgeometry{textwidth=600pt}

% \layout

\ExtraMaterial{
\begin{problem}
  За всеки от следните езици $L$, постройте минимален краен детерминиран автомат $\A$, който разпознава езика $L$, където:
  \marginnote{$\card{\omega}{a} \df $ броят на срещанията на буквата $a$ в думата $\omega$, $\abs{\omega} \df $ дължината на $\omega$.}
  \begin{enumerate}[a)]
  \item 
    $L = \{a^nb\mid n \geq 0\}$;
  \item
    $L = \{a,b\}^\star\setminus\{\varepsilon\}$;
  \item
    $L = \{\omega \in \{a,b\}^\star \mid \card{\omega}{a}\text{ и }\card{\omega}{b}\text{ са четни}\}$;
  \item
    $L = \{\omega \in \{a,b\}^\star \mid \card{\omega}{a}\text{ е четно}\ \&\ \card{\omega}{b}\text{ е нечетно}\}$;
  \item
    $L = \{a^nb^m\mid n,m \geq 0\}$;
  \item
    $L = \{a^nb^m\mid n,m \geq 1\}$;
  \item
    $L = \{a,b\}^\star \setminus \{a\}$;
  \item
    $L = \{\omega \in \{a,b\}^\star \mid \card{\omega}{a} \geq 2\ \lor\ \card{\omega}{b} \leq 3 \}$;
  \item
    $L = \{\omega \in \{a,b\}^\star \mid \card{\omega}{a} \geq 2\ \&\ \card{\omega}{b} \geq 1\}$;
  \item
    $L = \{\omega \in \{a,b\}^\star \mid \text{на всяка нечетна позиция на }\omega\text{ е буквата }a\}$;
  \item
    $L = \{\omega \in \{a,b\}^\star \mid \card{\omega}{a}\text{ е четно }\&\ \card{\omega}{b} \leq 1\}$;
  \item
    $L = \{\omega \in \{a,b\}^\star \mid \abs{\omega} \leq 3\}$;
  \item
    $L = \{\omega \in \{a,b\}^\star \mid \omega \text{ не започва с }ab\}$;
  \item
    $L = \{\omega \in \{a,b\}^\star \mid \omega \text{ завършва с }ab\text{ или ba}\}$;
  \item
    $L = \{\omega \in \{a,b\}^\star \mid \omega\text{ започва или завършва с } a\}$;
  \item
    $L = \{\omega \in \{a,b\}^\star \mid \omega\text{ започва с }a \iff \omega\text{ завършва с }b\}$;
  \item
    $L = \{\omega \in \{a,b\}^\star \mid \abs{\omega} \equiv 0\ (\bmod\ 2)\ \&\ \card{\omega}{a} = 1\}$;
  \item
    $L = \{\omega \in \{a,b\}^\star \mid \text{ всяко }a\text{ в }\omega\text{ се следва веднага от поне едно }b\}$;
  \item
    $L = \{\omega \in \{a,b\}^\star \mid \abs{\omega} \equiv 0 \bmod 3\}$;
  \item
    $L = \{\omega \in \{a,b\}^\star \mid \card{\omega}{a} \equiv 1 \bmod 3\}$;
  \item
    $L = \{\omega \in \{a,b\}^\star \mid \card{\omega}{a} \equiv 0 \bmod 3\ \&\ \card{\omega}{b} \equiv 1 \bmod 2\}$;
  \item
    $L = \{\omega \in \{a,b\}^\star \mid \card{\omega}{a} \equiv 0 \bmod 2\ \vee\ \card{\omega}{b} = 2\}$;
  \item
    $L = \{\omega \in \{a,b\}^\star \mid \omega \text{ съдържа равен брой срещания на }ab\text{ и на }ba\}$;
  \item
    $L = \{\omega_1 \sharp \omega_2 \sharp \omega_3 \mid |\omega_1| \geq 2\ \&\ |\omega_2| \geq 3\ \&\ |\omega_3| \geq 4\ \&\ \omega_i \in \{a,b\}^\star\text{
      за }i = 1,2,3\}$.
  \end{enumerate}
\end{problem}
}

\begin{extra2}
\begin{problem}
  Нека $\Sigma = \{a,b\}$.  Проверете дали $L$ е регулярен, където
  % \begin{multicols}{2}
  \begin{enumerate}[a)]
  % \item
  %   $L = \{\alpha^{\texttt{rev}} \mid \alpha \in L_0\}$, където $L_0$ е регулярен;
  \item
    % \marginnote{$\alpha = a^pb^p$}
    $L = \{a^ib^i\ \mid\ i\in\Nat\}$;
  \item
    $L = \{a^ib^i\ \mid\ i,j\in\Nat\ \&\ i\neq j\}$;
  \item
    % \marginnote{$\alpha = a^{p+1}b^p$.}
    $L = \{a^ib^j\ \mid\ i > j\}$;
  \item
    $L = \{a^nb^m \mid n\text{ дели }m\}$.
  \item
    $L = \{a^{2n}\ \mid\ n\geq 1\}$;
  \item
    $L = \{a^mb^na^{m+n}\ \mid\ m\geq 1\ \&\ n\geq 1\}$;
  \item
    $L = \{a^{n.m}\mid n,m\text{ са прости числа}\}$;
  \item
    $L = \{\omega\in\{a,b\}^\star \mid \card{\omega}{a} = \card{\omega}{b}\}$;
  \item
    % \marginnote{$\alpha = a^pba^pb$}
    $L = \{\omega\omega\mid \omega\in\{a,b\}^\star\}$;
  \item
    $L = \{\omega\omega^\rev\mid \omega\in\{a,b\}^\star\}$;
  \item
    $L = \{\alpha\beta\beta \in \{a,b\}^\star\mid \beta \neq \varepsilon\}$;
  \item
    $L = \{a^nb^nc^n\mid n\geq 0\}$;
  \item
    $L = \{\omega\omega\omega\mid \omega\in \Sigma^\star\}$;
  \item
    $L = \{a^{2^n}\mid n\geq 0\}$;
  \item
    $L = \{a^mb^n\mid n\neq m\}$;
  \item
    $L = \{a^{n!}b^{n!}\mid n\neq 1\}$;
  \item
    $L = \{a^{f_n} \mid f_0 = f_1 = 1\ \&\ f_{n+2} = f_{n+1} + f_{n}\}$;
  \item
    $L = \{\alpha \in \Sigma^\star \mid \abs{\ \card{\alpha}{a} - \card{\alpha}{b}\ } \leq 2\}$;
  \item
    $L = \{\alpha\beta\alpha \mid \alpha,\beta \in \Sigma^\star\ \&\ \abs{\beta} \leq \abs{\alpha}\}$;
  \item
    $L = \{\beta\gamma\gamma^\rev\mid \beta, \gamma \in \Sigma^\star\ \&\ \abs{\beta} \leq \abs{\gamma}\}$;
  \item
    $L = \{c^ka^nb^m \mid k,m,n > 0\ \&\ n \neq m\}$;
  \item
    $L = \{c^ka^nb^n \mid k > 0\ \&\ n \geq 0\}\cup\{a,b\}^\star$;
  \item
    $L = \{\omega \in \{a,b\}^\star \mid \card{\omega}{a}\text{ не дели }\card{\omega}{b}\}$;
  \item
    $L = \{\omega \in \{a,b\}^\star \mid \card{\omega}{a} < \card{\omega}{b}\}$;
  \item
    $L = \{\omega \in \{a,b\}^\star \mid \card{\omega}{a} = 2\card{\omega}{b}\}$;
  \item
    $L = \{\omega \in \{a,b\}^\star \mid \abs{\ \card{\omega}{a} - \card{\omega}{b}\ } \leq 3\}$.
  \end{enumerate}    
\end{problem}
\end{extra2}

% \restoregeometry

\begin{problem}
  Докажете, че следните езици са регулярни:
  \begin{enumerate}[a)]
  \item
    $L = \{\alpha \in \{a,b\}^\star \mid \text{ за всяка представка $\omega$ на $\alpha$ имаме }\abs{\ \card{\omega}{a} - \card{\omega}{b}\ } \leq 2 \}$;
  \item
    $L = \{\alpha \in \{a,b\}^\star \mid \text{ за някоя представка $\omega$ на $\alpha$ имаме }\abs{\ \card{\omega}{a} - \card{\omega}{b}\ } > 2 \}$;
  \item
    $L = \{\alpha \in \{a,b\}^\star \mid \text{ за някоя наставка $\omega$ на $\alpha$ имаме }\abs{\ \card{\omega}{a} - \card{\omega}{b}\ } > 2 \}$.
  \end{enumerate}
\end{problem}



% \begin{problem}
%   Нека $L$ е регулярен език над азбуката $\Sigma$. Докажете, че 
%   \[\text{Infix}(L) = \{\alpha \mid (\exists \beta,\gamma \in \Sigma^\star)[\beta\alpha\gamma \in L]\}\]
%   също е регулярен език.
% \end{problem}
% \begin{hint}
%   Най-лесно е да се построи автомат за $\text{Infix}(L)$ като се използва автомата за $L$.
% \end{hint}

\begin{problem}
  Нека $\Sigma = \{a,b,c,d\}$.
  Да се докаже, че езикът
  \[L = \{a_1a_2\cdots a_{2n} \in \Sigma^\star \mid (\forall j \in [1,n])[a_{2j-1} = a_{2j}]\ \&\ d\text{ се среща $\leq 3$ пъти}\}\]
  е регулярен.
\end{problem}

\begin{problem}
  Нека $L_1$ и $L_2$ са регулярни езици. Докажете, че $L$ също е регулярен език, където
  \[L = \{\alpha \mid (\exists \beta,\gamma)[\beta\alpha\gamma \in L_1]\ \&\ \alpha \in L_2 \vee \alpha^{\texttt{rev}} \in L_2\}.\]
\end{problem}

\begin{definition}
  \index{хомоморфизъм}
  Да фиксираме две азбуки $\Sigma_1$ и $\Sigma_2$.
  Хомоморфизъм е изображение $h:\Sigma^\star_1 \to \Sigma^\star_2$ със свойството, че
  за всеки две думи $\alpha,\beta\in\Sigma^\star_1$,
  \[h(\alpha\beta) = h(\alpha)\cdot h(\beta).\]
\end{definition}

Лесно се съобразява, че за всеки хомоморфизъм $h$, $h(\varepsilon) = \varepsilon$.

\begin{problem}
  Нека $L \subseteq \Sigma^\star_1$ е регулярен език и $h:\Sigma^\star_1\to\Sigma^\star_2$ е хомоморфизъм.
  Тогава
  $h(L) = \{h(\alpha) \in \Sigma^\star_2 \mid \alpha \in L\}$ е регулярен.
\end{problem}
\begin{hint}
  Индукция по построението на регулярни езици.
  % \begin{itemize}[-]
  % \item 
  %   За $L = \{a\}$, $h(L) = \{h(a)\}$.
  % \item
  %   $h(\emptyset) = \emptyset$.
  % \item
  %   Нека $L_1 = \L(r_1)$ и $L_2 = \L(r_2)$.
  %   Ще докажем, че $h(\L(r_1\cdot r_2))$ е регулярен.
  %   \begin{align*}
  %     h(\L(r_1\cdot r_2)) & = h(L_1\cdot L_2) & (\text{деф. на конкатенация})\\
  %     & = \{h(\gamma) \mid \gamma \in L_1 \cdot L_2\}\\
  %     & = \{h(\alpha\beta) \mid \alpha\in L_1\ \&\ \beta\in L_2\}\\
  %     & = \{h(\alpha)\cdot h(\beta) \mid \alpha \in L_1\ \&\ \beta \in L_2\} & (h\text{ е хомоморфизъм})\\
  %     & = \{\omega\gamma \mid \omega \in h(L_1)\ \&\ \gamma \in h(L_2)\}\\
  %     & = h(L_1)\cdot h(L_2).
  %   \end{align*}
  %   По И.П. имаме, че $h(L_1)$ и $h(L_2)$ са регулярни езици.
  %   Следователно, 
  %   \[h(\L(r_1\cdot r_2)) = h(L_1)\cdot h(L_2)\]
  %   е регулярен език.
  % \item
  %   От горното свойство имаме също, че за всяко $n$, $h(L^n) = h(L)^n$.
  % \item
  %   Освен това, 
  %   \begin{align*}
  %     h(\bigcup_n L_n) & = \{h(\alpha) \mid (\exists n)[\alpha \in L_n]\}\\
  %     & = \bigcup \{h(\alpha) \mid \alpha \in L_n\}\\
  %     & = \bigcup_n h(L_n).
  %   \end{align*}
  % \item
  %   Нека $L = \L(r^\star)$.
  %   Ще докажем, че $h(L^\star)$ е регулярен език.
  %   \begin{align*}
  %     h(L^\star) & = h(\bigcup_n L^n) & (\text{деф. на звезда на Клини})\\
  %     & = \bigcup_n h(L^n) & (\text{от горното свойство})\\
  %     & = \bigcup_n h(L)^n & (\text{от по-горното свойство})\\
  %     & = h(L)^\star & (\text{по деф.}).
  %   \end{align*}
  % \end{itemize}
\end{hint}

\begin{problem}
  Нека $L\subseteq \Sigma^\star_2$ е регулярен език и $h:\Sigma^\star_1\to\Sigma^\star_2$ е хомоморфизъм.
  Тогава езикът
  $h^{-1}(L) = \{\alpha \in \Sigma^\star_1 \mid h(\alpha) \in L\}$ е регулярен.  
\end{problem}
\begin{hint}
  Конструкция на автомат за $h^{-1}(L)$ при даден автомат за $L$.
  % Нека $\A$ е КДА разпознаващ езика $L$.
  % Ще построим $\A' = \pair{Q,\Sigma_1, \delta', s, F}$,
  % където дефинираме функцията на преходите $\delta'$ като $\delta'(q,a) = \delta^\star(q,h(a))$.
  % Понеже $h$ е хомоморфизъм, лесно се доказва с индукция по дължината на думата $\alpha \in \Sigma^\star_1$,
  % че $\delta'^\star(q,\alpha) = \delta^\star(q,h(\alpha))$.
  % Сега лесно се вижда, че $h^{-1}(\L(\A)) = \L(\A')$, защото:
  % \begin{align*}
  %   \alpha \in \L(\A') & \iff \delta'^\star(s,\alpha) \in F\\
  %   & \iff \delta^\star(s,h(\alpha)) \in F\\
  %   & \iff h(\alpha) \in \L(\A)\\
  %   & \iff \alpha \in h^{-1}(\L(\A)).
  % \end{align*}
\end{hint}

\begin{problem}
  Нека $\Sigma_1$ и $\Sigma_2$ са непресичащи се азбуки, а $L_1$ и $L_2$ са езици съответно над $\Sigma_1$ и $\Sigma_2$.
  За една дума $\omega \in (\Sigma_1 \cup \Sigma_2)^\star$, нека с $\omega_i \in \Sigma^\star_i$ да означим редицата от букви от $\Sigma_i$
  в реда, в който се срещат в $\omega$. Да разгледаме следния език
  \[L_1 \oplus L_2 = \{\omega \in (\Sigma_1 \cup \Sigma_2)^\star \mid \omega_1 \in L_1\ \&\ \omega_2 \in L_2\ \&\ |\omega_1| = |\omega_2|\}.\]
  \begin{enumerate}[a)]
  \item
    \marginnote{Да}
    Вярно ли е, че ако $L_1$ е краен, то $L_1 \oplus L_2$ е регулярен език?
  \item
    \marginnote{Не}
    Вярно ли е, че ако $L_1$ и $L_2$ са регулярни езици, то $L_1 \oplus L_2$ е регулярен език?
  \end{enumerate}
\end{problem}

\begin{problem}
  Нека $\Sigma_1$ и $\Sigma_2$ са непресичащи се азбуки, а $L_1$ и $L_2$ са езици съответно над $\Sigma_1$ и $\Sigma_2$.
  За една дума $\omega \in (\Sigma_1 \cup \Sigma_2)^\star$, нека с $\omega_i \in \Sigma^\star_i$ да означим редицата от букви от $\Sigma_i$
  в реда, в който се срещат в $\omega$. Да разгледаме следния език
  \[L_1 \oplus L_2 = \{\omega \in (\Sigma_1 \cup \Sigma_2)^\star \mid \omega_1 \in L_1\ \&\ \omega_2 \in L_2\}.\]
  \marginnote{Да}
  Вярно ли е, че ако $L_1$ и $L_2$ са регулярни езици, то $L_1 \oplus L_2$ е регулярен език?
\end{problem}

%%% Local Variables:
%%% mode: latex
%%% TeX-master: "../eai"
%%% End:



\subsection{Не толкова лесни задачи}

{\bf Това вече е добре да се махне}
\begin{problem}
  Докажете, че няма полиномиален алгоритъм за детерминизация на краен недетерминиран автомат.
\end{problem}
\begin{hint}
  За произволно $n$, разгледайте недетерминирания автомат $\A_n$, за който
  \[(\forall \alpha,\beta \in \{0,1\}^\star)[|\alpha| = |\beta| = n \implies (\alpha\beta \in \L(\A_n) \iff \alpha \neq \beta)].\]
  Този автомат ще има $2n+2$ състояния.

  Допуснете, че за него съществува детерминиран автомат $\D_n$ с $< 2^n$ на брой състояния.
  Разгледайте всички думи с дължина $n$, $\omega_1,\omega_2,\dots,\omega_{2^n}$.
  Приложете принципа на Дирихле и достигнете до противоречие.
\end{hint}


\begin{problem}
  \marginpar{\cite[стр. 84]{papadimitriou}}
  При дадени езици $L$, $L'$ над азбуката $\Sigma$, да разгледаме:
  \begin{enumerate}[a)]
  \item
    $\texttt{Pref}(L) = \{\alpha \in \Sigma^\star \mid (\exists \beta \in \Sigma^\star)[\alpha\beta \in L]\}$;
  \item
    $\texttt{NoPref}(L) = \{\alpha \in L \mid \text{ не съществува префикс на $\alpha$ в $L$}\}$;
  \item
    $\texttt{NoExtend}(L) = \{\alpha \in L \mid \text{ $\alpha$ не е префикс на никоя дума от $L$}\}$;
  \item
    $\mbox{Suf}(L) = \{\beta \in \Sigma^\star \mid (\exists \alpha \in \Sigma^\star)[\alpha\beta \in L]\}$;
  \item
    $\text{Infix}(L) = \{\alpha \mid (\exists \beta,\gamma \in \Sigma^\star)[\beta\alpha\gamma \in L]\}$;
  \item 
    $\frac{1}{2}(L) = \{\omega \in \Sigma^\star \mid (\exists \alpha \in \Sigma^\star)[\omega\alpha \in L\ \&\ \abs{\omega} = \abs{\alpha}]\}$;
  \item
    \marginpar{right quotient of $L$ by $L'$}
    $L/L' = \{\alpha \in \Sigma^\star \mid (\exists \beta \in L')[\alpha\beta \in L ] \}$;
  \item
    $L^{-1}(L') = \{ \beta \mid (\exists \alpha \in L)[ \alpha\beta \in L']\}$;
  \item
    $\mbox{Max}(L) = \{\alpha \in \Sigma^\star \mid (\forall \beta\in\Sigma^\star)[\beta \neq \varepsilon\implies \alpha\beta \not\in L]\}$.
  \end{enumerate}
  За всички тези езици, докажете, че са регулярни при условие, че $L$ и $L'$ са регулярни.
  \marginpar{Тази конструкция няма да бъде ефективна}
  Освен това, докажете, че $L/L'$ е регулярен и при условието, че $L$ е регулярен, но $L'$ е произволен език над азбуката $\Sigma$.
\end{problem}
\begin{hint}
  \begin{enumerate}[a)]
  \item 
    Индукция по дефиницията на регулярен израз.
  \item[в)]
    Най-лесно е да се построи автомат за $\text{Infix}(L)$ като се използва автомата за $L$.
  \item[г)]
    Конструкция с автомат за $L$ и автомат за $L^{\texttt{rev}}$.
  \end{enumerate}
\end{hint}

\begin{problem}
  \marginpar{\cite[стр. 75]{kozen}; \cite[стр. 89]{papadimitriou}}
  Да фиксираме азбука само с един символ $\Sigma = \{a\}$.
  Да положим за всяко $p,q\in\Nat$, 
  \[L(p,q) = \{a^k \mid (\exists n\in\Nat)[k = p+q\cdot n]\}.\]
  Ако за един език $L$ съществуват константи $p_1,\dots,p_k$ и $q_1,\dots,q_k$, такива че 
  \[L = \bigcup_{1\leq i \leq k} L(p_i,q_i),\]
  то казваме, че $L$ е {\em породен от аритметични прогресии}.
  \begin{enumerate}[a)]
  \item 
    Докажете, че $L \subseteq \{a\}^\star$ е регулярен език точно тогава, когато $L$ е породен от аритметична прогресия.
  \item
    За произволна азбука $\Sigma$, докажете, че ако $L \subseteq \Sigma^\star$ е регулярен език,
    то езикът $\{a^{\abs{\omega}} \mid \omega \in L\}$  е породен от аритметична прогресия.    
  \end{enumerate}
\end{problem}
\begin{hint}
  \begin{enumerate}[a)]
  \item 
    За едната посока, разгледайте ДКА за $L$.
  \item
    За втората част, разгледайте $h:\Sigma\to\{a\}$ деф. като $(\forall b\in\Sigma)[h(b) = a]$.
    Докажете, че $h$ е поражда хомоморфизъм между $\Sigma^\star$ и $\{a\}^\star$.
    Тогава $h(L) = \{a^{\abs{\omega}} \mid \omega \in L\}$, а
    ние знаем, че регулярните езици са затворени относно хомоморфни образи.  
  \end{enumerate}
\end{hint}

\begin{problem}
  Вярно ли е, че:
  \begin{itemize}
  \item 
    $\{a^m \mid a^{m^2} \in L(p,q)\}$ е регулярен език ?
  \item
    $\{a^m \mid a^{2^m} \in L(p,q)\}$ е регулярен език ?
  \end{itemize}
\end{problem}


\begin{problem}
  За даден език $L$ над азбуката $\Sigma$, да разгледаме езиците:
  \begin{enumerate}[a)]
  \item
    $L' = \{\alpha \in \Sigma^\star \mid (\exists \beta\in\Sigma^\star)[\abs{\alpha} = 2\abs{\beta}\ \&\ \alpha\beta \in L]\}$;
  \item 
    $L'' = \{\alpha \in \Sigma^\star \mid (\exists \beta\in\Sigma^\star)[2\abs{\alpha} = \abs{\beta}\ \&\ \alpha\beta \in L]\}$;
  \item 
    $\frac{1}{3}(L) = \{\alpha \in \Sigma^\star \mid (\exists \beta,\gamma \in \Sigma^\star)[\abs{\alpha} = \abs{\beta} = \abs{\gamma}\ \&\ \alpha\beta\gamma \in L]\}$;
  \item
    $\frac{2}{3}(L) = \{\beta \in \Sigma^\star \mid (\exists \beta,\gamma \in \Sigma^\star)[\abs{\alpha} = \abs{\beta} = \abs{\gamma}\ \&\ \alpha\beta\gamma \in L]\}$;
  \item
    $\frac{3}{3}(L) = \{\gamma \in \Sigma^\star \mid (\exists \beta,\gamma \in \Sigma^\star)[\abs{\alpha} = \abs{\beta} = \abs{\gamma}\ \&\ \alpha\beta\gamma \in L]\}$;
  \item
    $\hat{L} = \{\alpha\gamma \in \Sigma^\star \mid (\exists \beta,\gamma \in \Sigma^\star)[\abs{\alpha} = \abs{\beta} = \abs{\gamma}\ \&\ \alpha\beta\gamma \in L]\}$;
  \item
    $\sqrt{L} = \{\alpha \mid (\exists \beta \in \Sigma^\star)[\abs{\beta} = \abs{\alpha}^2\ \&\ \alpha\beta \in L]\}$;
  \item
    $\log(L) = \{\alpha \mid (\exists \beta \in \Sigma^\star)[\abs{\beta} = 2^{\abs{\alpha}}\ \&\ \alpha\beta \in L]\}$;
  \end{enumerate}
  Проверете ако $L$ е регулярен, то кои от горните езици също са регулярни.
\end{problem}

\begin{problem}
  Да разгледаме езика
  \[L = \{\omega \in \{0,1\}^\star \mid \omega\text{ съдържа равен брой поднизове }01\text{ и }10\}.\]
  Например, $101 \in L$, защото съдържа по веднъж $10$ и $01$.
  $1010 \not\in  L$, защото съдържа два пъти $10$ и само веднъж $01$.
  Докажете, че $L$ е регулярен.
\end{problem}

\begin{problem}
  Нека $L$ е регулярен език над азбуката $\{a,b\}$. Докажете, че следните езици са регулярни:
  \begin{enumerate}[a)]
  \item 
    $\texttt{Diff}_1(L) \df \{\alpha \in L \mid (\exists \beta \in L)[|\alpha| = |\beta|\ \&\ \alpha \text{ се различава от $\beta$ в една позиция}]\}$;
  \item
    $\texttt{Diff}_n(L) \df \{\alpha \in L \mid (\exists \beta \in L)[n \leq |\alpha| = |\beta|\ \&\ \alpha \text{ се различава от $\beta$ в $n$ позиции}]\}$;
  \item
    $\texttt{Diff}(L) \df \{\alpha \in L \mid (\exists \beta \in L)[|\alpha| = |\beta|\ \&\ \alpha \text{ се различава от $\beta$ във всяка  позиция}]\}$;
  \end{enumerate}
\end{problem}
\begin{hint}
  Ако $L = \L(\A)$, то правим декартово произведение на $\A$ плюс флаг дали сме направили грешка.

  Не е ли по-лесно с индукция по построението на регулярните езици ?
\end{hint}


% \begin{problem}
%   \marginpar{(\cite{kozen}, стр. 75)}
%   Да фиксираме азбука само с един символ $\Sigma = \{a\}$.
%   Множеството $U$ е {\em породен от аритметична прогресия}, ако съществуват числа $q \geq 0$ и $p > 0$,
%   такива че $(\forall n \geq q)[n \in U\ \iff\ n+p \in U]$.
%   Докажете, че $L \subseteq \{a\}^\star$ е регулярен точно тогава, когато множеството $\{m \mid a^m \in L\}$
%   е породено от аритметична прогресия.
% \end{problem}
% \begin{hint}
%   Разгледайте КДА за $L$.
% \end{hint}

% \begin{hint}
%   \begin{itemize}
%   \item 
%     Докажете, че за всяко $p,q \in \Nat$, $L(p,q)$ е регулярен език.
%   \item
%     Докажете, че за крайно много $p_0,\dots,p_k$, $q_0,\dots,q_k$,
%     $\bigcup_{i \leq k}L(p_i,q_i)$ е регулярен език.
%   \item
%     С индукция по построението на регулярните езици, 
%     докажете, че ако $L$ е регулярен, то $L$ може да се представи
%     като крайно обединение на езици породени от аритметични прогресии.
%     Съществената част от доказателството се състои в следното:
%     \begin{itemize}
%     \item 
%       \marginpar{$L(p_1,q_1)\cdot L(p_2,q_2) = L(p_1+p_2,\mbox{НОД}(q_1,q_2))\setminus F$, където $F$ е крайно м-во, и ако $q_1 = q_2$, то $F = \emptyset$}
%       езикът $L(p_1,q_1) \cdot L(p_2,q_2)$ може да се представи като крайно обединение 
%       на езици породени от артиметични прогресии.
%     \item
%       езикът $L(p,q)^\star$ може да се представи като крайно обединение 
%       на езици породени от артиметични прогресии.
%     \end{itemize}
%   \end{itemize}
% \end{hint}



\begin{problem}
  \marginpar{Да обърнем внимание, че езикът $L = \{a^nb^mc^k \mid n+m = k\}$ не е регулярен.}
  Да разгледаме азбуката:
  \[\Sigma_3 = \left\{\begin{bmatrix} 0\\0\\0\end{bmatrix},\begin{bmatrix} 0\\0\\1\end{bmatrix},\begin{bmatrix} 0\\1\\0\end{bmatrix},\begin{bmatrix} 0\\1\\1\end{bmatrix},\dots,\begin{bmatrix} 1\\1\\1\end{bmatrix}\right\}.\]
  Докажете, че 
  $L = \left\{\begin{bmatrix} \alpha\\\beta\\\gamma\end{bmatrix} \in \Sigma^\star_3 \mid \bin{\alpha}+\bin{\beta} = \bin{\gamma}\right\}$
  е автоматен език.
\end{problem}
\ifhints
\begin{hint}
  Доста по-удобно е да построим автомат $\A$, такъв че $\L(\A) = L^{\texttt{rev}}$.
  Да започнем с състоянието $q_{\scriptscriptstyle{=}}$, за което искаме да имаме свойството, че за произволно състояние $q$,
  \[\delta^\star(q, \tiny{ \begin{bmatrix} \alpha\\ \beta \\ \gamma\end{bmatrix} }) = q_{\scriptscriptstyle{=}}  \iff \bin{\alpha^{\rev}} + \bin{\beta^{\rev}} = \bin{\gamma^{\rev}}.\]
  Понеже за $\bin{\varepsilon} + \bin{\varepsilon} = \bin{\varepsilon}$, състоянието $q_{\scriptscriptstyle{=}}$ ще бъде начално и финално за $\A$.

  Нека $\bin{\alpha}+\bin{\beta} = \bin{\gamma}$. Тогава:
  \begin{align*}
    & \delta(q_{\scriptscriptstyle{=}},\tiny{ \begin{bmatrix} 0\\ 0 \\ 0\end{bmatrix} }) \df q_{\scriptscriptstyle{=}} & \comment\text{ защото }\bin{0\alpha} + \bin{0\beta} = \bin{0\gamma}\\
    & \delta(q_{\scriptscriptstyle{=}},\tiny{ \begin{bmatrix} 0\\ 1 \\ 1\end{bmatrix} }) \df q_{\scriptscriptstyle{=}} & \comment\text{ защото }\bin{0\alpha} + \bin{1\beta} = \bin{1\gamma}\\
    & \delta(q_{\scriptscriptstyle{=}},\tiny{ \begin{bmatrix} 1\\ 0 \\ 1\end{bmatrix} }) = q_{\scriptscriptstyle{=}} & \comment\text{ защото }\bin{1\alpha} + \bin{0\beta} = \bin{1\gamma}
  \end{align*}
  Остана случая $\bin{1\alpha} + \bin{1\beta} = \bin{10\gamma}$. Този случай е по-специален и трябва да бъде разгледан отделно.
  Трябва да отидем в състояние $q_1$, в което ще помним, че третия ред трябва да започва с $1$-ца. Затова имаме следния преход:
  \[\delta(q_{\scriptscriptstyle{=}},\tiny{ \begin{bmatrix} 1\\ 1 \\ 0\end{bmatrix} }) \df q_1.\]
  
  За останалите $\gamma \in \Sigma_3$ имаме, че
  \[\delta(q_{\scriptscriptstyle{=}},\gamma) \df q_{\texttt{err}},\]
  където $q_{\texttt{err}}$ е състоянието, от което не можем да излезем.
  
  Така трябва да дефинираме функцията на преходите, че за състоянието $q_1$ трябва да е изпълнено, че за произволно $q$,
  \[\delta^\star(q, \tiny{ \begin{bmatrix} \alpha\\ \beta \\ \gamma\end{bmatrix} }) = q_{\scriptscriptstyle{1}}  \iff \bin{\alpha^{\rev}} + \bin{\beta^{\rev}} = \bin{1\gamma^{\rev}}.\]
  Да разгледаме сега случая $\bin{\alpha} + \bin{\beta} = \bin{1\gamma}$. Тогава:
  \begin{align*}
    & \delta(q_1,\tiny{ \begin{bmatrix} 0\\ 0 \\ 1\end{bmatrix} }) \df q_{\scriptscriptstyle{=}} & \comment\text{ защото }\bin{0\alpha} + \bin{0\beta} = \bin{1\gamma}\\
    & \delta(q_1,\tiny{ \begin{bmatrix} 1\\ 1 \\ 1\end{bmatrix} }) \df q_{1} & \comment\text{ защото }\bin{1\alpha} + \bin{1\beta} = \bin{11\gamma}\\
    & \delta(q_1,\tiny{ \begin{bmatrix} 1\\ 0 \\ 0\end{bmatrix} }) \df q_{1} & \comment\text{ защото }\bin{1\alpha} + \bin{0\beta} = \bin{10\gamma}\\
    & \delta(q_1,\tiny{ \begin{bmatrix} 0\\ 1 \\ 0\end{bmatrix} }) \df q_{1} & \comment\text{ защото }\bin{0\alpha} + \bin{1\beta} = \bin{10\gamma}\\
    & \delta(q_1, \gamma) \df q_{\texttt{err}} & \comment\text{ за останалите }\gamma \in \Sigma_3
  \end{align*}
\end{hint}
\fi

\begin{problem}
  Да разгледаме азбуката:
  \[\Sigma_2 = \left\{\begin{bmatrix} 0\\0\end{bmatrix},\begin{bmatrix} 0\\1\end{bmatrix},\begin{bmatrix} 1\\0\end{bmatrix},\begin{bmatrix} 1\\1\end{bmatrix}\right\}.\]
  Една дума над азбуката $\Sigma_2$ ни дава два реда от $0$-ли и $1$-ци, които ще разглеждаме като числа в двоична бройна система.
  Да разгледаме езиците:
  \begin{itemize}
  \item 
    $L_1 = \left\{\begin{bmatrix} \alpha\\ \beta \end{bmatrix} \in \Sigma^\star_2 \mid \ov{\alpha}_{(2)} < \ov{\beta}_{(2)}\right\}$;
  \item
    $L_2 = \left\{\begin{bmatrix} \alpha\\ \beta \end{bmatrix} \in \Sigma^\star_2 \mid 3(\ov{\alpha}_{(2)}) = \ov{\beta}_{(2)}\right\}$;
  \item
    $L_3 = \left\{\begin{bmatrix} \alpha\\ \beta \end{bmatrix} \in \Sigma^\star_2 \mid \alpha = \beta^{\rev}\right\}$;
  \end{itemize}
  Докажете, че  $L_1$ и $L_2$ са автоматни, а $L_3$ не е автоматен.
\end{problem}
\ifhints
\begin{hint}
  Ще построим автомат $\A = \FA$ за езика $L^{\rev}_1$.
  За улеснение, в рамките на тази задача ще пишем:
  \begin{itemize}
  \item 
    $\alpha \equiv \beta$, ако $\ov{\alpha^{\rev}}_{(2)} = \ov{\beta^{\rev}}_{(2)}$,
  \item
    $\alpha \prec \beta$, ако $\ov{\alpha^{\rev}}_{(2)} < \ov{\beta^{\rev}}_{(2)}$,
  \item
    $\alpha \succ \beta$, ако $\ov{\alpha^{\rev}}_{(2)} > \ov{\beta^{\rev}}_{(2)}$.
  \end{itemize}
  Нека състоянията на автомата са $Q = \{q_{\scriptscriptstyle{=}},q_{\scriptscriptstyle{<}},q_{\scriptscriptstyle{>}}\}$.
  Искаме да е изпълнено свойствата:
  \begin{itemize}
  \item 
    $\delta^\star(q_{\scriptscriptstyle{=}}, \scriptsize{\begin{bmatrix} \alpha\\ \beta\end{bmatrix}}) = q_{\scriptscriptstyle{=}}$ точно тогава, когато $\alpha \equiv \beta$;
  \item 
    $\delta^\star(q_{\scriptscriptstyle{=}}, \scriptsize{\begin{bmatrix} \alpha\\ \beta\end{bmatrix}}) = q_{\scriptscriptstyle{<}}$ точно тогава, когато $\alpha \prec \beta$;
  \item 
    $\delta^\star(q_{\scriptscriptstyle{=}}, \scriptsize{\begin{bmatrix} \alpha\\ \beta\end{bmatrix}}) = q_{\scriptscriptstyle{>}}$ точно тогава, когато $\alpha \succ \beta$.
  \end{itemize}
  Множеството от финални състояния ще бъде $F = \{q_{\scriptscriptstyle{<}}\}$, а началното състояние $\qstart = q_{\scriptscriptstyle{=}}$.
  За да дефинираме функцията на преходите, трябва да разгледа няколко случая, в зависимост от това какво е отношението между $\alpha$ и $\beta$.
  \begin{itemize}
  \item
    Нека $\alpha \equiv \beta$. Тогава:  
    \begin{itemize}
    \item 
      $\alpha 0 \equiv \beta 0$ и $\alpha 1 \equiv \beta 1$. Следователно,
      \[\delta(q_{\scriptscriptstyle{=}},\scriptsize{\begin{bmatrix} 0\\0\end{bmatrix}}) = \delta(q_{\scriptscriptstyle{=}},\scriptsize{\begin{bmatrix} 1\\1\end{bmatrix}}) = q_{\scriptscriptstyle{=}}.\]
    \item
      $\alpha 0 \prec \beta 1$. Следователно,
      \[\delta(q_{\scriptscriptstyle{=}},\scriptsize{\begin{bmatrix} 0\\1\end{bmatrix}}) = q_{\scriptscriptstyle{>}}.\]
    \item
      $\alpha 1 \succ \beta 0$. Следователно,
      \[\delta(q_{\scriptscriptstyle{=}},\scriptsize{\begin{bmatrix} 1\\0\end{bmatrix}}) = q_{\scriptscriptstyle{<}}.\]
    \end{itemize}
  \item 
    Нека $\alpha \prec \beta$. Тогава:
    \begin{itemize}
    \item 
      $\alpha 0 \prec \beta 0$, $\alpha 1 \prec \beta 1$, $\alpha 0 \prec \beta 1$. Следователно,
      \[\delta(q_{\scriptscriptstyle{<}},\scriptsize{\begin{bmatrix} 0\\0\end{bmatrix}}) = \delta(q_{\scriptscriptstyle{<}},\scriptsize{\begin{bmatrix} 1\\1\end{bmatrix}}) = \delta(q_{\scriptscriptstyle{<}},\scriptsize{\begin{bmatrix} 0\\1\end{bmatrix}}) = q_{\scriptscriptstyle{<}}.\]
    \item
      $\alpha 1 \succ \beta 0$. Следователно,
      \[\delta(q_{\scriptscriptstyle{<}},\scriptsize{\begin{bmatrix} 1\\0\end{bmatrix}}) = q_{\scriptscriptstyle{>}}.\]
    \end{itemize}
  \item
    Нека $\alpha \succ \beta$. Тогава:
    \begin{itemize}
    \item 
      $\alpha 0 \succ \beta 0$, $\alpha 1 \succ \beta 1$, $\alpha 1 \succ \beta 0$. Следователно,
      \[\delta(q_{\scriptscriptstyle{>}},\scriptsize{\begin{bmatrix} 0\\0\end{bmatrix}}) = \delta(q_{\scriptscriptstyle{>}},\scriptsize{\begin{bmatrix} 1\\1\end{bmatrix}}) = \delta(q_{\scriptscriptstyle{>}},\scriptsize{\begin{bmatrix} 1\\0\end{bmatrix}}) = q_{\scriptscriptstyle{>}}.\]
    \item
      $\alpha 0 \prec \beta 1$. Следователно,
      \[\delta(q_{\scriptscriptstyle{>}},\scriptsize{\begin{bmatrix} 0\\1\end{bmatrix}}) = q_{\scriptscriptstyle{<}}.\]
    \end{itemize}
  \end{itemize}
  Докажете, че за така дефинирания автомат $\A$, $\L(\A) = L^{\texttt{rev}}_1$.
\end{hint}
\fi


\newpage

За дума $\alpha = a_1 \cdot a_2 \cdots a_n$, нека да означим
\begin{itemize}
\item 
  $\alpha[i,j] = a_i \cdot a_{i+1} \cdots a_j$,
\item
  $\alpha[i..] = a_i \cdot a_{i+1} \cdots a_n$.
\end{itemize}

Нека също така да означим $\Sigma^{\geq k} = \{ \alpha \in \Sigma^\star \mid |\alpha| \geq k\}$.

\begin{problem}
  Нека $L$ е регулярен език.
  Тогава езиците
  \begin{itemize}
  \item
    $L_1 = \{ \alpha \in \Sigma^\star \mid (\exists i)(\exists j)[\ \alpha[i,j] \in L\ ]\}$;
  \item
    $L_2 = \{ \alpha \in \Sigma^\star \mid (\forall i)(\forall j)[\ \alpha[i,j] \in L\ ]\}$;
  \item 
    $L_3 = \{ \alpha \in \Sigma^\star \mid (\forall i)(\exists j)[\ \alpha[i,j] \in L\ ]\}$;
  \item
    $L_4 = \{ \alpha \in \Sigma^\star \mid (\exists i)(\forall j)[\ \alpha[i,j] \in L\ ]\}$.
  \end{itemize}
  също са регулярни.
\end{problem}
\begin{hint}
  Лесно се вижда, че
  \[L_1 = \Sigma^\star \cdot L \cdot \Sigma^\star,\]
  както и следното
  \[L_2 = \ov{\Sigma^\star \cdot \ov{L} \cdot \Sigma^\star}.\]
  Да обърнем внимание, че
  \begin{align*}
    \alpha \in \Sigma^\star \cdot L & \iff (\exists \beta \in \Sigma^\star)(\exists \gamma \in L)[ \alpha = \beta \cdot \gamma]\\
                                    & \iff (\exists j)[\ \alpha[j..] \in L\ ].
  \end{align*}
  Аналогично получаваме, че
    \begin{align*}
      \alpha \in L\cdot \Sigma^\star & \iff (\exists \gamma \in L)(\exists \beta \in \Sigma^\star)[ \alpha = \gamma \cdot \beta ]\\
                                     & \iff (\exists j)[\ \alpha[1,j] \in L\ ].
  \end{align*}
  Нека да разгледаме по-подробно следния език:
  \begin{align*}
    \ov{L}_3 & = \{\alpha \in \Sigma^\star \mid (\exists i)(\forall j)[\ \alpha[i,j] \not\in L\ ]\}\\
             & = \{\alpha \in \Sigma^\star \mid (\exists i)(\forall j)[\ \alpha[i,j] \in \ov{L}\ ]\}\\
             & = \{\alpha \in \Sigma^\star \mid (\forall j)[\ \alpha[1,j] \in \Sigma^\star \cdot \ov{L}\ ]\}\\
  \end{align*}
  Така получаваме, че:
  \begin{align*}
    L_3 & = \{\alpha \in \Sigma^\star \mid (\exists j)[\ \alpha[1,j] \not\in \Sigma^\star \cdot \ov{L}\ ]\}\\
        & = \{\alpha \in \Sigma^\star \mid (\exists j)[\ \alpha[1,j] \in \ov{\Sigma^\star \cdot \ov{L}}\ ]\}\\
        & = \{\alpha \in \Sigma^\star \mid \alpha \in (\ov{\Sigma^\star \cdot \ov{L}}) \cdot \Sigma^\star \ ]\}.
  \end{align*}
  Оттук заключваме, че
  \[L_3 = (\ov{\Sigma^\star \cdot \ov{L}}) \cdot \Sigma^\star.\]
  Сега лесно можем да съобразим, че
  \[ L_4 = \ov{ \ov{(\Sigma^\star \cdot L)} \cdot \Sigma^\star}.\]
\end{hint}

\begin{problem}
  Нека $L$ е регулярен език над азбуката $\Sigma$.
  За произволно естествено число $k$, докажете, че езикът
  \[L_k = \{ \alpha \in \Sigma^\star \mid (\forall i)(\exists j)[\ |\alpha[i..]| \geq k \implies \alpha[i,j] \in L\ ]\}\]
  е регулярен.
\end{problem}
\begin{hint}
  \begin{align*}
    L_k & = \{ \alpha \in \Sigma^\star \mid (\forall i)(\exists j)[\ |\alpha[i..]| \geq k \implies \alpha[i,j] \in L\ ]\}\\
        & = \{ \alpha \in \Sigma^\star \mid (\forall i)[\ |\alpha[i..]| \geq k \implies (\exists j)[\alpha[i,j] \in L]\ ]\}\\
        & = \{ \alpha \in \Sigma^\star \mid (\forall i)[\ \alpha[i..] \in \Sigma^{\geq k} \implies \alpha[i..] \in L \cdot \Sigma^\star\ ]\}.
  \end{align*}
  Тогава
  \begin{align*}
    \ov{L}_k & = \{ \alpha \in \Sigma^\star \mid (\exists i)[\ \alpha[i..] \in \Sigma^{\geq k}\ \&\ \alpha[i..] \not\in L \cdot \Sigma^\star\ ]\}\\
             & = \{ \alpha \in \Sigma^\star \mid (\exists i)[\ \alpha[i..] \in \Sigma^{\geq k}\ \&\ \alpha[i..] \in \ov{L \cdot \Sigma^\star}\ ]\}\\
             & = \{ \alpha \in \Sigma^\star \mid (\exists i)[\ \alpha[i..] \in \Sigma^{\geq k} \cap \ov{L \cdot \Sigma^\star} \ ]\}\\
             & = \{ \alpha \in \Sigma^\star \mid \alpha \in \Sigma^\star \cdot (\Sigma^{\geq k} \cap \ov{L \cdot \Sigma^\star}) \}.
  \end{align*}
  Сега е ясно, че
  \[L_k = \ov{\Sigma^\star \cdot (\Sigma^{\geq k} \cap \ov{L \cdot \Sigma^\star})}.\]
\end{hint}



% \begin{problem}
%   За думите от вида $\alpha = a_1 \cdot a_2 \cdots a_n$, да означим
%   \[\alpha[i,j,k] = a_i \cdot a_{i+1} \cdots a_j \cdot a_k \cdot a_{k+1} \cdots a_n.\]
  
%   Тогава езикът
%   \[L_5 = \{ \alpha \in \Sigma^\star \mid (\exists i)(\forall j)(\exists k)[\ \alpha[i,j,k] \in L\ ]\}\]
%   също е регулярен.
% \end{problem}


%%% Local Variables:
%%% mode: latex
%%% TeX-master: "../eai"
%%% End:



\begin{extra}
  Нека да вземем назаем от \texttt{python} нотацията за \texttt{slices} на масив и
  ако $\alpha = a_0 \cdot a_1 \cdots a_{n-1}$, то нека 
  \begin{align*}
    \alpha\slice{i:j} & \df
                  \begin{cases}
                    a_i \cdots a_{m-1}, & \text{ ако }i < j\ \&\ m = \min\{j,|\alpha|\}\\
                    \varepsilon, & \text{ иначе}
                  \end{cases}\\
    \alpha\slice{i:} & \df \alpha\slice{i:|\alpha|}\\
    \alpha\slice{:i} & \df \alpha\slice{0:i}.
  \end{align*}
Нека също така $\Sigma^{\geq k} \df \{ \alpha \in \Sigma^\star \mid |\alpha| \geq k\}$.

\begin{problem}
  Нека $L$ е регулярен език. Тогава езиците
  \begin{itemize}
  \item
    $L_1 = \{ \alpha \in \Sigma^\star \mid (\exists i)(\exists j)[\ \alpha\slice{i:j} \in L\ ]\}$;
  \item
    $L_2 = \{ \alpha \in \Sigma^\star \mid (\forall i)(\forall j)[\ i < j \implies \alpha\slice{i:j} \in L\ ]\}$;
  \item 
    $L_3 = \{ \alpha \in \Sigma^\star \mid (\forall i)(\exists j)[\ i < j\ \&\ \alpha\slice{i:j} \in L\ ]\}$;
  \item
    $L_4 = \{ \alpha \in \Sigma^\star \mid (\exists i)(\forall j)[\ i < j \implies \alpha\slice{i:j} \in L\ ]\}$.
  \end{itemize}
  също са регулярни.
\end{problem}
\begin{hint}
  Лесно се вижда, че
  \[L_1 = \Sigma^\star \cdot L \cdot \Sigma^\star,\]
  както и следното
  \[L_2 = \ov{\Sigma^\star \cdot \ov{L} \cdot \Sigma^\star}.\]
  Да обърнем внимание, че
  \begin{align*}
    \alpha \in \Sigma^\star \cdot L & \iff (\exists \beta \in \Sigma^\star)(\exists \gamma \in L)[ \alpha = \beta \cdot \gamma]\\
                                    & \iff (\exists j)[\ \alpha\slice{j:} \in L\ ].
  \end{align*}
  Аналогично получаваме, че
    \begin{align*}
      \alpha \in L\cdot \Sigma^\star & \iff (\exists \gamma \in L)(\exists \beta \in \Sigma^\star)[ \alpha = \gamma \cdot \beta ]\\
                                     & \iff (\exists j)[\ \alpha\slice{:j} \in L\ ].
  \end{align*}
  Нека да разгледаме по-подробно следния език:
  \begin{align*}
    \ov{L}_3 & = \{\alpha \in \Sigma^\star \mid (\exists i)(\forall j)[\ i < j \implies \alpha\slice{i:j} \not\in L\ ]\}\\
             & = \{\alpha \in \Sigma^\star \mid (\exists i)(\forall j)[\ i < j \implies \alpha\slice{i:j} \in \ov{L}\ ]\}\\
             & = \{\alpha \in \Sigma^\star \mid (\forall j)[\ \alpha\slice{:j} \in \Sigma^\star \cdot \ov{L}\ ]\}
  \end{align*}
  Така получаваме, че:
  \begin{align*}
    L_3 & = \{\alpha \in \Sigma^\star \mid (\exists j)[\ \alpha\slice{:j} \not\in \Sigma^\star \cdot \ov{L}\ ]\}\\
        & = \{\alpha \in \Sigma^\star \mid (\exists j)[\ \alpha\slice{:j} \in \ov{\Sigma^\star \cdot \ov{L}}\ ]\}\\
        & = \{\alpha \in \Sigma^\star \mid \alpha \in (\ov{\Sigma^\star \cdot \ov{L}}) \cdot \Sigma^\star \ ]\}.
  \end{align*}
  Оттук заключваме, че
  \[L_3 = (\ov{\Sigma^\star \cdot \ov{L}}) \cdot \Sigma^\star.\]
  Сега лесно можем да съобразим, че
  \[ L_4 = \ov{ \ov{(\Sigma^\star \cdot L)} \cdot \Sigma^\star}.\]
\end{hint}

\begin{problem}
  Нека $L$ е регулярен език над азбуката $\Sigma$.
  За произволно естествено число $k$, докажете, че езикът
  \[L_k = \{ \alpha \in \Sigma^\star \mid (\forall i)(\exists j)[\ |\alpha\slice{i:}| \geq k \implies \alpha\slice{i:j} \in L\ ]\}\]
  е регулярен.
  С други думи, $L_k$ съдържа тези думи, за които от всяка позиция, с изключение на последните $k$, започва дума в езика $L$.
\end{problem}
\begin{hint}
  Да разпишем по-подробно дефиницията на езика $L_k$.
  \begin{align*}
    L_k & = \{ \alpha \in \Sigma^\star \mid (\forall i)(\exists j)[\ |\alpha\slice{i:}| \geq k \implies \alpha\slice{i:j} \in L\ ]\}\\
        & = \{ \alpha \in \Sigma^\star \mid (\forall i)[\ |\alpha\slice{i:}| \geq k \implies (\exists j)[\alpha\slice{i:j} \in L]\ ]\}\\
        & = \{ \alpha \in \Sigma^\star \mid (\forall i)[\ \alpha\slice{i:} \in \Sigma^{\geq k} \implies \alpha\slice{i:} \in L \cdot \Sigma^\star\ ]\}\\
        & = \{ \alpha \in \Sigma^\star \mid (\forall i)[\ \alpha\slice{i:} \not\in \Sigma^{\geq k}\ \lor\ \alpha\slice{i:} \in L \cdot \Sigma^\star\ ]\}\\
        & = \{ \alpha \in \Sigma^\star \mid \neg (\exists i)[\ \alpha\slice{i:} \in \Sigma^{\geq k}\ \&\ \alpha\slice{i:} \not\in L \cdot \Sigma^\star\ ]\}.
  \end{align*}
  Сега е ясно, че:
  \begin{align*}
    \ov{L}_k & = \{ \alpha \in \Sigma^\star \mid (\exists i)[\ \alpha\slice{i:} \in \Sigma^{\geq k}\ \&\ \alpha\slice{i:} \not\in L \cdot \Sigma^\star\ ]\}\\
             & = \{ \alpha \in \Sigma^\star \mid (\exists i)[\ \alpha\slice{i:} \in \Sigma^{\geq k}\ \&\ \alpha\slice{i:} \in \ov{L \cdot \Sigma^\star}\ ]\}\\
             & = \{ \alpha \in \Sigma^\star \mid (\exists i)[\ \alpha\slice{i:} \in \Sigma^{\geq k} \cap \ov{L \cdot \Sigma^\star}\ ]\}\\
             & = \{ \alpha \in \Sigma^\star \mid \alpha \in \Sigma^\star \cdot (\Sigma^{\geq k} \cap \ov{L \cdot \Sigma^\star}) \}\\
             & = \Sigma^\star \cdot (\Sigma^{\geq k} \cap \ov{L \cdot \Sigma^\star}).
  \end{align*}
  Тогава
  \[L_k = \ov{\Sigma^\star \cdot (\Sigma^{\geq k} \cap \ov{L \cdot \Sigma^\star})}.\]
\end{hint}

\end{extra}


%%% Local Variables:
%%% mode: latex
%%% TeX-master: "../eai"
%%% End:


%%% Local Variables:
%%% mode: latex
%%% TeX-master: "../eai"
%%% End:

\chapter{Безконтекстни езици и стекови автомати}

\section{Безконтекстни граматики}

\index{граматика!безконтекстна}
\mynote{В \cite{papadimitriou} дефиницията е различна. Там $\Sigma \subseteq V$. На англ. {\em context-free grammar}. Други срещани наименования на български са {\em контекстно-свободна}, {\em контекстно-независима}.}

В Раздел \ref{sect:regular-grammar} въведохме понятието граматика. След това видяхме как можем да опишем регулярните езици
със специален вид граматики, които нарекохме регулярни граматики.
Сега ще разгледаме още един вид граматики, които описват по-широк клас от езици.

\begin{itemize}
\item 
  Една граматика $G = (V, \Sigma, R, S)$ се нарича {\bf безконтекстна}, ако 
  имаме ограничението, че $R \subseteq V\times (V\cup\Sigma)^\star$.
\item
  \index{език!безконтекстен}
  $L$ се нарича {\bf безконтекстен език}, ако съществува безконтекстна граматика $G$, за която 
  $L = \L(G) = \{\omega \in \Sigma^\star \mid S \derive{\star} \omega\}$.
\end{itemize}

\begin{remark}
  Очевидно е, че всяка регулярна граматика е безконтекстна. Следователно, 
  {\em всеки регулярен език е безконтекстен.}
\end{remark}

Като първи пример нека да видим, че това включване е {\em строго}, т.е. съществува безконтекстен език, който не е регулярен.
Да напомним, че вече видяхме, че езикът $L = \{a^nb^n \mid n\in\Nat\}$ не е регулярен.

\begin{example}
  Да разгледаме безконтекстната граматика $G$ зададена със следните правила:
  \begin{align*}
    & S \to aSb \mid \varepsilon.
  \end{align*}
  Лесно се съобразява, че $\L(G) = \{a^nb^n \mid n\in\Nat\}$.
\end{example}

\begin{example}
  Да разгледаме безконтекстната граматика $G$ зададена със следните правила:
  \begin{align*}
    & S \to aSc\ |\  B\\
    & B \to bBc\ |\ \varepsilon.
  \end{align*}
  Лесно се съобразява, че $\L(G) = \{a^nb^kc^{n+k} \mid n,k\in\Nat\}$.
\end{example}

\begin{example}
  Да разгледаме граматика с правила
  \begin{align*}
    & S \to S + S\ |\ S * S\ |\ (S)\ |\ V\\
    & V \to x\ |\ y\ |\ z
  \end{align*}

  Думата $x * y + z$ има две различни дървета на извод.

  \begin{framed}
    \qtreecenterfalse
    \Tree [.$S$ [.$S$ [.$V$ $x$ ] ] $*$ [.$S$ [.$S$ [.$V$ $y$ ] ] $+$ [.$S$ [.$V$ $z$ ] ] ] ]
    \hskip 0.4in
    \Tree [.$S$ [.$S$ [.$S$ [.$V$ $x$ ] ] $*$ [.$S$ [.$V$ $y$ ] ] ]  $+$  [.$S$ [.$V$ $z$ ] ] ]
  \end{framed}
  
  
  Да разгледаме граматика с правила
  \begin{align*}
    & S \to E + S\ |\ E\\
    & E \to V * E\ |\ V\ |\ (S) * E\ |\ (S)\\
    & V \to x\ |\ y\ |\ z
  \end{align*}
  Сега думата $x * y + z$ има само едно дърво на извод.

  \begin{framed}
    \Tree [.$S$ [.$E$ [.$V$ $x$ ] $*$ [.$E$ [.$V$ $y$ ] ] ] $+$ [.$S$ [.$E$ [.$V$ $z$ ] ] ] ]
  \end{framed}
\end{example}

\begin{example}
  \begin{align*}
    & S \to \texttt{if } S \texttt{ then } S \texttt{ else }S\ |\ \texttt{ if }S \texttt{ then }S\ |\ V\\
    & V \to x\ |\ y\ |\ z
  \end{align*}

  Ние искаме следната граматика:
  \begin{align*}
    & S \to M\ |\ U\\
    & M \to \texttt{if } S \texttt{ then } M \texttt{ else }M\ |\ X\\
    & U \to \texttt{if } S \texttt{ then } S\ |\ \texttt{if } S \texttt{ then } M \texttt{ else }U
  \end{align*}
\end{example}

\begin{example}
  Да разгледаме граматика с правила
  \begin{align*}
    & S \to E\\
    & E \to E + P\ |\ P\\
    & P \to P * N\ |\ N\\
    & N \to (E)\ |\ a.
  \end{align*}
\end{example}

\begin{framed}
  \begin{thm}
    Всеки регулярен език е безконтекстен.
  \end{thm}
\end{framed}
\begin{proof}
  Ще направим индукция по построението на регулярните езици.
  \begin{itemize}
  \item
    Всеки от езиците $\emptyset$, $\{\varepsilon\}$ и $\{a\}$, за всяка буква $a \in \Sigma$ е безконтекстен.
  \item
    Нека $L_1$ и $L_2$ са безконтекстни езици. Тогава:
    \begin{itemize}
    \item
      $L_1 \cup L_2$ е безконтекстен език.
    \item
      $L_1 \cdot L_2$ е безконтекстен език.
    \item
      $L^\star_1$ е безконтекстен език.
    \end{itemize}
  \end{itemize}
\end{proof}

\begin{proposition}
  \label{pr:grammar:add}
  Нека $\alpha \derive{n} \beta$. Тогава за произволна дума $\gamma \in (V \cup \Sigma)^\star$ е изпълнено, че:
  \begin{align*}
    & \alpha\gamma \derive{n} \beta \gamma,\\
    & \gamma\alpha \derive{n} \gamma \beta.
  \end{align*}
\end{proposition}

\begin{proposition}
  \label{pr:grammar:concat}
  Ако $\alpha_1 \derive{n_1} \beta_1, \dots, \alpha_k \derive{n_k} \beta_k$, тогава
  \[\alpha_1\cdots\alpha_k \derive{n} \beta_1\cdots\beta_k,\]
  където $n = \sum^k_{i=1} n_i$.
\end{proposition}
\begin{proof}
  Индукция по $k \geq 1$.
  \begin{itemize}
  \item
    За $k = 1$ е очевидно. В този случай $n_1 = n$.
  \item
   Нека $k > 1$. Тогава от И.П. за $k-1$ имаме, че
   $\alpha_1\cdots\alpha_{k-1} \derive{n'} \beta_1\cdots\beta_{k-1}$ и $n' = \sum^{k-1}_{i=1} n_i$.
   От \Proposition{grammar:add} имаме, че
   \[\alpha_1\cdots\alpha_{k-1}\alpha_{k} \derive{n'} \beta_1\cdots\beta_{k-1}\alpha_k.\]
   Понеже $\alpha_k \derive{n_k} \beta_k$, отново от \Proposition{grammar:add} получаваме, че
   \[\beta_1\cdots\beta_{k-1}\alpha_k \derive{n_k} \beta_1 \cdots \beta_{k-1}\beta_k.\]
   Сега обединяваме двата извода и получаваме, че
   \[\alpha_1\cdots\alpha_{k} \derive{n} \beta_1\cdots\beta_k,\]
   където $n = \sum^k_{i=1} n_i$.
  \end{itemize}
\end{proof}

\begin{framed}
  \begin{proposition}
    \label{pr:grammar:divide}
    Нека $\alpha_1\cdots \alpha_k \derive{n}_G \beta$.
    Тогава съществуват думи $\beta_1,\dots,\beta_k$, такива че за $i = 1,\dots, k$ е изпълнено, че
    $\alpha_i \derive{n_i} \beta_i$, където $\beta = \beta_1\cdots \beta_k$ и $n = \sum^k_{i = 1}n_i$.
  \end{proposition}
\end{framed}
\begin{proof}
  Индукция по $n$.
  \begin{itemize}
  \item
    Нека $n = 0$. Тогава $\beta = \alpha_1 \cdots \alpha_k$ и е ясно, че в този случай $\beta_i = \alpha_i$ и $n_i = 0$.
  \item
    Нека $n > 0$ и $\alpha_1\cdots \alpha_k \derive{n} \beta$. Тогава за някое $i$, $\alpha_i \to_G \alpha'_i$ и
    като приложим \Proposition{grammar:add} получаваме, че
    \[\alpha_1\cdots\alpha_i\cdots\alpha_k \to_G \alpha_1\cdots\alpha'_i\cdots\alpha_k.\]
    Според дефиницията на релацията $\derive{n}$ имаме, че
    \[\alpha_1\cdots\alpha'_i\cdots\alpha_k \stackrel{n-1}{\to}_G \beta.\]
    От И.П. получаваме, че съществуват думи $\beta_1,\dots,\beta_k$, такива че $\beta = \beta_1 \cdots \beta_k$
    и $\alpha_j \derive{n_j} \beta_j$ за $j \neq i$ и $\alpha'_i \derive{n'_i} \beta_i$, като
    \[n-1 = n'_i + \sum_{j\neq i} n_j.\]
    Понеже имаме, че $\alpha_i \to_G \alpha'_i \derive{n'_i}\beta_i$,
    то е ясно, че за $n_i = n'_i + 1$ имаме $\alpha_i \derive{n_i} \beta_i$ и
    \[n = \sum^{k}_{i=1} n_i.\]
  \end{itemize}
\end{proof}




%%% Local Variables: 
%%% mode: latex
%%% TeX-master: "../eai"
%%% End: 

\newpage
\subsection{Равен брой леви и десни скоби}

Тук ще разглеждаме азбука $\Sigma$, която включва буквите $\texttt{[}$ и $\texttt{]}$.
Нека за по-голяма яснота да положим
\begin{align*}
  & \texttt{left}(\alpha) \df \card{\alpha}{\texttt{[}} & \comment{\text{брой срещания на $\texttt{[}$ в $\alpha$}}\\
  & \texttt{right}(\alpha) \df \card{\alpha}{\texttt{]}}. & \comment{\text{брой срещания на $\texttt{]}$ в $\alpha$}}
\end{align*}

\begin{problem}
  \label{prob:nanb}
  Нека $\omega$ е произволна дума над азбуката $\{\texttt{[}, \texttt{]}\}$. 
  Тогава:
  \begin{enumerate}[a)]
  \item 
    ако $\texttt{left}(\omega) = \texttt{right}(\omega) + 1$, то съществуват думи $\omega_1$, $\omega_2$, за които е изпълнено:
    \begin{itemize}
    \item 
      $\omega = \omega_1 \texttt{[} \omega_2$;
    \item
      $\texttt{left}(\omega_1) = \texttt{right}(\omega_1)$;
    \item
      $\texttt{left}(\omega_2) = \texttt{right}(\omega_2)$.
    \end{itemize}
  \item
    ако $\texttt{right}(\omega) = \texttt{left}(\omega) + 1$, то съществуват думи $\omega_1$, $\omega_2$, за които е изпълнено:
    \begin{itemize}
    \item 
      $\omega = \omega_1 \texttt{]} \omega_2$;
    \item
      $\texttt{left}(\omega_1) = \texttt{right}(\omega_1)$;
    \item
      $\texttt{left}(\omega_2) = \texttt{right}(\omega_2)$.
    \end{itemize}
  \end{enumerate}
\end{problem}
\begin{hint}
  \marginpar{Другият случай е аналогичен}
  Ще се съсредоточим върху случая, когато $\omega$ е дума, за която $\texttt{left}(\omega) = \texttt{right}(\omega) + 1$.
  Ще докажем а) с индукция по дължината на думата.
  \begin{itemize}
  \item 
    $\abs{\omega} = 1$. Тогава $\omega_1 = \omega_2 = \varepsilon$ и $\omega = \texttt{[}$.
  \item
    Да приемем, че твърдението а) е вярно за думи с дължина $\leq n$.
  \item
    $\abs{\omega} = n+1$. Ще разгледаме два случая, в зависимост от първия символ на $\omega$.
    \begin{itemize}
    \item 
      Случаят $\omega = \texttt{[}\omega'$ е очевиден. (Защо?)
    \item
      Интересният случай е $\omega = \texttt{]}\omega'$.    
      Тогава $\omega = \texttt{]}^{i+1}\texttt{[}\omega'$, за някое $i \in \Nat$.
      Да разгледаме думата $\omega''$, която се получава от $\omega$
      като премахнем първото срещане на думата $\texttt{][}$, т.е. 
      $\omega'' = \texttt{]}^i\omega'$ и $\abs{\omega''} = n-1$.
      Понеже от $\omega$ сме премахнали равен брой леви и десни скоби, то
      $\texttt{left}(\omega'') = \texttt{right}(\omega'')+1$.
      Според {\bf И.П.} за $\omega''$ са изпълнени свойствата:
      \begin{itemize}
      \item 
        $\omega'' = \omega''_1\texttt{[}\omega''_2$;
      \item
        $\texttt{left}(\omega''_1) = \texttt{right}(\omega''_1)$;
      \item
        $\texttt{left}(\omega''_2) = \texttt{right}(\omega''_2)$.
      \end{itemize}
      Понеже $\texttt{]}^i$ е префикс на $\omega''_1$, за да получим обратно $\omega$, трябва 
      да прибавим премахнатата част $\texttt{][}$ веднага след $\texttt{]}^i$ в $\omega''_1$.
    \end{itemize}
  \end{itemize}
\end{hint}

\begin{problem}
  За произволна дума $\omega \in \{ \texttt{[}, \texttt{]} \}^\star$, 
  докажете, че ако $\texttt{left}(\omega) > \texttt{right}(\omega)$, то съществуват думи $\omega_1$ и $\omega_2$,
  за които са изпълнени свойствата:
  \begin{itemize}
  \item 
    $\omega = \omega_1 \texttt{[} \omega_2$;
  \item
    $\texttt{left}(\omega_1) \geq \texttt{right}(\omega_1)$;
  \item
    $\texttt{left}(\omega_2) \geq \texttt{right}(\omega_2)$.
  \end{itemize}
\end{problem}

\begin{framed}
  \begin{problem}
    Да се докаже, че езикът 
    \[L = \{\ \alpha \in \{\texttt{[}, \texttt{]}\}^\star\mid \texttt{left}(\alpha) = \texttt{right}(\alpha)\ \}\]
    е безконтекстен.
  \end{problem}  
\end{framed}
\begin{hint}
  \marginpar{  Алтернативна граматика за езика $L$ е
    \[S \to \varepsilon\ |\ \texttt{[}S\texttt{]}\ |\ \texttt{]}S\texttt{[}\ |\ SS.\]}
  Една възможна граматика $G$ е следната: 
  \[S \to \texttt{[}S\texttt{]}S\ |\ \texttt{]}S\texttt{[}S\ |\ \varepsilon.\]
  % Например, да разгледаме извода на думата $aabbba$ в тази граматика:
  % \begin{align*}
  %   S & \to aSbS \to aaSbSbS \to aa\varepsilon bSbS \to aab\varepsilon bS \to aabbbSaS\\
  %   & \to aabbb\varepsilon a S \to aabbba.
  % \end{align*}
  
  Като следствие от \Problem{nanb} може лесно да се изведе, че за думи $\omega$, за които $\texttt{left}(\omega) = \texttt{right}(\omega)$,
  е изпълнено следното:
  \begin{enumerate}[a)]
  \item 
    ако $\omega = \texttt{[}\omega'$, то са изпълнени свойствата:
    \begin{itemize}
    \item 
      $\omega = \texttt{[}\omega_1\texttt{]}\omega_2$;
    \item
      $\texttt{left}(\omega_1) = \texttt{right}(\omega_1)$;
    \item
      $\texttt{left}(\omega_2) = \texttt{right}(\omega_2)$.
    \end{itemize}
  \item
    ако $\omega = \texttt{]}\omega'$, то са изпълнени свойствата:
    \begin{itemize}
    \item 
      $\omega = \texttt{]}\omega_1\texttt{[}\omega_2$;
    \item
      $\texttt{left}(\omega_1) = \texttt{right}(\omega_1)$;
    \item
      $\texttt{left}(\omega_2) = \texttt{right}(\omega_2)$.
    \end{itemize}
  \end{enumerate}

  Сега първо ще проверим, че $L \subseteq \L(G)$.
  За целта ще докажем с {\em пълна индукция} по дължината на думата $\omega$, че за всяка дума $\omega$ със свойството $\texttt{left}(\omega) = \texttt{right}(\omega)$ е изпълнено
  $S \rightarrow^\star \omega$.
  \begin{itemize}
  \item 
    Нека $\abs{\omega} = 0$. Тогава $S \rightarrow \varepsilon$.
  \item
    Да приемем, че за всяка дума с дължина $\leq k$ твърдението е вярно.
  \item
    Нека $\abs{\omega} = k+1$. Имаме два случая.
    \begin{itemize}
    \item 
      $\omega = \texttt{[}\omega^\prime$, т.е. от а) на \Problem{nanb}, 
      $\omega = \texttt{[}\omega_1\texttt{]}\omega_2$ и $\texttt{left}(\omega_1) = \texttt{right}(\omega_1)$, $\texttt{left}(\omega_2) = \texttt{right}(\omega_2)$.
      Тогава $\abs{\omega_1} \leq k$ и по И.П. $S \rightarrow^\star \omega_1$.
      Аналогично, $S \rightarrow^\star \omega_2$.
      Понеже имаме правило $S \rightarrow \texttt{[}S\texttt{]}S$, заключаваме че 
      $S \to^\star \texttt{[}\omega_1\texttt{]}\omega_2$.
    \item
      $\omega = \texttt{]}\omega^\prime$, т.е. свойство б), $\omega = \texttt{]}\omega_1\texttt{[}\omega_2$ и 
      $\texttt{left}(\omega_1) = \texttt{right}(\omega_1)$, $\texttt{left}(\omega_2) = \texttt{right}(\omega_2)$.
      Този случай се разглежда аналогично.
    \end{itemize}
  \end{itemize}
  
  Преминаваме към доказателството на другата посока, т.е. $\L(G) \subseteq L$.
  Тук с индукция по дължината на извода $\ell$ ще докажем, че
  $S \stackrel{\ell}{\to} \omega$, то $\omega \in M$,
  където
  \[M = \{\omega \in \{a,b,S\}^\star \mid \texttt{left}(\omega) = \texttt{right}(\omega)\}.\]
  \begin{itemize}
  \item 
    Ясно, че $S \stackrel{0}{\rightarrow} S$ и $S \in M$.
  \item
    Да разгледаме дума $\omega$, за която $S \stackrel{k+1}{\to} \omega$.
    Това означава, че съществува дума $\alpha$, за която
    \[S \stackrel{k}{\to} \alpha \to \omega.\]
    От {\bf И.П.} имаме, че $\alpha \in M$.
    Нека $\omega$ се получава от $\alpha$ с прилагане на правило от вида $S \to \gamma$.
    Разглеждаме всички варианти за думата $\alpha \in M$ и за правилото $S \to \gamma$ в граматиката $G$
    за да докажем, че $\omega \in M$.
    Удобно е да представим всички случаи в таблица.
    \begin{center}
      \begin{tabular}{| c | c | c |}
        \hline
        От И.П. за $\alpha$ & правило на $G$ & $\omega$ \\ \hline
        $\in M$ & $S \to \texttt{[}S\texttt{]}S$ & $\in M$ \\ \hline
        $\in M$ & $S \to \texttt{]}S\texttt{[}S$ & $\in M$ \\ \hline
        $\in M$ & $S \to \varepsilon$ & $\in M$ \\ \hline
      \end{tabular}
    \end{center}    
    Във всички случаи лесно се установява, че $\omega \in M$.
  \end{itemize}
  Така за всяка дума $\omega \in \L(G)$ следва, че
  \[\omega \in \Sigma^\star \cap M = L.\]
\end{hint}

%%% Local Variables:
%%% mode: latex
%%% TeX-master: "../eai"
%%% End:

\newpage
\subsection{Балансирани скоби}

Нека $\alpha$ е дума над азбука, която включва буквите $\texttt{[}$ и $\texttt{]}$. 
Ще казваме, че че $\alpha$ е {\bf балансирана}, ако са изпълнени свойствата:
\begin{itemize}
\item 
  $\texttt{left}(\alpha) = \texttt{right}(\alpha)$;
\item
  За всеки префикс $\gamma$ на $\alpha$,
  $\texttt{left}(\gamma) \geq \texttt{right}(\gamma)$.
\end{itemize}

\begin{framed}
  \begin{problem}
    Докажете, че езикът 
    \[L = \{\ \alpha \in \{\texttt{[},\texttt{]}\}^\star \mid \alpha\text{ е балансирана дума}\ \}\]
    е безконтекстен.
  \end{problem}  
\end{framed}
\begin{hint}
  \marginpar{\cite[стр. 135]{kozen}}
  \marginpar{\writedown Докажете, че езикът $L$ не е регулярен! }
  Да разгледаме граматиката $G$ с правила
  \[S \to \texttt{[}S\texttt{]}\ |\ SS\ |\ \varepsilon.\]
  Ще докажем, че $L = \L(G)$.
  
  Първо ще докажем включването $\L(G) \subseteq L$.
  Да разгледаме \[M \df \{\ \alpha \in \{\texttt{[},\texttt{]}, S\}^\star \mid \alpha\text{ е балансирана}\ \}.\]
  
  Нека $S \to^\star_G \alpha$. Ще докажем с индукция по дължината $\ell$ на извода на $\alpha$ от $S$,
  че $\alpha \in M$. Случаят, когато $\ell = 0$ е очевиден.
  Нека $S \to^{\ell}_G \beta \to^1_G \alpha$.
  От {\bf И.П.} имаме, че $\beta \in M$, т.е. $\beta$ е балансирана.

  Лесно се съобразява, че във всички случаи за думите $\beta$ и $\alpha$ имаме следното:
  \begin{center}
    \begin{tabular}{| c | c | c |}
      \hline
      $\text{от И.П. }$ & $\text{правило на }G$ & $\text{извод}$ \\ \hline
      $\beta \in M$ & $S \rightarrow \texttt{[}S\texttt{]}$ & $\alpha \in M$ \\ \hline
      $\beta \in M$ & $S \rightarrow SS$ & $\alpha \in M$ \\ \hline
      $\beta \in M$ & $S \rightarrow \varepsilon$ & $\alpha \in M$ \\ \hline
    \end{tabular}
  \end{center}

  За включването $L \subseteq \L(G)$, нека $\alpha \in L$.
  Ще докажем с индукция по дължината на думата, че $\alpha \in \L(G)$.
  Ясно е, че във всички нетривиални случаи можем да запишем думата $\alpha$ като $\alpha = \texttt{[}\beta\texttt{]}$.
  Проблемът е, че в общия случай не е ясно дали можем да приложим индукционното предположение за $\beta$,
  защото е възможно $\beta \not\in L$. Например, $\alpha = \texttt{[][]}$.
  Тогава $\beta = \texttt{][} \not \in L$.
  Поради тази причина, трябва да сме по-внимателни и да разгледаме два случая.
  \begin{itemize}
  \item 
    \marginpar{\comment т.е. $\beta \neq \varepsilon$ и $\beta \neq \alpha$}
    Нека $\alpha$ има {\em същински} префикс $\beta \in L$.
    Понеже $\alpha \in L$, лесно се съобразява, че $\alpha = \beta\gamma$ и $\gamma \in L$.
    Сега можем да приложим {\bf И.П.} за $\beta$ и $\gamma$ и да получим, че 
    $\beta \in \L(G)$ и $\gamma \in \L(G)$, т.е.
    $S \to^\star_G \beta$ и $S \to^\star_G \gamma$.
    Понеже имаме правило $S \to_G SS$, то е ясно, че $\alpha \in \L(G)$.
  \item
    Нека $\alpha$ да няма същински префикс $\gamma \in L$.
    Ясно е, че тогава $\alpha = \texttt{[}\beta\texttt{]}$, за някое $\beta$
    и $\texttt{left}(\beta) = \texttt{right}(\beta)$.
    Да видим защо $\beta \in L$.
    
    Ако $\beta \in L$, то ще можем да приложим {\bf И.П.} за $\beta$ и ще сме готови.
    За всеки префикс $\gamma$ на $\beta$ имаме, че $\texttt{[}\gamma$ е префикс на $\alpha$,
    и понеже $\alpha \in L$, то $\texttt{left}(\texttt{[}\gamma) \geq \texttt{right}(\texttt{[}\gamma)$.
    Възможно ли е $\texttt{left}(\gamma) < \texttt{right}(\gamma)$ ?
    Това може да се случи единствено ако $\texttt{left}(\texttt{[}\gamma) = \texttt{right}(\texttt{[}\gamma)$.
    Но тогава $\texttt{[}\gamma$ е същински префикс на $\alpha$, за който $\texttt{[}\gamma \in L$,
    което противоречи на случая, който разглеждаме.
    Това означава, че за произволен префикс $\gamma$ на $\beta$,
    $\texttt{left}(\gamma) \geq \texttt{right}(\gamma)$ и оттук $\beta \in L$ и можем да приложим {\bf И.П.}
    Тогава $S \to^\star_G \beta$ и чрез правилото $S \to_G \texttt{[}S\texttt{]}$
    получаваме, че $\alpha \in \L(G)$.    
  \end{itemize}
\end{hint}

%%% Local Variables:
%%% mode: latex
%%% TeX-master: "../eai"
%%% End:

\newpage
\section{Алгоритми}

Тук ще докажем, че съществуват {\em полиномиални} алгоритми, които за дадена безконтекстна граматика $G$ проверяват:
\begin{itemize}
\item
  дали $\L(G) = \emptyset$;
\item
  дали $|\L(G)| = \infty$;
\item
  дали $|\L(G)| < \infty$;
\item
  по дадена дума $\alpha$ дали $\alpha \in \L(G)$.
\end{itemize}

\subsection{Опростяване на безконтекстни граматики}

\subsubsection*{Премахване на безполезните променливи}

Нека е дадена безконтекстната граматика $G = \CFG$.
\mynote{\cite[стр. 88]{hopcroft1}, \cite[стр. 256]{hopcroft2}.}
Една променлива $A$ се нарича {\bf полезна}, ако съществува извод от следния вид:
\[S \derive{\star}_G \alpha A \beta \derive{\star}_G \gamma,\]
където $\gamma \in \Sigma^\star$, а $\alpha,\beta \in (V \cup \Sigma)^\star$.
Това означава, че една променлива е полезна, ако участва в извода на някоя дума в езика на граматиката.
Една променлива се нарича {\bf безполезна}, ако не е полезна.
Целта ни е да получим еквивалентна граматика $G'$ без безполезни променливи.
Ще решим задачата като разгледаме две леми.

\begin{lemma}
  \label{lem:useless1}
  Нека е дадена безконтекстната граматика $G$ и $\L(G) \neq \emptyset$.
  Съществува алгоритъм, който намира граматика $G'$, за която
  $\L(G) = \L(G')$ и със свойството, че  за всяка променлива $A' \in V'$, съществува дума $\alpha \in \Sigma^\star$,
  за която $A' \derive{\star}_{G'} \alpha$.
\end{lemma}
\begin{hint}
  Целта ни е да намерим множеството $\texttt{Gen}$ от променливи, които генерират думи, т.е. търсим множеството
  \[\texttt{Gen} \df \{A \in V \mid A \yield{\star} \alpha\ \&\ \alpha \in \Sigma^\star \}.\]
  За целта ще построим редица от множества $\texttt{Gen[n]}$ по следния начин:
  \mynote{Всяка итерация на алгоритъма отнема $\mathcal{O}(|G|)$ време. Следователно, изпълнението на целия алгоритъм отнема $\mathcal{O}(|G|^2)$ време.}
  \begin{itemize}
  \item 
    $\texttt{Gen[0]} \df \emptyset$;
  \item
    $\texttt{Gen[n+1]} \df \{A\in V \mid (\exists \alpha \in (\Sigma \cup \texttt{Gen[n]})^\star)[A \to_G \alpha] \}$.
  \item
    Спираме, когато стигнем до такова $n$, за което $\texttt{Gen[n]} = \texttt{Gen[n+1]}$. Лесно се съобразява, че
    $(\forall k \geq n)[\texttt{Gen[n]} = \texttt{Gen[k]}]$.
  \end{itemize}
  Трябва да докажем, че $\texttt{Gen} = \texttt{Gen[n]}$.
  \mynote{\writedown Докажете сами!}
  Това ще направим като докажем, че за всяко $k$, е изпълнена еквивалентността:
  \[A \in \texttt{Gen[k]} \iff (\exists \alpha\in\Sigma^\star)[A \yield{\leq k} \alpha]\]
  Дефинираме $G'$ като $V' = \texttt{Gen[n]}$ и правилата на $G'$ са само тези правила на $G$, в които участват променливи от $V'$ и букви от $\Sigma$.
\end{hint}

\begin{framed}
  \begin{corollary}
    Съществува {\em полиномиален} алгоритъм, който определя за всяка безконтекстна граматика $G$ дали $\L(G) = \emptyset$.
  \end{corollary}  
\end{framed}
\begin{proof}
  Прилагаме алгоритъма от \Lemma{useless1} и намираме множеството от променливи $V'$.
  Тогава $\L(G) = \emptyset$ точно тогава, когато $S \not\in V'$.  
\end{proof}

\begin{lemma}
  \label{lem:useless2}
  Съществува алгоритъм, който по дадена безконтекстна граматика $G = \CFG$, намира $G' = \pair{V',\Sigma,S,R'}$, $\L(G') = \L(G)$,
  със свойството, че за всяко $X \in V'$ съществуват $\alpha, \beta \in (V'\cup\Sigma)^\star$,
  за които $S \derive{\star} \alpha X \beta$,
  т.е. всяка променлива в $G'$ е достижима от началната променлива $S$.
\end{lemma}
\begin{hint}
  Нашата цел е да намерим множеството
  \[\texttt{Reach} \df \{B \in V \mid S \yield{\star} \alpha B \beta\text{, за някои }\alpha,\beta \in (\Sigma \cup V)^\star\}.\]
  Новата граматика $G'$ ще има променливи $V' \df \texttt{Reach}$,
  като правилата на граматика $G'$ са същите като тези на $G$, но ограничени до $V'$.
  Лесно се доказава, че $\L(G) = \L(G')$.
  Остава да намерим множеството $\texttt{Reach}$.
  За да постигнем тази цел, ще започнем да строим множествата $\texttt{Reach[i]} \subseteq V$ по следния начин:
  \begin{align*}
    \texttt{Reach[0]} \df & \{S\}\\
    % \texttt{Reach[i+1]}
    % \df & \{ B \in V \mid (\exists A \in \texttt{Reach[i]})[ A \to_G \alpha B \beta\text{, за някои }\alpha,\beta \in (V \cup \Sigma)^\star]\}\\
                        %   & \cup \texttt{Reach[i]}\\
                         % \df & \{ B \in V \mid (\exists A \in \texttt{Reach[i]})[ A \yield{\leq 1} \alpha B \beta\text{, за някои }\alpha,\beta \in (V \cup \Sigma)^\star]\},
  % \end{align*}
  % което може да се запише и така:
  % \begin{align*}
    \texttt{Reach[i+1]} \df & \{ B \in V \mid (\exists A \in \texttt{Reach[i]})[ A \to_G \alpha B \beta\text{, за някои }\alpha,\beta \in (V \cup \Sigma)^\star]\}\\
                            & \cup \texttt{Reach[i]}.    
  \end{align*}
  Докажете, че за всяко $i$ е изпълнено:
  \[\texttt{Reach[i]} = \{B \in V \mid S \yield{\leq i} \alpha B \beta\text{, за някои }\alpha,\beta \in (\Sigma \cup V)^\star\}.\]
  \mynote{\writedown Защо сме сигурни, че ще достигнем такава стъпка?}
  Спираме да строим тези множества, когато достигнем до стъпка $n$, за която
  \[\texttt{Reach[n]} = \texttt{Reach[n+1]}.\]
  Съобразете, че за това $n$ е изпълнено, че $\texttt{Reach[n]} = \texttt{Reach}$, т.е.
  \[\texttt{Reach[n]} = \{B \in V \mid S \yield{\star} \alpha B \beta\text{, за някои }\alpha,\beta \in (\Sigma \cup V)^\star\}.\]
\end{hint}

\begin{extra2}
  \begin{example}
    Да разгледаме безконтекстната граматика $G$ с правила:
    \begin{align*}
      & S \to AB\ |\ aA\\
      & A \to a\ |\ aAa\\
      & B \to SB\ |\ BC\\
      & C \to \varepsilon\ |\ cC.
    \end{align*}
    Първо да намерим променливите, от които се извеждат думи.
    \begin{itemize}
    \item
      $\texttt{Gen[0]} = \emptyset$;
    \item 
      $\texttt{Gen[1]} = \{A, C\}$, защото $A \to a$ и $C \to \varepsilon$;
    \item
      $\texttt{Gen[2]} = \{A, C, S\}$, защото $S \to aA$;
    \item
      не можем да добавим $B$ към $\texttt{Gen[3]}$, следователно $\texttt{Gen[3]} = \texttt{Gen[2]}$.
    \end{itemize}
    Получаваме граматиката $G'$:
    \begin{align*}
      & S \to aA\\
      & A \to a\ |\ aAa\\
      & C \to \varepsilon\ |\ cC.
    \end{align*}
    Сега премахваме променливите и буквите, които не са достижими от началната промелива $S$.
    \begin{itemize}
    \item
      $\texttt{Reach[0]} = \{S\}$;
    \item
      $\texttt{Reach[1]} = \{S,A\}$, защото $S \to aAa$;
    \item
      $\texttt{Reach[2]} = \{S,A\}$.
    \end{itemize}
    Така получаваме граматиката $G''$ със следните правила:
    \begin{align*}
      & S \to aA\\
      & A \to a\ |\ aAa.
    \end{align*}
  \end{example}
\end{extra2}

\begin{theorem}
  За всяка безконтекстна граматика $G$, за която $\L(G) \neq \emptyset$, съществува еквивалетнтна на нея безконтекстна граматика $G'$ без безполезни правила.
  Освен това, граматиката $G'$ може да се намери за полиномиално време.
\end{theorem}
\begin{hint}
  \mynote{\writedown Защо е важна последователността на прилагане? Например, да разгледаме граматиката
    \begin{align*}
      & S \to AB\ |\ ab\\
      & A \to aA\ |\ \varepsilon\\
      & B \to bB.      
    \end{align*}
    Ако първо приложим процедурата от \Lemma{useless2}, то нищо няма да премахнем, защото $A$ и $B$ са достижими от $S$.
    След това, ако приложим процедурата от \Lemma{useless1}, то ще премахнем всички правила, в които участва $B$.
    Така $A$ става недостижима от $S$ и трябва пак да приложим процедурата от \Lemma{useless2}.
    Алгоритъмът описан тук е квадратичен. Има и линеен такъв. Вижте \cite[стр. 296]{hopcroft2}}
  Прилагаме върху $G$ първо процедурата от \Lemma{useless1} и след това върху резултата прилагаме процедурата от \Lemma{useless2}.
\end{hint}

\begin{extra}
  \begin{problem}
    Проверете дали $\L(G) = \emptyset$, където правилата на $G$ са:
    \begin{align*}
      & S \to AS\ |\ BC\\
      & A \to a\ |\ BA\ |\ SB\\
      & B \to b\ |\ BC\ |\ AB\\
      & C \to CB\ |\ SC\ |\ AS.
    \end{align*}
  \end{problem}
  % \begin{hint}
  %   Лесно се проверява, че:
  %   \begin{align*}
  %     \texttt{Gen[1]} & = \{A,B\}\\
  %     \texttt{Gen[2]} & = \{A,B\}.
  %   \end{align*}
  % \end{hint}
\end{extra}

%%% Local Variables:
%%% mode: latex
%%% TeX-master: "../eai"
%%% End:


\subsubsection*{Премахване на $\varepsilon$-правила}
\index{$\varepsilon$-правила}

\begin{lemma}
  Съществува {\em експоненциален} алгоритъм, такъв че превръща всяка безконтекстна граматика $G$ в безконтекстна граматика $G'$ без правила от вида $A \to \varepsilon$,
  и $\L(G') = \L(G) \setminus \{\varepsilon\}$.
\end{lemma}
\begin{hint}
  За да премахнем правилата от вида $A \to \varepsilon$ от граматиката, първо трябва да намерим множеството
  от променливи
  \[\texttt{E} = \{A \in V \mid A \yield{\star} \varepsilon\}.\]
  За да направим това, следваме процедурата:
  \begin{enumerate}[1)]
  \item 
    \mynote{На всяка стъпка трябва да обходим цялата граматика, следователно всяка итерация отнема $\mathcal{O}(|G|)$ време.}
    Първо % трябва да намерим множеството
    % \[\texttt{E} = \{A \in V \mid A \yield{\star}_G \varepsilon\}.\]
    % Това правим като
    строим множествата $\texttt{E[n]}$ по следния начин:
    \begin{itemize}[-]
    \item
      $\texttt{E[0]} \df \emptyset$;
    \item
      \mynote{Обърнете внимание, че имаме $\texttt{E[n]}^\star$, а
        не просто $\texttt{E[n]}$. Също така да напомним, че $\emptyset^\star = \{\varepsilon\}$.}
      $\texttt{E[n+1]} \df \{ B \in V \mid (\exists \alpha \in \texttt{E[n]}^\star)[B \to_G \alpha ]\}$.
    \item
      Докажете, че за всяко $n$ е изпълнено, че
      \[\texttt{E[n]} = \{A \in V \mid A \yield{\leq n} \varepsilon\}.\]
    %   $A \in \texttt{E[n]}$ точно тогава, когато съществува дърво на извод $P$, за което
    % $A = \texttt{root}(P)$, $\texttt{yield}(P) = \varepsilon$ и $\texttt{height}(P) \leq n$.
    \item
      \mynote{Намираме $\texttt{E}$ за време $\mathcal{O}(|G|^2)$.}
      Спираме на първото $n$, за което $\texttt{E[n]} = \texttt{E[n+1]}$.
      Лесно се съобразява, че за това $n$ е изпълнено, че
      \[\texttt{E[n]} = \{A \in V \mid A \yield{\star} \varepsilon\}.\]
    \end{itemize}
  \item
    Строим множеството от правила $R'$, в което няма $\varepsilon$-правила по следния начин.
    За всяко правило $A \to_G X_1\cdots X_k$,
    добавяме към $R'$ всички правила от вида $A \to_G Y_1\cdots Y_k$, където:
    \mynote{Броят на правилата може да се увеличи експоненциално, защото в най-лошия случай извеждаме всички подмножества на дадено множество от променливи}
    \begin{itemize}[-]
    \item 
      ако $X_i \not\in \texttt{E}$, то $Y_i = X_i$;
    \item
      ако $X_i \in \texttt{E}$, то $Y_i = X_i$ или $Y_i = \varepsilon$;
    \item
      не всички $Y_i$ са $\varepsilon$.
    \end{itemize}
    \mynote{\writedown Докажете сами!}
    Лесно се съобразява, че $\L(G') = \L(G) \setminus \{\varepsilon\}$.
    % \begin{itemize}
    % \item
    %   Нека $A \stackrel{l}{\to_G} \omega$ за $\omega \neq \varepsilon$. Тогава $A \to^\star_{G'} \omega$.
    % \item
    %   Нека $A \stackrel{l}{\to_{G'}} \omega$ за $\omega \neq \varepsilon$. Тогава $A \to^\star_{G} \omega$.
    % \end{itemize}
  \end{enumerate}
\end{hint}

\begin{extra2}
  \begin{example}
  Нека е дадена граматиката $G$ с правила
  \begin{align*}
    & S \to D\\
    & D \to AD\ |\ b\\
    & A \to AB\ |\ BC\ |\ a\\
    & B \to AA\ |\ UC\\
    & C \to \varepsilon\ |\ CA\ |\ a\\
    & U \to \varepsilon\ |\ aUb.
  \end{align*}

  Тогава
  \begin{itemize}[-]
  \item
    $\texttt{E[0]} = \emptyset$;
  \item
    $\texttt{E[1]} = \{C,U\}$;
  \item
    $\texttt{E[2]} = \{C,U,B\}$;
  \item
    $\texttt{E[3]} = \{C,U,B,A\}$;
  \item
    $\texttt{E[4]} = \texttt{E[3]}$. Следователно,
    \[\texttt{E} = \{X \in V \mid X \yield{\star} \varepsilon\} = \{C,U,B,A\}.\]
  \end{itemize}
  
  Това означава, че $\varepsilon \not\in \L(G)$.
  Граматиката $G'$ без $\varepsilon$-правила, за която $\L(G') = \L(G)$ има следните правила
  \begin{align*}
    & S \to D \\
    & D\to AD\ |\ D\ |\ b \\
    & A \to A\ |\ B\ |\ C\ |\ AB\ |\ BC\ |\ a \\
    & B\to A\ |\ C\ |\ AA\ |\ U\ |\ UC\\
    & C \to C\ |\ A\ |\ CA\ |\ a\\
    & U \to aUb\ |\ ab.
  \end{align*}
\end{example}  
\end{extra2}

%%% Local Variables:
%%% mode: latex
%%% TeX-master: "../eai"
%%% End:


\subsubsection*{Премахване на преименуващи правила}
\index{преименуващи правила}
\mynote{В \cite[стр. 263]{hopcroft2} преименуващите правила се наричат \emph{unit productions}.}
Едно правило в граматиката $G$ се нарича {\bf преименуващо}, ако е от вида $A \to B$.
Нека е дадена граматика $G = \CFG$.
Ще построим еквивалентна граматика $G'$ без преименуващи правила.
В началото нека в $G'$ да добавим всички правила от $G$, които не са преименуващи.
След това, за всякa променлива $A$, за която $A \derive{\star}_G B$,
ако $B \to \alpha$ е правило в граматиката $G$, което не е преименуващо,
то добавяме към $G'$ правилото $A \to \alpha$.

\begin{lemma}
  Нека $G$ е безконтекстна граматика без $\varepsilon$-правила.
  Съществува {\em полиномиален} алгоритъм, такъв че превръща всяка безконтекстна граматика $G$ в безконтекстна граматика $G'$ без преименуващи правила
  и $\L(G') = \L(G) \setminus \{\varepsilon\}$.
\end{lemma}
\begin{hint}
  Първо намираме множеството от двойки
  \[\texttt{Ren} = \{(A,B) \in V\times V \mid A \derive{\star} B\}\]
  като строим множествата по такъв начин, че да е изпълнено свойството:
  \[\texttt{Ren[n]} = \{(A,B) \in V\times V \mid A \derive{\leq n} B\}\]

  \begin{align*}
    & \texttt{Ren[0]} \df \{(A,A) \mid A \in V\}\\
    & \texttt{Ren[n+1]} \df \texttt{Ren[n]} \cup \{(A,B) \mid (\exists C)[ A \to_G C\ \&\ (C,B) \in \texttt{Ren[n]}]\}.
  \end{align*}
  
  % \begin{itemize}
  % \item
  %   % Ясно е, че $\texttt{Ren[0]} \df \{(A,A) \mid A \in V\}$.
  %   $
  % \item
  %   Нека имаме $\texttt{Ren[n]}$. Тогава дефинираме
  %   \mynote{Всяка итерация отнема $\mathcal{O}(|G|)$ време.}
  %   \[\texttt{Ren[n+1]} \df \texttt{Ren[n]} \cup \{(A,B) \mid (\exists C)[ A \to_G C\ \&\ (C,B) \in \texttt{Ren[n]}]\}.\]
  % \item
  \mynote{$|\texttt{Ren}|$ има големина $\mathcal{O}(|G|^2)$.}
  Спираме на първото $n$, за което $\texttt{Ren[n]} = \texttt{Ren[n+1]}$. Тогава $\texttt{Ren} = \texttt{Ren[n]}$.
  % \end{itemize}
  
  Нека $R'_0 \df R \setminus (V\times V)$ са правилата на $R$, които не са преименуващи. Тогава
  \[R' \df  \{\ (A,\alpha) \in V \times (V\cup\Sigma)^\star \mid (\exists B)[(A,B) \in \texttt{Ren}\ \&\ (B,\alpha) \in R'_0]\ \}.\]
\end{hint}

% \ExtraMaterial{
\begin{extra}
\begin{multicols}{2}
    \begin{example}
      Нека е дадена граматиката $G$ с правила  
      \begin{align*}
        & S \to B\ |\ CC\ |\ b\\
        & A \to S\ |\ SB\\
        & B \to C\ |\ BC\\
        & C \to AB\ |\ a\ |\ b.
      \end{align*}
      Да намерим първо множеството $\texttt{Ren}$.
      \begin{align*}
        \texttt{Ren[0]} = \{ & (S,S), (A,A), (B,B), (C,C)\};\\
        \texttt{Ren[1]} = \{ & (S,S), (S,B), (A,A), (A,S),\\
                             & (B,B), (B,C), (C,C)\};\\
        \texttt{Ren[2]} = \{ & (S,S),(S,B), (S,C), (B,B),\\
                             & (B,C), (C,C), (A,A), (A,S),\\
                             & (A,B)\};\\
        \texttt{Ren[3]} = \{ & (S,S),(S,B), (S,C), (B,B),\\
                             & (B,C), (C,C), (A,A), (A,S),\\
                             & (A,B), (A,C)\};\\
        \texttt{Ren[4]} = \{ & (S,S),(S,B), (S,C), (B,B),\\
                             & (B,C), (C,C), (A,A), (A,S),\\
                             & (A,B), (A,C)\}.
      \end{align*}
      
      Получихме, че $\texttt{Ren[3]} = \texttt{Ren[4]}$.
      Оттук можем да заключим следното:
      \begin{align*}
        & A \derive{\star} A,B,S,C\\
        & B \derive{\star} B,C\\
        & S \derive{\star} S,B,C\\
        & C \derive{\star} C.          
      \end{align*}
      % \end{itemize}
      
      Първо добавяме към $R'$ правилата, които не са преименуващи, а именно:
      \begin{align*}
        & A \to SB\\
        & B \to BC\\
        & C \to AB\ |\ a\ |\ b\\
        & S \to CC\ |\ b.
      \end{align*}
      \begin{itemize}
      \item 
        Понеже имаме,че $A \derive{\star} B,S,C$, то добавяме към $R'$ правилата:
        \[A \to BC\ |\ AB\ |\ a\ |\ b\ |\ CC.\]
      \item
        Понеже имаме,че $B \derive{\star}_G C$, то добавяме към $R'$ правилата:
        \[B \to AB\ |\ a\ |\ b.\]
      \item
        Понеже имаме, че $S \derive{\star}_G B,C$, то добавяме към $R'$ правилата:
        \[S \to BC\ |\ AB\ |\ a\ |\ b.\]
      \end{itemize}
      Накрая получаваме, че граматиката $G'$ има правила
      \begin{align*}
        & S \to BC\ |\ AB\ |\ CC\ |\ a\ |\ b\\
        & A \to BS\ |\ BC\ |\ AB\ |\ a\ |\ b\ |\ CC\\
        & B \to AB\ |\ a\ |\ b\ |\ BC\\
        & C \to AB\ |\ a\ |\ b.
      \end{align*}
    \end{example}
  \end{multicols}
\end{extra}

\begin{extra}
\begin{problem}
  Премахнете преименуващите правила от граматиката $G$, като запазите езика, ако $G$ има следните правила:
    \begin{align*}
      & S \to C\ |\ CC\ |\ b\\
      & A \to B\\
      & B \to S\ |\ C\ |\ BC\\
      & C \to a\ |\ AB;
    \end{align*}
\end{problem}
\end{extra}

%%% Local Variables:
%%% mode: latex
%%% TeX-master: "../eai"
%%% End:


\subsubsection*{Премахване на дългите правила}

Едно правило се нарича дълго, ако е от вида $A \to \beta$, където $|\beta| \geq 3$.
Да разгледаме едно дълго правило в граматиката от вида $A \to x_1x_2\cdots x_k$, 
където $k \geq 3$ и $x_i \in V \cup \Sigma$. За да получим еквивалентна граматика без това дълго правило,
добавяме нови променливи $A_1,\dots, A_{k-2}$, и правила
\[A \to x_1A_1,\ A_1 \to x_2A_2, \dots,\ A_{k-2} \to x_{k-1}x_k.\]


\begin{problem}
  Нека е дадена граматиката  $G = \pair{\{S,A,B,C\}, \{a,b\}, S, R}$.
  Използвайте обща конструкция, за да премахнете ,,дългите'' правила от $ G$ като при това получите 
  безконтестна граматика $G_1$ с език $\L(G) = \L(G_1)$, където правилата на граматиката са:
  % \begin{enumerate}[a)]
  % \item
  %   \begin{align*}
  %     & S \to \varepsilon\ |\ ab\ |\ aAba\\
  %     & A\to aBCb\\
  %     & B\to bbb\\
  %     & C\to aC\ |\ aCaC;
  %   \end{align*}
  % \item
    \begin{align*}
      & S\to CC\ |\ b\\
      & A\to BSB\ |\ a\\
      & B\to ba\ |\ BC\\
      & C\to BaSA\ |\ a\ |\ b.
    \end{align*}
  % \end{enumerate}
\end{problem}

%%% Local Variables:
%%% mode: latex
%%% TeX-master: "../eai"
%%% End:


\subsection{Нормална Форма на Чомски}
\index{Чомски}
%[стр. 99 от \cite{sipser}]
\index{нормална форма на Чомски}
Една безконтекстна граматика е в {\bf нормална форма на Чомски}, ако
всяко правило е от вида
\[A \rightarrow BC\mbox{ и }A \rightarrow a,\]
като $A, B, C$.
\mynote{Ако искаме $\varepsilon$ да бъде в езика на граматиката, то добавяме нова начална променлива $S_0$ и правило $S_0 \to_G \varepsilon$.}
% Освен това, позволяваме правилото $S\to\varepsilon$.
% \footnote{В \cite[стр. 151]{papadimitriou} дефиницията е малко по-различна.
% Там дефинират $G$ да бъде в нормална форма на Чомски ако $R \subseteq V\times(V\cup\Sigma)^2$.
% В този случай губим езиците $\{\varepsilon\}$ и $\{a\}$, за $a\in\Sigma$.}

\begin{framed}
  \begin{theorem}
    Всеки безконтекстен език $L$, който не съдържа $\varepsilon$, се поражда от безконтекстна граматика в нормална форма на Чомски.
  \end{theorem}
\end{framed}
\begin{proof}
%  \marginpar{Броят на правилата може да се увеличи експоненциално.}
  Нека имаме безконтекстна граматика $G$, за която $L = \L(G)$.
  Ще построим безконтекстна граматика $G^\prime$ в нормална форма на Чомски, $L = \L(G^\prime)$.
  % [стр. 99 от \cite{sipser}]
  Следваме следната процедура:
  \begin{itemize}
  % \item
  %   Добавяме нов начален символ $S_0$ и правило $S_0 \to S$.
  \item
    Премахваме дългите правила.
    Това можем да направим за време $\mathcal{O}(|G|)$
    като новата граматика ще има дължина $\mathcal{O}(|G|)$.
  \item
    \mynote{Важно е, че преди това сме премахнали дългите правила. Вижте \cite[стр. 296]{hopcroft2}.}
    Премахваме $\varepsilon$-правилата.
    Това можем да направим за време $\mathcal{O}(|G|^2)$
    като новата граматика ще има дължина $\mathcal{O}(|G|)$.
  \item
    Премахваме преименуващите правила.
    Това можем да направим за време $\mathcal{O}(|G|^2)$
    като новата граматика ще има дължина $\mathcal{O}(|G|^2)$.
  \item
    За правила от вида $A\to u_1 u_2$, където $u_1, u_2 \in V \cup \Sigma$, 
    заменяме всяка буква $u_i$ с новата променлива $U_i$
    и добавяме правилото $U_i\to u_i$.
    Например, правилото $A \to aB$ се заменя с правилото $A \to XB$ и добавяме правилото $X \to a$,
    където $X$ е нова променлива.
    Това можем да направим за време $\mathcal{O}(|G|)$ и новата граматика ще има дължина $\mathcal{O}(|G|)$.
  \item
    Ако искаме $\varepsilon$ да бъде в езика на граматиката, то добавяме нова начална променлива $S_0$
    и правило $S_0 \to_G \varepsilon$.
  \end{itemize}
\end{proof}

\begin{theorem}
  При дадена безконтекстна граматика $G$, можем да намерим еквивалентна
  на нея граматика $G'$ в нормална форма на Чомски за време $\mathcal{O}(|G|^2)$,
  като получената граматика е с дължина $\mathcal{O}(|G|^2)$.
\end{theorem}

\begin{theorem}
  \mynote{\cite[стр. 137]{hopcroft1}. Важно е, че алгоритмите са полиномиални. От \Lemma{pumping-context} имаме експоненциални алгоритми.}
  Съществуват \emph{полиномиални} алгоритми, които определят по дадена безконтекстна граматика $G$ дали:
  \begin{enumerate}[a)]
  \item
    $\abs{\L(G)} < \infty$;
  \item
    $\abs{\L(G)} = \infty$.
  \end{enumerate}
\end{theorem}
\begin{proof}
  Нека е дадена една безконтекстна граматика $G$.
  Нека да разгледаме граматиката $G'$ в НФЧ {\em без безполезни променливи}, за която $\L(G) = \L(G')$.
  От граматиката $G' = \pair{V',\Sigma,S,R'}$ строим граф с възли променливите от $V'$ като
  за $A,B \in V'$ имаме ребро $A \to B$ точно тогава, когато съществува $C \in V'$,
  за което $A \to BC$ или $A \to CB$ е правило в $R'$.
  Сега ще съобразим, че ако в получения граф имаме цикъл, то $\L(G') = \infty$.

  Да разгледаме един такъв цикъл в графа:
  \[A_0 \to A_1 \to A_2 \to \cdots \to A_n \to A_0.\]
  Това означава, че 
  \[A_0 \derive{1}_G \alpha_1 A_1 \beta_1 \derive{1}_G \alpha_2 A_2 \beta_2 \derive{1}_G \cdots \derive{1}_G \alpha_n A_n \beta_n \derive{1}_G \alpha_{n+1} A_0 \beta_{n+1}.\]
  Понеже граматиката е в НФЧ имаме, че $|\alpha_i\beta_i| = i$.
  Понеже няма безполезни символи в граматика и тя е в НФЧ следва, че съществуват думи $\omega,\rho \in \Sigma^\star$, за които $\alpha_{n+1} \derive{\star} \omega$ и $\beta_{n+1} \derive{\star} \rho$,
  където $|\omega\rho| \geq n+1$, защото $|\alpha_{n+1}\beta_{n+1}| = n+1$.
  Понеже няма безполезни символи в граматиката, то съществуват $\delta, \gamma \in \Sigma^\star$, за които
  $S \derive{\star} \delta  A_0 \gamma$. Получаваме:
  \[ S \derive{\star}_G \delta A_0 \gamma \derive{\star}_G \delta \alpha_{n+1} A_0 \beta_{n+1} \gamma \derive{\star}_G \delta \omega A_0 \rho\gamma \derive{\star}_G \cdots \derive{\star}_G \delta \omega^i A_0 \rho^i \gamma \derive{\star}_G \delta\omega^i\alpha\rho^i\gamma,\]
  където $A_0 \derive{\star}_G \alpha$, за някое $\alpha \in \Sigma^\star$.

  Ако в графът няма цикли, то езикът $\L(G)$ е краен, защото ако от променливата $A$ най-дългият път има дължина $k+1$,
  то за променливите $B$ и $C$, за които имаме правилото $A\to BC$, най-дългият път от $B$ и $C$ има дължина най-много $k$.

  Ако най-дългият път от $A$ има дължина $k$, то всяко дърво на извод $P$, за което $\texttt{root}(P) = A$ и $\texttt{yield}(P) \in \Sigma^\star$,
  е такова, че $\texttt{height}(P) \leq k$. От \Corollary{tree:upper-bound} следва, че всички думи, които се извеждат от $A$ са най-много $2^{k}$ на брой,
  защото всяко такова дърво на извод е двоично и $\texttt{yield}(P) \leq 2^k$.
\end{proof}

%%% Local Variables:
%%% mode: latex
%%% TeX-master: "../eai"
%%% End:


\subsection{Проблемът за принадлежност}

\begin{theorem}
  Съществува {\em полиномиален} алгоритъм относно дължината на входната дума, който проверява дали дадена дума принадлежни на граматиката $G$.
\end{theorem}
% \begin{proof}[стр. 154 от \cite{papadimitriou}]
Можем да приемем, че $G = \CFG$ е граматика в нормална форма на Чомски.
% Нека $\alpha = a_0a_1\dots a_{n-1}$ е дума, за която искаме да проверим дали $\alpha \in \L(G)$.
\mynote{Това е алгоритъм на Cocke, Younger и Kasami (CYK), които откриват идеята за този алгоритъм независимо един от друг. Той е пример за динамично
  програмиране \cite[стр. 192]{kozen}, \cite[стр. 141]{shallit}. При вход дума $\alpha$, алгоритъмът работи за време $\mathcal{O}(\abs{\alpha}^3)$.}
\begin{algorithm}[H]
  \caption{Проверка дали $\alpha \in \L(G)$}
  \label{alg:belongs-to-grammar}
  \begin{algorithmic}[1]
    \State $n = \texttt{len}(\alpha)$ \Comment{Вход дума $\alpha = a_0a_1\cdots a_{n-1}$}
    \ForAll{$i < n$}
    \State $\texttt{V[i][i]} := \{A \in V \mid A\to_G \alpha\slice{i}\}$ \label{alg:cyk:initial}
    \ForAll{$j \in (i,n)$}
    % \If{$i = j$}
    % \State $\texttt{V[i][i]} = \{A \in V \mid A\to_G \alpha\slice{i}\}$ \label{alg:cyk:initial}
    % \Else
    \State{$\texttt{V[i][j]} := \emptyset$}
    % \EndIf
    \EndFor
    \EndFor
    \ForAll{$s \in [1, n)$} \Comment{Дължина на интервала} \label{alg:cyk:first-loop}
    \ForAll{$i < n-s$}\Comment{Начало на интервала}
    \ForAll{$k \in [i, i + s)$}\Comment{Разделяне на интервала}
    \If{$\exists A\to BC \in R\ \&\ B \in \texttt{V[i][k]}\ \&\ C\in \texttt{V[k+1][i+s]}$}
    \State $\texttt{V[i][i+s]} := \texttt{V[i][i+s]} \cup \{A\}$ \label{alg:cyk:add-variable}
    \EndIf
    \EndFor
    \EndFor
    \EndFor
    \If{$S \in \texttt{V[0][n-1]}$}
    \State \Return \texttt{True}\Comment{Има извод на думата от $S$}
    \Else
    \State \Return \texttt{False}
    \EndIf
  \end{algorithmic}
\end{algorithm}

\begin{lemma}
  \mynote{Да напомним, че според нашата нотация,
  \[\alpha\slice{i:j} = a_i\cdots a_{j-1}.\]}
  Нека е дадена безконтекстната граматика $G$ в нормална форма на Чомски и думата $\alpha$.
  Всеки път, точно преди да изпълни ред (\ref{alg:cyk:first-loop}) от програмата описана в Алгоритъм \ref{alg:belongs-to-grammar},
  е изпълнено, че за всяко $i < n-s$,
  \[\texttt{V[i][i+s]} = \{A \in V \mid A \derive{\star}_G \alpha\slice{i:i+s+1}\}.\]
\end{lemma}
\begin{proof}
  Пълна индукция по $s$.
  За $s = 0$ е ясно, защото от ред (\ref{alg:cyk:initial}) и от факта, че граматиката е в нормална форма на Чомски следва, че за всяко $i < n$, 
  \[\texttt{V[i][i]} = \{A \in V \mid A \to_G \alpha\slice{i}\} = \{A \in V \mid A \derive{\star}_G \alpha\slice{i:i+1}\}.\]
  
  Сега ще докажем твърдението за $s > 0$, т.е. ще докажем, че за всяко $i < n-s$ е изпълнено, че:
  \[\texttt{V[i][i+s]} = \{A \in V \mid A \derive{\star}_G \alpha\slice{i:i+s+1}\}.\]
  Да разгледаме двете посоки на това равенство.
  \begin{description}
  \item[($\subseteq$)]
    Нека $A \in \texttt{V[i][i+s]}$.
    Единствената стъпка на алгоритъма, при която може да сме добавили променливата $A$ към множеството $\texttt{V[i][i+s]}$ е ред (\ref{alg:cyk:add-variable}).
    Тогава имаме, че съществува $k$, което $i \leq k < i+s$, и 
    \begin{align*}
      & B \in \texttt{V[i][k]}, & \comment k = i + \overbrace{(k - i)}^{s'}\\
      & C \in \texttt{V[k+1][i+s]}, & \comment i + s = (k+1)+\underbrace{(i+s-k-1)}_{s''}\\
      & A\to_G BC.
    \end{align*}
    Понеже $s' < s$ и $s'' < s$, от \IndHyp имаме, че
    \begin{align*}
      & B \derive{\star}_G \alpha\slice{i:k+1} \\
      & C \derive{\star}_G \alpha\slice{k+1:i+s+1}. 
    \end{align*}
    Заключаваме веднага, че $A \derive{\star}_G \alpha\slice{i:i+s+1}$.
  \item[($\supseteq$)]
    Нека $A \derive{\star}_G \alpha\slice{i:i+s+1}$ и да разгледаме първото правило, което сме приложили в този извод.
    Понеже $G$ е в нормална форма на Чомски, правилото е от вида $A \to_G BC$ и тогава, според \Proposition{grammar:divide-2}, съществува $k$, за което $i \leq k < i+s$, и
    \begin{align*}
      & B \derive{\star} \alpha\slice{i:k+1} \\
      & C \derive{\star} \alpha\slice{k+1:i+s+1}.
    \end{align*}
    От \IndHyp получаваме, че $B \in \texttt{V[i][k]}$ и $C \in \texttt{V[k+1][i+s]}$.
    Тогава от ред (\ref{alg:cyk:add-variable}) на алгоритъма е ясно, че $A \in \texttt{V[i][i+s]}$.
  \end{description}
\end{proof}

{\scriptsize

\begin{multicols}{2}
  
\begin{example}
  Нека е дадена граматиката $G$ в нормална форма на Чомски с правила 
  \begin{align*}
    & S\rightarrow a\ |\ AB\ |\ AC\\
    & A\rightarrow a\\
    & B\rightarrow b\\
    & C\rightarrow SB\ |\ AS.
  \end{align*}
  Ще приложим CYK алгоритъма за да проверим дали $aaabb \in \L(G)$.
  
  \begin{itemize}
  \item 
    За $s = 0$ имаме, че:
    \begin{itemize}
    \item 
      $\texttt{V[0][0]} = \texttt{V[1][1]} = \texttt{V[2][2]} = \{S,A\}$;
    \item
      $\texttt{V[3][3]} = \texttt{V[4][4]} = \{B\}$.
    \end{itemize}
  \item
    За $s = 1$ имаме, че:
    \begin{itemize}
    \item
      $\texttt{V[0][1]} = \texttt{V[1][2]} = \{C\}$;
    \item
      $\texttt{V[2][3]} = \{S,C\}$;
    \item
      $\texttt{V[3][4]} = \emptyset$.
    \end{itemize}
  \item
    За $s = 2$ имаме, че:
    \begin{itemize}
    \item
      $\texttt{V[0][2]} = \{S\} \cup \emptyset$;
    \item
      $\texttt{V[1][3]} = \{S,C\} \cup \emptyset$;
    \item
      $\texttt{V[2][4]} = \emptyset \cup \{C\}$.
    \end{itemize}
  \item
    За $s = 3$ имаме, че:
    \begin{itemize}
    \item
      $\texttt{V[0][3]} = \{S,C\} \cup \emptyset \cup \emptyset = \{S,C\}$;
    \item
      $\texttt{V[1][4]} = \{S\} \cup \emptyset \cup \{C\} = \{S,C\}$.
    \end{itemize}
  \item
    За $s = 4$ имаме, че:
    \begin{itemize}
    \item 
      $\texttt{V[0][4]} = \{S,C\} \cup \emptyset \cup \emptyset \cup \{C\}= \{S,C\}$.
    \end{itemize}
  \end{itemize}
  Понеже $S \in \texttt{V[0][4]}$, то $aaabb \in \L(G)$.
\end{example}
\end{multicols}

}


%%% Local Variables:
%%% mode: latex
%%% TeX-master: "../eai"
%%% End:


%%% Local Variables: 
%%% mode: latex
%%% TeX-master: "../eai"
%%% End: 

\newpage
\section{Лема за покачването}

\begin{lemma}[за покачването (безконтекстни езици)]
  \index{лема за покачването!безконтекстни езици}
  \label{lem:pumping-context} 
  \marginpar{\cite[стр. 123]{sipser1}, \cite[стр. 125]{hopcroft1}, \cite[стр. 148]{kozen}}
  За всеки безконтекстен език $L$ съществува $p>0$, такова
  че ако $\alpha\in L$, за която $\abs{\alpha} \geq p$, то съществува разбиване на думата на пет части, $\alpha=xyuvw$,
  \marginpar{Ще казваме, че $p$ е константа на покачването}
  за което е изпълнено:
  \begin{enumerate}[1)]
  \item
    $\abs{yv}\geq 1$,
  \item
    $\abs{yuv}\leq p$, и
  \item
    $(\forall i\geq 0)[xy^iuv^iw\in L]$.
\end{enumerate}
\end{lemma}
\begin{proof}
  Нека $G$ е граматиката за езика $L$. Да положим
  \[b = \max\{\ |\gamma| \mid A \to \gamma \text{ е правило в }G\ \}.\]
  Нека
  \[p = b^{\abs{V} + 1}.\]
  Ще покажем, че $p$ е константа на покачването за граматиката $G$.
  
  Да разгледаме произволна дума $\alpha \in \L(G)$, за която $\abs{\alpha} \geq p$.
  Понеже $\L(G)$ е безкраен език, то такива думи съществуват.
  Тогава от \Cor{tree:upper-bound} следва, че ако $P = (T,\lambda)$ е дърво на извод за $\alpha$ в $G$, за което $X_\varepsilon = S$, то
  \[b^{|V| + 1}\leq |\alpha| \leq b^{\texttt{height}(T)}.\]
  Оттук следва, че
  \[|V| + 1 \leq \texttt{height}(T).\]
  Нека измежду всички дървета на извод $P = (T,\lambda)$ за $\alpha$ в граматиката $G$ сме избрали такова с минимална мнощност на $T$.
  Да фиксираме максимален път $\pi$ в $T$, т.е. дума $\pi \in T$ и $|\pi| = \texttt{height}(T)$.
  Ясно е, че
  \marginpar{Броят на върховете по пътя $\pi$ е 1 + броя на ребрата.}
  \[|V| + 1 \leq |\{ X_\gamma \mid \gamma \prec \pi\}| = \texttt{height}(T).\]
  Тук нарочно не броим $X_\pi$, защото $X_\pi \in \Sigma$, а ние се интересуваме само от променливите.
  От принципа на Дирихле следва, че съществува $\beta \prec \pi$, за което
  \begin{equation}
    \label{eq:10}
    X_\beta \in \{X_\gamma \mid \beta \prec \gamma \prec \pi\}.
  \end{equation}
  Нека $\beta$ е най-дългата дума, за която е изпълнено свойството (\ref{eq:10}).
  Да означим с $\gamma$ една от думите, за която $\beta \prec \gamma \prec \pi$ и $X_\gamma = X_\beta$.
  От избора на $\beta$ следва, че
  \begin{equation}
    \label{eq:11}
    \texttt{height}(T_\beta) = |\{ X_\gamma \mid \beta \preceq \gamma \prec \pi\}| \leq |V| + 1,
  \end{equation}
  защото ако допуснем противното, то ще следва, че $|\{ X_\gamma \mid \beta \prec \gamma \prec \pi\}| \geq |V| + 1$ и от принципа на Дирихле ще следва, че
  съществува по-дълга от $\beta$ дума, за която е изпълнено свойството (\ref{eq:10}).
  
  Нека $u = \texttt{yield}(P_\gamma)$. Тогава $yuv = \texttt{yield}(P_\beta)$, за някои думи $y,v \in \Sigma^\star$.
  Това означава, че думата $\alpha$ може да се запише като $\alpha = xyuvw$, за някои $x, w \in \Sigma^\star$.
  % \marginpar{По пътя $\pi$ е възможно да срещнем много различни двойки повтарящи се променливи, ние избрахме възможно най-долната.}
  Освен това имаме свойствата:
  \begin{enumerate}[1)]
  \item
    Да видим първо защо $\abs{yv}\geq 1$.
    Ако допуснем, че $|yv| = 0$, то това означава, че $\alpha = xuw$ и тази дума
    би имала дърво на извод $(P \setminus P_\beta) \odot P_\gamma$, защото:
    \begin{itemize}
    \item 
      $\texttt{yield}(P_\gamma) = u$;
    \item
      $\texttt{yield}(P \setminus P_\beta) = x X_\beta w$;
    \item
      $X_\alpha = X_\beta$.
    \end{itemize}
    Но това противоречи на избора на $P$ като дървото на извод за думата $\alpha$ с минимален брой елементи в $T$.
  \item
    $\abs{yuv} \leq p$, защото $yuv = \texttt{yield}(T_\beta)$ и оттук следва, че:
    \begin{align*}
      \abs{yuv} & \leq b^{\texttt{height}(T_\beta)} & \comment\text{от \Cor{tree:upper-bound}}\\
                & \leq b^{|V|+1} & \comment\text{от (\ref{eq:11})}\\
                & = p
    \end{align*}
  \item
    Остана само да видим защо за всяко естествено число $i$ е изпълнено, че $xy^iuv^iw \in L$.
    Това е най-сложното разсъждение в лемата, затова нека да го разгледаме внимателно.
    Понеже $\beta \prec \gamma$, нека $\rho$ е тази дума, за която $\gamma = \beta \cdot \rho$.
    Ясно е, че $P_\beta\setminus P_\gamma = (P\setminus P_\gamma)_\beta$.
    \begin{itemize}
    \item
      Нека $i = 0$. Тогава $xy^0uv^0w = xuv$ има дърво на извод $(P \setminus P_\beta ) \odot P_\gamma$,
      защото от избора на $\gamma$ и $\beta$ имаме, че:
      \begin{itemize}
      \item 
        $\texttt{yield}(P_\gamma) = u$;
      \item
        $\texttt{yield}(P \setminus P_\beta) = x X_\beta w$;
      \item
        $X_\alpha = X_\beta$.
      \end{itemize}
    \item
      Нека $i = 1$.
      Ясно е от условието, че $xyuvw \in L$, но нека да направим наблюдението, че дървото на извод $P$ за думата $\alpha$ в $G$
      може да се запише и така:
      \[P = (P \setminus P_\beta) \odot (P_\beta \setminus P_\rho) \odot P_\gamma, \]
      защото:
      \begin{itemize}
      \item
        $\texttt{yield}(P\setminus P_\beta) = x X_\beta w$;
      \item
        $\texttt{yield}(P_\beta \setminus P_\rho) = y X_\gamma v$;
      \item
        $\texttt{yield}(P_\gamma) = u$;
      \item
        $X_\beta = X_\gamma$.
      \end{itemize}
      Сега вече имаме идея как да обобщим горната конструкция за $i > 1$.
    \item
      Нека $i > 1$. Тогава $xy^iuv^iw$ има дърво на извод
      \[(P \setminus P_\beta) \odot (P_\beta \setminus P_\rho)^{(i)} \odot P_\gamma,\]
      защото:
      \begin{itemize}
      \item
        $\texttt{yield}(P\setminus P_\beta) = x X_\beta w$;
      \item
        $\texttt{yield}((P_\beta \setminus P_\rho)^{(i)}) = y^i X_\gamma v^i$ от \Prob{tree:iteration};
      \item
        $\texttt{yield}(P_\gamma) = u$;
      \item
        $X_\beta = X_\gamma$.
      \end{itemize}
    \end{itemize}

  \end{enumerate}
\end{proof}

\begin{cor}
  \label{cor:pumping-context-free}
  \marginpar{Ако $L$ е краен език, то е ясно, че $L$ е безконтекстен.}
  Нека $L$ е произволен {\bf безкраен} език. Нека също така е изпълнено, че:
  \begin{description}
  \item[($\forall$)]
    за {\em всяко} естествено число $p \geq 1$,
  \item[($\exists$)]
    можем да намерим дума $\alpha \in L$, $\abs{\alpha}\geq p$, такава че
  \item[($\forall$)]
    за {\em всяко} разбиване на думата на пет части, $\alpha = xyuvw$, със свойствата $\abs{yv} \geq 1$ и $\abs{yuv} \leq p$,
  \item[($\exists$)]
    можем да посочим $i \in \Nat$, за което е изпълнено, че $xy^iuv^iw \not\in L$.
  \end{description}  
  \marginpar{\writedown Докажете! Аналогично е на \Cor{pumping-reg}}
  Тогава $L$ {\bf не} е безконтекстен език.
\end{cor}

\begin{cor}
  \marginpar{\writedown Докажете!}
  Нека $G$ е безконтекстна граматика и $p$ е константата на покачването за $G$.
  Тогава $\abs{\L(G)} = \infty$ точно тогава, когато съществува $\alpha \in \L(G)$, за която $p \leq \abs{\alpha} < 2p$.
\end{cor}
% \begin{proof}
%   Ако съществува дума $\alpha \in L$, за която $\abs{\alpha} \geq p$, то от \Lem{pumping-context} следва,
%   че $\abs{L} = \infty$, защото $\alpha = xyuvw$ и $xy^iuv^iw \in L$, за всяко $i\in\Nat$.

%   За другата посока, нека сега $\abs{L} = \infty$.
%   Да изберем най-късата дума $\alpha \in L$, за която $\abs{\alpha} \geq p$.
%   Ще докажем, че $p \leq \abs{\alpha} < 2p$. За целта да допуснем, че $\abs{\alpha} \geq 2p$.
%   Тогава от \Lem{pumping-context} следва, че $\alpha = xyuvw$, $\abs{yv} \geq 1$, $\abs{yuv} \leq p$, $xy^0uv^0w = xuw \in L$.
%   Ако $\abs{xuw} < p$, то $\abs{yv} > p$, защото $\abs{yv} + \abs{xuw} = \abs{\alpha} \geq 2p$, и следователно $\abs{yuv} > p$, което е противоречие.
%   Следва, че $\abs{\alpha} > \abs{xuw} \geq p$.
%   Получихме, че думата $xuw\in L$ и $\abs{xuw} \geq p$. Това е противоречие с минималността на $\alpha$.
% \end{proof}

% \begin{framed}
%   \Lem{pumping-context} е полезна, когато искаме да докажем, че даден език $L$ {\bf не} е безконтекстен.
%   За целта, доказваме отрицанието на свойствата от \Lem{pumping-context} за $L$, т.е.
%   за всяка константа $p$, ние намираме дума $\alpha \in L$, $\abs{\alpha}\geq p$, такава че за всяко разбиване на думата на пет части, $\alpha = xyuvw$,
%   със свойствата $\abs{yv} \geq 1$ и $\abs{yuv} \leq p$, е изпълнено, че $(\exists i)[xy^iuv^iw \not\in L]$.
% \end{framed}

\begin{remark}
  Алгоритъм за проверка дали един безконтекстен език е безкраен следвайки горния критерий би 
  имал експоненциална сложност относно $|G|$.
\end{remark}

\begin{problem}
  \label{prob:anbncn}
  Докажете, че езикът 
  \[L = \{a^nb^nc^n\ \mid\ n\in\Nat\}\]
  не е безконтекстен.
\end{problem}
\begin{proof}
  \begin{description}
  \item[$(\forall)$]
    Разглеждаме произволна константа $p \geq 1$.
  \item[$(\exists)$]
    Избираме дума $\alpha \in L$, $\abs{\alpha} \geq p$.
    В случая, нека $\alpha = a^pb^pc^p$.
  \item[$(\forall)$]
    Разглеждаме произволно разбиване $xyuvw = \alpha$, за което $\abs{yuv} \leq p$ и $1 \leq \abs{yv}$.
  \item[$(\exists)$]
    Ще изберем $i$, за което $xy^iuv^iw \not\in L$.
    Знаем, че поне едно от $y$ и $v$ не е празната дума.
    Имаме няколко случая за $y$ и $v$.
    \begin{itemize}
    \item
      $y$ и $v$ са думи съставени от една буква.
      В този случай получаваме, че $xy^2uv^2w$ има различен брой букви $a$, $b$ и $c$.
    \item
      $y$ или $v$ е съставена от две букви.
      Тогава е възможно да се окаже, че $xy^2uv^2w$ да има равен брой $a$, $b$ и $c$,
      но тогава редът на буквите е нарушен.
    \item
      понеже $\abs{yuv} \leq p$, то не е възможно в $y$ или $v$ да се срещат и трите букви.
    \end{itemize}  
    Оказа се, че във всички възможни случаи за $y$ и $v$, 
    $xy^2uv^2w \not\in L$.
  \end{description}
  Така от \Cor{pumping-context-free} следва, че езикът $L$ не е безконтекстен.
\end{proof}

\begin{problem}
  Докажете, че езикът
  \[\L(a^\star b^\star c^\star) \setminus \{a^nb^nc^n \mid n \in \Nat\}\]
  е безконтекстен.
\end{problem}

\begin{problem}
  Докажете, че езикът
  \[L = \{a^ib^jc^k\ \mid\ 0 \leq i \leq j \leq k\}\]
  не е безконтекстен.
\end{problem}
\begin{proof}
  \begin{description}
  \item[$(\forall)$]
     Разглеждаме произволна константа $p \geq 1$.
   \item[$(\exists)$]
     Избираме дума $\alpha \in L$, $\abs{\alpha} \geq p$.
     В случая, нека $\alpha = a^pb^pc^p$.
   \item[$(\forall)$]
     Разглеждаме произволно разбиване $xyuvw = \alpha$, за което $\abs{yuv} \leq p$ и $1 \leq \abs{yv}$.
     Знаем, че поне една от $y$ и $v$ не е празната дума.
   \item[$(\exists)$] Ще намерим $i \in \Nat$, за което $xy^iuv^iw \not\in L$.
    \begin{itemize}
    \item
      $y$ и $v$ са съставени от една буква.
      Имаме три случая.
      \begin{enumerate}[i)]
      \item
        $a$ не се среща в $y$ и $v$.
        Тогава $xy^0vu^0w$ съдържа повече $a$ от $b$ или $c$.
      \item
        $b$ не се среща в $y$ и $v$.
        \begin{itemize}
        \item 
          Ако $a$ се среща в $y$ или $v$, тогава $xy^2uv^2w$ съдържа повече $a$ от $b$.
        \item
          Ако $c$ се среща в $y$ или $v$, тогава $xy^0uv^0w$ съдържа по-малко $c$ от $b$.
        \end{itemize}
      \item
        $c$ не се среща в $y$ и $v$.
        Тогава $xy^2uv^2w$ съдържа повече $a$ или $b$ от $c$.
      \end{enumerate}      
     \item
       $y$ или $v$ е съставена от две букви.
       Тук разглеждаме $xy^2uv^2w$ и съобразяваме, че редът на буквите е нарушен.
     \end{itemize}    
   \end{description}
\end{proof}

\begin{problem}
  Докажете, че езикът 
  \[L = \{\ \alpha\alpha\mid \alpha\in \{a,b\}^\star\ \}\]
  не е безконтекстен.
\end{problem}
\begin{hint}
  \begin{itemize}
  \item 
    Защо $\omega = a^pba^pb$ не става ?
  \item
    Защо $\omega = a^pb^{2p}a^p$ не става ?
  \item
    Разгледайте $\omega = a^pb^pa^pb^p$.
  \end{itemize}
\end{hint}

\begin{framed}
  \begin{prop}
    Безконтекстните езици {\bf не} са затворени относно сечение и допълнение.
  \end{prop}
\end{framed}
\begin{hint}
  Да разгледаме езика
  \[L_0 = \{a^nb^nc^n\mid n\in\Nat\},\] за който вече знаем от \Prob{anbncn}, че не е безконтекстен.
  Да вземем също така и безконтекстните езици 
  \marginpar{\writedown Защо са безконтекстни?}
  \[L_1 = \{a^nb^nc^m\mid n,m\in\Nat\},\ L_2 = \{a^mb^nc^n\mid n,m\in\Nat\},\]
  \begin{itemize}
  \item 
    Понеже $L_0 = L_1\cap L_2$, то заключаваме, че безконтекстните езици не са затворени 
    относно операцията сечение.
  \item
    \marginpar{Озн. $\ov{L} = \Sigma^\star \setminus L$}
    Да допуснем, че безконтекстните езици са затворени относно операцията допълнение.
    Тогава  $\ov{L}_1$ и $\ov{L}_2$ са безконтекстни.
    Знаем, че безконтекстните езици са затворени относно обединение. 
    Следователно, езикът $L_3 = \ov{L}_1 \cup \ov{L}_2$ също е безконтекстен.
    Понеже допуснахме, че безконтекстните са затворени относно допълнение, то $\ov{L}_3$ също е безконтекстен.
    Но тогава получаваме, че езикът
    \[L_0 = L_1 \cap L_2 = \ov{\ov{L}_1 \cup \ov{L}_2} = \ov{L}_3\]
    е безконтекстен, което е противоречие.
  \end{itemize}

  Друг пример, с който може да се види, че безконтекстните езици не са затворени относно допълнение е 
  като се докаже, че езикът
  \[\{a,b\}^\star \setminus \{\alpha\alpha\mid \alpha\in \{a,b\}^\star\}\]
  е безконтекстен.
  Това следва лесно като се използва \Prob{equal-but-different}.
\end{hint}

\begin{problem}
  Докажете, че езикът 
  \[L = \{\alpha\beta\alpha^{rev} \mid \alpha,\beta \in \{a,b\}^\star\ \&\ |\alpha| = |\beta|\}\]
  не е безконтекстен.
\end{problem}
\begin{hint}
  \begin{itemize}
  \item
    Защо не става ако разгледаме думата $\alpha = a^pb^pa^p$ ?
  \item 
    Защо не става ако разгледаме думата $\alpha = a^p b^p a^{2p} b^p a^p$ ?
  \item
    Разгледайте $\alpha = a^p b^p a^p b^p b^p a^p$.
    Покачване с повече от $p$ би трябвало да свърши работа.
  \end{itemize}
\end{hint}


\begin{problem}
  Докажете, че езикът 
  \[L = \{\alpha\beta\alpha \mid \alpha,\beta \in \{a,b\}^\star\}\]
  не е безконтекстен.
\end{problem}
\begin{hint}
  \begin{itemize}
  \item 
    Защо не става с $\omega = a^pba^pb$ ?
  \item
    Защо не става с $\omega = ab^pab^p$ ?
  \item
    Пробвайте с $\omega = a^pb^pa^pb^p$.
  \end{itemize}
\end{hint}

% \begin{framed}
%   \begin{prop}
%     Безконтекстните езици {\bf не} са затворени относно сечение и допълнение.
%   \end{prop}
% \end{framed}
% \begin{proof}
%   Да разгледаме езика
%   \[L_0 = \{a^nb^nc^n\mid n\in\Nat\},\] за който вече знаем от \Prob{anbncn}, че не е безконтекстен.
%   Да вземем също така и безконтекстните езици 
%   \marginpar{\writedown Защо са безконтекстни?}
%   \[L_1 = \{a^nb^nc^m\mid n,m\in\Nat\},\ L_2 = \{a^mb^nc^n\mid n,m\in\Nat\},\]
%   \begin{itemize}
%   \item 
%     Понеже $L_0 = L_1\cap L_2$, то заключаваме, че безконтекстните езици не са затворени 
%     относно операцията сечение.
%   \item
%     \marginpar{Озн. $\ov{L} = \Sigma^\star \setminus L$}
%     Да допуснем, че безконтекстните езици са затворени относно операцията допълнение.
%     Тогава  $\ov{L}_1$ и $\ov{L}_2$ са безконтекстни.
%     Знаем, че безконтекстните езици са затворени относно обединение. 
%     Следователно, езикът $L_3 = \ov{L}_1 \cup \ov{L}_2$ също е безконтекстен.
%     Ние допуснахме, че безконтекстните са затворени относно допълнение, следователно $\ov{L}_3$
%     също е безконтекстен.
%     Но тогава получаваме, че езикът
%     \[L_0 = L_1 \cap L_2 = \ov{\ov{L}_1 \cup \ov{L}_2} = \ov{L}_3\]
%     е безконтекстен, което е противоречие.
%   \end{itemize}
% \end{proof}


\begin{problem}
  Докажете, че езикът
  \[L = \{\alpha\sharp\beta \mid \alpha\text{ е подниз на }\beta\}\]
  не е безконтекстен.
\end{problem}
\begin{hint}
  \begin{itemize}
  \item 
    Защо не става ако вземем $\omega = a^p \sharp a^p$ ?
  \item 
    Защо не става ако вземем $\omega = a^pb \sharp a^pb$ ?
  \item
    Разгледайте $\omega = a^pb^p\sharp a^pb^p$.
  \end{itemize}
\end{hint}


\begin{problem}
  Вярно ли е, че следните езици са безконтекстни:
  \begin{enumerate}[a)]
  \item 
    $L = \{\alpha\sharp\beta \mid \alpha,\beta \in \{0,1\}^\star\ \&\ \ov{\alpha}_{(2)} + 1 = \ov{\beta}_{(2)} \}$;
  \item
    $L = \{\alpha\sharp\beta^{rev} \mid \alpha,\beta \in \{0,1\}^\star\ \&\ \ov{\alpha}_{(2)} + 1 = \ov{\beta}_{(2)} \}$ ?
  \end{enumerate}
\end{problem}


%%% Local Variables: 
%%% mode: latex
%%% TeX-master: "../eai"
%%% End: 

\newpage
\section{Недетерминирани стекови автомати}

\index{автомат!недетерминиран стеков}
\mynote{На англ. {\em Push-down automaton}. В този курс няма да разглеждаме детерминирани стекови автомати. Когато кажем стеков автомат, ще имаме предвид недетерминиран стеков автомат.
  Означаваме $\Sigma_\varepsilon \df \Sigma \cup \{\varepsilon\}$ и $\Gamma^{\leq 2} \df \{\varepsilon\} \cup \Gamma \cup \Gamma^2$.}
{\bf Недетерминиран стеков автомат} е седморка от вида
\[P = \PDA,\] където:
\begin{itemize}
\item
  $Q$ е крайно множество от състояния;
\item  
  $\Sigma$ е крайна входна азбука;
\item
  $\Gamma$ е крайна стекова азбука;
\item
  $\sharp \in \Gamma$ е символ за дъно на стека;
\item
  \mynote{Дефиницията на стеков автомат има много вариации, всички еквивалентни помежду си.}
  $\Delta:Q\times\Sigma_\varepsilon\times \Gamma \rightarrow \Ps(Q\times\Gamma^{\leq 2})$ 
  е функция на преходите;    
\item
  $\qstart \in Q$ е начално състояние;
\item
  $\qaccept \in Q$ е заключителното състояние.
\end{itemize}

\mynote{На англ. Instanteneous description}
{\em Моментно описание} (или конфигурация) на изчислението със стеков автомат представлява тройка от вида $(q,\alpha,\gamma) \in Q\times\Sigma^\star\times\Gamma^\star$,
т.е. автоматът се намира в състояние $q$, думата, която остава да се прочете е $\alpha$,
а съдържанието на стека е думата $\gamma$.
Удобно е да въведем бинарната релация $\vdash_P$ над $Q\times\Sigma^\star\times\Gamma^\star$,
която ще ни казва как моментното описание на автомата $P$ се променя след изпълнение на една стъпка:

\begin{figure}[H]
  \begin{subfigure}[b]{0.5\textwidth}
    \begin{prooftree}
      \AxiomC{$\Delta(q,b,A) \ni (p,\beta)$}
      \AxiomC{$b \in \Sigma$}
      \BinaryInfC{$(q,b\alpha,A\gamma) \vdash_P (p,\alpha,\beta\gamma)$}
    \end{prooftree}
  \end{subfigure}
  ~
  \begin{subfigure}[b]{0.5\textwidth}
    \begin{prooftree}
      \AxiomC{$\Delta(q,\varepsilon,A) \ni (p,\beta)$}
      \UnaryInfC{$(q,\alpha,A\gamma) \vdash_P (p,\alpha,\beta\gamma)$}
    \end{prooftree}
  \end{subfigure}
\end{figure}

% \begin{align*}
%   (p,\varepsilon) \in \Delta(q,x,A) & \implies (q,x\alpha,A\gamma) \vdash_P (p,\alpha,\gamma)\\
%   (p,\beta) \in \Delta(q,\varepsilon,Y) & \implies (q,\alpha,Y\gamma) \vdash_P (p,\alpha,\beta\gamma).
% \end{align*}
Рефлексивното и транзитивно затваряне на $\vdash_P$ ще означаваме с $\vdash^\star_P$.
Сега вече можем да дадем дефиниция на език, разпознаван от стеков автомат $P$.
\mynote{Възможно е да се даде и друга еквивалентна дефиниция - разпознаване с празен стек.}
Езикът $\L(P)$, който се разпознава от $P$, има следната дефиниция:
\[\L(P) = \{\ \omega \in \Sigma^\star \mid (\qstart,\omega,\sharp) \vdash^\star_P (\qaccept,\varepsilon,\varepsilon)\ \}.\]

\subsection{Примери}

\begin{extra}
\begin{example}
  \label{ex:anbn}
  За езика $L = \{a^nb^n\mid n\in\Nat\}$, да разгледаме $P = \PDA$, където
  \begin{itemize}
  \item
    $Q \df \{q,p,f\}$;
  \item
    $\qstart \df q$ и $\qaccept \df f$;
  \item
    $\Sigma \df \{a,b\}$ и $\Gamma \df \{\sharp,a\}$;
  \item
    \mynote{Тук получаваме детерминистичен стеков автомат.}
    Релацията на преходите $\Delta$ има следната дефиниция:
    \begin{enumerate}[(1)]
    \item
      $\Delta(q,a,\sharp) \df \{(q, a\sharp)\}$;
    \item
      $\Delta(q,a,a) \df \{(q, aa)\}$; \quad \comment{трупаме $a$-та в стека}
    \item 
      $\Delta(q,\varepsilon,\sharp) \df \{(f,\varepsilon)\}$;\quad \comment{трябва да разпознаем и думата $\varepsilon$}
    \item 
      $\Delta(q, b, a) \df \{(p,\varepsilon)\}$; \quad \comment{Започваме да четем само $b$-та}
    \item 
      $\Delta(p, b, a) \df \{(p,\varepsilon)\}$; \quad \comment{Чистим $a$-тата от стека}
    \item
      $\Delta(p, \varepsilon, \sharp) \df \{(f, \varepsilon)\}$.
    \item
      За всички останали тройки $(r,x,y)$, нека $\Delta(r,x,y) \df \emptyset$.
    \end{enumerate}
  \end{itemize}
  
  Да видим как думата $a^2b^2$ се разпознава от стековия автомат $P$:
  \begin{align*}
    (q, a^2b^2, \sharp) & \vdash_P (q, ab^2, a\sharp) & \comment{\text{правило }(1)}\\
                        & \vdash_P (q, b^2, aa\sharp) & \comment{\text{правило }(2)}\\
                        & \vdash_P (p, b, a\sharp) & \comment{\text{правило }(4)}\\
                        & \vdash_P (p, \varepsilon, \sharp) & \comment{\text{правило }(5)}\\
                        & \vdash_P (f, \varepsilon, \varepsilon) & \comment{\text{правило }(6)}
  \end{align*}
  % \mynote{\writedown Докажете, че $L = \L(P)$!}
  Получихме, че $(\qstart, a^2b^2, \sharp) \vdash^\star_P (\qaccept, \varepsilon, \varepsilon)$, откъдето следва, че $a^2b^2 \in \L(P)$.

  \begin{enumerate}[a)]
  \item
    Докажете с индукция по $n$, че за всяко естествено число $n$ са изпълнени свойствата:
    \begin{align}
      & (q, a^n\beta, \sharp) \vdash^n_P (q, \beta, a^n\sharp) \label{eq:anbn:1}\\
      & (p, b^n, a^n\sharp) \vdash^n_P (p, \varepsilon,\sharp). \label{eq:anbn:2}
    \end{align}
    Заключете, че $L \subseteq \L(P)$.
  \item
    Докажете, че с индукция по $n$, че за всяко естествено число $n$ са изпълнени свойствата:
    \begin{align}
      (q, \alpha\beta, \sharp) \vdash^n_P (q, \beta, \gamma\sharp)  & \implies \alpha = \gamma = a^n \label{eq:anbn:3}\\
      (p, \beta, \gamma\sharp) \vdash^n_P (p, \varepsilon, \sharp) & \implies \beta = b^n\ \&\ \gamma = a^n. \label{eq:anbn:4}
    \end{align}
    Оттук заключете, че $\L(P) \subseteq L$.    
  \end{enumerate}
\end{example}

\begin{example}
  \label{ex:omega-omega-r}
  Езикът $L = \{\ \omega\omega^{\rev} \mid \omega \in \{a,b\}^\star\ \}$ се разпознава от стеков автомат
  \[P = \PDA,\] където:
  \begin{itemize}
  \item 
    $Q \df \{q,p,f\}$ и $\qstart \df q$, $\qaccept \df f$;
  \item
    $\Sigma \df \{a,b\}$, $\Gamma \df \{a, b, \sharp\}$;
  \item
    Функцията на преходите $\Delta$ има следната дефиниция:
    \mynote{За всички липсващи твойки  в дефиницията на $\Delta$ приемаме, че $\Delta$ връща $\emptyset$}
    \begin{enumerate}[(1)]
    \item
     $\Delta(q, x, \sharp) \df \{(q, x\sharp)\}$, където $x \in \{a,b\}$;
    % \item 
    %   $\Delta(q, a, \sharp) \df \{(q, a\sharp)\}$;
    % \item 
    %   $\Delta(q, b, \sharp) \df \{(q, b\sharp)\}$;
    \item
      $\Delta(q, \varepsilon, \sharp) \df \{(q,\varepsilon)\}$;
    \item
      $\Delta(q, x, x) \df \{(q, xx), (p, \varepsilon)\}$, където $x \in \{a,b\}$;
    \item
      $\Delta(q, a, b) \df \{(q, ab)\}$;
    \item
      $\Delta(q, b, a) \df \{(q, ba)\}$;
    % \item
      % $\Delta(q, b, b) \df \{(q, bb), (p, \varepsilon)\}$;
    \item
      $\Delta(p, x, x) \df \{(p,\varepsilon)\}$, където $x \in \{a,b\}$;
    % \item
    %   $\Delta(p, b, b) \df \{(p,\varepsilon)\}$;
    \item
      $\Delta(p, \varepsilon, \sharp) \df \{(f,\varepsilon)\}$;
    \end{enumerate}
  \end{itemize}
  Основното наблюдение, което трябва да направим за да разберем конструкцията на автомата е, че
  всяка дума от вида $\omega\omega^{\texttt{rev}}$ може да се запише като $\omega_1aa\omega^{\texttt{rev}}_1$ или $\omega_1bb\omega^{\texttt{rev}}_1$.
  Да видим защо $P$ разпознава думата $abaaba$ с празен стек.
  Започваме по следния начин:
  \begin{align*}
    (q, abaaba,\sharp) & \vdash_P (q, baaba, a\sharp)   & \comment{\text{правило }(1)}\\
                       & \vdash_P (q, aaba, ba\sharp)   & \comment{\text{правило }(5)}\\
                       & \vdash_P (q, aba,  aba\sharp). & \comment{\text{правило }(4)}
  \end{align*}
  Сега можем да направим два избора как да продължим. Състоянието $p$ служи за маркер, което ни казва, че вече сме започнали 
  да четем $\omega^{\texttt{rev}}$. Поради тази причина, продължаваме така:
  \begin{align*}
    (q, aba, aba\sharp) & \vdash_P (p, ba, ba\sharp) & \comment{\text{правило }(3)}\\
                        & \vdash_P (p, a, a\sharp) & \comment{\text{правило }(6)}\\
                        & \vdash_P (p, \varepsilon, \sharp) & \comment{\text{правило }(6)}\\
                        & \vdash_P (f,\varepsilon,\varepsilon). & \comment{\text{правило }(7)}
  \end{align*}
  Да проиграем още един пример. Да видим защо думата $aba$ не се извежда от автомата.
  \begin{align*}
    (q, aba, \sharp) & \vdash_P (q, ba, a\sharp) & \comment{\text{правило }(1)}\\
                     & \vdash_P (q, a, ba\sharp) & \comment{\text{правило }(5)}\\
                     & \vdash_P (q, \varepsilon, aba\sharp). & \comment{\text{правило }(4)}
  \end{align*}
  От последното моментно описание на автомата нямаме нито един преход, следователно
  думата $aba$ не се разпознава от $P$.
  \mynote{\writedown Докажете, че $L = \L(P)$!}
  \begin{enumerate}[a)]
  \item
    \mynote{За (\ref{eq:omega-omega-r:1}) приложете индукция по дължината на думата $\alpha$. За индукционната стъпка разгледайте $\alpha$ като $\alpha = \alpha' x$.}
    Докажете с индукция пo $n$, за всяко естествено число $n$ са изпълнени свойствата:
    \begin{align}
      & |\alpha| = n\ \implies\ (q, \alpha\beta, \sharp) \vdash^n_P (q, \beta, \alpha^{\texttt{rev}}\sharp) \label{eq:omega-omega-r:1}\\
      & |\beta| = n\ \implies\ (p, \beta, \beta\sharp) \vdash^n_P (p, \varepsilon, \sharp). \label{eq:omega-omega-r:2}
    \end{align}
    Оттук заключете, че $L \subseteq \L(P)$.
  \item
    Докажете с индукця по $n$, че за всяко естествено число $n$ са изпълнени свойствата:
    \begin{align}
      (q, \alpha\beta, \sharp) \vdash^n_P (q, \beta, \gamma\sharp) & \implies \gamma = \alpha^{\rev}\ \&\ |\alpha| = n \label{eq:omega-omega-r:3}\\
      (p, \beta, \gamma\sharp) \vdash^n_P (p, \varepsilon, \sharp) & \implies \gamma = \beta\ \&\ |\beta| = n. \label{eq:omega-omega-r:4}
    \end{align}
    Оттук заключете, че $\L(P) \subseteq L$.
  \end{enumerate}
\end{example}

% \begin{problem}
%   Постройте \emph{детерминистичен} стеков автомат за езика $L = \{\omega \$ \omega^{\rev} \mid \omega \in \{a,b\}^\star \}$.
% \end{problem}

\begin{example}
  \mynote{От \Problem{equal-number-parentheses} знаем, че този език е безконтекстен.}
  Езикът $L = \{\ \omega \in \{a,b\}^\star \mid \abs{\omega}_a = \abs{\omega}_b\}$
  се разпознава от стековия автомат $P = \PDA$, където:
  \begin{itemize}
  \item 
    $Q = \{q,f\}$;
  \item
    $\Sigma = \{a,b\}$;
  \item
    $\Gamma = \{a, b, \sharp\}$;
  \item
    $\qstart = q$ и $\qaccept = f$;
  \item
    Можем да дефинираме релацията на преходите $\Delta$ по следния начин:
    \begin{enumerate}[(1)]
    \item 
      $\Delta(q, \varepsilon, \sharp) = \{(f, \varepsilon)\}$;
    \item
      $\Delta(q, x, \sharp) = \{(q, x\sharp)\}$, където $x \in \{a,b\}$;
    \item
      $\Delta(q, x, x) = \{(q, xx)\}$, където $x \in \{a,b\}$;
    \item
      $\Delta(q, a, b) = \{(q, \varepsilon)\}$;
    \item
      $\Delta(q, b, a) = \{(q, \varepsilon)\}$.
    \end{enumerate}
  \end{itemize}
  Да видим защо думата $abbbaa \in \L(P)$.
  \begin{align*}
    (q, abbbaa, \sharp) & \vdash_P (q, bbbaa,\ a\sharp) & \comment{\text{правило }(2)}\\
                        & \vdash_P (q, bbaa,\ \sharp) & \comment{\text{правило }(5)}\\
                        & \vdash_P (q, baa,\ b\sharp) & \comment{\text{правило }(2)}\\
                        & \vdash_P (q, aa, bb\sharp) & \comment{\text{правило }(3)}\\
                        & \vdash_P (q, a,\ b\sharp) & \comment{\text{правило }(4)}\\
                        & \vdash_P (q, \varepsilon,\ \sharp) & \comment{\text{правило }(4)}\\
                        & \vdash_P (f, \varepsilon,\ \varepsilon). & \comment{\text{правило }(1)}
  \end{align*}

  \begin{enumerate}[a)]
  \item
    Докажете с пълна индукция по дължината на думата $\gamma$, че за произволна дума $\gamma \in \{a, b\}^\star$, е изпълнено, че:
    \begin{align}
      (\forall n)[a^n\gamma \in L & \implies (q, \gamma, a^n\sharp) \vdash^\star_P (q, \varepsilon, \sharp)] \label{eq:omega-ab:1}\\
      (\forall n)[b^n\gamma \in L & \implies (q, \gamma, b^n\sharp) \vdash^\star_P (q, \varepsilon, \sharp)]. \label{eq:omega-ab:2}
    \end{align}
    % Ще докажем едновременно \Property{eq:omega-ab:1} и \Property{eq:omega-ab:2} с пълна индукция по дължината на думата $\gamma$.

    % Да разгледаме произволна дума $\gamma$. Случаят, когато $\abs{\gamma} = 0$ е тривиален. 
    % Нека $\abs{\gamma} > 0$ и да приемем, че $a^n\gamma \in L$.
    % Ясно е, че можем да представим $\gamma$ като $\gamma = a^kb\gamma'$, за някое $k$.
    % \begin{itemize}
    % \item
    %   Ако $n+k = 0$, то $a^n\gamma = b\gamma' \in L$ и прилагаме \IndHyp за \Property{eq:omega-ab:2} с думата $\gamma'$ и получаваме, че:
    %   \begin{align*}
    %     (q, \overbrace{b\gamma'}^{\gamma},\sharp) & \vdash (q,\gamma',b\sharp) & \comment{\text{правило (2)}}\\
    %                          & \vdash^\star (q, \varepsilon,\sharp) & \comment\text{\IndHyp}
    %   \end{align*}
    % \item
    %   Ако $n+k>0$, то $a^{n+k-1}\gamma' \in L$ и прилагаме \IndHyp за \Property{eq:omega-ab:1} с думата $\gamma'$ и получаваме, че:
    %   \begin{align*}
    %     (q, \overbrace{a^{k}b\gamma'}^{\gamma}, a^n\sharp) & \vdash^\star_P (q, b\gamma', a^{n+k}\sharp) & \comment{\text{правило (3)}}\\
    %                                                         & \vdash^\star_P (q, \gamma', a^{n+k-1}\sharp) & \comment\text{правило (5)}\\
    %                                                         & \vdash^\star_P (q, \varepsilon, \sharp). & \comment\text{\IndHyp}
    %   \end{align*}
    % \end{itemize}
    % Аналогично се доказва и \Property{eq:omega-ab:2}.
    
    Оттук заключете, че $L \subseteq \L(P)$.
  \item
    Докажете с пълна индукция по дължината на думата $\gamma$, че за произволна дума $\gamma \in \{a, b\}^\star$ е изпълнено, че:
    \begin{align}
      (\forall n)[(q, \gamma, a^n\sharp) \vdash^\star_P (q, \varepsilon, \sharp) & \implies a^n\gamma \in L] \label{eq:omega-ab:3}\\
      (\forall n)[(q, \gamma, b^n\sharp) \vdash^\star_P (q, \varepsilon, \sharp) & \implies b^n\gamma \in L]. \label{eq:omega-ab:4}
    \end{align}

    % Нека имаме изчислението $(q, \gamma, a^n\sharp) \vdash^\star_P (q, \varepsilon, \sharp)$.
    % Да представим думата $\gamma$ като $\gamma = a^kb\gamma'$.
    
    % \begin{itemize}
    % \item
    %   Ако $n+k = 0$, то можем да разбием изчислението по следния начин:
    %   \begin{align*}
    %     (q, b\gamma', \sharp) & \vdash_P (q, \gamma',b\sharp) \\
    %                           & \vdash^{\star} (q,\varepsilon,\sharp).
    %   \end{align*}
    %   Тогава от \IndHyp за \Property{eq:omega-ab:4} следва, че $a^n\gamma = b\gamma' \in L$.
    % \item
    %   Ако $n+k > 0$, то можем да разбием изчислението по следния начин:
    %   \begin{align*}
    %     (q, a^kb\gamma', a^n\sharp) & \vdash^\star_P (q, b\gamma',a^{n+k}\sharp) \\
    %                                 & \vdash_P (q, \gamma',a^{n+k-1}\sharp) \\
    %                                 & \vdash^{\star} (q,\varepsilon,\sharp).
    %   \end{align*}
    %   Тогава от \IndHyp за \Property{eq:omega-ab:3} следва, че $a^{n+k-1}\gamma' \in L$, но оттук
    %   веднага получаваме, че $a^na^kb\gamma' \in L$.
    % \end{itemize}
    % Аналогично се доказва и \Property{eq:omega-ab:4}.
    
    Оттук заключете, че $\L(P) \subseteq L$.
  \end{enumerate}
\end{example}

\begin{example}
  \mynote{От \Problem{balanced-parentheses} знаем, че този език е безконтекстен.}
  Езикът $L = \{\ \omega \in \{a,b\}^\star \mid \omega\text{ е балансирана дума}\ \}$
  се разпознава от стековия автомат $P = \PDA$, където:
  \begin{itemize}
  \item 
    $Q = \{q,f\}$;
  \item
    $\qstart = q$ и $\qaccept = f$;
  \item
    $\Sigma = \{a,b\}$ и $\Gamma = \{a, \sharp\}$;
  \item
    Можем да дефинираме релацията на преходите $\Delta$ по следния начин:
    \mynote{\writedown Докажете, че $L = \L(P)$!}
    \begin{enumerate}[(1)]
    \item 
      $\Delta(q, \varepsilon, \sharp) = \{(f, \varepsilon)\}$;
    \item
      $\Delta(q, a, \sharp) = \{(q, a\sharp)\}$;
    \item
      $\Delta(q, a, a) = \{(q, aa)\}$;
    \item
      $\Delta(q, b, a) = \{(q, \varepsilon)\}$;
    \end{enumerate}
  \end{itemize}  
  \begin{enumerate}[(a)]
  \item
    \mynote{Индукция по дължината на думата $\beta$.}
    Докажете, че за произволно естествено число $n$ и произволна дума $\beta \in \{a, b\}^\star$, 
    е изпълнено, че:
    \[a^n\beta \in L\ \implies (q, \beta, a^n\sharp) \vdash^\star_P (q, \varepsilon, \sharp).\]
    Оттук заключете, че $L \subseteq \L(P)$.
  \item
    \mynote{Индукция по броя на стъпките в изчислението на стековия автомат.}
    Докажете, че за произволно естествено число $n$ и произволна дума $\beta \in \{a, b\}^\star$, е изпълнено, че:
    \[(q,\beta,a^n\sharp) \vdash^\star_P (q, \varepsilon, \sharp)\ \implies\ a^n\beta \in L.\]
    Оттук заключете, че $\L(P) \subseteq L$.
  \end{enumerate}
\end{example}
\end{extra}


%%% Local Variables:
%%% mode: latex
%%% TeX-master: "../eai"
%%% End:


\subsection{Теорема за еквивалентност}

\begin{important}
  \begin{lemma}
    \mynote{\cite[стр. 136]{papadimitriou}}
    За всяка безконтекстна граматика $G$,
    съществува стеков автомат $P$, такъв че $\L(G) = \L(P)$.
  \end{lemma}
\end{important}
\begin{proof}
  % \mynote{Доказателството в \cite[стр. 117]{sipser3} не ми харесва}
  \mynote{Тук е важно, че дефинирахме най-ляв извод в граматика.}
  Нека е дадена безконтекстната граматика $G = \CFG$ в нормална форма на Чомски.
  Нашата цел е да построим стеков автомат
  \[P = \PDA,\] който разпознава $\L(G)$.
  \begin{itemize}
  \item
    $Q = \{\qstart,p,\qaccept\}$;
  \item
    $\Gamma = \Sigma \cup V \cup \{\sharp\}$;
  \item
    Релацията на преходите $\Delta$ дефинираме по следния начин:
    \mynote{Понеже граматиката е в нормална форма на Чомски, то $|\alpha| \leq 2$ и удовлетворяваме дефиницията на $\Delta$.}
    \begin{enumerate}[(1)]
    \item 
      $\Delta(\qstart, \varepsilon, \sharp ) = \{(p,S\sharp)\}$;
    \item
      $\Delta(p,\varepsilon,A) = \{(p,\alpha)\mid A\to_G \alpha\}, \text{ за всяка променлива }A \in V$;
    \item
      $\Delta(p,a,a) = \{(p,\varepsilon)\}, \text{ за всяка буква } a \in \Sigma$;
    \item
      $\Delta(p,\varepsilon,\sharp) = \{(\qaccept, \varepsilon)\}$.
    \end{enumerate}
  \end{itemize}

  \mynote{Ако $\gamma$ не е празната дума, то $\gamma$ започва с променлива.}
  Ще докажем, че за всяка променлива $A \in V$, за всяка дума $\alpha \in \Sigma^\star$, $\gamma \in (V \cdot \Sigma^\star)^\star$ и $\delta \in (V \cup \Sigma)^\star$, то е изпълнено, че:
  \begin{enumerate}[(a)]
  \item
    ако $S \lderive{\star} \alpha \gamma$, то $(p, \alpha, S\sharp) \vdash^\star_P (p, \varepsilon, \gamma\sharp)$;
  \item
    ако $(p, \alpha, \delta\sharp) \vdash^\star_P (p, \varepsilon, \sharp)$, то $\delta \lderive{\star} \alpha$.
  \end{enumerate}
  Тогава, ако вземем $\delta = S$ и $\gamma = \varepsilon$, то ще получим, че
  \begin{align*}
    \alpha \in \L(G) & \iff S \lderive{\star} \alpha\\
                     & \iff (p,\alpha,S\sharp) \vdash^\star_P (p, \varepsilon, \sharp) & \comment{\text{от (а) и (б)}}\\
                     & \iff (\qstart,\alpha,\sharp) \vdash^\star_P (\qaccept, \varepsilon, \varepsilon) & \comment{\text{от деф. на }P}\\
                     & \iff \alpha \in \L(P).
  \end{align*}

  Сега преминаваме към доказателствата на двете твърдения.

  % \begin{enumerate}[(a)]
  % \item
  За (а), индукция по дължината $\ell$ на извода $S \lderive{\ell} \alpha\gamma$.
    % \begin{itemize}
    % \item

  Нека $\ell = 0$. Този случай е тривиален, защото тогава $\alpha = \varepsilon$ и $\gamma = S$.
  Ясно е, че $(p,\varepsilon,S\sharp) \vdash^0_P (p,\varepsilon,S\sharp)$.
      % \item
      
      Нека $\ell > 0$ и $S \lderive{\ell} \alpha\gamma$. Това означава, че този извод може да се запише по следния начин:
      \begin{prooftree}
        \AxiomC{$S\lderive{\ell-1} \alpha_1A\gamma_2$}
        \AxiomC{$\alpha_1 \in \Sigma^\star$}
        \AxiomC{$A \to_G \alpha_2\gamma_1$}
        \RightLabel{\scriptsize{(\Proposition{grammar:context-left-step})}}
        \TrinaryInfC{$S \lderive{\ell} \underbrace{\alpha_1\alpha_2}_{\alpha}\underbrace{\gamma_1\gamma_2}_{\gamma}$}
      \end{prooftree}
      Тогава от \IndHyp имаме, че
      \begin{equation}
        \label{eq:5}
        (p, \alpha_1, S\sharp) \vdash^\star_P (p, \varepsilon, A\gamma_2\sharp).
      \end{equation}
      Получаваме следното изчисление на стековия автомат:
      \begin{align*}
        (p, \overbrace{\alpha_1\alpha_2}^{\alpha}, S\sharp) & \vdash^\star_P (p, \alpha_2, A\gamma_2\sharp) & \comment{\text{от (\ref{eq:5})}}\\
                                                            & \vdash_P (p, \alpha_2, \alpha_2\gamma_1\gamma_2\sharp) & \comment{\text{ред (2) от деф. на }\Delta}\\
                                                            & \vdash^\star_P (p, \varepsilon, \underbrace{\gamma_1\gamma_2}_{\gamma}\sharp) & \comment{\text{ред (3) от деф. на }\Delta}.
      \end{align*}
    % \end{itemize}
  За (б), индукция по броя на стъпките $\ell$ в изчислението на стековия автомат.
    % \begin{itemize}
    % \item
    
      Нека $\ell = 0$. Тогава е ясно, че единствената възможност е $\alpha = \varepsilon$ и $\delta = \varepsilon$.
      Тогава $\varepsilon \derive{\star}_G \varepsilon$.
      % \item
      
      Нека $\ell > 0$ и $(p, \alpha, \delta \sharp) \vdash^{\ell}_P (p, \varepsilon, \sharp)$.      
      Имаме три избора за първата стъпка в това изчисление.
      
      Започваме със случая, когато $\Delta(p,a,a) \ni (p,\varepsilon)$. Това означава, че със сигурност можем да представим думите $\alpha$ и $\delta$ като
      $\alpha = a\beta$ и $\delta = a\rho$, за някои $\beta$ и $\rho$, и да разбием изчислението по следния начин:
      \[(p, \underbrace{a\beta}_{\alpha}, \underbrace{a\rho}_{\delta} \sharp) \vdash_P (p,\beta,\rho\sharp ) \vdash^{\ell-1}_P (p, \varepsilon, \sharp).\]
      Сега можем да приложим индукционното предположение и да получим извода:
      \begin{prooftree}
        \AxiomC{$a \in \Sigma$}
        \AxiomC{$(p,\beta,\rho\sharp ) \vdash^{\ell-1}_P (p, \varepsilon, \sharp)$}
        \RightLabel{\scriptsize{\IndHyp}}
        \UnaryInfC{$\rho \lderive{\star} \beta$}
        \RightLabel{\scriptsize{(\Proposition{left-derivation:padding})}}
        \BinaryInfC{$\underbrace{a \rho}_{\delta} \lderive{\star} \underbrace{a\beta}_{\alpha}$}
      \end{prooftree}
    % \item
      Вторият случай е за $\Delta(p,\varepsilon,A) \ni (p,a)$. Според конструкцията на стековия автомат, това означава, че със сигурност имаме правилото $A \to_G a$, а думата $\delta$
      може да се представи като $\delta = A\rho$, за някое $\rho$, и изчислението може да се разбие по следния начин:
      \[(p, \alpha, \underbrace{A\rho}_{\delta} \sharp) \vdash_P (p,\alpha,a\rho\sharp ) \vdash^{\ell-1}_P (p, \varepsilon, \sharp).\]
      Сега прилагаме индукционното предположение и получаваме извода:
      \begin{prooftree}
        \AxiomC{$A \to_G a$}
        \AxiomC{}
        \LeftLabel{\scriptsize{(0)}}
        \UnaryInfC{$a \lderive{0} a$}
        \LeftLabel{\scriptsize{(1)}}
        \BinaryInfC{$A \lderive{1} a$}
        \AxiomC{$\rho \in (V\cup\Sigma)^\star$}
        \LeftLabel{\scriptsize{(\Proposition{left-derivation:padding})}}
        \BinaryInfC{$A\rho \lderive{1} a\rho$}
        \AxiomC{$(p,\alpha,a\rho\sharp ) \vdash^{\ell-1}_P (p, \varepsilon, \sharp)$}
        \RightLabel{\scriptsize{\IndHyp}}
        \UnaryInfC{$a\rho \lderive{\star} \alpha$}
        \LeftLabel{\scriptsize{(\Proposition{left-derivation:context-step})}}
        \BinaryInfC{$\underbrace{A\rho}_{\delta} \lderive{\star} \alpha$}
      \end{prooftree}
    % \item
      Последният случай е ако $\Delta(p,\varepsilon,A) \ni (p,BC)$. Според конструкцията на стековия автомат, това означава, че имаме правилото $A \to_G BC$,
      думата $\delta$ може да се представи като $\delta = A\rho$ и изчислението може да се разбие по следния начин:
      \[(p, \alpha, \underbrace{A\rho}_{\delta} \sharp) \vdash_P (p,\alpha, BC\rho\sharp ) \vdash^{\ell-1}_P (p, \varepsilon, \sharp).\]
      Сега прилагаме индукционното предположение и получаваме извода:
      \begin{prooftree}
        \AxiomC{$A \to_G BC$}
        \AxiomC{}
        \RightLabel{\scriptsize{(0)}}
        \UnaryInfC{$BC \lderive{0} BC$}
        \LeftLabel{\scriptsize{(1)}}
        \BinaryInfC{$A \lderive{1} BC$}
        \AxiomC{$\rho \in (V\cup\Sigma)^\star$}
        \LeftLabel{\scriptsize{(\Proposition{left-derivation:padding})}}
        \BinaryInfC{$A\rho \lderive{1} BC\rho$}
        \AxiomC{$(p,\alpha, BC\rho\sharp ) \vdash^{\ell-1}_P (p, \varepsilon, \sharp)$}
        \RightLabel{\scriptsize{\IndHyp}}
        \UnaryInfC{$BC\rho \lderive{\star} \alpha$}
        \LeftLabel{\scriptsize{(\Proposition{left-derivation:context-step})}}
        \BinaryInfC{$\underbrace{A\rho}_{\delta} \lderive{\star} \alpha$}
      \end{prooftree}
    % \end{itemize}
  % \end{itemize}
% \end{enumerate}
\end{proof}

\begin{important}
  \begin{lemma}
    За всеки стеков автомат $P$, съществува безконтекстна граматика $G$, такава че $\L(P) = \L(G)$.
  \end{lemma}
\end{important}
\begin{proof}
  Нека е даден стековия автомат
  \[P = \PDA.\]
  \mynote{Граматиката, която ще получим няма да бъде в нормална форма на Чомски.}
  Ще дефинираме безконтекстна граматика $G$, за която $\L(P) = \L(G)$.
  Променливите на граматика са 
  \[V = \{[q,A,p] \mid q,p \in Q\ \&\ A \in \Gamma\}.\]
  Правилата на $G$ са следните:
  \begin{itemize}
  \item
    Началната променлива е $S \df [\qstart,\sharp,\qaccept]$;
  \item
    Нека имаме $(r,BC) \in \Delta(q, a, A)$, където $a \in \Sigma_\varepsilon$.
    Тогава добавяме правилата:
    \[[q,A,p] \to_G a[r,B,q'][q',C,p],\]
    за всеки две състояния $q'$ и $p$.
  \item
    Нека имаме $(r,B) \in \Delta(q, a, A)$, където $a \in \Sigma_\varepsilon$.
    Тогава добавяме правилата:
    \[[q,A,p] \to_G a[r,B,p],\]
    за всяко състояние $p \in Q$.
  \item
    Нека имаме $(p,\varepsilon) \in \Delta(q,a,A)$, където $a \in \Sigma_\varepsilon$.
    Тогава добавяме правилата:
    \[[q,A,p] \to_G a.\]
  \end{itemize}
  След като вече сме обяснили какви правила включва граматиката $G$,
  трябва да докажем, че за произволна дума $\alpha \in \Sigma^\star$, произволни състояния $q$ и $p$,
  и произволен символ $A \in \Gamma$, е изпълнено, че:
  \[[q,A,p] \lderive{\star} \alpha\ \text{точно тогава, когато}\ (q,\alpha,A) \vdash^\star_{P} (p,\varepsilon,\varepsilon).\]
  \begin{description}
  \item[$(\Rightarrow)$]
    С пълна индукция по броят на стъпките $\ell$ в изчислението на стековия автомат $P$ ще докажем, че за произволно $\ell \geq 1$,
    \[\text{ако }(q,\alpha,A) \vdash^\ell_P (p,\varepsilon,\varepsilon)\text{, то } [q,A,p] \lderive{\star} \alpha.\]
    Ако $\ell = 1$, то е лесно, защото $\alpha = a \in \Sigma_\varepsilon$.
    Тогава $(p,\varepsilon) \in \Delta(q,a,A)$ и според конструкцията на граматиката $G$ имаме правилото $[q,A,p] \to_G a$.
    
    Ако $\ell > 1$, то в зависимост от първата стъпка на изчислението, имаме два случая.
    Нека $\alpha = a\beta$, където $a \in \Sigma_\varepsilon$.
    \begin{itemize}
    \item 
      Ако $\Delta(q,a,A) \ni (r,B)$, то можем да разбием изчислението по следния начин:
      \[(q,a\beta,A) \vdash_P (r,\beta,B) \vdash^{\ell-1}_P (p, \varepsilon, \varepsilon).\]
      Сега можем да приложим индукционното предположение и да получим извода:
      \begin{prooftree}
        \AxiomC{$\Delta(q,a,A) \ni (r,B)$}
        \LeftLabel{\scriptsize{(деф.)}}
        \UnaryInfC{$[q,A,p] \to_G a[r,B,p]$}
        \AxiomC{$(r,\beta,B) \vdash^{\ell-1}_P (p, \varepsilon, \varepsilon)$}
        \RightLabel{\scriptsize{\IndHyp}}
        \UnaryInfC{$[r,B,p] \lderive{\star} \beta$}
        \AxiomC{$a \in \Sigma$}
        \RightLabel{\scriptsize{(\Proposition{left-derivation:padding})}}
        \BinaryInfC{$a[r,B,p] \lderive{\star} a\beta$}
        \RightLabel{\scriptsize{(1)}}
        \BinaryInfC{$[q,A,p] \lderive{\star} \underbrace{a\beta}_{\alpha}$}
      \end{prooftree}
    \item
      Ако $\Delta(q, a, A) \ni (r, BC)$, то можем да разбием изчислението по следния начин:
      \[(q, a\beta, A) \vdash_P (r, \beta, BC) \vdash^{\ell-1}_P (p, \varepsilon, \varepsilon).\]      
      За $\ell-1$ стъпки трябва да стигнем от стек с големина $2$ до празен стек.
      Това означава, че можем да разбием думата $\beta$ на две части, $\beta = \beta_1\beta_2$, със свойството, че след като прочетем $\beta_1$,
      то стекът има големина $1$ и след като прочетем $\beta_2$, то стекът е празен.
      \mynote{Да обърнем внимание, че в междинните стъпки от двете изчисления, стекът може да расте.}
      Това означава, че съществува състояние $q'$, за което можем да разбием изчислението на две части по следния начин:
      \begin{align*}
        & (r, \beta_1, B) \vdash^{\ell_1}_P (q',\varepsilon,\varepsilon)\\
        & (q', \beta_2, C) \vdash^{\ell_2}_P (p,\varepsilon,\varepsilon),\\
        & \ell_1 + \ell_2 = \ell - 1.
      \end{align*}
      Понеже $\ell_1 < \ell$ и $\ell_2 < \ell$, от \IndHyp имаме следното:
      \begin{align*}
        & (r, \beta_1, B) \vdash^{\ell_1}_P (q', \varepsilon, \varepsilon) \implies [r, B, q'] \lderive{\star} \beta_1\\
        & (q', \beta_2, C) \vdash^{\ell_2}_P (p, \varepsilon, \varepsilon) \implies [q', C, p] \lderive{\star} \beta_2.
      \end{align*}
      Понеже имаме, че $\Delta(q,a,A) \ni (r,BC)$, то според дефиницията на стековия автомат, в граматиката имаме правилото
      \[[q,A,p] \to_G a[r,B,q'][q',C,p].\]
      Обединявайки всичко, получаваме извода
      \begin{prooftree}
        \AxiomC{$\Delta(q,a,A) \ni (r,BC)$}
        \RightLabel{\scriptsize{(деф.)}}
        \UnaryInfC{$[q,A,p] \to_G a[r,B,q'][q',C,p]$}
        \AxiomC{$(r, \beta_1, B) \vdash^{\ell_1}_P (q', \varepsilon, \varepsilon)$}
        \LeftLabel{\scriptsize{\IndHyp}}
        \UnaryInfC{$[r, B, q'] \lderive{\star} \beta_1$}
        \AxiomC{$(q', \beta_2, C) \vdash^{\ell_2}_P (p, \varepsilon, \varepsilon)$}
        \LeftLabel{\scriptsize{\IndHyp}}
        \UnaryInfC{$[q', C, p] \lderive{\star} \beta_2$}
        \LeftLabel{\scriptsize{(\ref{pr:left-derivation:concat2})}}
        \BinaryInfC{$[r,B,q'][q',C,p] \lderive{\star} \beta_1\beta_2$}
        \LeftLabel{\scriptsize{(\ref{pr:left-derivation:padding})}}
        \UnaryInfC{$a[r,B,q'][q',C,p] \lderive{\star} a\beta_1\beta_2$}
        \BinaryInfC{$[q,A,p] \lderive{\star} \underbrace{a\beta_1\beta_2}_{\alpha}$}
      \end{prooftree}
    \end{itemize}
  \item[$(\Leftarrow)$]
    За тази посока, с пълна индукция по дължината на извода $\ell$ в граматиката $G$ ще докажем, че за произволна дължина на извода $\ell \geq 1$,
    \[\text{ако }[q,A,p] \lderive{\ell} \alpha\text{, то }(q,\alpha,A) \vdash^\star_P (p,\varepsilon,\varepsilon).\]
    \begin{itemize}
    \item
      Нека $\ell = 1$. Тогава $\alpha = a \in \Sigma_\varepsilon$ и $[q,A,p] \to_G a$.
      Според дефиницията на граматиката, правилото $[q,A,p] \to_G a$ е добавено към граматиката, защото в стековият автомат имаме $\Delta(q,a,A) \ni (p,\varepsilon)$
    \item
      Нека $\ell > 1$. Тогава думата $\alpha$ може да се представи като $\alpha = a \beta$, където $a \in \Sigma_\varepsilon$, и според правилата на граматиката $G$ имаме два случая.
      \mynote{Тук отново е възможно $a = \varepsilon$. Това не е проблем, защото правим индукция по дължината на извода, а не по дължината на думата $\alpha$.}
      Да приемем, че имаме следния извод:
      \begin{prooftree}
        \AxiomC{$[q,A,p] \to_G a[r,B,p]$}
        \AxiomC{$[r,B,p] \lderive{\ell-1} \beta$}
        \LeftLabel{\scriptsize{правило (1)}}
        \BinaryInfC{$[q,A,p] \lderive{\ell} \underbrace{a\beta}_{\alpha}$}
      \end{prooftree}
      Тогава директно прилагаме \IndHyp и получаваме, че
      $(r, \beta, B) \vdash^\star_P (p, \varepsilon, \varepsilon)$ и накрая получаваме, че $(q, a\beta, A) \vdash^\star_P (p, \varepsilon, \varepsilon)$.
      
      Сега да разгледаме втория случай:
      \begin{prooftree}
        \AxiomC{$[q,A,p] \to_G a[r,B,q'][q',C,p]$}
        \AxiomC{$[r,B,q'][q',C,p] \lderive{\ell-1} \beta$}
        \BinaryInfC{$[q,A,p] \lderive{\ell} \underbrace{a\beta}_{\alpha}$}
      \end{prooftree}
      \mynote{Важно е, че $\beta \in \Sigma^\star$. Иначе няма да можем да приложим \Proposition{grammar:divide-2}.}
      Понеже $\beta \in \Sigma^\star$, можем да приложим \Proposition{grammar:divide-2} за $\lderive{\star}$ и оттам следва,
      че имаме разбиване на думата $\beta$ като $\beta = \beta_1\beta_2$, където 
      \begin{align*}
        & [r,B,q'] \lderive{\ell_1} \beta_1\\
        & [q',C,p] \lderive{\ell_2} \beta_2,\\
        & \ell_1 + \ell_2 = \ell - 1.
      \end{align*}
      Понеже $\ell_1 < \ell$ и $\ell_2 < \ell$, от \IndHyp получаваме, че 
      \begin{align*}
        & [r,B,q'] \lderive{\ell_1} \beta_1 \implies (r,\beta_1,B) \vdash^\star_P (q',\varepsilon,\varepsilon) \\
        & [q',C,p] \lderive{\ell_2} \beta_2 \implies (q',\beta_2,C) \vdash^\star_P (p,\varepsilon,\varepsilon).
      \end{align*}
      Правилото $[q,A,p] \rightarrow_G a[r,B,q'][q',C,p]$ 
      е добавено в граматиката, защото $\Delta(q,a,A) \ni (r, BC)$. 
      Обединявайки всичко, което знаем, получаваме:
      \begin{align*}
        (q, a\beta, A) & \vdash_P (r, \beta_1\beta_2, BC)\\
                       & \vdash^\star_P (q', \beta_2, C)\\
                       & \vdash^\star_P (p, \varepsilon, \varepsilon).
      \end{align*}    
    \end{itemize}
  \end{description}
\end{proof}

Предишните две леми ни дават следната теорема.
\begin{important}
  \begin{theorem}
    \label{th:push-down-context-free}
    Класът на езиците, които се разпознават от недетерминиран стеков автомат съвпада с
    класа на безконтекстните езици.
  \end{theorem}
\end{important}

\begin{extra}
\begin{example}
  Нека е дадена граматиката $G$ с правила 
  \begin{align*}
    & S \to AS\ |\ BS\ |\ \varepsilon\\
    & A \to aA\ |\ a\\
    & B \to Bb\ |\ b.
  \end{align*}
  Ще построим стеков автомат $P = \PDA$, такъв че $\L(P) = \L(G)$.
  \begin{itemize}
  \item
    $\Sigma = \{a,b\}$;
  \item 
    $\Gamma = \{A,S,B,a,b,\sharp\}$;
  \item
    $Q = \{\qstart,q,\qaccept\}$;
  \item
    Дефинираме релацията на преходите, следвайки конструкцията от \Theorem{push-down-context-free}:
    \begin{itemize}
    \item
      $\Delta(\qstart, \varepsilon, \sharp) = \{(q, S\sharp)\}$;
    \item 
      $\Delta(q, \varepsilon, S) = \{(q, AS), (q, BS), (q, \varepsilon)\}$;
    \item
      $\Delta(q, \varepsilon, A) = \{(q, aA), (q, a)\}$;
    \item
      $\Delta(q, \varepsilon, B) = \{(q, Bb), (q, b)\}$;
    \item
      $\Delta(q, a, a) = \{(q, \varepsilon)\}$;
    \item
      $\Delta(q, b, b) = \{(q, \varepsilon)\}$;
    \item
      $\Delta(q, \varepsilon, \sharp) = \{(\qaccept,\varepsilon)\}$;
    \end{itemize}
  \end{itemize}
\end{example}
\end{extra}


От \Proposition{context-free:pumping:non-closure} знаем, че безконтекстните езици не
са затворени относно операцията сечение, т.е. възможно е $L_1$ и $L_2$ да са безконтекстни
езици, но $L_1 \cap L_2$ да не е безконтекстен.
Оказва се обаче, че безконтекстните езици са затворени относно сечение с регулярен език.

\begin{important}
  \begin{theorem}\label{th:intersection-context-reg}
    Нека $L$ e безконтекстен език и $R$ е регулярен език.
    Тогава тяхното сечение $L \cap R$ е безконтекстен език.
  \end{theorem}
\end{important}
\begin{hint}
  \mynote{\cite[стр. 144]{papadimitriou}}
  Нека имаме стеков автомат
  \[P = \pair{Q',\Sigma,\Gamma,\sharp, \Delta', \qstart', \qaccept'}, \text{ където } \L(P) = L,\]
  и детерминиран краен автомат 
  \[\A = \pair{Q'', \Sigma, \qstart'', \delta'', F''}, \text{ където } \L(\A) = R.\]
  \mynote{Сравнете с конструкцията от \Proposition{automata-cap}.}
  Ще определим нов стеков автомат $\M = \PDA$, където:
  \begin{itemize}
  \item 
    $Q \df Q' \times Q''$;
  \item
    $\qstart \df \pair{\qstart',\qstart''}$;
  \item
    $F \df \{\qaccept'\} \times F''$;
  \item 
    Функцията на преходите $\Delta$ е дефинирана както следва:
    \begin{itemize}
    \item 
      \mynote{Симулираме едновременно изчислението и на двата автомата.}
      Ако $(r_1,Z) \in \Delta'(q_1, a, Y)$, то
      \[(\pair{r_1,\delta''(q_2,a)}, Z) \in \Delta(\pair{q_1,q_2},a,Y).\]
    \item
      \mynote{Нищо не четем от входната дума, следователно правим празен ход на $\A$}
      Ако $(r_1,Z) \in \Delta'(q_1,\varepsilon,Y)$ и всяко $q_2 \in Q''$, то
      \[(\pair{r_1,q_2}, Z) \in \Delta(\pair{q_1,q_2},\varepsilon,Y).\]
    \item
      \mynote{\writedown Докажете, че $\L(\M) = \L(P) \cap \L(\A)$ !}
      $\Delta$ не съдържа други преходи;
    \end{itemize}
  \end{itemize}

  \begin{itemize}
  \item
    \mynote{Индукция по броя стъпки в изчислението на $\M$.}
    Докажете, че ако $(\pair{q_1,q_2},\alpha,\gamma) \vdash^\star_\M (\pair{p_1,p_2},\varepsilon,\varepsilon)$, то
    \[(q_1,\alpha,\gamma) \vdash^\star_P (p_1,\varepsilon,\varepsilon)\text{ и }(q_2,\alpha) \vdash^\star_\A (p_2,\varepsilon).\]
  \item
    \mynote{Индукция по броя стъпки в изчислението на $P$.}
    Докажете, че ако $(q_1,\alpha,\gamma) \vdash^\star_P (p_1,\varepsilon,\varepsilon)$ и $(q_2,\alpha) \vdash^\star_\A (p_2,\varepsilon)$, то
    \[(\pair{q_1,q_2},\alpha,\gamma) \vdash^\star_\M (\pair{p_1,p_2},\varepsilon,\varepsilon).\]
  \end{itemize}
  
\end{hint}

\Theorem{intersection-context-reg} е удобна, когато искаме да докажем, че даден език не е безконтекстен.
С нейна помощ можем да сведем езика до друг, за който вече знаем, че не е безконтекстен.

\begin{extra}
  \begin{example}
    Да разгледаме езика $L = \{\omega \in \{a,b,c\}^\star \mid \card{\omega}{a} = \card{\omega}{b} = \card{\omega}{c}\}$.
    Да допуснем, че $L$ е безконтекстен език. Тогава, според \Theorem{intersection-context-reg}, $L^\prime = L \cap \L(a^\star b^\star c^\star)$ също е безконтекстен език.
    Но $L^\prime = \{a^nb^nc^n \mid n \in \Nat\}$, за който знаем от \Example{context-free:pumping:anbncn}, че {\em не} е безконтекстен.
    Достигнахме до противоречие. Следователно, $L$ не е безконтекстен език.
  \end{example}
\end{extra}


%%% Local Variables:
%%% mode: latex
%%% TeX-master: "../eai"
%%% End:

\newpage
\section{Допълнителни задачи}

\begin{problem}
  Постройте регулярен израз за езика на следната граматика:
  \begin{align*}
    & S \to S + S\ |\ S * S\ |\ A\\
    & A \to KL\ |\ LK\\
    & K \to 0K\ |\ \varepsilon\\
    & L \to 1K\ |\ \varepsilon.
  \end{align*}
\end{problem}


\begin{problem}
  Докажете, че следните езици са безконтекстни.
  \begin{enumerate}[a)]
  \item
    \marginpar{$S \rightarrow aSa\ \vert\ bSb\ \vert\ \varepsilon$}
    $L = \{ww^R \mid w \in \{a,b\}^\star\}$;
  \item
    \marginpar{$S \rightarrow aSa\ \vert\ bSb\ \vert\ a\vert\ b\ \vert\ \varepsilon$}
    $L = \{w \in \{a,b\}^\star \mid w = w^R\}$;
  \item
    $L = \{a^nb^{2m}c^{n} \mid m,n \in \Nat\}$;
  \item
    $L = \{a^nb^{m}c^{m}d^n \mid m,n \in \Nat\}$;
  \item
    \marginpar{Обединение на два езика}
    $L = \{a^nb^{2k} \mid n,k \in \Nat\ \&\ n \neq k\}$;
  \item
    \marginpar{$S \rightarrow aSb | aS | a$}
    $L = \{a^nb^k \mid n > k\}$;
  \item
    $L = \{a^nb^k \mid n \geq 2k\}$;
  % \item
  %   \marginpar{$S \rightarrow aSc | B,\ B \rightarrow bBc | \varepsilon$}
  %   $L = \{a^nb^mc^{n+m}\mid n,m \in \Nat\}$;
  % \item
  %   \marginpar{$S \rightarrow aSc | aS | B$, $B\rightarrow bBc | bB | \varepsilon$}
  %   $L = \{a^nb^kc^m \mid n + k \geq m\}$;
  \item
    \marginpar{$S \rightarrow aSc | aS | aB | bB$,\\$B\rightarrow bBc | bB | \varepsilon$}
    $L = \{a^nb^kc^m \mid n + k \geq m+1\}$;
  \item
    $L = \{a^nb^kc^m \mid n + k \geq m+2\}$;
  \item
    \marginpar{$S \rightarrow aSc | aS | B | Bc$,\\$B\rightarrow bBc | bB | \varepsilon$}
    $L = \{a^nb^kc^m \mid n + k + 1 \geq m\}$;
  \item
    $L = \{a^nb^mc^{2k} \mid n \neq 2m\ \&\ k \geq 1\}$;
  \item
    $L = \{a^nb^kc^m \mid n + k \leq m\}$;
  \item
    $L = \{a^nb^kc^m \mid n + k \leq m+1\}$;
  \item
    \marginpar{Обединение на три езика}
    $L = \{a^nb^mc^k \mid n, m, k \text{ не са страни на триъгълник}\}$.
  \item
    $L = \{a,b\}^\star \setminus \{a^{2n}b^n \mid n\in\Nat\}$;
  \item
    $L = \{\alpha \in \{a,b\}^\star\mid N_a(\alpha) = N_b(\alpha) + 1\}$;
  \item
    $L = \{\alpha \in \{a,b\}^\star\mid N_a(\alpha) \geq N_b(\alpha)\}$;
  \item
    $L = \{\alpha \in \{a,b\}^\star\mid N_a(\alpha) > N_b(\alpha)\}$;
  \item
    $L = \{\alpha\sharp\beta \mid \alpha,\beta \in \{a,b\}^\star\ \&\ \alpha \neq \beta\}$;
  \item
    $L = \{\alpha\beta \in \{a,b\}^\star \mid\ |\alpha| = |\beta|\ \&\ \alpha \neq \beta\}$;
  \item
    $L = \{\alpha \in \{a,b\}^\star \mid \text{ във всеки префикс $\beta$ на $\alpha$, } N_b(\beta) \leq N_a(\beta)\}$;
  \item
    $L = \{\alpha \sharp \beta \mid \alpha,\beta \in \{a,b\}^\star\ \&\ \alpha^R\mbox{ е поддума на }\beta \}$.
  \item
    $L = \{\omega_1 \sharp \omega_2 \sharp \cdots \sharp \omega_n \mid n\geq 2\ \&\ \omega_1,\omega_2,\dots,\omega_n \in \{a,b\}^\star\ \&\ \abs{\omega_1} = \abs{\omega_2}\}$;
  \item
    $L = \{\omega_1 \sharp \omega_2 \sharp \cdots \sharp \omega_n \mid n\geq 2\ \&\ \omega_1,\dots,\omega_n \in \{a,b\}^\star\ \&\ (\exists i \neq j)[\abs{\omega_i} = \abs{\omega_j}]\}$;
  \item
    $L = \{\omega_1 \sharp \omega_2 \sharp \cdots \sharp \omega_n \mid n\geq 2\ \&\ (\forall i\in[1,n])[\omega_i \in \{a,b\}^\star\ \&\ \abs{\omega_i} = \abs{\omega_{n+1-i}}]\}$.
  \end{enumerate}
\end{problem}
\begin{hint}
  \begin{enumerate}
  \item[р)]
    $S\to EaE$, $E \to aEbE | bEaE | \varepsilon$.
  \item[с)]
    $S\to E | SaS$, $E \to aEbE | bEaE | \varepsilon$.
  \item[у)]
    Разгледайте граматиката:
    \begin{align*}
      & S \to AaR\ |\ BbR\ |\ E\\
      & A \to XAX\ |\ bR\sharp\\
      & B \to XBX\ |\ aR\sharp\\
      & E \to XEX\ |\ XR\sharp\ |\ \sharp XR\\
      & R \to XR\ |\ \varepsilon\\
      & X \to a\ |\ b.
    \end{align*}
    Имаме, че за произволни думи $\alpha,\beta,\gamma,\delta \in \{a,b\}^\star$,
    \begin{align*}
      & S \to^\star \alpha b \gamma \sharp \beta a \delta\ \&\ |\alpha| = |\beta|,\\
      & S \to^\star \alpha a \gamma \sharp \beta b \delta\ \&\ |\alpha| = |\beta|, \text{ или}\\
      & S \to^\star \alpha \sharp \beta\ \&\ |\alpha| \neq |\beta|\\
    \end{align*}
  \item[ф)]
    Разгледайте граматиката:
    \begin{align*}
      & S \to XSX\ |\ A\\
      & A \to aBb\ |\ bBa\\
      & B \to XBX\ |\ \varepsilon\\
      & X \to a\ |\ b.
    \end{align*}
    Тази граматика е по-хитра:
    \begin{align*}
      & S \to AB\ |\ BA\\
      & A \to XAX\ |\ a\\
      & B \to XBX\ |\ b\\
      & X \to a\ |\ b.
    \end{align*}
    
  \item[х)]
    $S \to aSbS\ |\ aS$.
  \end{enumerate}
\end{hint}

\begin{problem}
  Проверете дали следните езици са безконтекстни:
  \begin{enumerate}[a)]
  \item
    $\{a^nb^{2n}c^{3n}\ \mid\ n\in\Nat\}$;
  \item
    $\{a^nb^{2n}c^{n}\ \mid\ n\in\Nat\}$;
  \item
    $\{a^nb^kc^ka^n\mid\ k \leq n\}$;
  \item
    $\{a^nb^mc^k\mid n < m < k\}$;
  \item
    $\{a^nb^nc^k\mid n \leq k \leq 2n\}$;
  \item
    $\{a^nb^mc^k\mid k = \min\{n,m\}\}$;
  \item
    $\{a^nb^nc^m\mid m \leq n\}$;
  \item
    $\{a^nb^mc^k\mid k = n\cdot m\}$;
  \item
    $L^\star$, където
    $L = \{\alpha\alpha^R \mid \alpha \in \{a,b\}^\star\}$;
  \item
    $\{www\mid w\in \{a,b\}^\star\}$;
  % \item
  %   $\{ww^R\mid w\in \{a,b\}^\star\}$;
  \item
    $\{a^{n^2}b^n\ \mid n \in \Nat\}$;
  \item
    $\{a^p\ \mid\ p\mbox{ е просто }\}$;
  \item
    $\{\omega \in \{a,b\}^\star \mid \omega = \omega^R\}$;
  \item
    $\{\omega^n \mid \omega \in \{a,b\}^\star\ \&\ N_b(\omega) = 2\ \&\ n \in \Nat\}$;
  \item
    $\{\omega c^n \omega^R \mid \omega \in \{a,b\}^\star\ \&\ n = \abs{\omega}\}$;
  \item
    % Дефиниция на подниз
    $\{w c x\mid w,x\in \{a,b\}^\star\ \&\ w\mbox{ е подниз на }x\}$;
  \item
    $\{x_1 \sharp x_2 \sharp \dots \sharp x_k\mid k\geq 2\ \&\ x_i\in a^\star\ \&\ (\exists i,j)[i \neq j\ \&\ x_i = x_j]\}$;
  \item
    $\{x_1 \sharp x_2 \sharp \dots \sharp x_k\mid k\geq 2\ \&\ x_i\in a^\star\ \&\ (\forall i,j \leq k)[i \neq j \iff x_i \neq x_j]\}$;
  \item
    $\{a^ib^jc^k\mid i,j,k\geq 0\ \&\ (i = j \vee j = k)\}$;
  \item
    % \marginpar{Разгл. $L' = L \cap L(a^*b^*c^*)$.}
    $\{\alpha \in \{a,b,c\}^\star\mid N_a(\alpha) > N_b(\alpha) > N_c(\alpha)\}$;
  \item
    $\{a,b\}^\star \setminus \{a^nb^n\mid n\in \Nat\}$;
  \item
    $\{a^nb^mc^k \mid m^2 = 2nk\}$;

  \item
    $L = \{a^nb^mc^ma^n \mid m,n\in\Nat\ \&\ n = m+42\}$;
  \item
    $L = \{babaabaaab\cdots ba^{n-1}ba^nb \mid n \geq 1\}$;
  \item
    $\{a^mb^nc^k\mid m = n \vee n = k \vee m = k\}$;
  \item
    $\{a^mb^nc^k\mid m \neq n \vee n \neq k \vee m \neq k\}$;
  \item
    $\{a^mb^nc^k\mid m = n \wedge n = k \wedge m = k\}$;
  \item
    $\{w \in \{a,b,c\}^\star\mid N_a(w) \neq N_b(w) \vee N_a(w) \neq N_c(w) \vee N_b(w) \neq N_c(w)\}$.
  \end{enumerate}
\end{problem}

\begin{problem}
  Докажете, че ако $L$ е безконтекстен език, то $L^R = \{\omega^R \mid \omega \in L\}$ 
  също е безконтекстен.
\end{problem}

\begin{problem}
  Нека $\Sigma = \{a,b,c,d,f,e\}$.
  Докажете, че езикът $L$ е безконтекстен, където за думите $\omega \in L$ са изпълнени свойствата:
  \begin{itemize}[-]
  \item 
    за всяко $n\in\Nat$, след всяко срещане на $n$ последнователни $a$-та
    следват $n$ последователни $b$-та, и $b$-та не се срещат по друг повод в $\omega$, и
  \item
    за всяко $m\in\Nat$, след всяко срещане на $m$ последнователни $c$-та
    следват $m$ последователни $d$-та, и $d$-та не се срещат по друг повод в $\omega$, и
  \item
    за всяко $k\in\Nat$, след всяко срещане на $k$ последнователни $f$-а
    следват $k$ последователни $e$-та, и $e$-та не се срещат по друг повод в $\omega$.
  \end{itemize}
\end{problem}

\begin{problem}
  Да разгледаме езиците:
  \begin{align*}
    & P = \{\alpha\in\{a,b,c\}^*\,|\, \alpha \text{ е палиндром с четна дължина}\} \\
    & L =  \{\beta b^n\,|\, n\in\mathbb{N}, \beta\in P^n\}.
  \end{align*}
  Да се докаже, че:
  \begin{enumerate}[a)]
  \item 
    $L$ не е регулярен;
  \item 
    $L$ е безконтекстен.
  \end{enumerate}
\end{problem}

\begin{problem}
  Нека $L_1$ е произволен регулярен език над азбуката $\Sigma$, 
  а $L_2$ е езика от всички думи палиндроми над $\Sigma$.
  Докажете, че $L$ е безконтекстен език, където:
  \[L = \{\alpha_1\alpha_2\cdots\alpha_{3n}\beta_1\cdots\beta_m\gamma_1\cdots\gamma_n \mid \alpha_i,\gamma_j \in L_1, \beta_k\in L_2, m,n \in \Nat\}.\]
\end{problem}

\begin{problem}
  Нека $L = \{\omega\in\{a,b\}^\star \mid N_a(\omega) = 2\}$.
  Да се докаже, че езикът $L' = \{\alpha^n \mid \alpha\in L, n \geq 0\}$ не е безконтекстен.
\end{problem}


\begin{problem}
  Нека $\Sigma = \{a,b,c\}$ и $L \subseteq \Sigma^\star$ е безконтестен език. Ако имаме дума 
  $\alpha \in \Sigma^\star$, тогава \emph{L-вариант} на $\alpha$ ще наричаме думата, която се получава като в $\alpha$ всяко едно 
  срещане на символа $a$ заменим с (евентуално различна) дума от $L$.
  Тогава, ако $M \subseteq \Sigma^*$ е произволен безконтестен език, да се докаже че езикът
  \begin{equation*}
    M' = \{\beta\in\Sigma^\star |\ \beta \text{ е $L$-вариант на } \alpha \in M \}
  \end{equation*}
  също е безконтекстен.
\end{problem}

\begin{problem}
  Докажете, че всеки безконтекстен език над азбуката $\Sigma = \{a\}$
  е регулярен.
\end{problem}

\begin{problem}
%  \marginpar{\cite{papadimitriou} стр. 149}
  Да фиксираме азбуката $\Sigma$.
  Нека $L$ е безконтекстен език, а $R$ е регулярен език.
  Докажете, че езикът
  $L/R = \{\omega \in \Sigma^\star \mid (\exists u \in R)[\omega u \in L]\}$
  е безконтекстен.
\end{problem}


\begin{problem}
  Нека е дадена граматиката $G = \pair{\{a,b\}, \{S,A,B,C\},S,R}$.
  Използвайте CYK-алгоритъма, за да проверите дали
  думата $\alpha$ принадлежи на $\L(G)$, където правилата на граматиката $R$ и думата $\alpha$
  са зададени като:
  \begin{enumerate}[a)]
  \item
    $R = \{S\rightarrow BA| CA|a, C\rightarrow BS|SA,A\rightarrow a, B\rightarrow b\}$, $\alpha=bbaaa$;
  \item
    $R =\{S\rightarrow AB|BC, A\rightarrow BA|a,B\rightarrow CC|b, C\rightarrow AB|a\}$, $\alpha=baaba$;
  \item
    $R = \{S\rightarrow AB, A\rightarrow AC|a|b,B\rightarrow CB|a, C\rightarrow a\}$, $\alpha=babaa$;
  \end{enumerate}
\end{problem}

\begin{problem}
  \marginpar{Интересно е също да се направи и безконтекстна граматика за $L$}
  Постройте стеков автомат за езика над азбуката $\{a,b,\sharp\}$:
  \[L = \{\omega_1 \sharp \omega_2 \sharp \cdots \sharp \omega_{2n} \mid n\in\Nat\ \&\ \sum^n_{i=1}\abs{\omega_{2i}} = \sum^{n}_{i=1}\abs{\omega_{2i-1}}\}.\]
\end{problem}

\begin{problem}
  Нека с $\code{\A}$ да означим думата над азбуката $\{0,1\}$, която кодира крайния автомат $\A$.
  Посочете кои от следните езици са регулярни/безконтекстни/разрешими/полуразрешими, където:
  \begin{enumerate}[a)]
  \item
    $L = \{\code{\A} \mid \A\text{ е краен автомат}\}$;
  \item
    $L = \{\code{\A} \mid \A\text{ е краен автомат с пет състояния}\}$;
  \item
    $L = \{\code{\A} \mid \A\text{ е краен детерминиран автомат}\}$;
  \item
    $L = \{\code{\A} \mid \A\text{ е краен детерминиран тотален автомат}\}$;
  \item
    $L = \{\code{\A}\cdot\omega \mid \omega \in \Sigma^\star\ \&\ \omega \in \L(\A)\}$;
  \item
    $L = \{\code{\A}\cdot\omega \mid \omega \in \Sigma^\star\ \&\ \omega \not\in \L(\A)\}$;
  \item
    $L = \{\code{\A} \cdot \code{\B} \mid \A, \B \text{ са крайни автомати и } \L(\A) = \L(\B)\}$;
  \end{enumerate}
  Обосновете се!
\end{problem}

\begin{problem}
  Нека $G = \CFG$ е {\em регулярна} граматика, т.е.
  всички правила на $G$ са от вида $A \to bC$ и $A \to \varepsilon$.
  Посочете кои от следните езици са безконтекстни, където:
  \begin{enumerate}[a)]
  \item 
    $L = \{\alpha\sharp\beta^{rev} \mid \alpha,\beta \in (V \cup \Sigma)^\star\ \&\ \alpha \vdash_G \beta\}$;
  \item 
    $L = \{\alpha\sharp\beta^{rev} \mid \alpha,\beta \in (V \cup \Sigma)^\star\ \&\ \alpha \vdash^\star_G \beta\}$;
  \item
    $L = \{\alpha\sharp\beta^{rev} \mid \alpha,\beta \in (V \cup \Sigma)^\star\ \&\ \alpha \not\vdash_G \beta\}$;
  \item
    $L = \{\alpha\sharp\beta^{rev} \mid \alpha,\beta \in (V \cup \Sigma)^\star\ \&\ \alpha \not\vdash^\star_G \beta\}$.
  \end{enumerate}
  Обосновете се!
\end{problem}

\begin{problem}
  Да разгледаме една {\em безконтекстна} граматика $G = \CFG$.
  Посочете кои от следните езици са безконтекстни, където:
  \begin{enumerate}[a)]
  \item 
    $L = \{\alpha\sharp\beta^{rev} \mid \alpha,\beta \in (V \cup \Sigma)^\star\ \&\ \alpha \vdash_G \beta\}$;
  \item
    $L = \{\alpha_1\sharp\alpha^{rev}_2\sharp\alpha_3\sharp\alpha^{rev}_4\sharp \cdots \mid \alpha_i \in (V \cup \Sigma)^\star\ \&\ \alpha_i \vdash_G \alpha_{i+1} \text{ за }i<n\}$;
  \item 
    $L = \{\alpha\sharp\beta^{rev} \mid \alpha,\beta \in (V \cup \Sigma)^\star\ \&\ \alpha \vdash^\star_G \beta\}$;
  \item
    $L = \{\alpha\sharp\beta^{rev} \mid \alpha,\beta \in (V \cup \Sigma)^\star\ \&\ \alpha \not\vdash_G \beta\}$;
  \item
    $L = \{\alpha\sharp\beta^{rev} \mid \alpha,\beta \in (V \cup \Sigma)^\star\ \&\ \alpha \not\vdash^\star_G \beta\}$.
  \end{enumerate}
  Обосновете се!
\end{problem}

\begin{problem}
  Да разгледаме една {\em неограничена} граматика $G = \CFG$.
  Посочете кои от следните езици са безконтекстни, където:
  \begin{enumerate}[a)]
  \item 
    $L = \{\alpha\sharp\beta^{rev} \mid \alpha,\beta \in (V \cup \Sigma)^\star\ \&\ \alpha \vdash_G \beta\}$;
  \item 
    $L = \{\alpha\sharp\beta^{rev} \mid \alpha,\beta \in (V \cup \Sigma)^\star\ \&\ \alpha \vdash^\star_G \beta\}$;
  \item
    $L = \{\alpha\sharp\beta^{rev} \mid \alpha,\beta \in (V \cup \Sigma)^\star\ \&\ \alpha \not\vdash_G \beta\}$;
  \item
    $L = \{\alpha\sharp\beta^{rev} \mid \alpha,\beta \in (V \cup \Sigma)^\star\ \&\ \alpha \not\vdash^\star_G \beta\}$.
  \end{enumerate}
  Обосновете се!
\end{problem}



%%% Local Variables:
%%% mode: latex
%%% TeX-master: "../eai"
%%% End:


%%% Local Variables:
%%% mode: latex
%%% TeX-master: "../eai"
%%% End:

\chapter{Машини на Тюринг}

\setlength{\epigraphwidth}{0.65\textwidth}\epigraph{Turing’s ‘Machines’. These machines are humans who calculate. \cite[§ 1096]{rpp1}.}




% \begin{framed}
%   {\bf Теза на Чърч-Тюринг:} Всеки алгоритъм може да се осъществи като машина на Тюринг.
% \end{framed}

% \section{Основни понятия}
\index{Тюринг}
\index{машина на Тюринг!детерминистична}
\mynote{Тук до голяма степен следваме \cite[Глава 3]{sipser3}. Понятието за машина на Тюринг има много еквивалентни дефиниции. }
{\em Детерминистична} машина на Тюринг ще наричаме осморка от вида 
\[\M = \TM,\] където:
\begin{itemize}
\item 
  $Q$ - крайно множество от състояния;
\item
  $\Sigma$ - крайна азбука за входа;
\item
  $\Gamma$ - крайна азбука за лентата, $\Sigma \subseteq \Gamma$;

\item
  $\blank$ - символ за празна клетка на лентата,  $\blank \in \Gamma \setminus \Sigma$;
\item
  $\qstart \in Q$ - начално състояние;
\item
  \mynote{Тези две състояния ще наричаме заключителни}
  $\qaccept \in Q$ - приемащо състояние;
\item
  $\qreject \in Q$ - отхвърлящо състояние, където $\qaccept \neq \qreject$;
\item
  % \mynote{Няма нужда да изискваме главата да остава върху същата клетка от лентата}
  \mynote{Това означава, че веднъж достигнем ли заключително състояние, не можем да правим повече преходи. Тук следваме \cite[стр. 169]{sipser3} и \cite[стр. 327]{hopcroft2}.}
  $\delta:Q'\times\Gamma \to Q\times \Gamma \times \{\goleft,\goright,\stay\}$ - тотална функция на преходите, където\\
  $Q'~=~Q~\setminus~\{\qaccept, \qreject\}$.
\end{itemize}

Всяка машина на Тюринг разполага с неограничено количество памет, която е представена като безкрайна (и в двете посоки) лента, разделена на клетки.
Всяка клетка съдържа елемент на $\Gamma$.
Сега ще опишем как $\M$ работи върху вход думата $\alpha \in \Sigma^\star$.
Първоначално безкрайната лента съдържа само думата $\alpha$. Останалите клетки на лентата съдържат символа $\blank$.

Освен това, $\M$ се намира в началното състояние $\qstart$ и главата за четене е върху най-левия символ на $\alpha$.
Работата на $\M$ е описана от функцията на преходите $\delta$.
  
\begin{itemize}
\item
  \index{машина на Тюринг!конфигурация}
  \index{машина на Тюринг!моментно описание}
  \mynote{На англ. instanteneous description.\\
    Понякога за удобство ще означаваме моментната конфигурация като $(q,\alpha\underline{x}\beta)$ вместо по-неудобното $(\alpha,q,x\beta)$.}
  Формално, {\bf моментната конфигурация} (или описание) на едно изчисление на машина на Тюринг
  е тройка от вида 
  \[(\alpha, q, \beta) \in \Gamma^\star\times Q \times \Gamma^+,\]
  като интерпретацията на тази тройка е, че машината се намира в състояние $q$ и лентата има вида
  \[\tape{\alpha\underline{x}\beta'},\]
  където $\beta = x\beta'$ и четящата глава на машината е поставена върху $x$.
\item
  Макар и да имаме безкрайна лента, моментната конфигурация, която може да се представи като {\em крайна} дума,
  описва цялото моментно състояние на машината на Тюринг.
\item
  \index{машина на Тюринг!начална конфигурация}
  {\bf Началната конфигурация} за входната дума $\alpha \in \Sigma^\star$ представлява тройката
  \[(\varepsilon, \qstart, \alpha\blank).\]
\item
  \index{машина на Тюринг!приемаща конфигурация}
  {\bf Приемаща конфигурация} представлява тройка от вида
  \[(\beta, \qaccept, \gamma).\]
\item
  \index{машина на Тюринг!отхвърляща конфигурация}
  {\bf Отхвърляща конфигурация} представлява тройка от вида
  \[(\beta, \qreject, \gamma).\]
\item
  \index{машина на Тюринг!заключителна конфигурация}
  Една конфигурация ще наричаме {\bf заключителна}, ако тя е или приемаща или отхвърляща.
\end{itemize}

Както за автомати, удобно е да дефинираме бинарна релация $\vdash_\M$ над $\Gamma^\star~\times~Q~\times~\Gamma^+$,
която ще казва как моментната конфигурация на машината $\M$ се променя при изпълнение на една стъпка.


За да направим това, удобно е първо да дефинираме бинарната релация $\vdash_{y,d}$ над $\Gamma^\star~\times~Q~\times~\Gamma^+$, която показва как една моментна конфигурация се променя, когато заменим символа на главата с $y$ и се придвижим на посока $d \in \{\goleft, \goright, \stay\}$.

\mynote{Дефиницията на релацията $\vdash_{y,d}$ не зависи от конкретна машина на Тюринг!}
\begin{important}
  \begin{figure}[H]
    \begin{subfigure}[b]{0.5\textwidth}
      \begin{prooftree}
        \AxiomC{}
        \UnaryInfC{$(\lambda, q, xz\rho) \vdash_{y,\goright} (\lambda y, p, z\rho)$}
      \end{prooftree}
      \vspace*{2mm}
    \end{subfigure}
    ~
    \begin{subfigure}[b]{0.5\textwidth}
      \begin{prooftree}
        \AxiomC{}
        \UnaryInfC{$(\lambda, q, x) \vdash_{y,\goright} (\lambda y, p, \blank)$}
      \end{prooftree}
      \vspace*{2mm}
    \end{subfigure}
    
    \begin{subfigure}[b]{0.5\textwidth}
      \begin{prooftree}
        \AxiomC{}
        \UnaryInfC{$(\lambda z, q, x\rho) \vdash_{y,\goleft} (\lambda, p, z y\rho)$}
      \end{prooftree}
    \end{subfigure}
    ~
    \begin{subfigure}[b]{0.5\textwidth}
      \begin{prooftree}
        \AxiomC{}
        \UnaryInfC{$(\varepsilon, q, x\rho) \vdash_{y,\goleft} (\varepsilon, p, \blank y \rho)$}
      \end{prooftree}
    \end{subfigure}
    
    \begin{prooftree}
      \AxiomC{}
      \UnaryInfC{$(\lambda, q, x\rho) \vdash_{y,\stay} (\lambda, p, y \rho)$}
    \end{prooftree}
    \caption{Дефиниция на релацията $\vdash_{y,d}$.}
  \end{figure}
\end{important}



\mynote{Ако няма опасност да се заблудим за коя точно машина на Тюринг $\M$ говорим, то е възможно да пишем просто $\vdash$ вместо $\vdash_\M$.}

Сега вече сме готови да дефинираме релацията $\vdash_\M$.

\begin{important}
  \begin{figure}[H]
    \centering
    \begin{prooftree}
      \AxiomC{$\delta(q,x) = (q',y,d)$}
      \AxiomC{$(\lambda, q, x\rho) \vdash_{y,d} (\lambda', q', \rho')$}
      \BinaryInfC{$(\lambda, q, x\rho) \vdash_{\M} (\lambda', q', \rho')$}
    \end{prooftree}
    \caption{Едностъпков преход в еднолентова детерминистична машина на Тюринг $\M$}
  \end{figure}
\end{important}

Сега за всяко естествено число $\ell$, ще дефинираме релацията $\vdash^{\ell}$,
която ще казва, че от конфигурацията $\kappa$ можем да достигнем до конфигурацията $\kappa'$ за $\ell$ на брой стъпки.

\begin{figure}[H]
  \begin{subfigure}[b]{0.5\textwidth}
    \begin{prooftree}
      \AxiomC{}
      \RightLabel{\scriptsize{(рефлексивност)}}
      \UnaryInfC{$\kappa \vdash^0 \kappa$}
    \end{prooftree}
  \end{subfigure}
  ~
  \begin{subfigure}[b]{0.5\textwidth}
    \begin{prooftree}
      \AxiomC{$\kappa \vdash \kappa''$}
      \AxiomC{$\kappa'' \vdash^{\ell} \kappa'$}
      \RightLabel{\scriptsize{(транзитивност)}}
      \BinaryInfC{$\kappa \vdash^{\ell+1}\kappa'$}
    \end{prooftree}
  \end{subfigure}
\end{figure}

\begin{itemize}
% \item
%   $(\lambda, q, \rho) \vdash^\star (\lambda',q',\rho') \dff (\exists \ell)[(\lambda, q, \rho) \vdash^\ell (\lambda', q',\rho')]$.
\item
  С $\vdash^\star$ ще означаваме рефлексивното и транзитивно затваряне на релацията $\vdash$ или с други думи,
  \[\kappa \vdash^\star \kappa' \dff (\exists \ell \in \Nat)[\kappa \vdash^\ell \kappa'].\]
\item
  Макар и една конфигурация $\kappa$ да преставлява тройка, то често ще бъде удобно да гледаме на $\kappa$ като на дума от $\Gamma^\star Q \Gamma^+$.
\item
  Важно свойство е, че ако $\kappa \vdash^\star \kappa'$, то $\abs{\kappa} \leq \abs{\kappa'}$.
\item
  % \mynote{Аналогично както на си думата \texttt{abc} се представя като \texttt{a,b,c,0}, тук
  % представяме думата $\alpha$ като $\alpha\blank$.}
  \mynote{Важно е да имаме $\blank$ след думата $\alpha$, защото е възможно $\alpha$ да е празната дума.}
  машината на Тюринг $\M$ {\bf приема} думата $\alpha$, ако за някои $\lambda, \rho \in \Gamma^\star$,
  \[(\varepsilon, \qstart, \alpha\blank) \vdash^\star_\M (\lambda, \qaccept, \rho).\]
\item
  Машината на Тюринг $\M$ {\bf отхвърля} думата $\alpha$, ако за  някои $\lambda, \rho \in \Gamma^\star$,
  \[(\varepsilon, \qstart, \alpha\blank) \vdash^\star_\M (\lambda, \qreject, \rho).\]
  % за някои $\gamma_1, \gamma_2 \in \Gamma^\star$.
\item
  Машината на Тюринг $\M$ {\bf не приема} думата $\alpha$, 
  ако $\M$ отхвърля $\alpha$ или $\M$ никога не завършва при начална конфигурация $(\varepsilon,\qstart,\alpha)$.
\item
  \index{машина на Тюринг!разрешител}
  \mynote{На англ. такава машина на Тюринг се нарича {\bf decider} \cite[стр. 170]{sipser3}. Може такива машини на Тюринг да се наричат и тотални \cite[стр. 213]{kozen}.
    Да се внимава, че в Манев понятията са различни.}
  Една машина на Тюринг се нарича {\bf разрешител}, ако при всеки вход достига до заключително състояние,
  т.е. достига до $\qaccept$ или $\qreject$.
\item 
  Езикът, който се {\bf разпознава} от машината $\M$ е:
  \[\L(\M) \df \{\alpha\in\Sigma^\star \mid (\varepsilon, \qstart, \alpha\blank) \vdash^\star_\M (\lambda, \qaccept, \rho), \text{ за някои }\lambda,\rho\in\Gamma^\star\}.\]
\item
  \index{език!полуразрешим}
  \mynote{На англ. {\bf semidecidable language}. В литературата се използва и названието {\bf рекурсивно номеруем език}.}
  Езикът $L$ се нарича {\bf полуразрешим}, ако съществува машина на Тюринг $\M$, за която
  $L = \L(\M)$.
  В този случай се казва, че $\M$ разпознава езика $L$.
  Ако една дума $\alpha \in L$, то след крайно много стъпки ще достигнем до състоянието $\qaccept$.
  Ако $\alpha \not\in L$, то не е ясно дали какво се случва с изчислението на $\M$ върху $\alpha$. Възможно е да достигнем до състоянието $\qreject$, но може да попаднем в безкрайно изчисление.
\item
  \index{език!разрешим}
  \mynote{На англ. {\bf decidable language}. В литературата се използва и названието {\bf рекурсивен език}.}
  Един език $L$ се нарича {\bf разрешим}, ако за него съществува {\em разрешител} $\M$, за която
  $L = \L(\M)$.
  В този случай се казва, че $\M$ разрешава езика $L$.
\end{itemize}

\begin{framed}
  \begin{proposition}
    Ако $L$ е разрешим език над азбуката $\Sigma$, то $\Sigma^\star \setminus L$ също е разрешим език.
  \end{proposition}
\end{framed}

От дефинициите е ясно, че всеки разрешим език е полуразрешим.
По-късно, ще видим, че съществуват полуразрешими езици, чиито допълнения не са полуразрешими,
т.е. не всеки полуразрешим език е разрешим.
Една от основните ни задачи ще бъде да класифицираме различни езици като (не)раз\-ре\-ши\-ми и (не)полуразрешими.
За да придобием по-добра интуиция за тези нови понятия, ще разгледаме подробно няколко примера.
Ще видим също как можем да изобразяваме функцията на преходите на $\M$ графично.


%%% Local Variables:
%%% mode: latex
%%% TeX-master: "../eai"
%%% End:


\section{Примери за разрешими езици}

\begin{example}
  \marginpar{Знаем, че $L$ не е безконтекстен}
  Да разгледаме езика $L = \{a^nb^nc^n \mid n\in\Nat\}$.
 
  Нека да въведем нов символ $d$, с който ще маркираме обработените символи $a$, $b$, $c$.
  Идеята на алгоритъма, който ще разгледаме е да маркира на всяка итерация по едно $a$, $b$, и $c$.
  Той завършва успешно ако всички символи на думата са маркирани.
  Нека първоначално думата е копирана върху лентата и четящата глава е върху първия символ на думата.
  \begin{enumerate}[(1)]
  \item 
    Чете $d$-та надясно по лентата докато срещне първото $a$ и го замества с $d$. Отива на стъпка (2).
    Ако символите свършат (т.е. достигне се $\blank$) преди да се достигне $a$,
    то алгоритъмът завършва успешно.
  \item
    Чете $d$-та надясно по лентата докато срещне първото $b$ и го замества с $d$.
    Отива на стъпка (3).
  \item
    Чете $d$-та надясно по лентата докато срещне първото $c$ и го замества с $d$.
  \item
    Връща четящата глава в началото на лентата, т.е. чете наляво докато не срещне символа $\blank$.
    Връща се в стъпка (1). 
  \end{enumerate}

  Нека сега да видим, че този алгоритъм може да се опише съвсем формално с машина на Тюринг.
  Ще построим машина на Тюринг $\M$, за която $L = \L(\M)$, където
  \begin{itemize}
  \item 
    $\Sigma = \{a,b,c\}$;
  \item
    $\Gamma = \{a,b,c,d,\blank\}$, за някой нов символ $d$;
  \item
    $Q = \{q_1,q_2,\dots,q_5\}$;
  \item
    $q_{accept} = q_5$;
  \item
    частичната функция на преходите $\delta:Q\times\Gamma \to Q\times\Gamma\times\{\goleft,\goright,\stay\}$
    е описана на схемата отдолу.
  \end{itemize}

  \begin{framed}
  \begin{figure}[H]
    \begin{center}
      \begin{tikzpicture}[->,>=stealth,thick,node distance=50pt]
        \tikzstyle{every state}=[circle,minimum size=10pt,auto]
        
        \node[state,initial]    (1) {$q_1$};
        \node[state]            (2) [right of=1]{$q_2$};
        \node[state]            (3) [right of=2]{$q_3$};
        \node[state]            (4) [right of=3]{$q_4$};
        \node[state,accepting]  (5) [below of=1]{$q_5$};
        % \node[state,accepting]  (6) [right of=5]{$6$};
        
        \begin{scope}[every node/.style={scale=.8}]
          \path
          (1) edge [loop above] node [above] {$d;\goright$} (1)
          (1) edge [bend right=15] node [left] {$\blank;\goright$} (5)
          % (1) edge [bend left=15] node [left] {$\{b,c\}$} (6)
          (1) edge [bend left=15] node [above] {$a/d;\goright$} (2)
          (2) edge [bend left=15] node [above] {$b/d;\goright$} (3)
          (2) edge [loop above] node   [above] {$\{a,d\};\goright$} (2)
          (3) edge [bend left=15] node [above] {$c/d;\goleft$} (4)
          (3) edge [loop above] node   [above] {$\{b,d\};\goright$} (3)
          (4) edge [loop right] node   [below right] {$\{a,b,d\};\goleft$} (4)
          (4) edge [in=65,out=115,above] node [above] {$\blank;\goright$} (1);
        \end{scope}
      \end{tikzpicture}
    \end{center}
    \caption{детерминистична частична машина на Тюринг $\M$, за която $\L(\M) = \{a^nb^nc^n \mid n \in \Nat\}$}
  \end{figure}
  \end{framed}

  Например,
  \begin{align*}
    & \delta(q_1, a) = (q_2, d, \goright)\\
    & \delta(q_4, \blank) = (q_1, \blank, \goright)\\
    & \delta(q_2, a) = (q_2, a, \goright).
  \end{align*}

  % Да проследим изчислението на думата $aabbcc$:
  
  % \[_1aabbcc \vdash d_2abbcc \vdash da_2bbcc \vdash dad_3bcc \vdash dadb_3cc \vdash dad_4bdc \vdash da_4dbdc \vdash \cdots \vdash\]
  % \[_4dadbdc \vdash\ _4\blank dadbdc \vdash\ _1dadbdc \vdash d_1adbdc \vdash dd_2dbdc \vdash ddd_2bdc \vdash dddd_3dc \vdash \]
  % \[ ddddd_3c \vdash dddddd_4 \vdash \cdots \vdash\ _4\blank dddddd \vdash\ _1dddddd \vdash \cdots \vdash dddddd_1\blank \vdash dddddd_5\blank.\]

  Съобразете, че тази машина на Тюринг може да се направи тотална като се добави ново състояние $\qreject$
  и за всяка двойка $(q,x)$, за която функцията на преходите не е дефинирана, да сочи към $\qreject = q_6$.
  Така можем да получим {\em тотална} машина на Тюринг за езика $L$, което означава, че 
  $L$ е не само полуразрешим, но {\em разрешим} език.

\begin{framed}
  \begin{figure}[H]
    \begin{center}
      \begin{tikzpicture}[->,>=stealth,thick,node distance=50pt]
        \tikzstyle{every state}=[circle,minimum size=10pt,auto]
        
        \node[state,initial]    (1) {$q_1$};
        \node[state]            (2) [right of=1]{$q_2$};
        \node[state]            (3) [right of=2]{$q_3$};
        \node[state]            (4) [right of=3]{$q_4$};
        \node[state,accepting]  (5) [below of=1]{$q_5$};
        \node[state]            (6) [below of=3]{$q_6$};
        
        \begin{scope}[every node/.style={scale=.8}]
          \path
          (1) edge [loop above] node [above] {$d;\goright$} (1)
          (1) edge [bend right=15] node [left] {$\blank;\stay$} (5)
          % (1) edge [bend left=15] node [left] {$\{b,c\}$} (6)
          (1) edge [bend left=15] node [above] {$a/d;\goright$} (2)
          (2) edge [bend left=15] node [above] {$b/d;\goright$} (3)
          (2) edge [loop above] node [above] {$\{a,d\};\goright$} (2)
          (3) edge [bend left=15] node [above] {$c/d;\goleft$} (4)
          (3) edge [loop above] node [above] {$\{b,d\};\goright$} (3)
          (4) edge [loop right] node [below right] {$\{a,b,d\};\goleft$} (4)
          (4) edge [in=65,out=115,above] node [above] {$\blank;\goright$} (1);

          \path
          (1) edge [dashed, bend right=15] node [left] {$\{b,c\};\stay$} (6)
          (2) edge [dashed, bend left=30] node [left] {$\{c,\blank\};\stay$} (6)
          (3) edge [dashed, bend left=25] node [right] {$\{a,\blank\};\stay$} (6)
          (5) edge [dashed, loop below] node [below] {$\{a,b,c,d,\blank\};\stay$} (5)
          (6) edge [dashed, loop right] node [right] {$\{a,b,c,d,\blank\};\stay$} (6);
        \end{scope}
      \end{tikzpicture}
    \end{center}
  \end{figure}
\end{framed}

\end{example}

\begin{example}
  \marginpar{Да напомним, че този език не е безконтекстен}
  \marginpar{В \cite[стр. 155]{hopcroft1} е дадено по-различно решение. Тук следваме \cite[стр. 173]{sipser3}. Там има малка грешка}
  Да разгледаме езика $L = \{\omega \sharp \omega \mid \omega\in\{a,b\}^\star\}$.
  Нека първо да видим, че можем неформално да опишем алгоритъм, който да разпознава думите на езика $L$.
  Нека една дума е копирана върху лентата и четящата глава е поставена върху първия символ от думата.
  \begin{enumerate}[(1)]
  \item 
    Чете $x$-ове надясно по лентата докато не срещне $a$ или $b$ и го замества с $x$.
    Запомня дали сме срещнали $a$ или $b$.
    Ако вместо $a$ или $b$ срещне $\sharp$, то отива на стъпка $(6)$.
  \item
    Чете $a$-та и $b$-та надясно по лентата докато не стигне $\sharp$. 
  \item
    Чете $c$-то надясно по лентата и всички следващи $x$-ове докато не срещне символа $a$ или $b$.
    Той трябва да е същия символ, който сме запаметили на стъпка $(1)$.
    Заместваме този символ с $x$.
  \item
    Чете $x$-ове наляво по лентата докато не стигне $\sharp$.
  \item
    Чете $a$-та и $b$-та по лентата докато не стигне $x$.
    Поставя четящата глава върху символа точно след първия $x$.
    Отива на стъпка $(1)$.
  \item
    Прочита $\sharp$ надясно по лентата и чете надясно $x$-ове докато не срещне $\blank$.
    Алгоритъмът завършва успешно.
  \end{enumerate}

  Ще построим машина на Тюринг $\M$, за която $L = \L(\M)$.
  \begin{itemize}
  \item 
    $\Sigma = \{a,b,\sharp\}$;
  \item
    $\Gamma = \{a,b,\sharp,x,\blank\}$;
  \item
    $Q = \{q_1,q_2,\dots,q_9\}$;
  \item
    $\qaccept = q_9$;
  \end{itemize}

  \begin{framed}
  \begin{figure}[H]
    \begin{center}
      \begin{tikzpicture}[->,>=stealth,thick,node distance=50pt]
        \tikzstyle{every state}=[circle,minimum size=10pt,auto,scale=.9]
        
        \node[state,initial]    (1) {$q_1$};
        \node[state]            (2) [above right of=1]{$q_2$};
        \node[state]            (3) [below right of=1]{$q_3$};
        \node[state]            (4) [right of=2]{$q_4$};
        \node[state]            (5) [right of=3]{$q_5$};
        \node[state]            (6) [below right of=4]{$q_6$};
        \node[state]            (7) [above of=6]{$q_7$};
        \node[state]            (8) [left of=3]{$q_8$};
        \node[state,accepting]  (9) [below left of=3]{$q_9$};
        
        \begin{scope}[every node/.style={scale=.8}]
          \path
          (1) edge [bend left=15] node [below right] {$a/x;\goright$} (2)
              edge [bend right=15] node [above right] {$b/x;\goright$} (3)
              edge [bend right=15] node [left] {$\sharp;\goright$} (8)
          (2) edge [loop above] node [above] {$\{a,b\};\goright$} (2)
              edge [bend left=15] node [above] {$\sharp;\goright$} (4)
          (3) edge [loop below] node [below] {$\{a,b\};\goright$} (3)
              edge [bend right=15] node [below] {$\sharp;\goright$} (5)
          (4) edge [loop above] node [above] {$x;\goright$} (4)
              edge [bend left=15] node [below left] {$a/x;\goleft$} (6)
          (5) edge [loop below] node [below] {$x;\goright$} (5)
              edge [bend right=15] node [above left] {$b/x;\goleft$} (6)
          (6) edge [loop right] node [right] {$x;\goleft$} (6)
              edge [bend right=15] node [right] {$\sharp;\goleft$} (7)
          (7) edge [loop right] node [right] {$\{a,b\};\goleft$} (7)
              edge [out=130,in=120,above,distance=2.5cm] node [above] {$x;\goright$} (1)
          (8) edge [loop left] node [left] {$x;\goright$} (8)
              edge [bend right=15] node [left] {$\blank;\stay$} (9);
        \end{scope}
      \end{tikzpicture}
    \end{center}
    \caption{детерминистична частична машина на г-н Тюринг $\M$, за която $\L(\M) = \{\omega\sharp\omega \mid \omega \in \{a,b\}^\star\}$}
  \end{figure}
\end{framed}

  Да проследим изчислението на думата $ab\sharp ab$.
  
  \begin{align*}
    (q_1, \underline{a}b\sharp ab) & \to (q_2, x\underline{b}\sharp ab) \to (q_2,xb\underline{\sharp}ab) \to (q_4, xb\sharp\underline{a}b) \to (q_6, xb\underline{\sharp}xb)\\
    & \to (q_7, x\underline{b}\sharp xb) \to (q_7, \underline{x}b\sharp xb) \to (q_1, x\underline{b}\sharp xb) \to (q_3, xx\underline{\sharp}xb)\\
    & \to (q_5, xx\sharp\underline{x}b) \to (q_5, xx\sharp x\underline{b}) \to (q_6, xx\sharp\underline{x}x) \to (q_6, xx\underline{\sharp} xx)\\
    & \to (q_7, x\underline{x}\sharp xx) \to (q_1, xx\underline{\sharp} xx) \to (q_8, xx\sharp\underline{x}x) \to (q_8, xx\sharp x\underline{x})\\
    & \to (q_8, xx\sharp xx\underline{\blank}) \to (q_9,xx\sharp xx\underline{\blank}).
  \end{align*}

  Може лесно да се съобрази, че тази машина на Тюринг може да се допълни до {\em тотална}.
  
\end{example}

%%% Local Variables:
%%% mode: latex
%%% TeX-master: "../eai"
%%% End:


\subsection*{Многолентови машини на Тюринг}
\index{машина на Тюринг!многолентова}
%Това е просто като имаш shift.
%Използват се при недет. машини

Машина на Тюринг с $k$ ленти има същата дефиниция като еднолентова машина на Тюринг
с единствената разлика, че
\[\delta: Q \times \Gamma^k\to Q \times \Gamma^k \times \{L,R,S\}^k.\]
Тук добавяме и възможността главата върху някои от лентите да стои на място.
\marginpar{В \cite[стр. 177]{sipser3} конструкцията е малко по-различна. Там съдържанието на всяка лента се поставя последователно върху една лента, като се разделят със специален символ}
\begin{prop}
  За всяка $k$-лентова машина на Тюринг $\M$ съществува еднолентова машина на Тюринг $\M'$,
  такава че $\L(\M) = \L(\M')$.
\end{prop}
\begin{proof}
  \marginpar{Тук на практика следваме \cite[стр. 162]{hopcroft1}}
  Нека $\M$ е $k$-лентова машина на Тюринг.
  Ще построим еднолентова машина на Тюринг $\M'$, за която $\L(\M) = \L(\M')$.
  Да означим $\hat\Gamma = \{\hat X \mid X \in \Gamma\}$.
  Тогава азбуката на лентата на $\M'$ ще бъде $\Gamma' = (\hat\Gamma \cup \Gamma)^{k}$.
  Сега вместо да имаме $k$ ленти ще имаме една лента, която представлява $k$-орка.
  За да симулираме $\M$, използваме символите $\hat X$ за да маркират позицията на главите на $\M$,
  като във всяка координата на лентата има точно по един символ от вида $\hat X$.
  % С $\$$ ще отблезяваме границите на всяка лента, в която можем да търсим маркера.
  За да определим следващия ход на машината $\M'$, ние трябва да сканираме лентата докато не 
  открием разположението на всичките $k$ на брой маркирани клетки. Тогава симулираме ход на $\M$
  и отново трябва да променим маркираните клетки.
\end{proof}

%%% Local Variables:
%%% mode: latex
%%% TeX-master: "../eai"
%%% End:


\section{Изчислими функции}

Една {\em тотална} функция $f:\Sigma^\star \to \Sigma^\star$ се нарича изчислима с машина на Тюринг $\M$, ако 
за всяка дума $\alpha \in \Sigma^\star$,
\[(\varepsilon, \qstart, \alpha) \vdash^\star_\M (\varepsilon, \qaccept, f(\alpha)).\]
Това означава, че машината на Тюринг $\M$ е тотална.

Лесно може да се съобрази, че тогава езикът
\[Graph(f) = \{\alpha\sharp f(\alpha) \mid \alpha \in \Sigma^\star\}\]
е разрешим.

\begin{problem}
  Докажете, че съществуват функции от вида $f:\Sigma^\star\to\Sigma^\star$, които не са изчислими с машина на Тюринг.
\end{problem}
\begin{hint}
  Всяка машина на Тюринг може да се кодира с естествено число.
  Това означава, че съществуват изброимо безкрайно много машини на Тюринг.
  От друга страна, съществуват неизброимо много функции от вида $f:\Sigma^\star \to \Sigma^\star$.
\end{hint}

% Нека е дадена функцията $f:\Nat^k \to \Nat$.
% Ще казваме, че $f$ е изчислима с машината на Тюринг $\M$,
% ако за всяко $n_1,\dots,n_k$ е изпълнено:
% \begin{itemize}
% \item 
%   Представяме всяко от числата $n_1,\dots,n_k$ в монадична бройна система
%   като лентата на $\M$ има вида:  
%   \[\dots \blank \blank \underbrace{1111\dots 11}_{n} \blank\blank\dots,\]
%   като изискваме главата на $\M$ да е позиционирана върху най-лявата единица.
%   Такава конфигурация ще наричаме {\bf стандартна начална конфигурация}.
% \item
%   Ако $f(n_1,\dots,n_k) = m$, то $\M$ завършва с резултат върху лентата
%   \[\dots \blank \blank \underbrace{1111\dots 11}_{m} \blank\blank\dots,\]
%   като главата на $\M$ е върху най-лявата 1-ца.
%   Такава конфигурация се нарича {\bf стандартна финална конфигурация}.
% \item
%   Ако $f(n_1,\dots,n_k)$ е недефенирана, то $\M$ няма да завърши в стандартна конфигурация, т.е.
%   или $\M$ ще работи безкрайно време, или ще завърши в конфигурация, която не е стандартна.
% \end{itemize}

\begin{example}
  Да разгледаме функцията $f:\{1\}^\star \to \{1\}^\star$
  дефинирана като $|f(\alpha)| = 2|\alpha|$.
  
\begin{framed}
\begin{figure}[H]
  \begin{center}
    \begin{tikzpicture}[->,>=stealth,thick,node distance=50pt]
      \tikzstyle{every state}=[circle,minimum size=10pt,auto,scale=.7]
      
      \node[state,initial below]    (1) {$q_0$};
      \node[state]            (2) [right of=1]{$q_1$};
      \node[state]            (3) [right of=2]{$q_2$};
      \node[state]            (4) [right of=3]{$q_3$};
      \node[state]            (5) [right of=4]{$q_4$};
      \node[state]            (6) [right of=5]{$q_5$};
      \node[state]            (7) [right of=6]{$q_6$};
      \node[state]            (8) [right of=7]{$q_7$};
      \node[state]            (9) [right of=8]{$q_8$};
      \node[state]            (10) [right of=9]{$q_9$};
      \node[state]            (11) [below of=8]{$q_{10}$};
      \node[state,accepting]  (12) [below of=11]{$q_{11}$};
      
      \begin{scope}[every node/.style={scale=.8}]
      \path
      (1) edge [bend left=15] node [above] {$1;\goleft$} (2)
      (1) edge [bend right=30] node [above] {$\blank;\stay$} (12)
      (2) edge [bend left=15] node [above] {$1;\goleft$} (3)
      (2) edge [bend right=15] node [below] {$\blank;\goleft$} (3)
      (3) edge [bend left=15] node [above] {$\blank/1;\goleft$} (4)
      (4) edge [bend left=15] node [above] {$\blank/1;\goright$} (5)
      (5) edge [loop below] node [below] {$1;\goright$} (5)
      (5) edge [bend left=15] node [above] {$\blank;\goright$} (6)
      (6) edge [loop below] node [below] {$1;\goright$} (6)
      (6) edge [bend left=15] node [above] {$\blank;\goleft$} (7)
      (7) edge [bend left=15] node [above] {$1/\blank;\goleft$} (8)
      (8) edge [bend left=15] node [above] {$1;\goleft$} (9)
      (9) edge [loop below] node [below] {$1;\goleft$} (9)
      (9) edge [bend right=15] node [below] {$\blank;\goleft$} (10)
      (10) edge [loop below] node [below] {$1;\goleft$} (10)
      (10) edge [out=140,in=60, above] node [below] {$\blank;\goright$} (2)
      (8) edge [] node [right] {$\blank;\goleft$} (11)
      (11) edge [loop left] node [left] {$1;\goleft$} (11)
      (11) edge [] node [right] {$\blank;\goright$} (12);
      \end{scope}
    \end{tikzpicture}
  \end{center}
\end{figure}
\end{framed}

\begin{align*}
  (q_0, \underline{1}1) & \vdash (q_1, \underline{\blank}11) \vdash  (q_2, \underline{\blank} \blank 11) \vdash  (q_3, \underline{\blank} 1 \blank 11)\\
                        & \vdash (q_4, 1\underline{1}\blank 11) \vdash (q_4, 11 \underline{\blank} 11) \vdash (q_5, 11\blank \underline{1}1)\\
                        & \vdash \cdots \vdash (q_7, 11 \blank \underline{1}\blank) \vdash \cdots
\end{align*}

\end{example}


\begin{example}
  Да видим защо тоталната функция $f:\{a,b\}^\star \to \{a,b\}^\star$, дефинирана като
  $f(\alpha) = \alpha\cdot\alpha$ е изчислима с машина на Тюринг.
  
  \begin{itemize}
  \item
    $\Sigma = \{a,b\}$;
  \item 
    $\Gamma = \{a,b,x,A,B\}$;
  \item
    $\qstart = q_0$;
  \item
    $\qaccept = q_6$
  \end{itemize}
  
  \begin{framed}
  \begin{figure}[H]
    \begin{center}
      \begin{tikzpicture}[->,>=stealth,thick,node distance=70pt]
        \tikzstyle{every state}=[circle,minimum size=10pt,auto,scale=.9]
        
        \node[state]            (1) {$q_0$};
        \node[state]            (2) [above of=1]{$q_1$};
        \node[state]            (3) [right of=2]{$q_2$};
        \node[state]            (4) [below of=1]{$q_3$};
        \node[state]            (5) [right of=4]{$q_4$};
        \node[state]            (6) [right of=1]{$q_5$};
        \node[state,accepting]  (7) [right of=6]{$q_6$};
        
        \begin{scope}[every node/.style={scale=.8}]
          \path
          (1) edge [bend left=15] node [left] {$a/x;\goright$} (2)
          (2) edge [loop above] node [above] {$\{a,b,A,B\};\goright$} (2)
          (2) edge [bend left=15] node [above] {$\blank/A;\goleft$} (3)
          (3) edge [loop right] node [right] {$\{a,b,A,B\};\goleft$} (3)
          (3) edge [bend right=15] node [right] {$x/a;\goright$} (1)
          (1) edge [bend right=15] node [left] {$b/x;\goright$} (4)
          (4) edge [loop below] node [below] {$\{a,b,A,B\};\goright$} (4)
          (4) edge [bend right=15] node [below] {$\blank/B;\goleft$} (5)
          (5) edge [loop right] node [right] {$\{a,b,A,B\};\goleft$} (5)
          (5) edge [bend left=15] node [right] {$x/b;\goright$} (1)
          (1) edge [loop left] node [left] {$A/a,B/b;\goright$} (1)
          (1) edge [bend left=15] node [above] {$\blank;\goleft$} (6)
          (6) edge [loop below] node [right] {$\{a,b\};\goleft$} (6)
          (6) edge [bend left=15] node [above] {$\blank;\goright$} (7);
        \end{scope}
      \end{tikzpicture}
    \end{center}
  \end{figure}
\end{framed}

Да проследим работата на $\M$ върху думата $ab$:

\begin{align*}
  (q_0, \underline{a}b) & \vdash (q_1, x\underline{b}) \vdash (q_1,xb\underline{\blank}) \vdash (q_2, x\underline{b}A) \vdash (q_2, \underline{x}bA)\\
                        & \vdash (q_0, a\underline{b}A) \vdash (q_3, ax\underline{A}) \vdash (q_3, axA\underline{\blank}) \vdash (q_4, ax\underline{A}B)\\
                        & \vdash (q_4, a\underline{x}AB) \vdash (q_0, ab\underline{A}B) \vdash (q_0,aba\underline{B}) \vdash (q_0, abab\underline{\blank})\\
                        & \vdash (q_5, aba\underline{b}) \vdash (q_5, ab\underline{a}b) \vdash (q_5, a\underline{b}ab) \vdash (q_5, \underline{a}bab)\\
                        & \vdash (q_5, \underline{\blank}abab) \vdash (q_6, \underline{a}bab).
\end{align*}
\end{example}

\begin{example}
  \marginpar{Изискваме $f(\alpha)$ да започва с $1$ за да може $f$ да бъде функция}
  Да разгледаме тоталната функция 
  \[f:\{0,1\}^\star \to 1\cdot\{0,1\}^\star,\]
  дефинирана като
  \[\ov{f(\alpha)}_{(2)} = \ov{\alpha}_{(2)} + 1.\]
  Нека да видим, че тази функция е изчислима с машина на Тюринг.

  \begin{itemize}
  \item 
    $\Sigma = \{0,1\}$;
  \item
    $\Gamma = \{0,1,\blank\}$;
  \item
    $\qstart = q_0$;
  \item
    $\qaccept = q_4$.
  \end{itemize}

  \begin{framed}
  \begin{figure}[H]
    \begin{center}
      \begin{tikzpicture}[->,>=stealth,thick,node distance=70pt]
        \tikzstyle{every state}=[circle,minimum size=10pt,auto,scale=.9]
        
        \node[state,initial below]    (0) {$q_0$};
        \node[state]            (1) [right of=0]{$q_1$};
        \node[state]            (2) [right of=1]{$q_2$};
        \node[state]            (3) [right of=2]{$q_3$};
        \node[state,accepting]  (4) [right of=3]{$q_4$};
        
        \begin{scope}[every node/.style={scale=.8}]
          \path
          (0) edge [loop above] node [above] {$0/\blank;\goright$} (0)
          (0) edge [bend left=15] node [above] {$1;\goright$} (1)
          (0) edge [bend right=30] node [below] {$\blank;\stay$} (2)
          (1) edge [loop above] node [above] {$\{0,1\};\goright$} (1)
          (1) edge [bend left=15] node [above] {$\blank;\goleft$} (2)
          (2) edge [loop above] node [above] {$1/0;\goleft$} (2)
          (2) edge [bend left=15] node [above] {$0/1;\goleft$} (3)
          (2) edge [bend right=30] node [below] {$\blank/1;\stay$} (4)
          (3) edge [loop above] node [above] {$\{0,1\};\goleft$} (3)
          (3) edge [bend left=15] node [above] {$\blank;\goright$} (4);
        \end{scope}
      \end{tikzpicture}
    \end{center}
  \end{figure}
\end{framed}

Да проследим изчислението на $\M$ върху вход $01011$.

\begin{align*}
  (q_0, 0\underline{1}011) & \vdash (q_0,\underline{1}011) \vdash (q_1, 1\underline{0}11) \vdash (q_1, 10\underline{1}1)\\
                           & \vdash (q_1, 101\underline{1}) \vdash (q_1, 1011\underline{\blank}) \vdash (q_2, 101\underline{1})\\
                           & \vdash (q_2, 10\underline{1}0) \vdash (q_2, 1\underline{0}00) \vdash (q_3, \underline{1}100)\\
                           & \vdash (q_3, \underline{\blank}1100) \vdash (q_4, \underline{1}100).
\end{align*}
\end{example}


\begin{problem}
  Да разгледаме азбуката $\Sigma = \{0,1,\dots,k-1\}$, където $k > 2$.
  Да разгледаме тоталната функция 
  \[f:\Sigma^\star \to (\Sigma\setminus\{0\})\cdot\Sigma^\star,\]
  дефинирана като
  \[\ov{f(\alpha)}_k = \ov{\alpha}_k + 1.\]
  Дефинирайте машина на Тюринг $\M$, която изчислява функцията $f$.
\end{problem}


%%% Local Variables:
%%% mode: latex
%%% TeX-master: "../eai"
%%% End:


\newpage
\section{Недетерминистични машини на Тюринг}
\index{машина на Тюринг!недетерминистична}

Една машина на Тюринг $\N$ се нарича недетерминистична, ако функцията на преходите има вида
\[\Delta_\N: Q\times \Gamma \to \Ps(Q \times \Gamma\times \{\goleft,\goright,\stay\}). \]

Отново можем да дефинираме бинарна релация $\vdash_\N$ над $\Gamma^\star \times Q \times \Gamma^\star$,
която ще казва как моментното описание на машината $\N$ се променя при изпълнение на една стъпка.
\begin{itemize}
\item
  Ако $\Delta_\N(q,z) \ni (p,y,\goright)$, то дефинираме $(\alpha, q, z\beta) \vdash_\N (\alpha y, p, \beta)$.
\item 
  Ако $\Delta_\N(q,z) \ni (p,y,\goleft)$, то дефинираме $(\alpha x, q, z\beta) \vdash_\N (\alpha , p, xy\beta)$.
\item 
  Ако $\Delta_\N(q,z) \ni (p,y,\stay)$, то дефинираме $(\alpha, q, z\beta) \vdash_\N (\alpha, p, y\beta)$.
\end{itemize}
С $\vdash^\star_\N$ ще означаваме рефлексивното и транзитивно затваряне на $\vdash_\N$.

Тогава за недетерминистична машина на Тюринг $\N$, 
\[\L(\N) = \{\alpha\in\Sigma^\star \mid (\varepsilon, \qstart, \alpha) \vdash^\star_\N (\beta, \qaccept, \gamma), \text{ за някои }\beta,\gamma\in\Gamma^\star\}.\]

\begin{remark}
  Върху дадена дума $\omega$, недетерминистичната машина на Тюринг $\N$ може да има много различни изчисления.
  Думата $\omega$ принадлежи на $\L(\N)$ ако съществува {\em поне едно} изчисление, което завършва в състоянието $\qaccept$.
  Възможно е много други изчисления за $\omega$ да завършват в $\qreject$ или да зациклят.
\end{remark}

Аналогично, дефинираме една недетерминистична машина на Тюринг $\N$ да бъде {\bf тотална}, ако за всяка дума и 
всяко изчисление на $\N$ върху $\omega$ завършва в $\qaccept$ или $\qreject$.

\begin{problem}
  \marginpar{\cite{hopcroft2}}
  $\N = (\{q_0,q_1,q_2,q_f\}, \{0,1\}, \{0,1,\blank\}, \Delta, q_0, \{q_f\})$,
  \begin{itemize}
  \item 
    $\Delta(q_0,0) = \{(q_0,1,\goright),(q_1,1,\goright)\}$;
  \item
    $\Delta(q_1,1) = \{(q_2,0,\goleft)\}$;
  \item
    $\Delta(q_2,1) = \{(q_0,1,\goright)\}$;
  \item
    $\Delta(q_1,\blank) = \{(q_f,\blank,\goright)\}$.
  \end{itemize}
  \marginpar{$\{0^{n+1}1^k \mid n,k\in\Nat\}$}
  Опишете $\L(\N)$.
\end{problem}

\begin{example}
  Да разгледаме езика  
  \[L = \{\alpha\sharp\beta \mid \alpha,\beta \in \{a,b\}^\star\ \&\ \alpha\text{ е подниз на }\beta\}.\]
  Ще видим, че този език е разрешим като построим недетерминистична машина на Тюринг $\N$,
  която разрешава този език.
  \begin{framed}
    \begin{figure}[H]
      \begin{center}
        \begin{tikzpicture}[->,>=stealth,thick,node distance=50pt]
          \tikzstyle{every state}=[circle,minimum size=10pt,scale=.9]
          
          \node[state,initial]    (1) {$q_0$};
          \node[state]            (2) [right of=1]{$q_1$};
          \node[state]            (3) [right of=2,node distance=70pt]{$q_2$};
          \node[state]            (4) [below of=3]{$q_3$};
          \node[state]            (5) [below right of=4,node distance=70pt]{$q_4$};
          \node[state]            (6) [right of=4]{$q_5$};
          \node[state]            (7) [above of=6]{$q_6$};
          \node[state]            (8) [right of=6,node distance=80pt]{$q_7$};
          \node[state]            (9) [right of=7,node distance=70pt]{$q_8$};
          \node[state,accepting]  (10)[below right of=5]{$q_{9}$};
          
          \begin{scope}[every node/.style={scale=.8}]
            \path
            (1) edge [loop above] node [above] {$\{a,b\};\goright$} (1)
            (1) edge [bend left=15] node [above] {$\sharp;\goright$} (2)
            (2) edge [loop above] node [above] {$a/\blank,b/\blank;\goright$} (2)
            (2) edge [bend left=15] node [above] {$\{a,b,\blank\};\goleft$} (3)
            (3) edge [loop above] node [above] {$\blank;\goleft$} (3)
            (3) edge [bend right=15] node [left] {$\sharp;\goleft$} (4)
            (4) edge [loop left] node [left] {$\{a,b\};\goleft$} (4)
            (4) edge [bend right=30] node [left] {$\blank;\goright$} (5)
            (5) edge [bend right=15] node [right] {$a/\blank;\goright$} (6)
            (6) edge [loop right] node [right] {$\{a,b\};\goright$} (6)
            (6) edge [bend right=15] node [right] {$\sharp;\goright$} (7)
            (7) edge [loop right] node [right] {$\blank;\goright$} (7)
            (7) edge [bend left=15] node [below] {$a/\blank;\goleft$} (3)
            (8) edge [loop right] node [right] {$\{a,b\};\goright$} (8)
            (5) edge [bend right=30] node [right] {$b/\blank;\goright$} (8)
            (8) edge [bend right=15] node [right] {$\sharp;\goright$} (9)
            (9) edge [loop right] node [right] {$\blank;\goright$} (9)
            (9) edge [bend right=45] node [above] {$b/\blank;\goleft$} (3)
            (5) edge [bend right=15] node [left] {$\sharp;\stay$} (10);
          \end{scope}
        \end{tikzpicture}
      \end{center}
    \end{figure}
  \end{framed}
  Да видим, че $\M$ успешно разпознава, че думата $ab\sharp aabb$ принадлежи на езика $L$.

  \begin{align*}
    (q_0, \underline{a}b\sharp aabb) & \vdash (q_0, a\underline{b}\sharp aabb) \vdash (q_0, ab\underline{\sharp} aabb) \vdash (q_1, ab\sharp\underline{a}abb) \vdash (q_1, ab\sharp\blank\underline{a}bb)\\
                                     & \vdash (q_2, ab\sharp\underline{\blank}abb) \vdash (q_2, ab\underline{\sharp}\blank abb) \vdash (q_3, a\underline{b}\sharp\blank abb) \vdash (q_3, \underline{a}b\sharp\blank abb)\\
                                     & \vdash (q_3, \underline{\blank}ab\sharp\blank abb) \vdash (q_4, \underline{a}b\sharp\blank abb) \vdash (q_5, \blank\underline{b}\sharp \blank abb) \vdash (q_5, \blank b\underline{\sharp} \blank abb)\\
                                     & \vdash (q_6, \blank b \sharp \underline{\blank} abb) \vdash (q_6, \blank b \sharp \blank \underline{a}bb) \vdash (q_2, \blank b \sharp \underline{\blank}\blank bb) \vdash (q_2, \blank b \underline{\sharp} \blank\blank bb)\\
                                     & \vdash (q_3, \blank \underline{b} \sharp \blank\blank bb) \vdash (q_3, \underline{\blank} b \sharp \blank\blank bb) \vdash (q_4,  \blank \underline{b} \sharp \blank\blank bb) \vdash (q_7, \blank \blank \underline{\sharp} \blank\blank bb)\\
                                     & \vdash (q_8, \blank \blank \sharp \underline{\blank}\blank bb)\vdash (q_8, \blank \blank \sharp \blank \underline{\blank} bb) \vdash (q_8, \blank \blank \sharp \blank \blank \underline{b}b)\\
    & \vdash (q_2, \blank \blank \sharp \blank \underline{\blank} \blank b) \vdash \cdots \vdash (q_4, \blank\blank\underline{\sharp}\blank\blank\blank b) \vdash (q_9, \blank\blank\underline{\sharp}\blank\blank\blank b)
  \end{align*}
\end{example}

\subsection*{Канонична подредба на $\Sigma^\star$}

\marginpar{За доказателството, че всяка НМТ е еквивалентна на ДМТ, е необходимо да фиксираме канонична подредба на думите над дадена азбука}
Нека $\Sigma = \{a_0,a_1,\dots,a_{k-1}\}$.
Подреждаме думите по ред на тяхната дължина.
Думите с еднаква дължина подреждаме по техния числов ред, т.е.
гледаме на буквите $a_i$ като числото $i$ в $k$-ична бройна система.
Тогава думите с дължина $n$ са числата от $0$ до $k^n-1$ записани в $k$-ична бройна система.
Ще означаваме с $\omega_i$ $i$-тата дума в $\Sigma^\star$ при тази подредба.

\begin{example}
  Ако $\Sigma = \{0,1\}$, то наредбата започва така:
  \[\varepsilon, 0, 1, \underbrace{00, 01, 10, 11}_{\text{от $0$ до $3$}}, \underbrace{000, 001, 010, 011, 100, 101, 110, 111}_{\text{от $0$ до $7$}}, 0000, 0001, \dots\]
  В този случай, $\omega_0 = \varepsilon$, $\omega_7 = 000$, $\omega_{13} = 110$.
\end{example}

\begin{problem}
  Нека $\Sigma = \{a_0,\dots,a_{k-1}\}$.
  Да разгледаме функцията $f:\Sigma^\star \to \Sigma^\star$, за която 
  $f(\alpha)$ е думата веднага след $\alpha$ в каноничната подредба на $\Sigma^\star$.
  Докажете, че $f$ е изчислима с машина на Тюринг.
\end{problem}
\begin{hint}
  Ако $\Sigma = \{0,1\}$, то машината на Тюринг има следния вид:
  \begin{framed}
    \begin{figure}[H]
      \begin{center}
        \begin{tikzpicture}[->,>=stealth,thick,node distance=70pt]
          \tikzstyle{every state}=[circle,minimum size=10pt,auto,scale=.9]
          
          \node[state,initial]    (1) [right of=0]{$q_1$};
          \node[state]            (2) [right of=1]{$q_2$};
          \node[state]            (3) [right of=2]{$q_3$};
          \node[state,accepting]  (4) [right of=3]{$q_4$};
          
          \begin{scope}[every node/.style={scale=.8}]
            \path
            (1) edge [loop above] node [above] {$\{0,1\};\goright$} (1)
            (1) edge [bend left=15] node [above] {$\blank;\goleft$} (2)
            (2) edge [loop above] node [above] {$1/0;\goleft$} (2)
            (2) edge [bend left=15] node [above] {$0/1;\goleft$} (3)
            (2) edge [bend right=30] node [below] {$\blank/0;\stay$} (4)
            (3) edge [loop above] node [above] {$\{0,1\};\goleft$} (3)
            (3) edge [bend left=15] node [above] {$\blank;\goright$} (4);
          \end{scope}
        \end{tikzpicture}
      \end{center}
    \end{figure}
  \end{framed}
\end{hint}


\begin{framed}
  \begin{thm}
    Ако $L$ се разпознава от {\em недетерминистична} машина на Тюринг $\N$, то $L$
    е разпознава и от {\em детерминистична} машина на Тюринг $\D$.
  \end{thm}
\end{framed}
\begin{proof}
  \marginpar{В \cite[стр. 164]{hopcroft1} не е добре обяснено.}
  Нека имаме недетерминистичната машина на Тюринг $\N$, за която $L = \L(\N)$.
  Една дума $\alpha$ принадлежи на $\L(\N)$ точно тогава, когато съществува изчисление,
  което започва с думата $\alpha$ върху лентата и след краен брой стъпки, следвайки функцията на преходите $\Delta_\N$,
  достига до състоянието $\qaccept$.
  Сложността идва от факта, че за думата $\alpha$ може да имаме много различни изчисления, 
  като само някои от тях завършват в $\qaccept$. Ще построим детерминистична машина на Тюринг,
  която последователно ще симулира всички възможни {\em крайни} изчисления за думата $\alpha$, докато 
  намери такова, което завършва в състоянието $\qaccept$.
  \marginpar{На практика това, което е правим е да представим всички възможни изчисления на $\N$ като $r$-разклонено дърво и да го обходим в широчина, докато не достигнем до $\qaccept$}
  
  Лесно се съобразява, че всяко изчисление на $\N$ може да се представи като 
  крайна редица от елементи на $Q \times \Gamma \times \{\goleft,\goright,\stay\}$.
  Понеже това множество е крайно, то можем на всяка такава тройка да
  съпоставим число в интервала $[1,r]$, където 
  \[r = 3 \cdot |Q| \cdot |\Gamma|.\]
  Оттук следва, че всяко изчисление на $\N$ може да се представи като крайна 
  редица от числа, всяко принадлежащо на интервала $[1,r]$.

  Детерминистичната машина на Тюринг $\D$ има три ленти.
  
  \begin{itemize}
  \item 
    На първата лента съхраняваме входящия низ и {\em тя никога не се променя}.
  \item
    На втората лента ще записваме последователно низове следвайки каноничната подредба на 
    думите над азбуката $\{1,2,\dots,r\}$.
  \item
    На третата лента симулираме изчислението на $\N$ върху думата от първата лента, използвайки изчислението, 
    което е описано на втората лента. Например, ако съдържанието на втората лента е $4,1,2$,
    това означава, че симулираме изчисление от три стъпки като на първата стъпка избираме четвъртата
    възможна тройка, на втората стъпка избираме първата възможна тройка, на третата стъпка избираме втората възможна тройка.
    
    Ако симулацията завърши в състоянието $\qaccept$ на $\N$, то машината $\D$ завършва успешно.
    В противен случай, на втората лента записваме следващия низ; изтриваме третата лента и започваме нова симулация.
  \end{itemize}
\end{proof}

\begin{prop}[Лема на Кьониг]
  \index{Кьониг}
  Ако $T$ е безкрайно дърво с крайно разклонение, то $T$ съдържа безкраен път.
\end{prop}
\begin{hint}
  Дефинираме безкрайния път на стъпки.
  На всяка стъпка избираме този наследник, който е корен на безкрайно дърво.
  Понеже $T$ е безкрайно дърво с крайно разклонение, на всяка стъпка можем да изберем такъв наследник.
\end{hint}

\begin{cor}
  Ако $L$ се разпознава от {\em тотална недетерминистична} машина на Тюринг $\N$, то $L$
  също се разпознава и от {\em тотална детерминистична} машина на Тюринг $\D$.
\end{cor}
\begin{proof}
  Да разгледаме дървото $T$, което представя всички изчисления на тоталната $\N$ при вход думата $\omega$.
  От лемата на Кьониг следва, че $T$ е крайно дърво, защото ако допуснем, че $T$ е безкрайно, то ще има безкрайно дълго изчисление на $\N$,
  което е невъзможно, понеже $\N$ винаги достига до финално състояние.
  \begin{itemize}
  \item 
    Ако $\N$ приема дадена дума $\omega$, то детерминистичната ни симулация на $\N$ ще достигне до изчисление, кодирано като път в $T$, 
    което завършва в състояние $\qaccept$.
  \item
    Ако $\N$ не приема дадена дума $\omega$, то детерминистичната ни симулация на $\N$ ще покаже, че всяко изчисление, кодирано като път в $T$, завършва в състояние $\qreject$.
  \end{itemize}
\end{proof}



%%% Local Variables:
%%% mode: latex
%%% TeX-master: "../eai"
%%% End:



\section{Основни свойства}

\begin{proposition}
  Ако $L$ е разрешим език над азбуката $\Sigma$, то $\Sigma^\star \setminus L$ също е разрешим език.
  \mynote{С други думи, разрешимите езици са затворени относно операцията допълнение.
    След малко в \Proposition{diagonal:accept} ще видим, че това твърдение не е изпълнено за полуразрешими езици.}
\end{proposition}
\begin{hint}
  Нека $L = \L(\M)$, където $\M$ е разрешител.
  Нека $\M'$ е същата като $\M$, само със сменени $\qaccept$ и $\qreject$ състояния.
  Тогава $\M'$ също е разрешител и $\ov{L} = \L(\M')$.
\end{hint}

\begin{proposition}
  Ако $L_1$ и $L_2$ са разрешими езици, то $L_1 \cup L_2$ е разрешим език.
  \mynote{С други думи, разрешимите езици са затворени относно операцията обединение.
    Като следствие получаваме, че всяко \emph{крайно} обединение на разрешими езици е разрешим език.

    \writedown Съобразете, че това твърдение е изпъленено и за полуразрешими езици.}
\end{proposition}
\begin{hint}
  Нека $L_1 = \L(\M_1)$ и $L_2 = \L(\M_2)$.
  Строим нова машина на Тюринг $\M$, която при вход думата $\alpha$
  симулира едновременно изчисленията на $\M_1$ и $\M_2$ върху $\alpha$.
  Това можем да направим като приемем, че $\M$ има две ленти - една за лентата на $\M_1$ и една за лентата на $\M_2$,
  като състоянията на $\M$ ще бъдат елементи на $Q_1 \times Q_2$.
  Ако една от двете машини достигне своето приемащо състояние, то $\M$ приема думата $\alpha$.
  Ако и двете машини достигнат своите отхвърлящи състояния, то $\M$ отхвърля думата $\alpha$.
\end{hint}

% \begin{proposition}
%   Ако $L_1$ и $L_2$ са полуразрешими езици, то $L_1 \cup L_2$ е полуразрешим език.
% \end{proposition}

\begin{proposition}
  Ако $L_1$ и $L_2$ са разрешими езици, то $L_1 \cap L_2$ е разрешим език.
  \mynote{С други думи, разрешимите езици са затворени относно операцията сечение.
    Като следствие получаваме, че всяко \emph{крайно} сечение на разрешими езици е разрешим език.

    \writedown Съобразете, че това твърдение е изпъленено и за полуразрешими езици.}
\end{proposition}
\begin{hint}
  Нека $L_1 = \L(\M_1)$ и $L_2 = \L(\M_2)$.
  Строим нова машина на Тюринг $\M$, която при вход думата $\alpha$
  симулира едновременно изчисленията на $\M_1$ и $\M_2$ върху $\alpha$.
  Ако и двете машини достигнат до приемащите си състояния, то $\M$ приема думата $\alpha$.
  Ако поне една от двете машини достигне до отхвърлящо състояние, то $\M$ отхвърля думата $\alpha$.
\end{hint}

% \begin{corollary}
%   Всяко крайно сечение на разрешими езици е разрешим език.
% \end{corollary}

% \begin{important}
%   \begin{theorem}
%     Разрешимите езици са затворени относно операциите обединение, сечение, допълнение.
%   \end{theorem}
% \end{important}


\index{Клини-Пост}
\begin{important}
  \begin{theorem}[Клини-Пост]
    $L$ и $\ov{L}$ са полуразрешими езици точно тогава, когато $L$ е разрешим език.
  \end{theorem}
\end{important}
\begin{hint}
  Посоката $(\Leftarrow)$ е ясна.
  За посоката $(\Rightarrow)$, нека $L = \L(\M_1)$ и $\ov{L} = \L(\M_2)$.
  Строим разрешител $\M$, която при вход думата $\alpha$ симулира едновременно изчисленията на $\M_1$ и $\M_2$ върху $\alpha$.
  Например, може $\M$ да има две ленти за симулацията на $\M_1$ и $\M_2$.
  Знаем със сигурност, че точно едно от двете симулирани изчисления ще завърши в приемащо състояние.
  Ако това е $\M_1$, то $\M$ приема $\alpha$.
  Ако това е $\M_2$, то $\M$ отхвърля $\alpha$.
\end{hint}


%%% Local Variables:
%%% mode: latex
%%% TeX-master: "../eai"
%%% End:


% \section{Машини на Тюринг като генератори}

% \mynote{\cite[стр. 168]{hopcroft1}}
% \mynote{\cite[стр. 180]{sipser3}}
% \mynote{На англ. се наричат {\em enumerators}}

% Нека да разгледаме един вариант на многолентовите машини на Тюринг, които ще наричаме {\bf генератори}.
% Нека машината на Тюринг да има две ленти, като в началото и двете ленти са празни.
% \begin{itemize}
% \item 
%   Първата лента ще служи за работна лента - върху нея можем да пишем и четем;
% \item
%   Втората лента служи единствено за изход - върху нея можем само да пишем пишем думи; не можем да четем какво вече сме написали върху нея и не можем да пишем върху вече записана клетка. Думите са разделени със специален символ - $\#$.
%   Това означава, че втората лента има вида
%   \[\omega_1\#\omega_2\#\cdots\#\omega_n\#\blank\blank\cdots\]
% \item
%   Езикът, които се извежда от такъв генератор е съставен от думите, които са изписани на изходната лента.
%   Такива езици ще наричаме {\bf изчислимо изброими}.
%   Обърнете внимание, че измежду думите на изходната лента е възможно да има повторения.
%   Ако езикът е безкраен, то машината ще работи безкрайно много време.
% \end{itemize}

% \begin{framed}
%   \begin{thm}
%     Един език $L$ е полуразрешим точно тогава, когато $L$ е изчислимо номеруем.
%   \end{thm}
% \end{framed}
% \begin{proof}
%   $(\Leftarrow)$ Нека $L$ да се номерира от генераторът $E$.
%   Машината на Тюринг $\M$, за която $L = \L(\M)$ ще работи по следния начин:
%   \begin{enumerate}[1)]
%   \item 
%     При вход думата $\omega$, $\M$ започва да симулира $E$;
%   \item
%     Когато се появи дума $\gamma$ върху изходната лента на $E$, сравняваме $\omega$ с $\gamma$;
%   \item
%     Ако $\omega = \gamma$, то отиваме в състоянието $q_{accept}$ на $\M$ и завършваме;
%   \item
%     В противен случай, отиваме обратно на стъпка $2)$.
%   \end{enumerate}

%   $(\Rightarrow)$ Нека сега $L = \L(\M)$. Целта ни е да изведем всички думи на $L$ върху изходната лента.
%   Основният проблем е, че за дадена дума $\omega$, не знаем за колко стъпки трябва да симулираме $\M$ за да сме сигурни дали думата $\omega \in \L(\M)$ или не. Оказва се, че можем да разрешим този проблем като позволяме да извеждаме повторящи се думи.
%   За целта, да подредим всички думи $\omega_1, \omega_2, \dots $ над азбуката $\Sigma$ спрямо каноничната наредба.
%   \begin{enumerate}[1)]
%   \item
%     Нека $s = 1$;
%   \item 
%     Симулираме $\M$ върху думите $\omega_1,\dots,\omega_s$ за $s$ стъпки;
%   \item
%     За всяка от тези думи $\omega_i$, които се приемат от $\M$, записваме ги върху изходната лента.
%   \item
%     Нека $s = s+1$; Отиваме обратно на стъпка $2)$.
%   \end{enumerate}
% \end{proof}

% \begin{remark}
%   В последната конструкция позволяваме думите на един полуразрешим език $L$ да се 
%   извеждат върху изходната лента многократно. Можем лесно да осигурим условието всяка дума на $L$
%   да се извежда точно по веднъж.
%   На стъпка $s = \pair{i,j}$, то проверяваме дали думата $\omega_i$ се приема успешно от $\M$
%   за {\em точно} $j$ на брой стъпки. Само тогава думата се записва на изходната лента.
  
%   Обърнете внимание, че не можем да осигурим условието думите да се извеждат във възходящ ред
%   относно каноничната наредба.
% \end{remark}

% \begin{framed}
%   \begin{thm}
%     Един език $L$ е разрешим точно тогава, когато съществува генератор за $L$, 
%     който изписва думите на $L$ във възходящ ред относно каноничната наредба.
%   \end{thm}
% \end{framed}
% \begin{proof}
%   $(\Rightarrow)$ Нека $L = \L(\M)$. Тази посока е лесна, защото $\M$ е тотална машина,
%   т.е. за всеки вход $\M$ завършва или в $q_{accept}$ или в $q_{reject}$.
%   \begin{enumerate}[1)]
%   \item 
%     Нека $s = 1$;
%   \item
%     Симулираме $\M$ върху думата $\omega_s$.
%   \item
%     Ако симулацията завърши в състояние $q_{accept}$, то записваме $\omega_s$
%     върху изходната лента. 
%   \item
%     Иначе ако симулацията завърши в състояние $q_{reject}$, то нищо не записваме върху изходната лента. 
%   \item
%     Нека $s = s+1$. Отиваме на стъпка $2)$.
%   \end{enumerate}

%   \mynote{Ако имам генератор $G$ за $L$ няма алгоритъм, който да ми каже дали $L$ е безкраен език или не. Това означава, че по код на $G$ няма как ефективно да получа код на $\M$}
%   $(\Leftarrow)$ Ако $L$ е краен, то е ясно, че мога да разпозная езика с краен автомат, което е частен случай на тотална машина на Тюринг.
%   По-интересният случай е когато $L$ е безкраен език.
%   Нека $L$ се генерира от машината на Тюринг $G$ като извежда думите на $L$ във възходящ ред.
%   \begin{itemize}
%   \item 
%     Вход дума $\omega$;
%   \item
%     Симулираме $G$ като гледаме думите, които се извеждат на изходната лента.
%     Ако срещнем думата $\omega$, то завършваме в състояние $q_{accept}$.
%   \item
%     Ако срещнем думата $\gamma$, която е по-голяма от $\omega$ относно каноничната наредба, 
%     то завършваме в състояние $q_{reject}$.
%   \end{itemize}
% \end{proof}

\subsection{Кодиране на машина на Тюринг}

\subsection*{Кодиране на преход}
Да разгледаме прехода $\delta(q_i,X_j) = (q_k,X_l,D_m)$.
Кодираме този преход по следния начин:
\[0^i10^j10^k10^l10^m\]
Да обърнем внимание, че в този двоичен код няма последователни единици и той 
започва и завършва с нула.


За да кодираме една машина на Тюринг $\M$ е достатъчно да кодираме функцията на преходите $\delta$.
Понеже $\delta$ е крайна функция, нека с числото $r$ да означим броя на всички възможни преходи.
По описания по-горе начин, нека $code_i$ е числото в двоичен запис, получено за $i$-тия преход на $\delta$.
Тогава кодът на $\M$ е следното число в двоичен запис:
\[\code{\M} \df 111\ \texttt{code}_1\ 11\ \texttt{code}_2\ 11\ \cdots\ 11\ \texttt{code}_r\ 111.\]
\begin{itemize}
\item
  Лесно се съобразява, че за две МТ $\M$ и $\M'$ с различни функции на преходите, имаме $\code{\M} \neq \code{\M'}$.
% \item
%   Ще казваме, че числото $r\in\Nat$ е {\bf код на} $\M$, ако $r$, записано в двоичен запис представлява думата $\code{\M}$.
%   Оттук нататък, когато пишем $\M_r$, ще имаме предвид машината на Тюринг с код $r$.
% \item
%   Ясно е, че не всяко естествено число е код на машина на Тюринг, но по дадено число $n$
%   има ефективна процедура, която ни казва дали $n$ е код на машина на Тюринг или не.
% \item
  % С $\pair{\M,\omega}$ ще означаваме кода на $\M$ при вход $\omega$ е числото с двоичен запис описанието на $\M$ и след това прикрепена думата $\omega$.
  % При едно число $r = \pair{M,\omega}$, лесно се намира кода на $\M$.
  % Просто започваме да четем двоичния запис на $r$ докато не срещнем за втори път $111$.
  % След това започва думата $\omega$.
% \item
%   Да въведем означението $\M_i$ за произволно ествестено число $i$.
%   Ако $i$ е код на машина на Тюринг $\M$, то $\M_i \df \M$.
%   Ако $i$ не е код на машина на Тюринг, то $\M_i$ е машина на Тюринг с празна функция на преходите.
\end{itemize}

\begin{example}
  Да се даде пример за кода на конкретна машина на Тюринг.
\end{example}


\begin{prop}
  Следните езици са разрешими:
  \begin{itemize}
  \item 
    $L = \{\code{\M} \mid \M\text{ е машина на Тюринг}\}$;
  \item
    $L = \{\code{\M} \mid \M\text{ е детерминистична машина на Тюринг}\}$.
  \end{itemize}
\end{prop}

\begin{remark}
  Следният език {\bf не} е разрешим:
  \[L_{\texttt{tot}} = \{\code{\M} \mid \M\text{ е тотална машина на Тюринг}\}.\]
\end{remark}


%%% Local Variables:
%%% mode: latex
%%% TeX-master: "../eai"
%%% End:


\subsection{Диагоналният език $L_{\texttt{diag}}$}

\newcommand{\Luniv}{L_{\texttt{univ}}}
\newcommand{\Lhalt}{L_{\texttt{halt}}}
\newcommand{\Laccept}{L_{\texttt{accept}}}

\index{език!неполуразрешим}
\begin{important}
  \begin{theorem}
    Езикът 
    \[L_{\texttt{diag}} \df \{\ \code{\M} \mid \M \text{ е машина на Тюринг и }\code{\M} \not\in L(\M)\ \}\]
    не се разпознава от машина на Тюринг, т.е. $L_{\texttt{diag}}$ {\bf не} е полуразрешим език.
  \end{theorem}
\end{important}
\begin{proof}
  Да допуснем, че $L_{\texttt{diag}}$ се разпознава от машина на Тюринг $\M_0$, т.е. 
  \[L_{\texttt{diag}} = \L(\M_0).\]
  Тогава да видим какво имаме за думата $\code{\M_0}$:
  \begin{align*}
    & \code{\M_0} \in L_{\texttt{diag}} \implies \code{\M_0} \in \L(\M_0) \implies \code{\M_0} \not\in L_{\texttt{diag}},\\
    & \code{\M_0} \not\in L_{\texttt{diag}} \implies \code{\M_0} \not\in \L(\code{\M_0}) \implies \code{\M_0} \in L_{\texttt{diag}}.
  \end{align*}
  Достигаме до противоречие.
\end{proof}

\index{език!полуразрешим}
\begin{important}
  \begin{proposition}
    Езикът 
    \[\Laccept \df \{\ \code{\M} \mid \text{$\M$ е машина на Тюринг и }\code{\M} \in \L(\M)\ \}\]
    е полуразрешим, но не е разрешим.
  \end{proposition}  
\end{important}
\begin{hint}
  Лесно се съобразява, че $\Laccept$ е полуразрешим.
  Дефинираме (многолентова) машина на Тюринг $\M'$, която работи по следния начин:
  \begin{itemize}
  \item
    вход дума $\alpha$;
  \item 
    $\M'$ проверява дали $\alpha$ има вида $\code{\M}$,
    за някоя машина на Тюринг $\M$;
  \item
    Ако $\alpha$ няма вида $\code{\M}$,
    то $\M'$ завършва като отхвърля думата $\alpha$.
  \item
    Ако $\alpha = \code{\M}$, 
    то $\M'$ симулира работата на $\M$ върху $\alpha$. Тогава:
    \begin{itemize}
    \item 
      Ако $\M$ завърши след краен брой стъпки като приема $\alpha$,
      то $\M'$ приема $\alpha$.
    \item
      Ако $\M$ завърши след краен брой стъпки като отхвърля $\alpha$,
      то $\M'$ отхвърля $\alpha$.
    \item
      Ако $\M$ никога не завършва върху $\alpha$,
      то $\M'$ също никога не завършва върху $\alpha$.
    \end{itemize}
  \end{itemize}
  Получаваме, че
  \[\alpha \in \Laccept \iff \alpha \in \L(\M'),\]
  откъдето следва, че $\Laccept$ е полуразрешим език.

  Ако допуснем, че $\Laccept$ е разрешим,
  то езикът $\{0,1\}^\star \setminus \Laccept$ би бил разрешим и тогава 
  \[L_{\texttt{diag}} = (\{0,1\}^\star \setminus \Laccept) \cap \{\code{\M} \mid \text{$\M$ е машина на Тюринг}\}\]
  ще е разрешим език, което е противоречие, защото ще следва, че $L_{\texttt{diag}}$ е разрешим език, а той не е е дори полуразрешим.
\end{hint}

\begin{corollary}
  Съществуват езици, които са полуразрешими, но не са разрешими.
\end{corollary}

\begin{corollary}
  Съществуват полуразрешими езици $L$, за които $\ov{L}$ не са полуразрешими.
\end{corollary}

\begin{problem}
  Докажете, че езикът
  \[\Lhalt = \{\code{\M} \mid \M\text{ е машина на Тюринг и }\M\text{ спира при вход }\code{\M}\}\]
  е полуразрешим, но не е разрешим.
\end{problem}



%%% Local Variables:
%%% mode: latex
%%% TeX-master: "../eai"
%%% End:


\subsection{Универсалният език $L_{\texttt{univ}}$}

\begin{framed}
  \begin{thm}
    Езикът 
    \[\Luniv \df \{\ \code{\M} \cdot \omega \mid \text{$\M$ е машина на Тюринг и }\omega\in \L(\M)\ \}\]
    е полуразрешим, но {\bf не} е разрешим.
  \end{thm}
\end{framed}
\begin{hint}
  \marginpar{Разсъждението е много сходно с това защо $\Laccept$ полуразрешим.}
  Първо да съобразим защо $\Luniv$ е полуразрешим език.
  Дефинираме машина на Тюринг $\M'$, която работи по следния начин:
  \begin{itemize}
  \item
    вход дума $\alpha$;
  \item 
    $\M'$ проверява дали $\alpha$ има вида $\code{\M} \cdot \omega$,
    за някоя машина на Тюринг $\M$ и дума $\omega$. Това става лесно, защото $\omega$
    започва веднага след второ срещане на $111$ в $\alpha$.
  \item
    Ако $\alpha = \code{\M} \cdot \omega$, 
    то $\M'$ симулира работата на $\M$ върху $\omega$.
    \begin{itemize}
    \item 
      Ако $\M$ завърши след краен брой стъпки като приеме $\omega$,
      то $\M'$ приема $\alpha$.
    \item
      Ако $\M$ завърши след краен брой стъпки като отхвърли $\omega$,
      то $\M'$ отхвърля $\alpha$.
    \item
      Ако $\M$ никога не завършва върху $\omega$,
      то очевидно $\M'$ също никога не завършва върху $\alpha$.
    \end{itemize}
  \item
    Ако $\alpha$ няма вида $\code{\M} \cdot \omega$,
    то $\M'$ завършва веднага като отхвърля думата $\alpha$.
  \end{itemize}
  Получаваме, че
  \[\alpha \in \Luniv \iff \alpha \in \L(\M').\]
  
  Сега да съобразим защо $\Luniv$ не е разрешим език.
  Имаме, че за произволна дума $\omega$,
  \begin{align*}
    \omega \in \Laccept & \iff (\exists \M)[\M\text{ е М.Т.}\ \&\ \omega = \code{\M}\ \&\ \omega \in \L(\M)]\\
                                       & \iff \omega \cdot \omega \in \Luniv.
  \end{align*}
  Ако допуснем, че $\Luniv$ е разрешим, то тогава $\Laccept$ е разрешим език, което е противоречие.
\end{hint}

\begin{cor}
  Езикът
  \[\ov{\Luniv} \df \{\code{\M} \cdot \omega \mid \code{\M} \text{ е машини на Тюринг и }\omega\not\in \L(\M)\}\]
  {\bf не} е полуразрешим.
\end{cor}





%%% Local Variables:
%%% mode: latex
%%% TeX-master: "../eai"
%%% End:


\section{Критерий за разрешимост}

\mynote{Сипсър нарича $\leq_m$ \emph{mapping reducibility} \cite[235]{sipser3}.}

\begin{important}
  Доказателството, че $\Luniv$ не е разрешим е пример за една обща схема, с която можем да докажем, че даден език не е разрешим:
  \begin{itemize}
  \item 
    Нека имаме езика $K$, за който вече знаем, че не е разрешим.
    В нашия пример, $K = \Laccept$.
  \item
    Питаме се дали някой друг език $L$ е разрешим.
  \item
    Намираме изчислима тотална функция $f$, за която е изпълнено, че:
    \[\omega \in K \iff f(\omega) \in L.\]
    В \Theorem{universal}, това е функцията $f(\omega) = \omega \sharp \omega$.
  \item
    В този случай ще означаваме $K \leq_m L$.
  \item
    Тогава, ако $L$ е разрешим ще следва, че $K$ е разрешим, което е противоречие.
  \end{itemize}
\end{important}

Сега искаме да разгледаме един критерий, който ще ни казва кога един език съставен от кодове на машини на Тюринг е разрешим. С негова помощ ще можем директно да решаваме наглед трудни задачи. Например,
в момента не е очевидно защо следния език не е разрешим:
\begin{align*}
  L_{\texttt{palin}}\{\omega \in \{0,1\}^\star \mid & \ \omega\text{ е код на машина на Тюринг и }\L(\M)\\
                                                    & \text{ съдържа само думи палиндроми}\}.
\end{align*}
След малко ще видим, че според критерия, който ще разгледаме, директно ще можем да заключим, че $L_{\texttt{palin}}$ не е разрешим. Да започнем с няколко примера.

\begin{proposition}
  Докажете, че езикът
  \[L_{\Sigma^\star} \df \{\omega \in \{0,1\}^\star \mid \omega\text{ е код на машина на Тюринг и }\L(\M_\omega) = \Sigma^\star\}\]
  не е разрешим.
\end{proposition}
\begin{proof}
  \mynote{$L_{\Sigma^\star}$ не е дори полуразрешим, но за момента не знаем как да докажем това.}
  Ще покажем, че съществува тотална изчислима функция $f$, за която:
  \[\alpha \in \Laccept \iff f(\alpha) \in L_{\Sigma^\star}.\]

  Ще опишем алгоритъм (формално машина на Тюринг), за който при входна думата $\omega$ прави следното:
  \begin{itemize}
  \item
    Ако $\omega$ не е код на машина на Тюринг, то връщаме $\omega$.
  \item
    Ако $\omega$ е код на машина на Тюринг $\M$, то тук става интересно.
    Връщаме код на друга машина на Тюринг $\M'$, която работи по следния начин:
    \mynote{За различни $\M$ получаваме различни $\M'$.}
    \begin{itemize}
    \item 
      Вход дума $\alpha$;
    \item
      Първоначално $\M'$ не обръща внимание на $\alpha$.
    \item
      $\M'$ симулира работата на $\M$ върху думата $\code{\M}$;
      \begin{itemize}
      \item 
        Ако след краен брой стъпки $\M$ завърши като приеме думата $\code{\M}$,
        то $\M'$ приема думата $\alpha$, т.е. $\M'$ завършва в състоянието $q_{\texttt{accept}}$.
      \item
        Ако след краен брой стъпки $\M$ завърши като отхрърли думата $\code{\M}$,
        то $\M'$ отхвърля думата $\alpha$, т.е. $\M'$ завършва в състоянието $q_{\texttt{reject}}$.
      \item
        В противен случай, $\M$ никога не завършва върху $\code{\M}$.
        Това означава, че $\M'$ никога не завършва върху входа $\alpha$
        и следователно $\M'$ не приема думата $\alpha$.
      \end{itemize}
    \end{itemize}    
  \end{itemize}
  Получаваме, че:
  \begin{align*}
    & \code{\M} \in \Laccept \implies \L(\M') = \Sigma^\star \implies \code{\M'} \in L_{\Sigma^\star},\\
    & \code{\M} \not\in \Laccept \implies \L(\M') = \emptyset \implies \code{\M'} \not\in L_{\Sigma^\star}.
  \end{align*}
  На практика гореописаният алгоритъм дефинира тоталната изчислима функция
  \begin{align*}
    f(\omega) =
    \begin{cases}
      \code{\M'}, & \text{ако }\omega = \code{\M}\\
      \omega, & \text{иначе}.
    \end{cases}
  \end{align*}
  Тогава получаваме, че
  \[\omega \in \Laccept \iff f(\omega) \in L_{\Sigma^\star}\]
  и ако допуснем, че $L_{\Sigma^\star}$ е разрешим език, то $\Laccept$ също ще е разрешим, което е противоречие.
\end{proof}

\begin{corollary}
  Езикът
  \[\ov{L}_{\emptyset} \df \{\omega \in \{0,1\}^\star \mid \omega \text{ е код на машина на Тюринг и }\L(\M_\omega) \neq \emptyset\}\]
  е полуразрешим, но не е разрешим.
\end{corollary}

\begin{corollary}
  Езикът
  \[L_{\emptyset} \df \{\omega \in \{0,1\}^\star \mid \omega\text{ е код на машина на Тюринг и }\L(\M_\omega) = \emptyset\}\]
  не е полуразрешим.
\end{corollary}
\begin{hint}
  Ако $L_{\emptyset}$ беше разрешим, то неговото допълнение
  \[\ov{L}_{\emptyset} = L_{\texttt{code}} \setminus L_{\texttt{Empty}}\]
  щеше да е разрешим език, което е противоречие.

  Ако $L_{\emptyset}$ беше полуразрешим, тогава, използвайки, че $\ov{L}_{\emptyset}$ е полуразрешим, от теоремата на Клини-Пост щеше да следва, че
  $L_{\emptyset}$ е разрешим, което е противоречие
\end{hint}

\begin{problem}
  Докажете, че езикът
  \[L_{\texttt{Dec}} = \{\omega \in \{0,1\}^\star \mid \omega \text{ е код на машина на Тюринг, която е разрешител }\}\]
  не е разрешим.
\end{problem}

\begin{important}
  \begin{proposition}
    Езикът
    \[L_{\texttt{reg}} \df \{\ \omega \mid \omega\text{ е код на машина на Тюринг и }\L(\M_\omega) \text{ е регулярен език}\ \}\]
    не е разрешим.
  \end{proposition}
\end{important}
\begin{proof}
  \mynote{\cite[стр. 219]{sipser3}}
  Да фиксираме един език, за който знаем, че не е регулярен, например, 
  $\{0^n1^n \mid n \in \Nat\}$.
  Дефинираме алгоритъм, за който по вход $\code{\M}$ 
  връща код на машината на Тюринг $\M'$, която работи по следния начин:
  \begin{itemize}
  \item 
    Вход думата $\alpha$;
  \item
    Ако $\alpha = 0^n1^n$, за някое $n$, то $\M'$ приема думата $\alpha$.
  \item
    Ако $\alpha$ не е от вида $0^n1^n$, тогава $\M'$ симулира $\M$ върху думата $\code{\M}$.
    \begin{itemize}
    \item 
      Ако след краен брой стъпки $\M$ завърши като приеме думата $\code{\M}$, то $\M'$ приема $\alpha$.
    \item
      Ако след краен брой стъпки $\M$ завърши като отхвърли думата $\code{\M}$, то $\M'$ отхвърля думата $\alpha$.
    \item
      В противен случай, $\M$ никога не завършва върху $\code{\M}$.
      Това означава, че $\M'$ никога не завършва върху входа $\alpha$
      и следователно $\M'$ не приема думата $\alpha$.
    \end{itemize}
  \end{itemize}
  \mynote{Използваме наготово, че $\Sigma^\star$ е регулярен език.}
  Получаваме, че:
  \begin{align*}
    & \code{\M} \in \Laccept \implies \L(\M') = \Sigma^\star \implies \code{\M'} \in L_{\texttt{reg}},\\
    & \code{\M} \not\in \Laccept \implies \L(\M') = \{0^n1^n \mid n \in \Nat\} \implies \code{\M'} \not\in L_{\texttt{reg}}.
  \end{align*}
  Сега вече трябва да е ясно, че следната тотална функция е изчислима:
  \begin{align*}
    f(\omega) =
    \begin{cases}
      \code{\M'}, & \text{ако }\omega = \code{\M}\\
      \omega, & \text{иначе}.
    \end{cases}
  \end{align*}
  Тогава получаваме, че
  \[\omega \in \Laccept \iff f(\omega) \in L_{\texttt{reg}}\]
  и ако допуснем, че $L_{\texttt{reg}}$ е разрешим език, то $\Laccept$ също ще е разрешим, което е противоречие.  
\end{proof}

Сега ще видим, че идеята, която следвахме в горните доказателства може да се обобщи.
Нека $\Ss$ е множество от полуразрешими езици над фиксирана азбука $\Sigma$.
% \mynote{$\Ss = \{L \mid L\text{ се разпознава от М.Т. с}\\\text{по-малко от 10 състояния}$. Това защо не върши работа?}
Например, 
\[\Ss = \{L \subseteq \Sigma^\star \mid L\text{ е регулярен език}\}.\]
Ще казваме, че $\Ss$ е свойство на полуразрешимите езици.
$\Ss$ е {\bf тривиално свойство}, ако $\Ss = \emptyset$ или $\Ss$ съдържа точно всички полуразрешими езици.
Нека разгледаме изброимото множество от всички машини на Тюринг, които разпознават езиците от $\Ss$.
Ще представим това множество като език от кодовете на тези машини на Тюринг, т.е.
\index{$\texttt{Code}(\Ss)$}
\[\texttt{Code}(\Ss) \df \{\omega \mid \text{$\omega$ е код на машина на Тюринг и } \L(\M_\omega) \in \Ss\}.\]
\index{$\texttt{Code}(L)$}
\mynote{Можем да дефинираме и $\texttt{Code}(L)$, което е безкрайно изброимо множество, ако $L$ е полуразрешим език.}

\begin{important}
  \begin{theorem}[Райс \cite{rice}]
    \index{Райс}
    \mynote{\cite[стр. 188]{hopcroft1}}
    За всяко нетривиално свойство $\Ss$ на полуразрешимите езици,
    $\texttt{Code}(\Ss)$ е неразрешим.
  \end{theorem}
\end{important}
\begin{proof}
  \mynote{Цел: да сведем ефективно $\Laccept$ към $L_\Ss$}
  Без ограничение на общността, нека $\emptyset \not\in \Ss$.
  Понеже $\Ss$ е нетривиално свойство, да разгледаме езика $L \in \Ss$,
  като $\M_L$ е машина на Тюринг, за която $\L(\M_L) = L$.
  Да разгледаме алгоритъм, който по дадена дума $\code{\M}$
  връща код на машина на Тюринг $\M'$, която зависи от $\M$ и от $\M_L$.
  Тя работи по следния начин:
  \begin{itemize}
  \item
    \mynote{Неформално описваме функцията $\delta$ за $\M'$}
    вход думата $\alpha$;
  \item
    първоначално $\M'$ не обръща внимание на $\alpha$;
  \item
    $\M'$ симулира $\M$ върху думата $\code{\M}$.
    \begin{itemize}
    \item
      \mynote{в този случай ще получим, че $\L(\M') = L$}
      ако след краен брой стъпки $\M$ завърши като приеме думата $\code{\M}$, то 
      $\M'$ симулира $\M_L$ върху входната дума $\alpha$;
      \begin{itemize}
      \item
        ако след краен брой стъпки $\M_L$ завърши като приеме думата $\alpha$, то 
        $\M'$ приема $\alpha$;
      \item 
        ако след краен брой стъпки $\M_L$ завърши като отхвърли думата $\alpha$, то 
        $\M'$ отхвърля $\alpha$;
      \item
        ако $\M_L$ никога не завършва върху $\alpha$, то 
        $\M'$ никога няма да завърши върху $\alpha$ и следователно $\M'$
        не приема $\alpha$.
      \end{itemize}
    \item
      \mynote{при тези два случая ще получим, че $\L(\M') = \emptyset$}
      ако след краен брой стъпки $\M$ завърши като отхвърли думата $\code{\M}$, то 
      $\M'$ отхвърля $\alpha$;
    \item
      Ако $\M$ никога не свършва върху $\code{\M}$, то $\M'$ никога няма да свърши върху $\alpha$,
      което означава, че $\M'$ не приема $\alpha$.      
    \end{itemize}
  \end{itemize}
  От всичко това следва, че така описаната машина на Тюринг $\M'$ има свойствата:
  \begin{align*}
    & \code{\M} \in \Laccept \implies \L(\M') = L \implies \L(\M') \in \Ss,\\
    & \code{\M} \not\in \Laccept \implies \L(\M') = \emptyset \implies \L(\M') \not\in \Ss.
  \end{align*}
  Сега вече трябва да е ясно, че гореописаният алгоритъм дефинира тоталната изчислима функция
  \begin{align*}
    f(\omega) =
    \begin{cases}
      \code{\M'}, & \text{ако }\omega = \code{\M}\\
      \omega, & \text{иначе}.
    \end{cases}
  \end{align*}
  Тогава получаваме, че
  \[\omega \in \Laccept \iff f(\omega) \in \texttt{Code}(\Ss)\]
  и ако допуснем, че $\texttt{Code}(\Ss)$ е разрешимо множество, то ще следва, че $\Laccept$ е разрешимо, което е противоречие.

  Ако $\emptyset \in \Ss$, то правим горните разсъждения за класа от езици
  \[\ov{\Ss} = \{ L \subseteq \Sigma^\star \mid L\text{ е полуразрешим език и } L \not\in\Ss\ \}.\]
  По аналогичен начин доказваме, че $\texttt{Code}(\ov{\Ss})$ не е разрешим език.
  Понеже 
  \[\texttt{Code}(\ov{\Ss}) = L_{\texttt{code}} \setminus \texttt{Code}(\Ss),\]
  то $\texttt{Code}(\Ss)$ също не е разрешим език.
\end{proof}

\begin{corollary}
  За всяко от следните свойства $\Ss$ на полуразрешимите езици, 
  $\texttt{Code}(\Ss)$ {\bf не} е разрешим език, където:
  \mynote{Тук няма нужда нищо да доказваме. Просто съобразяваме, че всяко от тези свойства на полуразрешимите езици е нетривиално.}
  \begin{enumerate}[a)]
  \item 
    $\Ss$ е свойството празнота, т.е. езикът
    \[\texttt{Code}(\Ss) = \{\code{\M} \mid \text{$\M$ е машина на Тюринг и } \L(\M) = \emptyset\}\]
    не е разрешим;
  \item 
    $\Ss$ е свойството за пълнота, т.е. езикът
    \[\texttt{Code}(\Ss) = \{\code{\M} \mid \text{$\M$ е машина на Тюринг и } \L(\M) = \Sigma^\star\}\]
    не е разрешим;
  \item
    $\Ss$ е свойството крайност, т.е. езикът
    \[\texttt{Code}(\Ss) = \{\code{\M} \mid \text{$\M$ е машина на Тюринг и }|\L(\M)| < \infty\}\]
    не е разрешим;
  \item
    $\Ss$ е свойството безкрайност, т.е. езикът
    \[\texttt{Code}(\Ss) = \{\code{\M} \mid \text{$\M$ е машина на Тюринг и }|\L(\M)| = \infty\}\]
    не е разрешим;
  \item
    $\Ss$ е свойството регулярност, т.е. езикът
    \[\texttt{Code}(\Ss) = \{\code{\M} \mid \text{$\M$ е машина на Тюринг и $\L(\M)$ е регулярен език}\}\]
    не е разрешим;
  \item
    \mynote{Това свойство е нетривиално, защото вече показахме, че $\{a^nb^nc^n \mid n \in \Nat\}$ е полуразрешим (дори разрешим) език, а знаем отдавна, че този език не е безконтекстен.}
    $\Ss$ е свойството безконтекстност, т.е. езикът
    \[\texttt{Code}(\Ss) = \{\code{\M} \mid \text{$\M$ е машина на Тюринг и $\L(\M)$ е безконтекстен}\}\]
    не е разрешим;
  \item
    \mynote{Тук също - вече сме разгледали примери за полуразрешими езици, които не са разрешими.}
    $\Ss$ е свойството разрешимост, т.е. езикът
    \[\texttt{Code}(\Ss) = \{\code{\M} \mid \text{$\M$ е машина на Тюринг и $\L(\M)$ е разрешим}\}\]
    не е разрешим.
  \end{enumerate}
\end{corollary}


%%% Local Variables:
%%% mode: latex
%%% TeX-master: "../eai"
%%% End:


\section{Критерии за полуразрешимост}

\begin{lemma}
  \mynote{Това означава, че ако $\texttt{Code}(\Ss)$ е полуразрешим език, то всеки език $L_0 \in \Ss$ притежава краен подезик, който също принадлежи на $\Ss$.}
  Нека $\Ss$ е свойство на полуразрешимите езици.
  Ако съществува безкраен език $L_0 \in \Ss$, който няма крайно подмножество в $\Ss$,
  то $\texttt{Code}(\Ss)$ не е полуразрешим език.  
\end{lemma}
\begin{hint}
  Нека $L_0 = \L(\M_0)$ като $L_0 \in \Ss$, но всеки краен подезик на $L_0$ не принадлежи на $\Ss$.
  Сега ще дефинираме тотална изчислима функция $f$, която при вход думата $\omega \in \{0,1\}^\star$ работи по следния начин:
  \begin{itemize}
  \item
    Ако $\omega$ не е код на машина на Тюринг, то $f(\omega) \df \omega$.
  \item
    Ако $\omega$ е код на машината на Тюринг $\M_\omega$, то тогава $f(\omega) = \code{\M'}$,
    където $\M'$ работи така:
    \begin{itemize}
    \item 
      вход думата $\alpha$;
    \item
      $\M'$ симулира работата на $\M_\omega$ върху думата $\omega$:
      \begin{itemize}
      \item
        ако $\M_\omega$ завърши за по-малко от $|\alpha|$ на брой стъпки като \emph{приеме} $\omega$, 
        то $\M'$ завършва веднага като \emph{отхвърля} $\alpha$;
      \item
        в противен случай, $\M'$ симулира работата на $\M_0$ върху $\alpha$.
      \end{itemize}
    \end{itemize}
    Така получаваме, че 
    \begin{align*}
      \L(\M') = 
      \begin{cases}
        \{\alpha \in L_0 \mid \abs{\alpha} < s_0\}, & \text{ако } \M_\omega \text{ приема }\omega\\
        L_0, & \text{ако }\M_\omega \text{ не приема }\omega,
      \end{cases}
    \end{align*}
    където $s_0$ е минималният брой стъпки необходими на $\M_\omega$ за да приеме думата $\omega$.
  \end{itemize}
  
  Заключаваме, че 
  \begin{align*}
    & \M_\omega \text{ не приема }\omega \implies \omega \in L_{\texttt{diag}} \implies \code{\M'} \in \texttt{Code}(\Ss)\\
    & \M_\omega \text{ приема }\omega \implies \omega \not\in L_{\texttt{diag}} \implies \code{\M'} \not\in \texttt{Code}(\Ss).
  \end{align*}
  Понеже
  \[\omega \in L_{\texttt{diag}} \iff f(\omega) \in \texttt{Code}(\Ss),\]
  то $\texttt{Code}(\Ss)$ не е полуразрешим, защото ние знаем, че $L_{\texttt{diag}}$ не е полуразрешим.
\end{hint}

\begin{cor}
  Следните езици {\bf не} са полуразрешими:
  \begin{itemize}
  \item 
    $L = \{\code{\M} \mid \abs{\L(\M)} = \infty\}$;
  \item
    $L = \{\code{\M} \mid \L(\M) = \Sigma^\star\}$;
  \item
    $L = \{\code{\M} \mid \L(\M)\text{ не е разрешим}\}$;
  \item
    $L = \{\code{\M} \mid \L(\M)\text{ не е полуразрешим}\}$;
  \item
    $L = \{\code{\M} \mid \L(\M)\text{ не е регулярен}\}$.
  \end{itemize}
\end{cor}

\begin{lemma}
  \mynote{Това означава, че ако $\texttt{Code}(\Ss)$ е полуразрешим език, ако $L_0 \in \Ss$ и $L_0 \subseteq L_1$, като $L_1$ е полуразрешим, то $L_1 \in \Ss$.}
  Нека $L_1$ е език в $\Ss$ и нека $L_2$ е полуразрешим език, като $L_1 \subset L_2$ и $L_2 \not\in\Ss$.
  Тогава $\texttt{Code}(\Ss)$ не е полуразрешим език.
\end{lemma}
\begin{hint}
  Нека $L_1 = \L(\M_1)$ и $L_2 = \L(\M_2)$.
  Ще дефинираме тотална изчислима функция $f$, която при вход думата $\omega \in \{0,1\}^\star$ работи по следния начин:
  \begin{itemize}
  \item
    Ако $\omega$ не е код на машина на Тюринг, то $f(\omega) = \omega$.
  \item
    Ако $\omega$ е код на машината на Тюринг $\M_\omega$, тогава $f(\omega)$ ще бъде код на машината на Тюринг $\hat{\M}$,
    която работи по следния начин:
    \begin{itemize}
    \item 
      вход думата $\alpha$;
    \item
      $\hat{\M}$ симулира едновременно две изчисления - $\M_1$ върху $\alpha$ и $\M_\omega$ върху $\omega$
      докато намери стъпка $s$, такава че:    
      \begin{itemize}
      \item 
        ако $\M_1$ завършва за $s$ на брой стъпки като приема думата $\alpha$, то $\hat{\M}$ завършва като приема думата $\alpha$;
      \item
        ако $\M_\omega$ завършва за $s$ на брой стъпки като приема думата $\omega$, 
        то $\hat{\M}$ симулира работата $\M_2$ върху $\alpha$.
      \item
        ако $\hat{\M}$ не намери такава стъпка, то е ясно, че $\hat{\M}$ никога не завършва върху $\alpha$.
      \end{itemize}
    \end{itemize}
    Получаваме, че:
    \begin{align*}
      \L(\hat{\M}) = 
      \begin{cases}
        L_2, & \text{ако $\M_\omega$ приема }\omega\\
        L_1, & \text{ако $\M_\omega$ не приема }\omega.
      \end{cases}
    \end{align*}
  \end{itemize}
  Заключаваме, че:
  \[\omega \in L_{\texttt{diag}} \iff f(\omega) \in \texttt{Code}(\Ss),\]
  защото $L_2 \not\in \Ss$, а $L_1 \in \Ss$.
  Това означава, че ефективно можем да сведем въпрос за принадлежност в $L_{\texttt{diag}}$
  към въпрос за принадлежност в $\texttt{Code}(\Ss)$.
  Следователно, ако $\texttt{Code}(\Ss)$ е полуразрешим език, то $L_{\texttt{diag}}$ е полуразрешим език, което е противоречие.  
\end{hint}

\begin{cor}
  Следните езици {\bf не} са полуразрешими:
  \begin{itemize}
  \item 
    $L = \{\code{\M} \mid \L(\M) \text{ е регулярен} \}$;
  \item
    $L = \{\code{\M} \mid \L(\M) \text{ е безконтекстен} \}$;
  \item
    $L = \{\code{\M} \mid \L(\M) \text{ е разрешим} \}$;
  \item
    $L = \{\code{\M} \mid \abs{\L(\M)} = 42\}$.
  \end{itemize}
\end{cor}


\begin{framed}
  \begin{theorem}[Райс-Шапиро]
    Нека $\texttt{Code}(\Ss)$ е полуразрешим език. Тогава е изпълнено, че:
    \[L \in \Ss \iff (\exists L_0 \subseteq \Sigma^\star )[L_0\text{ е краен и }L_0 \subseteq L \implies L_0 \in \Ss].\]
  \end{theorem}
\end{framed}


% % \section{Проблеми за безконтекстни езици}

% % \begin{lemma}
% %   Нека е дадена $\M = \TM$.
% %   Тогава езикът 
% %   \[L = \{\alpha\sharp\beta^R \mid \alpha,\beta \in \Gamma^\star Q \Gamma^\star\ \&\  \alpha \vdash_\M \beta\}\]
% %   е безконтекстен.
% % \end{lemma}
% % \begin{proof}
% %   Ще покажем, че съществува стеков автомат $P$, за който $\L_S(P) = L$.
% %   Четем буквата $X$. Тогава:
% %   \begin{itemize}
% %   \item 
% %     ако $\delta_\M(q,X) =(p,Y,R)$, то слагаме $Yp$ на върха на стека;
% %   \item
% %     ако $\delta_\M(q,X) =(p,Y,L)$, то ако $Z$ е върха на стека, заменяме $Z$ с $pZY$;
% %   \end{itemize}
% % \end{proof}



% % \begin{thm}
% %   Неразрешим е проблемът за проверка дали при дадени две произволни безконтекстни граматики $G_1$ и $G_2$,
% %   $\L(G_1) \cap \L(G_2) = \emptyset$.  
% % \end{thm}

% % \begin{thm}
% %   Неразрешим е проблемът за проверка дали при дадена произволна безконтекстна граматика $G$,
% %   $\L(G) = \Sigma^\star$.  
% % \end{thm}


% % \section{Въпроси}

% % Вярно ли е, че следният проблем е {\em разрешим}:
% % \begin{itemize}
% % \item
% %   за произволна безконтекстна граматика $G$, проверява дали $\L(G) = \emptyset$?
% % \item
% %   за произволна безконтекстна граматика $G$, проверява дали $\L(G) = \Sigma^\star$?
% % \item
% %   за произволни безконтекстни граматики $G_1$ и $G_2$, проверява дали $\L(G_1) \cap \L(G_2) = \emptyset$?
% % \item
% %   за произволни безконтекстни граматики $G_1$ и $G_2$, проверява дали $\L(G_1) \cap \L(G_2) = \Sigma^\star$?
% % \item
% %   за произволни безконтекстни граматики $G_1$ и $G_2$, проверява дали $\L(G_1) = \L(G_2)$?
% % \item
% %   за произволни безконтекстни граматики $G_1$ и $G_2$, проверява дали $\L(G_1) \subseteq \L(G_2)$?
% % \item
% %   за произволна безконтекстна граматика $G$ и произволен регулярен израз $r$,
% %   проверява дали $\L(G) = \L(r)$?
% % \item
% %   за произволна безконтекстна граматика $G$ и произволен регулярен израз $r$,
% %   проверява дали $\L(G) \subseteq \L(r)$?
% % \item
% %   за произволна безконтекстна граматика $G$ и произволен регулярен израз $r$,
% %   проверява дали $\L(r) \subseteq \L(G)$?
% % \item
% %   за произволни безконтекстни граматики $G_1$ и $G_2$, проверява дали $\L(G_1) \subseteq \L(G_2)$ 
% %   е безконтекстен език ?
% % \item
% %   за произволна безконтекстна граматика $G$, проверява дали $\Sigma^\star \setminus \L(G)$
% %   е безконтекстен език ?
% % \item
% %   за произволна безконтекстна граматика $G$, проверява дали $\L(G)$ е регулярен език?
% % \end{itemize}


%%% Local Variables:
%%% mode: latex
%%% TeX-master: "../eai"
%%% End:


% 
\subsection*{Безконтекстен език за преходите в машина на Тюринг}

% \subsection*{Валидни и невалидни изчисления на машини на Тюринг}
\marginpar{\cite{hopcroft1} стр. 201}
Да разгледаме машината на Тюринг $\M$.

Една дума $\omega$ описва конфигурация на машина на Тюринг,
ако $\omega \in \Gamma^\star Q \Gamma^\star$.

\begin{framed}
  \begin{prop}
    Да фиксираме една машина на Тюринг $\M$. 
    Тогава следните езици за безконтекстни:
    \begin{itemize}
    \item 
      $\texttt{ValidStep}(\M) \df \{\ \alpha\#\beta^{rev} \mid \alpha,\beta \in \Gamma^\star Q \Gamma^\star\ \&\ \alpha \vdash_\M \beta\ \}$;
    \item
      $\texttt{ValidStep}'(\M)\df \{\ \alpha^{rev}\#\beta \mid \alpha,\beta \in \Gamma^\star Q \Gamma^\star\ \&\ \alpha \vdash_\M \beta\ \}$;
    \item
      $\texttt{InvalidStep}(\M) \df \{\ \alpha\#\beta^{rev} \mid \alpha,\beta \in \Gamma^\star Q \Gamma^\star\ \&\  \alpha \not\vdash_\M \beta\ \}$;
    \item
      $\texttt{InvalidStep}'(\M) \df \{\ \alpha^{rev}\#\beta \mid \alpha,\beta \in \Gamma^\star Q \Gamma^\star\ \&\ \alpha \not\vdash_\M \beta\ \}$.
    \end{itemize}
  \end{prop}  
\end{framed}

\begin{hint}

  Да напомним първо как дефинираме релацията $\vdash_\M$:
  \begin{align*}
    & (\alpha_1z, q, x\alpha_2) \vdash_\M  (\alpha_1 zy, p, \alpha_2) & \comment{\text{ ако } q \overset{x/y;\goright}{\longrightarrow} p} \\
    & (\alpha_1z, q, x\alpha_2) \vdash_\M (\alpha_1, p ,zy\alpha_2) & \comment{\text{ ако } q \overset{x/y;\goleft}{\longrightarrow} p} \\
    & (\alpha_1z, q, x\alpha_2) \vdash_\M (\alpha_1z, p, y\alpha_2) & \comment{\text{ ако } q \overset{x/y;\stay}{\longrightarrow} p}.
  \end{align*}

  Думите в езика $\texttt{ValidStep}(\M)$ кодират релацията $\vdash_\M$. Това означава, че всяка дума на 
  $\texttt{ValidStep}(\M)$ има някое от следните представяния:
  \begin{align*}
    & \alpha_1zqx\alpha_2 \sharp \alpha^{rev}_2 p y z \alpha^{rev}_1 & \comment{\text{ ако } q \overset{x/y;\goright}{\longrightarrow} p} \\
    & \alpha_1zqx\alpha_2 \sharp \alpha^{rev}_2 y z p \alpha^{rev}_1 & \comment{\text{ ако } q \overset{x/y;\goleft}{\longrightarrow} p} \\
    & \alpha_1zqx\alpha_2 \sharp \alpha^{rev}_2 y p z\alpha^{rev}_1 & \comment{\text{ ако } q \overset{x/y;\stay}{\longrightarrow} p}
  \end{align*}

  Ще опишем неформално стеков автомат $P$ за езика $\texttt{ValidStep}(\M)$.
  Нека 
  \[Q^{P} \df \{r_q \mid q \in Q^\M\} \cup \{r, \hat{r}\}.\]

  \begin{itemize}
  \item
    Първо четем $\alpha_1$ и я записваме в стека като $\alpha^{rev}_1$.
    Това правим като дефинираме функцията на преходите като 
    \[(\forall a,z \in \Sigma)[\ \delta_{P}(r,a,z) \df \{(r,az)\}\ ].\]
  \item 
    Правим това докато не срещнем някое $q \in Q^\M$. Тогава трябва да направим преход на $\M$.
    Тук трябва да внимаваме, защото за да направим преход, трябва да знаем състоянието $q$ и да прочетем следващия символ.
    Един начин да разрешим този проблем е като запомним кое състояние сме прочели на машината на Тюринг в състоянията на стековия автомат:
    \[(\forall q \in Q^\M)(\forall z \in \Sigma)[\ \delta_{P}(r,q,z) = \{(r_q,z)\}\ ].\]
    \begin{itemize}
    \item 
      \marginpar{Стекът представлява $z\alpha^{rev}_1$}
      ако $\delta_\M(q,x) = (p,y,\goright)$, то слагаме $yp$ на върха на стека, т.е.
      \[\delta_{P}(r_q,x,z) = \{(\hat{r}, ypz)\}.\]
    \item
      ако $\delta_\M(q,x) =(p,y,\goleft)$, то ако $z$ е върха на стека, заменяме $z$ с $pzy$, т.е.
      \[\delta_{P}(r_q,x,z) = \{(\hat{r}, pzy)\}.\]
    \item
      ако $\delta_\M(q,x) =(p,y,\stay)$, то ако $z$ е върха на стека, заменяме $z$ с $ypz$, т.е.
      \[\delta_{P}(r_q,x,z) = \{(\hat{r}, ypz)\}.\]
    \end{itemize}
  \item
    Сега вече сме в състояние $\hat{r}$ и остава да прочетем $\alpha_2$ и да я запишем в стека като $\alpha^{rev}_2$:
    \[\delta_{P}(\hat{r},x,z) = \{(\hat{r}, xz)\}.\]
  \end{itemize}
\end{hint}

\begin{remark}
  Да обърнем внимание, че горната конструкция на стековия автомат $P$ е {\bf ефективна}, т.е.
  съществува алгоритъм, който при вход машина на Тюринг $\M$ връща като изход стеков автомат $P$ за езика $\texttt{ValidStep}(\M)$.
  С други думи, езикът 
  \[\{\code{\M} \cdot \code{P} \mid \L(P) = \texttt{ValidStep}(\M)\}\]
  е разрешим.
\end{remark}

\subsection*{История на машина на Тюринг}
\marginpar{история на приемащо изчисление}

Дума от вида  $\omega_1 \sharp \omega^{rev}_2 \sharp \omega_3 \sharp \omega^{rev}_4\sharp\omega_5\cdots$
се нарича {\bf история на приемащо изчисление} на машината на Тюринг $\M$, ако
\begin{itemize}
\item
  $\omega_i \in \Gamma^\star Q \Gamma^\star$, т.е. $\omega_i$ описва моментна конфигурация
  и $\omega_i$ не започва и не завършва на $\blank$.
\item
  $\omega_1 \in \qstart\Sigma^\star$ описва начална конфигурация.
\item
  $\omega_n \in \Gamma^\star \cdot\{\qaccept\} \cdot \Gamma^\star$ описва приемаща конфигурация.
\item 
  $\omega_i \vdash_\M \omega_{i+1}$ за $i = 1,\dots,n-1$.
\end{itemize}

\begin{lemma}
  \marginpar{\cite{hopcroft1}, стр. 201}
  Нека да означим с $\texttt{Accept}(\M)$ езикът от историите на всички приемащи изчисления за машината на Тюринг $\M$.
  Тогава 
  \[\texttt{Accept}(\M) = L_1 \cap L_2,\]
  където $L_1$ и $L_2$ са безконтекстни езици.
  Освен това, граматиките на $L_1$ и $L_2$ могат ефективно да бъдат построени от $\M$.
\end{lemma}
\begin{hint}
  Да разгледаме езиците:
  \begin{align*}
    & L_1 \df (\texttt{ValidStep}(\M)\sharp)^\star(\{\varepsilon\}\cup \Gamma^\star \cdot \{\qaccept\} \cdot \Gamma^\star\sharp)\\
    & L_2 \df \qstart\Sigma^\star \sharp (\texttt{ValidStep}(\M)\sharp)^\star(\{\varepsilon\}\cup \Gamma^\star \cdot \{\qaccept\} \cdot \Gamma^\star\sharp),
  \end{align*}
  за които е ясно, че са безконтекстни.
\end{hint}

\begin{thm}
  Езикът
  \[L = \{\code{G_1}\cdot\code{G_2} \mid \text{$G_1$ и $G_2$ са безконт. грам. и }\L(G_1) \cap \L(G_2) = \emptyset\}\]
  е неразрешим.
\end{thm}
\begin{hint}
  По дадена дума $\code{\M}$, можем ефективно да намерим $G_1$ и $G_2$, за които
  $\L(G_1) \cap \L(G_2)$ са точно валидните изчисления на $\M$.
  Тогава ако $L$ е разрешим език, то $L_{\texttt{Empty}}$ е разрешим език, което е противоречие.
\end{hint}

\begin{lemma}
  За всяка машина на Тюринг $\M$, $\overline{\texttt{Accept}(\M)}$ е безконтекстен език.
\end{lemma}
\begin{hint}
  Една дума $\alpha$ не е история на приемащо изчисление, ако е изпълнено някое от следните условия:
  \begin{itemize}
  \item 
    \marginpar{Можем да опишем това свойство с регулярен език}
    $\alpha$ не е от вида $\omega_1 \sharp \omega_2 \sharp \cdots \sharp \omega_n$,
    където $\omega_i \in \Gamma^\star Q \Gamma^\star$, или
  \item
    ако $\alpha$ е от вида $\omega_1 \sharp \omega_2 \sharp \cdots \sharp \omega_n$,
    където $\omega_i \in \Gamma^\star Q \Gamma^\star$, тогава:
    \begin{itemize}
    \item 
      $\omega_1 \not\in \qstart \Gamma^\star$, или
    \item
      $\omega_n \not\in \Gamma^\star \cdot \{\qaccept\} \cdot \Gamma^\star$, или
    \item
      $\omega_i \not\vdash_\M \omega^{rev}_{i+1}$, за някое нечетно $i$, или
    \item
      $\omega^{rev}_i \not\vdash_\M \omega_{i+1}$, за някое четно $i$.
    \end{itemize}
  \end{itemize}
  Думите притежаващи някое от тези свойства могат да се опишат като обединение на три регулярни езика и двата безконтекстни езика.
\end{hint}

\begin{framed}
  \begin{thm}
    За дадена азбука $\Sigma$, 
    езикът 
    \[\texttt{All}_{\texttt{CFG}} = \{\code{G} \mid G\text{ е безконтекстна граматика и }\L(G) = \Sigma^\star\}\]
    е неразрешим.
  \end{thm}
\end{framed}
\begin{hint}
  По дадена дума $\code{\M}$, можем ефективно да намерим $G$, за която
  $\L(G)$ са точно невалидните изчисления на $\M$.
  Тогава ако допуснем, че $L$ е разрешим език, то $L_{\texttt{Empty}}$ е разрешим, което е противоречие.
\end{hint}

\begin{cor}
  Следните езици не са разрешими:
  \begin{enumerate}[a)]
  \item
    $L = \{\code{G_1}\cdot\code{G_2} \mid \text{$G_1$ и $G_2$ са безконт. грам. и }\L(G_1) = \L(G_2)\}$;
  \item
    $L = \{\code{G_1}\cdot\code{G_2} \mid \text{$G_1$ и $G_2$ са безконт. грам. и }\L(G_1) \subseteq \L(G_2)\}$;
  \item 
    $L = \{\code{G}\cdot r \mid \text{$G$ е безконт. грам. и $r$ е рег. израз и }\L(G) = \L(r)\}$;
  \item
    $L = \{\code{G}\cdot \code{\A} \mid \text{$G$ е безконт. грам. и $\A$ е ДКА и }\L(G) = \L(\A)\}$;
  \item 
    $L = \{\code{G}\cdot r \mid \text{$G$ е безконт. грам. и $r$ е рег. израз и }\L(r) \subseteq \L(G)\}$;
  \item
    $L = \{\code{G}\cdot \code{\A} \mid \text{$G$ е безконт. грам. и $\A$ е ДКА и }\L(\A) \subseteq \L(G)\}$;
  \end{enumerate}
\end{cor}

\begin{remark}
  Добре е да обърнем внимание, че езикът 
  \[L = \{\code{G}\cdot \code{\A} \mid \text{$G$ е безконт. грам. и $\A$ е ДКА и }\L(G) \subseteq \L(\A)\}\]
  е разрешим.
  Това е така, защото $\L(G) \subseteq \L(\A) \iff \L(G) \cap \L(\ov{\A}) = \emptyset$,
  защото сечението на безконтекстен и регулярен език е безконтекстен език.
\end{remark}

\newpage

\begin{framed}
  \begin{prop}
    Езикът 
    \[\texttt{Reg} = \{\code{G} \mid G\text{ е безконтекстна граматика и $\L(G)$ е регулярен}\}\]
    не е разрешим.
  \end{prop}
\end{framed}
\begin{hint}
  Да фиксираме език $L_0$, който е безконтекстен, но не е регулярен.
  За произволен език $L$, да разгледаме езика
  \[\hat{L} \df L_0 \sharp \Sigma^\star\ \cup\ \Sigma^\star \sharp L.\]
  Първо ще докажем, че: 
  \begin{equation}
    \label{eq:2}
    L = \Sigma^\star\ \iff\ \hat{L}\text{ е регулярен}.
  \end{equation}
  Да отбележим, че можем ефективно да получим от безконтекстна граматика $G$ за $L$
  безконтекстна граматика $\hat{G}$ за $\hat{L}$.
  Нека да означим с $\texttt{conv}$ изчислимата функция, за която
  $\texttt{conv}(\code{G}) = \code{\hat{G}}$.

  \begin{itemize}
  \item 
    Ако $L = \Sigma^\star$, то $\hat{L}$ е регулярен, защото тогава
    $\hat{L} = \Sigma^\star \sharp \Sigma^\star$ е очевидно регулярен.
  \item
    \marginpar{Ако $L$ е регулярен, то $L/_\beta \df \{\alpha \mid \alpha\beta \in L\}$ е регулярен}  
    Ако $L \neq \Sigma^\star$, то нека да фиксираме дума $\omega \not\in L$.
    Ако допуснем, че $\hat{L}$ е регулярен, то езикът
    $\hat{L}/_{\sharp\omega} = L_0$ ще е регулярен, което е противоречие с избора на $L_0$.
  \end{itemize}
  
  Нека да означим
  \[\texttt{Full} \df \{\code{G} \mid G\text{ е безконтекстна граматика и }\L(G) = \Sigma^\star\}.\]
  От (\ref{eq:2}) имаме, че 
  \[\code{G} \in \texttt{Full}\ \iff\ \texttt{conv}(\code{G}) \in \texttt{Reg}.\]
  
  Ако допуснем, че $\texttt{Reg}$ е разрешим език, то тогава ще следва, че
  $\texttt{Full}$ е разрешим език, за което вече знаем, че не е вярно.
\end{hint}

% \begin{cor}
%   Нека $G_1$ и $G_2$ са произволни безконтекстни граматики, а $r$ е произволен регулярен израз.
%   Следните проблеми са неразрешими:
%   \begin{enumerate}
%   \item 
%     $\L(G_1) = \L(G_2)$;
%   \item
%     $\L(G_2) \subseteq \L(G_1)$;
%   \item
%     $\L(G_1) = \L(r)$;
%   \item
%     $\L(r) \subseteq \L(G_1)$.
%   \end{enumerate}
% \end{cor}

%%% Local Variables:
%%% mode: latex
%%% TeX-master: "../eai"
%%% End:

% \begin{theorem}[Грейбах 1963]
  \index{Грейбах}
  \mynote{\cite[стр. 205]{hopcroft1}}
  Нека $\mathcal{C}$ е клас от езици, за който съществува ефективно кодиране $\code{L}$ на езиците в $\mathcal{C}$ и който е:
  \mynote{По дадена дума $\omega$ можем ефективно да проверим дали тя кодира език от $\mathcal{C}$ или не.}
  \begin{itemize}
  \item 
    ефективно затворен относно обединение;
  \item
    ефективно затворен относно конкатенация с регулярен език;
  \item
    "$= \Sigma^\star$" е неразрешим за достатъчно голяма $\Sigma$.
  \end{itemize}
  \mynote{Съществуват езици от $\mathcal{C}$, които не притежават свойството $P$ и такива, които го притежават.}
  Нека $P$ е нетривиално свойство на $\mathcal{C}$, което е изпълнено за всеки регулярен език и ако $L \in P$,
  то $L/_a \in P$, където
  \[L/_a = \{\omega \mid \omega a \in L\}.\]
  Тогава езикът $\{\code{L} \mid P(L)\ \&\ L \in \mathcal{C}\}$ е неразрешим.
\end{theorem}
\begin{hint}
  Да фиксираме език $L_0 \in \Cs$, за който {\em не е изпълнено} свойството $P$.
  Нека да приемем, че $L_0 \subseteq \Sigma^\star$, която е достатъчно голяма азбука, за която
  въпроса ``$= \Sigma^\star$'' е неразрешим.
  За произволен език $L \in \mathcal{C}$, да разгледаме езика
  \[\hat{L} \df L_0 \sharp \Sigma^\star\ \cup\ \Sigma^\star \sharp L.\]
  Ясно е, че $\hat{L}\in \mathcal{C}$, защото $\mathcal{C}$ е ефективно затворен относно конкатенация с регулярен език и относно обединение. 
  Първо ще докажем, че: 
  \begin{equation}
    \label{eq:2}
    L = \Sigma^\star\ \iff\ \code{\hat{L}} \in P.
  \end{equation}

  \begin{itemize}
  \item 
    Ако $L = \Sigma^\star$, то $\hat{L}$ е регулярен, защото тогава
    $\hat{L} = \Sigma^\star \sharp \Sigma^\star$ е очевидно регулярен и от избора на $P$, $\code{\hat{L}} \in \mathcal{C}$.
  \item
    \mynote{Ако $\code{L} \in P$ , то за $L/_\beta \df \{\alpha \mid \alpha\beta \in L\}$ е изпълнено $P$.}  
    Ако $L \neq \Sigma^\star$, то нека да фиксираме дума $\omega \not\in L$.
    Ако допуснем, че $\code{\hat{L}} \in P$, то езикът
    за езикът $\hat{L}/_{\sharp\omega} = L_0$ също ще е изпълнено свойството $P$, което е противоречие с избора на $L_0$.
  \end{itemize}

  От (\ref{eq:2}) следва, че $P$ е разрешимо свойство точно тогава, когато въпросът ''$=\Sigma^\star$'' за езиците от $\mathcal{C}$ е разрешим, което е противоречие.
\end{hint}

\begin{corollary}
  Въпросът дали една безконтекстна граматика описва регулярен език е неразрешим.
  По-точно, езикът
  \[\texttt{Reg} = \{\code{G} \mid G\text{ е безконтекстна граматика и }\L(G)\text{ е регулярен език}\}\]
  е неразрешим.
\end{corollary}
\begin{proof}
  Ясно е, че имаме ефективно кодиране на безконтекстните граматики $\code{G}$ и освен това те са
  ефективно затворени относно конкатенация с регулярен език и относно обединение.
  Вече знаем от \Theorem{computations:all-cfg}, че $= \Sigma^\star$ за безконтекстни граматики е неразрешим за достатъчно голяма азбука $\Sigma$.
  Тогава от теоремата на Грейбах следва, че $\texttt{Reg}$ е неразрешим език.
\end{proof}


%%% Local Variables:
%%% mode: latex
%%% TeX-master: "../eai"
%%% End:

% \newpage
% \section{Неограничени граматики}
\index{граматика!неограничена}

\begin{definition}
  \mynote{\cite[стр. 220]{hopcroft1}}
  \mynote{На англ. unrestricted grammar}
  \mynote{Според йерархията на Чомски, това е граматика от тип 0}
  Граматиката $G = (V,\Sigma,R,S)$
  се нарича неограничена граматика, 
  ако правилата $R$ са от вида $\alpha \to \beta$,
  където $\alpha,\beta \in (V\cup\Sigma)^\star$.
\end{definition}

\begin{lemma}
  За всеки полуразрешим език $L$, $L = \L(G)$, за някоя неограничена граматика $G$.  
\end{lemma}
\begin{proof}
  Нека $L = \L(\M)$, където 
  \[\M = \TM\] е детерминистична машина на Тюринг,
  като искаме лентата да е безкрайна само отдясно и входната дума $\alpha$ е
  поставена в началото на лентата.
  Ще построим граматика $G = \CFG$, където 
  \[V = ((\Sigma\cup\{\varepsilon\})\times\Gamma) \cup \{A_1,A_2,A_3\}.\]
  Правилата на $G$ са следните:
  \begin{enumerate}[1)]
  \item 
    $A_1 \to sA_2$;
  \item
    $A_2 \to [a,a]A_2$, за всяка $a\in\Sigma$;
  \item
    $A_2 \to A_3$;
  \item
    $A_3 \to [\varepsilon,\blank]A_3$;
  \item
    $A_3 \to \varepsilon$;
  \item
    $q[a,X] \to [a,Y]p$, за всяка $a \in \Sigma\cup\{\varepsilon\}$, всяко $q\in Q$, $X,Y \in\Gamma$, 
    за които $\delta(q,X) = (p,Y,R)$;
  % \item
  %   $q[a,X] \to p[a,Y]$, за всяка $a \in \Sigma\cup\{\varepsilon\}$, всяко $q\in Q$, $X,Y \in\Gamma$, 
  %   за които $\delta(q,X) = (p,Y,N)$;
  \item
    $[b,Z]q[a,X] \to p[b,Z][z,Y]$, за всяко $X,Y,Z \in \Gamma$, $a,b\in\Sigma\cup\{\varepsilon\}$, $q\in Q$,
    за които $\delta(q,X) = (p,Y,L)$;
  \item
    $[a,X]q \to qaq$, $q[a,X] \to qaq$, $q \to \varepsilon$, за всяко $a\in\Sigma\cup\{\varepsilon\}$, $X\in\Gamma$,
    и $q \in F$.
  \end{enumerate}
  
  Лесно се вижда, че, използвайки правилата 1) и 2), за всяко $n$, имаме
  \[A_1 \to^\star s[a_1,a_1]\cdots[a_n,a_n]A_2,\]
  където $a_i \in \Sigma$.

  Нека $\M$ приема думата $\alpha = a_1\cdots a_n$.
  Това означава, че за някое $m$, $\M$ използва не повече от $m$ клетки от лентата отдясно на входната дума.
  Ясно е, че имаме
  \[A_1 \to^\star s[a_1,a_1]\cdots[a_n,a_n][\varepsilon,\blank]^m.\]
  Оттук нататък, можем да използваме само правилата 6), 7), 8), докато не срещнем финално състояние.
  С индукция по броя на стъпки в $\M$, можем да докаже, че ако е изпълнено
  $(\varepsilon,s,a_1\cdots a_n) \vdash^\star_\M (X_1\cdots X_{r-1},q,X_r\cdots X_l)$, 
  то \[s[a_1,a_1]\dots[a_n,a_n][\varepsilon,\blank]^m \rightarrow^\star_G [a_1,X_1]\cdots[a_{r-1},X_{r-1}]q[a_r,X_r]\cdots[a_{n+m},X_{n+m}],\]
  където $a_1,\dots,a_n \in \Sigma$, $a_{n+1},\dots,a_{n+m} = \varepsilon$, $X_1,\dots,X_{n+m} \in \Gamma$ и
  $X_{l+1} = X_{l+2} = \dots = X_{n+m} = \blank$.
  
  Най-накрая, ако $q \in F$, то можем да използваме правилата от 9) и да докажем, че
  \[[a_1,X_1]\cdots[a_{t-1},X_{t-1}]q[a_t,X_t]\cdots[a_{n+m},X_{n+m}] \rightarrow^\star_G a_1\cdots a_n.\]
  
  Така доказахме, че ако $\alpha \in \L(\M)$, то $\alpha \in \L(G)$, т.е. $\L(\M) \subseteq \L(G)$.
  За да докажем обратната посока, трябва да направим подобни разсъждения.
\end{proof}

\begin{lemma}
  Ако $L = \L(G)$, където $G$ е неограничена граматика, то $L$ е полуразрешим език.
\end{lemma}
\mynote{Доказателствата в \cite{hopcroft1} и \cite{papadimitriou} са различни}
\begin{proof}
  $\M$ ще бъде недетерминистична машина с три ленти.
  \begin{enumerate}[1)]
  \item
    Записваме входната дума $\omega$ на първата лента на $\M$.
    Тя никога не се променя.
  \item
    На втората лента ще имаме думата $\gamma \in (V\cup\Sigma)^\star$.
    В началото $\gamma := S$.
  \item 
    Недетерминистично избираме правило $\alpha \to \beta$ от граматиката $G$.
  \item
    Недетерминистично избираме $\gamma_0,\gamma_1 \in (V\cup\Sigma)^\star$, за които 
    $\gamma = \gamma_0\alpha\gamma_1$.
    Тогава $\gamma := \gamma_0\beta\gamma_1$.
    Ако няма такива $\gamma_0$ и $\gamma_1$, то $\M$ ,,зацикля'' - текущият опит за извеждане на $\omega$ пропада.
  \item
    Сравняваме съдържанието на първите две ленти, т.е. проверяваме дали $\omega = \gamma$.
    Ако $\omega = \gamma$, то спираме и казваме, че $\M$ разпознава думата $\omega$.
    Ако $\omega \neq \gamma$, то се връщаме на стъпка 3).
  \end{enumerate}

  \begin{algorithm}[H]
  \caption{}
%  \label{alg:}
  \begin{algorithmic}[1]
    \State $\gamma:= S$
    \ForAll{$\alpha\to\beta \in R$}
    \If{$(\exists \gamma_0,\gamma_1\in (V\cup\Sigma)^\star)[\gamma = \gamma_0\alpha\gamma_1]$}
    \State $\gamma := \gamma_0\beta\gamma_1$
    \Else ...
    \EndIf
    \EndFor
  \end{algorithmic}
\end{algorithm}

\end{proof}

\begin{example}
  Граматика за $L = \{a^nb^nc^n \mid n\in\Nat\}$.
\end{example}

%%% Local Variables:
%%% mode: latex
%%% TeX-master: "../eai"
%%% End:


% \section{Контекстни граматики}

\index{граматика!контекстна}
Казваме, че $G = (V,\Sigma,R,S)$ е {\bf контекстна граматика}, ако правилата на $G$ са от вида
\[\alpha_1 A \alpha_2 \to \alpha_1 \beta \alpha_2,\]
където $\alpha_1,\alpha_2 \in (V\cup\Sigma)^\star$ и $\beta \in (V\cup\Sigma)^+$.

\begin{problem}
  Езикът $L = \{a^nb^nc^n \mid n \in \Nat\ \&\ n > 0\}$ е контекстен.
\end{problem}
\begin{hint}
  Разгледайте контекстната граматика $G$ зададена със следните правила:
  \begin{align*}
    & S \to aSBC\ |\ aBC,\hspace*{0.2cm} CB \to BC\\
    & aB \to ab,\ bB \to bb,\ bC \to bc,\ cC \to cc.
  \end{align*}

  Докажете, че
  \begin{itemize}
  \item
    $S \to^\star_Ga^n(BC)^n$;
  \item
    $(BC)^n \to^\star_G B^nC^n$;
  \item
    $aB^n \to^\star_G ab^n$;
  \item
    $bC^n \to^\star_G bc^n$.
  \end{itemize}
\end{hint}

\begin{proposition}
  Класът на безконтекстните езици строго се включва в класа на контекстните езици.
\end{proposition}

\begin{proposition}
  Всеки контекстен език е разрешим.
\end{proposition}

\begin{proposition}
  Съществува разрешим език, който не е контекстен.
\end{proposition}


%%% Local Variables:
%%% mode: latex
%%% TeX-master: "../eai"
%%% End:


\section{Сложност}

\begin{itemize}
\item
  \index{машина на Тюринг!детерминистично полиномиално ограничена}
  Казваме, че детерминистичната машина на Тюринг $\M$ е {\bf полиномиално ограничена}, ако 
  същестува полином $p(x)$, такъв че няма дума $\omega$,
  за която $\M$ да извършва при вход $\omega$ повече от $p(|\omega|)$ стъпки.
\item
  \index{език!детерминистично полиномиално разрешим}
  Езикът $L$ се нарича {\bf детерминистично полиномиално разрешим},
  ако съществува полиномиално ограниченен детерминистичен разрешител $\M$, за който $L = \L(\M)$.
  Нека
  \[\mathcal{P} \df \{L \subseteq \Sigma^\star \mid L\text{ е полиномиално разрешим с ДМТ}\}.\]
\item
  \index{машина на Тюринг!полиномиално ограничена}
  Казваме, че детерминистичната машина на Тюринг $\M$ е {\bf експоненциално ограничена}, ако 
  същестува полином $p(x)$, такъв че няма дума $\omega$,
  за която $\M$ да извършва при вход $\omega$ повече от $2^{p(|\omega|)}$ стъпки.
\item
  \index{език!детерминистично експоненциално разрешим}
  Езикът $L$ се нарича {\bf детерминистично експоненциално разрешим},
  ако съществува експоненциално ограниченен детерминистичен разрешител $\M$, за който $L = \L(\M)$.
  Нека
  \[\mathcal{EXP} \df \{L \subseteq \Sigma^\star \mid L\text{ е експоненциално разрешим с ДМТ}\}.\]
\item
  \index{машина на Тюринг!недетерминистично полиномиално ограничена}
  Казваме, че недетерминистичната машина на Тюринг $\N$ е {\bf полиномиално ограничена}, ако 
  същестува полином $p(x)$, такъв че няма дума $\omega$,
  за която $\N$ да извършва при вход $\omega$ повече от $p(|\omega|)$ стъпки.
\item
  \index{език!недетерминистично полиномиално разрешим}
  Езикът $L$ се нарича {\bf недетерминистично полиномиално разрешим},
  ако съществува полиномиално ограничена недетерминистичен разрешител $\N$,
  за който $L = \L(\N)$. Нека
  \[\mathcal{NP} \df \{L \subseteq \Sigma^\star \mid L\text{ е полиномиално разрешим с НМТ}\}.\]
\end{itemize}

% \begin{framed}
%   \begin{dfn}
%     \begin{align*}
%       & \mathcal{P} \df \{L \subseteq \Sigma^\star \mid L\text{ е полиномиално разрешим с ДМТ}\};\\
%       & \mathcal{EXP} \df \{L \subseteq \Sigma^\star \mid L\text{ е експоненциално разрешим с ДМТ}\};\\
%       & \mathcal{NP} \df \{L \subseteq \Sigma^\star \mid L\text{ е полиномиално разрешим с НМТ}\}.
%     \end{align*}
%   \end{dfn}
% \end{framed}

\begin{problem}
  Докажете, че класът $\mathcal{P}$ е затворен относно допълнение, обединение, сечение и конкатенация.
\end{problem}

\begin{problem}
  Докажете, че класът $\mathcal{P}$ е затворен относно операцията звезда на Клини.
\end{problem}




\begin{thm}
  $\mathcal{NP} \subseteq \mathcal{EXP}$.
\end{thm}

\begin{proposition}
  За азбука $\Sigma$ от поне две букви, можем да обобщим някои от резултатите от предишните глави:
  \[\texttt{REG} \subsetneqq \texttt{CFG} \subsetneqq \mathcal{P}.\]
\end{proposition}
\begin{hint}
  Езикът $\{a^nb^nc^n \mid n \in \Nat\} \in \mathcal{P}$,
  но не е безконтекстен.
\end{hint}


%%% Local Variables:
%%% mode: latex
%%% TeX-master: "../eai"
%%% End:


\section{Задачи}

\begin{problem}
  Вярно ли е, че следните езици са разрешими?
  \begin{enumerate}[a)]
  \item
    $\{\code{\A}\cdot \omega \mid \A \text{ е ДКА и } \omega \in \L(\A)\}$;
  \item
    $\{\code{\A} \mid \A \text{ е ДКА и } \L(\A)\text{ е безкраен език}\}$;
  \item
    $\{\code{\A} \mid \A \text{ е ДКА и }\L(\A) = \{0,1\}^\star\}$;
  \item 
    $\{\code{\A} \mid \A \text{ е ДКА и }\L(\A)\text{ съдържа поне една дума с равен брой нули и единици}\}$;
  \item
    $\{\code{\A} \mid \A \text{ е ДКА и }\L(\A)\text{ съдържа поне една дума палиндром}\}$;
  \item
    $\{\code{\A} \mid \A \text{ е ДКА и }\L(\A)\text{ не съдържа дума с нечетен брой единици}\}$;
  \item
    $\{\code{\A}\cdot\code{\B} \mid \A\text{ и }\B \text{ са ДКА и }\L(\A) \subseteq \L(\B)\}$;
  \item
    $\{\code{\A}\cdot\code{\B} \mid \A\text{ и }\B \text{ са ДКА и }\L(\A) = \L(\B)\}$;
  \end{enumerate}
\end{problem}

\begin{problem}
  Вярно ли е, че следните езици са разрешими?
  \begin{enumerate}[a)]
  \item
    $\{\code{G} \cdot \omega \mid G \text{ е безконтекстна граматика и } \omega \in \L(G)\}$;
  \item
    $\{\code{G} \mid G \text{ е безконтекстна граматика и } \L(G) = \emptyset\}$;
  \item 
    $\{\code{G} \mid G \text{ е безконтекстна граматика над }\{0,1\}^\star\text{ и }\L(1^\star) \subseteq \L(G)\}$;
  \item 
    $\{\code{G} \mid G \text{ е безконтекстна граматика над }\{0,1\}^\star\text{ и }\L(1^\star) \subseteq \L(G)\}$;
  \item 
    $\{\code{G} \mid G \text{ е безконтекстна граматика над }\{0,1\}^\star\text{ и }\varepsilon \in \L(G)\}$;
  \item
    $\{\code{G}\cdot 0^k \mid G \text{ е безконтекстна граматика над }\{0,1\}^\star\text{ и }|\L(G)| \leq k\}$;
  \item
    $\{\code{G} \mid G \text{ е безконтекстна граматика над }\{0,1\}^\star\text{ и }|\L(G)| = \infty\}$;
  \end{enumerate}
\end{problem}


\begin{problem}
  Докажете, че езикът
  \[L = \{\code{\M}\sharp\omega \mid \M\text{ прави движение наляво при работата си върху вход }\omega\}\]
  е разрешим.
\end{problem}
\begin{hint}
  Нужно е да симулирате работата на $\M$ върху $\omega$ само за $|\omega| + |Q^\M| + 1$ на брой стъпки.
\end{hint}

\begin{problem}
  Докажете, че езикът
  \[L = \{\code{\M}\sharp\omega \mid \M\text{ прави опит за движение наляво от най-лявата клетка при работата си върху вход }\omega\}\]
  е разрешим.
\end{problem}


%%% Local Variables:
%%% mode: latex
%%% TeX-master: "../eai"
%%% End:


% \section{Варианти на машини на Тюринг}

\begin{itemize}
\item
  Четящата глава не отива наляво от началната позиция;
\item
  Никога на пише $\blank$.
\end{itemize}

Така имаме по-лесен начин за представянето на моментното описание на едно изчисление на машина на Тюринг като дума:
\[\hat{\Gamma}^\star \cdot Q \cdot \hat{\Gamma}^\star \cup \hat{\Gamma}^\star \cdot \{\blank\}^\star \cdot Q \cdot \{\blank\},\]
където $\hat\Gamma = \Gamma \setminus\{\blank\}$.


%%% Local Variables:
%%% mode: latex
%%% TeX-master: "../eai"
%%% End:


% \section{Линейни автомати}
\mynote{На англ. linear bounded automaton}
\index{линеен автомат}

{\bf Линеен автомат} е машина на Тюринг, на която не се позволява четящата глава да излиза извън частта от лентата, върху която първоначално е записана входната дума.

\begin{theorem}
  Езикът
  \[L = \{\code{\M}\sharp \omega \mid \M\text{ е линеен автомат и } \omega \in \L(\M)\}\]
  е разрешим.
\end{theorem}
\begin{proof}
  Това е лесно, защото изчислението на $\M$ върху входната дума $\omega$
  може да се намира в една от $|Q|\cdot|\Gamma|^{|\omega|}\cdot |\omega|$ конфигурации.
\end{proof}


\begin{theorem}
  Езикът
  \[L = \{\code{\M} \mid \M\text{ е линеен автомат и } \L(\M) = \emptyset\}\]
  е неразрешим.
\end{theorem}
\begin{proof}
  
\end{proof}


%%% Local Variables:
%%% mode: latex
%%% TeX-master: "../eai"
%%% End:


% \section{Проблемът за съответствието на Пост}\label{sect:turing:pcp}

\mynote{На англ. Post's correspondence problem \cite[стр. 392]{hopcroft2}, но по-добре е обяснено в \cite[стр. 227]{sipser3}. Тези двойки от думи се наричат домино.}

Пример за проблема за съответствието на Пост се нарича всяка крайна редица от елементи на $\Sigma^\star \times \Sigma^\star$,
които ние ще означаваме така:
\[\begin{bmatrix} \alpha_1\\ \beta_1\end{bmatrix},\begin{bmatrix} \alpha_2\\ \beta_2\end{bmatrix},\dots,\begin{bmatrix} \alpha_n\\ \beta_n\end{bmatrix}.\]
Всяка една редица от този вид се нарича {\em пример} за \PCP.
\mynote{Ако $|\alpha_i| = |\beta_i|$ за всяко $i$, то задачата е тривиална.}
Една непразна редица от индекси $i_1,i_2,\dots,i_n$ се нарича {\em решение} на \PCP примера, ако е изпълнено, че:
\[\alpha_{i_1}\alpha_{i_2}\cdots\alpha_{i_n} = \beta_{i_1}\beta_{i_2}\cdots\beta_{i_n}.\]

\begin{problem}
  \mynote{\cite[стр. 239]{sipser3}}
  Намерете решение на следния пример за \PCP:
  \[\begin{bmatrix}ab\\ abab\end{bmatrix},\begin{bmatrix} b\\ a\end{bmatrix},\begin{bmatrix} aba\\ b\end{bmatrix},\begin{bmatrix} aa\\ a\end{bmatrix}.\]
\end{problem}
\begin{solution}
  \[\begin{bmatrix}ab\cdot ab \cdot aba \cdot b \cdot b \cdot aa \cdot aa\\abab \cdot abab \cdot b \cdot a \cdot a \cdot a \cdot a\end{bmatrix}\]
\end{solution}


\subsection*{Модифициран проблем за съответствието }

\mynote{Тук искаме винаги да започваме с първото домино.}
Казваме, че \MPCP има решение, ако съществува произволна редица от индекси $i_1,\dots,i_n$ (може и празна), такава че:
\[\alpha_1\alpha_{i_1}\cdots\alpha_{i_n} = \beta_1\beta_{i_1}\cdots\beta_{i_n}.\]

\begin{lemma}
  Съществува алгоритъм, който свежда \MPCP към \PCP.
\end{lemma}
\begin{proof}
  Нека имаме пример за \MPCP:
  \[\begin{bmatrix} \alpha_1\\ \beta_1\end{bmatrix},\begin{bmatrix} \alpha_2\\ \beta_2\end{bmatrix},\dots,\begin{bmatrix} \alpha_k\\ \beta_k\end{bmatrix} .\]
  Нека символите $\star,\$$ не са от $\Sigma$.
  Нека за думата $\alpha = a_1\cdots a_n$ да дефинираме следните операции:
  \begin{align*}
    & \star\alpha = \star a_1 \star a_2\cdots \star a_n\\
    & \alpha\star = a_1\star a_2\star\cdots a_n \star\\
    & \star\alpha\star = \star a_1\star a_2 \star \cdots a_n\star.
  \end{align*}
  Тогава на базата на горния пример за \MPCP, строим пример за \PCP:
  \[\begin{bmatrix*}[l] \star\alpha_1\star\\ \star\beta_1\end{bmatrix*},\begin{bmatrix} \alpha_1\star\\ \star \beta_1\end{bmatrix},\dots,\begin{bmatrix} \alpha_k\star\\ \star\beta_k\end{bmatrix},\begin{bmatrix*}[r] \$\\ \star\$\end{bmatrix*}.\]
  Така ние показахме, че
  \[\MPCP \leq_m \PCP.\]
\end{proof}

\begin{corollary}
  Ако $\PCP$ е разрешим, то $\MPCP$ също е разрешим.
\end{corollary}

Ясно е, че проблемът на Пост е полуразрешим. Сега ще видим, че той не е разрешим.

\begin{framed}
  \begin{theorem}[Е. Пост \cite{pcp}]\index{Пост}
    Проблемът за съответствието на Пост е неразрешим при азбука $\Sigma$ с поне два символа.
  \end{theorem}
\end{framed}
\begin{hint}
  \mynote{Лесно се съобразява, че за азбука $\Sigma$ само с една буква проблемът е разрешим.}
  Нека приемем, че работим с машини на Тюринг, които не движат главата си наляво от левия край на лентата.
  Ще докажем, че $\Luniv \leq_m \MPCP$. Вече знаем, че \MPCP се свежда алгоритмично към \PCP, т.е. $\MPCP \leq_m \PCP$.
  Това означава, че ще опишем работата на тотална изчислима функция $f$, за която
  \[\gamma \in \Luniv \iff f(\gamma) \in \MPCP.\]
  Сега неформално ще опишем работата на функцията $f$.
  Нека фиксираме символа $\sharp \not \in \Gamma$.
  \begin{enumerate}[1)]
  \item
    Нека имаме като вход дума $\gamma = \code{\M}\sharp \alpha$.
  \item
    \mynote{Горната част на доминото се опитва да настигне долната част.}
    Започваме като добавяме за думата $\alpha = a_1\cdots a_n$ над азбуката $\Sigma$ следната двойка
    $\begin{bmatrix*}[l] \sharp\\ \sharp qa_1\cdots a_n\sharp\end{bmatrix*}$.
  \item
    Ако $\delta(q,a) = (p,b,\goright)$, то добавяме двойката
    $\begin{bmatrix*}[l] qa\\ bp\end{bmatrix*}$.
  \item
    Ако $\delta(q,\blank) = (p,b,\goright)$, то добавяме и двойката $\begin{bmatrix*}[l] q\sharp\\ bp\sharp\end{bmatrix*}$.
  \item
    \mynote{Тук е важно, че не позволяваме четящата глава да се мести по-наляво от първата клетка върху която е четящата глава при стартиране на изчислението.}
    Ако $\delta(q,a) = (p,b,\goleft)$, то добавяме двойките
    $\begin{bmatrix*}[l] xqa\\ pxb\end{bmatrix*}$.
  \item
    Ако $\delta(q,\blank) = (p,b,\goleft)$, то добавяме двойките
    $\begin{bmatrix*}[l] xq\sharp\\ pxb\sharp\end{bmatrix*}$.
  \item
    Ако $\delta(q,a) = (p,b,\stay)$, то добавяме двойката
    $\begin{bmatrix*}[l] qa\\ pb\end{bmatrix*}$.
  \item
    за всеки $x \in \Gamma$, добавяме $\begin{bmatrix} x\\ x\end{bmatrix}$.
    Освен това, добавяме и двойката $\begin{bmatrix} \sharp\\ \sharp\end{bmatrix}$.
  \item
    \mynote{Когато достигнем до приемащо състояние, то започваме да трием съдържанието на доминото за да можем да изравним двете части на доминото.}
    За всеки $x \in \Gamma$, добавяме двойката
    $\begin{bmatrix*}[l] x\qaccept\\ \qaccept\end{bmatrix*}$ и $\begin{bmatrix*}[l] \qaccept x\\ \qaccept\end{bmatrix*}$.
  \item
    За да завършим, добавяме двойката
    $\begin{bmatrix*}[r] \qaccept\sharp\sharp\\ \sharp\end{bmatrix*}$.
  \end{enumerate}
\end{hint}

\begin{corollary}
  Проблемът за еднозначност на безконтекстна граматика е неразрешим.
\end{corollary}
\begin{hint}
  Нека да означим
  \[\texttt{AMBIG} = \{\code{G} \mid G \text{ е нееднозначна безконтекстна граматика}\}.\]
  Да разгледаме един пример за $\PCP$ над азбуката $\Sigma$:
  \[\begin{bmatrix} \alpha_1\\ \beta_1\end{bmatrix},\begin{bmatrix} \alpha_2\\ \beta_2\end{bmatrix},\dots,\begin{bmatrix} \alpha_n\\ \beta_n\end{bmatrix}.\]
  По него можем ефективно да построим следната безконтекстна граматика:
  \begin{align*}
    & S \to A\ |\ B\\
    & A \to \alpha_1A c_1\ |\ \alpha_2 A c_2\ |\ \cdots\ |\ \alpha_n A c_n\ |\ \alpha_1c_1\ |\ \alpha_2c_2\ |\ \cdots\ |\ \alpha_nc_n\\
    & B \to \beta_1B c_1\ |\ \beta_2 B c_2\ |\ \cdots\ |\ \beta_n B c_n\ |\ \beta_1c_1\ |\ \beta_2c_2\ |\ \cdots\ |\ \beta_nc_n,
  \end{align*}
  където $c_1,\dots,c_n \not \in \Sigma$.
  Лесно се съобразява, че горния пример за $\PCP$ има решение точно тогава, когато безконтекстната граматика е нееднозначна.
  С други думи, показахме, че
  \[\PCP \leq_m \texttt{AMBIG}.\]
\end{hint}

\begin{corollary}\label{cor:pcp:grammar-intersect}
  Проблемът за сечение на две безконтекстни граматики е неразрешим.
\end{corollary}
\begin{hint}
  Нека да означим
  \[\INTERSECT = \{\code{G_1}\sharp\code{G_2} \mid \L(G_1) \cap \L(G_2) \neq \emptyset \}.\]
  Да разгледаме един пример за $\PCP$ над азбуката $\Sigma$:
  \[\begin{bmatrix} \alpha_1\\ \beta_1\end{bmatrix},\begin{bmatrix} \alpha_2\\ \beta_2\end{bmatrix},\dots,\begin{bmatrix} \alpha_n\\ \beta_n\end{bmatrix}.\]
  По него можем ефективно да построим следните две безконтекстни граматика с правила:
  \begin{align*}
    & S_1 \to \alpha_1S_1 c_1\ |\ \alpha_2 S_1 c_2\ |\ \cdots\ |\ \alpha_n S_1 c_n\ |\ \alpha_1c_1\ |\ \alpha_2c_2\ |\ \cdots\ |\ \alpha_nc_n\\
    & S_2 \to \beta_1S_2 c_1\ |\ \beta_2 S_2 c_2\ |\ \cdots\ |\ \beta_n S_2 c_n\ |\ \beta_1c_1\ |\ \beta_2c_2\ |\ \cdots\ |\ \beta_nc_n,
  \end{align*}
  където $c_1,\dots,c_n \not \in \Sigma$.
  Така показахме, че
  \[\PCP \leq_m \INTERSECT.\]
\end{hint}



%%% Local Variables:
%%% mode: latex
%%% TeX-master: "../eai"
%%% End:


% \section*{Бележки}

% \begin{itemize}
% \item
%   За основните дефиниции следваме основно Глава 3 от \cite{sipser3}.
% \item 
%   За въпросите за неразрешимост следваме основно Глава 8 от \cite{hopcroft1}.
% % \item
% %   За по-задълбочено запознаване с теория на изчислимостта, добри уводни книги са
% %   \cite{ditchev-soskov} и \cite{nikolova-soskova}.
% \end{itemize}


%%% Local Variables:
%%% mode: latex
%%% TeX-master: "../eai"
%%% End:


% \include{outro}

\bibliographystyle{amsalpha}
\bibliography{eai}

\printindex

\end{document}

%%% Local Variables: 
%%% mode: latex
%%% TeX-master: t
%%% End: 

