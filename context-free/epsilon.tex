\subsubsection*{Премахване на $\varepsilon$-правила}
\index{$\varepsilon$-правила}

\begin{lemma}
  Съществува {\em експоненциален} алгоритъм, такъв че превръща всяка безконтекстна граматика $G$ в безконтекстна граматика $G'$ без правила от вида $A \to \varepsilon$,
  и $\L(G') = \L(G) \setminus \{\varepsilon\}$.
\end{lemma}
\begin{hint}
  За да премахнем правилата от вида $A \to \varepsilon$ от граматиката, следваме процедурата:
  \begin{enumerate}[1)]
  \item 
    \mynote{На всяка стъпка трябва да обходим цялата граматика, следователно всяка итерация отнема $\mathcal{O}(|G|)$ време.}
    Първо % трябва да намерим множеството
    % \[\texttt{E} = \{A \in V \mid A \yield{\star}_G \varepsilon\}.\]
    % Това правим като
    строим множествата $\texttt{E[n]}$ по следния начин:
    \begin{itemize}[-]
    \item
      $\texttt{E[0]} \df \emptyset$;
    \item
      \mynote{Обърнете внимание, че имаме $\texttt{E[n]}^\star$, а
        не просто $\texttt{E[n]}$. Също така да напомним, че $\emptyset^\star = \{\varepsilon\}$.}
      $\texttt{E[n+1]} \df \{ B \in V \mid (\exists \alpha \in \texttt{E[n]}^\star)[B \to_G \alpha ]\}$.
    \item
      Докажете, че за всяко $n$ е изпълнено, че
      \[\texttt{E[n]} = \{A \in V \mid A \yield{\leq n} \varepsilon\}.\]
    %   $A \in \texttt{E[n]}$ точно тогава, когато съществува дърво на извод $P$, за което
    % $A = \texttt{root}(P)$, $\texttt{yield}(P) = \varepsilon$ и $\texttt{height}(P) \leq n$.
    \item
      \mynote{Намираме $\texttt{E}$ за време $\mathcal{O}(|G|^2)$.}
      Спираме на първото $n$, за което $\texttt{E[n]} = \texttt{E[n+1]}$.
      Лесно се съобразява, че за това $n$ е изпълнено, че
      \[\texttt{E[n]} = \{A \in V \mid A \yield{\star} \varepsilon\}.\]
    \end{itemize}
  \item
    Строим множеството от правила $R'$, в което няма $\varepsilon$-правила по следния начин.
    За всяко правило $A \to_G X_1\cdots X_k$,
    добавяме към $R'$ всички правила от вида $A \to_G Y_1\cdots Y_k$, където:
    \mynote{Броят на правилата може да се увеличи експоненциално, защото в най-лошия случай извеждаме всички подмножества на дадено множество от променливи}
    \begin{itemize}[-]
    \item 
      ако $X_i \not\in \texttt{E}$, то $Y_i = X_i$;
    \item
      ако $X_i \in \texttt{E}$, то $Y_i = X_i$ или $Y_i = \varepsilon$;
    \item
      не всички $Y_i$ са $\varepsilon$.
    \end{itemize}
    \mynote{\writedown Докажете сами!}
    Лесно се съобразява, че $\L(G') = \L(G) \setminus \{\varepsilon\}$.
    % \begin{itemize}
    % \item
    %   Нека $A \stackrel{l}{\to_G} \omega$ за $\omega \neq \varepsilon$. Тогава $A \to^\star_{G'} \omega$.
    % \item
    %   Нека $A \stackrel{l}{\to_{G'}} \omega$ за $\omega \neq \varepsilon$. Тогава $A \to^\star_{G} \omega$.
    % \end{itemize}
  \end{enumerate}
\end{hint}

{\scriptsize
\begin{multicols}{2}
  \begin{example}
  Нека е дадена граматиката $G$ с правила
  \begin{align*}
    & S \to D\\
    & D \to AD\ |\ b\\
    & A \to AB\ |\ BC\ |\ a\\
    & B \to AA\ |\ UC\\
    & C \to \varepsilon\ |\ CA\ |\ a\\
    & U \to \varepsilon\ |\ aUb.
  \end{align*}

  Тогава
  \begin{itemize}[-]
  \item
    $\texttt{E[0]} = \emptyset$;
  \item
    $\texttt{E[1]} = \{C,U\}$;
  \item
    $\texttt{E[2]} = \{C,U,B\}$;
  \item
    $\texttt{E[3]} = \{C,U,B,A\}$;
  \item
    $\texttt{E[4]} = \texttt{E[3]}$. Следователно,
    \[\texttt{E} = \{X \in V \mid X \derive{\star} \varepsilon\} = \{C,U,B,A\}.\]
  \end{itemize}
  
  Това означава, че $\varepsilon \not\in \L(G)$.
  Граматиката $G'$ без $\varepsilon$-правила, за която $\L(G') = \L(G)$ има следните правила
  \begin{align*}
    & S \to D \\
    & D\to AD\ |\ D\ |\ b \\
    & A \to A\ |\ B\ |\ C\ |\ AB\ |\ BC\ |\ a \\
    & B\to A\ |\ C\ |\ AA\ |\ U\ |\ UC\\
    & C \to C\ |\ A\ |\ CA\ |\ a\\
    & U \to aUb\ |\ ab.
  \end{align*}
\end{example}
\end{multicols}

}

%%% Local Variables:
%%% mode: latex
%%% TeX-master: "../eai"
%%% End:
