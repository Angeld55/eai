\subsubsection*{Премахване на $\varepsilon$-правила}
\index{$\varepsilon$-правила}
За да премахнем правилата от вида $A \to \varepsilon$, следваме процедурата:
\marginpar{Броят на правилата може да се увеличи експоненциално, защото в най-лошия случай извеждаме всички подмножества на дадено множество от променливи}
\begin{enumerate}[1)]
\item 
  Намираме множеството $E = \{A \in V \mid A \to^\star \varepsilon\}$ по следния начин.
  Първо, $E := \{A \in V \mid A \to \varepsilon\}$.
  След това, за всяко правило от вида $B \to X_1\cdots X_k$, 
  ако всяко $X_i \in E$, то добавяме $B$ към $E$.
\item
  Строим множеството от правила $R'$, в което няма правила $\varepsilon$-правила по следния начин.
  За всяко правило $A \to x_1\cdots x_k$, където $x_i \in V\cup\Sigma$,
  добавяме към $R'$ всички правила от вида $A \to y_1\cdots y_k$, където:
  \begin{itemize}[-]
  \item 
    ако $x_i \not\in E$, то $y_i = x_i$;
  \item
    ако $x_i \in E$, то $y_i = x_i$ или $y_i = \varepsilon$;
  \item
    не всички $y_i$-та са $\varepsilon$.
  \end{itemize}
\end{enumerate}

\begin{example}
  Нека е дадена граматиката $G$ с правила
  \begin{align*}
    & S \to D\\
    & D \to AD\ |\ b\\
    & A \to AB\ |\ BC\ |\ a\\
    & B \to AA\ |\ UC\\
    & C \to \varepsilon\ |\ CA\ |\ a\\
    & U \to \varepsilon\ |\ aUb.
  \end{align*}
  % \[S\rightarrow D,D\rightarrow AD|b,A\rightarrow AB|BC|a, B\rightarrow AA|EC,C\rightarrow \varepsilon|CA|a, E\rightarrow \varepsilon|aEb.\]
  Тогава $E = \{X \in V \mid X \rightarrow^\star_G \varepsilon\} = \{A,B,C,U\}$.
  Това означава, че $\varepsilon \not\in \L(G)$.
  Граматиката $G'$ без $\varepsilon$-правила, за която $\L(G') = \L(G)$ има следните правила
  \begin{align*}
    & S \to D\\
    & D\to AD\ |\ D\ |\ b\\
    & A \to A\ |\ B\ |\ C\ |\ AB\ |\ BC\ |\ a\\
    & B\to A\ |\ E\ |\ C\ |\ AA\ |\ UC\\
    & C \to C\ |\ A\ |\ CA\ |\ a\\
    & U \to aUb\ |\ ab.
  \end{align*}
\end{example}

%%% Local Variables:
%%% mode: latex
%%% TeX-master: "../eai"
%%% End:
