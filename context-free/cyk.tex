\subsection{Проблемът за принадлежност}

\begin{theorem}
  Съществува {\em полиномиален} алгоритъм относно дължината на входната дума, който проверява дали дадена дума принадлежни на граматиката $G$.
\end{theorem}
% \begin{proof}[стр. 154 от \cite{papadimitriou}]
Можем да приемем, че $G = \CFG$ е граматика в нормална форма на Чомски.
% Нека $\alpha = a_0a_1\dots a_{n-1}$ е дума, за която искаме да проверим дали $\alpha \in \L(G)$.
\mynote{Това е алгоритъм на Cocke, Younger и Kasami (CYK), които откриват идеята за този алгоритъм независимо един от друг. Той е пример за динамично
  програмиране \cite[стр. 192]{kozen}, \cite[стр. 141]{shallit}. При вход дума $\alpha$, алгоритъмът работи за време $\mathcal{O}(\abs{\alpha}^3)$.}
\begin{algorithm}[H]
  \caption{Проверка дали $\alpha \in \L(G)$}
  \label{alg:belongs-to-grammar}
  \begin{algorithmic}[1]
    \State $n = \texttt{len}(\alpha)$ \Comment{Вход дума $\alpha = a_0a_1\cdots a_{n-1}$}
    \ForAll{$i < n$}
    \State $\texttt{V[i][i]} := \{A \in V \mid A\to_G \alpha\slice{i}\}$ \label{alg:cyk:initial}
    \ForAll{$j \in (i,n)$}
    % \If{$i = j$}
    % \State $\texttt{V[i][i]} = \{A \in V \mid A\to_G \alpha\slice{i}\}$ \label{alg:cyk:initial}
    % \Else
    \State{$\texttt{V[i][j]} := \emptyset$}
    % \EndIf
    \EndFor
    \EndFor
    \ForAll{$s \in [1, n)$} \Comment{Дължина на интервала} \label{alg:cyk:first-loop}
    \ForAll{$i < n-s$}\Comment{Начало на интервала}
    \ForAll{$k \in [i, i + s)$}\Comment{Разделяне на интервала}
    \If{$\exists A\to BC \in R\ \&\ B \in \texttt{V[i][k]}\ \&\ C\in \texttt{V[k+1][i+s]}$}
    \State $\texttt{V[i][i+s]} := \texttt{V[i][i+s]} \cup \{A\}$ \label{alg:cyk:add-variable}
    \EndIf
    \EndFor
    \EndFor
    \EndFor
    \If{$S \in \texttt{V[0][n-1]}$}
    \State \Return \texttt{True}\Comment{Има извод на думата от $S$}
    \Else
    \State \Return \texttt{False}
    \EndIf
  \end{algorithmic}
\end{algorithm}

\begin{lemma}
  Нека е дадена безконтекстната граматика $G$ в нормална форма на Чомски и думата $\alpha$.
  Всеки път, точно преди да изпълни ред (\ref{alg:cyk:first-loop}) от програмата описана в Алгоритъм \ref{alg:belongs-to-grammar},
  е изпълнено, че за всяко $i < n-s$,
  \[\texttt{V[i][i+s]} \df \{A \in V \mid A \derive{\star}_G \alpha\slice{i:i+s+1}\}.\]
\end{lemma}
\begin{proof}
  Пълна индукция по $s$.
  За $s = 0$ е ясно, защото от ред (\ref{alg:cyk:initial}) и от факта, че граматиката е в нормална форма на Чомски следва, че за всяко $i < n$, 
  \[\texttt{V[i][i]} = \{A \in V \mid A \to_G \alpha\slice{i}\} = \{A \in V \mid A \derive{\star}_G \alpha\slice{i:i+1}\}.\]
  
  Сега ще докажем твърдението за $s > 0$, т.е. ще докажем, че за всяко $i < n-s$ е изпълнено, че:
  \[\texttt{V[i][i+s]} = \{A \in V \mid A \derive{\star}_G \alpha\slice{i:i+s+1}\}.\]
  Да разгледаме двете посоки на това равенство.
  \begin{description}
  \item[($\subseteq$)]
    Нека $A \in \texttt{V[i][i+s]}$.
    Единствената стъпка на алгоритъма, при която може да сме добавили променливата $A$ към множеството $\texttt{V[i][i+s]}$ е ред (\ref{alg:cyk:add-variable}).
    Тогава имаме, че съществува $k$, което $i \leq k < i+s$, и 
    \begin{align*}
      & B \in \texttt{V[i][k]},\\
      & C \in \texttt{V[k+1][i+s]},\\
      & A\to_G BC.
    \end{align*}
    Понеже $k-i < s$ и $i + s - k < s$, от \IndHyp имаме, че
    \begin{align*}
      & B \derive{\star}_G \alpha\slice{i:k+1} & \comment i+(k-i) = k\\
      & C \derive{\star}_G \alpha\slice{k+1:i+s+1}. & \comment (k+1)+(i+s-k) = (k+1)+s
    \end{align*}
    Заключаваме веднага, че 
    \[A \derive{\star}_G \alpha\slice{i:i+s+1}.\]
  \item[($\supseteq$)]
    Нека $A \derive{\star}_G \alpha\slice{i:i+s+1}$ и да разгледаме първото правило, което сме приложили в този извод.
    Понеже $G$ е в нормална форма на Чомски, правилото е от вида $A \to_G BC$ и тогава съществува някое $k$, за което $i \leq k < i+s$, и
    \begin{align*}
      & B \derive{\star} \alpha\slice{i:k+1} \\
      & C \derive{\star} \alpha\slice{k+1:i+s+1}.
    \end{align*}
    От \IndHyp получаваме, че $B \in \texttt{V[i][k]}$ и $C \in \texttt{V[k+1][i+s]}$.
    Тогава от ред (\ref{alg:cyk:add-variable}) на алгоритъма е ясно, че 
    \[A \in \texttt{V[i][i+s]}.\]
  \end{description}
\end{proof}

{\scriptsize

\begin{multicols}{2}
  
\begin{example}
  Нека е дадена граматиката $G$ в нормална форма на Чомски с правила 
  \begin{align*}
    & S\rightarrow a\ |\ AB\ |\ AC\\
    & A\rightarrow a\\
    & B\rightarrow b\\
    & C\rightarrow SB\ |\ AS.
  \end{align*}
  Ще приложим CYK алгоритъма за да проверим дали $aaabb \in \L(G)$.
  
  \begin{itemize}
  \item 
    За $s = 0$ имаме, че:
    \begin{itemize}
    \item 
      $\texttt{V[0][0]} = \texttt{V[1][1]} = \texttt{V[2][2]} = \{S,A\}$;
    \item
      $\texttt{V[3][3]} = \texttt{V[4][4]} = \{B\}$.
    \end{itemize}
  \item
    За $s = 1$ имаме, че:
    \begin{itemize}
    \item
      $\texttt{V[0][1]} = \texttt{V[1][2]} = \{C\}$;
    \item
      $\texttt{V[2][3]} = \{S,C\}$;
    \item
      $\texttt{V[3][4]} = \emptyset$.
    \end{itemize}
  \item
    За $s = 2$ имаме, че:
    \begin{itemize}
    \item
      $\texttt{V[0][2]} = \{S\} \cup \emptyset$;
    \item
      $\texttt{V[1][3]} = \{S,C\} \cup \emptyset$;
    \item
      $\texttt{V[2][4]} = \emptyset \cup \{C\}$.
    \end{itemize}
  \item
    За $s = 3$ имаме, че:
    \begin{itemize}
    \item
      $\texttt{V[0][3]} = \{S,C\} \cup \emptyset \cup \emptyset = \{S,C\}$;
    \item
      $\texttt{V[1][4]} = \{S\} \cup \emptyset \cup \{C\} = \{S,C\}$.
    \end{itemize}
  \item
    За $s = 4$ имаме, че:
    \begin{itemize}
    \item 
      $\texttt{V[0][4]} = \{S,C\} \cup \emptyset \cup \emptyset \cup \{C\}= \{S,C\}$.
    \end{itemize}
  \end{itemize}
  Понеже $S \in \texttt{V[0][4]}$, то $aaabb \in \L(G)$.
\end{example}
\end{multicols}

}


%%% Local Variables:
%%% mode: latex
%%% TeX-master: "../eai"
%%% End:
