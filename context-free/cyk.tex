\subsection{Проблемът за принадлежност}

\begin{thm}
  Съществува {\em полиномиален} алгоритъм относно дължината на входната дума, който проверява дали дадена дума принадлежни на граматиката $G$.
  \marginpar{За дума $\alpha$, алгоритъмът работи за време $\mathcal{O}(\abs{\alpha}^3)$.}
\end{thm}
% \begin{proof}[стр. 154 от \cite{papadimitriou}]
Можем да приемем, че $G = \CFG$ е граматика в нормална форма на Чомски.
Нека $\alpha = a_1a_2\dots a_n$ е дума, за която искаме да проверим дали $\alpha \in \L(G)$.
\marginpar{Това е алгоритъм на Cocke, Younger и Kasami (CYK), които откриват идеята за този алгоритъм независимо един от друг. Той е пример за динамично програмиране \cite[стр. 192]{kozen}.}
\begin{algorithm}[H]
  \caption{Проверка дали $\alpha \in \L(G)$}
  \label{alg:belongs-to-grammar}
  \begin{algorithmic}[1]
    \State $n := \texttt{len}(\alpha)$ \Comment{Вход дума $\alpha = a_1\cdots a_n$}
    \ForAll{$i\in [1,n]$}
    \State $V[i,i] := \{A \in V \mid A\to_G a_i\}$
    \EndFor
    \ForAll{$i,j \in [1,n]\ \&\ i \neq j$}
    \State $V[i,j] := \emptyset$
    \EndFor      
    \ForAll{$s \in [1, n)$} \Comment{Дължина на интервала}
    \ForAll{$i \in [1, n-s]$}\Comment{Начало на интервала}
    \ForAll{$k \in [i, i + s)$}\Comment{Разделяне на интервала}
    \If{$\exists A\to BC \in R\ \&\ B \in V[i,k]\ \&\ C\in V[k+1,i+s]$}
    \State $V[i,i+s] := V[i,i+s] \cup \{A\}$
    \EndIf
    \EndFor
    \EndFor
    \EndFor
    \If{$S \in V[1,n]$}
    \State \Return \texttt{True}\Comment{Има извод на думата от $S$}
    \Else
    \State \Return \texttt{False}
    \EndIf
  \end{algorithmic}
\end{algorithm}

\begin{lemma}
  Нека е дадена безконтекстната граматика $G$ в нормална форма на Чомски и думата $\alpha$.
  Всеки път, точно преди да изпълни ред 6 от програмата описана в Алгоритъм \ref{alg:belongs-to-grammar},
  е изпълнено, че за всяко $i = 1,2,\dots,n-s$,
  \[V[i,i+s] = \{A \in V \mid A \rightarrow^\star_G a_i\dots a_{i+s}\}.\]
\end{lemma}
\begin{proof}
  Пълна индукция по $s$.
  За $s = 0$ е ясно, защото от редове 2 и 3 и от факта, че граматиката е в нормална форма на Чомски следва, че за всяко $i = 1, \dots, n$,
  \[V[i,i] = \{A \in V \mid A \to_G a_i\} = \{A \in V \mid A \to^\star_G a_i\}.\]

  Нека твърдението е вярно за $s < n$. Ще докажем твърдението за $s+1$, т.е. ще докажем, че за всяко $i = 1,\dots,n-s-1$
  е изпълнено, че:
  \[V[i,i+s+1] = \{A \in V \mid A \rightarrow^\star_G a_i\dots a_{i+s+1}\}.\]
  Да разгледаме двете посоки на това равенство.
  % Да разгледаме $A \in V[i,i+s+1]$.
  \begin{description}
  \item[($\subseteq$)]
    Нека $A \in V[i,i+s+1]$.
    Единствената стъпка на алгоритъма, при която може да сме добавили променливата $A$ към множеството $V[i,i+s+1]$ е ред 10.
    Тогава имаме, че съществува $k$, $i \leq k < i+s+1$, за което
    \begin{align*}
      & B \in V[i,k],\\
      & C \in V[k+1,i+s+1],\\
      & A\to_G BC.
    \end{align*}
    Понеже $k-i < s+1$ и $i + s - k < s+1$, от {\bf И.П.} имаме, че
    \begin{align*}
      & B \to^\star_G a_i\cdots a_k\\
      & C \to^\star_G a_{k+1}\cdots a_{i+s+1}.
    \end{align*}
    Заключаваме веднага, че 
    \[A \to^\star_G a_i\cdots a_{i+s+1}.\]
  \item[($\supseteq$)]
    Нека $A \to^\star_G a_i\cdots a_{i+s+1}$ и да разгледаме първото правило, което сме приложили в този извод.
    Понеже $G$ е в нормална форма на Чомски, правилото е от вида $A \to_G BC$ и тогава съществува някое $k$, $i \leq k < i+s+1$, за което
    \begin{align*}
      & B \to^\star a_i\cdots a_{k}\\
      & C \to^\star a_{k+1}\cdots a_{i+s+1}.
    \end{align*}
    От {\bf И.П.} получаваме, че $B \in V[i,k]$ и $C \in V[k+1,i+s+1]$.
    Тогава от ред 10 на алгоритъма е ясно, че 
    \[A \in V[i,i+s+1].\]
  \end{description}
\end{proof}

\begin{example}
  Нека е дадена граматиката $G$ в нормална форма на Чомски с правила 
  \begin{align*}
    & S\rightarrow a\ |\ AB\ |\ AC\\
    & A\rightarrow a\\
    & B\rightarrow b\\
    & C\rightarrow SB\ |\ AS.
  \end{align*}
  Ще приложим CYK алгоритъма за да проверим дали думата $aaabb \in \L(G)$.
  \begin{itemize}
  \item 
    За $s = 0$ имаме, че:
    \begin{itemize}
    \item 
      $V[1,1] = V[2,2] = V[3,3] = \{S,A\}$;
    \item
      $V[4,4] = V[5,5] = \{B\}$.
    \end{itemize}
  \item
    За $s = 1$ имаме, че:
    \begin{itemize}
    \item
      $V[1,2] = V[2,3] = \{C\}$;
    \item
      $V[3,4] = \{S,C\}$;
    \item
      $V[4,5] = \emptyset$.
    \end{itemize}
  \item
    За $s = 2$ имаме, че:
    \begin{itemize}
    \item
      $V[1,3] = \{S\} \cup \emptyset$;
    \item
      $V[2,4] = \{S,C\} \cup \emptyset$;
    \item
      $V[3,5] = \emptyset \cup \{C\}$.
    \end{itemize}
  \item
    За $s = 3$ имаме, че:
    \begin{itemize}
    \item
      $V[1,4] = \{S,C\} \cup \emptyset \cup \emptyset = \{S,C\}$;
    \item
      $V[2,5] = \{S\} \cup \emptyset \cup \{C\} = \{S,C\}$.
    \end{itemize}
  \item
    За $s = 4$ имаме, че:
    \begin{itemize}
    \item 
      $V[1,5] = \{S,C\} \cup \emptyset \cup \emptyset \cup \{C\}= \{S,C\}$.
    \end{itemize}
  \end{itemize}
  Понеже $S \in V[1,5]$, то $aaabb \in \L(G)$.
\end{example}


%%% Local Variables:
%%% mode: latex
%%% TeX-master: "../eai"
%%% End:
