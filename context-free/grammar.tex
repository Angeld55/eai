\section{Извод в безконтекстна граматика}

\index{граматика!безконтекстна}
\mynote{В \cite{papadimitriou} дефиницията е различна. Там $\Sigma \subseteq V$. На англ. {\em context-free grammar}. Други срещани наименования на български са
  {\em контекстно-свободна}, {\em контекстно-независима}. Тук всички правила са от вида $A \to \alpha$, където $\alpha \in (V\cup\Sigma)^\star$.}

В Раздел \ref{sect:regular-grammar} въведохме понятието неограничена граматика. След това видяхме как можем да опишем регулярните езици
със специален вид граматики, които нарекохме регулярни граматики.
Сега ще разгледаме още един вид граматики, които описват по-широк клас от езици.

Една граматика $G = (V, \Sigma, R, S)$ се нарича {\bf безконтекстна}, ако 
имаме ограничението, че $R \subseteq V\times (V\cup\Sigma)^\star$.

Дефинираме релацията $\alpha \derive{\ell}_G \beta$ за произволни думи $\alpha,\beta \in (V\cup\Sigma)^\star$ по следния начин:

\begin{prooftree}
  \AxiomC{}
  \RightLabel{\scriptsize{правило (0)}}
  \UnaryInfC{$\alpha \derive{0}_G \alpha$}
\end{prooftree}

\begin{prooftree}
  \AxiomC{$A \to_G \gamma$}
  \AxiomC{$\gamma \derive{\ell}_G \beta$}
  \RightLabel{\scriptsize{правило (1)}}
  \BinaryInfC{$A \derive{\ell+1}_G \beta$}
\end{prooftree}

\begin{prooftree}
  \AxiomC{$\alpha_1 \derive{\ell_1}_G \beta_1$}
  \AxiomC{$\alpha_2 \derive{\ell_2}_G \beta_2$}
  \RightLabel{\scriptsize{правило (2)}}
  \BinaryInfC{$\alpha_1\cdot\alpha_2 \derive{\ell_1+\ell_2}_G \beta_1\cdot \beta_2$}
\end{prooftree}

\mynote{В частност имаме, че:
\begin{prooftree}
  \AxiomC{$A \to_G \beta$}
  \UnaryInfC{$A \derive{1}_G \beta$}
\end{prooftree}
}

Нека $\derive{\star}_G$ е рефлексивното и транзитивно затваряне на релацията $\derive{1}_G$. С други думи,
\[ \alpha \derive{\star}_G \beta\ \iff\ (\exists \ell\in\Nat)[\ \alpha \derive{\ell}_G \beta\ ].\]

\index{език!безконтекстен}
Един език $L$ се нарича {\bf безконтекстен}, ако съществува безконтекстна граматика $G$, за която 
$L = \L(G) = \{\omega \in \Sigma^\star \mid S \derive{\star} \omega\}$.

\begin{remark}
  Очевидно е, че всяка регулярна граматика е безконтекстна. Следователно, 
  {\em всеки регулярен език е безконтекстен.}
\end{remark}

\begin{remark}
  Като частен случай на \Proposition{unrestricted-grammar:context-general-step} получаваме свойството:
  \begin{prooftree}
    \AxiomC{$A \derive{\ell_1} \rho B \delta$}
    \AxiomC{$B \derive{\ell_2} \beta$}
    \RightLabel{\scriptsize{правило (3)}}
    \BinaryInfC{$A \derive{\ell_1+\ell_2} \rho\beta\delta$,}
  \end{prooftree}
  което ще означим като правило (3), защото ще го използваме често.
\end{remark}

\begin{extra}
\begin{example}
  Да разгледаме безконтекстната граматика $G$, която има следните правила:
  \begin{align*}
    & S \to AS\ |\ \varepsilon\\
    & A \to aAb\ |\ ab.
  \end{align*}
  Да видим защо думата $aabbab \in \L(G)$. Ако следваме формално правилата за извод, получаваме следното:
  \begin{prooftree}
    \AxiomC{$S \to_G AS$}
    \AxiomC{$A \derive{0}_G A$}
    \AxiomC{$S \to_G AS$}
    \AxiomC{$S \to_G \varepsilon$}
    \AxiomC{}
    \RightLabel{\scriptsize{(0)}}
    \UnaryInfC{$\varepsilon \derive{0}_G \varepsilon$}
    \RightLabel{\scriptsize{(2)}}
    \BinaryInfC{$S \derive{1}_G \varepsilon$}
    \RightLabel{\scriptsize{(3)}}
    \BinaryInfC{$S \derive{2}_G A$}
    \RightLabel{\scriptsize{(2)}}
    \BinaryInfC{$AS \derive{2}_G AA$}
    \RightLabel{\scriptsize{(1)}}
    \BinaryInfC{$S \derive{3}_G AA$}
  \end{prooftree}
  Освен това имаме и следния формален извод:
  \begin{prooftree}
    \AxiomC{$A \to_G aAb$}
    \AxiomC{}
    \LeftLabel{\scriptsize{(0)}}
    \UnaryInfC{$a \derive{0}_G a$}
    \AxiomC{$A \to_G ab$}
    \AxiomC{}
    \RightLabel{\scriptsize{(0)}}
    \UnaryInfC{$b \derive{0}_G b$}
    \RightLabel{\scriptsize{(2)}}
    \BinaryInfC{$Ab \derive{1}_G abb$}
    \RightLabel{\scriptsize{(2)}}
    \BinaryInfC{$aAb \derive{1}_G aabb$}
    \RightLabel{\scriptsize{(1)}}
    \BinaryInfC{$A \derive{2}_G aabb$}
  \end{prooftree}
  Аналогично,
  \begin{prooftree}
    \AxiomC{$A \to_G ab$}
    \AxiomC{}
    \RightLabel{\scriptsize{(0)}}
    \UnaryInfC{$ab \derive{0}_G ab$}
    \RightLabel{\scriptsize{(1)}}
    \BinaryInfC{$A \derive{1}_G ab$}
  \end{prooftree}
  Обединявайки всичко, получаваме:
  \begin{prooftree}
    \AxiomC{$S \derive{3}_G AA$}
    \AxiomC{$A \derive{2}_G aabb$}
    \RightLabel{\scriptsize{(3)}}
    \BinaryInfC{$S \derive{5}_G aabbA$}
    \AxiomC{$A \derive{1}_G ab$}
    \RightLabel{\scriptsize{(3)}}
    \BinaryInfC{$S \derive{6}_G aabbab$}
  \end{prooftree}
  Можем да приложим правилата за извод в различен ред и пак да получим същия краен резултат.
  Например:
  \begin{prooftree}
    \AxiomC{$S \to_G AS$}
    \AxiomC{$A \derive{2} aabb$}
    \RightLabel{\scriptsize{(3)}}
    \BinaryInfC{$S \derive{3}_G aabbS$}
    \RightLabel{\scriptsize{(3)}}
    \AxiomC{$S \derive{2}_G A$}
    \BinaryInfC{$S \derive{5}_G aabbA$}
    \AxiomC{$A \derive{1}_G ab$}
    \RightLabel{\scriptsize{(3)}}
    \BinaryInfC{$S \derive{6}_G aabbab$}
  \end{prooftree}
\end{example}
\end{extra}


Следващото твърдение ни дава едно свойство, което е вярно за безконтекстни граматики, но не и за неограничени граматики.
\begin{proposition}\label{pr:grammar:divide-2}
  \mynote{Тук $\gamma_1,\gamma_2,\beta \in (V\cup\Sigma)^\star$.}
  Нека $\gamma_1$ и $\gamma_2$ са непразни думи, за които $\gamma_1\gamma_2 \derive{\ell} \beta$. Тогава
  съществуват числа $\ell_1$, $\ell_2$ и думи $\beta_1$ и $\beta_2$, за които:
  $\gamma_1 \derive{\ell_1} \beta_1$ и $\gamma_2 \derive{\ell_2} \beta_2$ и $\beta = \beta_1\beta_2$ и $\ell = \ell_1 + \ell_2$.
\end{proposition}
\begin{proof}  
Индукция по дължината на извода $\ell$.
\begin{itemize}
\item
  Нека $\ell = 0$. Тогава $\beta_1 = \gamma_1$ и $\beta_2 = \gamma_2$.
\item
  Нека $\ell = 1$. Без ограничение на общността, можем да предположим, че имаме следния извод:
  \mynote{Разсъждаваме аналогично, ако $\gamma_2 = \delta A \rho$.}
  \begin{prooftree}
    \AxiomC{$\overbrace{\delta A \rho}^{\gamma_1} \derive{0} \delta A \rho$}
    \AxiomC{$A \to \alpha$}
    \LeftLabel{\scriptsize{(\Proposition{unrestricted-grammar:context-general-step})}}
    \BinaryInfC{$\gamma_1 \derive{1} \delta\alpha\rho$}
    \AxiomC{$\gamma_2 \derive{0} \gamma_2$}
    \LeftLabel{\scriptsize{правило (2)}}
    \BinaryInfC{$\gamma_1\gamma_2 \derive{1} \underbrace{\delta\alpha\rho}_{\beta_1}\underbrace{\gamma_2}_{\beta_2}$}
  \end{prooftree}
\item
  Нека $\ell > 1$. Тогава да разгледаме следната ситуация:
  \begin{prooftree}
    \AxiomC{$\vdots$}
    \LeftLabel{\scriptsize{$(\ell_1 > 0)$}}
    \UnaryInfC{$\gamma_1\gamma_2 \derive{\ell_1} \delta$}
    \AxiomC{$\vdots$}
    \RightLabel{\scriptsize{$(\ell_2 > 0)$}}
    \UnaryInfC{$\delta \derive{\ell_2} \beta$}
    \RightLabel{\scriptsize{(\Proposition{unrestricted-grammar:general-step})}}
    \BinaryInfC{$\gamma_1\gamma_2 \derive{\ell_1+\ell_2} \beta$}
  \end{prooftree}
  Понеже $\ell_1 < \ell$, то можем да приложим \IndHyp, според което можем да разбием $\delta$ като $\delta = \delta_1\delta_2$ и да получим извода:
  \begin{prooftree}
    \AxiomC{$\gamma_1 \derive{\ell'_1} \delta_1$}
    \AxiomC{$\gamma_2 \derive{\ell''_1} \delta_2$}
    \RightLabel{\scriptsize{правило (2)}}
    \BinaryInfC{$\gamma_1\gamma_2 \derive{\ell_1} \delta_1\delta_2$}
  \end{prooftree}
  Понеже $\ell_2 < \ell$, то отново можем да приложим \IndHyp, според което можем да разбием $\beta$ като $\beta = \beta_1\beta_2$ и да получим извода:
  \begin{prooftree}
    \AxiomC{$\delta_1 \derive{\ell'_2} \beta_1$}
    \AxiomC{$\delta_2 \derive{\ell''_2} \beta_2$}
    \RightLabel{\scriptsize{правило (2)}}
    \BinaryInfC{$\delta_1\delta_2 \derive{\ell_2} \beta_1\beta_2$}
  \end{prooftree}
  Сега можем да обобщим всичко със следния извод:
  \begin{prooftree}
    \AxiomC{$\gamma_1 \derive{\ell'_1} \delta_1$}
    \AxiomC{$\gamma_2 \derive{\ell''_1} \delta_2$}
    \LeftLabel{\scriptsize{(2)}}
    \BinaryInfC{$\gamma_1\gamma_2 \derive{\ell_1} \delta_1\delta_2$}
    \AxiomC{$\delta_1 \derive{\ell'_2} \beta_1$}
    \AxiomC{$\delta_2 \derive{\ell''_2} \beta_2$}
    \RightLabel{\scriptsize{(2)}}
    \BinaryInfC{$\delta_1\delta_2 \derive{\ell_2} \beta_1\beta_2$}
    \RightLabel{\scriptsize{(\Proposition{unrestricted-grammar:general-step})}}
    \BinaryInfC{$\gamma_1\gamma_2 \derive{\ell_1+\ell_2} \beta_1\beta_2$}
  \end{prooftree}
\end{itemize}
\end{proof}

\begin{proposition}\label{pr:grammar:divide}
  Нека $G$ е безконтекстна граматика и нека $X_1\cdots X_k \derive{\ell}_G \beta$, където $X_i \in V \cup \Sigma$ и $k \geq 2$.
  Тогава съществуват думи $\beta_1,\dots,\beta_k$, такива че за $i = 1,\dots, k$ е изпълнено, че
  $X_i \derive{\ell_i} \beta_i$, където $\beta = \beta_1\cdots \beta_k$ и $\ell = \sum^k_{i = 1}\ell_i$.
\end{proposition}
\begin{hint}
  \mynote{Тук е възможно $X_i = a \in \Sigma$. Тогава $a \derive{0} a$ и $\beta_i = a$.}
  Пълна индукция по $k$ като използвате \Proposition{grammar:divide-2}.
\end{hint}
  
\begin{important}
  \begin{theorem}
    Всеки регулярен език е безконтекстен.
  \end{theorem}
\end{important}
\begin{hint}
  Ще направим индукция по построението на регулярните езици.
  \begin{itemize}
  \item
    Всеки от езиците $\emptyset$, $\{\varepsilon\}$ и $\{a\}$, за всяка буква $a \in \Sigma$ е безконтекстен.
  \item
    Нека $L_1$ и $L_2$ са безконтекстни езици. Тогава:
    \begin{itemize}
    \item
      $L_1 \cup L_2$ е безконтекстен език.
    \item
      $L_1 \cdot L_2$ е безконтекстен език.
    \item
      $L^\star_1$ е безконтекстен език.
    \end{itemize}
  \end{itemize}
\end{hint}



%%% Local Variables:
%%% mode: latex
%%% TeX-master: "../eai"
%%% End:
