\section{Безконтекстни граматики}
\index{граматика!безконтекстна}
\marginpar{В \cite{papadimitriou} дефиницията е различна. Там $\Sigma \subseteq V$}
\marginpar{На англ. {\em context-free grammar}}
\marginpar{Други срещани наименования на български са {\em контекстно-свободна}, {\em контекстно-независима}}
В Раздел \ref{sect:regular-grammar} въведохме понятието граматика. След това видяхме как можем да опишем регулярните езици
със специален вид граматики, които нарекохме регулярни граматики.
Сега ще разгледаме още един вид граматики, които описват по-широк клас от езици.

\begin{itemize}
\item 
  Една граматика $G = (V, \Sigma, R, S)$ се нарича {\bf безконтекстна}, ако 
  имаме ограничението, че $R \subseteq V\times (V\cup\Sigma)^\star$.
\item
  \index{език!безконтекстен}
  $L$ се нарича {\bf безконтекстен език}, ако съществува безконтекстна граматика $G$, за която 
  $L = \L(G) = \{\omega \in \Sigma^\star \mid S \to^\star_G \omega\}$.
\end{itemize}

\begin{remark}
  Очевидно е, че всяка регулярна граматика е безконтекстна. Следователно, 
  {\em всеки регулярен език е безконтекстен.}
\end{remark}

Като първи пример нека да видим, че това включване е {\em строго}, т.е. съществува безконтекстен език, който не е регулярен.
Да напомним, че вече видяхме, че езикът $L = \{a^nb^n \mid n\in\Nat\}$ не е регулярен.

\begin{example}
  Да разгледаме безконтекстната граматика $G$ зададена със следните правила:
  \begin{align*}
    & S \to aSb \mid \varepsilon.
  \end{align*}
  Лесно се съобразява, че $\L(G) = \{a^nb^n \mid n\in\Nat\}$.
\end{example}

\begin{example}
  Да разгледаме безконтекстната граматика $G$ зададена със следните правила:
  \begin{align*}
    & S \to aSc\ |\  B\\
    & B \to bBc\ |\ \varepsilon.
  \end{align*}
  Лесно се съобразява, че $\L(G) = \{a^nb^kc^{n+k} \mid n,k\in\Nat\}$.
\end{example}

\begin{example}
  Да разгледаме граматика с правила
  \begin{align*}
    & S \to S + S\ |\ S * S\ |\ (S)\ |\ V\\
    & V \to x\ |\ y\ |\ z
  \end{align*}

  Думата $x * y + z$ има две различни дървета на извод.

  Да разгледаме граматика с правила
  \begin{align*}
    & S \to E + S\ |\ E\\
    & E \to V * E\ |\ V\ |\ (S) * E\ |\ (S)\\
    & V \to x\ |\ y\ |\ z
  \end{align*}
  Сега думата $x * y + z$ има само едно дърво на извод.
\end{example}

\begin{example}
  \begin{align*}
    & S \to \textbf{if } S \textbf{ then } S \textbf{ else }S | \textbf{ if }S \textbf{ then }S | V\\
    & V \to x\ |\ y\ |\ z
  \end{align*}

  Ние искаме следната граматика:
  \begin{align*}
    & S \to M | U\\
    & M \to \textbf{if } S \textbf{ then } M \textbf{ else }M | X\\
    & U \to \textbf{if } S \textbf{ then } S | \textbf{if } S \textbf{ then } M \textbf{ else }U
  \end{align*}
\end{example}



\begin{example}
  Да разгледаме граматика с правила
  \begin{align*}
    & S \to E\\
    & E \to E + P\ |\ P\\
    & P \to P * N\ |\ N\\
    & N \to (E)\ |\ a.
  \end{align*}
\end{example}

\begin{problem}
  Докажете, че езикът $L = \{a^mb^nc^k\mid m+n \geq k\}$ е безконтекстен.
\end{problem}
\begin{proof}
  Да разгледаме граматиката $G$ с правила
  \begin{align*}
    S& \rightarrow aSc\ |\ aS\ |\ B\\
    B& \rightarrow bBc\ |\  bB\ |\ \varepsilon.
  \end{align*}
  
  Лесно се вижда с индукция по $n$, че за всяко $n$ имаме свойствата:
  \marginpar{\ding{45} Докажете!}
  \begin{itemize}
  \item 
    $S \rightarrow^\star a^nSc^n$,
  \item
    $S \rightarrow^\star a^nS$,
  \item
    $B \rightarrow^\star a^nBc^n$,
  \item
    $B \rightarrow^\star b^nB$.
  \end{itemize}
  Комбинирайки горните свойства, можем да видим, че за всяко $n \geq k$,
  \begin{itemize}
  \item 
    $S \rightarrow^\star a^nSc^k$,
  \item
    $B \rightarrow^\star b^nBc^k$.
  \end{itemize}
  За да докажем, че $L \subseteq L(G)$, 
  да разгледаме една дума $\omega \in L$, т.е. $\omega = a^mb^nc^k$, където $m+n \geq k$.
  Имаме два случая:
  \begin{itemize}
  \item 
    $k \leq m$, т.е. $m = k+l$ и $m+n = k+l+n$.
    Тогава имаме изводите:
    \[S \rightarrow^\star a^kSc^k,\ S \rightarrow^\star a^lS,\ S \rightarrow B,\ B \rightarrow^\star b^nB,\ B \rightarrow \varepsilon.\]
    Обединявайки всичко това, получаваме:
    \[S \rightarrow^\star a^mb^nc^k.\]
  \item
    $k > m$, т.е. $k = m+l$, за някое $l > 0$, и $m+n = k+r = m+l+r$, за някое $r$.
    Тогава имаме изводите:
    \[S \rightarrow^\star a^mSc^m,\ S\rightarrow B,\ B\rightarrow^\star b^lBc^l,\ B\rightarrow b^rB,\ B\rightarrow\varepsilon,\]
    и отново получаваме $S \rightarrow^\star a^mb^nc^k$.
  \end{itemize}
  Така доказахме, че $\omega \in \L(G)$.
  
  Сега ще докажем, че $\L(G) \subseteq L$.
  С индукция по дължината на извода $l$,
  ще докажем, че ако $S \stackrel{l}{\rightarrow}\omega$, то $\omega \in M$, където
  \[M = \{a^nSc^k\mid n\geq k\}\cup\{a^nb^mBc^k\mid n+m\geq k\}\cup\{a^nb^mc^k\mid n+m\geq k\}.\]
  
  Ако $l = 0$, то е ясно, че $S \stackrel{0}{\rightarrow} S$ и $S \in M$.

  Нека $l > 0$ и $S \stackrel{l-1}{\rightarrow} \alpha \rightarrow \omega$.
  От {\bf И.П.} имаме, че $\alpha \in M$. Нека $\omega$ се получава от $\alpha$ с прилагане на правилото $C \rightarrow \gamma$.
  Разглеждаме всички варианти за думата $\alpha \in M$ и за правилото $C\rightarrow \gamma$ в граматиката $G$
  за да докажем, че  $\omega \in M$.
  Удобно е да представим всички случаи в таблица.
  \begin{center}
    \begin{tabular}{| c | c | c |}
      \hline
      $\alpha\in M$ & $C \rightarrow \gamma$ & $\omega \in M?$ \\ \hline
      $a^nSc^k$ & $S \rightarrow aSc$ & $a^{n+1}Sc^{k+1}$ \\ \hline
      $a^nSc^k$ & $S \rightarrow aS$ & $a^{n+1}Sc^{k}$ \\ \hline
      $a^nSc^k$ & $S \rightarrow B$ & $a^{n}Bc^{k}$ \\ \hline
      $a^nb^mBc^k$ & $B \rightarrow bBc$ & $a^nb^{m+1}Bc^{k+1}$\\ \hline
      $a^nb^mBc^k$ & $B \rightarrow bB$ & $a^nb^{m+1}Bc^{k}$\\ \hline
      $a^nb^mBc^k$ & $B \rightarrow \varepsilon$ & $a^nb^{m}c^{k}$\\ \hline
    \end{tabular}
  \end{center}
  Във всички случаи се установява, че $\omega \in M$.
  Сега, за всяка дума $\omega \in L(G)$ следва, че
  \[\omega \in \Sigma^\star \cap M = \{a^mb^nc^k\mid m+n \geq k\}.\]
\end{proof}


\begin{problem}
  \marginpar{
    $S \to aS \mid aSc \mid aB \mid bB$\\
    $B \to bB \mid bBc \mid \varepsilon$
}
  Докажете, че езикът $L = \{a^mb^nc^k\mid m+n \geq k + 1\}$ е безконтекстен.  
\end{problem}

\begin{problem}
  \label{pr:nanb}
  Нека $\omega$ е произволна дума над азбуката $\{a,b\}$. 
  Тогава:
  \begin{enumerate}[a)]
  \item 
    ако $N_a(\omega) = N_b(\omega) + 1$, то съществуват думи $\omega_1$, $\omega_2$, за които
    $\omega = \omega_1 a \omega_2$, $N_a(\omega_1) = N_b(\omega_1)$ и $N_a(\omega_2) = N_b(\omega_2)$.
  \item
    ако $N_b(\omega) = N_a(\omega) + 1$, то съществуват думи $\omega_1$, $\omega_2$, за които
    $\omega = \omega_1 b \omega_2$, $N_a(\omega_1) = N_b(\omega_1)$ и $N_a(\omega_2) = N_b(\omega_2)$.
  \end{enumerate}
\end{problem}
\begin{proof}
  Пълна индукция по дължината на думата $\omega$, за които $N_a(\omega) = N_b(\omega)+1$.
  \begin{itemize}
  \item 
    $\abs{\omega} = 1$. Тогава $\omega_1 = \omega_2 = \varepsilon$ и $\omega = a$.
  \item
    $\abs{\omega} = n+1$. Ще разгледаме два случая, в зависимост от първия символ на $\omega$.
    \begin{itemize}
    \item 
      Случаят $\omega = a\omega'$ е лесен. (Защо?)
    \item
      Интересният случай е $\omega = b\omega'$.    
      Тогава $\omega = b^{i+1}a\omega'$. Да разгледаме думата $\omega''$, която се получава от $\omega$
      като премахнем първото срещане на думата $ba$, т.е. 
      $\omega'' = b^i\omega'$ и $\abs{\omega''} = n-1$.
      Понеже от $\omega$ сме премахнали равен брой $a$-та и $b$-та, $N_a(\omega'') = N_b(\omega'')+1$.
      Според {\bf И.П.} за $\omega''$, можем да запишем думата като $\omega'' = \omega''_1a\omega''_2$
      и $N_a(\omega''_1) = N_b(\omega''_1)$, $N_a(\omega''_2) = N_b(\omega''_2)$.
      Понеже $b^i$ е префикс на $\omega''_1$, за да получим обратно $\omega$, трябва 
      да прибавим премахнатата част $ba$ веднага след $b^i$ в $\omega''_1$.
    \end{itemize}
  \end{itemize}
\end{proof}

\begin{problem}
  За произволна дума $\omega \in \{a,b\}^\star$, 
  докажете, че ако $N_a(\omega) > N_b(\omega)$, то съществуват думи $\omega_1$ и $\omega_2$,
  за които $\omega = \omega_1 a \omega_2$ и $N_a(\omega_1) \geq N_b(\omega_1)$, $N_a(\omega_2) \geq N_b(\omega_2)$.
\end{problem}

\begin{problem}
  Да се докаже, че езикът $L = \{\alpha \in \{a,b\}^\star\mid N_a(\alpha) = N_b(\alpha)\}$ 
  е безконтекстен.
\end{problem}
\begin{proof}
  \marginpar{  Алтернативна граматика за езика $L$ е
    \[S \rightarrow \varepsilon\ |\ aSb\ |\ bSa\ |\ SS.\]}
  Една възможна граматика $G$ е следната: 
  \[S \rightarrow aSbS\vert bSaS \vert\varepsilon.\]
  Например, да разгледаме извода на думата $aabbba$ в тази граматика:
  \begin{align*}
    S & \to aSbS \to aaSbSbS \to aa\varepsilon bSbS \to aab\varepsilon bS \to aabbbSaS\\
    & \to aabbb\varepsilon a S \to aabbba.
  \end{align*}
  
  Като следствие от \Prob{nanb} може лесно да се изведе, че за думи $\omega$, за които $N_a(\omega) = N_b(\omega)$,
  е изпълнено следното:
  \begin{enumerate}[a)]
  \item 
    ако $\omega = a\omega'$, то
    $\omega = a\omega_1b\omega_2$ и $N_a(\omega_1) = N_b(\omega_1)$, $N_a(\omega_2) = N_b(\omega_2)$;
  \item
    ако $\omega = b\omega'$, то
    $\omega = b\omega_1a\omega_2$ и $N_a(\omega_1) = N_b(\omega_1)$, $N_a(\omega_2) = N_b(\omega_2)$.
  \end{enumerate}

  Сега първо ще проверим, че $L \subseteq L(G)$.
  За целта ще докажем с {\em пълна индукция} по дължината на думата $\omega$, че за всяка дума $\omega$ със свойството $N_a(\omega) = N_b(\omega)$ е изпълнено
  $S \rightarrow^\star \omega$.
  \begin{itemize}
  \item 
    Нека $\abs{\omega} = 0$. Тогава $S \rightarrow \varepsilon$.
  \item
    Нека $\abs{\omega} = k+1$. Имаме два случая.
    \begin{itemize}
    \item 
      $\omega = a\omega^\prime$, т.е. от свойство а), $\omega = a\omega_1b\omega_2$ и $N_a(\omega_1) = N_b(\omega_1)$, $N_a(\omega_2) = N_b(\omega_2)$.
      Тогава $\abs{\omega_1} \leq k$ и по И.П. $S \rightarrow^\star \omega_1$.
      Аналогично, $S \rightarrow^\star \omega_2$.
      Понеже имаме правило $S \rightarrow aSbS$, заключаваме че $S \rightarrow^\star a\omega_1b\omega_2$.
    \item
      $\omega = b\omega^\prime$, т.е. свойство б), $\omega = b\omega_1a\omega_2$ и $N_a(\omega_1) = N_b(\omega_1)$, $N_a(\omega_2) = N_b(\omega_2)$.
      Този случай се разглежда аналогично.
    \end{itemize}
  \end{itemize}
  
  Преминаваме към доказателството на другата посока, т.е. $L(G) \subseteq L$.
  Тук с индукция по дължината на извода $l$ ще докажем, че
  $S \stackrel{l}{\rightarrow} \omega$, то $\omega \in M$,
  където
  \[M = \{\omega \in \{a,b,S\}^\star \mid N_a(\omega) = N_b(\omega)\}.\]
  За $l = 0$  е ясно, че $S \stackrel{0}{\rightarrow^\star} S$.
  За $l = k+1$, то $S \stackrel{k}{\rightarrow^\star} \alpha \rightarrow \omega$.
  От {\bf И.П.} имаме, че $\alpha \in M$.
  Нека $\omega$ се получава от $\alpha$ с прилагане на правилото $C \rightarrow \gamma$.
  Разглеждаме всички варианти за думата $\alpha \in M$ и за правилото $C\rightarrow \gamma$ в граматиката $G$
  за да докажем, че  $\omega \in M$.
  Удобно е да представим всички случаи в таблица.
  \begin{center}
    \begin{tabular}{| c | c | c |}
      \hline
      $\alpha$ & $C \rightarrow \gamma$ & $\omega$ \\ \hline
      $\in M$ & $S \rightarrow aSbS$ & $\in M$ \\ \hline
      $\in M$ & $S \rightarrow bSaS$ & $\in M$ \\ \hline
      $\in M$ & $S \rightarrow \varepsilon$ & $\in M$ \\ \hline
    \end{tabular}
  \end{center}
  Във всички случаи лесно се установява, че $\omega \in M$.
  Така за всяка дума $\omega \in L(G)$ следва, че
  \[\omega \in \Sigma^\star \cap M = L.\]
\end{proof}

\begin{problem}
  Нека $L_1$ и $L_2$ са произволни безконтекстни езици.
  Докажете, че:
  \begin{itemize}
  \item 
    $L_1 \cup L_2$ е безконтекстен език;
  \item
    $L^\star_1$ е безконтекствен език;
  \end{itemize}
\end{problem}

\begin{problem}
  Да разгледаме граматиката $G$ с правила
  \[S \to AA\ |\ B,\ A \to B\ |\ bb,\ B \to aa\ |\ aB.\]
  Да се намери езика на тази граматика и да се докаже, че граматиката разпознава точно този език.
\end{problem}



%%% Local Variables: 
%%% mode: latex
%%% TeX-master: "../eai"
%%% End: 
