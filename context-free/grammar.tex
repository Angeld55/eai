\section{Безконтекстни граматики}\label{sect:context-free:derive}
\index{граматика!безконтекстна}
\mynote{В \cite{papadimitriou} дефиницията е различна. Там $\Sigma \subseteq V$. На англ. {\em context-free grammar}. Други срещани наименования на български са
  {\em контекстно-свободна}, {\em контекстно-независима}. Тук всички правила са от вида $A \to \alpha$, където $\alpha \in (V\cup\Sigma)^\star$. В частност имаме, че:
  \begin{prooftree}
    \AxiomC{$A \to_G \beta$}
    \UnaryInfC{$A \derive{1}_G \beta$}
  \end{prooftree}}

В Раздел \ref{sect:unrestricted-grammar} въведохме понятието неограничена граматика. След това видяхме как можем да опишем регулярните езици
със специален вид граматики, които нарекохме регулярни граматики.
Сега ще разгледаме още един вид граматики, които описват по-широк клас от езици.

Една граматика $G = (V, \Sigma, R, S)$ се нарича {\bf безконтекстна}, ако 
имаме ограничението, че $R \subseteq V\times (V\cup\Sigma)^\star$.
Да повторим дефиницията на релацията $\alpha \derive{\ell}_G \beta$ от Раздел~\ref{sect:unrestricted-grammar} в частния случай, когато граматиката е безконтекстна.

\begin{important}
  \begin{figure}[H]
    \begin{subfigure}[b]{0.5\textwidth}
      \begin{prooftree}
        \AxiomC{}
        \RightLabel{\scriptsize{правило (0)}}
        \UnaryInfC{$\alpha \derive{0}_G \alpha$}
      \end{prooftree}
    \end{subfigure}
    ~
    \begin{subfigure}[b]{0.5\textwidth}
      \begin{prooftree}
        \AxiomC{$(A,\alpha) \in R$}
        \AxiomC{$\lambda\alpha\rho \derive{\ell}_G \beta$}
        \RightLabel{\scriptsize{правило (1)}}
        \BinaryInfC{$\lambda A \rho \derive{\ell+1}_G \beta$}
      \end{prooftree}
    \end{subfigure}
    \caption{Правила за извод в безконтекстна граматика}
  \end{figure}
\end{important}



% \mynote{На практика във всички учебници дефинират $\alpha \derive{1}_G \beta$ точно тогава когато $\alpha$ може да се представи като $\alpha = \rho A \delta$, имаме правило в граматиката $A \to_G \gamma$ и $\beta = \rho \gamma \delta$. }

% Нека дефинираме релацията $\derive{\star}_G$ е рефлексивното и транзитивно затваряне на релацията $\derive{1}_G$. С други думи,
% \[ \alpha \derive{\star}_G \beta\ \dff\ (\exists \ell\in\Nat)[\ \alpha \derive{\ell}_G \beta\ ].\]

\index{език!безконтекстен}
Нека официално да обявим, че един език $L$ се нарича {\bf безконтекстен}, ако съществува безконтекстна граматика $G$, за която 
$L = \L(G) = \{\omega \in \Sigma^\star \mid S \derive{\star} \omega\}$.

% \begin{important}
% \begin{remark}
%   Очевидно е, че всяка регулярна граматика е безконтекстна. Следователно, 
%   {\em всеки регулярен език е безконтекстен.}
% \end{remark}
% \end{important}

\begin{remark}
  Като частен случай на \Proposition{unrestricted-grammar:context-general-step} получаваме свойството:
  \begin{prooftree}
    \AxiomC{$\alpha \derive{\ell_1} \rho B \delta$}
    \AxiomC{$B \derive{\ell_2} \beta$}
    \RightLabel{\scriptsize{правило (2)}}
    \BinaryInfC{$\alpha \derive{\ell_1+\ell_2} \rho\beta\delta$,}
  \end{prooftree}
  което ще означим като правило (2), защото ще го използваме често.
\end{remark}

\begin{extra}
\begin{example}
  Да разгледаме безконтекстната граматика $G$, която има следните правила:
  \begin{align*}
    & S \to AS\ |\ \varepsilon\\
    & A \to aAb\ |\ ab.
  \end{align*}
  \mynote{За момента не е ясно как можем да проверим, че примерно думата $abbba \not\in \L(G)$. Този въпрос ще разгледаме по-нататък.}
  Да видим защо думата $aabbab \in \L(G)$. Ако следваме формално правилата за извод, получаваме следното:
  \begin{prooftree}
    \AxiomC{$S \to_G AS$}
    \AxiomC{$A \derive{0}_G A$}
    \AxiomC{$S \to_G AS$}
    \AxiomC{$S \to_G \varepsilon$}
    \AxiomC{}
    \RightLabel{\scriptsize{(0)}}
    \UnaryInfC{$\varepsilon \derive{0}_G \varepsilon$}
    \RightLabel{\scriptsize{(1)}}
    \BinaryInfC{$S \derive{1}_G \varepsilon$}
    \RightLabel{\scriptsize{(2)}}
    \BinaryInfC{$S \derive{2}_G A$}
    \RightLabel{\scriptsize{(\Proposition{unrestricted-grammar:concat2})}}
    \BinaryInfC{$AS \derive{2}_G AA$}
    \RightLabel{\scriptsize{(1)}}
    \BinaryInfC{$S \derive{3}_G AA$}
  \end{prooftree}
  Освен това имаме и следния формален извод:
  \begin{prooftree}
    \AxiomC{$A \to_G aAb$}
    \AxiomC{}
    \LeftLabel{\scriptsize{(0)}}
    \UnaryInfC{$a \derive{0}_G a$}
    \AxiomC{$A \to_G ab$}
    \AxiomC{}
    \RightLabel{\scriptsize{(0)}}
    \UnaryInfC{$abb \derive{0}_G abb$}
    \RightLabel{\scriptsize{(1)}}
    \BinaryInfC{$Ab \derive{1}_G abb$}
    \RightLabel{\scriptsize{(\Proposition{unrestricted-grammar:concat2}})}
    \BinaryInfC{$aAb \derive{1}_G aabb$}
    \RightLabel{\scriptsize{(1)}}
    \BinaryInfC{$A \derive{2}_G aabb$}
  \end{prooftree}
  Аналогично,
  \begin{prooftree}
    \AxiomC{$A \to_G ab$}
    \AxiomC{}
    \RightLabel{\scriptsize{(0)}}
    \UnaryInfC{$ab \derive{0}_G ab$}
    \RightLabel{\scriptsize{(1)}}
    \BinaryInfC{$A \derive{1}_G ab$}
  \end{prooftree}
  Обединявайки всичко, получаваме:
  \begin{prooftree}
    \AxiomC{$S \derive{3}_G AA$}
    \AxiomC{$A \derive{2}_G aabb$}
    \RightLabel{\scriptsize{(2)}}
    \BinaryInfC{$S \derive{5}_G aabbA$}
    \AxiomC{$A \derive{1}_G ab$}
    \RightLabel{\scriptsize{(2)}}
    \BinaryInfC{$S \derive{6}_G aabbab$}
  \end{prooftree}
  Можем да приложим правилата за извод в различен ред и пак да получим същия краен резултат.
  Например:
  \begin{prooftree}
    \AxiomC{$S \to_G AS$}
    \AxiomC{$A \derive{2} aabb$}
    \RightLabel{\scriptsize{(2)}}
    \BinaryInfC{$S \derive{3}_G aabbS$}
    \RightLabel{\scriptsize{(2)}}
    \AxiomC{$S \derive{2}_G A$}
    \BinaryInfC{$S \derive{5}_G aabbA$}
    \AxiomC{$A \derive{1}_G ab$}
    \RightLabel{\scriptsize{(2)}}
    \BinaryInfC{$S \derive{6}_G aabbab$}
  \end{prooftree}
\end{example}
\end{extra}


Следващото твърдение ни дава едно свойство, което е вярно за безконтекстни граматики, но не и за неограничени граматики.
\begin{proposition}\label{pr:grammar:divide-2}
  \mynote{Тук $\gamma_1,\gamma_2,\beta \in (V\cup\Sigma)^\star$.}
  Нека $\gamma_1$ и $\gamma_2$ са непразни думи, за които $\gamma_1\gamma_2 \derive{\ell} \beta$. Тогава
  съществуват числа $\ell_1$, $\ell_2$ и думи $\beta_1$ и $\beta_2$, за които:
  $\gamma_1 \derive{\ell_1} \beta_1$ и $\gamma_2 \derive{\ell_2} \beta_2$ и $\beta = \beta_1\beta_2$ и $\ell = \ell_1 + \ell_2$.
\end{proposition}
\begin{proof}  
Индукция по дължината на извода $\ell$.
\begin{itemize}
\item
  Нека $\ell = 0$. Тогава $\beta_1 = \gamma_1$ и $\beta_2 = \gamma_2$.
% \item
%   Нека $\ell = 1$. Без ограничение на общността, според \Proposition{grammar:alternative-def}, да предположим, че сме приложили правило в $\gamma_1$.
%   Получаваме извода
%   \mynote{Разсъждаваме аналогично, ако $\gamma_2 = \delta A \rho$.}
%   \begin{prooftree}
%     \AxiomC{$\overbrace{\delta A \rho}^{\gamma_1} \derive{0} \delta A \rho$}
%     \AxiomC{$A \to_G \alpha$}
%     \LeftLabel{\scriptsize{(\Proposition{grammar:alternative-def})}}
%     \BinaryInfC{$\gamma_1 \derive{1} \delta\alpha\rho$}
%     \AxiomC{$\gamma_2 \derive{0} \gamma_2$}
%     \LeftLabel{\scriptsize{правило (2)}}
%     \BinaryInfC{$\gamma_1\gamma_2 \derive{1} \underbrace{\delta\alpha\rho}_{\beta_1}\underbrace{\gamma_2}_{\beta_2}$}
%   \end{prooftree}
\item
  Нека $\ell > 0$. Тогава да разгледаме следната ситуация:
  % \begin{prooftree}
  %   \AxiomC{$\vdots$}
  %   \LeftLabel{\scriptsize{$(\ell_1 > 0)$}}
  %   \UnaryInfC{$\gamma_1\gamma_2 \derive{\ell_1} \delta$}
  %   \AxiomC{$\vdots$}
  %   \RightLabel{\scriptsize{$(\ell_2 > 0)$}}
  %   \UnaryInfC{$\delta \derive{\ell_2} \beta$}
  %   \RightLabel{\scriptsize{(\Proposition{unrestricted-grammar:general-step})}}
  %   \BinaryInfC{$\gamma_1\gamma_2 \derive{\ell_1+\ell_2} \beta$}
  % \end{prooftree}
  % Понеже $\ell_1 < \ell$, то можем да приложим \IndHyp, според което можем да разбием $\delta$ като $\delta = \delta_1\delta_2$ и да получим извода:
  \begin{prooftree}
    \AxiomC{$(A,\alpha) \in R$}
    \AxiomC{$\lambda\alpha\rho \derive{\ell-1}\beta$}
    \BinaryInfC{$\underbrace{\lambda A \rho}_{\gamma_1\gamma_2} \derive{\ell} \beta$}
  \end{prooftree}

  Без ограничение на общността, нека $A$ е част от $\gamma_1$. Това означава, че
  $\gamma_1 = \lambda A \rho_1$ и $\rho = \rho_1\gamma_2$. Получаваме следния извод:
  \begin{prooftree}
    \AxiomC{$(A,\alpha) \in R$}
    \RightLabel{\scriptsize{\IndHyp}}
    \AxiomC{$\overbrace{\lambda\alpha\rho_1}^{\gamma_3}\gamma_2 \derive{\ell-1}\beta$}
    \UnaryInfC{$\gamma_3 \derive{\ell_1'} \beta_1\ \&\ \gamma_2 \derive{\ell_2} \beta_2$}
    \LeftLabel{\scriptsize{правило (1)}}
    \BinaryInfC{$\underbrace{\lambda A \rho_1}_{\gamma_1} \derive{\ell'_1+1} \beta_1\ \&\ \gamma_2 \derive{\ell_2} \beta_2$}
  \end{prooftree}
  
  
  % \begin{prooftree}
  %   \AxiomC{$\gamma_1 \derive{\ell'_1} \delta_1$}
  %   \AxiomC{$\gamma_2 \derive{\ell''_1} \delta_2$}
  %   \LeftLabel{\scriptsize{($\ell'_1+\ell''_1 = \ell_1$)}}
  %   \RightLabel{\scriptsize{правило (2)}}
  %   \BinaryInfC{$\gamma_1\gamma_2 \derive{\ell_1} \underbrace{\delta_1\delta_2}_{\delta}$}
  % \end{prooftree}
  % Понеже $\ell_2 < \ell$, то отново можем да приложим \IndHyp и да получим извода:
  % \begin{prooftree}
  %   \AxiomC{$\delta_1 \derive{\ell'_2} \beta_1$}
  %   \AxiomC{$\delta_2 \derive{\ell''_2} \beta_2$}
  %   \LeftLabel{\scriptsize{($\ell'_2+\ell''_2 = \ell_2$)}}
  %   \RightLabel{\scriptsize{правило (2)}}
  %   \BinaryInfC{$\underbrace{\delta_1\delta_2}_{\delta} \derive{\ell_2} \underbrace{\beta_1\beta_2}_{\beta}$}
  % \end{prooftree}
  % Сега можем да обобщим всичко със следния извод:
  % \begin{prooftree}
  %   \AxiomC{$\gamma_1 \derive{\ell'_1} \delta_1$}
  %   \AxiomC{$\gamma_2 \derive{\ell''_1} \delta_2$}
  %   \LeftLabel{\scriptsize{(2)}}
  %   \BinaryInfC{$\gamma_1\gamma_2 \derive{\ell_1} \delta_1\delta_2$}
  %   \AxiomC{$\delta_1 \derive{\ell'_2} \beta_1$}
  %   \AxiomC{$\delta_2 \derive{\ell''_2} \beta_2$}
  %   \RightLabel{\scriptsize{(2)}}
  %   \BinaryInfC{$\delta_1\delta_2 \derive{\ell_2} \beta_1\beta_2$}
  %   \RightLabel{\scriptsize{(\Proposition{unrestricted-grammar:general-step})}}
  %   \BinaryInfC{$\gamma_1\gamma_2 \derive{\ell_1+\ell_2} \beta_1\beta_2$}
  % \end{prooftree}
\end{itemize}
\end{proof}

\begin{proposition}\label{pr:grammar:divide}
  Нека $G$ е безконтекстна граматика и нека $X_1\cdots X_k \derive{\ell}_G \beta$, където $X_i \in V \cup \Sigma$ и $k \geq 2$.
  Тогава съществуват думи $\beta_1,\dots,\beta_k$, такива че за $i = 1,\dots, k$ е изпълнено, че
  $X_i \derive{\ell_i} \beta_i$, където $\beta = \beta_1\cdots \beta_k$ и $\ell = \sum^k_{i = 1}\ell_i$.
\end{proposition}
\begin{hint}
  \mynote{Тук е възможно $X_i = a \in \Sigma$. Тогава $a \derive{0} a$ и $\beta_i = a$.}
  Пълна индукция по $k$ като използвате \Proposition{grammar:divide-2}.
\end{hint}

Следващото твърдение е важно, но е трудно да не пропуснем доказателството му.
\begin{important}
  \begin{proposition}
    Безконтекстните езици са затворени относно операциите обединение, конкатенация и звезда на Клини.
  \end{proposition}
\end{important}

Оттук можем да извлечем второ доказателство на твърдението, че всеки регулярен език е безконтекстен.

\begin{proposition}
  Всеки регулярен език е безконтекстен.
\end{proposition}
\begin{hint}
  Можем да подходим поне по два начина.
  \begin{itemize}
  \item
    Да проведем индукция по построението на регулярните езици и да докажем така, че всеки регулярен език е безконтекстен.
  \item
    Просто да съобразим, че всяка регулярна граматика е всъщност и безконтекстна.
  \end{itemize} 
\end{hint}



%%% Local Variables:
%%% mode: latex
%%% TeX-master: "../eai"
%%% End:
