\section{Извод в безконтекстна граматика}

\index{граматика!безконтекстна}
\mynote{В \cite{papadimitriou} дефиницията е различна. Там $\Sigma \subseteq V$. На англ. {\em context-free grammar}. Други срещани наименования на български са
  {\em контекстно-свободна}, {\em контекстно-независима}. Тук всички правила са от вида $A \to \alpha$, където $\alpha \in (V\cup\Sigma)^\star$.}

В Раздел \ref{sect:regular-grammar} въведохме понятието неограничена граматика. След това видяхме как можем да опишем регулярните езици
със специален вид граматики, които нарекохме регулярни граматики.
Сега ще разгледаме още един вид граматики, които описват по-широк клас от езици.

Една граматика $G = (V, \Sigma, R, S)$ се нарича {\bf безконтекстна}, ако 
имаме ограничението, че $R \subseteq V\times (V\cup\Sigma)^\star$.

Дефинираме релацията $\alpha \derive{\ell}_G \beta$ за произволни думи $\alpha,\beta \in (V\cup\Sigma)^\star$ по следния начин:

\begin{prooftree}
  \AxiomC{}
  \RightLabel{\scriptsize{правило (0)}}
  \UnaryInfC{$\alpha \derive{0}_G \alpha$}
\end{prooftree}

\begin{prooftree}
  \AxiomC{$A \to_G \gamma$}
  \AxiomC{$\gamma \derive{\ell}_G \beta$}
  \RightLabel{\scriptsize{правило (1)}}
  \BinaryInfC{$A \derive{\ell+1}_G \beta$}
\end{prooftree}

\begin{prooftree}
  \AxiomC{$\alpha_1 \derive{\ell_1}_G \beta_1$}
  \AxiomC{$\alpha_2 \derive{\ell_2}_G \beta_2$}
  \RightLabel{\scriptsize{правило (2)}}
  \BinaryInfC{$\alpha_1\cdot\alpha_2 \derive{\ell_1+\ell_2}_G \beta_1\cdot \beta_2$}
\end{prooftree}

\mynote{В частност имаме, че:
\begin{prooftree}
  \AxiomC{$A \to_G \beta$}
  \UnaryInfC{$A \derive{1}_G \beta$}
\end{prooftree}
}

Нека $\derive{\star}_G$ е рефлексивното и транзитивно затваряне на релацията $\derive{1}_G$. С други думи,
\[ \alpha \derive{\star}_G \beta\ \iff\ (\exists \ell\in\Nat)[\ \alpha \derive{\ell}_G \beta\ ].\]

\index{език!безконтекстен}
Един език $L$ се нарича {\bf безконтекстен}, ако съществува безконтекстна граматика $G$, за която 
$L = \L(G) = \{\omega \in \Sigma^\star \mid S \derive{\star} \omega\}$.

\begin{remark}
  Очевидно е, че всяка регулярна граматика е безконтекстна. Следователно, 
  {\em всеки регулярен език е безконтекстен.}
\end{remark}

\begin{remark}
  Като частен случай на \Proposition{unrestricted-grammar:context-general-step} получаваме свойството:
  \begin{prooftree}
    \AxiomC{$A \derive{\ell_1} \rho B \delta$}
    \AxiomC{$B \derive{\ell_2} \beta$}
    \RightLabel{\scriptsize{правило (3)}}
    \BinaryInfC{$A \derive{\ell_1+\ell_2} \rho\beta\delta$,}
  \end{prooftree}
  което ще означим като правило (3), защото ще го използваме често.
\end{remark}

\begin{extra}
\begin{example}
  Да разгледаме безконтекстната граматика $G$, която има следните правила:
  \begin{align*}
    & S \to AS\ |\ \varepsilon\\
    & A \to aAb\ |\ ab.
  \end{align*}
  Да видим защо думата $aabbab \in \L(G)$. Ако следваме формално правилата за извод, получаваме следното:
  \begin{prooftree}
    \AxiomC{$S \to_G AS$}
    \AxiomC{$A \derive{0}_G A$}
    \AxiomC{$S \to_G AS$}
    \AxiomC{$S \to_G \varepsilon$}
    \AxiomC{}
    \RightLabel{\scriptsize{(0)}}
    \UnaryInfC{$\varepsilon \derive{0}_G \varepsilon$}
    \RightLabel{\scriptsize{(2)}}
    \BinaryInfC{$S \derive{1}_G \varepsilon$}
    \RightLabel{\scriptsize{(3)}}
    \BinaryInfC{$S \derive{2}_G A$}
    \RightLabel{\scriptsize{(2)}}
    \BinaryInfC{$AS \derive{2}_G AA$}
    \RightLabel{\scriptsize{(1)}}
    \BinaryInfC{$S \derive{3}_G AA$}
  \end{prooftree}
  Освен това имаме и следния формален извод:
  \begin{prooftree}
    \AxiomC{$A \to_G aAb$}
    \AxiomC{}
    \LeftLabel{\scriptsize{(0)}}
    \UnaryInfC{$a \derive{0}_G a$}
    \AxiomC{$A \to_G ab$}
    \AxiomC{}
    \RightLabel{\scriptsize{(0)}}
    \UnaryInfC{$b \derive{0}_G b$}
    \RightLabel{\scriptsize{(2)}}
    \BinaryInfC{$Ab \derive{1}_G abb$}
    \RightLabel{\scriptsize{(2)}}
    \BinaryInfC{$aAb \derive{1}_G aabb$}
    \RightLabel{\scriptsize{(1)}}
    \BinaryInfC{$A \derive{2}_G aabb$}
  \end{prooftree}
  Аналогично,
  \begin{prooftree}
    \AxiomC{$A \to_G ab$}
    \AxiomC{}
    \RightLabel{\scriptsize{(0)}}
    \UnaryInfC{$ab \derive{0}_G ab$}
    \RightLabel{\scriptsize{(1)}}
    \BinaryInfC{$A \derive{1}_G ab$}
  \end{prooftree}
  Обединявайки всичко, получаваме:
  \begin{prooftree}
    \AxiomC{$S \derive{3}_G AA$}
    \AxiomC{$A \derive{2}_G aabb$}
    \RightLabel{\scriptsize{(3)}}
    \BinaryInfC{$S \derive{5}_G aabbA$}
    \AxiomC{$A \derive{1}_G ab$}
    \RightLabel{\scriptsize{(3)}}
    \BinaryInfC{$S \derive{6}_G aabbab$}
  \end{prooftree}
  Можем да приложим правилата за извод в различен ред и пак да получим същия краен резултат.
  Например:
  \begin{prooftree}
    \AxiomC{$S \to_G AS$}
    \AxiomC{$A \derive{2} aabb$}
    \RightLabel{\scriptsize{(3)}}
    \BinaryInfC{$S \derive{3}_G aabbS$}
    \RightLabel{\scriptsize{(3)}}
    \AxiomC{$S \derive{2}_G A$}
    \BinaryInfC{$S \derive{5}_G aabbA$}
    \AxiomC{$A \derive{1}_G ab$}
    \RightLabel{\scriptsize{(3)}}
    \BinaryInfC{$S \derive{6}_G aabbab$}
  \end{prooftree}
\end{example}
\end{extra}

Следващото твърдение ни дава едно свойство, което е вярно за безконтекстни граматики, но не и за неограничени граматики.
\begin{proposition}\label{pr:grammar:divide}
  Нека $G$ е безконтекстна граматика и нека $X_1\cdots X_k \derive{\ell}_G \beta$, където $X_i \in V \cup \Sigma$.
  Тогава съществуват думи $\beta_1,\dots,\beta_k$, такива че за $i = 1,\dots, k$ е изпълнено, че
  $X_i \derive{\ell_i} \beta_i$, където $\beta = \beta_1\cdots \beta_k$ и $\ell = \sum^k_{i = 1}\ell_i$.
\end{proposition}
\begin{hint}
  \mynote{Тук е възможно $X_i = a \in \Sigma$. Тогава $a \derive{0} a$ и $\beta_i = a$.}
  Ще докажем твърдението с индукция по $k \geq 2$.
  \begin{itemize}
  \item
    За $k = 2$, то твърдението е точно правило $(2)$.
  \item
    Нека $k > 2$.
    Тогава имаме следното:
    \begin{prooftree}
      \AxiomC{$X_1 \derive{\ell_1} \beta_1$}
      \AxiomC{$X_2 \derive{\ell_2} \beta_2$}
      \AxiomC{$\dots$}
      \AxiomC{$X_k \derive{\ell_k} \beta_k$}
      \RightLabel{\scriptsize{($\ell'' = \sum^k_{i=2}\ell_i$)}}
      \LeftLabel{\scriptsize{\IndHyp}}
      \TrinaryInfC{$X_2\cdots X_k \derive{\ell''} \beta_2\cdots\beta_k$}
      \LeftLabel{\scriptsize{правило (2)}}
      \RightLabel{\scriptsize{$(\ell = \ell_1+\ell'')$}}
      \BinaryInfC{$X_1X_2\cdots X_k \derive{\ell}_G \beta_1\beta_2\cdots\beta_k$}
    \end{prooftree}
  \end{itemize}
\end{hint}

\begin{important}
  \begin{theorem}
    Всеки регулярен език е безконтекстен.
  \end{theorem}
\end{important}
\begin{hint}
  Ще направим индукция по построението на регулярните езици.
  \begin{itemize}
  \item
    Всеки от езиците $\emptyset$, $\{\varepsilon\}$ и $\{a\}$, за всяка буква $a \in \Sigma$ е безконтекстен.
  \item
    Нека $L_1$ и $L_2$ са безконтекстни езици. Тогава:
    \begin{itemize}
    \item
      $L_1 \cup L_2$ е безконтекстен език.
    \item
      $L_1 \cdot L_2$ е безконтекстен език.
    \item
      $L^\star_1$ е безконтекстен език.
    \end{itemize}
  \end{itemize}
\end{hint}


За произволна безконтекстна граматика $G$, дефинираме релацията $X \yield{\ell} \alpha$, където $X \in V \cup \Sigma$ и $\alpha \in (V\cup\Sigma)^\star$, по следния начин:
\mynote{Интуитивно, $\yield{\star}$ е аналог на BFS, докато $\derive{\star}$ е аналог на DFS.}
\begin{prooftree}
  \AxiomC{}
  \RightLabel{\scriptsize{правило (0)}}
  \UnaryInfC{$X \yield{0} X$}
\end{prooftree}

\begin{prooftree}
  \AxiomC{$X \to_G X_1\cdots X_n$}
  \AxiomC{$X_1 \yield{\ell_1} \gamma_1$}
  \AxiomC{$\cdots$}
  \AxiomC{$X_n \yield{\ell_n} \gamma_n$}
  \LeftLabel{\scriptsize{($\ell = \max\{\ell_1,\dots,\ell_n\})$}}
  \RightLabel{\scriptsize{правило (1)}}
  \QuaternaryInfC{$X \yield{\ell+1} \gamma_1\cdots\gamma_n$}
\end{prooftree}

\mynote{ Съобразете, че имаме:
\begin{prooftree}
  \AxiomC{$X \to_G \gamma$}
  \UnaryInfC{$X \yield{1} \gamma$.}
\end{prooftree}}

Нека $\yield{\star}$ е рефлексивното и транзитивно затваряне на релацията $\yield{\ell}$, т.е.
\[X \yield{\star} \gamma \dff (\exists \ell\in\Nat)[X \yield{\ell} \gamma].\]


\begin{proposition}
  За произволни променливи $A$, $B$ и думи $\alpha_1,\alpha_2, \alpha_3$ е изпълнено, че:
  \begin{prooftree}
    \AxiomC{$A \yield{\ell_1}\alpha_1B\alpha_3$}
    \AxiomC{$B \yield{\ell_2} \alpha_2$}
    \BinaryInfC{$A \yield{\ell_1+\ell_2} \alpha_1\alpha_2\alpha_3$}
  \end{prooftree}
\end{proposition}
\begin{hint}
  Индукция по $\ell_1$.
\end{hint}

\begin{proposition}
  За произволна променлива $A$ и произволни думи $\beta$ и $\gamma$ е изпълнено, че:
  \begin{prooftree}
    \AxiomC{$A \yield{\ell} \beta A \gamma$}
    \AxiomC{$i \in \Nat$}
    \BinaryInfC{$A \yield{\ell \cdot i} \beta^i A \gamma^i$}
  \end{prooftree}
  
\end{proposition}
\begin{hint}
  Индукция по $i$.
\end{hint}

\begin{lemma}
  Нека $G$ е безконтекстна граматика, $X \in V \cup \Sigma$ и $\beta \in (V \cup \Sigma)^\star$.
  Тогава ако $X \derive{\star}_G \beta$, то $X \yield{\star}_G \beta$.
\end{lemma}  
\begin{proof}
  С пълна индукция по $\ell$ ще докажем, че ако $X \derive{\ell} \beta$, то $X \yield{\star} \beta$.
  \begin{itemize}
  \item
    $\ell = 0$, т.е. $X \derive{0} X$.
    Тогава е ясно, че $X \yield{\star} X$.
  \item
    Нека $\ell > 0$ и $X \derive{\ell} \beta$.
    Според правилата на извод в граматика имаме извода

    \begin{prooftree}
      \AxiomC{$X \to_G X_0X_1\cdots X_k$}
      \AxiomC{$X_0X_1\cdots X_k \derive{\ell-1} \beta$}
      \RightLabel{\scriptsize{правило (1)}}
      \BinaryInfC{$X \derive{\ell} \beta$}
    \end{prooftree}

    От \Proposition{grammar:divide} знаем, че съществува разбиване на $\beta$ на $k+1$ части, така че:
    \begin{itemize}
    \item
      $\beta = \beta_0 \cdots \beta_{k}$;
    \item
      $X_i \derive{\ell_i} \beta_i$, за всяко $i = 0,\dots,k$;
    \item
      $\ell-1 = \sum^k_{i=1} \ell_i$.
    \end{itemize}
    % От \IndHyp имаме, че $X_i \yield{\star} \beta_i$ за всяко $i = 0,1, \dots, k$.
    Получаваме следното:
    \begin{prooftree}
      \AxiomC{$X \to_G X_0\cdots X_k$}
      \AxiomC{$X_0 \derive{\ell_0} \beta_0$}
      \RightLabel{\scriptsize{\IndHyp}}
      \UnaryInfC{$X_0 \yield{\star} \beta_0$}
      \AxiomC{$\cdots$}
      \AxiomC{$X_k \derive{\ell_k} \beta_k$}
      \RightLabel{\scriptsize{\IndHyp}}
      \UnaryInfC{$X_k \yield{\star} \beta_k$}
      \QuaternaryInfC{$X \yield{\star} \underbrace{\beta_0\cdots\beta_k}_{\beta}$}
    \end{prooftree}
  \end{itemize}
\end{proof}

\begin{lemma}
  Нека $G$ е безконтекстна граматика, $X \in V \cup \Sigma$ и $\gamma \in (V \cup \Sigma)^\star$.
  Тогава ако $X \yield{\star}_G \gamma$, то $X \derive{\star}_G \gamma$.
\end{lemma}
\begin{proof}
  С пълна индукция по $\ell$ ще докажем, че ако $X \yield{\ell} \gamma$, то $X \derive{\star} \gamma$.
  \begin{itemize}
  \item
    Нека $\ell = 0$. Това означава, че $X \yield{0} X$. Ясно е, че $X \derive{\star} X$.
  \item
    Нека $\ell > 0$. Тогава имаме следното:
    \begin{prooftree}
      \AxiomC{$X \to_G X_0\cdots X_n$}
      \AxiomC{$X_0 \yield{\ell_0} \gamma_0$}
      \AxiomC{$\cdots$}
      \AxiomC{$X_n \yield{\ell_n} \gamma_n$}
      \RightLabel{\scriptsize{($\ell = 1 + \max\{\ell_0,\dots,\ell_n\})$}}
      \QuaternaryInfC{$X \yield{\ell} \gamma_0\cdots\gamma_n$}
    \end{prooftree}
    Получаваме, че:
    \begin{prooftree}
      \AxiomC{$X \to_G X_0 \cdots X_{n}$}
      \AxiomC{$X_0 \yield{\ell_0} \gamma_0$}
      \RightLabel{\scriptsize{\IndHyp}}
      \UnaryInfC{$X_0 \derive{\star} \gamma_0$}
      \AxiomC{$\cdots$}
      \AxiomC{$X_n \yield{\ell_n} \gamma_n$}
      \RightLabel{\scriptsize{\IndHyp}}
      \UnaryInfC{$X_n \derive{\star} \gamma_n$}
      \TrinaryInfC{$X_0 \cdots X_{n} \derive{\star} \gamma_0\cdots\gamma_{n}$}
      \BinaryInfC{$X \derive{\star} \underbrace{\gamma_0\cdots\gamma_{n}}_{\gamma}$}
    \end{prooftree}
  \end{itemize}
\end{proof}

Комбинирайки предишните две леми получаваме следната теорема.
\begin{framed}
  \begin{theorem}\label{th:grammar:yield-derive-equivalent}
    Нека $G$ е безконтекстна граматика, $X \in V \cup \Sigma$ и $\beta \in (V \cup \Sigma)^\star$.
    Тогава $X \yield{\star}_G \gamma$ точно тогава, когато $X \derive{\star}_G \gamma$.
  \end{theorem}  
\end{framed}

\begin{corollary}
  $\L(G) = \{\alpha \in \Sigma^\star \mid S \yield{\star}\alpha\}$.
\end{corollary}

Да напомним, че в Раздел~\ref{sect:regular:minimisation} дефинирахме език $\L_\A(q)$.
Сега ще дефинираме език $\L_G(A)$, за произволна променлива $A$.
\[\L_G(A) \df \{\alpha \in \Sigma^\star \mid A \yield{\star}\alpha\}.\]
Ясно е, че $\L(G) = \L_G(S)$.
Също така, да дефинираме апроксимациите $\L^\ell_G(A)$ на $\L_G(A)$ по следния начин:
\[\L^\ell_G(A) \df \{\alpha \in \Sigma^\star \mid A \yield{\leq \ell} \alpha\}.\]
Следните свойства са ясни:
\begin{itemize}
\item
  $\L^0_G(A) = \emptyset$;
\item
  $\L^n_G(A)$ е краен език за всяко $n$;
\item
  $\L^n_G(A) \subseteq \L^{n+1}_G(A)$ за всяко $n$;
\item
  $\L_G(A) = \bigcup_{n\geq 0}\L^n_G(A)$.  
\end{itemize}

Следната характеризация на $\L^\ell_G(A)$ ще е важна за нас, когато искаме да докажем,
че една безконтекстна граматика разпознава даден език.

\begin{proposition}\label{pr:grammar:yield-approximation}
  Нека $G$ е произволна безконтекстна граматика и $A$ е променлива в $G$.
  Тогава имаме следното:
  \begin{align*}
    \L^0_G(A) & = \emptyset\\
    \L^{\ell+1}_G(A) & = \bigcup\{\{\alpha_1\}\cdot \L^\ell_G(A_1) \cdots \{\alpha_n\} \cdot \L^\ell_G(A_n) \cdot \{\alpha_{n+1}\} \mid A \to_G \alpha_1A_1\cdots\alpha_nA_n\alpha_{n+1}\}.
  \end{align*}
\end{proposition}
\begin{proof}
  Нека $\alpha \in \L^{\ell+1}_G(A)$, т.е. $A \yield{\leq \ell+1} \alpha$. Имаме следния извод:
  \begin{prooftree}
    \AxiomC{$A \to_G \alpha_1B_1\cdots\alpha_n B_n\alpha_{n+1}$}
    \AxiomC{$B_1 \yield{\leq\ell} \beta_1$}
    \AxiomC{$\cdots$}
    \AxiomC{$B_n \yield{\leq \ell} \beta_n$}
    \QuaternaryInfC{$A \yield{\leq\ell+1} \underbrace{\alpha_1\beta_1\cdots\alpha_n\beta_n\alpha_{n+1}}_{\alpha}$}
  \end{prooftree}
  Понеже $B_i \yield{\leq \ell} \beta_i$, то от \IndHyp имаме, че $\beta_i \in \L^\ell_G(B_i)$.
  Тогава 
  \[\alpha \in \bigcup\{\{\alpha_1\}\cdot \L^\ell_G(A_1) \cdots \{\alpha_n\} \cdot \L^\ell_G(A_n) \cdot \{\alpha_{n+1}\} \mid A \to_G \alpha_1A_1\cdots\alpha_nA_n\alpha_{n+1}\}\]

  Обратно, нека сега $\alpha = \alpha_1\beta_1\cdots\alpha_n\beta_n\alpha_{n+1}$, където $A \to_G \alpha_1B_1\cdots\alpha_nB_n\alpha_{n+1}$,
  $\beta_i \in \L^\ell_G(B_i)$ и $\alpha = \alpha_1\beta_1\cdots\alpha_n\beta_n\alpha_{n+1}$.
  Получаваме следния извод:
  \begin{prooftree}
    \AxiomC{$A \to_G \alpha_1B_1\cdots\alpha_nB_n\alpha_{n+1}$}
    \AxiomC{$\beta_1 \in \L^\ell_G(B_1)$}
    \RightLabel{\scriptsize{\IndHyp}}
    \UnaryInfC{$B_1 \yield{\leq \ell }\beta_1 $}
    \AxiomC{$\cdots$}
    \AxiomC{$\beta_n \in \L^\ell_G(B_n)$}
    \RightLabel{\scriptsize{\IndHyp}}
    \UnaryInfC{$B_n \yield{\leq \ell }\beta_n$}
    \QuaternaryInfC{$A \yield{\leq \ell+1} \underbrace{\alpha_1\beta_1\cdots\alpha_n\beta_n\alpha_{n+1}}_{\alpha}$}
  \end{prooftree}
\end{proof}



\begin{example}
  \begin{align*}
    & S \to \texttt{if } S \texttt{ then } S \texttt{ else }S\ |\ \texttt{ if }S \texttt{ then }S\ |\ V\\
    & V \to x\ |\ y\ |\ z
  \end{align*}

  Ние искаме следната граматика:
  \begin{align*}
    & S \to M\ |\ U\\
    & M \to \texttt{if } S \texttt{ then } M \texttt{ else }M\ |\ X\\
    & U \to \texttt{if } S \texttt{ then } S\ |\ \texttt{if } S \texttt{ then } M \texttt{ else }U
  \end{align*}
\end{example}

\begin{example}
  Да разгледаме граматика с правила
  \begin{align*}
    & S \to E\\
    & E \to E + P\ |\ P\\
    & P \to P * N\ |\ N\\
    & N \to (E)\ |\ a.
  \end{align*}
\end{example}


%%% Local Variables:
%%% mode: latex
%%% TeX-master: "../eai"
%%% End:
