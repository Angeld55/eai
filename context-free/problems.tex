\section{Допълнителни задачи}

\subsection{Равен брой леви и десни скоби}

Тук ще разглеждаме азбука $\Sigma$, която включва буквите $\texttt{[}$ и $\texttt{]}$.
Нека за по-голяма яснота да положим
\begin{align*}
  & \texttt{left}(\alpha) \df \card{\alpha}{\texttt{[}} & \comment{\text{брой срещания на $\texttt{[}$ в $\alpha$}}\\
  & \texttt{right}(\alpha) \df \card{\alpha}{\texttt{]}}. & \comment{\text{брой срещания на $\texttt{]}$ в $\alpha$}}
\end{align*}

\begin{problem}
  \label{prob:nanb}
  Нека $\omega$ е произволна дума над азбуката $\{\texttt{[}, \texttt{]}\}$. 
  Тогава:
  \begin{enumerate}[a)]
  \item 
    ако $\texttt{left}(\omega) = \texttt{right}(\omega) + 1$, то съществуват думи $\omega_1$, $\omega_2$, за които е изпълнено:
    \begin{itemize}
    \item 
      $\omega = \omega_1 \texttt{[} \omega_2$;
    \item
      $\texttt{left}(\omega_1) = \texttt{right}(\omega_1)$;
    \item
      $\texttt{left}(\omega_2) = \texttt{right}(\omega_2)$.
    \end{itemize}
  \item
    ако $\texttt{right}(\omega) = \texttt{left}(\omega) + 1$, то съществуват думи $\omega_1$, $\omega_2$, за които е изпълнено:
    \begin{itemize}
    \item 
      $\omega = \omega_1 \texttt{]} \omega_2$;
    \item
      $\texttt{left}(\omega_1) = \texttt{right}(\omega_1)$;
    \item
      $\texttt{left}(\omega_2) = \texttt{right}(\omega_2)$.
    \end{itemize}
  \end{enumerate}
\end{problem}
\begin{hint}
  \marginpar{Другият случай е аналогичен}
  Ще се съсредоточим върху случая, когато $\omega$ е дума, за която $\texttt{left}(\omega) = \texttt{right}(\omega) + 1$.
  Ще докажем а) с индукция по дължината на думата.
  \begin{itemize}
  \item 
    $\abs{\omega} = 1$. Тогава $\omega_1 = \omega_2 = \varepsilon$ и $\omega = \texttt{[}$.
  \item
    Да приемем, че твърдението а) е вярно за думи с дължина $\leq n$.
  \item
    $\abs{\omega} = n+1$. Ще разгледаме два случая, в зависимост от първия символ на $\omega$.
    \begin{itemize}
    \item 
      Случаят $\omega = \texttt{[}\omega'$ е очевиден. (Защо?)
    \item
      Интересният случай е $\omega = \texttt{]}\omega'$.    
      Тогава $\omega = \texttt{]}^{i+1}\texttt{[}\omega'$, за някое $i \in \Nat$.
      Да разгледаме думата $\omega''$, която се получава от $\omega$
      като премахнем първото срещане на думата $\texttt{][}$, т.е. 
      $\omega'' = \texttt{]}^i\omega'$ и $\abs{\omega''} = n-1$.
      Понеже от $\omega$ сме премахнали равен брой леви и десни скоби, то
      $\texttt{left}(\omega'') = \texttt{right}(\omega'')+1$.
      Според {\bf И.П.} за $\omega''$ са изпълнени свойствата:
      \begin{itemize}
      \item 
        $\omega'' = \omega''_1\texttt{[}\omega''_2$;
      \item
        $\texttt{left}(\omega''_1) = \texttt{right}(\omega''_1)$;
      \item
        $\texttt{left}(\omega''_2) = \texttt{right}(\omega''_2)$.
      \end{itemize}
      Понеже $\texttt{]}^i$ е префикс на $\omega''_1$, за да получим обратно $\omega$, трябва 
      да прибавим премахнатата част $\texttt{][}$ веднага след $\texttt{]}^i$ в $\omega''_1$.
    \end{itemize}
  \end{itemize}
\end{hint}

\begin{problem}
  За произволна дума $\omega \in \{ \texttt{[}, \texttt{]} \}^\star$, 
  докажете, че ако $\texttt{left}(\omega) > \texttt{right}(\omega)$, то съществуват думи $\omega_1$ и $\omega_2$,
  за които са изпълнени свойствата:
  \begin{itemize}
  \item 
    $\omega = \omega_1 \texttt{[} \omega_2$;
  \item
    $\texttt{left}(\omega_1) \geq \texttt{right}(\omega_1)$;
  \item
    $\texttt{left}(\omega_2) \geq \texttt{right}(\omega_2)$.
  \end{itemize}
\end{problem}

\begin{framed}
  \begin{problem}
    Да се докаже, че езикът 
    \[L = \{\ \alpha \in \{\texttt{[}, \texttt{]}\}^\star\mid \texttt{left}(\alpha) = \texttt{right}(\alpha)\ \}\]
    е безконтекстен.
  \end{problem}  
\end{framed}
\begin{hint}
  \marginpar{  Алтернативна граматика за езика $L$ е
    \[S \to \varepsilon\ |\ \texttt{[}S\texttt{]}\ |\ \texttt{]}S\texttt{[}\ |\ SS.\]}
  Една възможна граматика $G$ е следната: 
  \[S \to \texttt{[}S\texttt{]}S\ |\ \texttt{]}S\texttt{[}S\ |\ \varepsilon.\]
  % Например, да разгледаме извода на думата $aabbba$ в тази граматика:
  % \begin{align*}
  %   S & \to aSbS \to aaSbSbS \to aa\varepsilon bSbS \to aab\varepsilon bS \to aabbbSaS\\
  %   & \to aabbb\varepsilon a S \to aabbba.
  % \end{align*}
  
  Като следствие от \Problem{nanb} може лесно да се изведе, че за думи $\omega$, за които $\texttt{left}(\omega) = \texttt{right}(\omega)$,
  е изпълнено следното:
  \begin{enumerate}[a)]
  \item 
    ако $\omega = \texttt{[}\omega'$, то са изпълнени свойствата:
    \begin{itemize}
    \item 
      $\omega = \texttt{[}\omega_1\texttt{]}\omega_2$;
    \item
      $\texttt{left}(\omega_1) = \texttt{right}(\omega_1)$;
    \item
      $\texttt{left}(\omega_2) = \texttt{right}(\omega_2)$.
    \end{itemize}
  \item
    ако $\omega = \texttt{]}\omega'$, то са изпълнени свойствата:
    \begin{itemize}
    \item 
      $\omega = \texttt{]}\omega_1\texttt{[}\omega_2$;
    \item
      $\texttt{left}(\omega_1) = \texttt{right}(\omega_1)$;
    \item
      $\texttt{left}(\omega_2) = \texttt{right}(\omega_2)$.
    \end{itemize}
  \end{enumerate}

  Сега първо ще проверим, че $L \subseteq \L(G)$.
  За целта ще докажем с {\em пълна индукция} по дължината на думата $\omega$, че за всяка дума $\omega$ със свойството $\texttt{left}(\omega) = \texttt{right}(\omega)$ е изпълнено
  $S \rightarrow^\star \omega$.
  \begin{itemize}
  \item 
    Нека $\abs{\omega} = 0$. Тогава $S \rightarrow \varepsilon$.
  \item
    Да приемем, че за всяка дума с дължина $\leq k$ твърдението е вярно.
  \item
    Нека $\abs{\omega} = k+1$. Имаме два случая.
    \begin{itemize}
    \item 
      $\omega = \texttt{[}\omega^\prime$, т.е. от а) на \Problem{nanb}, 
      $\omega = \texttt{[}\omega_1\texttt{]}\omega_2$ и $\texttt{left}(\omega_1) = \texttt{right}(\omega_1)$, $\texttt{left}(\omega_2) = \texttt{right}(\omega_2)$.
      Тогава $\abs{\omega_1} \leq k$ и по И.П. $S \rightarrow^\star \omega_1$.
      Аналогично, $S \rightarrow^\star \omega_2$.
      Понеже имаме правило $S \rightarrow \texttt{[}S\texttt{]}S$, заключаваме че 
      $S \to^\star \texttt{[}\omega_1\texttt{]}\omega_2$.
    \item
      $\omega = \texttt{]}\omega^\prime$, т.е. свойство б), $\omega = \texttt{]}\omega_1\texttt{[}\omega_2$ и 
      $\texttt{left}(\omega_1) = \texttt{right}(\omega_1)$, $\texttt{left}(\omega_2) = \texttt{right}(\omega_2)$.
      Този случай се разглежда аналогично.
    \end{itemize}
  \end{itemize}
  
  Преминаваме към доказателството на другата посока, т.е. $\L(G) \subseteq L$.
  Тук с индукция по дължината на извода $\ell$ ще докажем, че
  $S \stackrel{\ell}{\to} \omega$, то $\omega \in M$,
  където
  \[M = \{\omega \in \{a,b,S\}^\star \mid \texttt{left}(\omega) = \texttt{right}(\omega)\}.\]
  \begin{itemize}
  \item 
    Ясно, че $S \stackrel{0}{\rightarrow} S$ и $S \in M$.
  \item
    Да разгледаме дума $\omega$, за която $S \stackrel{k+1}{\to} \omega$.
    Това означава, че съществува дума $\alpha$, за която
    \[S \stackrel{k}{\to} \alpha \to \omega.\]
    От {\bf И.П.} имаме, че $\alpha \in M$.
    Нека $\omega$ се получава от $\alpha$ с прилагане на правило от вида $S \to \gamma$.
    Разглеждаме всички варианти за думата $\alpha \in M$ и за правилото $S \to \gamma$ в граматиката $G$
    за да докажем, че $\omega \in M$.
    Удобно е да представим всички случаи в таблица.
    \begin{center}
      \begin{tabular}{| c | c | c |}
        \hline
        От И.П. за $\alpha$ & правило на $G$ & $\omega$ \\ \hline
        $\in M$ & $S \to \texttt{[}S\texttt{]}S$ & $\in M$ \\ \hline
        $\in M$ & $S \to \texttt{]}S\texttt{[}S$ & $\in M$ \\ \hline
        $\in M$ & $S \to \varepsilon$ & $\in M$ \\ \hline
      \end{tabular}
    \end{center}    
    Във всички случаи лесно се установява, че $\omega \in M$.
  \end{itemize}
  Така за всяка дума $\omega \in \L(G)$ следва, че
  \[\omega \in \Sigma^\star \cap M = L.\]
\end{hint}

%%% Local Variables:
%%% mode: latex
%%% TeX-master: "../eai"
%%% End:

\subsection{Балансирани скоби}

Нека $\alpha$ е дума над азбука, която включва буквите $\texttt{[}$ и $\texttt{]}$. 
Ще казваме, че че $\alpha$ е {\bf балансирана}, ако са изпълнени свойствата:
\begin{itemize}
\item 
  $\texttt{left}(\alpha) = \texttt{right}(\alpha)$;
\item
  За всеки префикс $\gamma$ на $\alpha$,
  $\texttt{left}(\gamma) \geq \texttt{right}(\gamma)$.
\end{itemize}

\begin{framed}
  \begin{problem}
    Докажете, че езикът 
    \[L = \{\ \alpha \in \{\texttt{[},\texttt{]}\}^\star \mid \alpha\text{ е балансирана дума}\ \}\]
    е безконтекстен.
  \end{problem}  
\end{framed}
\begin{hint}
  \marginpar{\cite[стр. 135]{kozen}}
  \marginpar{\writedown Докажете, че езикът $L$ не е регулярен! }
  Да разгледаме граматиката $G$ с правила
  \[S \to \texttt{[}S\texttt{]}\ |\ SS\ |\ \varepsilon.\]
  Ще докажем, че $L = \L(G)$.
  
  Първо ще докажем включването $\L(G) \subseteq L$.
  Да разгледаме \[M \df \{\ \alpha \in \{\texttt{[},\texttt{]}, S\}^\star \mid \alpha\text{ е балансирана}\ \}.\]
  
  Нека $S \to^\star_G \alpha$. Ще докажем с индукция по дължината $\ell$ на извода на $\alpha$ от $S$,
  че $\alpha \in M$. Случаят, когато $\ell = 0$ е очевиден.
  Нека $S \to^{\ell}_G \beta \to^1_G \alpha$.
  От {\bf И.П.} имаме, че $\beta \in M$, т.е. $\beta$ е балансирана.

  Лесно се съобразява, че във всички случаи за думите $\beta$ и $\alpha$ имаме следното:
  \begin{center}
    \begin{tabular}{| c | c | c |}
      \hline
      $\text{от И.П. }$ & $\text{правило на }G$ & $\text{извод}$ \\ \hline
      $\beta \in M$ & $S \rightarrow \texttt{[}S\texttt{]}$ & $\alpha \in M$ \\ \hline
      $\beta \in M$ & $S \rightarrow SS$ & $\alpha \in M$ \\ \hline
      $\beta \in M$ & $S \rightarrow \varepsilon$ & $\alpha \in M$ \\ \hline
    \end{tabular}
  \end{center}

  За включването $L \subseteq \L(G)$, нека $\alpha \in L$.
  Ще докажем с индукция по дължината на думата, че $\alpha \in \L(G)$.
  Ясно е, че във всички нетривиални случаи можем да запишем думата $\alpha$ като $\alpha = \texttt{[}\beta\texttt{]}$.
  Проблемът е, че в общия случай не е ясно дали можем да приложим индукционното предположение за $\beta$,
  защото е възможно $\beta \not\in L$. Например, $\alpha = \texttt{[][]}$.
  Тогава $\beta = \texttt{][} \not \in L$.
  Поради тази причина, трябва да сме по-внимателни и да разгледаме два случая.
  \begin{itemize}
  \item 
    \marginpar{\comment т.е. $\beta \neq \varepsilon$ и $\beta \neq \alpha$}
    Нека $\alpha$ има {\em същински} префикс $\beta \in L$.
    Понеже $\alpha \in L$, лесно се съобразява, че $\alpha = \beta\gamma$ и $\gamma \in L$.
    Сега можем да приложим {\bf И.П.} за $\beta$ и $\gamma$ и да получим, че 
    $\beta \in \L(G)$ и $\gamma \in \L(G)$, т.е.
    $S \to^\star_G \beta$ и $S \to^\star_G \gamma$.
    Понеже имаме правило $S \to_G SS$, то е ясно, че $\alpha \in \L(G)$.
  \item
    Нека $\alpha$ да няма същински префикс $\gamma \in L$.
    Ясно е, че тогава $\alpha = \texttt{[}\beta\texttt{]}$, за някое $\beta$
    и $\texttt{left}(\beta) = \texttt{right}(\beta)$.
    Да видим защо $\beta \in L$.
    
    Ако $\beta \in L$, то ще можем да приложим {\bf И.П.} за $\beta$ и ще сме готови.
    За всеки префикс $\gamma$ на $\beta$ имаме, че $\texttt{[}\gamma$ е префикс на $\alpha$,
    и понеже $\alpha \in L$, то $\texttt{left}(\texttt{[}\gamma) \geq \texttt{right}(\texttt{[}\gamma)$.
    Възможно ли е $\texttt{left}(\gamma) < \texttt{right}(\gamma)$ ?
    Това може да се случи единствено ако $\texttt{left}(\texttt{[}\gamma) = \texttt{right}(\texttt{[}\gamma)$.
    Но тогава $\texttt{[}\gamma$ е същински префикс на $\alpha$, за който $\texttt{[}\gamma \in L$,
    което противоречи на случая, който разглеждаме.
    Това означава, че за произволен префикс $\gamma$ на $\beta$,
    $\texttt{left}(\gamma) \geq \texttt{right}(\gamma)$ и оттук $\beta \in L$ и можем да приложим {\bf И.П.}
    Тогава $S \to^\star_G \beta$ и чрез правилото $S \to_G \texttt{[}S\texttt{]}$
    получаваме, че $\alpha \in \L(G)$.    
  \end{itemize}
\end{hint}

%%% Local Variables:
%%% mode: latex
%%% TeX-master: "../eai"
%%% End:

\subsection{Лесни задачи}

\begin{extra}

\begin{problem}
  Постройте регулярен израз за езика на следната граматика:
  \begin{align*}
    & S \to S + S\ |\ S * S\ |\ A\\
    & A \to KL\ |\ LK\\
    & K \to 0K\ |\ \varepsilon\\
    & L \to 1K\ |\ \varepsilon.
  \end{align*}
\end{problem}

\begin{problem}
  Докажете, че следните езици са безконтекстни.
  \begin{enumerate}[a)]
  \item
    \mynote{$S \rightarrow aSa\ \vert\ bSb\ \vert\ a\vert\ b\ \vert\ \varepsilon$}
    $L = \{\omega \in \{a,b\}^\star \mid \omega = \omega^{\rev}\}$;
  \item
    $L = \{a^nb^{2m}c^{n} \mid m,n \in \Nat\}$;
  \item
    $L = \{a^nb^{m}c^{m}d^n \mid m,n \in \Nat\}$;
  \item
    \mynote{Обединение на два езика}
    $L = \{a^nb^{2k} \mid n,k \in \Nat\ \&\ n \neq k\}$;
  \item
    \mynote{$S \rightarrow aSb | aS | a$}
    $L = \{a^nb^k \mid n > k\}$;
  \item
    $L = \{a^nb^k \mid n \geq 2k\}$;
  \item
    \mynote{$S \rightarrow aSc | aS | aB | bB$,\\$B\rightarrow bBc | bB | \varepsilon$}
    $L = \{a^nb^kc^m \mid n + k \geq m+1\}$;
  \item
    $L = \{a^nb^kc^m \mid n + k \geq m+2\}$;
  \item
    \mynote{$S \rightarrow aSc | aS | B | Bc$,\\$B\rightarrow bBc | bB | \varepsilon$}
    $L = \{a^nb^kc^m \mid n + k + 1 \geq m\}$;
  \item
    $L = \{a^nb^mc^{2k} \mid n \neq 2m\ \&\ k \geq 1\}$;
  \item
    $L = \{a^nb^kc^m \mid n + k \leq m\}$;
  \item
    $L = \{a^nb^kc^m \mid n + k \leq m+1\}$;
  \item
    \mynote{Обединение на три езика}
    $L = \{a^nb^mc^k \mid n, m, k \text{ не са страни на триъгълник}\}$;
  \item
    $L = \{a,b\}^\star \setminus \{a^{2n}b^n \mid n\in\Nat\}$;
  \item
    \mynote{$S\to EaE$, $E \to aEbE\ |\ bEaE\ |\ \varepsilon$}
    $L = \{\alpha \in \{a,b\}^\star\mid \card{\alpha}{a} = \card{\alpha}{b} + 1\}$;
  \item
    $L = \{\alpha \in \{a,b\}^\star\mid \card{\alpha}{a} \geq \card{\alpha}{b}\}$;
  \item
    $L = \{\alpha \in \{a,b\}^\star\mid \card{\alpha}{a} > \card{\alpha}{b}\}$;
  \item
    $L = \{\alpha \in \{a,b\}^\star \mid \text{ във всеки префикс $\beta$ на $\alpha$, } \card{\beta}{b} \leq \card{\beta}{a}\}$;
  \item
    $L = \{\alpha \sharp \beta \mid \alpha,\beta \in \{a,b\}^\star\ \&\ \alpha^{\rev}\mbox{ е поддума на }\beta \}$.
  \item
    $L = \{\omega_1 \sharp \omega_2 \sharp \cdots \sharp \omega_n \mid n\geq 2\ \&\ \omega_1,\omega_2,\dots,\omega_n \in \{a,b\}^\star\ \&\ \abs{\omega_1} = \abs{\omega_2}\}$;
  \item
    $L = \{\omega_1 \sharp \omega_2 \sharp \cdots \sharp \omega_n \mid n\geq 2\ \&\ \omega_1,\dots,\omega_n \in \{a,b\}^\star\ \&\ (\exists i \neq j)[\ \abs{\omega_i} = \abs{\omega_j}\ ]\}$;
  \item
    $L = \{\omega_1 \sharp \omega_2 \sharp \cdots \sharp \omega_n \mid n\geq 2\ \&\ (\forall i\in[1,n])[\ \omega_i \in \{a,b\}^\star\ \&\ \abs{\omega_i} = \abs{\omega_{n+1-i}}\ ]\}$.
  \end{enumerate}
\end{problem}

\begin{problem}
  Проверете дали следните езици са безконтекстни:
  \begin{enumerate}[a)]
  \item
    $\{a^nb^{2n}c^{3n}\ \mid\ n\in\Nat\}$;
  % \item
  %   $\{a^nb^{2n}c^{n}\ \mid\ n\in\Nat\}$;
  \item
    $\{a^nb^kc^ka^n\mid\ k \leq n\}$;
  \item
    $\{a^nb^mc^k\mid n < m < k\}$;
  \item
    $\{a^nb^nc^k\mid n \leq k \leq 2n\}$;
  \item
    $\{a^nb^mc^k\mid k = \min\{n,m\}\}$;
  \item
    $\{a^nb^nc^m\mid m \leq n\}$;
  \item
    $\{a^nb^mc^k\mid k = n\cdot m\}$;
  \item
    $L^\star$, където
    $L = \{\alpha\alpha^{\rev} \mid \alpha \in \{a,b\}^\star\}$;
  \item
    $\{www\mid w\in \{a,b\}^\star\}$;
  \item
    $\{a^{n^2}b^n\ \mid n \in \Nat\}$;
  \item
    $\{a^p\ \mid\ p\mbox{ е просто }\}$;
  \item
    $\{\omega \in \{a,b\}^\star \mid \omega = \omega^R\}$;
  \item
    $\{\omega^n \mid \omega \in \{a,b\}^\star\ \&\ \card{\omega}{b} = 2\ \&\ n \in \Nat\}$;
  \item
    $\{\omega c^n \omega^R \mid \omega \in \{a,b\}^\star\ \&\ n = \abs{\omega}\}$;
  \item
    % Дефиниция на подниз
    $\{w c x\mid w,x\in \{a,b\}^\star\ \&\ w\mbox{ е подниз на }x\}$;
  \item
    $\{x_1 \sharp x_2 \sharp \dots \sharp x_k\mid k\geq 2\ \&\ x_i\in a^\star\ \&\ (\exists i,j)[i \neq j\ \&\ x_i = x_j]\}$;
  \item
    $\{x_1 \sharp x_2 \sharp \dots \sharp x_k\mid k\geq 2\ \&\ x_i\in a^\star\ \&\ (\forall i,j \leq k)[i \neq j \iff x_i \neq x_j]\}$;
  \item
    $\{a^ib^jc^k\mid i,j,k\geq 0\ \&\ (i = j \vee j = k)\}$;
  \item
    % \marginpar{Разгл. $L' = L \cap L(a^*b^*c^*)$.}
    $\{\alpha \in \{a,b,c\}^\star\mid \card{\alpha}{a} > \card{\alpha}{b} > \card{\alpha}{c}\}$;
  \item
    $\{a,b\}^\star \setminus \{a^nb^n\mid n\in \Nat\}$;
  \item
    $\{a^nb^mc^k \mid m^2 = 2nk\}$;

  \item
    $L = \{a^nb^mc^ma^n \mid m,n\in\Nat\ \&\ n = m+42\}$;
  \item
    $L = \{\sharp a \sharp aa \sharp aaa \sharp \cdots \sharp a^{n-1}\sharp a^n\sharp \mid n \geq 1\}$;
  \item
    $\{a^mb^nc^k\mid m = n \vee n = k \vee m = k\}$;
  \item
    $\{a^mb^nc^k\mid m \neq n \vee n \neq k \vee m \neq k\}$;
  \item
    $\{a^mb^nc^k\mid m = n \wedge n = k \wedge m = k\}$;
  \item
    $\{\omega \in \{a,b,c\}^\star\mid \card{\omega}{a} \neq \card{\omega}{b} \vee \card{\omega}{a} \neq \card{\omega}{c} \vee \card{\omega}{b} \neq \card{\omega}{c}\}$.
  \end{enumerate}
\end{problem}

\end{extra}

%%% Local Variables:
%%% mode: latex
%%% TeX-master: "../eai"
%%% End:


\begin{problem}
  Докажете, че езикът $L = \{a^nb^{kn} \mid k,n > 0\}$ не е безконтекстен.
\end{problem}
\ifhints
\begin{hint}
  Да разгледаме ситуацията $\alpha \in L$ и
  $\alpha = xyuvw$, където $y = a^i$ и $v = b^j$
  Интересният случай е когато $0 < i,j < p$.
  \begin{itemize}
  \item 
    Нека $i = j$.
    Разглеждаме думата $xy^{p^2+1}uv^{p^2+1}w$.
    Това означава дали $p+p^2i$ дели $p^2+p^2i$, т.е.
    дали $1 + pi$ дели $p+pi$.
    \begin{align*}
      & p + pi = k(1+pi), \text{ за някое }1 \leq k < p\\
      & p = k + pi(k-1), \text{ за някое }1 \leq k < p\\
    \end{align*}
    Достигаме до противоречие.
  \item
    Нека $i > j$, т.е. $i \geq j+1$.
    Отново разглеждаме думата $xy^{p^2+1}uv^{p^2+1}w$.
    Това означава дали $p+p^2i$ дели $p^2+p^2j$, т.е.
    дали $1 + pi$ дели $p+pj$, но
    $1+pi \geq 1 + p(j+1) > p + pj$.
    Противоречие.
  \item
    Нека $i < j$ и тогава нека $j = mi + r$, където $p > i > r \geq 0$
    Разглеждаме думата $xy^{mp^2+1}uv^{mp^2+1}w$.
    Това означава дали $p+mp^2i$ дели $p^2+mp^2j$, т.е.
    дали $1 + mpi$ дели $p+mpj$.
    \[p+pmj = k(1+pmi), \text{ за някое }1 \leq k.\]
    \begin{itemize}
    \item 
      Възможно ли е $k \geq p$? Тогава:
      \begin{align*}
        & p + pmj = k(1+pmi) \geq p(1+pmi) \geq p(1+pj)\\
        & p(1+mj) \geq p(1+pj)\\
        & m \geq p.
      \end{align*}
      Достигаме до противоречие.
      Следователно, $1 \leq k < p$.
    \item
      Възможно ли е $k \leq m$? Тогава:
      \begin{align*}
        & p + pmj = k(1+pmi) \leq m(1+pmi) \leq m(1+pj)\\
        & p(1+mj) \leq m(1+pj)\\
        & p + pmj \leq m + pmj\\
        & p \leq m.
      \end{align*}
      Достигаме до противоречие, защото $m < p$.
    \item
      Заключаваме, че $1 \leq m < k < p$. Тогава:
      \begin{align*}
        & p + pmj = k(1+pmi)\\
        & p = k + pm(ki-j).
      \end{align*}
      Понеже $m < k$, то $j < (m+1)i \leq ki$,
      т.е. $ki-j > 0$. Достигаме до противоречие.
    \end{itemize}
  \end{itemize}
\end{hint}
\fi

За всеки две думи с равна дължина, дефинираме функцията $\texttt{diff}$ по следния начин:
\begin{align*}
  & \texttt{diff}(\varepsilon,\varepsilon) = 0\\
  & \texttt{diff}(a \cdot \alpha, b\cdot \beta) =
    \begin{cases}
      \texttt{diff}(\alpha,\beta), & \text{ако }a = b\\
      1 + \texttt{diff}(\alpha,\beta), & \text{ако }a \neq b\\
    \end{cases}
\end{align*}

\begin{problem}
  За всеки от следните езици, отговорете дали са безконтекстни, като се обосновете:
  \begin{enumerate}[a)]
  \item 
    \ifhints
    \marginpar{Да}
    \fi
    $\{\alpha \sharp \beta \mid \alpha,\beta \in \{a,b\}^\star\ \&\ |\alpha| = |\beta|\ \&\ \texttt{diff}(\alpha,\beta^{\rev}) = 1\}$;
  \item
    \ifhints
    \marginpar{Да}
    \fi
    $\{\alpha \sharp \beta \mid \alpha, \beta \in \{a,b\}^\star\ \&\ |\alpha| = |\beta|\ \&\ \texttt{diff}(\alpha,\beta^{\rev}) \geq 1\}$;
  \item
    \ifhints 
    \marginpar{Не}
    \fi
    $\{\alpha \sharp \beta \mid \alpha, \beta \in \{a,b\}^\star\ \&\ |\alpha| = |\beta|\ \&\ \texttt{diff}(\alpha,\beta) \geq 1\}$;
  \item
    \ifhints 
    \marginpar{Да}
    \fi
    $\{\alpha\beta \mid \alpha, \beta \in \{a,b\}^\star\ \&\ |\alpha| = |\beta|\ \&\ \texttt{diff}(\alpha,\beta) \geq 1\}$;
  \item
    \ifhints
    \marginpar{Не}
    \fi
    $\{\alpha\beta \mid \alpha, \beta \in \{a,b\}^\star\ \&\ |\alpha| = |\beta|\ \&\ \texttt{diff}(\alpha,\beta) = 1\}$;
  \end{enumerate}
\end{problem}    

\begin{problem}
    Да разгледаме езика
    \[D_1 \df \{\alpha_1 \sharp \cdots \sharp \alpha_n \mid n \geq 2\ \&\ |\alpha_i| = |\alpha_{i+1}|\ \&\ \texttt{diff}(\alpha_i,\alpha^{\rev}_{i+1}) = 1\}.\]
    \begin{itemize}
    \item 
      Докажете, че $D_1$ не е безконтекстен.
    \item
      Докажете, че $D_1$ може да се представи като сечението на два безконтекстни езика.
    \end{itemize}    
\end{problem}
\begin{hint}
  Да разгледаме безконтекстния език 
  \[L_1 = \{\alpha_1 \sharp \alpha_2 \mid |\alpha_1| = |\alpha_{2}|\ \&\ \texttt{diff}(\alpha_1,\alpha^{\rev}_{2}) = 1\}.\]
  Тогава
  \begin{align*}
    D_1 =\ & L_1 \cdot (\sharp L_1)^\star \cdot (\{\varepsilon\} \cup \sharp\{a,b\}^\star)\ \cap \\
           & \{a,b\}^\star \cdot (\sharp L_1)^\star \cdot (\{\varepsilon\} \cup \sharp\{a,b\}^\star).
  \end{align*}
\end{hint}

\begin{problem}
  За произволен език $L$, дефинираме езика
  \[\texttt{Diff}_n(L) = \{\alpha \in L \mid (\exists \beta \in L)[|\alpha| = |\beta|\ \&\ \texttt{diff}(\alpha,\beta) = n]\}.\]
  Вярно ли е, че:
  \begin{itemize}
  \item 
    ако $L$ е регулярен, то $\texttt{Diff}_n(L)$ е регулярен?
  \item
    ако $L$ е безконтекстен, то $\texttt{Diff}_n(L)$ е безконтекстен?
  \end{itemize}
\end{problem}
\begin{hint}
  Мисля, че най-лесно става като се разгледа съответно автомата или стековия автомат.
\end{hint}

\begin{problem}
  \marginpar{\cite[стр. 158]{sipser3}}
  Нека $L_1$ и $L_2$ са езици. Дефинираме
  \[L_1 \triangle L_2 \df \{\alpha\beta \mid \alpha \in L_1\ \&\ \beta \in L_2\ \&\ |\alpha| = |\beta|\}.\]
  Докажете, че
  \begin{enumerate}[a)]
  \item 
    ако $L_1$ и $L_2$ са регулярни езици, то е възможно $L_1 \triangle L_2$ да не е регулярен;
  \item
    ако $L_1$ и $L_2$ са регулярни езици, то $L_1 \triangle L_2$ е безконтекстен;
  \item
    ако $L_1$ и $L_2$ са безконтекстни езици, то е възможно $L_1 \triangle L_2$ да не е безконтекстен.
  \end{enumerate}
\end{problem}

\begin{problem}
  Докажете, че ако $L$ е безконтекстен език, то 
  \[L^{\rev} = \{\omega^{\rev} \mid \omega \in L\}\]
  също е безконтекстен.
\end{problem}


\begin{problem}
  Нека $\Sigma = \{a,b,c,d,f,e\}$.
  Докажете, че езикът $L$ е безконтекстен, където за думите $\omega \in L$ са изпълнени свойствата:
  \begin{itemize}[-]
  \item 
    за всяко $n\in\Nat$, след всяко срещане на $n$ последнователни $a$-та
    следват $n$ последователни $b$-та, и $b$-та не се срещат по друг повод в $\omega$, и
  \item
    за всяко $m\in\Nat$, след всяко срещане на $m$ последнователни $c$-та
    следват $m$ последователни $d$-та, и $d$-та не се срещат по друг повод в $\omega$, и
  \item
    за всяко $k\in\Nat$, след всяко срещане на $k$ последнователни $f$-а
    следват $k$ последователни $e$-та, и $e$-та не се срещат по друг повод в $\omega$.
  \end{itemize}
\end{problem}

\begin{problem}
  Да разгледаме езиците:
  \begin{align*}
    & P = \{\alpha\in\{a,b,c\}^*\,|\, \alpha \text{ е палиндром с четна дължина}\} \\
    & L =  \{\beta b^n\,|\, n\in\mathbb{N}, \beta\in P^n\}.
  \end{align*}
  Да се докаже, че:
  \begin{enumerate}[a)]
  \item 
    $L$ не е регулярен;
  \item 
    $L$ е безконтекстен.
  \end{enumerate}
\end{problem}

\begin{problem}
  Нека $L_1$ е произволен регулярен език над азбуката $\Sigma$, 
  а $L_2$ е езика от всички думи палиндроми над $\Sigma$.
  Докажете, че $L$ е безконтекстен език, където:
  \[L = \{\alpha_1 \cdots\alpha_{3n}\beta_1\cdots\beta_m\gamma_1\cdots\gamma_n \mid \alpha_i,\gamma_j \in L_1, \beta_k\in L_2, m,n \in \Nat\}.\]
\end{problem}

\begin{problem}
  Нека $L = \{\omega\in\{a,b\}^\star \mid \card{\omega}{a} = 2\}$.
  Да се докаже, че езикът $L' = \{\alpha^n \mid \alpha\in L, n \geq 0\}$ не е безконтекстен.
\end{problem}


\begin{problem}
  Нека $\Sigma = \{a,b,c\}$ и $L \subseteq \Sigma^\star$ е безконтестен език. Ако имаме дума 
  $\alpha \in \Sigma^\star$, тогава \emph{L-вариант} на $\alpha$ ще наричаме думата, която се получава като в $\alpha$ всяко едно 
  срещане на символа $a$ заменим с (евентуално различна) дума от $L$.
  Тогава, ако $M \subseteq \Sigma^*$ е произволен безконтестен език, да се докаже че езикът
  \begin{equation*}
    M' = \{\beta\in\Sigma^\star |\ \beta \text{ е $L$-вариант на } \alpha \in M \}
  \end{equation*}
  също е безконтекстен.
\end{problem}

\begin{problem}
  Докажете, че всеки безконтекстен език над азбуката $\Sigma = \{a\}$
  е регулярен.
\end{problem}
\ifhints
\begin{hint}
  
\end{hint}
\fi

\begin{problem}
%  \marginpar{\cite{papadimitriou} стр. 149}
  Да фиксираме азбуката $\Sigma$.
  Нека $L$ е безконтекстен език, а $R$ е регулярен език.
  Докажете, че езикът
  \[L/R = \{\alpha \in \Sigma^\star \mid (\exists \beta \in R)[\alpha \beta \in L]\}\]
  е безконтекстен.
\end{problem}
\ifhints
\begin{hint}
  
\end{hint}
\fi

\begin{problem}
  Нека е дадена граматиката $G = \pair{\{a,b\}, \{S,A,B,C\},S,R}$.
  Използвайте CYK-алгоритъма, за да проверите дали
  думата $\alpha$ принадлежи на $\L(G)$, където правилата на граматиката и думата $\alpha$
  са зададени като:
  \begin{enumerate}[a)]
  \item
    $S \to BA\ |\ CA\ |\ a,\ C\to BS\ |\ SA,\ A\to a,\ B\to b$,\\
    $\alpha=bbaaa$;
  \item
    $S \to AB\ |\ BC,\ A\to BA\ |\ a,\ B\to CC\ |\ b,\ C\to AB\ |\ a$,\\
    $\alpha=baaba$;
  \item
    $S \to AB,\ A\to AC\ |\ a\ |\ b,\ B\to CB\ |\ a,\ C\to a$,\\
    $\alpha=babaa$.
  \end{enumerate}
\end{problem}

\begin{problem}
  Нека $L$ е безконтекстен език над азбуката $\Sigma$.
  Докажете, че следните езици са безконтекстни:
  \begin{enumerate}[a)]
  \item 
    $\texttt{Pref}(L) = \{\alpha \in \Sigma^\star \mid (\exists \beta \in \Sigma^\star)[\alpha\cdot\beta \in L]\}$;
  \item 
    $\texttt{Suff}(L) = \{\beta \in \Sigma^\star \mid (\exists \alpha \in \Sigma^\star)[\alpha\cdot\beta \in L]\}$;
  \item
    $\texttt{Infix}(L) = \{\beta \in \Sigma^\star \mid (\exists \alpha,\gamma \in \Sigma^\star)[\alpha \cdot \beta \cdot \gamma \in L]\}$;
  \end{enumerate}
\end{problem}

\begin{problem}
  Докажете, че езикът
  \[L = \{\ \omega_1 \sharp \omega_2 \sharp \cdots \sharp \omega_{2n} \mid n \geq 1\ \&\ \sum^n_{i=1}\abs{\omega_{2i-1}} = \sum^{n}_{i=1}\abs{\omega_{2i}}\ \}\]
  е безконтекстен.
\end{problem}
\begin{hint}
  Най-лесно става със стеков автомат, като са необходими две допълнителни букви за азбуката на стека.

  Една възможна безконтекстна граматика е следната:
  \begin{align*}
    & E \to XEX\ |\ \sharp O\ |\ O\sharp\ |\ \sharp E\sharp\\
    & O \to OO\ |\ EE\ |\ \varepsilon\\
    & X \to a\ |\ b,
  \end{align*}
  където началната променлива е $E$.
\end{hint}

\begin{proposition}
  Да разгледаме езиците:
  \begin{align*}
    & L_{\text{even}} = \{\ \omega_1 \sharp \omega_2 \sharp \cdots \sharp \omega_{2n} \mid n\in\Nat\ \&\ \sum^n_{i=1}\abs{\omega_{2i-1}} = \sum^{n}_{i=1}\abs{\omega_{2i}}\ \}\\
    & L_{\text{odd}} = \{\ \omega_1 \sharp \omega_2 \sharp \cdots \sharp \omega_{2n+1} \mid n\in\Nat\ \&\ \sum^n_{i=0}\abs{\omega_{2i+1}} = \sum^{n}_{i=1}\abs{\omega_{2i}}\ \}.
  \end{align*}
  Тогава:
  \begin{align*}
    (\forall \alpha)[\ \alpha\in L_{\text{even}}\ \implies\ (\exists \alpha_1,\alpha_2)[\ & \alpha \in \{a,b\}^\star\sharp\{a,b\}^\star\ \lor\\
                                                                                        & ( \alpha = \alpha_1\alpha_2\ \&\ \alpha_1 \in L_{\text{even}}\ \&\ \alpha_2 \in L_{\text{odd}})\ \lor\\
                                                                                        &( \alpha = \alpha_1\alpha_2\ \&\ \alpha_1 \in L_{\text{odd}}\ \&\ \alpha_2 \in L_{\text{even}})\ ]\ ]\\
    (\forall \alpha)[\ \alpha\in L_{\text{odd}}\ \implies\ (\exists \alpha_1,\alpha_2)[\ &( \alpha = \alpha_1\alpha_2\ \&\ \alpha_1 \in L_{\text{even}}\ \&\ \alpha_2 \in L_{\text{even}})\ \lor\\
                                                                                        &( \alpha = \alpha_1\alpha_2\ \&\ \alpha_1 \in L_{\text{odd}}\ \&\ \alpha_2 \in L_{\text{odd}})\ ]\ ].
  \end{align*}
\end{proposition}
\begin{hint}
  Да разгледаме една дума $\alpha = \omega_1 \sharp \omega_2 \sharp \cdots \sharp \omega_{2n} \in L_{\text{even}}$.
  Не е възможно $|\omega_{2i-1}| < |\omega_{2i}|$ за $i = 1,\dots,n$, защото тогава бихме имали 
  \[\sum^n_{i=1}|\omega_{2i-1}|\ <\ \sum^n_{i=1}|\omega_{2i}|.\]
  Нека $|\omega_{2i-1}| \geq |\omega_{2i}|$ за първото такова $i$.
  Тогава
  \[\alpha = \omega_1 \sharp \omega_2 \sharp \cdots \sharp\omega'_{2i-1}\omega''_{2i-1}\sharp\omega_{2i}\sharp\cdots \omega_{2n},\]
  където $|\omega''_{2i-1}| = |\omega_{2i}|$.
  Като махнем от $\alpha$ подниза $\omega''_{2i-1}\sharp\omega_{2i}$, 
  получаваме думата
  \[\alpha' = \omega_1 \sharp \omega_2 \sharp \cdots \sharp\omega'_{2i-1}\sharp\cdots \omega_{2n} \in L_{\text{odd}}.\]
  Понеже $|\alpha'| < |\alpha|$, от И.П. знаем, че $\alpha' = \alpha_1\alpha_2$, такива че
  $\alpha_1,\alpha_2 \in L_{\text{even}}$ или $\alpha_1,\alpha_2 \in L_{\text{odd}}$.
\end{hint}


% \begin{problem}
%   Нека с $\code{\A}$ да означим думата над азбуката $\{0,1\}$, която кодира крайния автомат $\A$.
%   Посочете кои от следните езици са регулярни/безконтекстни/разрешими/полуразрешими, където:
%   \begin{enumerate}[a)]
%   \item
%     $L = \{\code{\A} \mid \A\text{ е краен автомат}\}$;
%   \item
%     $L = \{\code{\A} \mid \A\text{ е краен автомат с пет състояния}\}$;
%   \item
%     $L = \{\code{\A} \mid \A\text{ е краен детерминиран автомат}\}$;
%   \item
%     $L = \{\code{\A} \mid \A\text{ е краен детерминиран тотален автомат}\}$;
%   \item
%     $L = \{\code{\A}\cdot\omega \mid \omega \in \Sigma^\star\ \&\ \omega \in \L(\A)\}$;
%   \item
%     $L = \{\code{\A}\cdot\omega \mid \omega \in \Sigma^\star\ \&\ \omega \not\in \L(\A)\}$;
%   \item
%     $L = \{\code{\A} \cdot \code{\B} \mid \A, \B \text{ са крайни автомати и } \L(\A) = \L(\B)\}$;
%   \end{enumerate}
%   Обосновете се!
% \end{problem}

\begin{problem}
  Нека $G = \CFG$ е {\em регулярна} граматика, т.е.
  всички правила на $G$ са от вида $A \to bC$ и $A \to \varepsilon$.
  Посочете кои от следните езици са безконтекстни, където:
  \begin{enumerate}[a)]
  \item 
    $L = \{\alpha\sharp\beta^{\rev} \mid \alpha,\beta \in (V \cup \Sigma)^\star\ \&\ \alpha \vdash_G \beta\}$;
  \item 
    $L = \{\alpha\sharp\beta^{\rev} \mid \alpha,\beta \in (V \cup \Sigma)^\star\ \&\ \alpha \vdash^\star_G \beta\}$;
  \item
    $L = \{\alpha\sharp\beta^{\rev} \mid \alpha,\beta \in (V \cup \Sigma)^\star\ \&\ \alpha \not\vdash_G \beta\}$;
  \item
    $L = \{\alpha\sharp\beta^{\rev} \mid \alpha,\beta \in (V \cup \Sigma)^\star\ \&\ \alpha \not\vdash^\star_G \beta\}$.
  \end{enumerate}
  Обосновете се!
\end{problem}

\begin{problem}
  Да разгледаме една {\em безконтекстна} граматика $G = \CFG$.
  Посочете кои от следните езици са безконтекстни, където:
  \begin{enumerate}[a)]
  \item 
    $L = \{\alpha\sharp\beta^{\rev} \mid \alpha,\beta \in (V \cup \Sigma)^\star\ \&\ \alpha \vdash_G \beta\}$;
  \item
    $L = \{\alpha_1\sharp\alpha^{\rev}_2\sharp\alpha_3\sharp\alpha^{rev}_4\sharp \cdots \mid \alpha_i \in (V \cup \Sigma)^\star\ \&\ \alpha_i \vdash_G \alpha_{i+1} \text{ за }i<n\}$;
  \item 
    $L = \{\alpha\sharp\beta^{\rev} \mid \alpha,\beta \in (V \cup \Sigma)^\star\ \&\ \alpha \vdash^\star_G \beta\}$;
  \item
    $L = \{\alpha\sharp\beta^{\rev} \mid \alpha,\beta \in (V \cup \Sigma)^\star\ \&\ \alpha \not\vdash_G \beta\}$;
  \item
    $L = \{\alpha\sharp\beta^{\rev} \mid \alpha,\beta \in (V \cup \Sigma)^\star\ \&\ \alpha \not\vdash^\star_G \beta\}$.
  \end{enumerate}
  Обосновете се!
\end{problem}

\begin{problem}
  Да разгледаме една {\em неограничена} граматика $G = \CFG$.
  Посочете кои от следните езици са безконтекстни, където:
  \begin{enumerate}[a)]
  \item 
    $L = \{\alpha\sharp\beta^{\rev} \mid \alpha,\beta \in (V \cup \Sigma)^\star\ \&\ \alpha \vdash_G \beta\}$;
  \item 
    $L = \{\alpha\sharp\beta^{\rev} \mid \alpha,\beta \in (V \cup \Sigma)^\star\ \&\ \alpha \vdash^\star_G \beta\}$;
  \item
    $L = \{\alpha\sharp\beta^{\rev} \mid \alpha,\beta \in (V \cup \Sigma)^\star\ \&\ \alpha \not\vdash_G \beta\}$;
  \item
    $L = \{\alpha\sharp\beta^{\rev} \mid \alpha,\beta \in (V \cup \Sigma)^\star\ \&\ \alpha \not\vdash^\star_G \beta\}$.
  \end{enumerate}
  Обосновете се!
\end{problem}



%%% Local Variables:
%%% mode: latex
%%% TeX-master: "../eai"
%%% End:
