\section{Недетерминирани стекови автомати}

\index{автомат!недетерминиран стеков}
\mynote{На англ. {\em Push-down automaton}. В този курс няма да разглеждаме детерминирани стекови автомати. Когато кажем стеков автомат, ще имаме предвид недетерминиран стеков автомат.
  Означаваме $\Sigma_\varepsilon \df \Sigma \cup \{\varepsilon\}$ и $\Gamma^{\leq 2} \df \{\varepsilon\} \cup \Gamma \cup \Gamma^2$.}
{\bf Недетерминиран стеков автомат} е седморка от вида
\[P = \PDA,\] където:
\begin{itemize}
\item
  $Q$ е крайно множество от състояния;
\item  
  $\Sigma$ е крайна входна азбука;
\item
  $\Gamma$ е крайна стекова азбука;
\item
  $\sharp \in \Gamma$ е символ за дъно на стека;
\item
  \mynote{Дефиницията на стеков автомат има много вариации, всички еквивалентни помежду си}
  $\Delta:Q\times\Sigma_\varepsilon\times \Gamma \rightarrow \Ps(Q\times\Gamma^{\leq 2})$ 
  е функция на преходите;    
\item
  $\qstart \in Q$ е начално състояние;
\item
  $\qaccept \in Q$ е заключителното състояние.
\end{itemize}

\mynote{На англ. Instanteneous description}
{\em Моментно описание} (или конфигурация) на изчислението със стеков автомат представлява тройка от вида $(q,\alpha,\gamma) \in Q\times\Sigma^\star\times\Gamma^\star$,
т.е. автоматът се намира в състояние $q$, думата, която остава да се прочете е $\alpha$,
а съдържанието на стека е думата $\gamma$.
Удобно е да въведем бинарната релация $\vdash_P$ над $Q\times\Sigma^\star\times\Gamma^\star$,
която ще ни казва как моментното описание на автомата $P$ се променя след изпълнение на една стъпка:

\begin{figure}[H]
  \begin{subfigure}[b]{0.5\textwidth}
    \begin{prooftree}
      \AxiomC{$\Delta(q,b,A) \ni (p,\beta)$}
      \AxiomC{$b \in \Sigma$}
      \BinaryInfC{$(q,b\alpha,A\gamma) \vdash_P (p,\alpha,\beta\gamma)$}
    \end{prooftree}
  \end{subfigure}
  ~
  \begin{subfigure}[b]{0.5\textwidth}
    \begin{prooftree}
      \AxiomC{$\Delta(q,\varepsilon,A) \ni (p,\beta)$}
      \UnaryInfC{$(q,\alpha,A\gamma) \vdash_P (p,\alpha,\beta\gamma)$}
    \end{prooftree}
  \end{subfigure}
\end{figure}

% \begin{align*}
%   (p,\varepsilon) \in \Delta(q,x,A) & \implies (q,x\alpha,A\gamma) \vdash_P (p,\alpha,\gamma)\\
%   (p,\beta) \in \Delta(q,\varepsilon,Y) & \implies (q,\alpha,Y\gamma) \vdash_P (p,\alpha,\beta\gamma).
% \end{align*}
Рефлексивното и транзитивно затваряне на $\vdash_P$ ще означаваме с $\vdash^\star_P$.
Сега вече можем да дадем дефиниция на език, разпознаван от стеков автомат $P$.
\mynote{Възможно е да се даде и друга еквивалентна дефиниция - разпознаване с празен стек.}
Езикът $\L(P)$, който се разпознава от $P$, има следната дефиниция:
\[\L(P) = \{\ \omega \in \Sigma^\star \mid (\qstart,\omega,\sharp) \vdash^\star_P (\qaccept,\varepsilon,\varepsilon)\ \}.\]

\subsection{Примери}

\begin{extra}
\begin{example}
  \label{ex:anbn}
  За езика $L = \{a^nb^n\mid n\in\Nat\}$, да разгледаме $P = \PDA$, където
  \begin{itemize}
  \item
    $Q \df \{q,p,f\}$;
  \item
    $\qstart \df q$ и $\qaccept \df f$;
  \item
    $\Sigma \df \{a,b\}$ и $\Gamma \df \{\sharp,a\}$;
  \item
    \mynote{Тук получаваме детерминистичен стеков автомат.}
    Релацията на преходите $\Delta$ има следната дефиниция:
    \begin{enumerate}[(1)]
    \item
      $\Delta(q,a,\sharp) \df \{(q, a\sharp)\}$;
    \item
      $\Delta(q,a,a) \df \{(q, aa)\}$; \quad \comment{трупаме $a$-та в стека}
    \item 
      $\Delta(q,\varepsilon,\sharp) \df \{(f,\varepsilon)\}$;\quad \comment{трябва да разпознаем и думата $\varepsilon$}
    \item 
      $\Delta(q, b, a) \df \{(p,\varepsilon)\}$; \quad \comment{Започваме да четем само $b$-та}
    \item 
      $\Delta(p, b, a) \df \{(p,\varepsilon)\}$; \quad \comment{Чистим $a$-тата от стека}
    \item
      $\Delta(p, \varepsilon, \sharp) \df \{(f, \varepsilon)\}$.
    \item
      За всички останали тройки $(r,x,y)$, нека $\Delta(r,x,y) \df \emptyset$.
    \end{enumerate}
  \end{itemize}
  
  Да видим как думата $a^2b^2$ се разпознава от стековия автомат $P$:
  \begin{align*}
    (q, a^2b^2, \sharp) & \vdash_P (q, ab^2, a\sharp) & \comment{\text{правило }(1)}\\
                        & \vdash_P (q, b^2, aa\sharp) & \comment{\text{правило }(2)}\\
                        & \vdash_P (p, b, a\sharp) & \comment{\text{правило }(4)}\\
                        & \vdash_P (p, \varepsilon, \sharp) & \comment{\text{правило }(5)}\\
                        & \vdash_P (f, \varepsilon, \varepsilon) & \comment{\text{правило }(6)}
  \end{align*}
  % \mynote{\writedown Докажете, че $L = \L(P)$!}
  Получихме, че $(\qstart, a^2b^2, \sharp) \vdash^\star_P (\qaccept, \varepsilon, \varepsilon)$, откъдето следва, че $a^2b^2 \in \L(P)$.

  \begin{enumerate}[a)]
  \item
    Докажете с индукция по $n$, че за всяко естествено число $n$ са изпълнени свойствата:
    \begin{align}
      & (q, a^n\beta, \sharp) \vdash^n_P (q, \beta, a^n\sharp) \label{eq:anbn:1}\\
      & (p, b^n, a^n\sharp) \vdash^n_P (p, \varepsilon,\sharp). \label{eq:anbn:2}
    \end{align}
    Заключете, че $L \subseteq \L(P)$.
  \item
    Докажете, че с индукция по $n$, че за всяко естествено число $n$ са изпълнени свойствата:
    \begin{align}
      (q, \alpha\beta, \sharp) \vdash^n_P (q, \beta, \gamma\sharp)  & \implies \alpha = \gamma = a^n \label{eq:anbn:3}\\
      (p, \beta, \gamma\sharp) \vdash^n_P (p, \varepsilon, \sharp) & \implies \beta = b^n\ \&\ \gamma = a^n. \label{eq:anbn:4}
    \end{align}
    Оттук заключете, че $\L(P) \subseteq L$.    
  \end{enumerate}
\end{example}

\begin{example}
  \label{ex:omega-omega-r}
  Езикът $L = \{\ \omega\omega^{\rev} \mid \omega \in \{a,b\}^\star\ \}$ се разпознава от стеков автомат
  \[P = \PDA,\] където:
  \begin{itemize}
  \item 
    $Q \df \{q,p,f\}$ и $\qstart \df q$, $\qaccept \df f$;
  \item
    $\Sigma \df \{a,b\}$, $\Gamma \df \{a, b, \sharp\}$;
  \item
    Функцията на преходите $\Delta$ има следната дефиниция:
    \mynote{За всички липсващи твойки  в дефиницията на $\Delta$ приемаме, че $\Delta$ връща $\emptyset$}
    \begin{enumerate}[(1)]
    \item
     $\Delta(q, x, \sharp) \df \{(q, x\sharp)\}$, където $x \in \{a,b\}$;
    % \item 
    %   $\Delta(q, a, \sharp) \df \{(q, a\sharp)\}$;
    % \item 
    %   $\Delta(q, b, \sharp) \df \{(q, b\sharp)\}$;
    \item
      $\Delta(q, \varepsilon, \sharp) \df \{(q,\varepsilon)\}$;
    \item
      $\Delta(q, x, x) \df \{(q, xx), (p, \varepsilon)\}$, където $x \in \{a,b\}$;
    \item
      $\Delta(q, a, b) \df \{(q, ab)\}$;
    \item
      $\Delta(q, b, a) \df \{(q, ba)\}$;
    % \item
      % $\Delta(q, b, b) \df \{(q, bb), (p, \varepsilon)\}$;
    \item
      $\Delta(p, x, x) \df \{(p,\varepsilon)\}$, където $x \in \{a,b\}$;
    % \item
    %   $\Delta(p, b, b) \df \{(p,\varepsilon)\}$;
    \item
      $\Delta(p, \varepsilon, \sharp) \df \{(f,\varepsilon)\}$;
    \end{enumerate}
  \end{itemize}
  Основното наблюдение, което трябва да направим за да разберем конструкцията на автомата е, че
  всяка дума от вида $\omega\omega^{\texttt{rev}}$ може да се запише като $\omega_1aa\omega^{\texttt{rev}}_1$ или $\omega_1bb\omega^{\texttt{rev}}_1$.
  Да видим защо $P$ разпознава думата $abaaba$ с празен стек.
  Започваме по следния начин:
  \begin{align*}
    (q, abaaba,\sharp) & \vdash_P (q, baaba, a\sharp)   & \comment{\text{правило }(1)}\\
                       & \vdash_P (q, aaba, ba\sharp)   & \comment{\text{правило }(5)}\\
                       & \vdash_P (q, aba,  aba\sharp). & \comment{\text{правило }(4)}
  \end{align*}
  Сега можем да направим два избора как да продължим. Състоянието $p$ служи за маркер, което ни казва, че вече сме започнали 
  да четем $\omega^{\texttt{rev}}$. Поради тази причина, продължаваме така:
  \begin{align*}
    (q, aba, aba\sharp) & \vdash_P (p, ba, ba\sharp) & \comment{\text{правило }(3)}\\
                        & \vdash_P (p, a, a\sharp) & \comment{\text{правило }(6)}\\
                        & \vdash_P (p, \varepsilon, \sharp) & \comment{\text{правило }(6)}\\
                        & \vdash_P (f,\varepsilon,\varepsilon). & \comment{\text{правило }(7)}
  \end{align*}
  Да проиграем още един пример. Да видим защо думата $aba$ не се извежда от автомата.
  \begin{align*}
    (q, aba, \sharp) & \vdash_P (q, ba, a\sharp) & \comment{\text{правило }(1)}\\
                     & \vdash_P (q, a, ba\sharp) & \comment{\text{правило }(5)}\\
                     & \vdash_P (q, \varepsilon, aba\sharp). & \comment{\text{правило }(4)}
  \end{align*}
  От последното моментно описание на автомата нямаме нито един преход, следователно
  думата $aba$ не се разпознава от $P$.
  \mynote{\writedown Докажете, че $L = \L(P)$!}
  \begin{enumerate}[a)]
  \item
    \mynote{За (\ref{eq:omega-omega-r:1}) приложете индукция по дължината на думата $\alpha$. За индукционната стъпка разгледайте $\alpha$ като $\alpha = \alpha' x$.}
    Докажете с индукция пo $n$, за всяко естествено число $n$ са изпълнени свойствата:
    \begin{align}
      & |\alpha| = n\ \implies\ (q, \alpha\beta, \sharp) \vdash^n_P (q, \beta, \alpha^{\texttt{rev}}\sharp) \label{eq:omega-omega-r:1}\\
      & |\beta| = n\ \implies\ (p, \beta, \beta\sharp) \vdash^n_P (p, \varepsilon, \sharp). \label{eq:omega-omega-r:2}
    \end{align}
    Оттук заключете, че $L \subseteq \L(P)$.
  \item
    Докажете с индукця по $n$, че за всяко естествено число $n$ са изпълнени свойствата:
    \begin{align}
      (q, \alpha\beta, \sharp) \vdash^n_P (q, \beta, \gamma\sharp) & \implies \gamma = \alpha^{\rev}\ \&\ |\alpha| = n \label{eq:omega-omega-r:3}\\
      (p, \beta, \gamma\sharp) \vdash^n_P (p, \varepsilon, \sharp) & \implies \gamma = \beta\ \&\ |\beta| = n. \label{eq:omega-omega-r:4}
    \end{align}
    Оттук заключете, че $\L(P) \subseteq L$.
  \end{enumerate}
\end{example}

% \begin{problem}
%   Постройте \emph{детерминистичен} стеков автомат за езика $L = \{\omega \$ \omega^{\rev} \mid \omega \in \{a,b\}^\star \}$.
% \end{problem}

\begin{example}
  \mynote{От \Problem{equal-number-parentheses} знаем, че този език е безконтекстен.}
  Езикът $L = \{\ \omega \in \{a,b\}^\star \mid \abs{\omega}_a = \abs{\omega}_b\}$
  се разпознава от стековия автомат $P = \PDA$, където:
  \begin{itemize}
  \item 
    $Q = \{q,f\}$;
  \item
    $\Sigma = \{a,b\}$;
  \item
    $\Gamma = \{a, b, \sharp\}$;
  \item
    $\qstart = q$ и $\qaccept = f$;
  \item
    Можем да дефинираме релацията на преходите $\Delta$ по следния начин:
    \begin{enumerate}[(1)]
    \item 
      $\Delta(q, \varepsilon, \sharp) = \{(f, \varepsilon)\}$;
    \item
      $\Delta(q, x, \sharp) = \{(q, x\sharp)\}$, където $x \in \{a,b\}$;
    \item
      $\Delta(q, x, x) = \{(q, xx)\}$, където $x \in \{a,b\}$;
    \item
      $\Delta(q, a, b) = \{(q, \varepsilon)\}$;
    \item
      $\Delta(q, b, a) = \{(q, \varepsilon)\}$.
    \end{enumerate}
  \end{itemize}
  Да видим защо думата $abbbaa \in \L(P)$.
  \begin{align*}
    (q, abbbaa, \sharp) & \vdash_P (q, bbbaa,\ a\sharp) & \comment{\text{правило }(2)}\\
                        & \vdash_P (q, bbaa,\ \sharp) & \comment{\text{правило }(5)}\\
                        & \vdash_P (q, baa,\ b\sharp) & \comment{\text{правило }(2)}\\
                        & \vdash_P (q, aa, bb\sharp) & \comment{\text{правило }(3)}\\
                        & \vdash_P (q, a,\ b\sharp) & \comment{\text{правило }(4)}\\
                        & \vdash_P (q, \varepsilon,\ \sharp) & \comment{\text{правило }(4)}\\
                        & \vdash_P (f, \varepsilon,\ \varepsilon). & \comment{\text{правило }(1)}
  \end{align*}

  \begin{enumerate}[a)]
  \item
    Докажете с пълна индукция по дължината на думата $\gamma$, че за произволна дума $\gamma \in \{a, b\}^\star$, е изпълнено, че:
    \begin{align}
      (\forall n)[a^n\gamma \in L & \implies (q, \gamma, a^n\sharp) \vdash^\star_P (q, \varepsilon, \sharp)] \label{eq:omega-ab:1}\\
      (\forall n)[b^n\gamma \in L & \implies (q, \gamma, b^n\sharp) \vdash^\star_P (q, \varepsilon, \sharp)]. \label{eq:omega-ab:2}
    \end{align}
    % Ще докажем едновременно \Property{eq:omega-ab:1} и \Property{eq:omega-ab:2} с пълна индукция по дължината на думата $\gamma$.

    % Да разгледаме произволна дума $\gamma$. Случаят, когато $\abs{\gamma} = 0$ е тривиален. 
    % Нека $\abs{\gamma} > 0$ и да приемем, че $a^n\gamma \in L$.
    % Ясно е, че можем да представим $\gamma$ като $\gamma = a^kb\gamma'$, за някое $k$.
    % \begin{itemize}
    % \item
    %   Ако $n+k = 0$, то $a^n\gamma = b\gamma' \in L$ и прилагаме \IndHyp за \Property{eq:omega-ab:2} с думата $\gamma'$ и получаваме, че:
    %   \begin{align*}
    %     (q, \overbrace{b\gamma'}^{\gamma},\sharp) & \vdash (q,\gamma',b\sharp) & \comment{\text{правило (2)}}\\
    %                          & \vdash^\star (q, \varepsilon,\sharp) & \comment\text{\IndHyp}
    %   \end{align*}
    % \item
    %   Ако $n+k>0$, то $a^{n+k-1}\gamma' \in L$ и прилагаме \IndHyp за \Property{eq:omega-ab:1} с думата $\gamma'$ и получаваме, че:
    %   \begin{align*}
    %     (q, \overbrace{a^{k}b\gamma'}^{\gamma}, a^n\sharp) & \vdash^\star_P (q, b\gamma', a^{n+k}\sharp) & \comment{\text{правило (3)}}\\
    %                                                         & \vdash^\star_P (q, \gamma', a^{n+k-1}\sharp) & \comment\text{правило (5)}\\
    %                                                         & \vdash^\star_P (q, \varepsilon, \sharp). & \comment\text{\IndHyp}
    %   \end{align*}
    % \end{itemize}
    % Аналогично се доказва и \Property{eq:omega-ab:2}.
    
    Оттук заключете, че $L \subseteq \L(P)$.
  \item
    Докажете с пълна индукция по дължината на думата $\gamma$, че за произволна дума $\gamma \in \{a, b\}^\star$ е изпълнено, че:
    \begin{align}
      (\forall n)[(q, \gamma, a^n\sharp) \vdash^\star_P (q, \varepsilon, \sharp) & \implies a^n\gamma \in L] \label{eq:omega-ab:3}\\
      (\forall n)[(q, \gamma, b^n\sharp) \vdash^\star_P (q, \varepsilon, \sharp) & \implies b^n\gamma \in L]. \label{eq:omega-ab:4}
    \end{align}

    % Нека имаме изчислението $(q, \gamma, a^n\sharp) \vdash^\star_P (q, \varepsilon, \sharp)$.
    % Да представим думата $\gamma$ като $\gamma = a^kb\gamma'$.
    
    % \begin{itemize}
    % \item
    %   Ако $n+k = 0$, то можем да разбием изчислението по следния начин:
    %   \begin{align*}
    %     (q, b\gamma', \sharp) & \vdash_P (q, \gamma',b\sharp) \\
    %                           & \vdash^{\star} (q,\varepsilon,\sharp).
    %   \end{align*}
    %   Тогава от \IndHyp за \Property{eq:omega-ab:4} следва, че $a^n\gamma = b\gamma' \in L$.
    % \item
    %   Ако $n+k > 0$, то можем да разбием изчислението по следния начин:
    %   \begin{align*}
    %     (q, a^kb\gamma', a^n\sharp) & \vdash^\star_P (q, b\gamma',a^{n+k}\sharp) \\
    %                                 & \vdash_P (q, \gamma',a^{n+k-1}\sharp) \\
    %                                 & \vdash^{\star} (q,\varepsilon,\sharp).
    %   \end{align*}
    %   Тогава от \IndHyp за \Property{eq:omega-ab:3} следва, че $a^{n+k-1}\gamma' \in L$, но оттук
    %   веднага получаваме, че $a^na^kb\gamma' \in L$.
    % \end{itemize}
    % Аналогично се доказва и \Property{eq:omega-ab:4}.
    
    Оттук заключете, че $\L(P) \subseteq L$.
  \end{enumerate}
\end{example}

\begin{example}
  \mynote{От \Problem{balanced-parentheses} знаем, че този език е безконтекстен.}
  Езикът $L = \{\ \omega \in \{a,b\}^\star \mid \omega\text{ е балансирана дума}\ \}$
  се разпознава от стековия автомат $P = \PDA$, където:
  \begin{itemize}
  \item 
    $Q = \{q,f\}$;
  \item
    $\qstart = q$ и $\qaccept = f$;
  \item
    $\Sigma = \{a,b\}$ и $\Gamma = \{a, \sharp\}$;
  \item
    Можем да дефинираме релацията на преходите $\Delta$ по следния начин:
    \mynote{\writedown Докажете, че $L = \L(P)$!}
    \begin{enumerate}[(1)]
    \item 
      $\Delta(q, \varepsilon, \sharp) = \{(f, \varepsilon)\}$;
    \item
      $\Delta(q, a, \sharp) = \{(q, a\sharp)\}$;
    \item
      $\Delta(q, a, a) = \{(q, aa)\}$;
    \item
      $\Delta(q, b, a) = \{(q, \varepsilon)\}$;
    \end{enumerate}
  \end{itemize}  
  \begin{enumerate}[(a)]
  \item
    \mynote{Индукция по дължината на думата $\beta$.}
    Докажете, че за произволно естествено число $n$ и произволна дума $\beta \in \{a, b\}^\star$, 
    е изпълнено, че:
    \[a^n\beta \in L\ \implies (q, \beta, a^n\sharp) \vdash^\star_P (q, \varepsilon, \sharp).\]
    Оттук заключете, че $L \subseteq \L(P)$.
  \item
    \mynote{Индукция по броя на стъпките в изчислението на стековия автомат.}
    Докажете, че за произволно естествено число $n$ и произволна дума $\beta \in \{a, b\}^\star$, е изпълнено, че:
    \[(q,\beta,a^n\sharp) \vdash^\star_P (q, \varepsilon, \sharp)\ \implies\ a^n\beta \in L.\]
    Оттук заключете, че $\L(P) \subseteq L$.
  \end{enumerate}
\end{example}
\end{extra}


%%% Local Variables:
%%% mode: latex
%%% TeX-master: "../eai"
%%% End:


\subsection{Теорема за еквивалентност}

\begin{important}
  \begin{lemma}
    \mynote{\cite[стр. 136]{papadimitriou}}
    За всяка безконтекстна граматика $G$,
    съществува стеков автомат $P$, такъв че $\L(G) = \L(P)$.
  \end{lemma}
\end{important}
\begin{proof}
  % \mynote{Доказателството в \cite[стр. 117]{sipser3} не ми харесва}
  \mynote{Тук е важно, че дефенирахме най-ляв извод в граматика.}
  Нека е дадена безконтекстната граматика $G = \CFG$ в нормална форма на Чомски.
  Нашата цел е да построим стеков автомат
  \[P = \PDA,\] който разпознава $\L(G)$.
  \begin{itemize}
  \item
    $Q = \{\qstart,p,\qaccept\}$;
  \item
    $\Gamma = \Sigma \cup V \cup \{\sharp\}$;
  \item
    Релацията на преходите $\Delta$ дефинираме по следния начин:
    \mynote{Понеже граматиката е в нормална форма на Чомски, то $|\alpha| \leq 2$ и удовлетворяваме дефиницията на $\Delta$.}
    \begin{enumerate}[(1)]
    \item 
      $\Delta(\qstart, \varepsilon, \sharp ) = \{(p,S\sharp)\}$;
    \item
      $\Delta(p,\varepsilon,A) = \{(p,\alpha)\mid A\to_G \alpha\}, \text{ за всяка променлива }A \in V$;
    \item
      $\Delta(p,a,a) = \{(p,\varepsilon)\}, \text{ за всяка буква } a \in \Sigma$;
    \item
      $\Delta(p,\varepsilon,\sharp) = \{(\qaccept, \varepsilon)\}$.
    \end{enumerate}
  \end{itemize}

  \mynote{Ако $\gamma$ не е празната дума, то $\gamma$ започва с променлива.}
  Ще докажем, че за всяка променлива $A \in V$, за всяка дума $\alpha \in \Sigma^\star$, $\gamma \in (V \cdot \Sigma^\star)^\star$ и $\delta \in (V \cup \Sigma)^\star$, то е изпълнено, че:
  \begin{enumerate}[(a)]
  \item
    ако $S \lderive{\star} \alpha \gamma$, то $(p, \alpha, S\sharp) \vdash^\star_P (p, \varepsilon, \gamma\sharp)$;
  \item
    ако $(p, \alpha, \delta\sharp) \vdash^\star_P (p, \varepsilon, \sharp)$, то $\delta \lderive{\star} \alpha$.
  \end{enumerate}
  Тогава, ако вземем $\delta = S$ и $\gamma = \varepsilon$, то ще получим, че
  \begin{align*}
    \alpha \in \L(G) & \iff S \lderive{\star} \alpha\\
                     & \iff (p,\alpha,S\sharp) \vdash^\star_P (p, \varepsilon, \sharp) & \comment{\text{от (а) и (б)}}\\
                     & \iff (\qstart,\alpha,\sharp) \vdash^\star_P (\qaccept, \varepsilon, \varepsilon) & \comment{\text{от деф. на }P}\\
                     & \iff \alpha \in \L(P).
  \end{align*}

  Сега преминаваме към доказателствата на двете твърдения.

  \begin{enumerate}[(a)]
  \item
    Индукция по дължината $\ell$ на извода $S \lderive{\ell} \alpha\gamma$.
    % \begin{itemize}
    % \item
      Нека $\ell = 0$. Този случай е тривиален, защото тогава $\alpha = \varepsilon$ и $\gamma = S$.
      Ясно е, че $(p,\varepsilon,S\sharp) \vdash^0_P (p,\varepsilon,S\sharp)$.
      % \item
      
      Нека $\ell > 0$ и $S \lderive{\ell} \alpha\gamma$. Това означава, че този извод може да се запише по следния начин:
      \begin{prooftree}
        \AxiomC{$S\lderive{\ell-1} \alpha_1A\gamma_2$}
        \AxiomC{$\alpha_1 \in \Sigma^\star$}
        \AxiomC{$A \to_G \alpha_2\gamma_1$}
        \RightLabel{\scriptsize{(\Proposition{grammar:context-left-step})}}
        \TrinaryInfC{$S \lderive{\ell} \underbrace{\alpha_1\alpha_2}_{\alpha}\underbrace{\gamma_1\gamma_2}_{\gamma}$}
      \end{prooftree}
      Тогава от \IndHyp имаме, че
      \begin{equation}
        \label{eq:5}
        (p, \alpha_1, S\sharp) \vdash^\star_P (p, \varepsilon, A\gamma_2\sharp).
      \end{equation}
      Получаваме следното изчисление на стековия автомат:
      \begin{align*}
        (p, \overbrace{\alpha_1\alpha_2}^{\alpha}, S\sharp) & \vdash^\star_P (p, \alpha_2, A\gamma_2\sharp) & \comment{\text{от (\ref{eq:5})}}\\
                                                            & \vdash_P (p, \alpha_2, \alpha_2\gamma_1\gamma_2\sharp) & \comment{\text{ред (2) от деф. на }\Delta}\\
                                                            & \vdash^\star_P (p, \varepsilon, \underbrace{\gamma_1\gamma_2}_{\gamma}\sharp) & \comment{\text{ред (3) от деф. на }\Delta}.
      \end{align*}
    % \end{itemize}
  \item
    Индукция по броя на стъпките $\ell$ в изчислението на стековия автомат.
    % \begin{itemize}
    % \item
    
      Нека $\ell = 0$. Тогава е ясно, че единствената възможност $\alpha = \varepsilon$ и $\delta = \varepsilon$.
      Тогава $\varepsilon \derive{\star}_G \varepsilon$.
      % \item
      
      Нека $\ell > 0$ и $(p, \alpha, \delta \sharp) \vdash^{\ell}_P (p, \varepsilon, \sharp)$.      
      Имаме три избора за първата стъпка в това изчисление.
      
      Започваме със случая, когато $\Delta(p,a,a) \ni (p,\varepsilon)$. Това означава, че $\alpha = a\beta$, $\delta = a\rho$ и
      $(p, \alpha, \delta \sharp) \vdash_P (p,\beta,\rho\sharp ) \vdash^{\ell-1}_P (p, \varepsilon, \sharp)$.
      Тук имаме следния извод:
      \begin{prooftree}
        \AxiomC{$a \lderive{0} a$}
        \AxiomC{$(p,\beta,\rho\sharp ) \vdash^{\ell-1}_P (p, \varepsilon, \sharp)$}
        \RightLabel{\scriptsize{\IndHyp}}
        \UnaryInfC{$\rho \lderive{\star} \beta$}
        \RightLabel{\scriptsize{правило (2)}}
        \BinaryInfC{$\underbrace{a \rho}_{\delta} \lderive{\star} \underbrace{a\beta}_{\alpha}$}
      \end{prooftree}
    % \item
      Вторият случай е ако $\Delta(p,\varepsilon,A) \ni (p,a)$. Това означава, че $A \to_G a$, $\delta = A\rho$ и
      \[(p, \alpha, \delta \sharp) \vdash_P (p,\alpha,a\rho\sharp ) \vdash^{\ell-1}_P (p, \varepsilon, \sharp).\]
      Можем да направим следния извод:
      \begin{prooftree}
        \AxiomC{$A \to_G a$}
        \AxiomC{$\rho \lderive{0} \rho$}
        \LeftLabel{\scriptsize{правило (2)}}
        \BinaryInfC{$A\rho \lderive{1} a\rho$}
        \AxiomC{$(p,\alpha,a\rho\sharp ) \vdash^{\ell-1}_P (p, \varepsilon, \sharp)$}
        \RightLabel{\scriptsize{\IndHyp}}
        \UnaryInfC{$a\rho \lderive{\star} \alpha$}
        \LeftLabel{\scriptsize{(\Proposition{unrestricted-grammar:general-step})}}
        \BinaryInfC{$\underbrace{A\rho}_{\delta} \lderive{\star} \alpha$}
      \end{prooftree}
    % \item
      Последният случай е ако $\Delta(p,\varepsilon,A) \ni (p,BC)$. Това означава, че $A \to_G BC$, $\delta = A\rho$ и
      $(p, \alpha, \delta \sharp) \vdash_P (p,\alpha, BC\rho\sharp ) \vdash^{\ell-1}_P (p, \varepsilon, \sharp)$.
      Можем да направим следния извод:
      \begin{prooftree}
        \AxiomC{$A \to_G BC$}
        \AxiomC{$\rho \lderive{0} \rho$}
        \LeftLabel{\scriptsize{правило (2)}}
        \BinaryInfC{$A\rho \lderive{1} BC\rho$}
        \AxiomC{$(p,\alpha, BC\rho\sharp ) \vdash^{\ell-1}_P (p, \varepsilon, \sharp)$}
        \RightLabel{\scriptsize{\IndHyp}}
        \UnaryInfC{$BC\rho \lderive{\star} \alpha$}
        \LeftLabel{\scriptsize{(\Proposition{unrestricted-grammar:general-step})}}
        \BinaryInfC{$\underbrace{A\rho}_{\delta} \lderive{\star} \alpha$}
      \end{prooftree}
    % \end{itemize}
  % \end{itemize}
\end{enumerate}
\end{proof}

\begin{important}
  \begin{lemma}
    За всеки стеков автомат $P$, съществува безконтекстна граматика $G$, такава че $\L(P) = \L(G)$.
  \end{lemma}
\end{important}
\begin{proof}
  Нека е даден стековия автомат
  \[P = \PDA.\]
  \mynote{Граматиката, която ще получим няма да бъде в нормална форма на Чомски.}
  Ще дефинираме безконтекстна граматика $G$, за която $\L(P) = \L(G)$.
  Променливите на граматика са 
  \[V = \{[q,A,p] \mid q,p \in Q\ \&\ A \in \Gamma\}.\]
  Правилата на $G$ са следните:
  \begin{itemize}
  \item
    Началната променлива е $S \df [\qstart,\sharp,\qaccept]$;
  \item
    Нека имаме $(r,BC) \in \Delta(q, a, A)$, където $a \in \Sigma_\varepsilon$.
    Тогава добавяме правилата:
    \[[q,A,p] \to_G a[r,B,q'][q',C,p],\]
    за всеки две състояния $q'$ и $p$.
  \item
    Нека имаме $(r,B) \in \Delta(q, a, A)$, където $a \in \Sigma_\varepsilon$.
    Тогава добавяме правилата:
    \[[q,A,p] \to_G a[r,B,p],\]
    за всяко състояние $p \in Q$.
  \item
    Нека имаме $(p,\varepsilon) \in \Delta(q,a,A)$, където $a \in \Sigma_\varepsilon$.
    Тогава добавяме правилата:
    \[[q,A,p] \to_G a.\]
  \end{itemize}
  След като вече сме обяснили какви правила включва граматиката $G$,
  трябва да докажем, че за произволна дума $\alpha \in \Sigma^\star$, произволни състояния $q$ и $p$,
  и произволен символ $A \in \Gamma$, е изпълнено, че:
  \[[q,A,p] \lderive{\star} \alpha\ \text{точно тогава, когато}\ (q,\alpha,A) \vdash^\star_{P} (p,\varepsilon,\varepsilon).\]
  \begin{description}
  \item[$(\Rightarrow)$]
    С пълна индукция по броят на стъпките $\ell$ в изчислението на стековия автомат $P$ ще докажем, че за произволно $\ell \geq 1$,
    \[\text{ако }(q,\alpha,A) \vdash^\ell_P (p,\varepsilon,\varepsilon)\text{, то } [q,A,p] \lderive{\star} \alpha.\]
    Ако $\ell = 1$, то е лесно, защото $\alpha = a \in \Sigma_\varepsilon$.
    Тогава $(p,\varepsilon) \in \Delta(q,a,A)$ и според конструкцията на граматиката $G$ имаме правилото $[q,A,p] \to_G a$.
    
    Ако $\ell > 1$, то в зависимост от първата стъпка на изчислението, имаме два случая.
    Нека $\alpha = a\beta$, където $a \in \Sigma_\varepsilon$.
    \begin{itemize}
    \item 
      Ако $\Delta(q,a,A) \ni (r,B)$, то имаме, че:
      \[(q,a\beta,A) \vdash_P (r,\beta,B) \vdash^{\ell-1}_P (p, \varepsilon, \varepsilon).\]
      Можем да направим следния извод:
      \begin{prooftree}
        \AxiomC{$\Delta(q,a,A) \ni (r,B)$}
        \LeftLabel{\scriptsize{(деф.)}}
        \UnaryInfC{$[q,A,p] \to_G a[r,B,p]$}
        \AxiomC{$(r,\beta,B) \vdash^{\ell-1}_P (p, \varepsilon, \varepsilon)$}
        \RightLabel{\scriptsize{\IndHyp}}
        \UnaryInfC{$[r,B,p] \lderive{\star} \beta$}
        \RightLabel{\scriptsize{(\Proposition{grammar:context-left-step}})}
        \BinaryInfC{$[q,A,p] \lderive{\star} \underbrace{a\beta}_{\alpha}$}
      \end{prooftree}
    \item
      Ако $\Delta(q, a, A) \ni (r, BC)$, то имаме, че:
      \[(q, a\beta, A) \vdash_P (r, \beta, BC) \vdash^{\ell-1}_P (p, \varepsilon, \varepsilon).\]      
      За $\ell-1$ стъпки трябва да стигнем от стек с големина $2$ до празен стек.
      Това означава, че можем да разбием думата $\beta$ на две части, $\beta = \beta_1\beta_2$, със свойството, че след като прочетем $\beta_1$,
      то стекът има големина $1$ и след като прочетем $\beta_2$, то стекът е празен.
      \mynote{Да обърнем внимание, че в междинните стъпки от двете изчисления, стекът може да расте.}
      Това означава, че съществува състояние $q'$, за което можем да разбием изчислението по следния начин:
      \begin{align*}
        & (r, \beta_1, B) \vdash^{\ell_1}_P (q',\varepsilon,\varepsilon)\\
        & (q', \beta_2, C) \vdash^{\ell_2}_P (p,\varepsilon,\varepsilon),\\
        & \ell_1 + \ell_2 = \ell - 1.
      \end{align*}
      Понеже $\ell_1 < \ell$ и $\ell_2 < \ell$, от \IndHyp имаме следното:
      \begin{align*}
        & (r, \beta_1, B) \vdash^{\ell_1}_P (q', \varepsilon, \varepsilon) \implies [r, B, q'] \lderive{\star} \beta_1\\
        & (q', \beta_2, C) \vdash^{\ell_2}_P (p, \varepsilon, \varepsilon) \implies [q', C, p] \lderive{\star} \beta_2.
      \end{align*}
      Понеже имаме, че $\Delta(q,a,A) \ni (r,BC)$, то в граматиката имаме правилото
      \[[q,A,p] \to_G a[r,B,q'][q',C,p].\]
      Обединявайки всичко, получаваме извода
      \begin{prooftree}
        \AxiomC{$\Delta(q,a,A) \ni (r,BC)$}
        % \LeftLabel{\scriptsize{(деф.)}}
        \UnaryInfC{$[q,A,p] \to_G a[r,B,q'][q',C,p]$}
        \AxiomC{$(r, \beta_1, B) \vdash^{\ell_1}_P (q', \varepsilon, \varepsilon)$}
        \LeftLabel{\scriptsize{\IndHyp}}
        \UnaryInfC{$[r, B, q'] \lderive{\star} \beta_1$}
        \AxiomC{$(q', \beta_2, C) \vdash^{\ell_1}_P (p, \varepsilon, \varepsilon)$}
        % \RightLabel{\scriptsize{\IndHyp}}
        \UnaryInfC{$[q', C, p] \lderive{\star} \beta_2$}
        \LeftLabel{\scriptsize{(2)}}
        \BinaryInfC{$[r,B,q'][q',C,p] \lderive{\star} \beta_1\beta_2$}
        \BinaryInfC{$[q,A,p] \lderive{\star} \underbrace{a\beta_1\beta_2}_{\alpha}$}
      \end{prooftree}
    \end{itemize}
  \item[$(\Leftarrow)$]
    За тази посока, с пълна индукция по дължината на извода $\ell$ в граматиката $G$ ще докажем, че за произволна дължина на извода $\ell \geq 1$,
    \[\text{ако }[q,A,p] \lderive{\ell} \alpha\text{, то }(q,\alpha,A) \vdash^\star_P (p,\varepsilon,\varepsilon).\]
    \begin{itemize}
    \item
      Нека $\ell = 1$. Тогава имаме $[q,A,p] \to_G \alpha$, където $\alpha \in \Sigma_\varepsilon$.
      Този случай е ясен от дефиницията на граматиката $G$, т.е. $\Delta(q,a,A) \ni (p,\varepsilon)$.
    \item
      Нека $\ell > 1$. Тогава имаме, че $\alpha = a\beta$ и според правилата на граматиката $G$ имаме два случая.
      \mynote{Тук отново е възможно $a = \varepsilon$. Това не е проблем, защото правим индукция по дължината на извода, а не по дължината на думата $\alpha$.}
      Да приемем, че имаме следния извод:
      \begin{prooftree}
        \AxiomC{$[q,A,p] \lderive{1} a[r,B,p]$}
        \AxiomC{$[r,B,p] \lderive{\ell-1} \beta$}
        \BinaryInfC{$[q,A,p] \lderive{\ell} \underbrace{a\beta}_{\alpha}$}
      \end{prooftree}
      Тогава директно прилагаме \IndHyp и получаваме, че
      $(r, \beta, B) \vdash^\star_P (p, \varepsilon, \varepsilon)$ и накрая получаваме, че $(q, a\beta, A) \vdash^\star_P (p, \varepsilon, \varepsilon)$.
      
      Сега да разгледаме втория случай:
      \begin{prooftree}
        \AxiomC{$[q,A,p] \lderive{1} a[r,B,q'][q',C,p]$}
        \AxiomC{$[r,B,q'][q',C,p] \lderive{\ell-1} \beta$}
        \BinaryInfC{$[q,A,p] \lderive{\ell} \underbrace{a\beta}_{\alpha}$}
      \end{prooftree}
      \mynote{Важно е, че $\beta \in \Sigma^\star$. Иначе няма да можем да приложим \Proposition{grammar:divide-2}.}
      Понеже $\beta \in \Sigma^\star$, можем да приложим \Proposition{grammar:divide-2} за $\lderive{\star}$ и оттам следва,
      че имаме разбиване на думата $\beta$ като $\beta = \beta_1\beta_2$, където 
      \begin{align*}
        & [r,B,q'] \lderive{\ell_1} \beta_1\\
        & [q',C,p] \lderive{\ell_2} \beta_2,\\
        & \ell_1 + \ell_2 = \ell - 1.
      \end{align*}
      Понеже $\ell_1 < \ell$ и $\ell_2 < \ell$, от \IndHyp получаваме, че 
      \begin{align*}
        & [r,B,q'] \lderive{\ell_1} \beta_1 \implies (r,\beta_1,B) \vdash^\star_P (q',\varepsilon,\varepsilon) \\
        & [q',C,p] \lderive{\ell_2} \beta_2 \implies (q',\beta_2,C) \vdash^\star_P (p,\varepsilon,\varepsilon).
      \end{align*}
      Правилото $[q,A,p] \rightarrow_G a[r,B,q'][q',C,p]$ 
      е добавено в граматиката, защото $\Delta(q,a,A) \ni (r, BC)$. 
      Обединявайки всичко, което знаем, получаваме:
      \begin{align*}
        (q, a\beta, A) & \vdash_P (r, \beta_1\beta_2, BC)\\
                       & \vdash^\star_P (q', \beta_2, C)\\
                       & \vdash^\star_P (p, \varepsilon, \varepsilon).
      \end{align*}    
    \end{itemize}
  \end{description}
\end{proof}

Предишните две леми ни дават следната теорема.
\begin{important}
  \begin{theorem}
    \label{th:push-down-context-free}
    Класът на езиците, които се разпознават от краен стеков автомат съвпада с
    класа на безконтекстните езици.
  \end{theorem}
\end{important}

\begin{extra}
\begin{example}
  Нека е дадена граматиката $G$ с правила 
  \begin{align*}
    & S \to AS\ |\ BS\ |\ \varepsilon\\
    & A \to aA\ |\ a\\
    & B \to Bb\ |\ b.
  \end{align*}
  Ще построим стеков автомат $P = \PDA$, такъв че $\L(P) = \L(G)$.
  \begin{itemize}
  \item
    $\Sigma = \{a,b\}$;
  \item 
    $\Gamma = \{A,S,B,a,b,\sharp\}$;
  \item
    $Q = \{\qstart,q,\qaccept\}$;
  \item
    Дефинираме релацията на преходите, следвайки конструкцията от \Theorem{push-down-context-free}:
    \begin{itemize}
    \item
      $\Delta(\qstart, \varepsilon, \sharp) = \{(q, S\sharp)\}$;
    \item 
      $\Delta(q, \varepsilon, S) = \{(q, AS), (q, BS), (q, \varepsilon)\}$;
    \item
      $\Delta(q, \varepsilon, A) = \{(q, aA), (q, a)\}$;
    \item
      $\Delta(q, \varepsilon, B) = \{(q, Bb), (q, b)\}$;
    \item
      $\Delta(q, a, a) = \{(q, \varepsilon)\}$;
    \item
      $\Delta(q, b, b) = \{(q, \varepsilon)\}$;
    \item
      $\Delta(q, \varepsilon, \sharp) = \{(\qaccept,\varepsilon)\}$;
    \end{itemize}
  \end{itemize}
\end{example}
\end{extra}

\begin{important}
  \begin{theorem}\label{th:intersection-context-reg}
    Нека $L$ e безконтекстен език и $R$ е регулярен език.
    Тогава тяхното сечение $L \cap R$ е безконтекстен език.
  \end{theorem}
\end{important}
\begin{hint}
  \mynote{\cite[стр. 144]{papadimitriou}}
  Нека имаме стеков автомат
  \[P = \pair{Q',\Sigma,\Gamma,\sharp, \Delta', \qstart', \qaccept'}, \text{ където } \L(P) = L,\]
  и детерминиран краен автомат 
  \[\A = \pair{Q'', \Sigma, \qstart'', \delta'', F''}, \text{ където } \L(\A) = R.\]
  \mynote{Сравнете с конструкцията от \Proposition{automata-cap}.}
  Ще определим нов стеков автомат $\M = \PDA$, където:
  \begin{itemize}
  \item 
    $Q \df Q' \times Q''$;
  \item
    $\qstart \df \pair{\qstart',\qstart''}$;
  \item
    $F \df \{\qaccept'\} \times F''$;
  \item 
    Функцията на преходите $\Delta$ е дефинирана както следва:
    \begin{itemize}
    \item 
      \mynote{Симулираме едновременно изчислението и на двата автомата.}
      Ако $(r_1,Z) \in \Delta'(q_1, a, Y)$, то
      \[(\pair{r_1,\delta''(q_2,a)}, Z) \in \Delta(\pair{q_1,q_2},a,Y).\]
    \item
      \mynote{Нищо не четем от входната дума, следователно правим празен ход на $\A$}
      Ако $(r_1,Z) \in \Delta'(q_1,\varepsilon,Y)$ и всяко $q_2 \in Q''$, то
      \[(\pair{r_1,q_2}, Z) \in \Delta(\pair{q_1,q_2},\varepsilon,Y).\]
    \item
      \mynote{\writedown Докажете, че $\L(\M) = \L(P) \cap \L(\A)$ !}
      $\Delta$ не съдържа други преходи;
    \end{itemize}
  \end{itemize}

  \begin{itemize}
  \item
    \mynote{Индукция по броя стъпки в изчислението на $\M$.}
    Докажете, че ако $(\pair{q_1,q_2},\alpha,\gamma) \vdash^\star_\M (\pair{p_1,p_2},\varepsilon,\varepsilon)$, то
    \[(q_1,\alpha,\gamma) \vdash^\star_P (p_1,\varepsilon,\varepsilon)\text{ и }(q_2,\alpha) \vdash^\star_\A (p_2,\varepsilon).\]
  \item
    \mynote{Индукция по броя стъпки в изчислението на $P$.}
    Докажете, че ако $(q_1,\alpha,\gamma) \vdash^\star_P (p_1,\varepsilon,\varepsilon)$ и $(q_2,\alpha) \vdash^\star_\A (p_2,\varepsilon)$, то
    \[(\pair{q_1,q_2},\alpha,\gamma) \vdash^\star_\M (\pair{p_1,p_2},\varepsilon,\varepsilon).\]
  \end{itemize}
  
\end{hint}

\Theorem{intersection-context-reg} е удобна, когато искаме да докажем, че даден език не е безконтекстен.
С нейна помощ можем да сведем езика до друг, за който вече знаем, че не е безконтекстен.

\begin{extra}
  \begin{example}
    Да разгледаме езика $L = \{\omega \in \{a,b,c\}^\star \mid \card{\omega}{a} = \card{\omega}{b} = \card{\omega}{c}\}$.
    Да допуснем, че $L$ е безконтекстен език. Тогава, според \Theorem{intersection-context-reg}, $L^\prime = L \cap \L(a^\star b^\star c^\star)$ също е безконтекстен език.
    Но $L^\prime = \{a^nb^nc^n \mid n \in \Nat\}$, за който знаем от \Example{anbncn}, че {\em не} е безконтекстен.
    Достигнахме до противоречие. Следователно, $L$ не е безконтекстен език.
  \end{example}
\end{extra}

\begin{important}
  \begin{remark}
    Не е вярно, че сечението на всеки два безконтекстни езика е безконтекстен език.
    Например, $L_1 = \{a^nb^nc^k \mid n,k\in\Nat\}$ и $L_2 = \{a^kb^nc^n \mid n,k\in\Nat\}$
    са безконтекстни езици, но ние знаем, че $L_1 \cap L_2 = \{a^nb^nc^n \mid n \in \Nat\}$
    не е безконтекстен.
  \end{remark}
\end{important}


%%% Local Variables:
%%% mode: latex
%%% TeX-master: "../eai"
%%% End:
