\section{Недетерминирани стекови автомати}

\index{автомат!недетерминиран стеков}
\marginpar{На англ. {\em Push-down automaton}}
\marginpar{В този курс няма да разглеждаме детерминирани стекови автомати. Когато кажем стеков автомат, ще имаме предвид недетерминиран стеков автомат.}
{\bf Недетерминиран стеков автомат} е седморка от вида
\[P = \PDA,\] където:
\begin{itemize}
\item
  $Q$ е крайно множество от състояния;
\item  
  $\Sigma$ е крайна входна азбука;
\item
  $\Gamma$ е крайна стекова азбука;
% \item
%   $\# \in \Gamma$ е символ за дъно на стека;
\item
  $\qstart \in Q$ е начално състояние;
\item
  \marginpar{Озн. $\Ps_{fin}(A)$ - крайните подмножества на $A$}
  \marginpar{Дефиницията на стеков автомат има много вариации, всички еквивалентни помежду си}
  $\Delta:Q\times(\Sigma \cup \{\varepsilon\})\times (\Gamma \cup \{\varepsilon\})\rightarrow \Ps_{fin}(Q\times\Gamma^\star)$ 
  е функция на преходите;    
\item
  $F\subseteq Q$ е множество от заключителни състояния.
\end{itemize}

\marginpar{Instanteneous description}
{\em Моментно описание} (или конфигурация) на изчислението със стеков автомат представлява тройка от вида $(q,\alpha,\gamma) \in Q\times\Sigma^\star\times\Gamma^\star$,
т.е. автоматът се намира в състояние $q$, думата, която остава да се прочете е $\alpha$,
а съдържанието на стека е думата $\gamma$.
Удобно е да въведем бинарната релация $\vdash_P$ над $Q\times\Sigma^\star\times\Gamma^\star$,
която ще ни казва как моментното описание на автомата $P$ се променя след изпълнение на една стъпка:
\begin{align*}
  (p,\beta) \in \Delta(q,x,Y) & \implies (q,x\alpha,Y\gamma) \vdash_P (p,\alpha,\beta\gamma)\\
  (p,\beta) \in \Delta(q,\varepsilon,Y) & \implies (q,\alpha,Y\gamma) \vdash_P (p,\alpha,\beta\gamma)\\
  (p,\varepsilon) \in \Delta(q,\varepsilon,\varepsilon) & \implies (q,\alpha,\gamma) \vdash_P (p,\alpha,\gamma).
\end{align*}
Рефлексивното и транзитивно затваряне на $\vdash_P$ ще означаваме с $\vdash^\star_P$.
Сега вече можем да дадем дефиниция на език, разпознаван от стеков автомат $P$.
% \begin{itemize}
% \item
Езикът $\L(P)$, който се разпознава от $P$, има следната дефиниция:
\[\L(P) = \{\omega \in \Sigma^\star \mid (\exists q \in F)(\exists \gamma \in \Gamma^\star)[\ (\qstart,\omega,\varepsilon) \vdash^\star_P (q,\varepsilon,\gamma)\ ]\}.\]    
% \item
%   \marginpar{Това означение не е стандартно. По-късно ще видим, че двете понятия съвпадат}
%   $\L_\emptyset(P)$ е езика, който се разпознава от $P$  {\bf с празен стек},
%   \[\L_\emptyset(P) = \{\omega \mid (\exists q \in Q)[\ (\qstart,\omega,\#) \vdash^\star_P (q,\varepsilon,\varepsilon)\ ]\}.\]
% \end{itemize}

\begin{example}
  \label{ex:anbn}
  За езика $L = \{a^nb^n\mid n\in\Nat\}$ съществува стеков автомат $P$, такъв че
  $L = \L(P)$.
  Да разгледаме $P = \PDA$, където
  \begin{itemize}
  \item
    $Q = \{\qstart,q,p,f\}$;
  \item
    $\Sigma = \{a,b\}$;
  \item
    $\Gamma = \{\sharp,a\}$, където символът $\sharp$ служи за дъно на стека, а броят на $a$-тата в стека ще показват колко букви $a$ сме прочели от думата;
  \item
    $F = \{f\}$;
  \item
    Релацията на преходите $\Delta$ има следната дефиниция:
    \begin{enumerate}[(1)]
    \item
      $\Delta(\qstart,\varepsilon,\varepsilon) = \{(q,\sharp)\}$;
    \item
      $\Delta(q,a,\sharp) = \{(q, a\sharp), (p, a\sharp)\}$;
    \item
      $\Delta(q,a,a) = \{(q, aa), (p, aa)\}$;
    \item 
      $\Delta(q,\varepsilon,\sharp) = \{(q,\varepsilon)\}$;
    % \item 
    %   $\Delta(\qstart,\varepsilon, a) = \emptyset$;
    % \item
    %   $\Delta(\qstart, b, a) = \emptyset$;
    % \item
    %   $\Delta(\qstart, b, \sharp) = \emptyset$;
    \item 
      $\Delta(p, b, a) = \{(p,\varepsilon)\}$;
    \item
      $\Delta(p, \varepsilon, \sharp) = \{(f, \varepsilon)\}$.
    % \item 
    %   $\Delta(p, b, \sharp) = \emptyset$;
    % \item 
    %   $\Delta(p, a, \sharp) = \emptyset$;
    % \item 
    %   $\Delta(p, a, A) = \emptyset$;
    % \item
    %   $\Delta(p, \varepsilon, \sharp) = \emptyset$.
    \item
      За всички останали тройки $(r,x,y)$, нека $\Delta(r,x,y) = \emptyset$.
    \end{enumerate}
  \end{itemize}
  
  Да видим как думата $a^2b^2$ се разпознава от автомата с празен стек:
  \begin{align*}
    (\qstart, a^2b^2, \varepsilon) & \vdash_P (q, a^2b^2, \sharp) & \comment{\text{правило }(1)}\\
                                   & \vdash_P (q, ab^2, a\sharp) & \comment{\text{правило }(2)}\\
                                   & \vdash_P (p, b^2, aa\sharp) & \comment{\text{правило }(3)}\\
                                   & \vdash_P (p, b, a\sharp) & \comment{\text{правило }(5)}\\
                                   & \vdash_P (p, \varepsilon, \sharp) & \comment{\text{правило }(5)}\\
                                   & \vdash_P (f, \varepsilon, \varepsilon) & \comment{\text{правило }(6)}
  \end{align*}
  \marginpar{\writedown Докажете, че $L = \L(P)$!}
  Поличихме, че
  \[(\qstart, a^2b^2, \varepsilon) \vdash^\star_P (f, \varepsilon, \varepsilon),\]
  откъдето следва, че $a^2b^2 \in \L(P)$.

  \begin{enumerate}[a)]
  \item
    \marginpar{Индукция по $n$}
    Докажете, че за всяко естествено число $n$,
    \begin{align*}
      & (q, a^nb^n, \sharp) \vdash^\star_P (q, b^n, a^n\sharp)\\
      & (p, b^n, a^n\sharp) \vdash^\star_P (p, \varepsilon,\sharp).
    \end{align*}
    Заключете, че $L \subseteq \L(P)$.
  \item
    \marginpar{Индукция по броя на стъпките в изчислението на стековия автомат}
    Докажете, че за всеки три думи $\alpha,\beta, \gamma \in \{a,b\}^\star$ е изпълнено, че:
    \begin{align*}
      (q, \alpha\beta, \sharp) \vdash^\star_P (q, \beta, \gamma\sharp)  & \implies \alpha = \gamma = a^n, \text{ за някое }n\\
      (p, \beta, \gamma\sharp) \vdash^\star_P (p, \varepsilon, \sharp) & \implies \beta = \gamma = b^k, \text{ за някое }k.
    \end{align*}
    Оттук заключете, че $\L(P) \subseteq L$.    
  \end{enumerate}
  
\end{example}

\begin{example}
  \label{ex:omega-omega-r}
  Езикът $L = \{\omega\omega^{rev} \mid \omega \in \{a,b\}^\star\}$ се разпознава от стеков автомат $P = \PDA$, където:
  \begin{itemize}
  \item 
    $Q = \{\qstart,q,p,\qaccept\}$;
  \item
    $\Sigma = \{a,b\}$;
  \item
    $\Gamma = \{a, b, \sharp\}$;
  \item
    $F = \{\qaccept\}$;
  \item
    Функцията на преходите $\Delta$ има следната дефиниция:
    \marginpar{За всички липсващи редове в дефиницията на $\Delta$ приемаме, че $\Delta$ връща $\emptyset$}
    \begin{enumerate}[(1)]
    \item
      $\Delta(\qstart,\varepsilon,\varepsilon) = \{(q, \sharp)\}$;
    \item 
      $\Delta(q, a, \sharp) = \{(q, a\sharp)\}$;
    \item 
      $\Delta(q, b, \sharp) = \{(q, b\sharp)\}$;
    \item
      $\Delta(q, \varepsilon, \sharp) = \{(q,\varepsilon)\}$;
    \item
      $\Delta(q, a, a) = \{(q, aa), (p, \varepsilon)\}$;
    \item
      $\Delta(q, a, b) = \{(q, ab)\}$;
    \item
      $\Delta(q, b, a) = \{(q, ba)\}$;
    \item
      $\Delta(q, b, b) = \{(q, bb), (p, \varepsilon)\}$;
    \item
      $\Delta(p, a, a) = \{(p,\varepsilon)\}$;
    \item
      $\Delta(p, b, b) = \{(p,\varepsilon)\}$;
    \item
      $\Delta(p, \varepsilon, \sharp) = \{(\qaccept,\varepsilon)\}$;
    \end{enumerate}
  \end{itemize}
  Основното наблюдение, което трябва да направим за да разберем конструкцията на автомата е, че
  всяка дума от вида $\omega\omega^{rev}$ може да се запише като $\omega_1aa\omega^{rev}_1$ или $\omega_1bb\omega^{rev}_1$.
  Да видим защо $P$ разпознава думата $abaaba$ с празен стек.
  Започваме по следния начин:
  \begin{align*}
    (\qstart, abaaba, \varepsilon) & \vdash_P (q, abaaba,\sharp)    & \comment{\text{правило }(1)} \\
                                   & \vdash_P (q, baaba, a\sharp)   & \comment{\text{правило }(2)}\\
                                   & \vdash_P (q, aaba, ba\sharp)   & \comment{\text{правило }(7)}\\
                                   & \vdash_P (q, aba,  aba\sharp). & \comment{\text{правило }(6)}
  \end{align*}
  Сега можем да направим два избора как да продължим. Състоянието $p$ служи за маркер, което ни казва, че вече сме започнали 
  да четем $\omega^{rev}$. Поради тази причина, продължаваме така:
  \begin{align*}
    (q, aba, aba\sharp) & \vdash_P (p, ba, ba\sharp) & \comment{\text{правило }(5)}\\
                    & \vdash_P (p, a, a\sharp) & \comment{\text{правило }(10)}\\
                    & \vdash_P (p, \varepsilon, \sharp) & \comment{\text{правило }(9)}\\
                    & \vdash_P (\qaccept,\varepsilon,\varepsilon). & \comment{\text{правило }(11)}
  \end{align*}
  Да проиграем още един пример. Да видим защо думата $aba$ не се извежда от автомата.
  \begin{align*}
    (\qstart, aba, \varepsilon) & \vdash_P (q,aba,\sharp) & \comment{\text{правило }(1)} \\
                                & \vdash_P (q, ba,a\sharp) & \comment{\text{правило }(2)}\\
                                & \vdash_P (q, a, ba\sharp) & \comment{\text{правило }(7)}\\
                                & \vdash_P (q, \varepsilon, aba\sharp). & \comment{\text{правило }(6)}
  \end{align*}
  От последното моментно описание на автомата нямаме нито един преход, следователно
  думата $aba$ не се разпознава от $P$.
  \marginpar{\writedown Докажете, че $L = \L(P)$!}
  \begin{enumerate}[a)]
  \item
    \marginpar{Индукция по дължината на думата $\alpha$}
    Докажете, че за всеки две думи $\alpha, \beta \in \{a,b\}^\star$ е изпълнено, че:
    \begin{align*}
      & (q, \alpha\beta, \sharp) \vdash^\star_P (q, \beta, \alpha^{rev}\sharp)\\
      & (p, \alpha, \alpha\sharp) \vdash^\star_P (p, \varepsilon, \sharp).
    \end{align*}
    Оттук заключете, че $L \subseteq \L(P)$.
  \item
    \marginpar{Индукция по броя на стъпките в изчислението на стековия автомат}
    Докажете, че за всеки три думи $\alpha,\beta, \gamma \in \{a,b\}^\star$ е изпълнено, че:
    \begin{align*}
      (q, \alpha\beta, \sharp) \vdash^\star_P (q, \beta, \gamma\sharp)  & \implies \gamma = \alpha^{rev}\\
      (p, \beta, \gamma\sharp) \vdash^\star_P (p, \varepsilon, \sharp) & \implies \gamma = \beta.
    \end{align*}
    Оттук заключете, че $\L(P) \subseteq L$.
  \end{enumerate}

\end{example}



\begin{example}
  Езикът 
  \[L = \{\ \omega \in \{\texttt{[},\texttt{]}\}^\star \mid \texttt{left}(\omega) = \texttt{right}(\omega)\ \}\]
  се разпознава от стековия автомат $P = \PDA$, където:
  \begin{itemize}
  \item 
    $Q = \{\qstart,q,\qaccept\}$;
  \item
    $\Sigma = \{\texttt{[},\texttt{]}\}$;
  \item
    $\Gamma = \{\texttt{[}, \texttt{]}, \#\}$;
  \item
    $F = \{\qaccept\}$;
  \item
    Можем да дефинираме релацията на преходите $\Delta$ по следния начин:
    \begin{enumerate}[(1)]
    \item
      $\Delta(\qstart, \varepsilon, \varepsilon) = \{(q,\#)\}$;
    \item 
      $\Delta(q, \varepsilon, \#) = \{(\qaccept, \varepsilon)\}$;
    \item
      $\Delta(q, \texttt{[}, \#) = \{(q, \texttt{[}\#)\}$;
    \item
      $\Delta(q, \texttt{]}, \#) = \{(q, \texttt{]}\#)\}$;
    \item
      $\Delta(q, \texttt{[}, \texttt{[}) = \{(q, \texttt{[[})\}$;
    \item
      $\Delta(q, \texttt{[}, \texttt{]}) = \{(q, \varepsilon)\}$;
    \item
      $\Delta(q, \texttt{]}, \texttt{[}) = \{(q, \varepsilon)\}$;
    \item
      $\Delta(q, \texttt{]}, \texttt{]}) = \{(q, \texttt{]]})\}$.
    \end{enumerate}
  \end{itemize}
  Да видим защо думата $\texttt{[]]][[} \in \L(P)$.
  \begin{align*}
    (\qstart, \texttt{[]]][[}, \varepsilon) & \vdash_P (q, \texttt{[]]][[}, \#) & \comment{\text{правило }(1)}\\
                                            & \vdash_P (q,\ \texttt{]]][[},\ \texttt{[}\#) & \comment{\text{правило }(3)}\\
                                            & \vdash_P (q,\ \texttt{]][[},\ \#) & \comment{\text{правило }(7)}\\
                                            & \vdash_P (q,\ \texttt{][[},\ \texttt{]}\#) & \comment{\text{правило }(4)}\\
                                            & \vdash_P (q,\ \texttt{[[},\ \texttt{]]}\#) & \comment{\text{правило }(8)}\\
                                            & \vdash_P (q,\ \texttt{[},\ \texttt{]}\#) & \comment{\text{правило }(6)}\\
                                            & \vdash_P (q,\ \varepsilon,\ \#) & \comment{\text{правило }(6)}\\
                                            & \vdash_P (\qaccept,\ \varepsilon,\ \varepsilon). & \comment{\text{правило }(2)}
  \end{align*}

  \begin{enumerate}[a)]
  \item
    \marginpar{Индукция по дължината на думата $\alpha$}
    Докажете, че за произволно естествено число $n$ и произволна дума $\alpha \in \{\texttt{[}, \texttt{]}\}^\star$, е изпълнено, че:
    \begin{align*}
      \texttt{[}^n\alpha \in L & \implies (q, \alpha, \texttt{[}^n\sharp) \vdash^\star_P (q, \varepsilon, \sharp)\\
      \texttt{]}^n\alpha \in L & \implies (q, \alpha, \texttt{]}^n\sharp) \vdash^\star_P (q, \varepsilon, \sharp).
    \end{align*}
    Оттук заключете, че $L \subseteq \L(P)$.
  \item
    \marginpar{Индукция по броя на стъпките в изчислението на стековия автомат }
    Докажете, че за произволно естествено число $n$ и произволна дума $\alpha \in \{\texttt{[}, \texttt{]}\}^\star$, е изпълнено, че:
    \begin{align*}
      (q, \alpha, \texttt{[}^n\sharp) \vdash^\star_P (q, \varepsilon, \sharp) & \implies \texttt{[}^n\alpha \in L\\
      (q, \alpha, \texttt{]}^n\sharp) \vdash^\star_P (q, \varepsilon, \sharp) & \implies \texttt{]}^n\alpha \in L.
    \end{align*}
    Оттук заключете, че $\L(P) \subseteq L$.
  \end{enumerate}
\end{example}

\begin{example}
  Езикът
  \[L = \{\ \omega \in \{\texttt{[},\texttt{]}\}^\star \mid \omega\text{ е балансирана дума}\ \}\]
  се разпознава от стековия автомат $P = \PDA$, където:
  \begin{itemize}
  \item 
    $Q = \{\qstart,q,\qaccept\}$;
  \item
    $\Sigma = \{\texttt{[},\texttt{]}\}$;
  \item
    $\Gamma = \{\texttt{[}, \sharp\}$;
  \item
    $F = \{\qaccept\}$;
  \item
    Можем да дефинираме релацията на преходите $\Delta$ по следния начин:
    \marginpar{\writedown Докажете, че $L = \L(P)$!}
    \begin{enumerate}[(1)]
    \item
      $\Delta(\qstart,\varepsilon,\varepsilon) = \{(q,\sharp)\}$;
    \item 
      $\Delta(q, \varepsilon, \sharp) = \{(\qaccept, \varepsilon)\}$;
    \item
      $\Delta(q, \texttt{[}, \sharp) = \{(q, \texttt{[}\sharp)\}$;
    % \item
    %   $\Delta(q, \texttt{]}, \sharp) = \emptyset$;
    \item
      $\Delta(q, \texttt{[}, \texttt{[}) = \{(q, \texttt{[[})\}$;
    % \item
    %   $\Delta(q, \texttt{[}, \texttt{]}) = \emptyset$;
    \item
      $\Delta(q, \texttt{]}, \texttt{[}) = \{(q, \varepsilon)\}$;
    % \item
    %   $\Delta(q, \texttt{]}, \texttt{]}) = \emptyset$.
    \end{enumerate}
  \end{itemize}  
  \begin{enumerate}[(a)]
  \item
    \marginpar{Индукция по дължината на думата $\alpha$}
    Докажете, че за произволно естествено число $n$ и произволна дума $\alpha \in \{\texttt{[}, \texttt{]}\}^\star$, 
    е изпълнено, че:
    \[\texttt{[}^n\alpha \in L\ \implies (q, \alpha, \texttt{[}^n\sharp) \vdash^\star_P (q, \varepsilon, \sharp).\]
    Оттук заключете, че $L \subseteq \L(P)$.
  \item
    \marginpar{Индукция по броя на стъпките в изчислението на стековия автомат.}
    Докажете, че за произволно естествено число $n$ и произволна дума $\alpha \in \{\texttt{[}, \texttt{]}\}^\star$, е изпълнено, че:
    \[(q,\alpha,\texttt{[}^n\sharp) \vdash^\star_P (q, \varepsilon, \sharp)\ \implies\ \texttt{[}^n\alpha \in L.\]
    Оттук заключете, че $\L(P) \subseteq L$.
  \end{enumerate}
\end{example}


% \begin{thm}
%   \marginpar{(\cite{hopcroft1}, стр. 114) }
%   Нека $L$ е произволен език над азбука $\Sigma$.
%   \begin{enumerate}[1)]
%   \item 
%     Ако съществува НСА $P$, за който $L = \L_F(P)$, то съществува НСА $P^\prime$, за който $L = \L_\emptyset(P^\prime)$.
%   \item
%     Ако съществува НСА $P$, за който $L = \L_\emptyset(P)$, то съществува НСА $P^\prime$, за който $L = \L_F(P^\prime)$.
%   \end{enumerate}
%   С други думи, езиците разпознавани от НСА с празен стек са точно езиците разпознавани от НСА с финално състояние.
% \end{thm}
% \begin{proof}
%   \begin{enumerate}[1)]
%   \item 
%     Нека $L = \L_F(P)$, където $P = \PDA$.
%     Ще построим $P^\prime$, така че да симулира $P$ и като отидем във финално състояние ще изпразним стека.
%     Нека
%     \[P^\prime = \langle{Q\cup\{q_e,\qstart'\},\Sigma,\Gamma \cup \{\$\},\$,\qstart',\Delta^\prime,\emptyset}\rangle,\]
%     където $\$ \not\in \Gamma$.
%     Важно е $P^\prime$ да има собствен нов символ за дъно на стека, защото е възможно за някоя дума $\alpha \not\in \L_F(P)$
%     стековият автомат $P$ да си изчисти стека и така да разпознаем повече думи.
%     \begin{itemize}
%     \item 
%       \marginpar{- започваме симулацията}
%       $\Delta'(\qstart',\varepsilon,\$) = \{(\qstart,\#\$)\}$;
%     \item
%       \marginpar{- симулираме $P$}
%       $\Delta'(q,a,X)$ включва множеството $\Delta(q,a,X)$, за всяко $q\in Q$, $a\in\Sigma_\varepsilon$, $X\in\Gamma$;
%     \item
%       \marginpar{- ако сме във финално, започваме да чистим стека}
%       $\Delta'(q,\varepsilon,X)$ съдържа също и елемента $(q_e,\varepsilon)$, за всяко $q\in F$, $X \in \Gamma \cup \{\$\}$;
%     \item
%       \marginpar{- изчистваме стека}
%       $\Delta'(q_e,\varepsilon,X) = \{(q_e,\varepsilon)\}$, за всяко $X \in \Gamma \cup \{\$\}$;
%     \item
%       $\Delta'$ няма други правила.
%     \end{itemize}
%   \item
%     Сега имаме $L = \L_S(P)$, където $P = \langle{Q,\Sigma,\Gamma,\#,\qstart,\Delta,\emptyset}\rangle$. 
%     Да положим
%     \[P^\prime = \langle{Q\cup\{\qstart',q_f\}, \Sigma, \Gamma \cup \{\$\}, \$, \qstart', \Delta^\prime, \{q_f\}}\rangle.\]
%     $P^\prime$ ще симулира $P$ като ще внимаваме кога $P$ изчиства символа $\#$. Тогава ще искаме да отидем във финалното състояние $q_f$.
%     \begin{itemize}
%     \item 
%       \marginpar{- започваме симулацията}
%       $\Delta'(\qstart',\varepsilon,\$) = \{(\qstart, \#\$)\}$;
%     \item
%       \marginpar{- симулираме $P$}
%       $\Delta'(q,a,X) = \Delta(q,a,X)$, за всяко $q \in Q$, $a \in \Sigma_\varepsilon$, $X \in \Gamma$;
%     \item
%       \marginpar{- щом сме стигнали до $\$$, значи $P$ е изчистил стека си}
%       $\Delta'(q,\varepsilon,\$) = \{(q_f,\varepsilon)\}$.
%     \end{itemize}
%   \end{enumerate}
% \end{proof}

% \begin{problem}
%   Като използвате стековия автомат от Пример \ref{ex:anbn}, дефинирайте автомат $P'$, за който
%   $\L_F(P') = \{a^nb^n \mid n\in\Nat\}$.
% \end{problem}



\begin{framed}
  \begin{lemma}
    За всяка безконтекстна граматика $G$,
    съществува стеков автомат $P$, такъв че $\L(G) = \L(P)$.
  \end{lemma}
\end{framed}
\begin{proof}
  \marginpar{\cite[стр. 136]{papadimitriou}}
  \marginpar{Доказателството в \cite[стр. 117]{sipser3} не ми харесва}
  Нека е дадена безконтекстната граматика $G = \CFG$.
  Нашата цел е да построим стеков автомат $P = \PDA$, който разпознава $\L(G)$.
  % Нека  \[P = \langle{\{\qstart,q,f\},\Sigma,\Sigma\cup V \cup\{\sharp\},q,\Delta,\{f\}}\rangle,\]
  % където релацията на преходите е:
  \begin{itemize}
  \item
    $Q = \{\qstart,q,\qaccept\}$;
  \item
    $\Gamma = \Sigma \cup V \cup \{\sharp\}$;
  \item
    $F = \{\qaccept\}$;
  \item
    Релацията на преходите $\Delta$ дефинираме по следния начин:
    \begin{enumerate}[(1)]
    \item 
      $\Delta(\qstart, \varepsilon, \varepsilon ) = \{(q,S\sharp)\}$;
    \item
      $\Delta(q,\varepsilon,A) = \{(q,\alpha)\mid A\to_G \alpha\}, \text{ за всяка променлива }A \in V$;
    \item
      $\Delta(q,a,a) = \{(q,\varepsilon)\}, \text{ за всяка буква } a \in \Sigma$;
    \item
      $\Delta(q,\varepsilon,\sharp) = \{(\qaccept, \varepsilon)\}$.
    \end{enumerate}
  \end{itemize}

  Ще докажем, че за всяка променлива $A \in V$, за всяка дума $\alpha \in \Sigma^\star$ и $\gamma \in (\Sigma \cup V)^\star$, то е изпълнено, че:
  \begin{enumerate}[(a)]
  \item
    ако $S \to^\star_G \alpha \gamma$, то $(q, \alpha, S\sharp) \vdash^\star_P (q, \varepsilon, \gamma\sharp)$;
  \item
    ако $(q, \alpha, A\gamma\sharp) \vdash^\star_P (q, \varepsilon, \sharp)$, то $A\gamma \to^\star_G \alpha$.
  \end{enumerate}
  Тогава, ако вземем $\gamma = \varepsilon$ и $A = S$, то ще получим, че
  \begin{align*}
    \alpha \in \L(G) & \iff S \to^\star_G \alpha\\
                     & \iff (q,\alpha,S\sharp) \vdash^\star_P (q, \varepsilon, \sharp) & \comment{\text{от (а) и (б)}}\\
                     & \iff (\qstart,\alpha,\varepsilon) \vdash^\star_P (\qaccept, \varepsilon, \varepsilon) & \comment{\text{от деф. на }\Delta}\\
                     & \iff \alpha \in \L(P).
  \end{align*}

  Сега преминаваме към доказателствата на двете твърдения.

  \begin{enumerate}[(a)]
  \item
    Индукция по дължината на извода $S \to^\star_G \alpha\gamma$.
    Базовият случай, когато дължината на извода е $0$, е тривиален, защото тогава $\alpha = \varepsilon$ и $\gamma = S$.
    Ясно е, че $(q,\varepsilon,S\sharp) \vdash^0_P (q,\varepsilon,S\sharp)$.

    Нека $S \to^{l+1}_G \alpha\gamma$. Това означава, че този извод може да се запише по следния начин:
    \[S \to^l_G \alpha_1A\gamma_2 \to_G \underbrace{\alpha_1\alpha_2}_{\alpha}\underbrace{\gamma_1\gamma_2}_{\gamma},\]
    където $A \to_G \alpha_2\gamma_1$ е правилото в граматиката, което сме приложили най-накрая. Тогава от И.П. имаме, че
    \begin{equation}
      \label{eq:5}
      (q, \alpha_1, S\sharp) \vdash^\star_P (q, \varepsilon, A\gamma_2\sharp).
    \end{equation}
    Тогава имаме следното изчисление на стековия автомат:
    \begin{align*}
      (q, \alpha_1\alpha_2, S\sharp) & \vdash^\star_P (q, \alpha_2, A\gamma_2\sharp) & \comment{\text{от (\ref{eq:5})}}\\
                                     & \vdash_P (q, \alpha_2, \alpha_2\gamma_1\gamma_2\sharp) & \comment{\text{ред (2) от деф. на }\Delta}\\
                                     & \vdash^\star_P (q, \varepsilon, \gamma_1\gamma_2\sharp) & \comment{\text{ред (3) от деф. на }\Delta}.
    \end{align*}
  \item
    Индукция по броя на стъпките в изчислението на стековия автомат.
    Нека $(q,\alpha, A\gamma\sharp) \vdash^1_P (q,\varepsilon,\sharp)$.
    Тогава е ясно, че $\alpha = \gamma = \varepsilon$ и $A \to_G \varepsilon$ е правило в граматиката.

    Нека $(q, \alpha, A\gamma \sharp) \vdash^{l+1}_P (q, \varepsilon, \sharp)$.
    Нека $A \to_G \alpha_1B\gamma_2$ е правило в граматиката.
    Според дефиницията на $\Delta$, ясно е, че тогава $\alpha = \alpha_1 \alpha_2$, за някое $\alpha_2 \in \Sigma^\star$.
    Тогава
    \begin{align*}
      (q, \alpha, A\gamma \sharp) & \vdash_P (q, \alpha_1\alpha_2, \alpha_1B\gamma_2\gamma\sharp) & \comment{\text{ред (2) от деф. на  }\Delta}\\
                                  & \vdash^\star_P (q, \alpha_2, B\gamma_2\gamma\sharp) & \comment{\text{ред (3) от деф. на }\Delta}\\
                                  & \vdash^\star_P (q, \varepsilon, \sharp) & \comment{\text{брой стъпки } \leq l}.
    \end{align*}
    Тогава от И.П. следва, че $B\gamma_2\gamma \to^\star_G \alpha_2$.
    Оттук получаваме, че
    \[A\gamma \to_G \alpha_1 B\gamma_2\gamma \to^\star_G \alpha_1 \alpha_2.\]
  \end{enumerate}
\end{proof}

\begin{prop}
  \marginpar{\writedown Докажете!}
  За всеки стеков автомат $P$,
  съществува стеков автомат $\hat{P}$, такъв че $\L(P) = \L(\hat{P})$,
  където
  \begin{itemize}
  \item
    $\hat{Q} = Q \cup \{\hat{q}_{\texttt{start}}, \qaccept\}$;
  \item
    $\hat{V} = V \cup \{\sharp\}$, $\sharp \not\in V$;
  \item
    $\hat{\Delta}(\hat{q}_{\texttt{start}},\varepsilon,\varepsilon) = \{(\qstart, \sharp)\}$;
  \item
    $\hat{F} = \{\qaccept\}$;
  \item
    всяко успешно изчисление на $\hat{P}$ завършва с празен стек.
  \end{itemize}
\end{prop}


\begin{framed}
  \begin{lemma}
    За всеки стеков автомат $P$, съществува безконтекстна граматика $G$, такава че $\L(P) = \L(G)$.
  \end{lemma}
\end{framed}
\begin{proof}
  Можем да приемем, че работим със стеков $P$ притежаващ свойствата на стековия автомат $\hat{P}$ от предишното твърдение.
  Ще дефинираме безконтекстна граматика $G$, за която $\L(P) = \L(G)$.
  Променливите на граматика са 
  \[V = \{[q,A,p] \mid q,p \in \hat{Q}, A \in \hat{\Gamma}\}.\]
  Правилата на $G$ са следните:
  \begin{itemize}
  \item
    Началната променлива е $[\qstart,\sharp,\qaccept]$;
  \item
    Нека имаме $(q_1,B_1\dots B_m) \in \Delta(q, a, A)$.
    Тогава добавяме правилата в граматиката:
    \[[q,A,p] \to_G a[q_1,B_1,q_2][q_2,B_2,q_3]\dots [q_m,B_m,p],\]
    за всеки $q,q_1,\dots,q_{m},p \in Q$, $a \in \Sigma \cup \{\varepsilon\}$.
  \item
    Нека имаме $(p,\varepsilon) \in \Delta(q,a,A)$.
    Тогава добавяме правилата в граматиката:
    \[[q,A,p] \to a,\]
    където $a \in \Sigma \cup \{\varepsilon\}$.
  \end{itemize}
  Трябва да докажем, че за произволна дума $\alpha \in \Sigma^\star$, произволни състояния $q,p \in Q$,
  и произволна буква $A \in \Gamma$, е изпълнено, че:
  \[[q,A,p] \rightarrow^\star_G \alpha\ \Leftrightarrow\ (q,\alpha,A) \vdash^\star_{P} (p,\varepsilon,\varepsilon).\]
  \begin{description}
  \item[$(\Rightarrow)$]
    С пълна индукция по броят на стъпките $l$ в изчислението на стековия автомата $P$ ще докажем, че:
    \[(q,\alpha,A) \vdash^\star_P (p,\varepsilon,\varepsilon)\ \implies\ [q,A,p] \to^\star_G \alpha.\]
    Ако $l = 1$, то е лесно, защото $\alpha \in \Sigma \cup\{\varepsilon\}$ и $m = 0$.
    Според конструкцията на граматиката $G$ имаме правилото $[q,A,p] \to a$.
    
    Ако $l > 1$, нека $\alpha = a\beta$, където е възможно $a = \varepsilon$. Тогава:
    \marginpar{Възможно е $a = \varepsilon$}
    \[(q,a\beta,A) \vdash_P (q_1,\beta,B_1\dots B_n) \vdash^{l-1}_P (p, \varepsilon, \varepsilon).\]
    Да разбием думата $\beta$ на $n$ части, $\beta = \beta_1\cdots \beta_n$, със свойството, че след като прочетем $\beta_i$ 
    сме премахнали променливата $B_i$ от върха на стека.
    Понеже най-накрая искаме стекът да е празен то със сигурност ще има такъва стъпка от изчислението.
    Това означава, че съществуват състояния $q_2,\dots,q_{n}$:
    \begin{align*}
      & (q_1, \beta_1, B_1) \vdash^{l_1}_P (q_{2},\varepsilon,\varepsilon)\\
      & (q_2, \beta_2, B_2) \vdash^{l_2}_P (q_{3},\varepsilon,\varepsilon)\\
      & \ \vdots\\
      & (q_n, \beta_n, B_n) \vdash^{l_n}_P (p,\varepsilon,\varepsilon),
    \end{align*}
    където $l_1+l_2+\cdots+l_n = l-1$.
    Сега по {\bf И.П.} получаваме:
    \begin{align*}
      & (q_1, \beta_1, B_1) \vdash^{l_1}_P (q_{2},\varepsilon,\varepsilon) \implies [q_1,B_1, q_{2}] \to^\star_G \beta_1\\
      & (q_2, \beta_2, B_2) \vdash^{l_2}_P (q_{3},\varepsilon,\varepsilon) \implies [q_2,B_2, q_{3}] \to^\star_G \beta_2\\
      & \ \vdots\\
      & (q_n, \beta_n, B_n) \vdash^{l_n}_P (p,\varepsilon,\varepsilon) \implies [q_n,B_n, p] \to^\star_G \beta_n.
    \end{align*}
    Понеже имаме правилото $\Delta(q,a,A) \ni (q_1,B_1\cdots B_n)$, то в граматиката имаме правилото
    \[[q,A,p] \rightarrow_G a[q_1,B_1,q_2]\dots[q_n,B_n,p].\]
    Обединявайки всичко, получаваме извода
    \[[q,A,p] \rightarrow^\star_G a\beta.\]
  \item[$(\Leftarrow)$]
    Този път с пълна индукция по дължината на извода $l$ ще докажем, че
    \[[q,A,p] \rightarrow^\star_G \alpha \implies (q,\alpha,A) \vdash^\star_P (p,\varepsilon,\varepsilon).\]
    Ако $l = 1$, то имаме $[q,A,p] \rightarrow \alpha$, където $\alpha = a$ или $\alpha = \varepsilon$.
    Ако $l > 1$, то имаме, че $\alpha = a\beta$ и за някое $n$, 
    \[[q,A,p] \rightarrow_G a[q_1,B_1,q_2][q_2,B_2,q_3]\dots[q_n,B_n,p] \rightarrow^{l-1}_G \beta.\]
    Отново нека $\beta = \beta_1\dots \beta_n$, където 
    \begin{align*}
      & [q_1,B_1,q_{2}] \to^{l_1}_G \beta_1\\
      & [q_2,B_2,q_{3}] \to^{l_2}_G \beta_2\\
      & \ \vdots\\
      & [q_{n},B_n, p] \rightarrow^{l_n}_G \beta_n,
    \end{align*}
    където $l_1 + l_2 + \cdots + l_n = l-1$.
    От {\bf И.П.} получаваме, че 
    \begin{align*}
      & [q_1,B_1,q_{2}] \to^{l_1}_G \beta_1 \implies (q_1,\beta_1,B_1) \vdash^\star_P (q_{2},\varepsilon,\varepsilon) \\
      & [q_2,B_2,q_{3}] \to^{l_2}_G \beta_2 \implies (q_2,\beta_2,B_2) \vdash^\star_P (q_{3},\varepsilon,\varepsilon)\\
      & \ \vdots\\
      & [q_{n},B_n, p] \rightarrow^{l_n}_G \beta_n \implies  (q_{n},\beta_n,B_n) \vdash^\star_P (p,\varepsilon,\varepsilon).
    \end{align*}
    Правилото $[q,A,p] \rightarrow_G a[q_1,B_1,q_2][q_2,B_2,q_3]\dots[q_n,B_n,p]$ е добавено в граматиката, 
    защото $\Delta(q,a,A) \ni (q_1, B_1B_2\cdots B_n)$. 
    Обединявайки всичко, което знаем, получаваме:
    \begin{align*}
      (q, a\beta, A) & \vdash_P (q_1, \beta_1\beta_2\cdots\beta_n, B_1B_2\cdots B_n)\\
                     & \vdash^\star_P (q_2, \beta_{2}\cdots\beta_n, B_2\cdots B_n)\\
                     & \dots\\
                     & \vdash^\star_P (q_n, \beta_n, B_n)\\
                     & \vdash^\star_P (p, \varepsilon, \varepsilon)
    \end{align*}
  \end{description}
\end{proof}
  
  
\begin{framed}
\begin{thm}
  \label{th:push-down-context-free}
  Класът на езиците, които се разпознават от краен стеков автомат съвпада с
  класа на безконтекстните езици.
\end{thm}
\end{framed}
% \begin{proof}
%   \marginpar{\cite[стр. 117]{hopcroft1}}
%   Ще разгледаме двете посоки на твърдението поотделно.
%   \begin{enumerate}[1)]
%   \item 
%     Нека е дадена безконтекстна граматика $G = \CFG$.
%     Нашата цел е да построим стеков автомат $P$, така че $\L_S(P) = \L(G)$.
%     Нека  \[P = \langle{\{q\},\Sigma,\Sigma\cup V,S,q,\Delta,\emptyset}\rangle,\]
%     където функцията на преходите е:
%     \begin{align*}
%       & \Delta(q,\varepsilon,A) = \{(q,\alpha)\mid A\to\alpha\mbox{ е правило в граматиката }G\}\\
%       & \Delta(q,a,a) = \{(q,\varepsilon)\}
%     \end{align*}
%   \item
%     Нека имаме $P = \langle{Q, \Sigma, \Gamma, \Delta, s, \#, \emptyset}\rangle$.
%     Ще дефинираме безконтекстна граматика $G$, за която $\L_S(P) = \L(G)$.
%     Променливите на граматика са 
%     \[V = \{[q,A,p] \mid q,p \in Q, A \in \Gamma\}.\]
%     Правилата на $G$ са следните:
%     \begin{itemize}
%     \item
%       $S \to [s,\#,q]$, за всяко $q \in Q$;
%     \item
%       $[q,A,p] \to a[q_1,B_1,q_2][q_2,B_2,q_3]\dots [q_m,B_m,p]$,
%       където 
%       \[(q_1,B_1\dots B_m) \in \Delta(q, a, A)\]
%       и произволни $q,q_1,\dots,q_{m},p \in Q$,
%       $a \in \Sigma \cup \{\varepsilon\}$.

%       Да обърнем внимание, че е възможно $m = 0$.
%       Това означава, че $(p,\varepsilon) \in \Delta(q, a, A)$ и тогава имаме правилото $[q,A,p] \to a$, където $a \in \Sigma \cup \{\varepsilon\}$.
%     \end{itemize}
%     Трябва да докажем, че:
%     \[[q,A,p] \rightarrow^\star_G \alpha\ \Leftrightarrow\ (q,\alpha,A) \vdash^\star_P (p,\varepsilon,\varepsilon).\]
%     \begin{description}
%     \item[$(\Rightarrow)$]
%       С пълна индукция по $i$, ще докажем, че 
%       \[(q,\alpha,A) \vdash^i_P (p,\varepsilon,\varepsilon)\ \implies\ [q,A,p] \to^\star_G \alpha.\]
%       Ако $i = 1$, то е лесно, защото $\alpha \in \Sigma \cup\{\varepsilon\}$ и $m = 0$.
%       Според конструкцията на граматиката $G$ имаме правилото $[q,A,p] \to a$.

%       Ако $i > 1$, нека $\alpha = a\beta$. Тогава:
%       \marginpar{Възможно е $a = \varepsilon$}
%       \[(q,a\beta,A) \vdash_P (q_1,\beta,B_1\dots B_n) \vdash^{i-1}_P (p, \varepsilon, \varepsilon).\]
%       Да разбием думата $\beta$ на $n$ части, $\beta = \beta_1\cdots \beta_n$, със свойството, че след като прочетем $\beta_i$ 
%       сме премахнали променливата $B_i$ от върха на стека. Това означава, че съществуват състояния $q_2,\dots,q_{n}$:
%       \begin{align*}
%         & (q_1, \beta_1, B_1) \vdash^{l_1}_P (q_{2},\varepsilon,\varepsilon)\\
%         & (q_2, \beta_2, B_2) \vdash^{l_2}_P (q_{3},\varepsilon,\varepsilon)\\
%         & \ \vdots\\
%         & (q_n, \beta_n, B_n) \vdash^{l_n}_P (p,\varepsilon,\varepsilon),
%       \end{align*}
%       където $l_1+l_2+\cdots+l_n = i-1$.
%       Сега по {\bf И.П.} получаваме:
%       \begin{align*}
%         & (q_1, \beta_1, B_1) \vdash^{l_1}_P (q_{2},\varepsilon,\varepsilon) \implies [q_1,B_1, q_{2}] \to^\star_G \beta_1\\
%         & (q_2, \beta_2, B_2) \vdash^{l_2}_P (q_{3},\varepsilon,\varepsilon) \implies [q_2,B_2, q_{3}] \to^\star_G \beta_2\\
%         & \ \vdots\\
%         & (q_n, \beta_n, B_n) \vdash^{l_n}_P (p,\varepsilon,\varepsilon) \implies [q_n,B_n, p] \to^\star_G \beta_n.
%       \end{align*}
%       Понеже имаме правилото $\Delta(q,a,A) \ni (q_1,B_1\cdots B_n)$, то в граматиката имаме правилото
%       \[[q,A,p] \rightarrow_G a[q_1,B_1,q_2]\dots[q_n,B_n,p].\]
%       Обединявайки всичко, получаваме извода
%       \[[q,A,p] \rightarrow^\star_G a\beta.\]
%     \item[$(\Leftarrow)$]
%       Отново с пълна индукция по $i$ ще докажем, че
%       \[[q,A,p] \rightarrow^i_G \alpha \implies (q,\alpha,A) \vdash^\star_P (p,\varepsilon,\varepsilon).\]
%       Ако $i = 1$, то имаме $[q,A,p] \rightarrow \alpha$, където $\alpha = a$ или $\alpha = \varepsilon$.
%       Ако $i > 1$, то имаме, че $\alpha = a\beta$ и за някое $n$, 
%       \[[q,A,p] \rightarrow_G a[q_1,B_1,q_2][q_2,B_2,q_3]\dots[q_n,B_n,p] \rightarrow^{i-1}_G \beta.\]
%       Отново нека $\beta = \beta_1\dots \beta_n$, където 
%       \begin{align*}
%         & [q_1,B_1,q_{2}] \to^{i_1}_G \beta_1\\
%         & [q_2,B_2,q_{3}] \to^{i_2}_G \beta_2\\
%         & \ \vdots\\
%         & [q_{n},B_n, p] \rightarrow^{i_n}_G \beta_n,
%       \end{align*}
%       където $i_1 + i_2 + \cdots + i_n = i-1$.
%       От {\bf И.П.} получаваме, че 
%       \begin{align*}
%         & [q_1,B_1,q_{2}] \to^{i_1}_G \beta_1 \implies (q_1,\beta_1,B_1) \vdash^\star_P (q_{2},\varepsilon,\varepsilon) \\
%         & [q_2,B_2,q_{3}] \to^{i_2}_G \beta_2 \implies (q_2,\beta_2,B_2) \vdash^\star_P (q_{3},\varepsilon,\varepsilon)\\
%         & \ \vdots\\
%         & [q_{n},B_n, p] \rightarrow^{i_n}_G \beta_n \implies  (q_{n},\beta_n,B_n) \vdash^\star_P (p,\varepsilon,\varepsilon).
%       \end{align*}
%       Правилото $[q,A,p] \rightarrow_G a[q_1,B_1,q_2][q_2,B_2,q_3]\dots[q_n,B_n,p]$ е добавено в граматиката, 
%       защото $\Delta(q,a,A) \ni (q_1, B_1B_2\cdots B_n)$. 
%       Обединявайки всичко, което знаем, получаваме:
%       \begin{align*}
%         (q, a\beta, A) & \vdash_P (q_1, \beta_1\cdots\beta_n, B_1\cdots B_n)\\
%         & \vdash^\star_P (q_2, \beta_{2}\cdots\beta_n, B_2\cdots B_n)\\
%         & \dots\\
%         & \vdash^\star_P (q_n, \beta_n, B_n)\\
%         & \vdash^\star_P (p, \varepsilon, \varepsilon)
%       \end{align*}
%     \end{description}
%   \end{enumerate}
% \end{proof}


\begin{example}
  Нека е дадена граматиката $G$ с правила 
  \begin{align*}
    & S \to ASB\ |\ \varepsilon\\
    & A \to aAa\ |\ a\\
    & B \to bBb\ |\ b.
  \end{align*}
  Ще построим стеков автомат $P = \PDA$, такъв че $\L(P) = \L(G)$.
  \begin{itemize}
  \item
    $\Sigma = \{a,b\}$;
  \item 
    $\Gamma = \{A,S,B,a,b\}$;
  % \item
  %   $\# = S$;
  \item
    $Q = \{\qstart,q,\qaccept\}$;
  \item
    $F = \{\qaccept\}$;
  \item
    Дефинираме релацията на преходите, следвайки конструкцията от \Th{push-down-context-free}:
    \begin{itemize}
    \item
      $\Delta(\qstart,\varepsilon,\varepsilon) = \{(q,S\sharp)\}$;
    \item 
      $\Delta(q,\varepsilon, S) = \{(q,ASB), (q,\varepsilon)\}$;
    \item
      $\Delta(q, \varepsilon, A) = \{(q, aAa), (q, a)\}$;
    \item
      $\Delta(q, \varepsilon, B) = \{(q, bBb), (q, b)\}$;
    \item
      $\Delta(q, a, a) = \{(q,\varepsilon)\}$;
    \item
      $\Delta(q, b, b) = \{(q,\varepsilon)\}$;
    \item
      $\Delta(q, \varepsilon, \sharp) = \{ (\qaccept,\varepsilon)\}$;
    \end{itemize}
  \end{itemize}
\end{example}


\begin{example}
  Да видим как по стековия автомат за езика 
  \[L = \{\omega\omega^{rev} \mid \omega \in \{a,b\}^\star\}\]
  от \Ex{omega-omega-r} можем да построим граматика, следвайки конструкцията от \Th{push-down-context-free}.

  \begin{itemize}
  \item
    Променливите на граматиката са $2 \cdot 3 \cdot 2 + 1 = 13$.
  \item
    Началната променлива е $[q,\sharp,\qaccept]$.
  \item
    Понеже $\Delta(q, a, \sharp) = \{(q, a\sharp)\}$, то имаме правилата
    \begin{align*}
      & [q, \sharp, q] \to a[q,a,q][q,\sharp,q]\ |\ a[q,a,p][p,\sharp,q]\\
      & [q, \sharp, p] \to a[q,a,q][q,\sharp,p]\ |\ a[q,a,p][p,\sharp,p].
    \end{align*}
  \item
    Понеже $\Delta(q, a, a) = \{(q, aa), (p, \varepsilon)\}$, то добавяме правилата:
    \begin{align*}
      & [q,a,q] \to a[q, a, q][q,a,q]\ |\ a[q,a,p][p,a,q]\\
      & [q,a,p] \to a\ |\ a[q, a, q][q,a,p]\ |\ a[q,a,p][p,a,p].
    \end{align*}
  \item
    \marginpar{\ding{45} Довършете граматиката и след това я опростете!}
    Понеже $\Delta(q, a, b) = \{(q, ab)\}$, то добавяме правилата:
    \begin{align*}
      & [q,b,q] \to a[q,a,q][q,b,q]\ |\ a[q,a,p][p,b,q]\\
      & [q,b,p] \to a[q,a,q][q,b,p]\ |\ a[q,a,p][p,b,p].
    \end{align*}
  \end{itemize}
\end{example}

\begin{framed}
  \begin{thm}
    \label{th:intersection-context-reg}
    Нека $L$ e безконтекстен език и $R$ е регулярен език.
    Тогава тяхното сечение $L \cap R$ е безконтекстен език.
  \end{thm}  
\end{framed}
\begin{proof}
  \marginpar{\cite[стр. 144]{papadimitriou}}
  Нека имаме стеков автомат
  \[P = \pair{Q',\Sigma,\Gamma,q'_{\texttt{start}}, \Delta', F'}, \text{ където } \L(P) = L,\]
  и краен автомат 
  \[\A = \pair{Q'', \Sigma, q''_{\texttt{start}}, \Delta'', F''}, \text{ където } \L(\A) = R.\]
  Ще определим нов стеков автомат $\M = \PDA$, където
  \begin{itemize}
  \item 
    $Q = Q' \times Q''$;
  \item
    $\qstart = \pair{q'_{\texttt{start}},q''_{\texttt{start}}}$;
  \item
    $F = F' \times F''$;
  \item 
    Функцията на преходите $\Delta$ е дефинирана както следва:
    \begin{itemize}
    \item 
      \marginpar{симулираме едновременно изчислението и на двата автомата}
      Ако $(r_1,Z) \in \Delta'(q_1, a, Y)$ и $r_2 \in \Delta''(q_2,a)$, то
      \[(\pair{r_1,r_2}, Z) \in \Delta(\pair{q_1,q_2},a,Y).\]
    \item
      \marginpar{празен ход на автомата $M_2$}
      Ако $(r_1,Z) \in \Delta'(q_1,\varepsilon,Y)$ и всяко $q_2 \in Q''$, то
      \[(\pair{r_1,q_2}, Z) \in \Delta(\pair{q_1,q_2},\varepsilon,Y).\]
    \item
      \marginpar{\writedown Докажете, че $\L(\M) = \L(\M_1) \cap \L(\M_2)$ !}
      $\Delta$ не съдържа други преходи;
    \end{itemize}
  \end{itemize}
\end{proof}

\Th{intersection-context-reg} е удобна, когато искаме да докажем, че даден език не е безконтекстен.
С нейна помощ можем да сведем езика до друг, за който вече знаем, че не е безконтекстен.

\begin{example}
  Езикът $L = \{\omega \in \{a,b,c\}^\star \mid N_a(\omega) = N_b(\omega) = N_c(\omega)\}$ не е безконтекстен.
  Да допуснем, че $L$ е безконтекстен език.
  Тогава \[L^\prime = L \cap \L(a^\star b^\star c^\star)\] също е безконтекстен език.
  Но $L^\prime = \{a^nb^nc^n \mid n \in \Nat\}$, за който знаем от \Prob{anbncn}, че {\em не} е безконтекстен.
  Достигнахме до противоречие. Следователно, $L$ не е безконтекстен език.
\end{example}

%%% Local Variables: 
%%% mode: latex
%%% TeX-master: "../eai"
%%% End: 
