\section{Недетерминирани стекови автомати}

\index{автомат!недетерминиран стеков}
\marginpar{На англ. {\em Push-down automaton}}% (стр. 157 от \cite{kozen})}
%Sipser p.102
{\bf Недетерминиран стеков автомат} е седморка от вида
\[P = \PDA,\] където:
\begin{itemize}
\item
  $Q$ е крайно множество от състояния;
\item  
  $\Sigma$ е крайна входна азбука;
\item
  $\Gamma$ е крайна стекова азбука;
\item
  $\# \in \Gamma$ е символ за дъно на стека;
\item
  $\qstart \in Q$ е начално състояние;
\item
  \marginpar{Озн. $\Ps_{fin}(A)$ - крайните подмножества на $A$}
  \marginpar{Дефиницията в \cite[стр. 131]{papadimitriou} е малко по-различна}
  $\Delta:Q\times(\Sigma \cup \{\varepsilon\})\times\Gamma\rightarrow \Ps_{fin}(Q\times\Gamma^\star)$ 
  е функция на преходите;    
\item
  $F\subseteq Q$ е множество от заключителни състояния.
\end{itemize}

\marginpar{Instanteneous description}
{\em Моментно описание} (или конфигурация) на изчислението със стеков автомат представлява тройка от вида $(q,\alpha,\gamma) \in Q\times\Sigma^\star\times\Gamma^\star$,
т.е. автоматът се намира в състояние $q$, думата, която остава да се прочете е $\alpha$,
а съдържанието на стека е думата $\gamma$.
Удобно е да въведем бинарната релация $\vdash_P$ над $Q\times\Sigma^\star\times\Gamma^\star$,
която ще ни казва как моментното описание на автомата $P$ се променя след изпълнение на една стъпка:
\[(q,x\alpha,Y\gamma) \vdash_P (p,\alpha,\beta\gamma), \text{ ако } \Delta(q,x,Y) \ni (p,\beta),\]
\[(q,\alpha,Y\gamma) \vdash_P (p,\alpha,\beta\gamma), \text{ ако } \Delta(q,\varepsilon,Y) \ni (p,\beta).\]
Рефлексивното и транзитивно затваряне на $\vdash_P$ ще означаваме с $\vdash^\star_P$.
Сега вече можем да дадем дефиниция на език, разпознаван от стеков автомат $P$.
\begin{itemize}
\item
  $\L_F(P)$ е езика, който се разпознава от $P$ {\bf с финално състояние},
  \[\L_F(P) = \{\omega \mid (\exists q \in F)(\exists \gamma \in \Gamma^\star)[\ (\qstart,\omega,\#) \vdash^\star_P (q,\varepsilon,\gamma)\ ]\}.\]    
\item
  \marginpar{Това означение не е стандартно. По-късно ще видим, че двете понятия съвпадат}
  $\L_\emptyset(P)$ е езика, който се разпознава от $P$  {\bf с празен стек},
  \[\L_\emptyset(P) = \{\omega \mid (\exists q \in Q)[\ (\qstart,\omega,\#) \vdash^\star_P (q,\varepsilon,\varepsilon)\ ]\}.\]
\end{itemize}

\begin{example}
  \label{ex:anbn}
  За езика $L = \{a^nb^n\mid n\in\Nat\}$ съществува стеков автомат $P$, такъв че
  $L = \L_\emptyset(P)$.
  Да разгледаме $P = \PDA$, където
  \begin{itemize}
  \item
    $Q = \{\qstart,p\}$;
  \item
    $\Sigma = \{a,b\}$;
  \item
    $\Gamma = \{\#,A\}$, където символът $\#$ служи за дъно на стека, а броят на $A$-тата в стека ще показват колко букви $a$ сме прочели от думата;
  \item
    $F = \emptyset$, защото разпознаваме с празен стек, а не с финално състояние;
  \item
    Функцията на преходите $\Delta$ има следната дефиниция:

    % \begin{align*}
    %   & \Delta(\qstart,a,\#) = \{(\qstart, A\#), (p, A\#)\} & \comment{\text{записваме още едно прочетено $a$}}\\
    %   & \Delta(\qstart,a,A) = \{(\qstart, AA), (p, AA)\} & \comment{\text{като допускаме и това да е последното $a$}}\\
    %   & \Delta(\qstart,\varepsilon,\#) = \{(\qstart,\varepsilon)\} & \comment{\text{трябва да приемем и празната дума}}\\
    %   & \Delta(\qstart,\varepsilon, A) = \emptyset & \\
    %   & \Delta(\qstart, b, A) = \emptyset & \comment{\text{не позволяваме разбъкване на $a$-та и $b$-та}}\\
    %   & \Delta(\qstart, b, \#) = \emptyset & \comment{\text{не можем да започнем с четене на $b$-та}}\\
    %   & \Delta(p, b, A) = \{(p,\varepsilon)\} & \comment{\text{намаляваме разликата между прочетените $a$-та и $b$-та}}\\
    %   & \Delta(p, b, \#) = \emptyset & \comment{\text{не позволяваме да прочетем повече $b$-та от $a$-та}}\\
    %   & \Delta(p, a, A) = \emptyset & \comment{\text{не позволяваме разбъкване на $a$-та и $b$-та}} \\
    %   & \Delta(p, a, \#) = \emptyset & \comment{\text{не позволяваме разбъкване на $a$-та и $b$-та}}.
    % \end{align*}

    За да бъде коректна дефиницията на $\Delta$ функцията, трябва да посочим стойността ѝ за $|Q| . (|\Sigma| + 1) . |\Gamma|$ на брой тройки.
    За удобство, обикновено изпускаме тези редове, за които $\Delta$ функцията връща $\emptyset$.
   
    \begin{enumerate}[(1)]
    \item
      $\Delta(\qstart,\varepsilon,\varepsilon) = \{(\qstart,\#)\}$;
    \item
      $\Delta(\qstart,a,\#) = \{(\qstart, A\#), (p, A\#)\}$;
    \item
      $\Delta(\qstart,a,A) = \{(\qstart, AA), (p, AA)\}$;
    \item 
      $\Delta(\qstart,\varepsilon,\#) = \{(\qstart,\varepsilon)\}$;
    \item 
      $\Delta(\qstart,\varepsilon, А) = \emptyset$;
    \item
      $\Delta(\qstart, b, A) = \emptyset$;
    \item
      $\Delta(\qstart, b, \#) = \emptyset$;
    \item 
      $\Delta(p, b, A) = \{(p,\varepsilon)\}$;
    \item
      $\Delta(p, \varepsilon, \#) = \{(p, \varepsilon)\}$.
    \item 
      $\Delta(p, b, \#) = \emptyset$;
    \item 
      $\Delta(p, a, \#) = \emptyset$;
    \item 
      $\Delta(p, a, A) = \emptyset$;
    \item
      $\Delta(p, \varepsilon, \#) = \emptyset$.
    \end{enumerate}
  \end{itemize}
  Да видим как думата $a^2b^2$ се разпознава от автомата с празен стек:
  \marginpar{\writedown Докажете, че $L = \L_\emptyset(P)$!}
  \begin{align*}
    (\qstart, a^2b^2, \varepsilon) & \vdash_P (\qstart, a^2b^2, \#) & \comment{\text{правило }(1)}\\
                                   & \vdash_P (\qstart, ab^2, A\#) & \comment{\text{правило }(2)}\\
                                   & \vdash_P (p, b^2, AA\#) & \comment{\text{правило }(3)}\\
                                   & \vdash_P (p, b, A\#) & \comment{\text{правило }(8)}\\
                                   & \vdash_P (p, \varepsilon, \#) & \comment{\text{правило }(8)}\\
                                   & \vdash_P (p, \varepsilon, \varepsilon) & \comment{\text{правило }(9)}
  \end{align*}
  Поличихме, че
  \[(\qstart, a^2b^2, \varepsilon) \vdash^\star_P (p, \varepsilon, \varepsilon),\]
  откъдето следва, че $a^2b^2 \in \L_\emptyset(P)$.
  
\end{example}

\begin{example}
  \label{ex:omega-omega-r}
  За езика $L = \{\omega\omega^{rev} \mid \omega \in \{a,b\}^\star\}$ съществува стеков автомат $P$, такъв че
  $L = \L_\emptyset(P)$.
  Нека $P = \PDA$, където:
  \begin{itemize}
  \item 
    $Q = \{\qstart,q,p\}$;
  \item
    $\Sigma = \{a,b\}$;
  \item
    $\Gamma = \{A, B, \#\}$;
  \item
    $F = \emptyset$, защото разпознаваме с празен стек;
  \item
    Функцията на преходите $\Delta$ има следната дефиниция:
    \marginpar{За всички липсващи редове в дефиницията на $\Delta$ приемаме, че $\Delta$ връща $\emptyset$}
    \begin{enumerate}[(1)]
    \item
      $\Delta(\qstart,\varepsilon,\varepsilon) = \{(q, \#)\}$;
    \item 
      $\Delta(q, a, \#) = \{(q, A\#)\}$;
    \item 
      $\Delta(q, b, \#) = \{(q, B\#)\}$;
    \item
      $\Delta(q, \varepsilon, \#) = \{(q,\varepsilon)\}$;
    \item
      $\Delta(q, a, A) = \{(q, AA), (p, \varepsilon)\}$;
    \item
      $\Delta(q, a, B) = \{(q, AB)\}$;
    \item
      $\Delta(q, b, A) = \{(q, BA)\}$;
    \item
      $\Delta(q, b, B) = \{(q, BB), (p, \varepsilon)\}$;
    \item
      $\Delta(p, a, A) = \{(p,\varepsilon)\}$;
    \item
      $\Delta(p, b, B) = \{(p,\varepsilon)\}$;
    \item
      $\Delta(p, \varepsilon, \#) = \{(p,\varepsilon)\}$;
    \end{enumerate}
  \end{itemize}
  Основното наблюдение, което трябва да направим за да разберем конструкцията на автомата е, че
  всяка дума от вида $\omega\omega^{rev}$ може да се запише като $\omega_1aa\omega^{rev}_1$ или $\omega_1bb\omega^{rev}_1$.
  Да видим защо $P$ разпознава думата $abaaba$ с празен стек.
  Започваме по следния начин:
  \begin{align*}
    (\qstart, abaaba, \varepsilon) & \vdash_P (q,abaaba,\#) & \comment{\text{правило }(1)} \\
                                   & \vdash_P (q,baaba,A\#) & \comment{\text{правило }(2)}\\
                                   & \vdash_P (q, aaba, BA\#) & \comment{\text{правило }(7)}\\
                                   & \vdash_P (q, aba, ABA\#). & \comment{\text{правило }(6)}
  \end{align*}
  Сега можем да направим два избора как да продължим. Състоянието $p$ служи за маркер, което ни казва, че вече сме започнали 
  да четем $\omega^{rev}$. Поради тази причина, продължаваме така:
  \begin{align*}
    (q, aba, ABA\#) & \vdash_P (p, ba, BA\#) & \comment{\text{правило }(5)}\\
                          & \vdash_P (p, a, A\#) & \comment{\text{правило }(10)}\\
                          & \vdash_P (p, \varepsilon, \#) & \comment{\text{правило }(9)}\\
                          & \vdash_P (p,\varepsilon,\varepsilon). & \comment{\text{правило }(11)}
  \end{align*}
  Да проиграем още един пример. Да видим защо думата $aba$ не се извежда от автомата.
  \begin{align*}
    (\qstart, aba, \varepsilon) & \vdash_P (q,aba,\#) & \comment{\text{правило }(1)} \\
                                & \vdash_P (q, ba,A\#) & \comment{\text{правило }(2)}\\
                                & \vdash_P (q, a, BA\#) & \comment{\text{правило }(7)}\\
                                & \vdash_P (q, \varepsilon, ABA\#). & \comment{\text{правило }(6)}
  \end{align*}
  \marginpar{\writedown Докажете, че $\L_\emptyset(P) = L$ !}
  От последното моментно описание на автомата нямаме нито един преход, следователно
  думата $aba$ не се разпознава от $P$ с празен стек.


  \begin{enumerate}[a)]
  \item
    Докажете, че ако $\alpha x \beta \in L$ и $(q,x\beta, \alpha\#) \vdash_P (q, \beta,\gamma\#)$, то $\gamma\beta \in L$.
  \item
    Докажете, че ако $(q,\alpha,\beta\#) \vdash^\star_P(q,\varepsilon,\varepsilon)$, то $\beta\alpha \in L$.
  \end{enumerate}

\end{example}




\begin{example}
  За езика 
  \[L = \{\ \omega \in \{\texttt{[},\texttt{]}\}^\star \mid \texttt{left}(\omega) = \texttt{right}(\omega)\ \}\] съществува стеков автомат $P$, такъв че
  $L = \L_\emptyset(P)$.

  Нека $P = \PDA$, където:
  \begin{itemize}
  \item 
    $Q = \{\qstart,q\}$;
  \item
    $\Sigma = \{\texttt{[},\texttt{]}\}$;
  \item
    $\Gamma = \{\texttt{[}, \texttt{]}, \#\}$;
  \item
    $F = \emptyset$, защото разпознаваме с празен стек;
  \item
    Можем да дефинираме релацията на преходите $\Delta$ по следния начин:
    \begin{enumerate}[(1)]
    \item
      $\Delta(\qstart, \varepsilon, \varepsilon) = \{(q,\#)\}$;
    \item 
      $\Delta(q, \varepsilon, \#) = \{(q, \varepsilon)\}$;
    \item
      $\Delta(q, \texttt{[}, \#) = \{(q, \texttt{[}\#)\}$;
    \item
      $\Delta(q, \texttt{]}, \#) = \{(q, \texttt{]}\#)\}$;
    \item
      $\Delta(q, \texttt{[}, \texttt{[}) = \{(q, \texttt{[[})\}$;
    \item
      $\Delta(q, \texttt{[}, \texttt{]}) = \{(q, \varepsilon)\}$;
    \item
      $\Delta(q, \texttt{]}, \texttt{[}) = \{(q, \varepsilon)\}$;
    \item
      $\Delta(q, \texttt{]}, \texttt{]}) = \{(q, \texttt{]]})\}$.
    \end{enumerate}
  \end{itemize}
  Да видим защо думата $\texttt{[]]][[} \in \L_\emptyset(P)$.
  \begin{align*}
    (\qstart, \texttt{[]]][[}, \varepsilon) & \vdash_P (q, \texttt{[]]][[}, \#) & \comment{\text{правило }(1)}\\
                                            & \vdash_P (q,\ \texttt{]]][[},\ \texttt{[}\#) & \comment{\text{правило }(3)}\\
                                            & \vdash_P (q,\ \texttt{]][[},\ \#) & \comment{\text{правило }(7)}\\
                                            & \vdash_P (q,\ \texttt{][[},\ \texttt{]}\#) & \comment{\text{правило }(4)}\\
                                            & \vdash_P (q,\ \texttt{[[},\ \texttt{]]}\#) & \comment{\text{правило }(8)}\\
                                            & \vdash_P (q,\ \texttt{[},\ \texttt{]}\#) & \comment{\text{правило }(6)}\\
                                            & \vdash_P (q,\ \varepsilon,\ \#) & \comment{\text{правило }(6)}\\
                                            & \vdash_P (q,\ \varepsilon,\ \varepsilon). & \comment{\text{правило }(2)}
  \end{align*}
\end{example}

\begin{example}
  За езика 
  \[L = \{\ \omega \in \{\texttt{[},\texttt{]}\}^\star \mid \omega\text{ е балансирана дума}\ \}\] съществува стеков автомат $P$, такъв че
  $L = \L_\emptyset(P)$.

  Нека $P = \PDA$, където:
  \begin{itemize}
  \item 
    $Q = \{\qstart\}$;
  \item
    $\Sigma = \{\texttt{[},\texttt{]}\}$;
  \item
    $\Gamma = \{\texttt{[}, \#\}$;
  \item
    $F = \emptyset$, защото разпознаваме с празен стек;
  \item
    Функцията на преходите $\Delta$ има следната дефиниция:
    \begin{enumerate}[(1)]
    \item 
      $\Delta(\qstart, \varepsilon, \#) = \{(\qstart, \varepsilon)\}$;
    \item
      $\Delta(\qstart, \texttt{[}, \#) = \{(\qstart, \texttt{[}\#)\}$;
    \item
      $\Delta(\qstart, \texttt{]}, \#) = \emptyset$;
    \item
      $\Delta(\qstart, \texttt{[}, \texttt{[}) = \{(\qstart, \texttt{[[})\}$;
    \item
      $\Delta(\qstart, \texttt{[}, \texttt{]}) = \emptyset$;
    \item
      $\Delta(\qstart, \texttt{]}, \texttt{[}) = \{(\qstart, \varepsilon)\}$;
    \item
      $\Delta(\qstart, \texttt{]}, \texttt{]}) = \emptyset$.
    \end{enumerate}
  \end{itemize}  

  \begin{enumerate}[(1)]
  \item 
    Ако $\texttt{[}^nx\alpha \in L$ и
    $(q, x\alpha, \texttt{[}^n\sharp) \vdash (q, \alpha, \texttt{[}^{l}\sharp)$,
    то $\texttt{[}^l\alpha \in L$.
    \begin{itemize}
    \item 
      Нека $|\alpha| = 0$. Тогава е ясно, че $x = \texttt{]}$ и $n = 1$.
      Тогава $l = 0$ и $\varepsilon \in L$.
    \item 
      Нека $|\alpha| > 0$. Тогава имаме два случая за $x$.
      \begin{itemize}
      \item 
        Ако $x = \texttt{[}$, то 
        \[(q, \texttt{[}\alpha, \texttt{[}^n\sharp) \vdash (q, \alpha, \texttt{[}^{n+1}\sharp).\]
        Щом $\texttt{[}^n\texttt{[}\alpha \in L$, то директно следва, че $\texttt{[}^{n+1}\alpha \in L$.
      \item
        Ако $x = \texttt{]}$, то според функцията на преходите, $n > 0$ и 
        \[(q, \texttt{]}\alpha, \texttt{[}^n\sharp) \vdash (q, \alpha, \texttt{[}^{n-1}\sharp).\]
        Щом $\texttt{[}^{n-1}\texttt{[}\texttt{]}\alpha \in L$, то лесно се проверява, че $\texttt{[}^{n-1}\alpha \in L$.
      \end{itemize}
    \end{itemize}
  \item
    Нека $\alpha \in L$.
    Ако $(q,\alpha,\sharp) \vdash^\star_P (q,\varepsilon,\varepsilon)$, то според (1),
    на всяка стъпка от изчислението имаме, че 
  \item
    Ако $(q,\alpha,\texttt{[}^n\sharp) \vdash^\star (q,\varepsilon,\varepsilon)$,
    то $\texttt{[}^n\alpha \in L$.
    Индукция по дължината на извода $l$.
    \begin{itemize}
    \item 
      Ако $(q,\alpha,\texttt{[}^n\sharp) \vdash^1_P (q,\varepsilon,\varepsilon)$, то е ясно, че $n = 0$ и $\alpha = \varepsilon$.
    \item
      Нека $(q,\alpha,\texttt{[}^n\sharp) \vdash^{l+1}_P (q,\varepsilon,\varepsilon)$.
      \begin{itemize}
      \item 
        \begin{align*}
          (q,\texttt{[}\alpha,\texttt{[}^n\sharp) & \vdash_P (q,\alpha,\texttt{[}^{n+1}\sharp) \\
                                                  & \vdash^l_P (q, \varepsilon, \varepsilon).
        \end{align*}
      \item
        \begin{align*}
          (q,\texttt{]}\alpha,\texttt{[}^n\sharp) & \vdash_P (q,\alpha,\texttt{[}^{n-1}\sharp) \\
                                                  & \vdash^l_P (q, \varepsilon, \varepsilon).
        \end{align*}
      \end{itemize}
    \end{itemize}
  \end{enumerate}

\end{example}


\begin{thm}
  \marginpar{(\cite{hopcroft1}, стр. 114) }
  Нека $L$ е произволен език над азбука $\Sigma$.
  \begin{enumerate}[1)]
  \item 
    Ако съществува НСА $P$, за който $L = \L_F(P)$, то съществува НСА $P^\prime$, за който $L = \L_\emptyset(P^\prime)$.
  \item
    Ако съществува НСА $P$, за който $L = \L_\emptyset(P)$, то съществува НСА $P^\prime$, за който $L = \L_F(P^\prime)$.
  \end{enumerate}
  С други думи, езиците разпознавани от НСА с празен стек са точно езиците разпознавани от НСА с финално състояние.
\end{thm}
\begin{proof}
  \begin{enumerate}[1)]
  \item 
    Нека $L = \L_F(P)$, където $P = \PDA$.
    Ще построим $P^\prime$, така че да симулира $P$ и като отидем във финално състояние ще изпразним стека.
    Нека
    \[P^\prime = \langle{Q\cup\{q_e,\qstart'\},\Sigma,\Gamma \cup \{\$\},\$,\qstart',\Delta^\prime,\emptyset}\rangle,\]
    където $\$ \not\in \Gamma$.
    Важно е $P^\prime$ да има собствен нов символ за дъно на стека, защото е възможно за някоя дума $\alpha \not\in \L_F(P)$
    стековият автомат $P$ да си изчисти стека и така да разпознаем повече думи.
    \begin{itemize}
    \item 
      \marginpar{- започваме симулацията}
      $\Delta'(\qstart',\varepsilon,\$) = \{(\qstart,\#\$)\}$;
    \item
      \marginpar{- симулираме $P$}
      $\Delta'(q,a,X)$ включва множеството $\Delta(q,a,X)$, за всяко $q\in Q$, $a\in\Sigma_\varepsilon$, $X\in\Gamma$;
    \item
      \marginpar{- ако сме във финално, започваме да чистим стека}
      $\Delta'(q,\varepsilon,X)$ съдържа също и елемента $(q_e,\varepsilon)$, за всяко $q\in F$, $X \in \Gamma \cup \{\$\}$;
    \item
      \marginpar{- изчистваме стека}
      $\Delta'(q_e,\varepsilon,X) = \{(q_e,\varepsilon)\}$, за всяко $X \in \Gamma \cup \{\$\}$;
    \item
      $\Delta'$ няма други правила.
    \end{itemize}
  \item
    Сега имаме $L = \L_S(P)$, където $P = \langle{Q,\Sigma,\Gamma,\#,\qstart,\Delta,\emptyset}\rangle$. 
    Да положим
    \[P^\prime = \langle{Q\cup\{\qstart',q_f\}, \Sigma, \Gamma \cup \{\$\}, \$, \qstart', \Delta^\prime, \{q_f\}}\rangle.\]
    $P^\prime$ ще симулира $P$ като ще внимаваме кога $P$ изчиства символа $\#$. Тогава ще искаме да отидем във финалното състояние $q_f$.
    \begin{itemize}
    \item 
      \marginpar{- започваме симулацията}
      $\Delta'(\qstart',\varepsilon,\$) = \{(\qstart, \#\$)\}$;
    \item
      \marginpar{- симулираме $P$}
      $\Delta'(q,a,X) = \Delta(q,a,X)$, за всяко $q \in Q$, $a \in \Sigma_\varepsilon$, $X \in \Gamma$;
    \item
      \marginpar{- щом сме стигнали до $\$$, значи $P$ е изчистил стека си}
      $\Delta'(q,\varepsilon,\$) = \{(q_f,\varepsilon)\}$.
    \end{itemize}
  \end{enumerate}
\end{proof}

\begin{problem}
  Като използвате стековия автомат от Пример \ref{ex:anbn}, дефинирайте автомат $P'$, за който
  $\L_F(P') = \{a^nb^n \mid n\in\Nat\}$.
\end{problem}

\begin{framed}
  \begin{lemma}
    За всяка безконтекстна граматика $G$,
    съществува стеков автомат $P$, такъв че $\L(G) = \L_S(G)$.
  \end{lemma}
\end{framed}
\begin{proof}
  \marginpar{\cite[стр. 136]{papadimitriou}}
  Нека е дадена безконтекстната граматика $G = \CFG$.
  Нашата цел е да построим стеков автомат $P$, така че $\L_\emptyset(P) = \L(G)$.
  Нека  \[P = \langle{\{q\},\Sigma,\Sigma\cup V,S,q,\Delta,\emptyset}\rangle,\]
  където функцията на преходите е:
  \begin{align*}
    & \Delta(q,\varepsilon,A) = \{(q,\alpha)\mid A\to_G \alpha\}, \text{ за всяка променлива }A \in V\\
    & \Delta(q,a,a) = \{(q,\varepsilon)\}, \text{ за всяка буква } a \in \Sigma.
  \end{align*}
  Ще докажем, че за всяка променлива $A \in V$, за всяка дума $\alpha \in \Sigma^\star$ и $\gamma \in (\Sigma \cup V)^\star$, то е изпълнено, че:
  \begin{enumerate}[(1)]
  \item
    ако $S \to^\star_G \alpha \gamma$, то $(q, \alpha, S) \vdash^\star_P (q, \varepsilon, \gamma)$;
  \item
    ако $(q, \alpha, S) \vdash^\star_P (q, \varepsilon, \gamma)$, то $S \to^\star_G \alpha \gamma$.
  \end{enumerate}
  Тогава, ако вземем променливата $\gamma = \varepsilon$, то ще получим, че
  \begin{align*}
    \alpha \in \L(G) & \iff S \to^\star_G \alpha \\
                     & \iff (q,\alpha,S) \vdash^\star_P (q, \varepsilon, \varepsilon)\\
                     & \iff \alpha \in \L_\emptyset(P).
  \end{align*}

  \begin{enumerate}[(1)]
  \item
    
  \item
    
  \end{enumerate}
  
\end{proof}

\begin{framed}
\begin{thm}
  \label{th:push-down-context-free}
  Класът на езиците, които се разпознават от краен стеков автомат съвпада с
  класа на безконтекстните езици.
\end{thm}
\end{framed}
\begin{proof}
  \marginpar{\cite[стр. 117]{hopcroft1}}
  Ще разгледаме двете посоки на твърдението поотделно.
  \begin{enumerate}[1)]
  \item 
    Нека е дадена безконтекстна граматика $G = \CFG$.
    Нашата цел е да построим стеков автомат $P$, така че $\L_S(P) = \L(G)$.
    Нека  \[P = \langle{\{q\},\Sigma,\Sigma\cup V,S,q,\Delta,\emptyset}\rangle,\]
    където функцията на преходите е:
    \begin{align*}
      & \Delta(q,\varepsilon,A) = \{(q,\alpha)\mid A\to\alpha\mbox{ е правило в граматиката }G\}\\
      & \Delta(q,a,a) = \{(q,\varepsilon)\}
    \end{align*}
  \item
    Нека имаме $P = \langle{Q, \Sigma, \Gamma, \Delta, s, \#, \emptyset}\rangle$.
    Ще дефинираме безконтекстна граматика $G$, за която $\L_S(P) = \L(G)$.
    Променливите на граматика са 
    \[V = \{[q,A,p] \mid q,p \in Q, A \in \Gamma\}.\]
    Правилата на $G$ са следните:
    \begin{itemize}
    \item
      $S \to [s,\#,q]$, за всяко $q \in Q$;
    \item
      $[q,A,p] \to a[q_1,B_1,q_2][q_2,B_2,q_3]\dots [q_m,B_m,p]$,
      където 
      \[(q_1,B_1\dots B_m) \in \Delta(q, a, A)\]
      и произволни $q,q_1,\dots,q_{m},p \in Q$,
      $a \in \Sigma \cup \{\varepsilon\}$.

      Да обърнем внимание, че е възможно $m = 0$.
      Това означава, че $(p,\varepsilon) \in \Delta(q, a, A)$ и тогава имаме правилото $[q,A,p] \to a$, където $a \in \Sigma \cup \{\varepsilon\}$.
    \end{itemize}
    Трябва да докажем, че:
    \[[q,A,p] \rightarrow^\star_G \alpha\ \Leftrightarrow\ (q,\alpha,A) \vdash^\star_P (p,\varepsilon,\varepsilon).\]
    \begin{description}
    \item[$(\Rightarrow)$]
      С пълна индукция по $i$, ще докажем, че 
      \[(q,\alpha,A) \vdash^i_P (p,\varepsilon,\varepsilon)\ \implies\ [q,A,p] \to^\star_G \alpha.\]
      Ако $i = 1$, то е лесно, защото $\alpha \in \Sigma \cup\{\varepsilon\}$ и $m = 0$.
      Според конструкцията на граматиката $G$ имаме правилото $[q,A,p] \to a$.

      Ако $i > 1$, нека $\alpha = a\beta$. Тогава:
      \marginpar{Възможно е $a = \varepsilon$}
      \[(q,a\beta,A) \vdash_P (q_1,\beta,B_1\dots B_n) \vdash^{i-1}_P (p, \varepsilon, \varepsilon).\]
      Да разбием думата $\beta$ на $n$ части, $\beta = \beta_1\cdots \beta_n$, със свойството, че след като прочетем $\beta_i$ 
      сме премахнали променливата $B_i$ от върха на стека. Това означава, че съществуват състояния $q_2,\dots,q_{n}$:
      \begin{align*}
        & (q_1, \beta_1, B_1) \vdash^{l_1}_P (q_{2},\varepsilon,\varepsilon)\\
        & (q_2, \beta_2, B_2) \vdash^{l_2}_P (q_{3},\varepsilon,\varepsilon)\\
        & \ \vdots\\
        & (q_n, \beta_n, B_n) \vdash^{l_n}_P (p,\varepsilon,\varepsilon),
      \end{align*}
      където $l_1+l_2+\cdots+l_n = i-1$.
      Сега по {\bf И.П.} получаваме:
      \begin{align*}
        & (q_1, \beta_1, B_1) \vdash^{l_1}_P (q_{2},\varepsilon,\varepsilon) \implies [q_1,B_1, q_{2}] \to^\star_G \beta_1\\
        & (q_2, \beta_2, B_2) \vdash^{l_2}_P (q_{3},\varepsilon,\varepsilon) \implies [q_2,B_2, q_{3}] \to^\star_G \beta_2\\
        & \ \vdots\\
        & (q_n, \beta_n, B_n) \vdash^{l_n}_P (p,\varepsilon,\varepsilon) \implies [q_n,B_n, p] \to^\star_G \beta_n.
      \end{align*}
      Понеже имаме правилото $\Delta(q,a,A) \ni (q_1,B_1\cdots B_n)$, то в граматиката имаме правилото
      \[[q,A,p] \rightarrow_G a[q_1,B_1,q_2]\dots[q_n,B_n,p].\]
      Обединявайки всичко, получаваме извода
      \[[q,A,p] \rightarrow^\star_G a\beta.\]
    \item[$(\Leftarrow)$]
      Отново с пълна индукция по $i$ ще докажем, че
      \[[q,A,p] \rightarrow^i_G \alpha \implies (q,\alpha,A) \vdash^\star_P (p,\varepsilon,\varepsilon).\]
      Ако $i = 1$, то имаме $[q,A,p] \rightarrow \alpha$, където $\alpha = a$ или $\alpha = \varepsilon$.
      Ако $i > 1$, то имаме, че $\alpha = a\beta$ и за някое $n$, 
      \[[q,A,p] \rightarrow_G a[q_1,B_1,q_2][q_2,B_2,q_3]\dots[q_n,B_n,p] \rightarrow^{i-1}_G \beta.\]
      Отново нека $\beta = \beta_1\dots \beta_n$, където 
      \begin{align*}
        & [q_1,B_1,q_{2}] \to^{i_1}_G \beta_1\\
        & [q_2,B_2,q_{3}] \to^{i_2}_G \beta_2\\
        & \ \vdots\\
        & [q_{n},B_n, p] \rightarrow^{i_n}_G \beta_n,
      \end{align*}
      където $i_1 + i_2 + \cdots + i_n = i-1$.
      От {\bf И.П.} получаваме, че 
      \begin{align*}
        & [q_1,B_1,q_{2}] \to^{i_1}_G \beta_1 \implies (q_1,\beta_1,B_1) \vdash^\star_P (q_{2},\varepsilon,\varepsilon) \\
        & [q_2,B_2,q_{3}] \to^{i_2}_G \beta_2 \implies (q_2,\beta_2,B_2) \vdash^\star_P (q_{3},\varepsilon,\varepsilon)\\
        & \ \vdots\\
        & [q_{n},B_n, p] \rightarrow^{i_n}_G \beta_n \implies  (q_{n},\beta_n,B_n) \vdash^\star_P (p,\varepsilon,\varepsilon).
      \end{align*}
      Правилото $[q,A,p] \rightarrow_G a[q_1,B_1,q_2][q_2,B_2,q_3]\dots[q_n,B_n,p]$ е добавено в граматиката, 
      защото $\Delta(q,a,A) \ni (q_1, B_1B_2\cdots B_n)$. 
      Обединявайки всичко, което знаем, получаваме:
      \begin{align*}
        (q, a\beta, A) & \vdash_P (q_1, \beta_1\cdots\beta_n, B_1\cdots B_n)\\
        & \vdash^\star_P (q_2, \beta_{2}\cdots\beta_n, B_2\cdots B_n)\\
        & \dots\\
        & \vdash^\star_P (q_n, \beta_n, B_n)\\
        & \vdash^\star_P (p, \varepsilon, \varepsilon)
      \end{align*}
    \end{description}
  \end{enumerate}
\end{proof}


\begin{example}
  Нека е дадена граматиката $G$ с правила 
  \begin{align*}
    & S \to ASB\ |\ \varepsilon\\
    & A \to aAa\ |\ a\\
    & B \to bBb\ |\ b.
  \end{align*}
  Ще построим стеков автомат $P = \PDA$, такъв че $\L_\emptyset(P) = \L(G)$.
  \begin{itemize}
  \item
    $\Sigma = \{a,b\}$;
  \item 
    $\Gamma = \{A,S,B,a,b\}$;
  \item
    $\# = S$;
  \item
    $Q = \{\qstart\}$;
  \item
    $F = \emptyset$;
  \item
    Дефинираме релацията на преходите, следвайки конструкцията от \Th{push-down-context-free}:
    \begin{itemize}
    \item 
      $\Delta(\qstart,\varepsilon, S) = \{\pair{\qstart,ASB}, \pair{\qstart,\varepsilon}\}$;
    \item
      $\Delta(\qstart, \varepsilon, A) = \{\pair{\qstart, aAa}, \pair{\qstart, a}\}$;
    \item
      $\Delta(\qstart, \varepsilon, B) = \{\pair{\qstart, bBb}, \pair{\qstart, b}\}$;
    \item
      $\Delta(\qstart, a, a) = \{\pair{\qstart,\varepsilon}\}$;
    \item
      $\Delta(\qstart, b, b) = \{\pair{\qstart,\varepsilon}\}$.
    \end{itemize}
  \end{itemize}
\end{example}


\begin{example}
  Да видим как по стековия автомат за езика 
  \[L = \{\omega\omega^{rev} \mid \omega \in \{a,b\}^\star\}\]
  от \Ex{omega-omega-r} можем да построим граматика, следвайки конструкцията от \Th{push-down-context-free}.

  \begin{itemize}
  \item
    Променливите на граматиката са $2 \cdot 3 \cdot 2 + 1 = 13$.
  \item
    Нека тук за улеснение да пишем $q$ вместо $\qstart$;
  \item 
    Започваме с правилата $S \to [q,\#,q]\ |\ [q,\#,p]$,
    защото началното състояние на автомата е $q$ и символът за дъно на стек е $\#$.
  \item
    Понеже $\Delta(q, a, \#) = \{(q, A\#)\}$, то имаме правилата
    \begin{align*}
      & [q, \#, q] \to a[q,A,q][q,\#,q]\ |\ a[q,A,p][p,\#,q]\\
      & [q, \#, p] \to a[q,A,q][q,\#,p]\ |\ a[q,A,p][p,\#,p].
    \end{align*}
  \item
    Понеже $\Delta(q, a, A) = \{(q, AA), (p, \varepsilon)\}$, то добавяме правилата:
    \begin{align*}
      & [q,A,q] \to a[q, A, q][q,A,q]\ |\ a[q,A,p][p,A,q]\\
      & [q,A,p] \to a\ |\ a[q, A, q][q,A,p]\ |\ a[q,A,p][p,A,p].
    \end{align*}
  \item
    \marginpar{\ding{45} Довършете граматиката и след това я опростете!}
    Понеже $\Delta(q, a, B) = \{(q, AB)\}$, то добавяме правилата:
    \begin{align*}
      & [q, B, q] \to a[q,A,q][q,B,q]\ |\ a[q,A,p][p,B,q]\\
      & [q, B, p] \to a[q,A,q][q,B,p]\ |\ a[q,A,p][p,B,p].
    \end{align*}
  \end{itemize}
\end{example}

\begin{thm}
  \label{th:intersection-context-reg}
  \marginpar{\cite[стр. 144]{papadimitriou}}
  Нека $L$ e безконтекстен език и $R$ е регулярен език.
  Тогава тяхното сечение $L \cap R$ е безконтекстен език.
\end{thm}
\begin{proof}
  Нека имаме стеков автомат
  \[\M_1 = \PDAn{1}, \text{ където } \L_F(\M_1) = L,\]
  \marginpar{всъщност няма нужда да е детерминиран}
  и краен детерминиран автомат 
  \[\M_2 = \FAn{2}, \text{ където } \L(\M_2) = R.\]
  Ще определим нов стеков автомат $\M = \PDA$, където
  \begin{itemize}
  \item 
    $Q = Q_1 \times Q_2$;
  \item
    $\qstart = \pair{s_1,s_2}$;
  \item
    $F = F_1 \times F_2$;
  \item 
    Функцията на преходите $\Delta$ е дефинирана както следва:
    \begin{itemize}
    \item 
      \marginpar{симулираме едновременно изчислението и на двата автомата}
      Ако $\Delta_1(q_1, a, b) \ni \pair{r_1,c}$
      и $\delta_2(q_2,a) = r_2$, то
      \[\Delta(\pair{q_1,q_2},a,b) \ni \pair{\pair{r_1,r_2}, c}.\]
    \item
      \marginpar{празен ход на автомата $M_2$}
      Ако $\Delta_1(q_1,\varepsilon,b) \ni \pair{r_1,c}$,
      то за всяко $q_2 \in Q_2$,
      \[\Delta(\pair{q_1,q_2},\varepsilon,b) \ni \pair{\pair{r_1,q_2},c}.\]    
    \item
      \marginpar{\writedown Докажете, че $\L(\M) = \L(\M_1) \cap \L(\M_2)$ !}
      $\Delta$ не съдържа други преходи;
    \end{itemize}
  \end{itemize}
\end{proof}

\Th{intersection-context-reg} е удобна, когато искаме да докажем, че даден език не е безконтекстен.
С нейна помощ можем да сведем езика до друг, за който вече знаем, че не е безконтекстен.

\begin{example}
  Езикът $L = \{\omega \in \{a,b,c\}^\star \mid N_a(\omega) = N_b(\omega) = N_c(\omega)\}$ не е безконтекстен.
  Да допуснем, че $L$ е безконтекстен език.
  Тогава \[L^\prime = L \cap \L(a^\star b^\star c^\star)\] също е безконтекстен език.
  Но $L^\prime = \{a^nb^nc^n \mid n \in \Nat\}$, за който знаем от \Prob{anbncn}, че {\em не} е безконтекстен.
  Достигнахме до противоречие. Следователно, $L$ не е безконтекстен език.
\end{example}

\begin{problem}
  Докажете, че езикът $L = \{\omega \in \{a,b\}^\star \mid N_a(\omega) = 2N_b(\omega)\}$ не е безконтекстен.
\end{problem}
\begin{hint}
  Разгледайте езика $L' = L \cap \L(a^\star b^\star a^\star)$.
\end{hint}


%%% Local Variables: 
%%% mode: latex
%%% TeX-master: "../eai"
%%% End: 
