\section{Неограничени граматики}\label{sect:unrestricted-grammar}
\index{граматика!неограничена}

\mynote{На англ. {\em unrestricted grammar}. Това е тип 0 граматики в йерархията на Чомски \cite[стр. 220]{hopcroft1}.}
{\bf Неограничена граматика} e наредена четворка от вида
\[G = (V, \Sigma, R, S),\]
където:
\begin{itemize}
\item
  $V$ е крайно множество от {\em променливи} (нетерминали);
\item
  $\Sigma$ е крайно множество от {\em букви} (терминали), като $\Sigma \cap V = \emptyset$;
\item
  \mynote{В \cite{hopcroft1} правилата се наричат {\em productions} или {\em production rules}.}
  $R \subseteq (V\cup\Sigma)^+ \times (V \cup \Sigma)^\star$ е крайно множество от {\em правила}.
  За по-добра яснота, обикновено правилата $(\alpha, \beta) \in R$ ще означаваме като 
  $\alpha \to_G \beta$. Когато е ясно за коя граматика говорим, ще пишем просто $\alpha \to \beta$.
\item
  $S \in V$ е началната променлива (нетерминал). 
\end{itemize}

\index{граматика!извод}
Удобно е да дефинираме извод на думата $\beta$ от думата $\alpha$ в граматиката $G$ за $\ell$ стъпки, което ще означаваме като $\alpha \derive{\ell}_G \beta$,
с индукция по броя на стъпките $\ell$ по следния начин:
\mynote{Обърнете внимание, че имаме недетерминизъм в тази дефиниция на извод. Също така, понякога ,за удобство, ще пишем просто $\derive{\ell}$ вместо $\derive{\ell}_G$,
  когато се знае за коя граматика говорим.}

\begin{figure}[H]
  \begin{subfigure}[b]{0.5\textwidth}
    \begin{prooftree}
      \AxiomC{}
      \RightLabel{\scriptsize{правило (0)}}
      \UnaryInfC{$\alpha \derive{0}_G \alpha$}
    \end{prooftree}
    \vspace*{2mm}
  \end{subfigure}
  ~
  \begin{subfigure}[b]{0.5\textwidth}
    \begin{prooftree}
      \AxiomC{$\alpha \to_G \gamma$}
      \AxiomC{$\gamma \derive{\ell}_G \beta$}
      \RightLabel{\scriptsize{правило (1)}}
      \BinaryInfC{$\alpha \derive{\ell+1}_G \beta$}
    \end{prooftree}
    \vspace*{2mm}
  \end{subfigure}
  \begin{subfigure}[b]{0.5\textwidth}
    \begin{prooftree}
      \AxiomC{$\alpha_1 \derive{\ell_1}_G \beta_1$}
      \AxiomC{$\alpha_2 \derive{\ell_2}_G \beta_2$}
      \RightLabel{\scriptsize{правило (2)}}
      \BinaryInfC{$\alpha_1\cdot\alpha_2 \derive{\ell_1+\ell_2}_G \beta_1\cdot \beta_2$}
    \end{prooftree}
  \end{subfigure}
\end{figure}




\mynote{В частност имаме следното:
\begin{prooftree}
  \AxiomC{$\alpha \to_G \beta$}
  \UnaryInfC{$\alpha \derive{1}_G \beta$}
\end{prooftree}
}

% Нека дефинираме $\derive{\star}_G$ като рефлексивното и транзитивно затваряне на релацията $\derive{1}_G$. С други думи,
Сега дефинираме релацията $\derive{\star}_G$ като
\[ \alpha \derive{\star}_G \beta\ \dff\ (\exists \ell\in\Nat)[\ \alpha \derive{\ell}_G \beta\ ].\]

Езикът, който се поражда от граматиката $G$ дефинираме по следния начин:
\[\L(G) \df \{\omega \in \Sigma^\star \mid S \derive{\star}_G \omega\}.\]


% \begin{proposition}\label{pr:unrestricted-grammar:add}
%   За произволно естествено число $\ell$ е изпълнено, че:
%   \begin{prooftree}
%     \AxiomC{$\alpha \derive{\ell}_G \beta$}
%     \AxiomC{$\gamma,\rho \in (V \cup \Sigma)^\star$}
%     \BinaryInfC{$\gamma\alpha\rho \derive{\ell} \gamma \beta \rho$}
%   \end{prooftree}
% \end{proposition}

\begin{proposition}\label{pr:unrestricted-grammar:concat}
  За всяко $k$ е изпълнено, че:
  \begin{prooftree}
    \AxiomC{$\alpha_1 \derive{\ell_1} \beta_1$}
    \AxiomC{$\dots$}
    \AxiomC{$\alpha_k \derive{\ell_k} \beta_k$}
    \RightLabel{\scriptsize{$(\ell = \sum^k_{i=1} \ell_i)$}}
    \TrinaryInfC{$\alpha_1\cdots\alpha_k \derive{\ell} \beta_1\cdots\beta_k$}
  \end{prooftree}
\end{proposition}
\begin{hint}
  Индукция по $k$.
\end{hint}

\begin{proposition}\label{pr:unrestricted-grammar:general-step}
  За произволни естествени числа $\ell_1$ и $\ell_2$ е изпълнено, че:
  \begin{prooftree}
    \AxiomC{$\alpha \derive{\ell_1} \beta$}
    \AxiomC{$\beta \derive{\ell_2} \gamma$}
    \BinaryInfC{$\alpha \derive{\ell_1+\ell_2} \gamma$}
  \end{prooftree}  
\end{proposition}
\begin{hint}
  Пълна индукция по $\ell_1$.
  \begin{itemize}
  \item
    Ако $\ell_1 = 0$, то е тривиално.
  \item
    Ако $\ell_1 = 1$, то следва от правило (1).
  \item
    Ако $\ell_1 > 1$, то имаме два случая за това как сме получили $\alpha \derive{\ell_1} \beta$.
    \begin{itemize}
    \item
      Първият случай е, ако сме приложили правило (1), т.е.
      \begin{prooftree}
        \AxiomC{$\alpha \to_G \delta$}
        \AxiomC{$\delta \derive{\ell_1-1} \beta$}
        \RightLabel{\scriptsize{(1)}}
        \BinaryInfC{$\alpha \derive{\ell_1} \beta$}
      \end{prooftree}
      Тогава получаваме следния извод:
      \begin{prooftree}
        \AxiomC{$\alpha \to_G \delta$}
        \AxiomC{$\delta \derive{\ell_1-1} \beta$}
        \AxiomC{$\beta \derive{\ell_2} \gamma$}
        \RightLabel{\scriptsize{\IndHyp}}
        \BinaryInfC{$\delta \derive{\ell_1+\ell_2-1} \gamma$}
        \RightLabel{\scriptsize{(1)}}
        \BinaryInfC{$\alpha \derive{\ell_1+\ell_2} \gamma$}
      \end{prooftree}
    \item
      \mynote{Защо сме сигурни, че можем да намерим такова разбиване ?}
      Вторият случай е, ако сме приложили правило (2), т.е. $\ell_1 = \ell'_1 + \ell''_1$ и $\ell'_1 > 0$ и $\ell''_1 > 0$ и имаме извода:
      \begin{prooftree}
        \AxiomC{$\alpha_1 \derive{\ell'_1} \beta_1$}
        \AxiomC{$\alpha_2 \derive{\ell''_1} \beta_2$}
        \RightLabel{\scriptsize{(2)}}
        \BinaryInfC{$\underbrace{\alpha_1\alpha_2}_{\alpha} \derive{\ell'_1+\ell''_2} \underbrace{\beta_1\beta_2}_{\beta}$}
      \end{prooftree}
      Тогава получаваме следния извод:
      \begin{prooftree}
        \AxiomC{$\alpha_1 \derive{\ell'_1} \beta_1$}
        \LeftLabel{\scriptsize{(2)}}
        \AxiomC{}
        \LeftLabel{\scriptsize{(0)}}
        \UnaryInfC{$\alpha_2 \derive{0} \alpha_2$}
        \LeftLabel{\scriptsize{(2)}}
        \BinaryInfC{$\alpha \derive{\ell'_1} \beta_1\alpha_2$}
        \AxiomC{}
        \RightLabel{\scriptsize{(0)}}
        \UnaryInfC{$\beta_1 \derive{0} \beta_1$}
        \AxiomC{$\alpha_2 \derive{\ell''_1} \beta_2$}
        \RightLabel{\scriptsize{(2)}}
        \BinaryInfC{$\beta_1\alpha_2 \derive{\ell''_1} \beta$}
        \AxiomC{$\beta \derive{\ell_2} \gamma$}
        \RightLabel{\scriptsize{\IndHyp}}
        \BinaryInfC{$\beta_1\alpha_2 \derive{\ell''_1+\ell_2} \gamma$}
        \RightLabel{\scriptsize{\IndHyp}}
        \BinaryInfC{$\alpha \derive{\ell_1+\ell_2} \gamma$}
      \end{prooftree}
    \end{itemize}
  \end{itemize}
\end{hint}

\begin{proposition}\label{pr:unrestricted-grammar:context-general-step}
  За произволни естествени числа $\ell_1$ и $\ell_2$ е изпълнено, че:
  \begin{prooftree}
    \AxiomC{$\alpha \derive{\ell_1} \rho\beta\delta$}
    \AxiomC{$\beta \derive{\ell_2} \gamma$}
    \BinaryInfC{$\alpha \derive{\ell_1+\ell_2} \rho\gamma\delta$}
  \end{prooftree}  
\end{proposition}
\begin{proof}
  Имаме следния извод:
  \begin{prooftree}
    \AxiomC{$\alpha \derive{\ell_1} \rho\beta\delta$}
    \AxiomC{}
    \LeftLabel{\scriptsize{(0)}}
    \UnaryInfC{$\rho \derive{0} \rho$}
    \AxiomC{$\beta \derive{\ell_2} \gamma$}
    \LeftLabel{\scriptsize{(2)}}
    \BinaryInfC{$\rho \beta \derive{\ell_2} \rho\gamma$}
    \AxiomC{}
    \RightLabel{\scriptsize{(0)}}
    \UnaryInfC{$\delta \derive{0} \delta$}
    \RightLabel{\scriptsize{(2)}}
    \BinaryInfC{$\rho \beta \delta \derive{\ell_2} \rho \gamma \delta$}
    \RightLabel{\scriptsize{(\Proposition{unrestricted-grammar:general-step})}}
    \BinaryInfC{$\alpha \derive{\ell_1+\ell_2} \rho\gamma\delta$}
  \end{prooftree}
\end{proof}



\begin{extra}
  \begin{remark}\label{rem:unrestricted-grammar:original-def}
    В повечето учебници авторите дефинират релацията $\derive{\star}_G$ като
    рефлексивното и транзитивното затваряне на релацията $\derive{}_G$, където
    \[\alpha \derive{}_G \beta \dff \alpha = \rho\gamma\delta\ \&\ \gamma \to_G \gamma'\ \&\ \beta = \rho\gamma'\delta\text{, за някои $\rho$ и $\delta$}.\]
    Както се вижда от следващото твърдение, релацията $\derive{1}_G$, която дефинирахме по-горе, е точно релацията $\derive{}_G$.
  \end{remark}
  
  % Както отбелязахме в \Remark{unrestricted-grammar:original-def}, в повечето учебници дават следната дефиниция на релацията $\derive{1}$, която тук ще формулираме като твърдение, което следва от нашите правила за извод.
  \begin{proposition}\label{pr:grammar:alternative-def}
    $\alpha \derive{1}_G \beta$ точно тогава, когато $\alpha \derive{}_G \beta$.
  \end{proposition}
  \begin{hint}
    За посоката $(\leftarrow)$, нека $\alpha \derive{}_G \beta$, т.е. $\alpha = \delta \gamma \rho$, $\gamma \to_G \gamma'$ и $\beta = \delta \gamma' \rho$. Тогава
    \begin{prooftree}
      \AxiomC{$\delta \gamma \rho \derive{0} \delta \gamma \rho$}
      \AxiomC{$\gamma \to_G \gamma'$}
      \RightLabel{\scriptsize{правило (2)}}
      \UnaryInfC{$\gamma \derive{1} \gamma'$}
      \RightLabel{\scriptsize{(\Proposition{unrestricted-grammar:context-general-step})}}
      \BinaryInfC{$\underbrace{\delta \gamma \rho}_{\alpha} \derive{1} \underbrace{\delta \gamma' \rho}_{\beta}$}
    \end{prooftree}
    За посоката $(\rightarrow)$, индукция по $\abs{\alpha}$. Имаме два случая.
    \begin{itemize}
    \item
      Ако $\alpha \to_G \beta$ е правило в граматиката $G$, то всичко е ясно, защото тогава $\rho = \delta = \varepsilon$.
    \item
      В противен случай, достигнали сме до извода $\alpha \derive{1} \beta$ като сме приложили правилото:
      \begin{prooftree}
        \AxiomC{$\alpha_1 \derive{\ell_1} \beta_1$}
        \AxiomC{$\alpha_2 \derive{\ell_2} \beta_2$}
        \BinaryInfC{$\underbrace{\alpha_1\alpha_2}_{\alpha} \derive{\ell_1+\ell_2} \underbrace{\beta_1\beta_2}_{\beta}$,}
      \end{prooftree}
    \end{itemize}
    където $\ell_1 + \ell_2 = 1$.
    Без ограничение на общността, нека $\ell_1 = 1$ и $\ell_2 = 0$. Понеже $\abs{\alpha_1} < \abs{\alpha}$, можем да приложим \IndHyp
    за $\alpha_1$. Получаваме, че $\alpha_1 \derive{}_G \beta_1$. Понеже $\ell_2 = 0$, то $\alpha_2 = \beta_2$.
    Заключваме, че $\underbrace{\alpha_1\alpha_2}_{\alpha} \derive{}_G \underbrace{\beta_1\beta_2}_{\beta}$.
  \end{hint}
\end{extra}


%%% Local Variables:
%%% mode: latex
%%% TeX-master: "../eai"
%%% End:
