\section{Неограничени граматики}\label{sect:unrestricted-grammar}
\index{граматика!неограничена}

\mynote{На англ. {\em unrestricted grammar}. Това са тип 0 граматиките в йерархията на Чомски \cite[стр. 220]{hopcroft1}.}
{\bf Неограничена граматика} e наредена четворка от вида
\[G = (V, \Sigma, R, S),\]
където:
\begin{itemize}
\item
  $V$ е крайно множество от {\em променливи} (нетерминали);
\item
  $\Sigma$ е крайно множество от {\em букви} (терминали), като $\Sigma \cap V = \emptyset$;
\item
  \mynote{В \cite{hopcroft1} правилата се наричат {\em productions} или {\em production rules}.}
  $R \subseteq (V\cup\Sigma)^+ \times (V \cup \Sigma)^\star$ е крайно множество от {\em правила}.
  За по-добра яснота, обикновено правилата $(\alpha, \beta) \in R$ ще означаваме като 
  $\alpha \to_G \beta$. Когато е ясно за коя граматика говорим, ще пишем просто $\alpha \to \beta$.
\item
  $S \in V$ е началната променлива (нетерминал). 
\end{itemize}

\index{граматика!извод}
Удобно е да дефинираме извод на думата $\beta$ от думата $\alpha$ в граматиката $G$ за $\ell$ стъпки, което ще означаваме като $\alpha \derive{\ell}_G \beta$,
с индукция по броя на стъпките $\ell$ по следния начин:
\mynote{В правило (1), нямаме никакви ограничения за $\lambda$ и $\rho$. Те са произволни елементи на $(V\cup\Sigma)^\star$, което означава, че може и да са празните думи.}
\begin{important}
  \begin{figure}[H]
    \begin{subfigure}[b]{0.5\textwidth}
      \begin{prooftree}
        \AxiomC{$\alpha \in (V\cup\Sigma)^\star$}
        \RightLabel{\scriptsize{правило (0)}}
        \UnaryInfC{$\alpha \derive{0}_G \alpha$}
      \end{prooftree}
    \end{subfigure}
    ~
    \begin{subfigure}[b]{0.5\textwidth}
      \begin{prooftree}
        \AxiomC{$(\alpha,\beta)\in R$}
        \AxiomC{$\lambda\beta\rho \derive{\ell} \gamma$}
        \RightLabel{\scriptsize{правило (1)}}
        \BinaryInfC{$\lambda\alpha\rho \derive{\ell+1}_G \gamma$}
      \end{prooftree}
    \end{subfigure}
    \caption{Правила за извод в неограничена граматика}
  \end{figure}
\end{important}

\mynote{Обърнете внимание, че в тази дефиниция на извод не определяме реда, в който прилагаме правилата на граматиката. Също така, понякога ,за удобство, ще пишем просто $\derive{\ell}$ вместо $\derive{\ell}_G$,
  когато се знае за коя граматика говорим.

  С други думи, $\derive{\star}_G$ е рефлексивното и транзитивно затваряне на релацията $\derive{1}_G$.
}
Сега дефинираме релацията $\derive{\star}_G$ като
\[ \alpha \derive{\star}_G \beta\ \dff\ (\exists \ell\in\Nat)[\ \alpha \derive{\ell}_G \beta\ ].\]
Езикът, който се поражда от граматиката $G$ дефинираме по следния начин:
\[\L(G) \df \{\omega \in \Sigma^\star \mid S \derive{\star}_G \omega\}.\]

За да можем да работим по-удобно с релацията за извод в граматика, ще започнем с няколко основни свойства. 

\begin{proposition}\label{pr:unrestricted-grammar:padding}
  За проиволно естествено число $\ell$ имаме извода:
  \begin{prooftree}
    \AxiomC{$\alpha \derive{\ell} \beta$}
    \AxiomC{$\lambda,\rho \in (V\cup\Sigma)^\star$}
    \BinaryInfC{$\lambda \alpha \rho \derive{\ell} \lambda \beta \rho$}
  \end{prooftree}
\end{proposition}
\begin{proof}
  Индукция по $\ell$.
  \begin{itemize}
  \item
    $\ell = 0$. Директно следва от правило $(0)$, защото тогава $\alpha = \beta$.
  \item
    $\ell > 0$. Според правилата за извод имаме следния случай:
    \begin{prooftree}
      \AxiomC{$(\alpha',\alpha'') \in R$}
      \AxiomC{$\gamma\alpha''\delta \derive{\ell-1} \beta$}
      \RightLabel{\scriptsize{правило (1)}}
      \BinaryInfC{$\underbrace{\gamma \alpha' \delta}_{\alpha} \derive{\ell} \beta$}
    \end{prooftree}
    Сега като използваме \IndHyp получаваме следния извод:
    \begin{prooftree}
      \AxiomC{$(\alpha',\alpha'') \in R$}
      \AxiomC{$\gamma\alpha''\delta \derive{\ell-1} \beta$}
      \RightLabel{\scriptsize{\IndHyp}}
      \UnaryInfC{$\lambda\gamma\alpha''\delta\rho \derive{\ell-1} \lambda\beta\rho$}
      \RightLabel{\scriptsize{правило (1)}}
      \BinaryInfC{$\lambda\underbrace{\gamma \alpha' \delta}_{\alpha}\rho \derive{\ell} \lambda\beta\rho$}
    \end{prooftree}
  \end{itemize}
\end{proof}


\begin{proposition}\label{pr:unrestricted-grammar:concat2}
  За произволни $\ell_1$ и $\ell_2$ имаме извода:
  \begin{prooftree}
    \AxiomC{$\alpha_1 \derive{\ell_1} \beta_1$}
    \AxiomC{$\alpha_2 \derive{\ell_2} \beta_2$}
    \BinaryInfC{$\alpha_1\alpha_2 \derive{\ell_1+\ell_2} \beta_1\beta_2$}
  \end{prooftree}
\end{proposition}
\begin{proof}
  Индукция по $\ell_1$.
  \begin{itemize}
  \item
    Ако $\ell_1 = 0$, то $\alpha_1 = \beta_1$ и тогава прилагаме \Proposition{unrestricted-grammar:padding}.
  \item
    Ако $\ell_1 > 0$, то имаме следното:
    \begin{prooftree}
      \AxiomC{$(\alpha'_1,\alpha''_1) \in R$}
      \AxiomC{$\lambda'\alpha''_1\rho' \derive{\ell_1-1} \beta_1$}
      \RightLabel{\scriptsize{правило (1)}}
      \BinaryInfC{$\underbrace{\lambda' \alpha'_1 \rho'}_{\alpha_1} \derive{\ell_1} \beta_1$}
    \end{prooftree}
    Сега прилагаме \IndHyp и получаваме следния извод:
    \begin{prooftree}
      \AxiomC{$(\alpha'_1,\alpha''_1) \in R$}
      \AxiomC{$\lambda'\alpha''_1\rho' \derive{\ell_1-1} \beta_1$}
      \AxiomC{$\alpha_2 \derive{\ell_2} \beta_2$}
      \RightLabel{\scriptsize\IndHyp}
      \BinaryInfC{$\lambda' \alpha''_1 \rho' \alpha_2 \derive{\ell_1-1+\ell_2} \beta_1\beta_2$}
      \RightLabel{\scriptsize{правило (1)}}
      \BinaryInfC{$\underbrace{\lambda'\alpha'_1\rho'}_{\alpha_1}\alpha_2 \derive{\ell_1+\ell_2} \beta_1\beta_2$}
    \end{prooftree}
  \end{itemize}
  
\end{proof}



\begin{proposition}\label{pr:unrestricted-grammar:concat}
  За всяко $k$ е изпълнено, че:
  \begin{prooftree}
    \AxiomC{$\alpha_1 \derive{\ell_1} \beta_1$}
    \AxiomC{$\dots$}
    \AxiomC{$\alpha_k \derive{\ell_k} \beta_k$}
    \RightLabel{\scriptsize{$(\ell = \sum^k_{i=1} \ell_i)$}}
    \TrinaryInfC{$\alpha_1\cdots\alpha_k \derive{\ell} \beta_1\cdots\beta_k$}
  \end{prooftree}
\end{proposition}
\begin{hint}
  Индукция по $k$.
\end{hint}

% \begin{proposition}\label{pr:unrestricted-grammar:general-step}
%   За произволни естествени числа $\ell_1$ и $\ell_2$ е изпълнено, че:
%   \begin{prooftree}
%     \AxiomC{$\alpha \derive{\ell_1} \beta$}
%     \AxiomC{$\rho \beta \delta \derive{\ell_2} \gamma$}
%     \BinaryInfC{$\rho \alpha \delta \derive{\ell_1+\ell_2} \gamma$}
%   \end{prooftree}  
% \end{proposition}
% \begin{hint}
%   Пълна индукция по $(\ell_1,\abs{\alpha})$ с лексикографската наредба.
%   \begin{itemize}
%   \item
%     Ако $\ell_1 = 0$, то е тривиално, защото тогава $\alpha = \beta$.
%   \item
%     Ако $\ell_1 > 0$, то имаме два случая в зависимост от това кое е последното правило, което сме приложили за да получим $\alpha \derive{\ell_1} \beta$.
%     \begin{itemize}
%     \item
%       Първият случай е, ако сме приложили правило (1), т.е.
%       \begin{prooftree}
%         \AxiomC{$\alpha \to_G \delta$}
%         \AxiomC{$\delta \derive{\ell_1-1} \beta$}
%         \RightLabel{\scriptsize{(1)}}
%         \BinaryInfC{$\alpha \derive{\ell_1} \beta$}
%       \end{prooftree}
%       Тогава получаваме следния извод:
%       \begin{prooftree}
%         \AxiomC{$\alpha \to_G \delta$}
%         \AxiomC{$\delta \derive{\ell_1-1} \beta$}
%         \AxiomC{$\beta \derive{\ell_2} \gamma$}
%         \RightLabel{\scriptsize{\IndHyp}}
%         \BinaryInfC{$\delta \derive{\ell_1+\ell_2-1} \gamma$}
%         \RightLabel{\scriptsize{(1)}}
%         \BinaryInfC{$\alpha \derive{\ell_1+\ell_2} \gamma$}
%       \end{prooftree}
%     \item
%       Вторият случай е, ако сме приложили правило (2), т.е. 
%       \begin{prooftree}
%         \AxiomC{$\alpha_1 \derive{\ell'_1} \beta_1$}
%         \AxiomC{$\alpha_2 \derive{\ell''_1} \beta_2$}
%         \AxiomC{$\alpha_1,\alpha_2 \in (V\cup\Sigma)^+$}
%         \RightLabel{\scriptsize{(2)}}
%         \TrinaryInfC{$\underbrace{\alpha_1\alpha_2}_{\alpha} \derive{\ell'_1+\ell''_1} \underbrace{\beta_1\beta_2}_{\beta}$}
%       \end{prooftree}
%       % Тук със сигурност имаме, че $\abs{\alpha_1} < \abs{\alpha}$ и $\abs{\alpha_2} < \abs{\alpha}$,
%       % което ни позволява да приложим индукционното предположение, защото със сигурност знаем, че
%       % $(\ell'_1,\abs{\alpha_1}) <_{lex} (\ell_1,\abs{\alpha})$ и 
%       % $(\ell''_1,\abs{\alpha_2}) <_{lex} (\ell_1,\abs{\alpha})$.

%       Ако $\ell'_1 > 0$ и $\ell''_1 > 0$, то това означава, че получаваме следния извод:
%       \begin{prooftree}
%         \AxiomC{$\alpha_1 \derive{\ell'_1} \beta_1$}
%         \LeftLabel{\scriptsize{(2)}}
%         \AxiomC{}
%         \LeftLabel{\scriptsize{(0)}}
%         \UnaryInfC{$\alpha_2 \derive{0} \alpha_2$}
%         \LeftLabel{\scriptsize{(2)}}
%         \BinaryInfC{$\alpha \derive{\ell'_1} \beta_1\alpha_2$}
%         \AxiomC{}
%         \RightLabel{\scriptsize{(0)}}
%         \UnaryInfC{$\beta_1 \derive{0} \beta_1$}
%         \AxiomC{$\alpha_2 \derive{\ell''_1} \beta_2$}
%         \RightLabel{\scriptsize{(2)}}
%         \BinaryInfC{$\beta_1\alpha_2 \derive{\ell''_1} \beta$}
%         \AxiomC{$\beta \derive{\ell_2} \gamma$}
%         \RightLabel{\scriptsize{\IndHyp}}
%         \BinaryInfC{$\beta_1\alpha_2 \derive{\ell''_1+\ell_2} \gamma$}
%         \RightLabel{\scriptsize{\IndHyp}}
%         \BinaryInfC{$\alpha \derive{\ell_1+\ell_2} \gamma$}
%       \end{prooftree}

%       Остава да разгледаме случая, когато $\ell'_1 = 0$ или $\ell''_1 = 0$.
%       Нека, без ограничение на общността, да приемем, че $\ell'_1 = 0$.
%       Тогава $\alpha_1 = \beta_1$ и $\ell''_1 = \ell_1$, но пак можем да използваме индукционното предположение, защото $\abs{\alpha_2} < \abs{\alpha}$ и оттук
%       $(\ell''_1,\abs{\alpha_2}) <_{\text{lex}} (\ell_1,\abs{\alpha})$. Така получаваме, че
      
%     \end{itemize}
%   \end{itemize}
% \end{hint}

\begin{proposition}\label{pr:unrestricted-grammar:context-general-step}
  За произволни естествени числа $\ell_1$ и $\ell_2$ е изпълнено, че:
  \begin{prooftree}
    \AxiomC{$\alpha \derive{\ell_1} \lambda\beta\rho$}
    \AxiomC{$\beta \derive{\ell_2} \gamma$}
    \BinaryInfC{$\alpha \derive{\ell_1+\ell_2} \lambda\gamma\rho$}
  \end{prooftree}  
\end{proposition}
\begin{proof}
  Сега ще докажем твърдението с индукция по $\ell_1$.
  \begin{itemize}
  \item
    $\ell_1 = 0$. Тогава всичко е ясно, защото тогава $\alpha = \lambda\beta\rho$ и получаваме извода:
    \begin{prooftree}
      \AxiomC{$\beta \derive{\ell_2} \gamma$}
      \RightLabel{\scriptsize{(\Proposition{unrestricted-grammar:padding})}}
      \UnaryInfC{$\underbrace{\lambda\beta\rho}_{\alpha} \derive{\ell_2} \lambda\gamma\rho$}
    \end{prooftree}
  \item
    $\ell_1 > 0$. От правилата за извод следва, че имаме следната ситуация:
    \begin{prooftree}
      \AxiomC{$(\alpha',\alpha'') \in R$}
      \AxiomC{$\lambda'\alpha''\rho' \derive{\ell_1-1} \lambda\beta\rho$}
      \BinaryInfC{$\underbrace{\lambda'\alpha'\rho'}_{\alpha} \derive{\ell_1} \lambda\beta\rho$}
    \end{prooftree}

    Използвайки \IndHyp получаваме следния извод:
    \begin{prooftree}
      \AxiomC{$(\alpha',\alpha'') \in R$}
      \AxiomC{$\lambda'\alpha''\rho' \derive{\ell_1-1} \lambda\beta\rho$}
      \AxiomC{$\beta \derive{\ell_2} \gamma$}
      \RightLabel{\scriptsize{\IndHyp}}
      \BinaryInfC{$\lambda'\alpha''\rho' \derive{\ell_1+\ell_2-1} \lambda\gamma\rho$}
      \RightLabel{\scriptsize{правило (1)}}
      \BinaryInfC{$\underbrace{\lambda'\alpha'\rho'}_{\alpha} \derive{\ell_1+\ell_2} \lambda\gamma\rho$}
    \end{prooftree}
  \end{itemize}
\end{proof}

\begin{corollary}\label{cor:unrestricted-grammar:context-general-step}
  За произволни естествени числа $\ell_1$ и $\ell_2$ е изпълнено, че:
  \begin{prooftree}
    \AxiomC{$\alpha \derive{\ell_1} \beta$}
    \AxiomC{$\beta \derive{\ell_2} \gamma$}
    \BinaryInfC{$\alpha \derive{\ell_1+\ell_2} \gamma$}
  \end{prooftree}  
\end{corollary}

% \begin{extra}
%   \begin{remark}\label{rem:unrestricted-grammar:original-def}
%     В повечето учебници авторите дефинират релацията $\derive{\star}_G$ като
%     рефлексивното и транзитивното затваряне на релацията $\derive{}_G$, където
%     \[\alpha \derive{}_G \beta \dff \alpha = \lambda\alpha'\rho\ \&\ (\alpha',\alpha'') \in R\ \&\ \beta = \lambda\alpha''\rho\text{, за някои $\lambda, \rho \in (V\cup\Sigma)^\star$}.\]
%     % Както се вижда от следващото твърдение, релацията $\derive{1}_G$, която дефинирахме по-горе, е точно релацията $\derive{}_G$.
%   \end{remark}
  
  % Както отбелязахме в \Remark{unrestricted-grammar:original-def}, в повечето учебници дават следната дефиниция на релацията $\derive{1}$, която тук ще формулираме като твърдение, което следва от нашите правила за извод.
  % \begin{problem}\label{prob:grammar:alternative-def}
  %   Докажете, че $\alpha \derive{1}_G \beta$ точно тогава, когато $\alpha \derive{}_G \beta$.
  % \end{problem}

  % \begin{problem}
  %   Докажете, че $\derive{\star}_G$ е транзитивното и рефлексивно затваряне на релацията $\derive{}_G$.
  % \end{problem}

  % \begin{hint}
  %   За посоката $(\leftarrow)$, нека $\alpha \derive{}_G \beta$, т.е. $\alpha = \lambda \gamma \rho$, $(\gamma,\gamma') \in R$ и $\beta = \lambda \gamma' \rho$. Тогава
  %   \begin{prooftree}
  %     \AxiomC{$(\gamma,\gamma') \in R$}
  %     \AxiomC{}
  %     \RightLabel{\scriptsize{правило (0)}}
  %     \UnaryInfC{$\delta \gamma \rho \derive{0} \delta \gamma \rho$}
  %     \RightLabel{\scriptsize{правило (1)}}
  %     \BinaryInfC{$\underbrace{\delta \gamma \rho}_{\alpha} \derive{1} \underbrace{\delta \gamma' \rho}_{\beta}$}
  %   \end{prooftree}
  %   За посоката $(\rightarrow)$, нека $\alpha \derive{1}_G \beta$. Тогава имаме следния извод:
  %   \begin{prooftree}
  %     \AxiomC{$(\alpha',\alpha'') \in R$}
  %     \AxiomC{$\lambda\alpha''\rho \derive{0} \lambda\alpha''\rho$}
  %     \BinaryInfC{$\underbrace{\lambda\alpha'\rho}_{\alpha} \derive{1} \underbrace{\lambda\alpha''\rho}_{\beta}$}
  %   \end{prooftree}
  % \end{hint}
% \end{extra}


%%% Local Variables:
%%% mode: latex
%%% TeX-master: "../eai"
%%% End:
