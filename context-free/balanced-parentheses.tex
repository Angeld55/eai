\subsection{Балансирани скоби}

Нека $\alpha$ е дума над азбука, която включва буквите $\texttt{[}$ и $\texttt{]}$. 
Ще казваме, че $\alpha$ е {\bf балансирана}, ако са изпълнени свойствата:
\begin{itemize}
\item 
  $\texttt{left}(\alpha) = \texttt{right}(\alpha)$;
\item
  За всеки префикс $\gamma$ на $\alpha$,
  $\texttt{left}(\gamma) \geq \texttt{right}(\gamma)$.
\end{itemize}

\begin{framed}
  \begin{problem}
    Докажете, че езикът 
    \[L = \{\ \alpha \in \{\texttt{[},\texttt{]}\}^\star \mid \alpha\text{ е балансирана дума}\ \}\]
    е безконтекстен.
  \end{problem}  
\end{framed}
\begin{hint}
  \marginpar{\cite[стр. 135]{kozen}}
  \marginpar{\writedown Докажете, че езикът $L$ не е регулярен! }
  Да разгледаме граматиката $G$ с правила
  \[S \to \texttt{[}S\texttt{]}\ |\ SS\ |\ \varepsilon.\]
  Ще докажем, че $L = \L(G)$.
  
  Първо ще докажем включването $\L(G) \subseteq L$.
  Да разгледаме \[M \df \{\ \alpha \in \{\texttt{[},\texttt{]}, S\}^\star \mid \alpha\text{ е балансирана}\ \}.\]
  
  Нека $S \to^\star_G \alpha$. Ще докажем с индукция по дължината $\ell$ на извода на $\alpha$ от $S$,
  че $\alpha \in M$. Случаят, когато $\ell = 0$ е очевиден.
  Нека $S \to^{\ell}_G \beta \to^1_G \alpha$.
  От {\bf И.П.} имаме, че $\beta \in M$, т.е. $\beta$ е балансирана.

  Лесно се съобразява, че във всички случаи за думите $\beta$ и $\alpha$ имаме следното:
  \begin{center}
    \begin{tabular}{| c | c | c |}
      \hline
      $\text{от И.П. }$ & $\text{правило на }G$ & $\text{извод}$ \\ \hline
      $\beta \in M$ & $S \rightarrow \texttt{[}S\texttt{]}$ & $\alpha \in M$ \\ \hline
      $\beta \in M$ & $S \rightarrow SS$ & $\alpha \in M$ \\ \hline
      $\beta \in M$ & $S \rightarrow \varepsilon$ & $\alpha \in M$ \\ \hline
    \end{tabular}
  \end{center}

  За включването $L \subseteq \L(G)$, нека $\alpha \in L$.
  Ще докажем с индукция по дължината на думата, че $\alpha \in \L(G)$.
  Ясно е, че във всички нетривиални случаи можем да запишем думата $\alpha$ като $\alpha = \texttt{[}\beta\texttt{]}$.
  Проблемът е, че в общия случай не е ясно дали можем да приложим индукционното предположение за $\beta$,
  защото е възможно $\beta \not\in L$. Например, $\alpha = \texttt{[][]}$.
  Тогава $\beta = \texttt{][} \not \in L$.
  Поради тази причина, трябва да сме по-внимателни и да разгледаме два случая.
  \begin{itemize}
  \item 
    \marginpar{\comment т.е. $\beta \neq \varepsilon$ и $\beta \neq \alpha$}
    Нека $\alpha$ има {\em същински} префикс $\beta \in L$.
    Понеже $\alpha \in L$, лесно се съобразява, че $\alpha = \beta\gamma$ и $\gamma \in L$.
    Сега можем да приложим {\bf И.П.} за $\beta$ и $\gamma$ и да получим, че 
    $\beta \in \L(G)$ и $\gamma \in \L(G)$, т.е.
    $S \to^\star_G \beta$ и $S \to^\star_G \gamma$.
    Понеже имаме правило $S \to_G SS$, то е ясно, че $\alpha \in \L(G)$.
  \item
    Нека $\alpha$ да няма същински префикс $\gamma \in L$.
    Ясно е, че тогава $\alpha = \texttt{[}\beta\texttt{]}$, за някое $\beta$
    и $\texttt{left}(\beta) = \texttt{right}(\beta)$.
    Да видим защо $\beta \in L$.
    
    Ако $\beta \in L$, то ще можем да приложим {\bf И.П.} за $\beta$ и ще сме готови.
    За всеки префикс $\gamma$ на $\beta$ имаме, че $\texttt{[}\gamma$ е префикс на $\alpha$,
    и понеже $\alpha \in L$, то $\texttt{left}(\texttt{[}\gamma) \geq \texttt{right}(\texttt{[}\gamma)$.
    Възможно ли е $\texttt{left}(\gamma) < \texttt{right}(\gamma)$ ?
    Това може да се случи единствено ако $\texttt{left}(\texttt{[}\gamma) = \texttt{right}(\texttt{[}\gamma)$.
    Но тогава $\texttt{[}\gamma$ е същински префикс на $\alpha$, за който $\texttt{[}\gamma \in L$,
    което противоречи на случая, който разглеждаме.
    Това означава, че за произволен префикс $\gamma$ на $\beta$,
    $\texttt{left}(\gamma) \geq \texttt{right}(\gamma)$ и оттук $\beta \in L$ и можем да приложим {\bf И.П.}
    Тогава $S \to^\star_G \beta$ и чрез правилото $S \to_G \texttt{[}S\texttt{]}$
    получаваме, че $\alpha \in \L(G)$.    
  \end{itemize}
\end{hint}

%%% Local Variables:
%%% mode: latex
%%% TeX-master: "../eai"
%%% End:
