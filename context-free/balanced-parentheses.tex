\subsection{Балансирани скоби}

\mynote{Например, думата $aabaabbb$ е балансирана, докато $abbaaab$ не е балансирана. Практическият смисъл на тази задача е, че можем да напишем граматика,
  която да разпознава, че за всяка отваряща скоба $\texttt{\{}$, която прочете, по-късно ще прочетем и съответната затваряща скоба $\texttt{\}}$.}
Нека $\alpha$ е дума над азбука, която включва буквите $a$ и $b$. 
Ще казваме, че $\alpha$ е {\bf балансирана}, ако са изпълнени свойствата:
\begin{itemize}
\item 
  $\texttt{left}(\alpha) = \texttt{right}(\alpha)$;
\item
  За всеки префикс $\gamma$ на $\alpha$,
  $\texttt{left}(\gamma) \geq \texttt{right}(\gamma)$.
\end{itemize}

\begin{framed}
  \begin{problem}
    \mynote{\writedown Докажете, че езикът $L$ не е регулярен! }
    Докажете, че езикът 
    \[L = \{\ \alpha \in \{a,b\}^\star \mid \alpha\text{ е балансирана дума}\ \}\]
    е безконтекстен.
  \end{problem}  
\end{framed}
\begin{hint}
  \mynote{\cite[стр. 135]{kozen}}
  Да разгледаме граматиката $G$ с правила
  \[S \to aSb\ |\ SS\ |\ \varepsilon.\]
  Ще докажем, че $L = \L(G)$.
  Имаме, че
  \begin{align*}
    \L^0_G(S) = & \emptyset\\
    \L^{\ell+1}_G(S) = & \{a\} \cdot \L^\ell_G(S) \cdot \{b\}\ \cup\\
                & \L^\ell_G(S) \cdot \L^\ell_G(S)\ \cup \\
                & \{\varepsilon\}.
  \end{align*}
  Първо да разгледаме коректност на граматиката, т.е. трябва да докажем, че за всяко $\ell$ е изпълнено следното:
  \begin{equation}
    \label{eq:balanced-correctness}
    \L^\ell_G(S) \subseteq \{\ \alpha \in \{a,b\}^\star \mid \alpha\text{ е балансирана дума}\ \}.
  \end{equation}
  Твърдението е очевидно за $\ell = 0$. Да приемем, че \Property{eq:balanced-correctness} е изпълнено за някое $\ell$.
  Ще докажем \Property{eq:balanced-correctness} за $\ell+1$.
  Да разгледаме произволна дума $\omega \in \L^{\ell+1}_G(S)$. Имаме три случая.
  \begin{itemize}
  \item
    Нека $\omega = \varepsilon$. Тогава е ясно, че $\omega$ е балансирана дума.
  \item
    Нека $\omega \in \{a\} \cdot \L^\ell_G(S) \cdot \{b\}$. Тогава $\omega = a \omega_1 b$ и от \IndHyp имаме, че
    $\omega_1$ е балансирана.
    Лесно се съобразява, че $\omega$ също е балансирана.
  \item
    Нека $\omega \in \L^\ell_G(S) \cdot \L^\ell_G(S)$. Тогава $\omega = \omega_1 \omega_2$, такива че $\omega_1,\omega_2 \in \L^\ell_G(S)$.
    От \IndHyp имаме, че $\omega_1$ и $\omega_2$ са балансирани.
    Лесно се съобразява, че $\omega$ също е балансирана.
  \end{itemize}
  Така доказахме, че
  \[\L_G(S)  = \bigcup_\ell \L^\ell_G(S) \subseteq \{\ \alpha \in \{a,b\}^\star \mid \alpha\text{ е балансирана дума}\ \}.\]

  Сега ще докажем пълнота на граматиката, т.е. следното свойство:
  \begin{equation}
    \label{eq:balanced-completeness}
    \{\ \alpha \in \{a,b\}^\star \mid \alpha\text{ е балансирана дума}\ \} \subseteq \L_G(S).
  \end{equation}
  Това ще направим с \emph{пълна} индукция по дължината на думите.
  Да разгледаме произволна балансирана дума $\omega$. Имаме няколко случая, които трябва да разгледаме.
  \begin{itemize}
  \item
    Ако $\omega = \varepsilon$, то е ясно, че $\omega \in \L_G(S)$.
  \item
    Нека сега $\omega \neq \varepsilon$. Ясно е, че със сигурност $\omega = a \omega_1 b$.
    Проблемът е, че в общия случай не е ясно дали можем да приложим \IndHyp за $\omega_1$,
    защото е възможно $\omega_1$ да не е балансирана. Например, ако $\omega = abab$, то $\omega_1 = ba$ не е балансирана.
    Поради тази причина, трябва да сме по-внимателни и да разгледаме два допълнителни случая.
    \begin{itemize}
    \item 
      \mynote{Тук $\omega_1 \neq \varepsilon$ и $\omega_1 \neq \alpha$.}
      Нека $\omega$ има {\em същински} префикс $\omega_1$, който да е балансирана дума.
      Понеже $\omega$ е балансирана дума, лесно се съобразява, че $\omega = \omega_1\omega_2$, $\omega_2$ също е балансирана дума.
      Сега можем да приложим \IndHyp за $\omega_1$ и $\omega_2$ и да получим, че 
      $\omega_1 \in \L_G(S)$ и $\omega_2 \in \L_G(S)$.
      Ясно е, че $\omega \in \L_G(S)$.
    \item
      Нека $\omega$ да няма същински префикс $\omega_1$, който е балансирана дума.
      Ясно е, че тогава $\omega = a\beta b$, за някое $\beta$. Да видим защо $\beta$ е балансирана дума.
      Ако $\beta$ е балансирана, то ще можем да приложим \IndHyp за $\beta$ и ще сме готови.
      
      За всеки префикс $\gamma$ на $\beta$ имаме, че $a\gamma$ е префикс на $\alpha$,
      и понеже $\alpha$ е балансирана, то $\texttt{left}(a\gamma) \geq \texttt{right}(a\gamma)$.
      Възможно ли е $\texttt{left}(\gamma) < \texttt{right}(\gamma)$ ?
      Това може да се случи единствено ако $\texttt{left}(a\gamma) = \texttt{right}(a\gamma)$.
      Но тогава $a\gamma$ е същински префикс на $\alpha$, за който $a\gamma$ е балансирана дума,
      което противоречи на случая, който разглеждаме.
      Това означава, че за произволен префикс $\gamma$ на $\beta$,
      $\texttt{left}(\gamma) \geq \texttt{right}(\gamma)$ и оттук $\beta$ е балансирана дума и можем да приложим \IndHyp.
      Тогава $\beta \in \L_G(S)$ и следователно $\alpha \in \{a\} \cdot \L_G(S) \cdot \{b\} \subseteq \L_G(S)$.
    \end{itemize}
  \end{itemize}
  Така доказахме \Property{eq:balanced-completeness}, което ни дава пълнота на граматиката спрямо езика.
\end{hint}

%%% Local Variables:
%%% mode: latex
%%% TeX-master: "../eai"
%%% End:
