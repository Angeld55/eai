\subsubsection*{Премахване на дългите правила}

Едно правило се нарича дълго, ако е от вида $A \to \beta$, където $|\beta| \geq 3$.
Да разгледаме едно дълго правило в граматиката от вида $A \to X_1X_2\cdots X_k$, където $k \geq 3$.
\marginpar{Новата граматика ще има дължина $\mathcal{O}(|G|)$.}
За да получим еквивалентна граматика без това дълго правило,
добавяме нови променливи $A_1,\dots, A_{k-2}$, и правила
\[A \to X_1A_1,\ A_1 \to X_2A_2, \dots,\ A_{k-2} \to X_{k-1}X_k.\]

\begin{problem}
  Нека е дадена граматиката  $G = \pair{\{S,A,B,C\}, \{a,b\}, S, R}$.
  Използвайте обща конструкция, за да премахнете ,,дългите'' правила от $ G$ като при това получите 
  безконтестна граматика $G_1$ с език $\L(G) = \L(G_1)$, където правилата на граматиката са:
  % \begin{enumerate}[a)]
  % \item
  %   \begin{align*}
  %     & S \to \varepsilon\ |\ ab\ |\ aAba\\
  %     & A\to aBCb\\
  %     & B\to bbb\\
  %     & C\to aC\ |\ aCaC;
  %   \end{align*}
  % \item
    \begin{align*}
      & S\to CC\ |\ b\\
      & A\to BSB\ |\ a\\
      & B\to ba\ |\ BC\\
      & C\to BaSA\ |\ a\ |\ b.
    \end{align*}
  % \end{enumerate}
\end{problem}

%%% Local Variables:
%%% mode: latex
%%% TeX-master: "../eai"
%%% End:
