\subsection*{Сечение с регулярен език}
От \Proposition{context-free:pumping:non-closure} знаем, че безконтекстните езици не
са затворени относно операцията сечение, т.е. възможно е $L_1$ и $L_2$ да са безконтекстни
езици, но $L_1 \cap L_2$ да не е безконтекстен.
Оказва се обаче, че безконтекстните езици са затворени относно сечение с регулярен език.

\begin{important}
  \begin{theorem}\label{th:intersection-context-reg}
    Нека $L$ e безконтекстен език и $R$ е регулярен език.
    Тогава тяхното сечение $L \cap R$ е безконтекстен език.
  \end{theorem}
\end{important}
\begin{hint}
  \mynote{Тук адаптираме доказателството от \cite[стр. 144]{papadimitriou}.}
  Нека имаме стеков автомат
  \[P = \pair{Q',\Sigma,\Gamma,\sharp, \Delta', \qstart', \qaccept'}, \text{ където } \L(P) = L,\]
  и детерминиран краен автомат 
  \[\A = \pair{Q'', \Sigma, \qstart'', \delta'', F''}, \text{ където } \L(\A) = R.\]
  \mynote{Сравнете с конструкцията от \Proposition{automata-cap}.}
  Ще определим нов стеков автомат $\M = \PDA$, където:
  \begin{itemize}
  \item 
    $Q \df Q' \times Q''$;
  \item
    $\qstart \df \pair{\qstart',\qstart''}$;
  \item
    $F \df \{\qaccept'\} \times F''$;
  \item 
    Функцията на преходите $\Delta$ е дефинирана както следва:
    \begin{itemize}
    \item 
      \mynote{Симулираме едновременно изчислението и на двата автомата.}
      Ако $(r_1,\gamma) \in \Delta'(q_1, a, Y)$, то
      \[ \Delta(\pair{q_1,q_2},a,Y) \ni (\pair{r_1,\delta''(q_2,a)}, \gamma).\]
    \item
      \mynote{Нищо не четем от входната дума, следователно правим празен ход на $\A$}
      Ако $(r_1,\gamma) \in \Delta'(q_1,\varepsilon,Y)$ и всяко $q_2 \in Q''$, то
      \[ \Delta(\pair{q_1,q_2},\varepsilon,Y) \ni (\pair{r_1,q_2}, \gamma).\]
    \item
      \mynote{\writedown Докажете, че $\L(\M) = \L(P) \cap \L(\A)$ !}
      $\Delta$ не съдържа други преходи;
    \end{itemize}
  \end{itemize}

  \begin{itemize}
  \item
    Докажете, че ако $(\pair{q_1,q_2},\alpha,\gamma) \vdash^\star_\M (\pair{p_1,p_2},\varepsilon,\varepsilon)$, то
    \[(q_1,\alpha,\gamma) \vdash^\star_P (p_1,\varepsilon,\varepsilon)\text{ и }(q_2,\alpha) \vdash^\star_\A (p_2,\varepsilon).\]
  \item
    Докажете, че ако $(q_1,\alpha,\gamma) \vdash^\star_P (p_1,\varepsilon,\varepsilon)$ и $(q_2,\alpha) \vdash^\star_\A (p_2,\varepsilon)$, то
    \[(\pair{q_1,q_2},\alpha,\gamma) \vdash^\star_\M (\pair{p_1,p_2},\varepsilon,\varepsilon).\]
  \end{itemize}
  
\end{hint}

\begin{extra}
  \Theorem{intersection-context-reg} е удобна, когато искаме да докажем, че даден език не е безконтекстен.
С нейна помощ можем да сведем езика до друг, за който вече знаем, че не е безконтекстен.

  \begin{example}
    Да разгледаме езика $L = \{\omega \in \{a,b,c\}^\star \mid \card{\omega}{a} = \card{\omega}{b} = \card{\omega}{c}\}$.
    Да допуснем, че $L$ е безконтекстен език. Тогава, според \Theorem{intersection-context-reg}, $L^\prime = L \cap \L(a^\star b^\star c^\star)$ също е безконтекстен език.
    Но $L^\prime = \{a^nb^nc^n \mid n \in \Nat\}$, за който знаем от \Example{context-free:pumping:anbncn}, че {\em не} е безконтекстен.
    Достигнахме до противоречие. Следователно, $L$ не е безконтекстен език.
  \end{example}
\end{extra}
