\section{Синтактични дървета}

\mynote{ На английски се нарича parse tree, syntax tree, derivation tree. Обикновено, например в \cite[стр. 123]{papadimitriou},  дават рекурсивна дефиниция на синтактично дърво.}

\begin{itemize}
% \item
%   За фиксирано $b \in \Nat$, ще разглеждаме думи $\alpha$ и $\beta$ над азбуката $\{0,1,\dots,b-1\}$.
% \item
%   С $\alpha \preceq \beta$ ще означаваме, че $\alpha$ е префикс на $\beta$, а с $\alpha \prec \beta$,
%   че $\alpha$ е \emph{същински} префикс на $\beta$, т.е. $\alpha \preceq \beta\ \&\ \alpha \neq \beta$.
% \item
%   \index{наредба!лексикографска}
%   Ще казваме, че $\alpha$ е лексикографски по-малка от $\beta$, което ще означаваме като $\alpha <_{\texttt{lex}} \beta$, ако
%   \[(\exists i < \min\{|\alpha|,|\beta|\})[\ \alpha\slice{:i} = \beta\slice{:i}\ \&\ \alpha\slice{i} < \beta\slice{i}\ ].\]
% \item
%   \index{дърво}
%   \mynote{Всеки възел в дървото еднозначно се определя от пътя от възела до корена.}
%   Непразното множество $T \subseteq \{0,1,\dots,b-1\}^\star$ се нарича {\bf дърво},
%   ако $T$ е затворено относно префикси, т.е. $\texttt{Pref}(T) = T$.
%   С други думи,
%   \[(\forall \alpha)(\forall \beta)[\ \alpha \in T\ \&\ \beta \preceq \alpha\ \implies\ \beta \in T\ ].\]
%   \mynote{При нас всички дървета са крайно разклонени.}
% \item
%   Нека да въведем следните означения:
%   \begin{align*}
%     & \high(T) \df \max\{\ \abs{\alpha}\ \mid\ \alpha \in T\ \} & \comment\text{височина}\\
%     & \successor_T(\alpha) \df \{ \alpha i \mid \alpha i \in T\ \&\ i < b\} & \comment\text{наследниците на }\alpha\\
%     & T_\alpha \df \alpha^{-1}(T) & \comment\text{поддървото на }\alpha\\
%     & \leaves(T) \df \{ \alpha \in T \mid \successor_T(\alpha) = \emptyset \}. & \comment\text{листата на }T
%   \end{align*}

%   \begin{figure}[H]
%     \centering
%     \begin{tikzpicture}
%       \coordinate (A) at (0,0);
%       \coordinate (B) at (-2,-3);
%       \coordinate (C) at (2,-3);
%       \coordinate (D) at (0,-1.5);
%       \coordinate (E) at (-1,-3);
%       \coordinate (F) at (1,-3);
%       \draw (A) -- node[above left]{$T$} (B) -- node[below]{$\alpha^{-1}(T)$}(C) -- (A);
%       \draw (D) -- (E);
%       \draw (D) -- (F);
%       \draw [photon] (A) -- node[left]{$\alpha$} (D);
%     \end{tikzpicture}
%     \caption{$\alpha^{-1}(T)$ е поддърво на $T$.}
%   \end{figure}  
\item
  Нека фиксираме граматиката $G = (\Sigma,V,S,R)$ и
  \[b \df \max\{\abs{\gamma} \mid A \to_G \gamma\text{ е правило в }G\}.\]
\item
  \index{синтактично дърво}
  Двойката $P = (T,\lambda)$ се нарича {\bf дърво на извод} съвместимо с $G$, ако са изпълнени свойствата:
  \begin{itemize}
  \item
    $T$ е крайно \emph{оптимално} дърво и $T \subseteq \{0,1,\dots,b-1\}^\star$.
  \item
    Имаме функция $\lambda: T \to V \cup \Sigma \cup \{\varepsilon\}$.
    Нека положим $X_\alpha \df \lambda(\alpha)$.
  \item
    \mynote{С други думи,
      \[\lambda(\alpha)\to_G\lambda(\alpha 0)\cdots\lambda(\alpha k).\] Това свойство е ключово, защото то ни казва, че дървото е съвместимо с правилата на граматиката $G$.}
    Ако $\alpha \in T$ и $|\successor_T(\alpha)| = k+1$, за някое $k \in \Nat$, то $X_\alpha \in V$,
    като имаме и 
    \[X_\alpha \to_G X_{\alpha 0} X_{\alpha 1} \cdots X_{\alpha k}.\]
  \end{itemize}
\item
  За дървото на извод $P$, нека $\texttt{root}(P) \df X_\varepsilon$.
\item
  Нека $\alpha_0, \alpha_1,\dots,\alpha_k$ са всички думи от множеството $\leaves(T)$
  подредени във възходящ ред относно лексикографската наредба. Тогава 
  \[\texttt{yield}(P) \df X_{\alpha_0} X_{\alpha_1}\cdots X_{\alpha_k}.\]
% \item
%   Нека $P = (T,\lambda)$ е дърво на извод съвместимо с граматиката $G$.
%   За всяко $\alpha \in T$, дефинираме $\lambda_\alpha:T_\alpha \to V \cup \Sigma \cup \{\varepsilon\}$ като
%   \[\lambda_\alpha(\beta) \df \lambda(\alpha \cdot \beta).\]
% \item
%   Нека $P = (T,\lambda)$ и $\alpha \in T$. Тогава
%   \[P_\alpha \df (T_\alpha, \lambda_\alpha).\]
\end{itemize}


\begin{extra}
\begin{example}
  Да разгледаме граматиката $G$ зададена със следните правила:
  \begin{align*}
    & S \to aS\ |\ aSc\ |\ B\\
    & B \to bB\ |\ bBc\ |\ \varepsilon.
  \end{align*}
  \mynote{По-нататък ще докажем, че езикът на граматиката $G$ е $\{a^nb^kc^\ell \mid n+k\geq \ell\}$.}
  Да разгледаме дървото на извод $P = (T, \lambda)$, където:

  \begin{framed}
    \begin{figure}[H]
    \qtreecenterfalse
    \Tree [.$\varepsilon$ $0$ [.$1$ $10$ [.$11$ [.$110$ $1100$ [.$1101$ $11010$ ] $1102$ ] ] $12$ ] ]
    \hskip 0.4in
    $\stackrel{\lambda}{\Rightarrow}$
    \hskip 0.4in
    \Tree [.$S$ $a$ [.$S$ $a$ [.$S$ [.$B$ $b$ [.$B$ $\varepsilon$ ] $c$ ] ] $c$ ] ]
    \caption{Дърво на извод за думата $aabcc$.}      
    \end{figure}
  \end{framed}
  \begin{multicols}{2}
  Имаме, че:
  \begin{itemize}
  \item
    $\high(P) = 5$;
  \item
    $\leaves(P) = \{0, 10, 1100, 11010, 1102, 12\}$;
  \item
    $\texttt{yield}(P) = aab\varepsilon cc = aabcc$.
  \item
    $\successor_T(110) = \{1100, 1101, 1102\}$;
  \item
    $\lambda(\varepsilon) = S$;
  \item
    $\lambda(0) = a$, $\lambda(1) = S$;
  \item
    Ясно е, че $\underbrace{\lambda(\varepsilon)}_{S} \to_G \underbrace{\lambda(0)\lambda(1)}_{aS}$;
  \item
    $\lambda(10) = a$, $\lambda(11) = S$, $\lambda(12) = c$;
  \item
    Ясно е, че $\underbrace{\lambda(1)}_{S} \to_G \underbrace{\lambda(10)\lambda(11)\lambda(12)}_{aSc}$;
  \item
    $\lambda(110) = B$;
  \item
    Ясно е, че $\underbrace{\lambda(11)}_{S} \to_G \underbrace{\lambda(110)}_{B}$;
  \item
    $\lambda(1100) = b$, $\lambda(1101) = B$, $\lambda(1102) = c$;
  \item
    Ясно е, че
    \[\underbrace{\lambda(110)}_{B} \to_G \underbrace{\lambda(1100)\lambda(1101)\lambda(1102)}_{bBc};\]
  \item
    $\lambda(11010) = \varepsilon$;
  \item
    Ясно е, че $\underbrace{\lambda(1101)}_{B} \to_G \underbrace{\lambda(11010)}_{\varepsilon}$;
  \end{itemize}
  От всичко по-горе следва, че $P = (T,\lambda)$ е дърво на извод за думата $aabcc$ в граматиката $G$.
  \end{multicols}
\end{example}
\end{extra}




% \begin{problem}
%   Нека $P = (T,\lambda)$ е дърво на извод съвместимо с $G$.
%   Докажете, че ако $\alpha \preceq \beta$, то $\texttt{yield}(T_\beta)$ е инфикс на $\texttt{yield}(T_\alpha)$.
% \end{problem}
% \begin{hint}
%   Нека $\beta = \alpha \cdot \gamma$. Тогава имаме следното дърво:
  
%   \begin{figure}[H]
%     \centering
%     \begin{tikzpicture}
%       \coordinate (A) at (0,0);
%       \coordinate (B) at (-3,-4);
%       \coordinate (C) at (3,-4);
%       \coordinate (D) at (0,-1.5);
%       \coordinate (E) at (-2,-4);
%       \coordinate (F) at (2,-4);
%       \coordinate (G) at (0,-2.5);
%       \coordinate (H) at (-1,-4);
%       \coordinate (I) at (1,-4);
      
%       \draw (A) -- node[left]{$T$} (B) -- (C) -- (A);
      
%       \draw (D) -- node[left]{$T_\alpha$}(E);
%       \draw (D) -- (F);
      
%       \draw (G) -- node[left]{$T_\beta$}(H);
%       \draw (G) -- (I);
      
%       \draw [photon] (A) -- node[left]{$\alpha$} (D);
%       \draw [photon] (D) -- node[below left]{$\gamma$} (G);
%     \end{tikzpicture}
%     \caption{}
%   \end{figure}
% \end{hint}

% \begin{problem}
%   Нека $P = (T,\lambda)$ и $P' = (T',\lambda')$ са дървета на извод съвместими с граматиката $G$ и нека
%   имаме думи $\omega_1, \omega_2 \in \Sigma^\star$, за които
%   \mynote{Това означава, че съществува $\alpha \in T$, за което $\lambda(\alpha) = \texttt{root}(P') = \lambda'(\varepsilon)$.}
%   \[\texttt{yield}(P) = \omega_1 \cdot \texttt{root}(P') \cdot \omega_2.\]
%   Дефинираме $P'' = (T'',\lambda'')$ по следния начин:
%   \begin{itemize}
%   \item
%     \mynote{Ясно е, че $\alpha \in T$ и $\alpha \in \alpha\cdot T'$, но понеже $X_\alpha = X'_\varepsilon$, то нямаме проблем.}
%     $T'' \df T \cup \{\alpha\} \cdot T'$;
%   \item
%     Сега трябва да дефинираме функцията $\lambda'' : T'' \to V \cup \Sigma \cup \{\varepsilon\}$.
    
%     $\lambda''(\gamma) \df
%     \begin{cases}
%       \lambda(\gamma), & \text{ако }\gamma \in T\\
%       \lambda'(\beta), & \text{ако }\gamma = \alpha \cdot \beta\ \&\ \beta \in T'
%     \end{cases}$

%     \mynote{Тук трябва да сме внимателни, защото двата случая на дефиницията на $\lambda''$  се засичат за $\gamma = \alpha$.
%       Понеже $\alpha \in \texttt{leaves}(T)$ и $\lambda(\alpha) = \lambda'(\varepsilon)$, то функцията $\lambda''$ е коректно дефинирана.}
%   \end{itemize}
%   \index{дърво на извод!конкатенация}
%   Тогава $P''$ е дърво на извод съвместимо с граматиката $G$ и
%   \[\texttt{yield}(P'') = \omega_1 \cdot \texttt{yield}(P') \cdot \omega_2.\]
%   Нека в такъв случай да означаваме $P'' = P \odot P'$ и ще казваме, че $P''$ е конкатенацията на $P$ и $P'$.
  
%   \begin{figure}[H]
%     \begin{subfigure}[t]{0.5\textwidth}
%       \centering
%       \begin{tikzpicture}
%         \coordinate (A) at (0,0);
%         \coordinate (B) at (-2,-2.5);
%         \coordinate (C) at (2,-2.5);
%         \coordinate (D) at (0,-2.5);
%         \coordinate (E) at (-1,-3.5);
%         \coordinate (F) at (1,-3.5);
        
%         \draw (A) -- node[above left]{$T$} (B) -- (C) -- (A);
%         \draw (D) -- node[left]{$T'$} (E) -- (F) -- (D);
        
%         \draw [photon] (A) -- node[left]{$\alpha$} (D);
%       \end{tikzpicture}
%       \caption{Дървото $T''$}
%       \end{subfigure}
%       $\stackrel{\lambda}{\Rightarrow}$
%       \begin{subfigure}[t]{0.5\textwidth}
%         \centering
%         \begin{tikzpicture}
%           \node (A) at (0,0) {${\scriptstyle X_\varepsilon}$};
%           \coordinate (B) at (-2,-2.5);
%           \coordinate (C) at (2,-2.5);
%           \node (D) at (0,-2.5) {${\scriptstyle X_\alpha}$};
%           \coordinate (E) at (-1,-3.5);
%           \coordinate (F) at (1,-3.5);
          
%           \draw (A) -- node[above left]{${\scriptstyle P\ =\ }$} (B) -- node[below]{$\omega_1$}(D) -- node[below]{$\omega_2$}(C) -- (A);
%           \draw (D) -- node[left]{${\scriptstyle P'\ =\ }$}(E) -- node[below]{$\texttt{yield}(P')$}(F) -- (D);
          
%           % \draw [photon] (A) -- node[left]{$\alpha$} (D);
%         \end{tikzpicture}
%         \caption{$P'' = P \odot P'$}
%       \end{subfigure}
%       \caption{Конкатенация на дървета}
%     \end{figure}
%   \end{problem}
  
%   Сега да разгледаме частния случай, когато разгледаме $P$ вместо $P'$ в горните условия. Тогава имаме, че $X_\varepsilon = X_\alpha$.
%   Дефинираме $n$-тата степен на дървото $P$ по следния начин:
%   \begin{itemize}
% \item
%   $P^{(0)} \df (T_0,\lambda_0)$, където $T_0 = \{\varepsilon\}$ и $\lambda_0(\varepsilon) = \texttt{root}(P)$;
% \item
%   $P^{(n+1)} \df P^{(n)} \odot P$.
% \end{itemize}

% \begin{problem}
%   Докажете, че $P \odot (P \odot P) = (P \odot P) \odot P$.
% \end{problem}


% \begin{problem}
%   Докажете, че $P^{(n+k)} = P^{(n)} \odot P^{(k)}$.
% \end{problem}

% \begin{framed}
%   \begin{problem}
%     \label{prob:tree:iteration}
%     Нека $\texttt{yield}(P) = \omega_1 \cdot \texttt{root}(P) \cdot \omega_2$, за някои думи $\omega_1, \omega_2 \in \Sigma^\star$.
%     Докажете, че за всяко естествено число $i$ е изпълнено, че:
%     \[\texttt{yield}(P^{(i)}) = \omega^i_1 \cdot \texttt{root}(P) \cdot \omega^i_2.\]
%   \end{problem}
% \end{framed}
% \begin{hint}
%   Картинката за $i = 2$ изглежда така:

%   \begin{figure}[H]
%     \begin{subfigure}[t]{0.5\textwidth}
%       \centering
%       \begin{tikzpicture}
%         \coordinate (A) at (0,0);
%         \coordinate (B) at (-2,-2.5);
%         \coordinate (C) at (2,-2.5);
%         \coordinate (D) at (0,-2.5);
%         \coordinate (E) at (-2,-5);
%         \coordinate (F) at (2,-5);
%         \coordinate (G) at (0,-5);
        
%         \draw (A) -- node[above left]{$T$} (B) -- (C) -- (A);
%         \draw (D) -- node[left]{$T$} (E) -- (F) -- (D);
        
%         \draw [photon] (A) -- node[left]{$\alpha$} (D);
%         \draw [photon] (D) -- node[left]{$\alpha$} (G);
%       \end{tikzpicture}
%       \caption{}
%     \end{subfigure}
%     $\stackrel{\lambda}{\Rightarrow}$
%     \begin{subfigure}[t]{0.5\textwidth}
%       \centering
%       \begin{tikzpicture}
%         \node (A) at (0,0) {${\scriptstyle X_\varepsilon}$};
%         \coordinate (B) at (-2,-2.5);
%         \coordinate (C) at (2,-2.5);
%         \node (D) at (0,-2.5) {${\scriptstyle X_\varepsilon}$};
%         \coordinate (E) at (-2,-5);
%         \coordinate (F) at (2,-5);
%         \node (G) at (0,-5) {${\scriptstyle X_\varepsilon}$};
        
%         \draw (A) -- node[above left]{${\scriptstyle P\ =\ }$} (B) -- node[below]{$\omega_1$}(D) -- node[below]{$\omega_2$}(C) -- (A);
%         \draw (D) -- node[above left]{${\scriptstyle P\ =\ }$}(E) -- node[below]{$\omega_1$} (G) -- node[below]{$\omega_2$} (F) -- (D);
%       \end{tikzpicture}
%       \caption{$\texttt{yield}(P^{(2)}) = \omega^2_1 \cdot \texttt{root}(P) \cdot \omega^2_2$}
%     \end{subfigure}
%     \caption{Степенуване на $P$.}
%   \end{figure}
% \end{hint}

% \subsection{Операции върху синтактични дървета}

% \begin{problem}
%   Нека $P = (T,\lambda)$ е дърво на извод съвместимо с граматиката $G$ и нека разгледаме един път $\alpha$ в дървото $T$.
%   % Дефинираме $P \setminus P_\alpha = (T',\lambda')$ по следния начин:
%   Дефинираме $\uppercut{P}{\alpha} = (T',\lambda')$ по следния начин:
%   \begin{itemize}
%   \item
%     \mynote{Съобразете, че $\alpha \in \texttt{front}(T')$.}
%     $T' = T \setminus \{ \gamma \in T\mid \alpha \prec \gamma\}$, т.е. махаме от $T$ същинските разширения на $\alpha$
%   \item
%     $\lambda'(\gamma) = \lambda(\gamma)$ за $\gamma \in T'$.
%   \end{itemize}
%   Докажете, че $\uppercut{P}{\alpha}$ е дърво на извод съвместимо с граматиката $G$ и 
%   \[P = (\uppercut{P}{\alpha}) \odot (\lowercut{P}{\alpha}).\]
% \end{problem}
% \begin{hint}
%   Имаме следната картинка:

%   \begin{figure}[H]
%     \begin{subfigure}[t]{0.5\textwidth}
%       \centering
%       \begin{tikzpicture}
%         \coordinate (A) at (0,0);
%         \coordinate (B) at (-2,-2.5);
%         \coordinate (C) at (2,-2.5);
%         \coordinate (D) at (-0.9,-2.5);
%         \coordinate (E) at (0.9,-2.5);
%         \coordinate (F) at (0,-1.5);
        
%         \coordinate (G) at (0,-2.3);
%         \coordinate (H) at (-0.9,-3.3);
%         \coordinate (I) at (0.9,-3.3);
        
%         \draw (A) -- node[above left]{${\scriptstyle T'\ =\ }$}(B) -- (D) -- (F) -- (E) -- (C) -- (A);
%         \draw (G) -- node[left]{${\scriptstyle T_\alpha\ =\ }$} (H) -- (I) -- (G);
        
%         \draw [photon] (A) -- node[left]{$\alpha$} (F);
%       \end{tikzpicture}
%       \caption{$T_\alpha = \alpha^{-1}(T)$}
%     \end{subfigure}
%     $\stackrel{\lambda}{\Rightarrow}$
%     \begin{subfigure}[t]{0.5\textwidth}
%       \centering
%       \begin{tikzpicture}
%         \node (A) at (0,0) {${\scriptstyle X_\varepsilon}$};
%         \coordinate (B) at (-2,-2.5);
%         \coordinate (C) at (2,-2.5);
%         \coordinate (D) at (-0.9,-2.5);
%         \coordinate (E) at (0.9,-2.5);
%         \node (F) at (0,-1.5) {${\scriptstyle X_\alpha}$};
        
%         \node (G) at (0,-2.3) {${\scriptstyle X_\alpha}$};
%         \coordinate (H) at (-0.9,-3.3);
%         \coordinate (I) at (0.9,-3.3);
        
%         \draw (A) -- node[above left]{${\scriptstyle \uppercut{P}{\alpha}}\ =\ $}(B) -- (D) -- (F) -- (E) -- (C) -- (A);
%         \draw (G) -- node[left]{${\scriptstyle \lowercut{P}{\alpha}}\ =\ $} (H) -- (I) -- (G);
%       \end{tikzpicture}
%       \caption{$\texttt{yield}(\uppercut{P}{\alpha}) = \omega_1 \cdot \texttt{root}(\lowercut{P}{\alpha}) \cdot \omega_2$}
%     \end{subfigure}
%   \end{figure}
% \end{hint}



%%% Local Variables:
%%% mode: latex
%%% TeX-master: "../eai"
%%% End:
