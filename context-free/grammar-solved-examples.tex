\subsection{Примерни задачи}

\begin{extra}

Да напомним, че в Раздел~\ref{sect:regular:minimisation} дефинирахме език $\L_\A(q)$.
Сега ще дефинираме език $\L_G(A)$, за произволна променлива $A$ по следния начин:
\[\L_G(A) \df \{\alpha \in \Sigma^\star \mid A \yield{\star}\alpha\}.\]
Ясно е, че $\L(G) = \L_G(S)$.
Също така, да дефинираме апроксимациите $\L^\ell_G(A)$ на $\L_G(A)$ така:
\[\L^\ell_G(A) \df \{\alpha \in \Sigma^\star \mid A \yield{\leq \ell} \alpha\}.\]
Следните свойства са ясни:
\begin{itemize}
\item
  $\L^0_G(A) = \emptyset$;
\item
  $\L^\ell_G(A)$ е краен език за всяко $\ell$;
\item
  $\L^\ell_G(A) \subseteq \L^{\ell+1}_G(A)$ за всяко $\ell$;
\item
  $\L_G(A) = \bigcup_{\ell\geq 0}\L^\ell_G(A)$.  
\end{itemize}

Следната характеризация на езиците $\L^\ell_G(A)$ ще е удобна за нас, когато искаме да докажем,
че една безконтекстна граматика разпознава даден език.

\begin{proposition}\label{pr:grammar:yield-approximation}
  Нека $G$ е произволна безконтекстна граматика и $A$ е променлива в $G$.
  Тогава имаме следното:
  \begin{align*}
    \L^0_G(A) & = \emptyset\\
    \L^{\ell+1}_G(A) & = \bigcup\{\{\alpha_1\}\cdot \L^\ell_G(A_1) \cdots \{\alpha_n\} \cdot \L^\ell_G(A_n) \cdot \{\alpha_{n+1}\} \mid A \to_G \alpha_1A_1\cdots\alpha_nA_n\alpha_{n+1}\}.
  \end{align*}
\end{proposition}
\begin{proof}
  Пълна индукция по $\ell$. Твърдението очевидно е изпълнено за $\ell = 0$.
  Нека $\alpha \in \L^{\ell+1}_G(A)$, т.е. $A \yield{\leq \ell+1} \alpha$. Имаме следния извод:
  \begin{prooftree}
    \AxiomC{$A \to_G \alpha_1B_1\cdots\alpha_n B_n\alpha_{n+1}$}
    \AxiomC{$B_1 \yield{\leq\ell} \beta_1$}
    \AxiomC{$\cdots$}
    \AxiomC{$B_n \yield{\leq \ell} \beta_n$}
    \QuaternaryInfC{$A \yield{\leq\ell+1} \underbrace{\alpha_1\beta_1\cdots\alpha_n\beta_n\alpha_{n+1}}_{\alpha}$}
  \end{prooftree}
  Понеже $B_i \yield{\leq \ell} \beta_i$, то от \IndHyp имаме, че $\beta_i \in \L^\ell_G(B_i)$.
  Тогава 
  \[\alpha \in \bigcup\{\{\alpha_1\}\cdot \L^\ell_G(A_1) \cdots \{\alpha_n\} \cdot \L^\ell_G(A_n) \cdot \{\alpha_{n+1}\} \mid A \to_G \alpha_1A_1\cdots\alpha_nA_n\alpha_{n+1}\}\]

  Обратно, нека сега $\alpha = \alpha_1\beta_1\cdots\alpha_n\beta_n\alpha_{n+1}$, където $A \to_G \alpha_1B_1\cdots\alpha_nB_n\alpha_{n+1}$,
  $\beta_i \in \L^\ell_G(B_i)$ и $\alpha = \alpha_1\beta_1\cdots\alpha_n\beta_n\alpha_{n+1}$.
  Получаваме следния извод:
  \begin{prooftree}
    \AxiomC{$A \to_G \alpha_1B_1\cdots\alpha_nB_n\alpha_{n+1}$}
    \AxiomC{$\beta_1 \in \L^\ell_G(B_1)$}
    \RightLabel{\scriptsize{\IndHyp}}
    \UnaryInfC{$B_1 \yield{\leq \ell }\beta_1 $}
    \AxiomC{$\cdots$}
    \AxiomC{$\beta_n \in \L^\ell_G(B_n)$}
    \RightLabel{\scriptsize{\IndHyp}}
    \UnaryInfC{$B_n \yield{\leq \ell }\beta_n$}
    \QuaternaryInfC{$A \yield{\leq \ell+1} \underbrace{\alpha_1\beta_1\cdots\alpha_n\beta_n\alpha_{n+1}}_{\alpha}$}
  \end{prooftree}
\end{proof}

\begin{corollary}\label{cor:grammar:yield-approximation}
  Нека $G$ е произволна безконтекстна граматика и $A$ е променлива в $G$. Тогава
  \[\L_G(A) = \bigcup\{\{\alpha_1\}\cdot \L_G(A_1) \cdots \{\alpha_n\} \cdot \L_G(A_n) \cdot \{\alpha_{n+1}\} \mid A \to_G \alpha_1A_1\cdots\alpha_nA_n\alpha_{n+1}\}.\]
\end{corollary}

Като първи пример, нека да видим, че съществува безконтекстен език, който не е регулярен.
\mynote{Да напомним, че вече видяхме, че езикът $L = \{a^nb^n \mid n\in\Nat\}$ не е регулярен.}
\begin{example}\label{ex:grammar:anbn}
  Да разгледаме безконтекстната граматика $G$ зададена със следните правила:
  \begin{align*}
    & S \to aSb \mid \varepsilon.
  \end{align*}
  Да разгледаме първите няколко члена на редицата от езици $\L^\ell_G(S)$:
  \begin{align*}
    & \L^0_G(S) = \emptyset\\
    & \L^1_G(S) = \{a\} \cdot \emptyset \cdot \{b\} \cup \{\varepsilon\} = \{\varepsilon\}\\
    & \L^{2}_G(S) = \{a\} \cdot \{\varepsilon\} \cdot \{b\} \cup \{\varepsilon\} = \{\varepsilon, ab\}\\
    & \L^{3}_G(S) = \{a\} \cdot \{\varepsilon, ab\} \cdot \{b\} \cup \{\varepsilon\} = \{\varepsilon,ab,aabb\} = \{a^nb^n \mid n < 3\}\\
    & \vdots
  \end{align*}
  Лесно се съобразява, че:
  \[\L^\ell_G(S) = \{a^nb^n \mid n < \ell\}.\]
  Заключаваме, че:
  \[\L(G) = \L_G(S) = \bigcup_{\ell}\L^\ell_G(S) = \{a^nb^n \mid n\in\Nat\}.\]
\end{example}

\begin{example}
  Да разгледаме безконтекстната граматика $G$ зададена със следните правила:
  \begin{align*}
    & S \to aSc\ |\  B\\
    & B \to bBc\ |\ \varepsilon.
  \end{align*}
  Да видим защо $\L(G) = \{a^nb^kc^{n+k} \mid n,k\in\Nat\}$.
  Първо ще докажем \emph{коректност} на граматиката. Това означава, че $G$ не генерира думи извън езика, т.е.
  $\L_G(S) \subseteq \{a^nb^kc^{n+k} \mid n,k\in\Nat\}$. За да направим това обаче, трябва да знаем също така и $\L_G(B)$.
  Формално казано, трябва да докажем, че за всяко $\ell$ е изпълнено следното:
  \begin{align}
    \L^\ell_G(S) & \subseteq \{a^nb^kc^{n+k} \mid n,k\in\Nat\} \label{eq:nknk1}\\
    \L^\ell_G(B) & \subseteq \{b^kc^k \mid k \in \Nat\} \label{eq:nknk2}.
  \end{align}
  Това ще направим с индукция по $\ell$.
  Да напомним, че според \Proposition{grammar:yield-approximation} имаме следните връзки:
  \begin{align*}
    & \L^0_G(S) = \emptyset\\
    & \L^{\ell+1}_G(S) = \{a\} \cdot \L^\ell_G(S) \cdot \{c\} \cup \L^\ell_G(B)\\
    & \L^0_G(B) = \emptyset\\
    & \L^{\ell+1}_G(B) = \{b\} \cdot \L^\ell_G(B) \cdot \{c\} \cup \{\varepsilon\}.
  \end{align*}
  Очевидно е, че \Property{eq:nknk1} и \Property{eq:nknk2} са изпълнени за $\ell = 0$.
  Да примем, че имаме \Property{eq:nknk1} и \Property{eq:nknk2} за някое $\ell$.
  \mynote{Обърнете внимание, че не можем да докажем \Property{eq:nknk1} независимо от \Property{eq:nknk2}.}
  Ще докажем свойствата и за $\ell+1$.
  \begin{itemize}
  \item
    Първо, нека $\alpha \in \L^{\ell+1}_G(S)$. Имаме два случая.
    \begin{itemize}
    \item
      Нека $\alpha \in \{a\} \cdot \L^\ell_G(S) \cdot \{c\}$. От \IndHyp следва, че
      $\alpha = a^{n+1}b^kc^{n+k+1}$ за някои естествени числа $n$ и $k$.
      Тогава е ясно, че $\alpha \in \{a^nb^kc^{n+k} \mid n,k\in\Nat\}$.
    \item
      Нека $\alpha \in  \L^\ell_G(B)$. От \IndHyp следва, че
      $\alpha \in \{b^kc^k \mid k \in \Nat\} \subseteq \{a^nb^kc^{n+k} \mid n,k\in\Nat\}$.
    \end{itemize}
  \item
    Второ, нека $\alpha \in \L^{\ell+1}_G(B)$. Имаме два случая.
    \begin{itemize}
    \item
      Нека $\alpha \in \{b\} \cdot \L^\ell_G(B) \cdot \{c\}$. От \IndHyp следва, че
      $\alpha = b^{k+1}c^{k+1}$ за някое естествено число $k$.
      Тогава е ясно, че $\alpha \in \{b^{k}c^{k} \mid k\in\Nat\}$.
    \item
      Нека $\alpha = \varepsilon$. В този случай също е ясно, че $\alpha \in \{b^{k}c^{k} \mid k\in\Nat\}$.
    \end{itemize}
  \end{itemize}
  Оттук заключаваме, че $\L_G(S) \subseteq \{a^nb^kc^{n+k} \mid n,k\in\Nat\}$.
  
  Сега да разгледаме пълнота на граматиката, което означава, че
  всяка дума от езика се генерира от $G$. С други думи, $\{a^nb^kc^{n+k} \mid n,k\in\Nat\} \subseteq \L_G(S)$.
  Първо ще докажем, че
  \begin{equation}
    \label{eq:4}
      \{b^kc^k \mid k \in \Nat\} \subseteq \L_G(B).
    \end{equation}
    Това ще направим с \emph{пълна} индукция по дължината на думата.
    Да разгледаме произволна дума $\alpha \in \{b^kc^k \mid k \in \Nat\}$.
    \begin{itemize}
    \item
      Нека $|\alpha| = 0$, т.е. $\alpha = \varepsilon$.
      Ясно е, че $\alpha \in \L_G(B)$.
    \item
      Нека $|\alpha| > 0$, т.е. $\alpha = b^{k+1}c^{k+1}$.
      От \IndHyp за \Property{eq:4} имаме $\alpha \in \{b\} \cdot \L_G(B) \cdot \{c\} \subseteq \L_G(B)$.
    \end{itemize}
    Вече сме готови да докажем, че:
    \begin{equation}
      \label{eq:9}
      \{a^nb^kc^{n+k} \mid n,k \in \Nat\} \subseteq \L_G(S).
    \end{equation}
    Това включване пак ще докажем с \emph{пълна} индукция по дължината на думата.
    Да разгледаме произволна дума $\alpha \in L$. 
    \begin{itemize}
    \item
      Нека $|\alpha| = 0$, т.е. $\alpha = \varepsilon$.
      Ясно е, че $\alpha \in \L_G(B) \subseteq \L_G(S)$.
    \item
      Нека сега $|\alpha| > 0$, т.е. $\alpha = a^nb^kc^{n+k}$, където $n > 0$ или $k > 0$. Да разгледаме два случая.
      \begin{itemize}
      \item
        Нека $n = 0$. Тогава $\alpha = b^kc^k$ и $k > 0$. Тогава от \Property{eq:4}
        следва, че $\alpha \in \L_G(B) \subseteq \L_G(S)$.
      \item                   
        Нека $n > 0$. Тогава от \IndHyp за \Property{eq:9} имаме
        $\alpha \in \{a\} \cdot \L_G(S) \cdot \{c\} \subseteq \L_G(S)$.
      \end{itemize}
    \end{itemize}
    Доказахме коректност и пълнота на граматиката и следователно $\L(G) = \{a^nb^kc^{n+k} \mid n,k\in\Nat\}$.
  \end{example}
  
  
  \begin{example}
    Нека да видим защо езикът $L = \{a^mb^nc^k\mid m+n \geq k\}$ е безконтекстен.
    Да разгледаме граматиката $G$ с правила
    \begin{align*}
      S& \rightarrow aSc\ |\ aS\ |\ B\\
      B& \rightarrow bBc\ |\  bB\ |\ \varepsilon.
    \end{align*}
    От \Proposition{grammar:yield-approximation} имаме, че:
    \begin{align*}
      \L^0_G(S) & = \emptyset\\
      \L^{\ell+1}_G(S) & = \{a\} \cdot \L^\ell_G(S) \cdot \{c\} \cup \{a\}\cdot \L^\ell_G(S) \cup \L^\ell_G(B)\\
      \L^0_G(B) & = \emptyset\\
      \L^{\ell+1}_G(B) & = \{b\} \cdot \L^\ell_G(B) \cdot \{c\} \cup \{b\} \cdot \L^\ell_G(B) \cup \{\varepsilon\}.
    \end{align*}
    Да предположим, че за произволно естествено число $\ell$ е изпълнено следното:
    \mynote{Тези две свойства ще бъдат нашето \IndHyp. Очевидно е, че те са изпълнени за $\ell = 0$.}
    \begin{align}
      \L^\ell_G(S) & \subseteq \{a^nb^mc^k \mid n+m \geq k\} \\
      \L^\ell_G(B)  & \subseteq \{ b^mc^k \mid m \geq k\}. 
    \end{align}
    Ще докажем, че
    \begin{align*}
      \L^{\ell+1}_G(S) & \subseteq \{a^nb^mc^k \mid n+m \geq k\}\\
      \L^{\ell+1}_G(B)  & \subseteq \{ b^mc^k \mid m \geq k\}.
    \end{align*}
    За първото включване, да разгледаме произволна дума $\alpha \in \L^{\ell+1}_G(S)$. Имаме три случая:
    \begin{itemize}
    \item
      Ако $\alpha \in \L^\ell_G(B)$, то от \IndHyp имаме, че
      \[\alpha \in \{b^mc^k \mid m \geq k\} \subseteq \{a^nb^mc^k \mid n+m \geq k\}.\]
    \item
      Ако $\alpha \in \{a\} \cdot \L^{\ell}_G(S)$, то от \IndHyp имаме, че
      \[\alpha \in \{a^{n+1}b^mc^k \mid n+m \geq k\} \subseteq \{a^nb^mc^k \mid n+m \geq k\}.\]
    \item
      Ако $\alpha \in \{a\} \cdot \L^{\ell}_G(S) \cdot \{c\}$, то от \IndHyp имаме, че
      \[\alpha \in \{a^{n+1}b^mc^{k+1} \mid n+m \geq k\} \subseteq \{a^nb^mc^k \mid n+m \geq k\}.\]
    \end{itemize}
    За второто включване, нека $\alpha \in \L^{\ell+1}_G(B)$. Имаме три случая за думата $\alpha$.
    \begin{itemize}
    \item
      Нека $\alpha \in \{b\} \cdot \L^\ell_G(B) \cdot \{c\}$. Тогава от \IndHyp имаме, че:
      \[\alpha \in \{b^{m+1}c^{k+1} \mid m \geq k\} \subseteq \{b^mc^k \mid  m \geq k\}.\]
    \item
      Нека $\alpha \in \{b\} \cdot \L^\ell_G(B)$. Тогава от \IndHyp имаме, че:
      \[\alpha \in \{b^{m+1}c^{k} \mid m \geq k\} \subseteq \{b^mc^k \mid m \geq k\}.\]
    \item
      Нека $\alpha \in \{\varepsilon\}$. Тогава е ясно, че имаме $\alpha \in \{b^mc^k \mid m \geq k\}$.
    \end{itemize}  
    
    \mynote{Така доказахме \emph{коректност} на граматиката.}
    Заключаваме, че
    \begin{align*}
      \L_G(S) & = \bigcup_\ell\L^\ell_G(S) \subseteq \{a^nb^mc^k \mid n+m \geq k\}\\
      \L_G(B) & = \bigcup_\ell\L^\ell_G(B) \subseteq \{a^nb^mc^k \mid n+m \geq k\}.
    \end{align*}
    
    \mynote{Сега ще докажем \emph{пълнота} на граматиката. Тук ще използваме представянията на $\L_G(S)$ и $\L_G(B)$ според \Corollary{grammar:yield-approximation}.}
    Сега трябва да докажем обратните включвания, а именно:
    \begin{align}
      & \{a^nb^mc^k \mid n+m \geq k\} \subseteq \L_G(S) \label{eq:anbmck:S}\\
      & \{b^mc^k \mid m \geq k\} \subseteq \L_G(B). \label{eq:anbmck:B}
    \end{align}
    
    Трябва да започнем първо със \Property{eq:anbmck:B}.
    Да разгледаме произволна дума $\alpha = b^mc^k$. Трябва да докажем, че $\alpha \in \L_G(B)$.
    Ще направим това с индукция по $m$.
    \begin{itemize}
    \item
      Нека $m = 0$. Това означава, че $\alpha = \varepsilon$. Ясно е, че $\varepsilon \in \L_G(B)$.
    \item
      Нека $m > 0$. Тук имаме два подслучая.
      \begin{itemize}
      \item
        Нека $m = k$. Тогава $\alpha = b \gamma c$ и имаме, че $\gamma = b^{m-1}c^{k-1}$.
        Можем да приложим \IndHyp за $\gamma$ и следователно $\gamma \in \L_G(B)$.
        Получаваме, че $\alpha \in \{b\} \cdot \L_G(B) \cdot \{c\} \subseteq \L_G(B)$.
      \item
        Нека $m > k$. Тогава $\alpha = b \gamma$ и имаме, че $\gamma = b^{m-1}c^k$.
        Можем да приложим \IndHyp за $\gamma$ и следователно $\gamma \in \L_G(B)$.
        Получаваме, че $\alpha \in \{b\} \cdot \L_G(B)\subseteq \L_G(B)$.
      \end{itemize}
    \end{itemize}
    Сега преминаваме към \Property{eq:anbmck:S}.
    Да разгледаме произволна дума $\alpha = a^nb^mc^k$. Трябва да докажем, че $\alpha \in \L_G(S)$.
    Ще направим това с индукция по $n$.
    \begin{itemize}
    \item
      Нека $n = 0$. Тогава $\alpha = b^mc^k$ и $m \geq k$.
      От \Property{eq:anbmck:B} следва, че $\alpha \in \L_G(B) \subseteq \L_G(S)$.
    \item
      Нека $n > 0$. Имаме два подслучая.
      \begin{itemize}
      \item
        Нека $n + m = k$. Тогава $\alpha = a\gamma c$ и $\gamma = a^{n-1}b^mc^{k-1}$.
        Можем да приложим \IndHyp за $\gamma$ и следователно $\gamma \in \L_G(S)$.
        Получаваме, че $\alpha \in \{a\} \cdot \L_G(S) \cdot \{c\} \subseteq \L_G(S)$.
      \item
        Нека $n + m > k$. Тогава $\alpha = a \gamma$ и $\gamma = a^{n-1}b^m c^k$.
        Можем да приложим \IndHyp за $\gamma$ и следователно $\gamma \in \L_G(S)$.
        Получаваме, че $\alpha \in \{a\} \cdot \L_G(S) \subseteq \L_G(S)$.
      \end{itemize}
    \end{itemize}
  \end{example}

\begin{problem}
  Докажете, че езикът $L = \{a^nb^mc^kd^\ell \mid n+k = m + \ell\}$ е безконтекстен.
\end{problem}
\begin{hint}
  Да разгледаме произволна дума от вида $\omega = a^n b^m c^k d^\ell$.
  Имаме два случая.
  \begin{itemize}
  \item
    Ако $n > \ell$, тогава имаме еквивалентността: $\omega \in L \iff (n-\ell) + k = m$.
  \item
    Ако $n \leq \ell$, тогава имаме еквивалентността: $\omega \in L \iff k = m + (\ell- n)$.
  \end{itemize}
  Това наблюдение ни подсказва, че трябва да разгледаме следните езици:
  \begin{align*}
    & L_1 = \{a^nb^mc^k \mid m = n+k\},\\
    & L_2 = \{b^mc^kd^\ell \mid k = m+\ell\}.
  \end{align*}
  Така получаваме, че
  \[L = \{a^n \cdot \omega \cdot d^n \mid n\in\Nat\ \&\ \omega \in L_1 \cup L_2\}.\]
  $L_1$ е безконтекстен език, защото може да се опише с безконтекстната граматика $G_1$ със следните правила:
  \[S_1 \to AC,\quad  A \to aAb\ |\ \varepsilon,\quad C \to bCc\ |\ \varepsilon.\]
  $L_1$ също е безконтекстен език, защото може да се опише с безконтекстната граматика $G_2$ със следните правила:
  \[S_2 \to BD,\quad B \to bBc\ |\ \varepsilon,\quad D \to cCd\ |\ \varepsilon.\]
  Тогава безконтекстната граматика $G$ за $L$ 
  съдържа правилата на граматиките $G_1$ и $G_2$, а също и правилата
  \[S \to aSd\ |\ S_1\ |\ S_2.\]
\end{hint}

\begin{problem}
  \label{prob:equal-but-different}
  \mynote{Ние вече знаем, че този език не е регулярен}
  Докажете, че езикът $L = \{\alpha\beta \in \{a,b\}^\star \mid\ |\alpha| = |\beta|\ \&\ \alpha \neq \beta\}$ е безконтекстен.
\end{problem}
\begin{hint}
  Да разгледаме една произволна дума $\omega$, където $\omega = \alpha\beta$, $|\alpha| = |\beta|$ и $\alpha \neq \beta$.
  Знаем, че същестува индекс $i < |\alpha|$, такъв че думата $\omega$ може да се представи така:
  \[\omega = \alpha\slice{:i} \cdot \alpha\slice{i} \cdot \alpha\slice{i+1:} \cdot \beta\slice{:i} \cdot \beta\slice{i} \cdot \beta\slice{i+1:},\]
  където $\alpha\slice{i} \neq \beta\slice{i}$.

  Нека $n = |\alpha| = |\beta|$ и да представим $n$ като $n = i+1+k$. Имаме два случая.
  \begin{itemize}
  \item
    Ако $k \geq i$, то можем да преставим $\omega$ по следния начин:
    \[\omega = \underbrace{\alpha\slice{:i}}_{\text{дълж. }i} \cdot \alpha\slice{i} \cdot \underbrace{\alpha\slice{i+1:2i+1}}_{\text{дълж. }i} \cdot \underbrace{\alpha\slice{2i+1:} \cdot \beta\slice{:i}}_{\text{дълж. }k} \cdot \beta\slice{i} \cdot \underbrace{\beta\slice{i+1:}}_{\text{дълж. }k}.\]
  \item
    Нека $k < i$, то можем да преставим $\omega$ по следния начин:
    \[\omega = \underbrace{\alpha\slice{:i}}_{\text{дълж. }i} \cdot \alpha\slice{i} \cdot \underbrace{\alpha\slice{i+1:} \cdot \beta\slice{:i-k}}_{\text{дълж. }i} \cdot \underbrace{\beta\slice{i-k:i}}_{\text{дълж. }k} \cdot \beta\slice{i} \cdot \underbrace{\beta\slice{i+1:}}_{\text{дълж. }k}.\]
  \end{itemize}
  От тези представяния на $\omega$ е ясно, че можем да изразим езика $L$ по следния начин:
  \[L = L_a \cdot L_b \cup L_b \cdot L_a,\]
  където:
  \begin{align*}
    & L_a = \{\alpha a \beta \mid \alpha,\beta \in \{a,b\}^\star\ \&\ |\alpha| = |\beta|\}\\
    & L_b = \{\alpha b \beta \mid \alpha,\beta \in \{a,b\}^\star\ \&\ |\alpha| = |\beta|\}.
  \end{align*}
  Сега разгледайте безконтекстната граматика $G$ със следните правила:
  \begin{align*}
    & S \to AB\ |\ BA\\
    & A \to XAX\ |\ a\\
    & B \to XBX\ |\ b\\
    & X \to a\ |\ b.
  \end{align*}
  Лесно се съобразява, че $L_a = \L_G(A)$ и $L_b = \L_G(B)$ и накрая $L = \L_G(S)$.
\end{hint}

\begin{problem}
 Докажете, че езикът $L = \{\alpha \sharp \beta \mid \alpha,\beta \in \{a,b\}^\star\ \&\ \alpha \neq \beta \}$ е безконтекстен.
\end{problem}
\begin{hint}
  Разгледайте граматиката:
  \begin{align*}
    & S \to AaR\ |\ BbR\ |\ E\\
    & A \to XAX\ |\ bR\sharp\\
    & B \to XBX\ |\ aR\sharp\\
    & E \to XEX\ |\ XR\sharp\ |\ \sharp XR\\
    & R \to XR\ |\ \varepsilon\\
    & X \to a\ |\ b.
  \end{align*}
  Имаме, че за произволни думи $\alpha,\beta,\gamma,\delta \in \{a,b\}^\star$,
  \begin{align*}
    & S \derive{\star} \alpha b \gamma \sharp \beta a \delta\ \&\ |\alpha| = |\beta|,\\
    & S \derive{\star} \alpha a \gamma \sharp \beta b \delta\ \&\ |\alpha| = |\beta|, \text{ или}\\
    & S \derive{\star} \alpha \sharp \beta\ \&\ |\alpha| \neq |\beta|.
  \end{align*}      
\end{hint}

\begin{problem}
  Докажете, че езикът $L = \{a^nb^mc^kd^\ell \mid n+k \geq m + \ell\}$ е безконтекстен.
\end{problem}

\begin{problem}
  \mynote{
    $S \to aS \mid aSc \mid aB \mid bB$\\
    $B \to bB \mid bBc \mid \varepsilon$
  }
  Докажете, че езикът $L = \{a^mb^nc^k\mid m+n \geq k + 1\}$ е безконтекстен.  
\end{problem}

\begin{problem}
  Докажете, че езикът $L = \{\alpha \sharp \beta \mid \alpha,\beta \in \{a,b\}^\star\ \&\ |\alpha| \neq |\beta| \}$ е безконтекстен.
\end{problem}

\begin{problem}
  Да разгледаме граматиката $G$ с правила
  \[S \to AA\ |\ B,\ A \to B\ |\ bb,\ B \to aa\ |\ aB.\]
  Да се намери езика на тази граматика и да се докаже, че граматиката разпознава точно този език.
\end{problem}

\end{extra}

%%% Local Variables:
%%% mode: latex
%%% TeX-master: "../eai"
%%% End:
