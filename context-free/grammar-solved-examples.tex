\subsection{Примерни задачи}

\begin{example}
  Езикът $L = \{a^mb^nc^k\mid m+n \geq k\}$ е безконтекстен.
\end{example}  
\begin{proof}
  Да разгледаме граматиката $G$ с правила
  \begin{align*}
    S& \rightarrow aSc\ |\ aS\ |\ B\\
    B& \rightarrow bBc\ |\  bB\ |\ \varepsilon.
  \end{align*}

  От \Proposition{grammar:yield-approximation} имаме, че:
  \begin{align*}
    \L^{\ell+1}_G(S) & = \{a\} \cdot \L^\ell_G(S) \cdot \{c\} \cup \{a\}\cdot \L^\ell_G(S) \cup \L^\ell_G(B)\\
    \L^{\ell+1}_G(B) & = \{b\} \cdot \L^\ell_G(B) \cdot \{c\} \cup \{b\} \cdot \L^\ell_G(B) \cup \{\varepsilon\}.
  \end{align*}

  Да сметнем първите няколко члена на редицата от апроксимации:
  \begin{align*}
    \L^0_G(S) & = \emptyset\\
    \L^0_G(B) & = \emptyset\\
    \L^1_G(S) & = \{a\} \cdot \emptyset \cdot \{c\} \cup \{a\}\cdot\emptyset \cup \emptyset \\
              & = \emptyset\\
    \L^1_G(B) & = \{b\} \cdot \emptyset \cdot \{c\} \cup \{b\} \cdot \emptyset \cup \{\varepsilon\} \\
              & = \{\varepsilon\}\\
    \L^2_G(S) & = \{a\} \cdot \emptyset \cdot \{c\} \cup \{a\}\cdot\emptyset \cup \{\varepsilon\} \\
              & = \{\varepsilon\}\\
    \L^2_G(B) & = \{b\} \cdot \{\varepsilon\} \cdot \{c\} \cup \{b\}\cdot \{\varepsilon\} \cup \{\varepsilon\} \\
              & = \{\varepsilon,b,bc\}\\
    \L^3_G(S) & = \{a\} \cdot \{\varepsilon\} \cdot \{c\} \cup \{a\}\cdot\{\varepsilon\} \cup \{\varepsilon,bc,b\} \\
              & = \{\varepsilon, a, b, ac, bc\}\\
              & = \{a^nb^mc^k \mid 2 > n+m \geq k\}\\
    \L^3_G(B) & = \{b\} \cdot \{\varepsilon,b,bc\} \cdot \{c\} \cup \{b\}\cdot \{\varepsilon,b,bc\} \cup \{\varepsilon,b,bc\} \\
              & = \{\varepsilon,b,bc,bc,bbc,bbcc,bb\}\\
              & = \{ b^mc^k \mid 2 \geq m \geq k\}
  \end{align*}
  Да предположим, че за произволно естествено число $\ell$ е изпълнено следното:
  \mynote{Това ще бъде нашето \IndHyp.}
  \begin{align*}
    \L^\ell_G(S) & = \{a^nb^mc^k \mid \ell - 1 > n+m \geq k\}\\
    \L^\ell_G(B)  & = \{ b^mc^k \mid \ell -1 \geq m \geq k\}.
  \end{align*}
  Ще докажем, че
  \begin{align*}
    \L^{\ell+1}_G(S) & = \{a^nb^mc^k \mid \ell > n+m \geq k\}\\
    \L^{\ell+1}_G(B)  & = \{ b^mc^k \mid \ell \geq m \geq k\}.
  \end{align*}

  \begin{itemize}
  \item
    Първо ще докажем, че
    \[\{a^nb^mc^k \mid \ell > n+m \geq k\} \subseteq \L^{\ell+1}_G(S).\]
    Да разгледаме думата $\alpha = a^nb^mc^k$ и $\ell > n+m \geq k$.
    \begin{itemize}
    \item
      Ако $\ell - 1 > n+m$, то от \IndHyp имаме, че
      \[\alpha \in \L^{\ell}_G(S) \subseteq \L^{\ell+1}_G(S).\]
    \item
      Нека сега $\ell-1 = n+m \geq k$.
      \begin{itemize}
      \item
        Ако $n = 0$, то $\alpha = b^mc^k$ и $\ell-1 = m \geq k$. Тогава от \IndHyp имаме, че
        \[\alpha \in \L^{\ell}_G(B) \subseteq \L^{\ell+1}_G(S).\]
      \item
        Нека сега $n > 0$. Тогава имаме следните два случая:
        \begin{itemize}
        \item 
          Ако $n + m > k$, то е ясно, че $(n-1) + m \geq k$. От \IndHyp имаме, че
          \[\alpha \in \{a\} \cdot \L^\ell_G(S) \subseteq \L^{\ell+1}_G(S).\]
        \item
          \mynote{Понеже $n > 0$, то и $k > 0$.}
          Ако $n + m = k$, то е ясно, че $(n-1) + m = k-1$. От \IndHyp имаме, че
          \[\alpha \in \{a\} \cdot \L^\ell_G(S) \cdot \{c\} \subseteq \L^{\ell+1}_G(S).\]
        \end{itemize}
      \end{itemize}
    \end{itemize}
  \item
    Сега да разгледаме включването
    \[\L^{\ell+1}_G(S) \subseteq \{a^nb^mc^k \mid \ell > n+m \geq k\}.\]
    Да разгледаме произволна дума $\alpha \in \L^{\ell+1}_G(S)$. Имаме три случая:
    \begin{itemize}
    \item
      Ако $\alpha \in \L^\ell_G(B)$, то от \IndHyp имаме, че
      \[\alpha \in \{b^mc^k \mid \ell > m \geq k\} \subseteq \{a^nb^mc^k \mid \ell > n+m \geq k\}.\]
    \item
      Ако $\alpha \in \{a\} \cdot \L^{\ell}_G(S)$, то от \IndHyp имаме, че
      \[\alpha \in \{a^{n+1}b^mc^k \mid \ell - 1 > n+m \geq k\} \subseteq \{a^nb^mc^k \mid \ell > n+m \geq k\}.\]
    \item
      Ако $\alpha \in \{a\} \cdot \L^{\ell}_G(S) \cdot \{c\}$, то от \IndHyp имаме, че
      \[\alpha \in \{a^{n+1}b^mc^{k+1} \mid \ell - 1 > n+m \geq k\} \subseteq \{a^nb^mc^k \mid \ell > n+m \geq k\}.\]
    \end{itemize}
  \end{itemize}

  Сега преминаваме към второто равенство.
  % \[\L^{\ell+1}_G(B) = \L^\ell_G(B) \cup \{b\} \cdot \L^\ell_G(B) \cdot \{c\} \cup \{b\} \cdot \L^\ell_G(B).\]  
  \begin{itemize}
  \item
    Първо ще докажем, че
    \[\{b^mc^k \mid \ell \geq m \geq k\} \subseteq \L^{\ell+1}_G(B).\]
    Да разгледаме думата $\alpha = b^mc^k$ и $\ell \geq m \geq k$.
    \begin{itemize}
    \item
      Ако $m = 0$, то $\alpha = \varepsilon$ и е ясно, че:
      \[\alpha \in \L^{1}_G(B) \subseteq \L^{\ell+1}_G(B).\]
    \item
      Ако $\ell > m$, то от \IndHyp имаме, че:
      \[\alpha \in \L^{\ell}_G(B) \subseteq \L^{\ell+1}_G(B).\]
    \item
      Нека сега $\ell = m \geq k$.
      \begin{itemize}
      \item
        Ако $\ell = m = k$, то от \IndHyp имаме, че:
        \[\alpha \in \{a\} \cdot \L^\ell_G(B) \cdot \{c\} \subseteq \L^{\ell+1}_G(B).\]
      \item
        Ако $\ell = m > k$, то от \IndHyp имаме, че:
        \[\alpha \in \{a\} \cdot \L^\ell_G(B) \subseteq \L^{\ell+1}_G(B).\]
      \end{itemize}
    \end{itemize}
  \item
    Сега да разгледаме включването
    \[\L^{\ell+1}_G(B) \subseteq \{b^mc^k \mid \ell \geq m \geq k\}.\]
    Нека $\alpha \in \L^{\ell+1}_G(B)$. Имаме три случая за думата $\alpha$.
    \begin{itemize}
    \item
      Нека $\alpha \in \{b\} \cdot \L^\ell_G(B) \cdot \{c\}$. Тогава от \IndHyp имаме, че:
      \[\alpha \in \{b^{m+1}c^{k+1} \mid \ell-1\geq m \geq k\} \subseteq \{b^mc^k \mid \ell \geq m \geq k\}.\]
    \item
      Нека $\alpha \in \{b\} \cdot \L^\ell_G(B)$. Тогава от \IndHyp имаме, че:
      \[\alpha \in \{b^{m+1}c^{k} \mid \ell-1\geq m \geq k\} \subseteq \{b^mc^k \mid \ell \geq m \geq k\}.\]
    \item
      Нека $\alpha \in \{\varepsilon\}$. Тогава е ясно, че имаме:
      \[\alpha \in \{b^mc^l \mid \ell \geq m \geq k\}.\]
    \end{itemize}
  \end{itemize}
\end{proof}

\begin{example}
  Езикът 
  \[L = \{a^nb^mc^kd^\ell \mid n+k = m + \ell\}\]
  е безконтекстен.
\end{example}
\begin{hint}
  Нека $G_1$ е безконтекстна граматика за езика
  \[L_1 = \{a^nb^mc^k \mid m = n+k\},\]
  където правилата на $G_1$ са
  \[S_1 \to AC,\quad  A \to aAb\ |\ \varepsilon,\quad C \to bCc\ |\ \varepsilon.\]
  Нека $G_2$ е безконтекстна граматика за езика 
  \[L_2 = \{b^mc^kd^\ell \mid k = m+\ell\},\]
  където правилата на $G_2$ са
  \[S_2 \to BD,\quad B \to bBc\ |\ \varepsilon,\quad D \to cCd\ |\ \varepsilon.\]
  Тогава граматиката $G$ за $L$ 
  съдържа правилата на граматиките $G_1$ и $G_2$, а също и правилата
  \[S \to aSd\ |\ S_1\ |\ S_2.\]
\end{hint}

\begin{problem}
  \label{prob:equal-but-different}
  \mynote{Ние вече знаем, че този език не е регулярен}
  Езикът
  \[L = \{\alpha\beta \in \{a,b\}^\star \mid\ |\alpha| = |\beta|\ \&\ \alpha \neq \beta\}\]
  е безконтекстен.
\end{problem}
\begin{hint}
  Разгледайте граматиката:
  \begin{align*}
    & S \to AB\ |\ BA\\
    & A \to XAX\ |\ a\\
    & B \to XBX\ |\ b\\
    & X \to a\ |\ b.
  \end{align*}
\end{hint}

\begin{problem}
 Докажете, че езикът
 \[L = \{\alpha \sharp \beta \mid \alpha,\beta \in \{a,b\}^\star\ \&\ \alpha \neq \beta \}\]
 е безконтекстен.
\end{problem}
\begin{hint}
  Разгледайте граматиката:
  \begin{align*}
    & S \to AaR\ |\ BbR\ |\ E\\
    & A \to XAX\ |\ bR\sharp\\
    & B \to XBX\ |\ aR\sharp\\
    & E \to XEX\ |\ XR\sharp\ |\ \sharp XR\\
    & R \to XR\ |\ \varepsilon\\
    & X \to a\ |\ b.
  \end{align*}
  Имаме, че за произволни думи $\alpha,\beta,\gamma,\delta \in \{a,b\}^\star$,
  \begin{align*}
    & S \to^\star \alpha b \gamma \sharp \beta a \delta\ \&\ |\alpha| = |\beta|,\\
    & S \to^\star \alpha a \gamma \sharp \beta b \delta\ \&\ |\alpha| = |\beta|, \text{ или}\\
    & S \to^\star \alpha \sharp \beta\ \&\ |\alpha| \neq |\beta|.
  \end{align*}      
\end{hint}






\begin{problem}
  Докажете, че езикът 
  \[L = \{a^nb^mc^kd^\ell \mid n+k \geq m + \ell\}\]
  е безконтекстен.
\end{problem}

\begin{problem}
  \mynote{
    $S \to aS \mid aSc \mid aB \mid bB$\\
    $B \to bB \mid bBc \mid \varepsilon$
  }
  Докажете, че езикът 
  \[L = \{a^mb^nc^k\mid m+n \geq k + 1\}\]
  е безконтекстен.  
\end{problem}

\begin{problem}
  Докажете, че езикът
  \[L = \{\alpha \sharp \beta \mid \alpha,\beta \in \{a,b\}^\star\ \&\ |\alpha| \neq |\beta| \}\]
  е безконтекстен.
\end{problem}

\begin{problem}
  Да разгледаме граматиката $G$ с правила
  \[S \to AA\ |\ B,\ A \to B\ |\ bb,\ B \to aa\ |\ aB.\]
  Да се намери езика на тази граматика и да се докаже, че граматиката разпознава точно този език.
\end{problem}

%%% Local Variables:
%%% mode: latex
%%% TeX-master: "../eai"
%%% End:
