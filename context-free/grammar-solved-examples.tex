\subsection{Примерни задачи}

\begin{extra}
  \begin{example}
    Да разгледаме безконтекстната граматика $G$ зададена със следните правила:
    \begin{align*}
      & S \to aSc\ |\  B\\
      & B \to bBc\ |\ \varepsilon.
    \end{align*}
    Да видим защо $\L(G) = \{a^nb^kc^{n+k} \mid n,k\in\Nat\}$.
    Първо ще докажем \emph{коректност} на граматиката. Това означава, че $G$ не генерира думи извън езика, т.е.
    $\L_G(S) \subseteq \{a^nb^kc^{n+k} \mid n,k\in\Nat\}$. За да направим това обаче, трябва да знаем също така и $\L_G(B)$.
    Формално казано, трябва да докажем, че за всяко $\ell$ е изпълнено следното:
    \begin{align}
      \L^\ell_G(S) & \subseteq \{a^nb^kc^{n+k} \mid n,k\in\Nat\} \label{eq:nknk1}\\
      \L^\ell_G(B) & \subseteq \{b^kc^k \mid k \in \Nat\} \label{eq:nknk2}.
    \end{align}
    Това ще направим с индукция по $\ell$.
    Да напомним, че според \Proposition{grammar:yield-approximation} имаме следните връзки:
    \begin{align*}
      & \L^0_G(S) = \emptyset\\
      & \L^{\ell+1}_G(S) = \{a\} \cdot \L^\ell_G(S) \cdot \{c\} \cup \L^\ell_G(B)\\
      & \L^0_G(B) = \emptyset\\
      & \L^{\ell+1}_G(B) = \{b\} \cdot \L^\ell_G(B) \cdot \{c\} \cup \{\varepsilon\}.
    \end{align*}
    Очевидно е, че \Property{eq:nknk1} и \Property{eq:nknk2} са изпълнени за $\ell = 0$.
    Да примем, че имаме \Property{eq:nknk1} и \Property{eq:nknk2} за някое $\ell$.
    \mynote{Обърнете внимание, че не можем да докажем \Property{eq:nknk1} независимо от \Property{eq:nknk2}.}
    Ще докажем свойствата и за $\ell+1$.
    \begin{itemize}
    \item
      Първо, нека $\alpha \in \L^{\ell+1}_G(S)$. Имаме два случая.
      \begin{itemize}
      \item
        Нека $\alpha \in \{a\} \cdot \L^\ell_G(S) \cdot \{c\}$. От \IndHyp следва, че
        $\alpha = a^{n+1}b^kc^{n+k+1}$ за някои естествени числа $n$ и $k$.
        Тогава е ясно, че $\alpha \in \{a^nb^kc^{n+k} \mid n,k\in\Nat\}$.
      \item
        Нека $\alpha \in  \L^\ell_G(B)$. От \IndHyp следва, че
        $\alpha \in \{b^kc^k \mid k \in \Nat\} \subseteq \{a^nb^kc^{n+k} \mid n,k\in\Nat\}$.
      \end{itemize}
    \item
      Второ, нека $\alpha \in \L^{\ell+1}_G(B)$. Имаме два случая.
      \begin{itemize}
      \item
        Нека $\alpha \in \{b\} \cdot \L^\ell_G(B) \cdot \{c\}$. От \IndHyp следва, че
        $\alpha = b^{k+1}c^{k+1}$ за някое естествено число $k$.
        Тогава е ясно, че $\alpha \in \{b^{k}c^{k} \mid k\in\Nat\}$.
      \item
        Нека $\alpha = \varepsilon$. В този случай също е ясно, че $\alpha \in \{b^{k}c^{k} \mid k\in\Nat\}$.
      \end{itemize}
    \end{itemize}
    Оттук заключаваме, че $\L_G(S) \subseteq \{a^nb^kc^{n+k} \mid n,k\in\Nat\}$.
    
    Сега да разгледаме пълнота на граматиката, което означава, че
    всяка дума от езика се генерира от $G$. С други думи, $\{a^nb^kc^{n+k} \mid n,k\in\Nat\} \subseteq \L_G(S)$.
    Първо ще докажем, че
    \begin{equation}
      \label{eq:4}
      \{b^kc^k \mid k \in \Nat\} \subseteq \L_G(B).
    \end{equation}
    Това включване ще докажем с \emph{пълна} индукция по дължината на думата.
    Да разгледаме произволна дума $\alpha \in \{b^kc^k \mid k \in \Nat\}$.
    \begin{itemize}
    \item
      Нека $|\alpha| = 0$, т.е. $\alpha = \varepsilon$.
      Ясно е, че $\alpha \in \L_G(B)$.
    \item
      Нека $|\alpha| > 0$, т.е. $\alpha = b^{k+1}c^{k+1}$.
      От \IndHyp за \Property{eq:4} следва, че $\alpha \in \{b\} \cdot \L_G(B) \cdot \{c\} \subseteq \L_G(B)$.
    \end{itemize}
    Вече сме готови да докажем, че:
    \begin{equation}
      \label{eq:9}
      \{a^nb^kc^{n+k} \mid n,k \in \Nat\} \subseteq \L_G(S).
    \end{equation}
    Това включване пак ще докажем с \emph{пълна} индукция по дължината на думата.
    Да разгледаме произволна дума $\alpha \in L$. 
    \begin{itemize}
    \item
      Нека $|\alpha| = 0$, т.е. $\alpha = \varepsilon$.
      Ясно е, че $\alpha \in \L_G(B) \subseteq \L_G(S)$.
    \item
      Нека сега $|\alpha| > 0$, т.е. $\alpha = a^nb^kc^{n+k}$, където $n > 0$ или $k > 0$. Да разгледаме два случая.
      \begin{itemize}
      \item
        Нека $n = 0$. Тогава $\alpha = b^kc^k$ и $k > 0$. Тогава от \Property{eq:4}
        следва, че $\alpha \in \L_G(B) \subseteq \L_G(S)$.
      \item                   
        Нека $n > 0$. Тогава от \IndHyp за \Property{eq:9} следва, че
        $\alpha \in \{a\} \cdot \L_G(S) \cdot \{c\} \subseteq \L_G(S)$.
      \end{itemize}
    \end{itemize}
    Доказахме коректност и пълнота на граматиката и следователно $\L(G) = \{a^nb^kc^{n+k} \mid n,k\in\Nat\}$.
  \end{example}
  
  
  \begin{example}
    Нека да видим защо езикът $L = \{a^mb^nc^k\mid m+n \geq k\}$ е безконтекстен.
    Да разгледаме граматиката $G$ с правила
    \begin{align*}
      S& \rightarrow aSc\ |\ aS\ |\ B\\
      B& \rightarrow bBc\ |\  bB\ |\ \varepsilon.
    \end{align*}
    От \Proposition{grammar:yield-approximation} имаме, че:
    \begin{align*}
      \L^0_G(S) & = \emptyset\\
      \L^{\ell+1}_G(S) & = \{a\} \cdot \L^\ell_G(S) \cdot \{c\} \cup \{a\}\cdot \L^\ell_G(S) \cup \L^\ell_G(B)\\
      \L^0_G(B) & = \emptyset\\
      \L^{\ell+1}_G(B) & = \{b\} \cdot \L^\ell_G(B) \cdot \{c\} \cup \{b\} \cdot \L^\ell_G(B) \cup \{\varepsilon\}.
    \end{align*}
    Да предположим, че за произволно естествено число $\ell$ е изпълнено следното:
    \mynote{Тези две свойства ще бъдат нашето \IndHyp. Очевидно е, че те са изпълнени за $\ell = 0$.}
    \begin{align}
      \L^\ell_G(S) & \subseteq \{a^nb^mc^k \mid n+m \geq k\} \\
      \L^\ell_G(B)  & \subseteq \{ b^mc^k \mid m \geq k\}. 
    \end{align}
    Ще докажем, че
    \begin{align*}
      \L^{\ell+1}_G(S) & \subseteq \{a^nb^mc^k \mid n+m \geq k\}\\
      \L^{\ell+1}_G(B)  & \subseteq \{ b^mc^k \mid m \geq k\}.
    \end{align*}
    За първото включване, да разгледаме произволна дума $\alpha \in \L^{\ell+1}_G(S)$. Имаме три случая:
    \begin{itemize}
    \item
      Ако $\alpha \in \L^\ell_G(B)$, то от \IndHyp имаме, че
      \[\alpha \in \{b^mc^k \mid m \geq k\} \subseteq \{a^nb^mc^k \mid n+m \geq k\}.\]
    \item
      Ако $\alpha \in \{a\} \cdot \L^{\ell}_G(S)$, то от \IndHyp имаме, че
      \[\alpha \in \{a^{n+1}b^mc^k \mid n+m \geq k\} \subseteq \{a^nb^mc^k \mid n+m \geq k\}.\]
    \item
      Ако $\alpha \in \{a\} \cdot \L^{\ell}_G(S) \cdot \{c\}$, то от \IndHyp имаме, че
      \[\alpha \in \{a^{n+1}b^mc^{k+1} \mid n+m \geq k\} \subseteq \{a^nb^mc^k \mid n+m \geq k\}.\]
    \end{itemize}
    
    За второто включване, нека $\alpha \in \L^{\ell+1}_G(B)$. Имаме три случая за думата $\alpha$.
    \begin{itemize}
    \item
      Нека $\alpha \in \{b\} \cdot \L^\ell_G(B) \cdot \{c\}$. Тогава от \IndHyp имаме, че:
      \[\alpha \in \{b^{m+1}c^{k+1} \mid m \geq k\} \subseteq \{b^mc^k \mid  m \geq k\}.\]
    \item
      Нека $\alpha \in \{b\} \cdot \L^\ell_G(B)$. Тогава от \IndHyp имаме, че:
      \[\alpha \in \{b^{m+1}c^{k} \mid m \geq k\} \subseteq \{b^mc^k \mid m \geq k\}.\]
    \item
      Нека $\alpha \in \{\varepsilon\}$. Тогава е ясно, че имаме $\alpha \in \{b^mc^k \mid m \geq k\}$.
    \end{itemize}  
    
    \mynote{Така доказахме \emph{коректност} на граматиката.}
    Заключаваме, че
    \begin{align*}
      \L_G(S) & = \bigcup_\ell\L^\ell_G(S) \subseteq \{a^nb^mc^k \mid n+m \geq k\}\\
      \L_G(B) & = \bigcup_\ell\L^\ell_G(B) \subseteq \{a^nb^mc^k \mid n+m \geq k\}.
    \end{align*}
    
    \mynote{Сега ще докажем \emph{пълнота} на граматиката.}
    Сега трябва да докажем обратните включвания, а именно:
    \begin{align}
      & \{a^nb^mc^k \mid n+m \geq k\} \subseteq \L_G(S) \label{eq:anbmck:S}\\
      & \{b^mc^k \mid m \geq k\} \subseteq \L_G(B). \label{eq:anbmck:B}
    \end{align}
    
    Трябва да започнем първо със \Property{eq:anbmck:B}.
    Да разгледаме произволна дума $\alpha = b^mc^k$. Трябва да докажем, че $\alpha \in \L_G(B)$.
    Ще направим това с индукция по $m$.
    \begin{itemize}
    \item
      Нека $m = 0$. Това означава, че $\alpha = \varepsilon$. Ясно е, че $\varepsilon \in \L_G(B)$.
    \item
      Нека $m > 0$. Тук имаме два подслучая.
      \begin{itemize}
      \item
        Нека $m = k$. Тогава $\alpha = b \gamma c$ и имаме, че $\gamma = b^{m-1}c^{k-1}$.
        Можем да приложим \IndHyp за $\gamma$ и следователно $\gamma \in \L_G(B)$.
        Получаваме, че $\alpha \in \{b\} \cdot \L_G(B) \cdot \{c\} \subseteq \L_G(B)$.
      \item
        Нека $m > k$. Тогава $\alpha = b \gamma$ и имаме, че $\gamma = b^{m-1}c^k$.
        Можем да приложим \IndHyp за $\gamma$ и следователно $\gamma \in \L_G(B)$.
        Получаваме, че $\alpha \in \{b\} \cdot \L_G(B)\subseteq \L_G(B)$.
      \end{itemize}
    \end{itemize}
    Сега преминаваме към \Property{eq:anbmck:S}.
    Да разгледаме произволна дума $\alpha = a^nb^mc^k$. Трябва да докажем, че $\alpha \in \L_G(S)$.
    Ще направим това с индукция по $n$.
    \begin{itemize}
    \item
      Нека $n = 0$. Тогава $\alpha = b^mc^k$ и $m \geq k$.
      От \Property{eq:anbmck:B} следва, че $\alpha \in \L_G(B) \subseteq \L_G(S)$.
    \item
      Нека $n > 0$. Имаме два подслучая.
      \begin{itemize}
      \item
        Нека $n + m = k$. Тогава $\alpha = a\gamma c$ и $\gamma = a^{n-1}b^mc^{k-1}$.
        Можем да приложим \IndHyp за $\gamma$ и следователно $\gamma \in \L_G(S)$.
        Получаваме, че $\alpha \in \{a\} \cdot \L_G(S) \cdot \{c\} \subseteq \L_G(S)$.
      \item
        Нека $n + m > k$. Тогава $\alpha = a \gamma$ и $\gamma = a^{n-1}b^m c^k$.
        Можем да приложим \IndHyp за $\gamma$ и следователно $\gamma \in \L_G(S)$.
        Получаваме, че $\alpha \in \{a\} \cdot \L_G(S) \subseteq \L_G(S)$.
      \end{itemize}
    \end{itemize}
  \end{example}
\end{extra}

\newpage
\begin{problem}
  Докажете, че езикът 
  \[L = \{a^nb^mc^kd^\ell \mid n+k = m + \ell\}\]
  е безконтекстен.
\end{problem}
\begin{hint}
  Да разгледаме произволна дума от вида $\omega = a^n b^m c^k d^\ell$.
  Имаме два случая.
  \begin{itemize}
  \item
    Ако $n \leq \ell$, тогава имаме еквивалентността:
    \[\omega \in L \iff k = m + (\ell- n).\]
  \item
    Ако $n > \ell$, тогава имаме еквивалентността:
    \[\omega \in L \iff (n-\ell) + k = m.\]
  \end{itemize}
  Това наблюдение ни подсказва, че трябва да разгледаме следните езици:
  \begin{align*}
    & L_1 = \{a^nb^mc^k \mid m = n+k\},\\
    & L_2 = \{b^mc^kd^\ell \mid k = m+\ell\}.
  \end{align*}
  Така получаваме, че
  \[L = \{a^n \cdot \omega \cdot d^n \mid n\in\Nat\ \&\ \omega \in L_1 \cup L_2\}.\]
  $L_1$ е безконтекстен език, защото може да се опише с безконтекстната граматика $G_1$ със следните правила:
  \[S_1 \to AC,\quad  A \to aAb\ |\ \varepsilon,\quad C \to bCc\ |\ \varepsilon.\]
  $L_1$ също е безконтекстен език, защото може да се опише с безконтекстната граматика $G_2$ със следните правила:
  \[S_2 \to BD,\quad B \to bBc\ |\ \varepsilon,\quad D \to cCd\ |\ \varepsilon.\]
  Тогава безконтекстната граматика $G$ за $L$ 
  съдържа правилата на граматиките $G_1$ и $G_2$, а също и правилата
  \[S \to aSd\ |\ S_1\ |\ S_2.\]
\end{hint}

\begin{problem}
  \label{prob:equal-but-different}
  \mynote{Ние вече знаем, че този език не е регулярен}
  Докажете, че езикът
  \[L = \{\alpha\beta \in \{a,b\}^\star \mid\ |\alpha| = |\beta|\ \&\ \alpha \neq \beta\}\]
  е безконтекстен.
\end{problem}
\begin{hint}
  Да разгледаме една произволна дума $\omega$, където $\omega = \alpha\beta$, $|\alpha| = |\beta|$ и $\alpha \neq \beta$.
  Знаем, че същестува индекс $i < |\alpha|$, такъв че думата $\omega$ може да се представи така:
  \[\omega = \alpha\slice{:i} \cdot \alpha\slice{i} \cdot \alpha\slice{i+1:} \cdot \beta\slice{:i} \cdot \beta\slice{i} \cdot \beta\slice{i+1:},\]
  където $\alpha\slice{i} \neq \beta\slice{i}$.

  Нека $n = |\alpha| = |\beta|$ и да представим $n$ като $n = i+1+k$. Имаме два случая.
  \begin{itemize}
  \item
    Ако $k \geq i$, то можем да преставим $\omega$ по следния начин:
    \[\omega = \underbrace{\alpha\slice{:i}}_{\text{дълж. }i} \cdot \alpha\slice{i} \cdot \underbrace{\alpha\slice{i+1:2i+1}}_{\text{дълж. }i} \cdot \underbrace{\alpha\slice{2i+1:} \cdot \beta\slice{:i}}_{\text{дълж. }k} \cdot \beta\slice{i} \cdot \underbrace{\beta\slice{i+1:}}_{\text{дълж. }k}.\]
  \item
    Нека $k < i$, то можем да преставим $\omega$ по следния начин:
    \[\omega = \underbrace{\alpha\slice{:i}}_{\text{дълж. }i} \cdot \alpha\slice{i} \cdot \underbrace{\alpha\slice{i+1:} \cdot \beta\slice{:i-k}}_{\text{дълж. }i} \cdot \underbrace{\beta\slice{i-k:i}}_{\text{дълж. }k} \cdot \beta\slice{i} \cdot \underbrace{\beta\slice{i+1:}}_{\text{дълж. }k}.\]
  \end{itemize}
  От тези представяния на $\omega$ е ясно, че можем да изразим езика $L$ по следния начин:
  \[L = L_a \cdot L_b \cup L_b \cdot L_a,\]
  където:
  \begin{align*}
    & L_a = \{\alpha a \beta \mid \alpha,\beta \in \{a,b\}^\star\ \&\ |\alpha| = |\beta|\}\\
    & L_b = \{\alpha b \beta \mid \alpha,\beta \in \{a,b\}^\star\ \&\ |\alpha| = |\beta|\}.
  \end{align*}
  Сега разгледайте безконтекстната граматика $G$ със следните правила:ю
  \begin{align*}
    & S \to AB\ |\ BA\\
    & A \to XAX\ |\ a\\
    & B \to XBX\ |\ b\\
    & X \to a\ |\ b.
  \end{align*}
  Лесно се съобразява, че:
  \begin{align*}
    & L_a = \L_G(A)\\
    & L_b = \L_G(B)\\
    & L = \L_G(S).
  \end{align*}
\end{hint}

\begin{problem}
 Докажете, че езикът
 \[L = \{\alpha \sharp \beta \mid \alpha,\beta \in \{a,b\}^\star\ \&\ \alpha \neq \beta \}\]
 е безконтекстен.
\end{problem}
\begin{hint}
  Разгледайте граматиката:
  \begin{align*}
    & S \to AaR\ |\ BbR\ |\ E\\
    & A \to XAX\ |\ bR\sharp\\
    & B \to XBX\ |\ aR\sharp\\
    & E \to XEX\ |\ XR\sharp\ |\ \sharp XR\\
    & R \to XR\ |\ \varepsilon\\
    & X \to a\ |\ b.
  \end{align*}
  Имаме, че за произволни думи $\alpha,\beta,\gamma,\delta \in \{a,b\}^\star$,
  \begin{align*}
    & S \to^\star \alpha b \gamma \sharp \beta a \delta\ \&\ |\alpha| = |\beta|,\\
    & S \to^\star \alpha a \gamma \sharp \beta b \delta\ \&\ |\alpha| = |\beta|, \text{ или}\\
    & S \to^\star \alpha \sharp \beta\ \&\ |\alpha| \neq |\beta|.
  \end{align*}      
\end{hint}






\begin{problem}
  Докажете, че езикът 
  \[L = \{a^nb^mc^kd^\ell \mid n+k \geq m + \ell\}\]
  е безконтекстен.
\end{problem}

\begin{problem}
  \mynote{
    $S \to aS \mid aSc \mid aB \mid bB$\\
    $B \to bB \mid bBc \mid \varepsilon$
  }
  Докажете, че езикът 
  \[L = \{a^mb^nc^k\mid m+n \geq k + 1\}\]
  е безконтекстен.  
\end{problem}

\begin{problem}
  Докажете, че езикът
  \[L = \{\alpha \sharp \beta \mid \alpha,\beta \in \{a,b\}^\star\ \&\ |\alpha| \neq |\beta| \}\]
  е безконтекстен.
\end{problem}

\begin{problem}
  Да разгледаме граматиката $G$ с правила
  \[S \to AA\ |\ B,\ A \to B\ |\ bb,\ B \to aa\ |\ aB.\]
  Да се намери езика на тази граматика и да се докаже, че граматиката разпознава точно този език.
\end{problem}

%%% Local Variables:
%%% mode: latex
%%% TeX-master: "../eai"
%%% End:
