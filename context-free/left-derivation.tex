\section{Най-ляв извод в граматика}
\index{граматика!най-ляв извод}
\mynote{Най-левият извод ще бъде важен за нас, когато разгледаме стековите автомати.}
В нашата дефиниция на извод, изборът върху коя променлива да приложим правило от граматиката е недетерминистичен.
В някои случаи, за нас ще бъде важно винаги да правим детерминистичен избор на това върху коя променлива прилагаме правило.

\begin{definition}
За две думи $\alpha,\beta \in (V\cup\Sigma)^\star$, дефинираме {\bf най-ляв извод} в граматиката $G$, $\alpha \lderive{\ell} \beta$, по следния начин:
\mynote{Новото е, че имаме изискването $\lambda \in \Sigma^\star$. Това на практика означава, че на всяка стъпка заместваме възможно най-лявата променлива.}
\begin{important}
\begin{figure}[H]
  \begin{subfigure}[b]{0.4\textwidth}
      \begin{prooftree}
        \AxiomC{}
        \RightLabel{\scriptsize{правило (0)}}
        \UnaryInfC{$\alpha \lderive{0} \alpha$}
      \end{prooftree}
    \end{subfigure}
    ~
    \begin{subfigure}[b]{0.4\textwidth}
      \begin{prooftree}
        \AxiomC{$(A,\alpha) \in R$}
        \AxiomC{$\lambda \alpha \rho \lderive{\ell} \beta$}
        \AxiomC{$\lambda \in \Sigma^\star$}
        \RightLabel{\scriptsize{правило (1)}}
        \TrinaryInfC{$\lambda A \rho \lderive{\ell+1} \beta$}
      \end{prooftree}
    \end{subfigure}
    \caption{Правила за най-ляв извод в безконтекстна граматика.}
  \end{figure}  
\end{important}
\end{definition}





\mynote{Интуитивно, $A \yield{\star} \alpha$ ни казва, че при обхождане на синтактичното дърво в широчина ние получаваме като листа думата $\alpha$. Ако
  $A\lderive{\star}\alpha$, то обхождаме синтактичното дърво с корен $A$ в дълбочина, като винаги избираме най-левия необходен клон.  Ако $A\derive{\star}\alpha$, то обхождаме синтактичното дърво в дълбочина без да имаме детерминистичен избор с кой необходен клон да продължим.}

\begin{proposition}\label{pr:left-derivation:padding}
  За проиволно естествено число $\ell$ имаме извода:
  \begin{prooftree}
    \AxiomC{$\alpha \lderive{\ell} \beta$}
    \AxiomC{$\lambda \in \Sigma^\star$}
    \AxiomC{$\rho \in (V\cup\Sigma)^\star$}
    \TrinaryInfC{$\lambda \alpha \rho \lderive{\ell} \lambda \beta \rho$}
  \end{prooftree}
\end{proposition}
\begin{proof}
  Индукция по $\ell$.
  \begin{itemize}
  \item
    $\ell = 0$. Директно следва от правило $(0)$.
  \item
    $\ell > 0$. Според правилата за извод имаме следния случай:
    \begin{prooftree}
      \AxiomC{$(A,\alpha') \in R$}
      \AxiomC{$\gamma\alpha'\delta \lderive{\ell-1} \beta$}
      \AxiomC{$\gamma \in \Sigma^\star$}
      \RightLabel{\scriptsize{правило (1)}}
      \TrinaryInfC{$\underbrace{\gamma A \delta}_{\alpha} \lderive{\ell} \beta$}
    \end{prooftree}
    Сега като използваме \IndHyp получаваме следния извод:
    \begin{prooftree}
      \AxiomC{$(A,\alpha') \in R$}
      \AxiomC{$\gamma\alpha'\delta \lderive{\ell-1} \beta$}
      \AxiomC{$\lambda \in \Sigma^\star$}
      \AxiomC{$\rho \in (V\cup\Sigma)^\star$}
      \RightLabel{\scriptsize{\IndHyp}}
      \TrinaryInfC{$\lambda\gamma\alpha'\delta\rho \lderive{\ell-1} \lambda\beta\rho$}
      \AxiomC{$\lambda\gamma \in \Sigma^\star$}
      \RightLabel{\scriptsize{правило (1)}}
      \TrinaryInfC{$\lambda\underbrace{\gamma A \delta}_{\alpha}\rho \lderive{\ell} \lambda\beta\rho$}
    \end{prooftree}
  \end{itemize}
\end{proof}



Удобно ще бъде да разгледаме следния аналог на \Proposition{unrestricted-grammar:context-general-step}.
\begin{proposition}\label{pr:left-derivation:context-step}
  Имаме извода:
  \begin{prooftree}
    \AxiomC{$\delta \lderive{\ell_1} \alpha B \gamma$}
    \AxiomC{$\alpha \in \Sigma^\star$}
    \AxiomC{$B \lderive{\ell_2}\beta$}
    \TrinaryInfC{$\delta \lderive{\ell_1+\ell_2}\alpha\beta\gamma$}
  \end{prooftree}
\end{proposition}
\begin{proof}
  Сега ще докажем твърдението с индукция по $\ell_1$.
  \begin{itemize}
  \item
    Нека $\ell_1 = 0$. Тук нещата са ясни, защото $\delta = \alpha B \gamma$ и имаме извода:
    \begin{prooftree}
      \AxiomC{$B \lderive{\ell_2} \beta$}
      \AxiomC{$\alpha \in \Sigma^\star$}
      \AxiomC{$\gamma \in (V \cup \Sigma)^\star$}
      \RightLabel{\scriptsize{(\Proposition{left-derivation:padding})}}
      \TrinaryInfC{$\underbrace{\alpha B \gamma}_{\delta} \lderive{\ell_2} \alpha\beta\gamma$}
    \end{prooftree}
  \item
    Нека $\ell_1 > 0$. Трябва да разбием извода с дължина $\ell_1$ за да можем да приложим индукционното предположение. От правилата за извод следва, че имаме следната ситуация:
    \begin{prooftree}
      \AxiomC{$(A,\alpha') \in R$}
      \AxiomC{$\lambda \alpha' \rho \lderive{\ell_1-1} \alpha B \gamma$}
      \AxiomC{$\lambda \in \Sigma^\star$}
      \RightLabel{\scriptsize{(1)}}
      \TrinaryInfC{$\underbrace{\lambda A \rho}_{\delta} \lderive{\ell_1} \alpha B \gamma$}
    \end{prooftree}

    Използвайки \IndHyp получаваме следния извод:
    \begin{prooftree}
      \AxiomC{$\lambda \in \Sigma^\star$}
      \AxiomC{$(A,\alpha') \in R$}
      \AxiomC{$\lambda\alpha'\rho \lderive{\ell_1-1} \alpha B \gamma$}
      \AxiomC{$B \lderive{\ell_2} \beta$}
      \AxiomC{$\alpha \in \Sigma^\star$}
      \RightLabel{\scriptsize{\IndHyp}}
      \TrinaryInfC{$\lambda\alpha'\rho \lderive{\ell_1+\ell_2-1} \alpha\beta\gamma$}
      \RightLabel{\scriptsize{(1)}}
      \TrinaryInfC{$\underbrace{\lambda A \rho}_{\delta} \lderive{\ell_1+\ell_2} \alpha\beta\gamma$}
    \end{prooftree}
  \end{itemize}  
\end{proof}

\begin{proposition}\label{pr:left-derivation:concat2}
  За произволни $\ell_1$ и $\ell_2$ имаме извода:
  \begin{prooftree}
    \AxiomC{$\alpha_1 \lderive{\ell_1} \beta_1$}
    \AxiomC{$\beta_1 \in \Sigma^\star$}
    \AxiomC{$\alpha_2 \lderive{\ell_2} \beta_2$}
    \TrinaryInfC{$\alpha_1\alpha_2 \lderive{\ell_1+\ell_2} \beta_1\beta_2$}
  \end{prooftree}
\end{proposition}


Следваща \Lemma{left-derivation-equivalence} е важна, защото тя на практика ни казва, че няма значение в какъв ред ще заместваме променливите в един извод на безконтекстна граматика.

\begin{important}
  \begin{lemma}\label{lem:left-derivation-equivalence}
% \mynote{Важно е да отбележим, че имаме горната еквивалентност само когато $\alpha \in \Sigma^\star$. Съобразете сами защо тази лема не е вярна в общия случай, когато позволим $\alpha$ да съдържа променливи.}
    За всяка безконтекстна граматика $G$, променлива $A$ и дума $\alpha \in \Sigma^\star$,
    \[A \lderive{\star} \alpha\text{ точно тогава, когато } A \derive{\star} \alpha.\]
    В частност, $\L(G) = \{\alpha\in\Sigma^\star \mid S \lderive{\star} \alpha\}$.
  \end{lemma}
\end{important}
\begin{proof}
  \mynote{Правило $(1)$ за $\lderive{\ell}$ е частен случай на правило $(1)$ за $\derive{\ell}$} 
  От правилата за извод е видно, че ако $A \lderive{\star} \alpha$, то $A \derive{\star} \alpha$.

  За другата посока, според \Theorem{grammar:yield-derive-equivalent} е достатъчно да се докаже, че ако $A \yield{\star} \alpha$ то $A \lderive{\star} \alpha$. С пълна индукция по $\ell$ ще докажем, че за произволно число $\ell$, ако $A \yield{\ell}\alpha$, то $A \lderive{\star}\alpha$.
  Да отбележим, че щом $A \in V$, а $\alpha \in \Sigma^\star$ е ясно, че $\ell > 0$.
  Това означава, че $A \yield{\ell}\alpha$ е получен чрез правило $(1)$ от дефиницията на релацията $\yield{\ell}$, т.е.
  \begin{prooftree}
    % \AxiomC{$A \to_G \alpha_1B_1\cdots\alpha_n B_n\alpha_{n+1}$}
    \AxiomC{$A \to_G X_1X_2\cdots X_n$}
    \AxiomC{$X_1 \yield{\ell_1} \alpha_1$}
    \AxiomC{$\cdots$}
    \AxiomC{$X_n \yield{\ell_n} \alpha_n$}
    \RightLabel{\scriptsize{(1)}}
    \QuaternaryInfC{$A \yield{\ell} \underbrace{\alpha_1\cdots\alpha_n}_{\alpha}$,}
  \end{prooftree}
  където $X_i \in V \cup \Sigma$ и $\ell = 1+\max\{\ell_1,\dots,\ell_n\}$.
  Започваме така:
  \begin{prooftree}
    \AxiomC{$A \to_G X_1X_2\cdots X_n$}
    \AxiomC{$X_1 \yield{\ell_1} \alpha_1$}
    \RightLabel{\scriptsize{\IndHyp}}
    \UnaryInfC{$X_1 \lderive{^\star}\alpha_1$}
    \LeftLabel{\scriptsize{(\Proposition{left-derivation:context-step})}}
    \BinaryInfC{$A \lderive{^\star}\alpha_1 X_2 \cdots X_n$}
  \end{prooftree}
  Продължаваме последователно за всяко $i < n$ правим да правим извода:
  \begin{prooftree}
    \AxiomC{$A \lderive{\star} \overbrace{\alpha_1\cdots\alpha_{i-1}}^{\in\Sigma^\star}X_{i}X_{i+1}\cdots X_n$}
    \AxiomC{$X_i \yield{\ell_{i+1}} \alpha_i$}
    \RightLabel{\scriptsize{\IndHyp}}
    \UnaryInfC{$X_i \lderive{^\star}\beta_i$}
    \LeftLabel{\scriptsize{(\Proposition{left-derivation:context-step})}}
    \BinaryInfC{$A \lderive{\star}\alpha_1\cdots\alpha_{i-1}\alpha_i X_{i+1} \cdots X_n$}
  \end{prooftree}
  Завършваме със следния извод:
  \begin{prooftree}
    \AxiomC{$A \lderive{\star} \overbrace{\alpha_1\alpha_2\cdots \alpha_{n-1}}^{\in\Sigma^\star} X_n$}
    \AxiomC{$X_n \yield{\ell_n} \alpha_n$}
    \RightLabel{\scriptsize{\IndHyp}}
    \UnaryInfC{$X_n \lderive{\star}\alpha_n$}
    \LeftLabel{\scriptsize{(\Proposition{left-derivation:context-step})}}
    \BinaryInfC{$A \lderive{^\star}\underbrace{\alpha_1\alpha_2\cdots\alpha_{n-1}\alpha_n}_{\alpha}$}
  \end{prooftree}
\end{proof}



\subsection*{Най-десен извод}

\begin{extra}
  \index{граматика!най-десен извод}

  \mynote{Единствената разлика между дефиницията на $\rderive{\star}$ и тази на $\derive{\star}$ е, че тук изискваме $\rho \in \Sigma^\star$.}

  За две думи $\alpha,\beta \in (V\cup\Sigma)^\star$, дефинираме {\bf най-десен извод} в граматиката $G$, $\alpha \rderive{\ell} \beta$, по следния начин:

  \begin{important}
    \begin{figure}[H]
      \begin{subfigure}[b]{0.4\textwidth}
        \begin{prooftree}
          \AxiomC{}
          \RightLabel{\scriptsize{правило (0)}}
          \UnaryInfC{$\alpha \rderive{0} \alpha$}
        \end{prooftree}
      \end{subfigure}
      ~
      \begin{subfigure}[b]{0.4\textwidth}
        \begin{prooftree}
          \AxiomC{$(A,\alpha) \in R$}
          \AxiomC{$\lambda \alpha \rho \rderive{\ell} \beta$}
          \AxiomC{$\rho \in \Sigma^\star$}
          \RightLabel{\scriptsize{правило (1)}}
          \TrinaryInfC{$\lambda A \rho \rderive{\ell+1} \beta$}
        \end{prooftree}
      \end{subfigure}
      \caption{Правила за най-десен извод в безконтекстна граматика.}
    \end{figure}
  \end{important}

  
%   \begin{framed}
%   \begin{figure}[H]
%     \begin{subfigure}[b]{0.5\textwidth}
%       \begin{prooftree}
%         \AxiomC{}
%         \RightLabel{\scriptsize{правило (0)}}
%         \UnaryInfC{$\alpha \rderive{0} \alpha$}
%       \end{prooftree}
%     \end{subfigure}
%     ~
%     \begin{subfigure}[b]{0.5\textwidth}
%       \begin{prooftree}
%         \AxiomC{$A \to_G \alpha$}
%         \AxiomC{$\alpha \rderive{\ell} \beta$}
%         \RightLabel{\scriptsize{правило (1)}}
%         \BinaryInfC{$A \rderive{\ell+1} \beta$}
%       \end{prooftree}
%     \end{subfigure}
%     \center
% \begin{prooftree}
%   \AxiomC{$\alpha_1 \rderive{\ell_1} \beta_1$}
%   \AxiomC{$\alpha_2 \rderive{\ell_2} \beta_2$}
%   \AxiomC{$\beta_2 \in \Sigma^\star$}
%   \RightLabel{\scriptsize{правило (2)}}
%   \TrinaryInfC{$\alpha_1\alpha_2 \rderive{\ell_1+\ell_2}\beta_1\beta_2$}
% \end{prooftree}
% \caption{Правила за най-десен извод в безконтекстна граматика.}
% \end{figure}
% \end{framed}

% \mynote{Интуитивно, $\yield{\star}$ е аналог на BFS, докато $\rderive{\star}$ е аналог на DFS като винаги се избира най-десния необходен клон.}

Удобно ще бъде да разгледаме следния аналог на \Proposition{unrestricted-grammar:context-general-step}.
\begin{problem}\label{prob:grammar:context-right-step}
  Докажете, че имаме извода:
  \begin{prooftree}
    \AxiomC{$\delta \rderive{\ell_1} \lambda B \rho$}
    \AxiomC{$B \rderive{\ell_2}\beta$}
    \AxiomC{$\rho \in \Sigma^\star$}
    \TrinaryInfC{$\delta \rderive{\ell_1+\ell_2}\lambda\beta\rho$}
  \end{prooftree}
\end{problem}

\begin{problem}
  Докажете, че за всяка безконтекстна граматика $G$, променлива $A$ и дума $\alpha \in \Sigma^\star$,
  \[A \rderive{\star} \alpha\text{ точно тогава, когато } A \derive{\star} \alpha.\]
\end{problem}

\end{extra}


%%% Local Variables:
%%% mode: latex
%%% TeX-master: "../eai"
%%% End:
