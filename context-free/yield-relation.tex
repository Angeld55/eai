\section{Извод върху синтактично дърво}

\mynote{Не знам да има учебник, в който формално да е въведена тази релация. Тя е удобна най-вече за решаване на задачи, както и за доказателството на лемата за покачването.}
\begin{definition}
  За произволна безконтекстна граматика $G$, дефинираме релацията $X \yield{\ell} \alpha$, където $X \in V \cup \Sigma$ и $\alpha \in (V\cup\Sigma)^\star$, по следния начин:
  \begin{important}
    \begin{prooftree}
      \AxiomC{}
      \RightLabel{\scriptsize{правило (0)}}
      \UnaryInfC{$X \yield{0} X$}
    \end{prooftree}
    
    \begin{prooftree}
      \AxiomC{$X \to_G X_1\cdots X_n$}
      \AxiomC{$X_1 \yield{\ell_1} \gamma_1$}
      \AxiomC{$\cdots$}
      \AxiomC{$X_n \yield{\ell_n} \gamma_n$}
      \LeftLabel{\scriptsize{($\ell = \sup\{\ell_1,\dots,\ell_n\})$}}
      \RightLabel{\scriptsize{правило (1)}}
      \QuaternaryInfC{$X \yield{\ell+1} \gamma_1\cdots\gamma_n$}
    \end{prooftree}
  \end{important}
\end{definition}

Да напомним, че по дефиниция, ако $n = 0$, то $X_1\cdots X_n = \varepsilon$.
Освен това, понеже $\sup(A)$ означава най-малкото естествено число по-голямо от всички елементи на $A$, то $\sup(\emptyset) = 0$.
Това означава, че ако в граматиката имаме правилото $A \to_G \varepsilon$, то според правило (1)
получаваме, че $A \yield{1} \varepsilon$.

\mynote{ Съобразете, че имаме:
\begin{prooftree}
  \AxiomC{$X \to_G \gamma$}
  \UnaryInfC{$X \yield{1} \gamma$.}
\end{prooftree}
Обърнете внимание също, че тази релация е рефликсивна, но не е транзитивна!}

Да дефинираме $\yield{\star}$ по следния начин:
\[X \yield{\star} \gamma \dff (\exists \ell\in\Nat)[X \yield{\ell} \gamma].\]

Релацията $\yield{\star}$ е удобна, когато не се интересуваме от реда, по който прилагаме правилата за извод в една безконтекстна граматика.

\begin{lemma}
  Нека $G$ е безконтекстна граматика, $X \in V \cup \Sigma$ и $\beta \in (V \cup \Sigma)^\star$.
  Тогава ако $X \derive{\star} \beta$, то $X \yield{\star} \beta$.
\end{lemma}  
\begin{proof}
  С пълна индукция по $\ell$ ще докажем, че ако $X \derive{\ell} \beta$, то $X \yield{\star} \beta$.
  \begin{itemize}
  \item
    $\ell = 0$, т.е. $X \derive{0} X$.
    Тогава е ясно, че $X \yield{\star} X$.
  \item
    Нека $\ell > 0$ и $X \derive{\ell} \beta$.
    Според правилата на извод в граматика имаме извода

    \begin{prooftree}
      \AxiomC{$X \to_G X_1\cdots X_k$}
      \AxiomC{$X_1\cdots X_k \derive{\ell-1} \beta$}
      % \RightLabel{\scriptsize{правило (1)}}
      \BinaryInfC{$X \derive{\ell} \beta$}
    \end{prooftree}

    \mynote{Естествено, че е възможно някои $X_i$ да са букви от $\Sigma$. Тогава $\beta_i = X_i$ и $X_i \derive{0}_G \beta_i$.}
    Щом имаме, че $X_1\cdots X_k \derive{\ell-1} \beta$, от \Proposition{grammar:divide} знаем, че съществува разбиване на $\beta$ на $k+1$ части, така че:
    \begin{itemize}
    \item
      $\beta = \beta_1 \cdots \beta_{k}$;
    \item
      $X_i \derive{\ell_i} \beta_i$, за всяко $i = 1,\dots,k$;
    \item
      $\ell-1 = \sum^k_{i=1} \ell_i$.
    \end{itemize}
    Понеже $\ell_i < \ell$ за всяко $i = 1,\dots,k$, получаваме следното:
    \begin{prooftree}
      \AxiomC{$X \to_G X_1\cdots X_k$}
      \AxiomC{$X_1 \derive{\ell_1} \beta_1$}
      \RightLabel{\scriptsize{\IndHyp}}
      \UnaryInfC{$X_1 \yield{\star} \beta_1$}
      \AxiomC{$\cdots$}
      \AxiomC{$X_k \derive{\ell_k} \beta_k$}
      \RightLabel{\scriptsize{\IndHyp}}
      \UnaryInfC{$X_k \yield{\star} \beta_k$}
      \RightLabel{\scriptsize{правило (1)}}
      \QuaternaryInfC{$X \yield{\star} \underbrace{\beta_1\cdots\beta_k}_{\beta}$}
    \end{prooftree}
  \end{itemize}
\end{proof}

\begin{lemma}
  Нека $G$ е безконтекстна граматика, $X \in V \cup \Sigma$ и $\gamma \in (V \cup \Sigma)^\star$.
  Тогава ако $X \yield{\star} \gamma$, то $X \derive{\star} \gamma$.
\end{lemma}
\begin{proof}
  С пълна индукция по $\ell$ ще докажем, че ако $X \yield{\ell} \gamma$, то $X \derive{\star} \gamma$.
  \begin{itemize}
  \item
    Нека $\ell = 0$. Това означава, че $X \yield{0} X$. Ясно е, че $X \derive{\star} X$.
  \item
    Нека $\ell > 0$. Тогава имаме следното:
    \begin{prooftree}
      \AxiomC{$X \to_G X_1\cdots X_n$}
      \AxiomC{$X_1 \yield{\ell_1} \gamma_1$}
      \AxiomC{$\cdots$}
      \AxiomC{$X_n \yield{\ell_n} \gamma_n$}
      \RightLabel{\scriptsize{($\ell = 1 + \sup\{\ell_1,\dots,\ell_n\})$}}
      \QuaternaryInfC{$X \yield{\ell} \underbrace{\gamma_1\cdots\gamma_n}_{\gamma}$}
    \end{prooftree}
    Понеже $\ell_i < \ell$ за всяко $i = 1,\dots,n$, получаваме следното:
    \begin{prooftree}
      \AxiomC{$X \to_G X_1 \cdots X_{n}$}
      \AxiomC{$X_1 \yield{\ell_1} \gamma_1$}
      \RightLabel{\scriptsize{\IndHyp}}
      \UnaryInfC{$X_1 \derive{\star} \gamma_1$}
      \AxiomC{$\cdots$}
      \AxiomC{$X_n \yield{\ell_n} \gamma_n$}
      \RightLabel{\scriptsize{\IndHyp}}
      \UnaryInfC{$X_n \derive{\star} \gamma_n$}
      \RightLabel{\scriptsize{(\ShortProposition{unrestricted-grammar:concat})}}
      \TrinaryInfC{$X_1 \cdots X_{n} \derive{\star} \gamma_1\cdots\gamma_{n}$}
      % \RightLabel{\scriptsize{правило (1)}}
      \BinaryInfC{$X \derive{\star} \underbrace{\gamma_1\cdots\gamma_{n}}_{\gamma}$}
    \end{prooftree}
  \end{itemize}
\end{proof}

Комбинирайки предишните две леми получаваме следната теорема.
\begin{framed}
  \begin{theorem}\label{th:grammar:yield-derive-equivalent}
    Нека $G$ е безконтекстна граматика, $X \in V \cup \Sigma$ и $\gamma \in (V \cup \Sigma)^\star$.
    Тогава $X \yield{\star} \gamma$ точно тогава, когато $X \derive{\star} \gamma$.
    В частност,
    \[\L(G) = \{\alpha \in \Sigma^\star \mid S \yield{\star}\alpha\}.\]
  \end{theorem}  
\end{framed}

Следващото твърдение ни казва, че има пряка връзка между релацията $\yield{\star}$ и синтактичните дървета.

\begin{important}
  \begin{proposition}\label{pr:yield-relation:parse-tree}
    Нека $G$ е безконтекстна граматика. Тогава
    $X \yield{\ell} \gamma$ точно тогава, когато съществува дърво на извод $P$ съвместимо с $G$, за което
    $\texttt{root}(P) = X$, $\texttt{yield}(P) = \gamma$ и $\texttt{height}(P) = \ell$.
  \end{proposition}
\end{important}
\begin{hint}
  Индукция по $\ell$.
\end{hint}


%%% Local Variables:
%%% mode: latex
%%% TeX-master: "../eai"
%%% End:
