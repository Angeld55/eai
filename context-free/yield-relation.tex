\section{Извод върху синтактично дърво}

За произволна безконтекстна граматика $G$, дефинираме релацията $X \yield{\ell} \alpha$, където $X \in V \cup \Sigma$ и $\alpha \in (V\cup\Sigma)^\star$, по следния начин:
\begin{prooftree}
  \AxiomC{}
  \RightLabel{\scriptsize{правило (0)}}
  \UnaryInfC{$X \yield{0} X$}
\end{prooftree}

\begin{prooftree}
  \AxiomC{$X \to_G X_1\cdots X_n$}
  \AxiomC{$X_1 \yield{\ell_1} \gamma_1$}
  \AxiomC{$\cdots$}
  \AxiomC{$X_n \yield{\ell_n} \gamma_n$}
  \LeftLabel{\scriptsize{($\ell = \max\{\ell_1,\dots,\ell_n\})$}}
  \RightLabel{\scriptsize{правило (1)}}
  \QuaternaryInfC{$X \yield{\ell+1} \gamma_1\cdots\gamma_n$}
\end{prooftree}

\mynote{ Съобразете, че имаме:
\begin{prooftree}
  \AxiomC{$X \to_G \gamma$}
  \UnaryInfC{$X \yield{1} \gamma$.}
\end{prooftree}
Обърнете внимание също, че тТази релация е рефликсивна, но не е транзитивна!}

Да дефинираме $\yield{\star}$ по следния начин:
\[X \yield{\star} \gamma \dff (\exists \ell\in\Nat)[X \yield{\ell} \gamma].\]

\begin{proposition}
  За произволни променливи $A$, $B$ и думи $\alpha_1,\alpha_2, \alpha_3$ е изпълнено, че:
  \begin{prooftree}
    \AxiomC{$A \yield{\ell_1}\alpha_1B\alpha_3$}
    \AxiomC{$B \yield{\ell_2} \alpha_2$}
    \BinaryInfC{$A \yield{\ell_1+\ell_2} \alpha_1\alpha_2\alpha_3$}
  \end{prooftree}
\end{proposition}
\begin{hint}
  Индукция по $\ell_1$.
\end{hint}

\begin{proposition}
  За произволна променлива $A$ и произволни думи $\beta$ и $\gamma$ е изпълнено, че:
  \begin{prooftree}
    \AxiomC{$A \yield{\ell} \beta A \gamma$}
    \AxiomC{$i \in \Nat$}
    \BinaryInfC{$A \yield{\ell \cdot i} \beta^i A \gamma^i$}
  \end{prooftree}
  
\end{proposition}
\begin{hint}
  Индукция по $i$.
\end{hint}

\begin{lemma}
  Нека $G$ е безконтекстна граматика, $X \in V \cup \Sigma$ и $\beta \in (V \cup \Sigma)^\star$.
  Тогава ако $X \derive{\star}_G \beta$, то $X \yield{\star}_G \beta$.
\end{lemma}  
\begin{proof}
  С пълна индукция по $\ell$ ще докажем, че ако $X \derive{\ell} \beta$, то $X \yield{\star} \beta$.
  \begin{itemize}
  \item
    $\ell = 0$, т.е. $X \derive{0} X$.
    Тогава е ясно, че $X \yield{\star} X$.
  \item
    Нека $\ell > 0$ и $X \derive{\ell} \beta$.
    Според правилата на извод в граматика имаме извода

    \begin{prooftree}
      \AxiomC{$X \to_G X_0X_1\cdots X_k$}
      \AxiomC{$X_0X_1\cdots X_k \derive{\ell-1} \beta$}
      \RightLabel{\scriptsize{правило (1)}}
      \BinaryInfC{$X \derive{\ell} \beta$}
    \end{prooftree}

    От \Proposition{grammar:divide} знаем, че съществува разбиване на $\beta$ на $k+1$ части, така че:
    \begin{itemize}
    \item
      $\beta = \beta_0 \cdots \beta_{k}$;
    \item
      $X_i \derive{\ell_i} \beta_i$, за всяко $i = 0,\dots,k$;
    \item
      $\ell-1 = \sum^k_{i=1} \ell_i$.
    \end{itemize}
    % От \IndHyp имаме, че $X_i \yield{\star} \beta_i$ за всяко $i = 0,1, \dots, k$.
    Получаваме следното:
    \begin{prooftree}
      \AxiomC{$X \to_G X_0\cdots X_k$}
      \AxiomC{$X_0 \derive{\ell_0} \beta_0$}
      \RightLabel{\scriptsize{\IndHyp}}
      \UnaryInfC{$X_0 \yield{\star} \beta_0$}
      \AxiomC{$\cdots$}
      \AxiomC{$X_k \derive{\ell_k} \beta_k$}
      \RightLabel{\scriptsize{\IndHyp}}
      \UnaryInfC{$X_k \yield{\star} \beta_k$}
      \RightLabel{\scriptsize{правило (1)}}
      \QuaternaryInfC{$X \yield{\star} \underbrace{\beta_0\cdots\beta_k}_{\beta}$}
    \end{prooftree}
  \end{itemize}
\end{proof}

\begin{lemma}
  Нека $G$ е безконтекстна граматика, $X \in V \cup \Sigma$ и $\gamma \in (V \cup \Sigma)^\star$.
  Тогава ако $X \yield{\star}_G \gamma$, то $X \derive{\star}_G \gamma$.
\end{lemma}
\begin{proof}
  С пълна индукция по $\ell$ ще докажем, че ако $X \yield{\ell} \gamma$, то $X \derive{\star} \gamma$.
  \begin{itemize}
  \item
    Нека $\ell = 0$. Това означава, че $X \yield{0} X$. Ясно е, че $X \derive{\star} X$.
  \item
    Нека $\ell > 0$. Тогава имаме следното:
    \begin{prooftree}
      \AxiomC{$X \to_G X_0\cdots X_n$}
      \AxiomC{$X_0 \yield{\ell_0} \gamma_0$}
      \AxiomC{$\cdots$}
      \AxiomC{$X_n \yield{\ell_n} \gamma_n$}
      \RightLabel{\scriptsize{($\ell = 1 + \max\{\ell_0,\dots,\ell_n\})$}}
      \QuaternaryInfC{$X \yield{\ell} \gamma_0\cdots\gamma_n$}
    \end{prooftree}
    Получаваме, че:
    \begin{prooftree}
      \AxiomC{$X \to_G X_0 \cdots X_{n}$}
      \AxiomC{$X_0 \yield{\ell_0} \gamma_0$}
      \RightLabel{\scriptsize{\IndHyp}}
      \UnaryInfC{$X_0 \derive{\star} \gamma_0$}
      \AxiomC{$\cdots$}
      \AxiomC{$X_n \yield{\ell_n} \gamma_n$}
      \RightLabel{\scriptsize{\IndHyp}}
      \UnaryInfC{$X_n \derive{\star} \gamma_n$}
      \RightLabel{\scriptsize{(\Proposition{unrestricted-grammar:concat})}}
      \TrinaryInfC{$X_0 \cdots X_{n} \derive{\star} \gamma_0\cdots\gamma_{n}$}
      \RightLabel{\scriptsize{правило (1)}}
      \BinaryInfC{$X \derive{\star} \underbrace{\gamma_0\cdots\gamma_{n}}_{\gamma}$}
    \end{prooftree}
  \end{itemize}
\end{proof}

Комбинирайки предишните две леми получаваме следната теорема.
\begin{framed}
  \begin{theorem}\label{th:grammar:yield-derive-equivalent}
    Нека $G$ е безконтекстна граматика, $X \in V \cup \Sigma$ и $\beta \in (V \cup \Sigma)^\star$.
    Тогава $X \yield{\star}_G \gamma$ точно тогава, когато $X \derive{\star}_G \gamma$.
    В частност,
    \[\L(G) = \{\alpha \in \Sigma^\star \mid S \yield{\star}\alpha\}.\]
  \end{theorem}  
\end{framed}

Да напомним, че в Раздел~\ref{sect:regular:minimisation} дефинирахме език $\L_\A(q)$.
Сега ще дефинираме език $\L_G(A)$, за произволна променлива $A$.
\[\L_G(A) \df \{\alpha \in \Sigma^\star \mid A \yield{\star}\alpha\}.\]
Ясно е, че $\L(G) = \L_G(S)$.
Също така, да дефинираме апроксимациите $\L^\ell_G(A)$ на $\L_G(A)$ по следния начин:
\[\L^\ell_G(A) \df \{\alpha \in \Sigma^\star \mid A \yield{\leq \ell} \alpha\}.\]
Следните свойства са ясни:
\begin{itemize}
\item
  $\L^0_G(A) = \emptyset$;
\item
  $\L^n_G(A)$ е краен език за всяко $n$;
\item
  $\L^n_G(A) \subseteq \L^{n+1}_G(A)$ за всяко $n$;
\item
  $\L_G(A) = \bigcup_{n\geq 0}\L^n_G(A)$.  
\end{itemize}

Следната характеризация на $\L^\ell_G(A)$ ще е важна за нас, когато искаме да докажем,
че една безконтекстна граматика разпознава даден език.

\begin{proposition}\label{pr:grammar:yield-approximation}
  Нека $G$ е произволна безконтекстна граматика и $A$ е променлива в $G$.
  Тогава имаме следното:
  \begin{align*}
    \L^0_G(A) & = \emptyset\\
    \L^{\ell+1}_G(A) & = \bigcup\{\{\alpha_1\}\cdot \L^\ell_G(A_1) \cdots \{\alpha_n\} \cdot \L^\ell_G(A_n) \cdot \{\alpha_{n+1}\} \mid A \to_G \alpha_1A_1\cdots\alpha_nA_n\alpha_{n+1}\}.
  \end{align*}
\end{proposition}
\begin{proof}
  Пълна индукция по $\ell$. Твърдението очевидно е изпълнено за $\ell = 0$.
  Нека $\alpha \in \L^{\ell+1}_G(A)$, т.е. $A \yield{\leq \ell+1} \alpha$. Имаме следния извод:
  \begin{prooftree}
    \AxiomC{$A \to_G \alpha_1B_1\cdots\alpha_n B_n\alpha_{n+1}$}
    \AxiomC{$B_1 \yield{\leq\ell} \beta_1$}
    \AxiomC{$\cdots$}
    \AxiomC{$B_n \yield{\leq \ell} \beta_n$}
    \QuaternaryInfC{$A \yield{\leq\ell+1} \underbrace{\alpha_1\beta_1\cdots\alpha_n\beta_n\alpha_{n+1}}_{\alpha}$}
  \end{prooftree}
  Понеже $B_i \yield{\leq \ell} \beta_i$, то от \IndHyp имаме, че $\beta_i \in \L^\ell_G(B_i)$.
  Тогава 
  \[\alpha \in \bigcup\{\{\alpha_1\}\cdot \L^\ell_G(A_1) \cdots \{\alpha_n\} \cdot \L^\ell_G(A_n) \cdot \{\alpha_{n+1}\} \mid A \to_G \alpha_1A_1\cdots\alpha_nA_n\alpha_{n+1}\}\]

  Обратно, нека сега $\alpha = \alpha_1\beta_1\cdots\alpha_n\beta_n\alpha_{n+1}$, където $A \to_G \alpha_1B_1\cdots\alpha_nB_n\alpha_{n+1}$,
  $\beta_i \in \L^\ell_G(B_i)$ и $\alpha = \alpha_1\beta_1\cdots\alpha_n\beta_n\alpha_{n+1}$.
  Получаваме следния извод:
  \begin{prooftree}
    \AxiomC{$A \to_G \alpha_1B_1\cdots\alpha_nB_n\alpha_{n+1}$}
    \AxiomC{$\beta_1 \in \L^\ell_G(B_1)$}
    \RightLabel{\scriptsize{\IndHyp}}
    \UnaryInfC{$B_1 \yield{\leq \ell }\beta_1 $}
    \AxiomC{$\cdots$}
    \AxiomC{$\beta_n \in \L^\ell_G(B_n)$}
    \RightLabel{\scriptsize{\IndHyp}}
    \UnaryInfC{$B_n \yield{\leq \ell }\beta_n$}
    \QuaternaryInfC{$A \yield{\leq \ell+1} \underbrace{\alpha_1\beta_1\cdots\alpha_n\beta_n\alpha_{n+1}}_{\alpha}$}
  \end{prooftree}
\end{proof}



\begin{example}
  \begin{align*}
    & S \to \texttt{if } S \texttt{ then } S \texttt{ else }S\ |\ \texttt{ if }S \texttt{ then }S\ |\ V\\
    & V \to x\ |\ y\ |\ z
  \end{align*}

  Ние искаме следната граматика:
  \begin{align*}
    & S \to M\ |\ U\\
    & M \to \texttt{if } S \texttt{ then } M \texttt{ else }M\ |\ X\\
    & U \to \texttt{if } S \texttt{ then } S\ |\ \texttt{if } S \texttt{ then } M \texttt{ else }U
  \end{align*}
\end{example}

\begin{example}
  Да разгледаме граматика с правила
  \begin{align*}
    & S \to E\\
    & E \to E + P\ |\ P\\
    & P \to P * N\ |\ N\\
    & N \to (E)\ |\ a.
  \end{align*}
\end{example}



%%% Local Variables:
%%% mode: latex
%%% TeX-master: "../eai"
%%% End:
