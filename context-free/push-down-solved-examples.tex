\subsection{Примери}

\begin{example}
  \label{ex:anbn}
  За езика $L = \{a^nb^n\mid n\in\Nat\}$ съществува стеков автомат $P$, такъв че
  $L = \L(P)$.
  Да разгледаме
  \[P = \PDA,\] където
  \begin{itemize}
  \item
    $Q = \{q,p,f\}$;
  \item
    $\qstart = q$;
  \item
    $\qaccept = f$;
  \item
    $\Sigma = \{a,b\}$;
  \item
    $\Gamma = \{\sharp,a\}$;
  \item
    Релацията на преходите $\Delta$ има следната дефиниция:
    \begin{enumerate}[(1)]
    \item
      $\Delta(q,a,\sharp) = \{(q, a\sharp), (p, a\sharp)\}$;
    \item
      $\Delta(q,a,a) = \{(q, aa), (p, aa)\}$;
    \item 
      $\Delta(q,\varepsilon,\sharp) = \{(f,\varepsilon)\}$\quad \comment{трябва да разпознаем и $\varepsilon$};
    \item 
      $\Delta(p, b, a) = \{(p,\varepsilon)\}$;
    \item
      $\Delta(p, \varepsilon, \sharp) = \{(f, \varepsilon)\}$.
    \item
      За всички останали тройки $(r,x,y)$, нека $\Delta(r,x,y) = \emptyset$.
    \end{enumerate}
  \end{itemize}
  
  Да видим как думата $a^2b^2$ се разпознава от автомата с празен стек:
  \begin{align*}
    (q, a^2b^2, \sharp) & \vdash_P (q, ab^2, a\sharp) & \comment{\text{правило }(1)}\\
                        & \vdash_P (p, b^2, aa\sharp) & \comment{\text{правило }(2)}\\
                        & \vdash_P (p, b, a\sharp) & \comment{\text{правило }(4)}\\
                        & \vdash_P (p, \varepsilon, \sharp) & \comment{\text{правило }(4)}\\
                        & \vdash_P (f, \varepsilon, \varepsilon) & \comment{\text{правило }(5)}
  \end{align*}
  \mynote{\writedown Докажете, че $L = \L(P)$!}
  Получихме, че
  \[(\qstart, a^2b^2, \sharp) \vdash^\star_P (\qaccept, \varepsilon, \varepsilon),\]
  откъдето следва, че $a^2b^2 \in \L(P)$.

  \begin{enumerate}[a)]
  \item
    Докажете с индукция по $n$, че за всяко естествено число $n$,
    \begin{align*}
      & (q, a^n\beta, \sharp) \vdash^n_P (q, \beta, a^n\sharp)\\
      & (p, b^n, a^n\sharp) \vdash^n_P (p, \varepsilon,\sharp).
    \end{align*}
    Заключете, че $L \subseteq \L(P)$.
  \item
    \mynote{Индукция по броя на стъпките в изчислението на стековия автомат}
    Докажете, че за всеки три думи $\alpha,\beta, \gamma \in \{a,b\}^\star$ е изпълнено, че:
    \begin{align*}
      (q, \alpha\beta, \sharp) \vdash^n_P (q, \beta, \gamma\sharp)  & \implies \alpha = \gamma = a^n\\
      (p, \beta, \gamma\sharp) \vdash^n_P (p, \varepsilon, \sharp) & \implies \beta = b^n\ \&\ \gamma = a^n,
    \end{align*}
    Оттук заключете, че $\L(P) \subseteq L$.    
  \end{enumerate}
\end{example}

\begin{example}
  \label{ex:omega-omega-r}
  Езикът $L = \{\ \omega\omega^{\rev} \mid \omega \in \{a,b\}^\star\ \}$ се разпознава от стеков автомат
  \[P = \PDA,\] където:
  \begin{itemize}
  \item 
    $Q = \{q,p,f\}$;
  \item
    $\qstart = q$, $\qaccept = f$;
  \item
    $\Sigma = \{a,b\}$, $\Gamma = \{a, b, \sharp\}$;
  \item
    Функцията на преходите $\Delta$ има следната дефиниция:
    \mynote{За всички липсващи редове в дефиницията на $\Delta$ приемаме, че $\Delta$ връща $\emptyset$}
    \begin{enumerate}[(1)]
    \item 
      $\Delta(q, a, \sharp) = \{(q, a\sharp)\}$;
    \item 
      $\Delta(q, b, \sharp) = \{(q, b\sharp)\}$;
    \item
      $\Delta(q, \varepsilon, \sharp) = \{(q,\varepsilon)\}$;
    \item
      $\Delta(q, a, a) = \{(q, aa), (p, \varepsilon)\}$;
    \item
      $\Delta(q, a, b) = \{(q, ab)\}$;
    \item
      $\Delta(q, b, a) = \{(q, ba)\}$;
    \item
      $\Delta(q, b, b) = \{(q, bb), (p, \varepsilon)\}$;
    \item
      $\Delta(p, a, a) = \{(p,\varepsilon)\}$;
    \item
      $\Delta(p, b, b) = \{(p,\varepsilon)\}$;
    \item
      $\Delta(p, \varepsilon, \sharp) = \{(f,\varepsilon)\}$;
    \end{enumerate}
  \end{itemize}
  Основното наблюдение, което трябва да направим за да разберем конструкцията на автомата е, че
  всяка дума от вида $\omega\omega^{\texttt{rev}}$ може да се запише като $\omega_1aa\omega^{\texttt{rev}}_1$ или $\omega_1bb\omega^{\texttt{rev}}_1$.
  Да видим защо $P$ разпознава думата $abaaba$ с празен стек.
  Започваме по следния начин:
  \begin{align*}
    (q, abaaba,\sharp) & \vdash_P (q, baaba, a\sharp)   & \comment{\text{правило }(1)}\\
                       & \vdash_P (q, aaba, ba\sharp)   & \comment{\text{правило }(6)}\\
                       & \vdash_P (q, aba,  aba\sharp). & \comment{\text{правило }(5)}
  \end{align*}
  Сега можем да направим два избора как да продължим. Състоянието $p$ служи за маркер, което ни казва, че вече сме започнали 
  да четем $\omega^{\texttt{rev}}$. Поради тази причина, продължаваме така:
  \begin{align*}
    (q, aba, aba\sharp) & \vdash_P (p, ba, ba\sharp) & \comment{\text{правило }(4)}\\
                        & \vdash_P (p, a, a\sharp) & \comment{\text{правило }(9)}\\
                        & \vdash_P (p, \varepsilon, \sharp) & \comment{\text{правило }(8)}\\
                        & \vdash_P (f,\varepsilon,\varepsilon). & \comment{\text{правило }(10)}
  \end{align*}
  Да проиграем още един пример. Да видим защо думата $aba$ не се извежда от автомата.
  \begin{align*}
    (q, aba, \sharp) & \vdash_P (q, ba, a\sharp) & \comment{\text{правило }(1)}\\
                     & \vdash_P (q, a, ba\sharp) & \comment{\text{правило }(6)}\\
                     & \vdash_P (q, \varepsilon, aba\sharp). & \comment{\text{правило }(5)}
  \end{align*}
  От последното моментно описание на автомата нямаме нито един преход, следователно
  думата $aba$ не се разпознава от $P$.
  \mynote{\writedown Докажете, че $L = \L(P)$!}
  \begin{enumerate}[a)]
  \item
    \mynote{Индукция по дължината на думата $\alpha$}
    Докажете, че за всеки две думи $\alpha, \beta \in \{a,b\}^\star$ е изпълнено, че:
    \begin{align*}
      & |\alpha| = n\ \implies\ (q, \alpha\beta, \sharp) \vdash^n_P (q, \beta, \alpha^{\texttt{rev}}\sharp)\\
      & |\beta| = n\ \implies\ (p, \beta, \beta\sharp) \vdash^n_P (p, \varepsilon, \sharp).
    \end{align*}
    Оттук заключете, че $L \subseteq \L(P)$.
  \item
    \mynote{Индукция по броя на стъпките в изчислението на стековия автомат}
    Докажете, че за всеки три думи $\alpha,\beta, \gamma \in \{a,b\}^\star$ е изпълнено, че:
    \begin{align*}
      (q, \alpha\beta, \sharp) \vdash^n_P (q, \beta, \gamma\sharp) & \implies \gamma = \alpha^{\rev}\ \&\ |\alpha| = n\\
      (p, \beta, \gamma\sharp) \vdash^n_P (p, \varepsilon, \sharp) & \implies \gamma = \beta\ \&\ |\beta| = n.
    \end{align*}
    Оттук заключете, че $\L(P) \subseteq L$.
  \end{enumerate}
\end{example}

\begin{example}
  Безконтекстният език
  \[L = \{\ \omega \in \{\texttt{[},\texttt{]}\}^\star \mid \texttt{left}(\omega) = \texttt{right}(\omega)\ \}\]
  се разпознава от стековия автомат
  \[P = \PDA,\] където:
  \begin{itemize}
  \item 
    $Q = \{q,f\}$;
  \item
    $\Sigma = \{\texttt{[},\texttt{]}\}$;
  \item
    $\Gamma = \{\texttt{[}, \texttt{]}, \sharp\}$;
  \item
    $\qstart = q$;
  \item
    $\qaccept = f$;
  \item
    Можем да дефинираме релацията на преходите $\Delta$ по следния начин:
    \begin{enumerate}[(1)]
    \item 
      $\Delta(q, \varepsilon, \sharp) = \{(f, \varepsilon)\}$;
    \item
      $\Delta(q, \texttt{[}, \sharp) = \{(q, \texttt{[}\sharp)\}$;
    \item
      $\Delta(q, \texttt{]}, \sharp) = \{(q, \texttt{]}\sharp)\}$;
    \item
      $\Delta(q, \texttt{[}, \texttt{[}) = \{(q, \texttt{[[})\}$;
    \item
      $\Delta(q, \texttt{[}, \texttt{]}) = \{(q, \varepsilon)\}$;
    \item
      $\Delta(q, \texttt{]}, \texttt{[}) = \{(q, \varepsilon)\}$;
    \item
      $\Delta(q, \texttt{]}, \texttt{]}) = \{(q, \texttt{]]})\}$.
    \end{enumerate}
  \end{itemize}
  Да видим защо думата $\texttt{[]]][[} \in \L(P)$.
  \begin{align*}
    (q, \texttt{[]]][[}, \sharp) & \vdash_P (q,\ \texttt{]]][[},\ \texttt{[}\sharp) & \comment{\text{правило }(2)}\\
                                 & \vdash_P (q,\ \texttt{]][[},\ \sharp) & \comment{\text{правило }(6)}\\
                                 & \vdash_P (q,\ \texttt{][[},\ \texttt{]}\sharp) & \comment{\text{правило }(3)}\\
                                 & \vdash_P (q,\ \texttt{[[},\ \texttt{]]}\sharp) & \comment{\text{правило }(7)}\\
                                 & \vdash_P (q,\ \texttt{[},\ \texttt{]}\sharp) & \comment{\text{правило }(5)}\\
                                 & \vdash_P (q,\ \varepsilon,\ \sharp) & \comment{\text{правило }(5)}\\
                                 & \vdash_P (f,\ \varepsilon,\ \varepsilon). & \comment{\text{правило }(1)}
  \end{align*}

  \begin{enumerate}[a)]
  \item
    \mynote{Индукция по дължината на думата $\alpha$}
    Докажете, че за произволно естествено число $n$ и произволна дума $\alpha \in \{\texttt{[}, \texttt{]}\}^\star$, е изпълнено, че:
    \begin{align*}
      \texttt{[}^n\alpha \in L & \implies (q, \alpha, \texttt{[}^n\sharp) \vdash^\star_P (q, \varepsilon, \sharp)\\
      \texttt{]}^n\alpha \in L & \implies (q, \alpha, \texttt{]}^n\sharp) \vdash^\star_P (q, \varepsilon, \sharp).
    \end{align*}
    Оттук заключете, че $L \subseteq \L(P)$.
  \item
    \mynote{Индукция по броя на стъпките в изчислението на стековия автомат }
    Докажете, че за произволно естествено число $n$ и произволна дума $\alpha \in \{\texttt{[}, \texttt{]}\}^\star$, е изпълнено, че:
    \begin{align*}
      (q, \alpha, \texttt{[}^n\sharp) \vdash^\star_P (q, \varepsilon, \sharp) & \implies \texttt{[}^n\alpha \in L\\
      (q, \alpha, \texttt{]}^n\sharp) \vdash^\star_P (q, \varepsilon, \sharp) & \implies \texttt{]}^n\alpha \in L.
    \end{align*}
    Оттук заключете, че $\L(P) \subseteq L$.
  \end{enumerate}
\end{example}

\begin{example}
  Езикът
  \[L = \{\ \omega \in \{\texttt{[},\texttt{]}\}^\star \mid \omega\text{ е балансирана дума}\ \}\]
  се разпознава от стековия автомат $P = \PDA$, където:
  \begin{itemize}
  \item 
    $Q = \{q,f\}$;
  \item
    $\qstart = q$;
  \item
    $\qaccept = f$;
  \item
    $\Sigma = \{\texttt{[},\texttt{]}\}$;
  \item
    $\Gamma = \{\texttt{[}, \sharp\}$;
  \item
    Можем да дефинираме релацията на преходите $\Delta$ по следния начин:
    \mynote{\writedown Докажете, че $L = \L(P)$!}
    \begin{enumerate}[(1)]
    \item 
      $\Delta(q, \varepsilon, \sharp) = \{(f, \varepsilon)\}$;
    \item
      $\Delta(q, \texttt{[}, \sharp) = \{(q, \texttt{[}\sharp)\}$;
    \item
      $\Delta(q, \texttt{[}, \texttt{[}) = \{(q, \texttt{[[})\}$;
    \item
      $\Delta(q, \texttt{]}, \texttt{[}) = \{(q, \varepsilon)\}$;
    \end{enumerate}
  \end{itemize}  
  \begin{enumerate}[(a)]
  \item
    \mynote{Индукция по дължината на думата $\alpha$}
    Докажете, че за произволно естествено число $n$ и произволна дума $\alpha \in \{\texttt{[}, \texttt{]}\}^\star$, 
    е изпълнено, че:
    \[\texttt{[}^n\alpha \in L\ \implies (q, \alpha, \texttt{[}^n\sharp) \vdash^\star_P (q, \varepsilon, \sharp).\]
    Оттук заключете, че $L \subseteq \L(P)$.
  \item
    \mynote{Индукция по броя на стъпките в изчислението на стековия автомат.}
    Докажете, че за произволно естествено число $n$ и произволна дума $\alpha \in \{\texttt{[}, \texttt{]}\}^\star$, е изпълнено, че:
    \[(q,\alpha,\texttt{[}^n\sharp) \vdash^\star_P (q, \varepsilon, \sharp)\ \implies\ \texttt{[}^n\alpha \in L.\]
    Оттук заключете, че $\L(P) \subseteq L$.
  \end{enumerate}
\end{example}

%%% Local Variables:
%%% mode: latex
%%% TeX-master: "../eai"
%%% End:
