\begin{extra}
  Добре е да обърнем внимание, че е възможно, ако $A \yield{\ell} \alpha$, то да има няколко синтактични дървета с корен променливата $A$,
  които да извеждат думата $\alpha$ и да имат една и съща височина $\ell$.
  \begin{example}
    Да разгледаме граматика $G_0$ с правила
    \begin{align*}
      & S \to S + S\ |\ S * S\ |\ (S)\\
      & N \to 0\ |\ 1\ |\ \dots\ |\ 9
    \end{align*}
    \mynote{Такъв вид граматики се наричат \emph{нееднозначни} (на англ. \emph{ambiguous}). Ако всяка дума от езика има единствено синтактично дърво, то тази граматика се нарича \emph{еднозначна} (на англ. \emph{unambiguous}). От практическа гледна точка е важно свойство дали една граматика е еднозначна или не.}
    В тази граматика нямаме приоритет между $+$ или $*$. Например,
    думата $5 * 6 + 2$ има две различни синтактични дървета $P_1$ и $P_2$, като и двете имат височина $4$ и
    $\texttt{yield}(P_1) = \texttt{yield}(P_2) = 5*6+2$.
    
    \begin{figure}[H]
      \centering
      \qtreecenterfalse
      \Tree [.$S$ [.$S$ [.$N$ $5$ ] ] $*$ [.$S$ [.$S$ [.$N$ $6$ ] ] $+$ [.$S$ [.$N$ $2$ ] ] ] ]
      \hskip 0.6in
      \Tree [.$S$ [.$S$ [.$S$ [.$N$ $5$ ] ] $*$ [.$S$ [.$N$ $6$ ] ] ]  $+$  [.$S$ [.$N$ $2$ ] ] ]
      \caption{Две синтактични дървета в граматиката $G_0$ за думата $5 * 6 + 2$.}
    \end{figure}

    \mynote{\todo Колко различни синтактични дървета за думата $1 + 5*6 + 2$ може да намерите?}
    
    Искаме да модифицираме $G_0$, така че $*$ да има по-висок приоритет спрямо $+$.
    За целта ще разгледаме граматиката $G_1$, която ще поражда същия език като $G_0$, със следните правила:
    \begin{align*}
      & S \to E + S\ |\ E\\
      & E \to N * E\ |\ N\ |\ (S) * E\ |\ (S)\\
      & N \to 0\ |\ 1\ |\ \dots\ |\ 9
    \end{align*}
    Сега думата $5 * 6 + 2$ има само едно дърво на извод.
    \mynote{$*$ има по-висок приоритет от $+$, защото $*$ се среща по-близо до листата.}
    Тук операцията $*$ е с по-висок приоритет в сравнение с операцията $+$.
    \begin{figure}[H]
      \centering
      \Tree [.$S$ [.$E$ [.$N$ $5$ ] $*$ [.$E$ [.$N$ $6$ ] ] ] $+$ [.$S$ [.$E$ [.$N$ $2$ ] ] ] ]
      \caption{Единствено синтактично дърво в граматиката $G_1$ за думата $5 * 6 + 2$.}
    \end{figure}

  \end{example}

% \begin{example}
%   \mynote{Тези примери трябва да се обяснят по-подробно.}
%   Да разгледаме следната граматика:
%   \begin{align*}
%     & S \to \texttt{if } S \texttt{ then } S \texttt{ else }S\ |\ \texttt{ if }S \texttt{ then }S\ |\ C\\
%     & C \to E < E\ |\ E > E\ |\ E == E\ |\ E\\
%     & E \to F + E\ |\ F\\
%     & F \to N * F\ |\ N\ |\ (E) * F\ |\ (E)\\
%     & N \to 0\ |\ 1\ |\ \cdots\ |\ 9
%     \end{align*}

%   Да разгледаме думата
%   \[\texttt{if } x \texttt{ then } \texttt{ if }y \texttt{ then }u\ \texttt{ else }z.\]
%   Имаме многозначност. 
%   Ние искаме следната граматика:
%   \begin{align*}
%     & S \to M\ |\ U\\
%     & M \to \texttt{if } S \texttt{ then } M \texttt{ else }M\ |\ X\\
%     & U \to \texttt{if } S \texttt{ then } S\ |\ \texttt{if } S \texttt{ then } M \texttt{ else }U
%   \end{align*}
% \end{example}

% \begin{example}
%   Да разгледаме граматика с правила
%   \begin{align*}
%     & S \to E\\
%     & E \to E + P\ |\ P\\
%     & P \to P * N\ |\ N\\
%     & N \to (E)\ |\ a.
%   \end{align*}
% \end{example}

\end{extra}
