\section{Контекстни граматики}
\index{граматика!контекстна}
\mynote{На англ. context-sensitive.}
Казваме, че $G = (V,\Sigma,R,S)$ е {\bf контекстна граматика}, ако правилата на $G$ са от вида
\[\rho A \delta \to \rho \alpha \delta,\]
където $\rho,\delta \in (V\cup\Sigma)^\star$ и $\alpha \in (V\cup\Sigma)^+$.

\begin{example}
  Езикът $L = \{a^nb^nc^n \mid n > 0\}$ е контекстен.
\end{example}
\begin{hint}
  Разгледайте контекстната граматика $G$ зададена със следните правила:
  \begin{align*}
    & S \to aSBC\ |\ aBC\\
    & CB \to BC\\
    & aB \to ab\\
    & bB \to bb\\
    & bC \to bc\\
    & cC \to cc.
  \end{align*}

  Докажете, че:
  \begin{itemize}
  \item
    $S \derive{\star} a^n(BC)^n$;
  \item
    $(BC)^n \derive{\star} B^nC^n$;
  \item
    $aB^n \derive{\star} ab^n$;
  \item
    $bC^n \derive{\star} bc^n$.
  \end{itemize}
  Оттук следва, че $L \subseteq \L(G)$.
\end{hint}

\begin{proposition}
  Класът на безконтекстните езици строго се включва в класа на контекстните езици.
\end{proposition}

\begin{proposition}
  Всеки контекстен език е разрешим.
\end{proposition}

\begin{proposition}
  Съществува разрешим език, който не е контекстен.
\end{proposition}


%%% Local Variables:
%%% mode: latex
%%% TeX-master: "../eai"
%%% End:
