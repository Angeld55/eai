\section{Контекстни граматики}
\index{граматика!контекстна}
\mynote{На англ. context-sensitive \cite[стр. 223]{hopcroft1}. На български може да се преведат и като контекстнозависими граматики. В йерархията на Чомски това са граматиките от тип 1.
  В дефиницията, позволяваме и правилото $S \to \varepsilon$, ако искаме да включим $\varepsilon$ в езика, като тогава изискваме $S$ да не се среща в дясна страна на правило.}

Разглеждането на контекстни граматики излиза извън целите на този курс. Добавяме този раздел единствено за пълнота на изложението.
Казваме, че $G = (V,\Sigma,R,S)$ е {\bf контекстна граматика}, ако правилата на $G$ са от вида
$\lambda A \rho \to \lambda \alpha \rho$, където $\lambda,\rho \in (V\cup\Sigma)^\star$ и $\alpha \in (V\cup\Sigma)^+$.

\begin{example}
  Езикът $L = \{a^nb^nc^n \mid n > 0\}$ е контекстен.
\end{example}
\begin{extra2}
  \begin{hint}
    Разгледайте контекстната граматика $G$ зададена със следните правила:
    \begin{align*}
      & S \to aSBC\ |\ aBC\\
      & CB \to CZ\\
      & CZ \to WZ\\
      & WZ \to WC\\
      & WC \to BC\\
      & aB \to ab\\
      & bB \to bb\\
      & bC \to bc\\
      & cC \to cc.
    \end{align*}
    Докажете, че за всяко $n > 0$ е изпълено следното:
    \begin{itemize}
    \item
      $S \derive{n} a^n(BC)^n$;
    \item
      $CB \derive{4}_G BC$.
    \item
      $(BC)^n \derive{4(n-1)} B^nC^n$;
    \item
      $aB^n \derive{n} ab^n$;
    \item
      $bC^n \derive{n} bc^n$.
    \end{itemize}
    Оттук лесно можем да докажем, че $L \subseteq \L(G)$.    
  \end{hint}
\end{extra2}


% Естествен въпрос е дали съществува формализъм подобен на краен автомат, с който може да се опишат точно контекстните езици. Оказва се, че линейно ограничените автомати описват точно контекстните езици.

%%% Local Variables:
%%% mode: latex
%%% TeX-master: "../eai"
%%% End:
