\subsubsection*{Премахване на преименуващи правила}
\index{преименуващи правила}
Преименуващите правила са от вида $A \to B$.
Нека е дадена граматика $G = \CFG$.
Ще построим еквивалентна граматика $G'$ без преименуващи правила.
В началото нека в $R'$ да добавим всички правила от $R$, които не са преименуващи.
След това, за всякa променлива $A$, за която $A \to^\star_G B$,
ако $B \to \alpha$ е правило в $R$, което не е преименуващо,
то добавяме към $R'$ правилото $A \to \alpha$.

\begin{lemma}
  Съществува {\em полиномиален} алгоритъм, такъв че превръща всяка безконтекстна граматика $G$ в безконтекстна граматика $G'$ без преименуващи правила
  и $\L(G') = \L(G) \setminus \{\varepsilon\}$.
\end{lemma}
\begin{hint}
  Първо намираме множеството от двойки
  \[\texttt{Ren} = \{(A,B) \in V\times V \mid A \to^\star_G B\}\]
  като строим множествата
  \[\texttt{Ren[n]} = \{(A,B) \in V\times V \mid A \stackrel{\leq n}{\to}_G B\}\]
  
  \begin{itemize}
  \item
    Ясно е, че $\texttt{Ren[0]} \df \{(A,A) \mid A \in V\}$.
  \item
    Нека имаме $\texttt{Ren[n]}$. Тогава дефинираме
    \marginpar{Всяка итерация отнема $\mathcal{O}(|G|)$ време.}
    \[\texttt{Ren[n+1]} \df \{(A,B) \in V\times V \mid (\exists C\in V)[ A \to_G C\ \&\ (C,B) \in \texttt{Ren[n]}]\}.\]
  \item
    \marginpar{$|\texttt{Ren}|$ има големина $\mathcal{O}(|G|^2)$.}
    Спираме на първото $n$, за което $\texttt{Ren[n]} = \texttt{Ren[n+1]}$. Тогава $\texttt{Ren} = \texttt{Ren[n]}$.
  \end{itemize}
  
  Нека $R'_0 \df R \setminus \texttt{Ren}$ са правилата на $R$, които не са преименуващи. Тогава
  \[R' \df R'_0 \cup \{\ (A,\alpha) \in V \times (V\cup\Sigma)^\star \mid (\exists B)[(A,B) \in \texttt{Ren}\ \&\ (B,\alpha) \in R'_0]\ \}.\]
\end{hint}

\begin{example}
  Нека е дадена граматиката $G$ с правила  
  \begin{align*}
    & S \to B\ |\ CC\ |\ b\\
    & A \to B\ |\ S\\
    & B \to C\ |\ BC\\
    & C \to AB\ |\ a\ |\ b.
  \end{align*}
  Първо добавяме към $R'$ правилата, които не са преименуващи, а именно:
  \begin{align*}
    & B \to BC\\
    & C \to AB\ |\ a\ |\ b\\
    & S \to CC\ |\ b.
  \end{align*}
  \begin{itemize}
  \item 
    Лесно се съобразява, че $A \to^\star_G B,S,C$.
    Добавяме правилата 
    \[A \to BC\ |\ AB\ |\ a\ |\ b\ |\ CC.\]
  \item
    Имаме $B \to^\star_G C$.
    Добавяме правилата $B \to AB\ |\ a\ |\ b$.
  \item
    Имаме $S \to^\star_G B,C$.
    Добавяме правилата $S \to BC\ |\ AB\ |\ a\ |\ b$.
  \end{itemize}
  Накрая получаваме, че граматиката $G'$ има правила
  \begin{align*}
    & S \to BC\ |\ AB\ |\ CC\ |\ a\ |\ b\\
    & A \to BC\ |\ AB\ |\ a\ |\ b\ |\ CC\\
    & B \to AB\ |\ a\ |\ b\ |\ BC\\
    & C \to AB\ |\ a\ |\ b.
  \end{align*}
\end{example}

\begin{problem}
  Премахнете преименуващите правила от граматиката $G$, като запазите езика, ако $G$ има следните правила:
    \begin{align*}
      & S \to C\ |\ CC\ |\ b\\
      & A \to B\\
      & B \to S\ |\ C\ |\ BC\\
      & C \to a\ |\ AB;
    \end{align*}
\end{problem}

%%% Local Variables:
%%% mode: latex
%%% TeX-master: "../eai"
%%% End:
