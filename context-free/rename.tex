\subsubsection*{Премахване на преименуващи правила}
\index{преименуващи правила}
Преименуващите правила са от вида $A \to B$.
Нека е дадена граматика $G = \CFG$.
Ще построим еквивалентна граматика $G'$ без преименуващи правила.
В началото нека в $R'$ да добавим всички правила от $R$, които не са преименуващи.
След това, за всякa променлива $A$, за която $A \derive{\star}_G B$,
ако $B \to \alpha$ е правило в $R$, което не е преименуващо,
то добавяме към $R'$ правилото $A \to \alpha$.

\begin{lemma}
  Съществува {\em полиномиален} алгоритъм, такъв че превръща всяка безконтекстна граматика $G$ в безконтекстна граматика $G'$ без преименуващи правила
  и $\L(G') = \L(G) \setminus \{\varepsilon\}$.
\end{lemma}
\begin{hint}
  Първо намираме множеството от двойки
  \[\texttt{Ren} = \{(A,B) \in V\times V \mid A \derive{\star} B\}\]
  като строим множествата по такъв начин, че:
  \[\texttt{Ren[n]} = \{(A,B) \in V\times V \mid A \derive{\leq n} B\}\]

  \begin{align*}
    & \texttt{Ren[0]} \df \{(A,A) \mid A \in V\}\\
    & \texttt{Ren[n+1]} \df \texttt{Ren[n]} \cup \{(A,B) \mid (\exists C)[ A \to_G C\ \&\ (C,B) \in \texttt{Ren[n]}]\}.
  \end{align*}
  
  % \begin{itemize}
  % \item
  %   % Ясно е, че $\texttt{Ren[0]} \df \{(A,A) \mid A \in V\}$.
  %   $
  % \item
  %   Нека имаме $\texttt{Ren[n]}$. Тогава дефинираме
  %   \mynote{Всяка итерация отнема $\mathcal{O}(|G|)$ време.}
  %   \[\texttt{Ren[n+1]} \df \texttt{Ren[n]} \cup \{(A,B) \mid (\exists C)[ A \to_G C\ \&\ (C,B) \in \texttt{Ren[n]}]\}.\]
  % \item
  \mynote{$|\texttt{Ren}|$ има големина $\mathcal{O}(|G|^2)$.}
  Спираме на първото $n$, за което $\texttt{Ren[n]} = \texttt{Ren[n+1]}$. Тогава $\texttt{Ren} = \texttt{Ren[n]}$.
  % \end{itemize}
  
  Нека $R'_0 \df R \setminus (V\times V)$ са правилата на $R$, които не са преименуващи. Тогава
  \[R' \df  \{\ (A,\alpha) \in V \times (V\cup\Sigma)^\star \mid (\exists B)[(A,B) \in \texttt{Ren}\ \&\ (B,\alpha) \in R'_0]\ \}.\]
\end{hint}

% \ExtraMaterial{
\begin{extra}
\begin{multicols}{2}
    \begin{example}
      Нека е дадена граматиката $G$ с правила  
      \begin{align*}
        & S \to B\ |\ CC\ |\ b\\
        & A \to S\ |\ SB\\
        & B \to C\ |\ BC\\
        & C \to AB\ |\ a\ |\ b.
      \end{align*}
      Да намерим първо $\texttt{Ren}$.
      \begin{align*}
        \texttt{Ren[0]} = \{ & (S,S), (A,A), (B,B), (C,C)\};\\
        \texttt{Ren[1]} = \{ & (S,S), (S,B), (A,A), (A,S),\\
                             & (B,B), (B,C), (C,C)\};\\
        \texttt{Ren[2]} = \{ & (S,S),(S,B), (S,C), (B,B), (B,C),\\
                             & (C,C), (A,A), (A,S), (A,B)\};\\
        \texttt{Ren[3]} = \{ & (S,S),(S,B), (S,C), (B,B), (B,C),\\
                             & (C,C), (A,A), (A,S), (A,B), (A,C)\};\\
        \texttt{Ren[4]} = \{ & (S,S),(S,B), (S,C), (B,B), (B,C),\\
                             & (C,C), (A,A), (A,S), (A,B), (A,C)\};\\
      \end{align*}
      
      Получихме, че $\texttt{Ren[3]} = \texttt{Ren[4]}$.
      Оттук можем да заключим следното:
      \begin{align*}
        & A \derive{\star} A,B,S,C\\
        & B \derive{\star} B,C\\
        & S \derive{\star} S,B,C\\
        & C \derive{\star} C.          
      \end{align*}
      % \end{itemize}
      
      Първо добавяме към $R'$ правилата, които не са преименуващи, а именно:
      \begin{align*}
        & A \to BS\\
        & B \to BC\\
        & C \to AB\ |\ a\ |\ b\\
        & S \to CC\ |\ b.
      \end{align*}
      \begin{itemize}
      \item 
        Понеже имаме,че $A \derive{\star} B,S,C$, то добавяме към $R'$ правилата:
        \[A \to BC\ |\ AB\ |\ a\ |\ b\ |\ CC.\]
      \item
        Понеже имаме,че $B \derive{\star}_G C$, то добавяме към $R'$ правилата:
        \[B \to AB\ |\ a\ |\ b.\]
      \item
        Понеже имаме, че $S \derive{\star}_G B,C$, то добавяме към $R'$ правилата:
        \[S \to BC\ |\ AB\ |\ a\ |\ b.\]
      \end{itemize}
      Накрая получаваме, че граматиката $G'$ има правила
      \begin{align*}
        & S \to BC\ |\ AB\ |\ CC\ |\ a\ |\ b\\
        & A \to BS\ |\ BC\ |\ AB\ |\ a\ |\ b\ |\ CC\\
        & B \to AB\ |\ a\ |\ b\ |\ BC\\
        & C \to AB\ |\ a\ |\ b.
      \end{align*}
    \end{example}
  \end{multicols}
\end{extra}

\begin{extra}
\begin{problem}
  Премахнете преименуващите правила от граматиката $G$, като запазите езика, ако $G$ има следните правила:
    \begin{align*}
      & S \to C\ |\ CC\ |\ b\\
      & A \to B\\
      & B \to S\ |\ C\ |\ BC\\
      & C \to a\ |\ AB;
    \end{align*}
\end{problem}
\end{extra}

%%% Local Variables:
%%% mode: latex
%%% TeX-master: "../eai"
%%% End:
