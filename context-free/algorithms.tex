\section{Алгоритми}

\subsection{Опростяване на безконтекстни граматики}

\subsubsection*{Премахване на безполезните променливи}

Нека е дадена безконтекстната граматика $G = \CFG$.
\marginpar{\cite[стр. 88]{hopcroft1}}
Една променлива $A$ се нарича {\bf полезна}, ако съществува извод от следния вид:
\[S \to^\star \alpha A \beta \to^\star \gamma,\]
където $\gamma \in \Sigma^\star$, а $\alpha,\beta \in (V \cup \Sigma)^\star$.
Това означава, че една променлива е полезна, ако участва в извода на някоя дума в езика на граматиката.
Една променлива се нарича {\bf безполезна}, ако не е полезна.
Целта ни е да получим еквивалентна граматика $G'$ без безполезни променливи.
Ще решим задачата като разгледаме две леми.

\begin{lemma}
  \label{lem:useless1}
  Нека е дадена безконтекстната граматика $G = \CFG$ и $\L(G) \neq \emptyset$.
  Съществува алгоритъм, който намира граматика $G' = \pair{V',\Sigma,S,R'}$, за която 
  $\L(G) = \L(G')$, и за всяка променлива $A' \in V'$, съществува дума $\alpha \in \Sigma^\star$,
  за която $A' \to^\star \alpha$.
\end{lemma}
\begin{hint}
  Да разгледаме следната проста итеративна процедура.
  \begin{algorithm}[H]
    \caption{Намираме $V' = \{A \in V\mid (\exists \alpha \in \Sigma^\star)[A \to^\star \alpha]\}$}
    \label{alg:useless}
    \begin{algorithmic}[1]
      \State $V' := \emptyset$
      \State $V'' := \{A \in V \mid (\exists \alpha \in \Sigma^\star)[A \to \alpha]\}$
      \While{$V' \neq V''$}
      \State $V' := V''$
      \State $V'' := V' \cup \{A \in V \mid (\exists \alpha \in (\Sigma \cup V')^\star)[A \to \alpha]\}$
      \EndWhile
      \State \Return $V'$
    \end{algorithmic}
  \end{algorithm}
  Трябва да докажем, че във $V'$ са точно полезните променливи за $G$.
  Очевидно е, че ако $A \in V'$, то $A$ е полезна променлива.
  \marginpar{\writedown Докажете!}
  За другата посока, с индукция по дължината на извода се доказва, че ако $A \to^\star_G \omega$,
  то $A \in V'$.
  
  Правилата на $G'$ са всички правила на $G$, в които участват променливи от $V'$ и букви от $\Sigma$.
\end{hint}

\begin{lemma}
  \label{lem:useless2}
  Съществува алгоритъм, който по дадена безконтекстна граматика $G = \CFG$, намира $G' = \pair{V',\Sigma',S,R'}$, $\L(G') = \L(G)$,
  със свойството, че за всяко $x \in V' \cup \Sigma'$ съществуват $\alpha, \beta \in (V'\cup\Sigma')^\star$,
  за които $S \to^\star \alpha x \beta$,
  т.е. всяка променлива или буква в $G'$ е достижима от началната променлива $S$.
\end{lemma}
\begin{hint}
  Намираме $V'$ и $\Sigma'$ итеративно, като в началото $V' = \{S\}$, $\Sigma' = \emptyset$.
  Ако $A \in V'$ и имаме правила в $G$:
  \[A \to \alpha_0\ |\ \alpha_1\ |\ \cdots\ |\ \alpha_n,\]
  то за всяко $i = 0,\dots,n$ добавяме всички променливи на $\alpha_i$ към $V'$ и всички нетерминали на $\alpha_i$ към $\Sigma'$.
\end{hint}

\begin{thm}
  За всяка безконтекстна граматика $G$, $\L(G) = \emptyset$ точно тогава, когато $S$ е безполезно правило в граматиката.
\end{thm}
\begin{proof}
  \marginpar{Защо е важна последователността на прилагане?}
  Нека е дадена безконтекстна граматика $G$ пораждаща $L$.
  Прилагаме върху $G$ първо процедурата от \Lem{useless1} и след това върху резултата прилагаме процедурата от \Lem{useless2}.
\end{proof}

\begin{example}
  Да разгледаме следната граматика $G$:
  \begin{align*}
    & S \to AB\ |\ aA\\
    & A \to a\ |\ aAa\\
    & B \to SB\ |\ BC\\
    & C \to \varepsilon\ |\ cC.
  \end{align*}
  Първо да намерим променливите, от които се извеждат думи.
  \begin{itemize}
  \item 
    $V_0 = \{A, C\}$, защото $A \to a$ и $C \to \varepsilon$;
  \item
    $V_1 = \{A, C, S\}$, защото $S \to aA$;
  \item
    не можем да добавим $B$ към $V_2$, следователно $V_1 = V_2$.
  \end{itemize}
  Получаваме граматиката $G'$:
  \begin{align*}
    & S \to aA\\
    & A \to a\ |\ aAa\\
    & C \to \varepsilon\ |\ cC.
  \end{align*}
  Сега премахваме променливите и буквите, които не са достижими от началната промелива $S$. Така получаваме граматиката $G''$:
  \begin{align*}
    & S \to aA\\
    & A \to a\ |\ aAa.
  \end{align*}
\end{example}

\begin{problem}
  Проверете дали $\L(G) = \emptyset$, където правилата на $G$ са:
  \begin{align*}
    & S \to AS\ |\ BC\\
    & A \to 0\ |\ BA\ |\ SB\\
    & B \to 1\ |\ BC\ |\ AB\\
    & C \to CB\ |\ SC\ |\ AS.
  \end{align*}
\end{problem}

\subsubsection*{Премахване на $\varepsilon$-правила}
\index{$\varepsilon$-правила}
За да премахнем правилата от вида $A \to \varepsilon$, следваме процедурата:
\marginpar{Броят на правилата може да се увеличи експоненциално, защото в най-лошия случай извеждаме всички подмножества на дадено множество от променливи}
\begin{enumerate}[1)]
\item 
  Намираме множеството $E = \{A \in V \mid A \to^\star \varepsilon\}$ по следния начин.
  Първо, $E := \{A \in V \mid A \to \varepsilon\}$.
  След това, за всяко правило от вида $B \to X_1\cdots X_k$, 
  ако всяко $X_i \in E$, то добавяме $B$ към $E$.
\item
  Строим множеството от правила $R'$, в което няма правила $\varepsilon$-правила по следния начин.
  За всяко правило $A \to x_1\cdots x_k$, където $x_i \in V\cup\Sigma$,
  добавяме към $R'$ всички правила от вида $A \to y_1\cdots y_k$, където:
  \begin{itemize}[-]
  \item 
    ако $x_i \not\in E$, то $y_i = x_i$;
  \item
    ако $x_i \in E$, то $y_i = x_i$ или $y_i = \varepsilon$;
  \item
    не всички $y_i$-та са $\varepsilon$.
  \end{itemize}
\end{enumerate}

\begin{example}
  Нека е дадена граматиката $G$ с правила
  \begin{align*}
    & S \to D\\
    & D \to AD\ |\ b\\
    & A \to AB\ |\ BC\ |\ a\\
    & B \to AA\ |\ UC\\
    & C \to \varepsilon\ |\ CA\ |\ a\\
    & U \to \varepsilon\ |\ aUb.
  \end{align*}
  % \[S\rightarrow D,D\rightarrow AD|b,A\rightarrow AB|BC|a, B\rightarrow AA|EC,C\rightarrow \varepsilon|CA|a, E\rightarrow \varepsilon|aEb.\]
  Тогава $E = \{X \in V \mid X \rightarrow^\star_G \varepsilon\} = \{A,B,C,U\}$.
  Това означава, че $\varepsilon \not\in \L(G)$.
  Граматиката $G'$ без $\varepsilon$-правила, за която $\L(G') = \L(G)$ има следните правила
  \begin{align*}
    & S \to D\\
    & D\to AD\ |\ D\ |\ b\\
    & A \to A\ |\ B\ |\ C\ |\ AB\ |\ BC\ |\ a\\
    & B\to A\ |\ E\ |\ C\ |\ AA\ |\ UC\\
    & C \to C\ |\ A\ |\ CA\ |\ a\\
    & U \to aUb\ |\ ab.
  \end{align*}
\end{example}

\subsubsection*{Премахване на преименуващи правила}
\index{преименуващи правила}
Преименуващите правила са от вида $A \to B$.
Нека е дадена граматика $G = \CFG$, в която има преименуващи правила.
Ще построим еквивалентна граматика $G'$ без преименуващи правила.
В началото нека в $R'$ да добавим всички правила от $R$, които не са преименуващи.
След това, за всякa променлива $A$, за която $A \to^\star_G B$,
ако $B \to \alpha$ е правило в $R$, което не е преименуващо,
то добавяме към $R'$ правилото $A \to \alpha$.

\begin{example}
  Нека е дадена граматиката $G$ с правила  
  \begin{align*}
    & S \to B\ |\ CC\ |\ b\\
    & A \to B\ |\ S\\
    & B \to C\ |\ BC\\
    & C \to AB\ |\ a\ |\ b.
  \end{align*}
  % \[A\rightarrow B|S,B\rightarrow C|BC,C\rightarrow AB|a|b,S\rightarrow B|CC|b.\]
  Първо добавяме към $R'$ правилата $B \to BC, C \to AB\ |\ a\ |\ b, S \to CC\ |\ b$.
  \begin{itemize}
  \item 
    Лесно се съобразява, че $A \to^\star_G B,S,C$.
    Добавяме правилата 
    \[A \to BC\ |\ AB\ |\ a\ |\ b\ |\ CC.\]
  \item
    Имаме $B \to^\star_G C$.
    Добавяме правилата $B \to AB\ |\ a\ |\ b$.
  \item
    Имаме $S \to^\star_G B,C$.
    Добавяме правилата $S \to BC\ |\ AB\ |\ a\ |\ b$.
  \end{itemize}
  Накрая получаваме, че граматиката $G'$ има правила
  \begin{align*}
    & S \to BC\ |\ AB\ |\ CC\ |\ a\ |\ b\\
    & A \to BC\ |\ AB\ |\ a\ |\ b\ |\ CC\\
    & B \to AB\ |\ a\ |\ b\ |\ BC\\
    & C \to AB\ |\ a\ |\ b.
  \end{align*}
\end{example}

\begin{problem}
  Премахнете преименуващите правила от граматиката $G$, като запазите езика, ако $G$ има следните правила:
    \begin{align*}
      & S \to C\ |\ CC\ |\ b\\
      & A \to B\\
      & B \to S\ |\ C\ |\ BC\\
      & C \to a\ |\ AB;
    \end{align*}
\end{problem}

\subsubsection*{Премахване на дългите правила}

Едно правило се нарича дълго, ако е от вида $A \to \beta$, където $|\beta| \geq 3$.
Да разгледаме едно дълго правило в граматиката от вида $A \to x_1x_2\cdots x_k$, 
където $k \geq 3$ и $x_i \in V \cup \Sigma$. За да получим еквивалентна граматика без това дълго правило,
добавяме нови променливи $A_1,\dots, A_{k-2}$, и правила
\[A \to x_1A_1,\ A_1 \to x_2A_2, \dots,\ A_{k-2} \to x_{k-1}x_k.\]


\begin{problem}
  Нека е дадена граматиката  $G = \pair{\{S,A,B,C\}, \{a,b\}, S, R}$.
  Използвайте обща конструкция, за да премахнете ,,дългите'' правила от $ G$ като при това получите 
  безконтестна граматика $G_1$ с език $\L(G) = \L(G_1)$, където правилата на граматиката са:
  % \begin{enumerate}[a)]
  % \item
  %   \begin{align*}
  %     & S \to \varepsilon\ |\ ab\ |\ aAba\\
  %     & A\to aBCb\\
  %     & B\to bbb\\
  %     & C\to aC\ |\ aCaC;
  %   \end{align*}
  % \item
    \begin{align*}
      & S\to CC\ |\ b\\
      & A\to BSB\ |\ a\\
      & B\to ba\ |\ BC\\
      & C\to BaSA\ |\ a\ |\ b.
    \end{align*}
  % \end{enumerate}
\end{problem}

\subsection{Нормална Форма на Чомски}
\index{Чомски}
%[стр. 99 от \cite{sipser}]
\index{нормална форма на Чомски}
Една безконтекстна граматика е в {\bf нормална форма на Чомски}, ако
всяко правило е от вида
\[A \rightarrow BC\mbox{ и }A \rightarrow a,\]
като $B, C$ {\em не могат} да бъдат променливата за начало $S$.
Освен това, позволяваме правилото $S\to\varepsilon$.
\footnote{В \cite[стр. 151]{papadimitriou} дефиницията е малко по-различна.
Там дефинират $G$ да бъде в нормална форма на Чомски ако $R \subseteq V\times(V\cup\Sigma)^2$.
В този случай губим езиците $\{\varepsilon\}$ и $\{a\}$, за $a\in\Sigma$.}

\begin{framed}
  \begin{thm}
    Всеки безконтекстен език $L$ се поражда от безконтекстна граматика в нормална форма на Чомски.
  \end{thm}
\end{framed}
\begin{proof}
%  \marginpar{Броят на правилата може да се увеличи експоненциално.}
  Нека имаме контекстно-свободна граматика $G$, за която $L = \L(G)$.
  Ще построим безконтекстна граматика $G^\prime$ в нормална форма на Чомски, $L = \L(G^\prime)$.
  % [стр. 99 от \cite{sipser}]
  Следваме следната процедура:
  \begin{itemize}
  \item
    Добавяме нов начален символ $S_0$ и правило $S_0 \to S$.
  \item
    \marginpar{Време $O(n)$}
    Премахваме дългите правила.
    Заменяме правилата от вида $A\to x_1x_2\dots x_n$, $n\geq 3$, $x_i \in V\cup\Sigma$, с
    правилата \[A\to x_1A_1,\ A_1\to x_2A_2,\ \dots,\ A_{n-2} \to x_{n-1}x_n.\]
    където $A_i$ са нови променливи.
  \item
    \marginpar{Време $O(n^2)$}
    Премахваме $\varepsilon$-правилата.
    За всяка променлива $A \neq S_0$ премахваме правилата от вида $A\to\varepsilon$.
    Това правим по следния начин.
    
    Ако имаме правило от вида $R \to Au$ или $R\to u A$, $u \in V \cup \Sigma$,
    то добавяме правилото $R\to u$.
    %Правим това за всяко срещане на променливата $A$ в дясната страна на правило.
    Например, 
    \begin{itemize}
    \item 
      ако имаме правило $R\to aA$, то добавяме правилото $R \to a$;
    \item
      ако имаме правило $R\to AA$, то добавяме правилото $R \to A$.
    \end{itemize}
    Ако имаме правило от вида $R\to A$, то добавяме правилото $R\to\varepsilon$
    само ако променливата $R$ още не е преминала през процедурата за премахване на $\varepsilon$.
  \item
    \marginpar{Време $O(n^2)$}
    \marginpar{Памет $O(n^2)$}
    Премахваме преименуващите правила, т.е. правила от вида $A\to B$.
    Заменяме всяко правило от вида $B \to \beta$ с $A\to \beta$,
    освен ако $A \to \beta$ е вече премахнато преименуващо правило.
  \item
    \marginpar{Време $O(n)$}
    За правила от вида $A\to u_1 u_2$, където $u_1, u_2 \in V \cup \Sigma$, 
    заменяме всяка буква $u_i$ с новата променлива $U_i$
    и добавяме правилото $U_i\to u_i$.
    Например, правилото $A \to aB$ се заменя с правилото $A \to XB$ и добавяме правилото $X \to a$,
    където $X$ е нова променлива.
  \end{itemize}
\end{proof}

\begin{thm}
  При дадена безконтекстна граматика $G$ с дължина $n$, можем да намерим еквивалентна
  на нея граматика $G'$ в нормална форма на Чомски за време $O(n^2)$,
  като получената граматика е с дължина $O(n^2)$.
\end{thm}


% \begin{problem}
%   Нека е дадена граматиката  $G = \pair{\{S,A,B,C,D,E\}, \{a,b\},S, R}$.
%   \begin{enumerate}[a)]
%   \item
%     Намерете множеството $\{X \in V \mid X \rightarrow^\star_G \varepsilon\}$.
%   \item
%     Вярно ли е, че $\varepsilon \in L(G)$?
%   \item
%     Постройте граматика $G_1$ без $\varepsilon$-правила, за която $L(G_1)=L(G)\setminus\{\varepsilon\}$.
%   \end{enumerate}
%   Множеството от правила $R$ на граматиката $G$ е зададено като:
%   \begin{enumerate}[a)]
%   \item
%     $R = \{S\rightarrow D,D\rightarrow AD|b,A\rightarrow ACB|BC|a, B\rightarrow ABCA|CEC,C\rightarrow \varepsilon|CA|a, E\rightarrow \varepsilon|aEb\}$;
%   \item
%     $R = \{S \rightarrow aD, D\rightarrow \varepsilon|ABBA|ADD,A\rightarrow DEB|a,B\rightarrow DDD|DC|b,C\rightarrow CCE|a, E\rightarrow \varepsilon|bEa\}$;
%   \item
%     $R = \{ S\rightarrow D,D\rightarrow AD|b,A\rightarrow AB|BC|a, B\rightarrow AB|CC, C\rightarrow \varepsilon|CA|a, E\rightarrow a|EB\}$;
%   \item
%     $R = \{ S \rightarrow AD|a, D\rightarrow \varepsilon|BB|AD,A\rightarrow DB|a,B\rightarrow DD|DC|b,C\rightarrow CE|a, E\rightarrow AB|b|EA\}$;
%   \item
%     $R =\{S\rightarrow AS|SB|SS,B\rightarrow CA|b, C\rightarrow AA|a|BA,A\rightarrow \varepsilon|BS\}$;
%   % \item
%   %   $R = \{S\rightarrow AB|AC,B\rightarrow \varepsilon |BC|b,A\rightarrow BB|CC|a,C\rightarrow CS|a\}$;
%   % \item
%   %   $R = \{S\rightarrow AS|SB|SS,B\rightarrow AC|b, C\rightarrow A|a|AB,A\rightarrow \varepsilon|BS\}$;
%   \item
%     $R = \{S\rightarrow BA|CA,B\rightarrow \varepsilon |BC|b,A\rightarrow BB|CC|a, C\rightarrow CS|a\}$;
%   \item
%     $R = \{S\rightarrow AS|b,A\rightarrow AC|BC|a, B\rightarrow BC|CC,C\rightarrow \varepsilon|CA|a\}$;
%   \item
%     $R = \{S\rightarrow \varepsilon|BA|AS,A\rightarrow SB|a,B\rightarrow SS|SC|b,
%     C\rightarrow CC|a\}$; 
%   \end{enumerate}
% \end{problem}


% \begin{problem}
%   Намерете безконтекстна граматика в нормална форма на Чомски за езиците от задача 6.
% \end{problem}


\subsection{Проблемът за принадлежност}

\begin{thm}
  Съществува {\em полиномиален} алгоритъм, който проверява дали дадена дума принадлежни на граматиката $G$.
  \marginpar{За дума $\alpha$, алгоритъмът работи за време $O(\abs{\alpha}^3)$}
\end{thm}
% \begin{proof}[стр. 154 от \cite{papadimitriou}]
Можем да приемем, че $G = \CFG$ е граматика в нормална форма на Чомски.
Нека $\alpha = a_1a_2\dots a_n$ е дума, за която искаме да проверим дали $\alpha \in \L(G)$.
\marginpar{Това е алгоритъм на Cocke, Younger и Kasami (CYK), който е пример за динамично програмиране (стр. 195 от \cite{kozen})}
\begin{algorithm}[H]
  \caption{Проверка дали $\alpha \in \L(G)$}
  \label{alg:belongs-to-grammar}
  \begin{algorithmic}[1]
    \State $n := \abs{\alpha}$ \Comment{Вход дума $\alpha = a_1\cdots a_n$}
    \ForAll{$i\in [1,n]$}
    \State $V[i,i] := \{A \in V \mid A\rightarrow a_i\}$
    \EndFor
    \ForAll{$i,j \in [1,n]\ \&\ i \neq j$}
    \State $V[i,j] := \emptyset$
    \EndFor      
    \ForAll{$s \in [1, n)$} \Comment{Дължина на интервала}
    \ForAll{$i \in [1, n-s]$}\Comment{Начало на интервала}
    \ForAll{$k \in [i, i + s)$}\Comment{Разделяне на интервала}
    \If{$\exists A\to BC \in R\ \&\ B \in V[i,k]\ \&\ C\in V[k+1,i+s]$}
    \State $V[i,i+s] := V[i,i+s] \cup \{A\}$
    \EndIf
    \EndFor
    \EndFor
    \EndFor
    \If{$S \in V[1,n]$}
    \State \Return \texttt{True}\Comment{Има извод на думата от $S$}
    \Else
    \State \Return \texttt{False}
    \EndIf
  \end{algorithmic}
\end{algorithm}

\begin{lemma}
  За дадена граматика в нормална форма на Чомски и дума $\alpha$, 
  за всяко $0 \leq s < \abs{\alpha}$, след $s$-тата итерация на алгоритъма (редове 6 - 10), за всяка позиция $i = 1,\dots,n-s$,
  \[V[i,i+s] = \{A \in V \mid A \rightarrow^\star_G a_i\dots a_{i+s}\}.\]
\end{lemma}
\begin{proof}
  Пълна индукция по $s$.
  За $s = 0$  е ясно. (Защо?)

  Нека твърдението е вярно за $s < n$. Ще докажем твърдението за $s+1$, т.е. за всяко $i = 1,\dots,n-s-1$,
  \[V[i,i+s+1] = \{A \in V \mid A \rightarrow^\star_G a_i\dots a_{i+s+1}\}.\]
  % Да разгледаме $A \in V[i,i+s+1]$.
  За едната посока, да разгледаме първoто правило в извода $A \to^\star_G a_i\cdots a_{i+s+1}$.
  Понеже $G$ е в НФЧ, то е от вида $A \to BC$ и тогава съществува някое $t$, за което 
  $B \to^\star a_i\cdots a_{i+t}$ и $C \to^\star a_{i+t+1}\cdots a_{i+s+1}$.
  От И.П. получаваме, че $B \in V[i,i+t]$ и $C \in V[i+t+1,i+s+1]$.
  Тогава от ред 10 на алгоритъма е ясно, че $A \in V[i,i+s+1]$.
  
  За другата посока, нека $A \in V[i,i+s+1]$.
  Единствената стъпка на алгоритъма, при която може да сме добавили $A$ към множеството $V[i,i+s+1]$ е ред 10.
  Тогава имаме, че съществува $k$, за което $B \in V[i,k]$, $C \in V[k+1,i+s+1]$, и $A\to BC$ е правило в граматиката $G$.
  От И.П. имаме, че $B \to^\star_G a_i\cdots a_k$ и $C \to^\star_G a_{k+1}\cdots a_{i+s+1}$.
  Заключаваме веднага, че $A \to^\star_G a_i\cdots a_{i+s+1}$.
\end{proof}

\begin{example}
  Нека е дадена граматиката $G$ с правила 
  \begin{align*}
    & S\rightarrow a\ |\ AB\ |\ AC\\
    & A\rightarrow a\\
    & B\rightarrow b\\
    & C\rightarrow SB\ |\ AS.
  \end{align*}
  Ще приложим $CYK$ алгоритъма за да проверим дали думата $aaabb \in \L(G)$.
  \begin{itemize}
  \item 
    $V[1,1] = V[2,2] = V[3,3] = \{S,A\}$;
    $V[4,4] = V[5,5] = \{B\}$.
  \item
    $V[1,2] = V[2,3] = \{C\}$;
    $V[3,4] = \{S,C\}$;
    $V[4,5] = \emptyset$.
  \item
    $V[1,3] = \{S\} \cup \emptyset$;
    $V[2,4] = \{S,C\} \cup \emptyset$;
    $V[3,5] = \emptyset \cup \{C\}$.
  \item
    $V[1,4] = \{S,C\} \cup \emptyset \cup \emptyset = \{S,C\}$;
    $V[2,5] = \{S\} \cup \emptyset \cup \{C\} = \{S,C\}$.
  \item
    $V[1,5] = \{S,C\} \cup \emptyset \cup \emptyset \cup \{C\}= \{S,C\}$.
  \end{itemize}
  Понеже $S \in V[1,5]$, то $aaabb \in \L(G)$.
\end{example}

\begin{thm}
  \marginpar{\cite[стр. 137]{hopcroft1}}
  Съществуват алгоритми, които определят по дадена безконтекстна граматика $G$ дали:
  \begin{enumerate}[a)]
  \item 
    $\abs{\L(G)} = 0$;
  \item
    $\abs{\L(G)} < \infty$;
  \item
    $\abs{\L(G)} = \infty$.
  \end{enumerate}
\end{thm}
\begin{proof}
  Нека е дадена една безконтекстна граматика $G$.
  \begin{description}
  \item[($\L(G) = \emptyset?$)]
    Прилагаме алгоритъма за премахване на безполезните променливи.
    Ако открием, че $S$ е безполезна променлива, то $\L(G) = \emptyset$.
  \item[($\abs{\L(G)} < \infty?$ или $\abs{\L(G)} = \infty?$)]
    Нека да разгледаме граматиката $G'$ в НФЧ без безполезни променливи, за която $\L(G) = \L(G')$.
    От граматиката $G' = \pair{V',\Sigma,S,R'}$ строим граф с възли променливите от $V'$ като
    за $A,B \in V'$ имаме ребро $A \to B$ точно тогава, когато съществува $C \in V'$,
    за което $A \to BC$ или $A \to CB$ е правило в $R'$.
    
    Ако в получения граф имаме цикъл, то $\L(G') = \infty$.
  \end{description}
\end{proof}

%%% Local Variables: 
%%% mode: latex
%%% TeX-master: "../eai"
%%% End: 
