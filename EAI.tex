\documentclass[a4paper, 11pt, oneside]{report}
%\usepackage[paperwidth=9cm, paperheight=12cm, top=0.5cm, bottom=0.5cm, left=0.0cm, right=0.5cm]{geometry}
%\special{papersize=9cm,12cm}


%%%%%%%%%%%%%
%% MARGINS %%
%%%%%%%%%%%%%

\setlength{\marginparsep}{0.5cm}
\setlength{\oddsidemargin}{0.3cm}
\setlength{\hoffset}{0cm}
\setlength{\marginparwidth}{110pt}

\let\oldmarginpar\marginpar

\renewcommand\marginpar[1]{\leavevmode\oldmarginpar{\raggedright\scriptsize #1}}

% \renewcommand\marginpar[1]{\-\oldmarginpar[\raggedleft\scriptsize #1]%
% {\raggedright\scriptsize #1}}
%\renewcommand\marginpar[1]{\oldmarginpar{\scriptsize #1}}

%%%%%%%%%%%%%%

%\usepackage{ucs}

% \usepackage{natbib}
% \usepackage{bibentry}

\usepackage[bulgarian]{babel}
\usepackage[utf8]{inputenc}
\usepackage[colorlinks=true, linkcolor=blue,pdfstartview=FitV,citecolor=green, urlcolor=blue]{hyperref}
\usepackage{pifont}
\usepackage{amssymb}
\usepackage{amsmath}
\usepackage{mathrsfs}
\usepackage{latexsym}
\usepackage{amsthm}
% \usepackage{paralist}
% \usepackage{enumerate}
\usepackage{makeidx}
\usepackage{layout}
\usepackage{framed}
\usepackage{bussproofs}
\usepackage{algorithm}
\floatname{algorithm}{Алгоритъм}
%\usepackage{algorithmic}
\usepackage{minted}
\usepackage[noend]{algpseudocode}
%\usepackage{algpseudocode}
\usepackage{float}

\usepackage{paralist}
\usepackage[shortlabels]{enumitem}
\setlist{leftmargin=*}

%%%%%%%%%%%%%%% TIKZ Package %%%%%%%%%%%%%%%%%%%%%%%
\usepackage{tikz}
\usepackage{pgf}
\usetikzlibrary{arrows,automata}
\usetikzlibrary{positioning}
\usetikzlibrary{backgrounds}
%%%%%%%%%%%%%%%%%%%%%%%%%%%%%%%%%%%%%%%%%%%%%%%%%%%%
\usepackage{caption}
\usepackage{subcaption}

\theoremstyle{definition}
\newtheorem{thm}{Теорема}
\newtheorem{crl}{Следствие}
\newtheorem{cor}{Следствие}
\newtheorem{lemma}{Лема}
\newtheorem{prop}{Твърдение}
\newtheorem{dfn}{Определение}
\newtheorem{problem}{Задача}
\newtheorem{example}{Пример}
\newtheorem{question}{Въпрос}
\newtheorem*{remark}{Забележка}
\renewenvironment{proof}{\noindent{\bf Доказателство.}\hspace*{1em}}{\qed\par}
\newenvironment{hint}{\noindent{\bf Упътване.}\hspace*{1em}}{\qed\par}
\newenvironment{solution}{\noindent{\bf Решение.}\hspace*{1em}}{\qed\par}

\newcommand{\A}{\mathcal{A}}
\newcommand{\B}{\mathcal{B}}
\renewcommand{\C}{\mathcal{C}}
\newcommand{\M}{\mathcal{M}}
\renewcommand{\L}{\mathcal{L}}
\newcommand{\D}{\mathcal{D}}
\newcommand{\R}{\mathbb{R}}
\newcommand{\Z}{\mathbb{Z}}
\newcommand{\N}{\mathcal{N}}
\newcommand{\Q}{\mathbb{Q}}
\newcommand{\Ls}{\mathscr{L}}
\newcommand{\Fs}{\mathscr{F}}
\newcommand{\Rs}{\mathscr{R}}
\newcommand{\Ps}{\mathscr{P}}
\newcommand{\As}{\mathscr{A}}
\newcommand{\Bs}{\mathscr{B}}
\newcommand{\Es}{\mathscr{E}}
\newcommand{\Is}{\mathscr{I}}
\newcommand{\Ss}{\mathscr{S}}
\newcommand{\xn}{x_{1},\dots,x_{n}}

\newcommand{\Nat}{\mathbb{N}}
\newcommand{\Int}{\mathbb{Z}}
\newcommand{\Real}{\mathbb{R}}

\newcommand{\xs}{overline{x}}

\newcommand{\ys}{overline{y}}

\newcommand{\zs}{overline{z}}
\newcommand{\ov}[1]{\overline{#1}}
\newcommand{\abs}[1]{\lvert{#1}\rvert}
\newcommand{\pair}[1]{\langle{#1}\rangle}
\newcommand{\writedown}{\ding{45}\ }

\newcommand{\FA}{\langle{Q,\Sigma,s,\delta,F}\rangle}
\newcommand{\FAn}[1]{\langle{Q_#1,\Sigma,s_#1,\delta_#1,F_#1}\rangle}
\newcommand{\NFA}{\langle{Q,\Sigma,s,\Delta,F}\rangle}
\newcommand{\NFAn}[1]{\langle{Q_#1,\Sigma,s_#1,\Delta_#1,F_#1}\rangle}
\newcommand{\PDA}{\langle{Q,\Sigma,\Gamma,\#,s,\Delta,F}\rangle}
\newcommand{\PDAn}[1]{\langle{Q_#1,\Sigma,\Gamma,\#,s_#1,\Delta_#1,F_#1}\rangle}
\newcommand{\CFG}{\langle{V,\Sigma,R,S}\rangle}
\newcommand{\TM}{\langle{Q,\Sigma,\Gamma,\delta,s,\blank,F}\rangle}

\renewcommand{\iff}{\ \leftrightarrow\ }
\newcommand{\df}{\stackrel{\text{деф}}{=}}
\newcommand{\dff}{\stackrel{\text{деф}}{\iff}}

\newcommand{\Th}[1]{{\em Теорема~\ref{th:#1}}}
\newcommand{\Lem}[1]{{\em Лема~\ref{lem:#1}}}
\newcommand{\Cor}[1]{{\em Следствие~\ref{cor:#1}}}
\newcommand{\Prob}[1]{{\em Задача~\ref{pr:#1}}}
\newcommand{\Prop}[1]{{\em Твърдение~\ref{pr:#1}}}
\newcommand{\Ex}[1]{{\em Пример~\ref{ex:#1}}}

%\newcommand*{\blank}{\allowbreak\textvisiblespace\allowbreak} % visible space
\newcommand*{\blank}{\sqcup}

% \setsecheadstyle{\large\usefont{T2A}{fag}{b}{r}} %\scshape
% \setsubsecheadstyle{\bfseries\sffamily}

% \renewcommand\familydefault{\sfdefault}

\title{Чернова на записки по ,,Езици, автомати, изчислимост''}
\author{Стефан Вътев\thanks{ел. поща: \href{mailto:stefanv@fmi.uni-sofia.bg}{stefanv@fmi.uni-sofia.bg}}}
%\\{\em Софийски Университет ,,Св. Климент Охридски''}}
%, \LaTeX\ файловете са \href{https://github.com/stefk0/EAI}{тук}}}
%, Факултет по математика и информатика, Софийски университет ,,Св. Климент Охридски''}}

% \abstract{test}

\makeindex
\begin{document}
\maketitle
% \layout

\tableofcontents

\chapter{Увод}
\label{ch:intro}

\section{Съждително смятане}\label{sect:propositional}
\marginpar{На англ. Propositional calculus}

Както при езиците за програмиране, всяка логика има свой синтаксис и семантика.
Тук ще разгледаме класическата съждителна логика, при която те са сравнително прости.

Съждителното смятане наподобява аритметичното смятане, като вместо аритметичните операции $+,-,\cdot,/$, 
имаме съждителни операции като $\neg, \wedge, \vee$.
Например, $(p\vee q) \wedge \neg  r$ е съждителна формула.
Освен това, докато аритметичните променливи приемат стойности числа, то
съждителните променливи приемат само стойности {\bf истина (1)} или {\bf неистина (0)}.

\marginpar{Това не е формална дефиниция, но за момента е достатъчно.}
{\bf Съждителна формула} наричаме съвкупността от съждителни променливи $p,q,r,\dots$, свързани със знаците за логически операции
$\neg$, $\vee$, $\wedge$, $\rightarrow$, $\leftrightarrow$ и скоби, определящи реда на операциите.

\subsection*{Съждителни операции}

\begin{itemize}
\item
  Отрицание $\neg$
\item 
  Дизюнкция $\vee$
\item
  Конюнкция $\wedge$
\item
  Импликация $\rightarrow$
\item
  Еквивалентност $\iff$
\end{itemize}

Ще използваме таблица за истинност за да определим стойностите на основните съждителни операции
при всички възможни набори на стойностите на променливите.

\[
\begin{array}{|c|c|c|c|c|c|c|c|c|}
  \hline
  p & q & \neg p & p \vee q & p \wedge q & p \rightarrow q & \neg p \vee q & p \iff q & (p \wedge q)\ \vee\ (\neg p\wedge \neg q) \\
  \hline
  0 & 0 & 1 & 0 & 0 & 1 & 1 & 1 & 1\\
  \hline
  0 & 1 & 1 & 1 & 0 & 1 & 1 & 0 & 0\\
  \hline
  1 & 0 & 0 & 1 & 0 & 0 & 0 & 0 & 0\\
  \hline
  1 & 1 & 0 & 1 & 1 & 1 & 1 & 1 & 1\\
  \hline
\end{array}
\]


{\bf Съждително верен} (валиден) е този логически израз, който има верностна стойност {\bf 1} при всички възможни набори на
стойностите на съждителните променливи в израза, т.е. стълбът на израза в таблицата за истинност трябва да съдържа само 
стойности {\bf 1}. 

Два съждителни израза $\varphi$ и $\psi$ са {\bf еквивалентни}, което означаваме $\varphi \equiv \psi$, ако са съставени от 
едни и същи съждителни променливи и двата израза имат едни и същи верностни стойности при всички комбинации от верностни 
стойности на променливите. С други думи, колоните на двата израза в съответните им таблици за истинност трябва да съвпадат.
Така например, от горната таблица се вижда, че 
$p\to q \equiv \neg p \vee q$ и $p \iff q \equiv (\neg{p}\wedge q)\ \vee\ (p\wedge \neg q)$.

\subsection{Съждителни закони}

\begin{enumerate}[I)]
\item
  {\bf Закон за идемпотентността}
  \[p \land p \equiv p\]
  \[p \lor p \equiv p\]
\item
    {\bf Комутативен закон}
    \[p\vee q \equiv q\vee p\] 
    \[p \wedge q \equiv q \wedge p\]
  \item
    {\bf Асоциативен закон}
    \[(p\vee q)\vee r \equiv p\vee(q\vee r)\]
    \[(p\ \wedge\ q)\ \wedge\ r \equiv p\ \wedge\ (q\ \wedge\ r)\]
  \item
    {\bf Дистрибутивен закон}
    \[p\ \wedge\ (q \vee r) \equiv (p\ \wedge q)\vee (p\ \wedge\ r)\]
    \[p\vee (q\ \wedge\ r) \equiv (p\vee q)\ \wedge\ (p\vee r)\]
  \item
    {\bf Закони на де Морган}
    \[\neg(p \wedge q) \equiv (\neg p \vee \neg q)\]
    \[\neg(p\vee q) \equiv (\neg p \wedge \neg q)\]
  \item
    {\bf Закон за контрапозицията}
    \[p\rightarrow q \equiv \neg q \rightarrow \neg p\]
  \item
    {\bf Обобщен закон за контрапозицията}
    \[(p \wedge q)\rightarrow r \equiv (p \wedge \neg r) \rightarrow \neg q\]
  \item
    {\bf Закон за изключеното трето}
    \[p\vee \neg p \equiv {\mathbf 1}\]
  \item
    {\bf Закон за силогизма (транзитивност)}
    \[ ((p\rightarrow q)\ \wedge\ (q\rightarrow r)) \rightarrow (p\rightarrow r) \equiv {\mathbf 1}\]
\end{enumerate}

Лесно се проверява с таблиците за истинност, че законите са валидни.

\subsection{Нормални форми}

\begin{itemize}
\item
  Конюнктивна нормална форма
\item
  Дизюнктивна нормална форма
\end{itemize}


%%% Local Variables:
%%% mode: latex
%%% TeX-master: "../eai"
%%% End:


\section{Предикати и квантори}

\subsection*{Квантори}

Свойствата или отношенията на елементите в произволно множество се наричат {\bf предикати}.
Нека да разгледаме един едноместен предикат $P(\cdot)$.

\bigskip
\begin{tabular}{|l|p{4.2cm}|p{4.5cm}|}
  \hline
  твърдение & Кога е истина? & Кога е неистина?\\
  \hline
  $\forall x P(x)$ & $P(x)$ е вярно за всяко $x$ & съществува $x$, за което $P(x)$ {\bf не} е вярно \\
  \hline
  $\exists x P(x)$ & съществува $x$, за което $P(x)$ е вярно & $P(x)$ {\bf не} е вярно за всяко $x$\\
  \hline
\end{tabular}  
\bigskip

\begin{enumerate}[(I)]
\item 
  {\bf Квантор за общност} $\forall x$.
  Записът $(\forall x \in A) P(x)$ означава, че за всеки елемент $a$ в $A$, 
  твърдението $P(a)$ има стойност истина.
  Например, $(\forall x\in\Real)[(x+1)^2 = x^2+2x+1]$.
\item
  {\bf Квантор за съществуване} $\exists x$.
  Записът $(\exists x \in A) P(x)$ означава, че съществува елемент $a$ в $A$, 
  за който твърдението $P(a)$ има стойност истина.
  Например, $(\exists x \in\mathbb{C})[x^2 = -1]$, но $(\forall x\in\Real)[x^2 \neq -1]$.
\end{enumerate}

% \begin{example}
%   \begin{itemize}
%   \item
%     За всяко естествено число, съществува по-голямо от него:
%     \[(\forall x\in\Nat)(\exists z\in\Nat)[x < z].\]
%   \item
%     Съществува естествено число, от което няма по-малко:
%     \[(\exists x\in\Nat)(\forall y\in\Nat)[x < y \vee x = y].\]
%     Нека да означим с $Zero(x)$ предиката, който казва, че $x$ е най-малкото число, т.е.
%     \[Zero(x) \equiv (\forall y)[x < y \vee x =y].\]
%   \item
%     Нека $S(x,y)$ да бъде предиката, който казва, че $y = x+1$ в естествените числа:
%     \[S(x,y) \equiv (x < y\ \wedge\ (\forall z\in\Nat)[x < z\ \rightarrow (z = y\ \vee\ y < z)].\]
%   \item
%     $One(x)$ - $x$ е числото $1$:
%     \[One(x) \equiv (\exists y)[Z(y)\ \wedge\ S(y,x)].\]
%   \item
%     $Div(x,y)$ - $x$ се дели на $y$:
%     \[Div(x,y) \equiv (\exists z)[x = y.z].\]
%   \item
%     $Prime(x)$ - $x$ е просто число:
%     \[Prime(x) \equiv x \geq 2\ \wedge\ (\forall y\in\Nat)[\neg (O(y)\ \wedge Z(y))\ \rightarrow\ \neg Div(x,y)].\]
%   \end{itemize}
% \end{example}


\subsection*{Закони на предикатното смятане}

\begin{enumerate}[(I)]
  \item
    $\neg\forall x P(x) \iff \exists x \neg P(x)$
  \item
    $\neg\exists x P(x) \iff \forall x \neg P(x)$
  \item
    $\forall x P(x) \iff \neg\exists x \neg P(x)$
  \item
    $\exists x P(x) \iff \neg\forall x \neg P(x)$
  \item
    $\forall x \forall y P(x,y) \iff \forall y\forall x P(x,y)$
  \item
    $\exists x\exists y P(x,y) \iff \exists y \exists x P(x,y)$  
  \item
    $\exists x\forall y P(x,y) \rightarrow \forall y \exists x P(x,y)$
\end{enumerate}

\bigskip
\begin{tabular}{|l|p{2.5cm}|p{3.2cm}|p{3cm}|}
  \hline
  \multicolumn{4}{|c|}{{\bf Закони на Де Морган за квантори}}\\
  \hline
  Твърдение & Еквивалентно твърдение & Кога е истина? & Кога е неистина?\\
  \hline
  $\neg \exists x P(x)$ & $\forall x \neg P(x)$ & за всяко $x$ $P(x)$ {\bf не} е вярно & съществува $x$, за което $P(x)$ е вярно \\
  \hline
  $\neg \forall x P(x)$ & $\exists x \neg P(x)$ & съществува $x$, за което $P(x)$ {\bf не} е вярно & $P(x)$ е вярно за всяко $x$\\
  \hline
\end{tabular}  
\bigskip

\begin{problem}
  Да означим с $K(x,y)$ твърдението ``$x$ познава $y$''.
  Изразете като предикатна формула следните твърдения.
  \begin{enumerate}[1)]
  \item
    \marginpar{$\forall x \exists y K(x,y)$}
    Всеки познава някого.
  \item
    \marginpar{$\exists x \forall y K(x,y)$}
    Някой познава всеки.
  \item
    \marginpar{$\exists x\forall y K(y,x)$}
    Някой е познаван от всички.
  \item
    \marginpar{$\forall x \exists y(K(x,y)\wedge \neg K(y,x)) $}
    Всеки знае някой, който не го познава.
  \item
    \marginpar{$\exists x \forall y(K(y,x)\ \rightarrow K(x,y))$}
    Има такъв, който знае всеки, който го познава.
  \item
    \marginpar{$(\forall x,y)(K(x,y)\ \&\ K(y,x) \to \exists z(K(x,z)\ \&\ K(y,z))$}
    Всеки двама познати имат общ познат.
  \end{enumerate}
\end{problem}

\begin{example}
  Нека $D \subseteq \Real$.
  Казваме, че $f:D \to \Real$ е {\em непрекъсната} в точката $x_0 \in D$, ако 
  \[(\forall \varepsilon > 0)(\exists \delta >0)(\forall x\in D)(\ |x_0 - x| < \delta\ \to\ |f(x_0) - f(x)| < \varepsilon\ ).\]
  Да видим какво означава $f$ да бъде {\em прекъсната} в точката $x_0 \in D$:
  \marginpar{$f$ е прекъсната в $x_0$ точно тогава, когато $f$ не е непрекъсната в $x_0$}
  \begin{align*}
    & \neg (\forall \varepsilon > 0)(\exists \delta >0)(\forall x\in D)(\ |x_0 - x| < \delta\ \to\ |f(x_0) - f(x)| < \varepsilon\ ) \equiv \\
    & (\exists \varepsilon > 0) \neg (\exists \delta >0)(\forall x\in D)(\ |x_0 - x| < \delta\ \to\ |f(x_0) - f(x)| < \varepsilon\ ) \equiv \\
    & (\exists \varepsilon > 0)(\forall \delta >0)\neg(\forall x\in D)(\ |x_0 - x| < \delta\ \to\ |f(x_0) - f(x)| < \varepsilon\ ) \equiv \\
    & (\exists \varepsilon > 0)(\forall \delta >0)(\exists x\in D)\neg(\ |x_0 - x| < \delta\ \to\ |f(x_0) - f(x)| < \varepsilon\ ) \equiv \\
    & (\exists \varepsilon > 0)(\forall \delta >0)(\exists x\in D)\neg(\ \neg (|x_0 - x| <\delta) \vee |f(x_0) - f(x)| < \varepsilon\ ) \equiv \\
    & (\exists \varepsilon > 0)(\forall \delta >0)(\exists x\in D)(\ \neg\neg (|x_0 - x| <\delta) \land \neg (|f(x_0) - f(x)| < \varepsilon)\ ) \equiv \\
    & (\exists \varepsilon > 0)(\forall \delta >0)(\exists x\in D)(\ |x_0 - x| < \delta\ \land\ |f(x_0) - f(x)| \geq \varepsilon\ ).
  \end{align*}
\end{example}


%%% Local Variables:
%%% mode: latex
%%% TeX-master: "../eai"
%%% End:


\section{Множества, релации, функции}\label{sect:intro:sets}
\index{множества}

\subsection*{Основни отношения между множества}

За произволни множества $A$ и $B$, ще казваме, че:
\begin{itemize}
\item
  $A$ е подмножество на $B$, което ще означаваме като $A \subseteq B$, ако:
  \[(\forall x)[x \in A \implies x \in B].\]
\item
  $A$ е равно на $B$, което ще означаваме като $A = B$, ако:
  \[(\forall x)[x \in A \iff x \in B],\]
  или
  \[A = B\ \iff\ A \subseteq B\ \&\ B \subseteq A.\]
\end{itemize}

\subsection*{Основни операции върху множества}

Ще разгледаме няколко операции върху произволни множества $A$ и $B$.
\begin{itemize}
\item
  \index{множества!сечение}
  \mynote{На англ. \emph{intersection}}
  {\bf Сечение}
  \[A\cap B = \{x\ \mid\ x\in A\ \wedge\ x\in B\}.\]
  Казано по-формално, $A\cap B$ е множеството, за което е изпълнено, че:
  \[(\forall x)[x \in A\cap B \iff (x\in A\ \land\ x \in B)].\]
  Примери:
  \begin{itemize}
  \item
    $A \cap A = A$, за всяко множество $A$.
  \item
    $A \cap \emptyset = \emptyset$, за всяко множество $A$.
  \item
    \mynote{Макар и $\emptyset$, $\{\emptyset\}$ и $\{1,2\}$ да са множества, те може да са елементи на други множества.}
    $\{1,\emptyset,\{\emptyset\}\} \cap \{\emptyset\} = \{\emptyset\}$.
  \item
    $\{1,2,\{1,2\}\} \cap \{1,\{1\}\} = \{1\}$.
  \end{itemize}
\item
  \index{множества!обединение}
  \mynote{На англ. \emph{union}}
  {\bf Обединение}
  \[A\cup B = \{x\ \mid x\in A\ \vee\ x\in B\}.\]
  $A\cup B$ е множеството, за което е изпълнено, че:
  \[(\forall x)[x \in A\cup B \iff (x\in A\ \lor\ x \in B)].\]
  Примери:
  \begin{itemize}
  \item
    $A \cup A = A$, за всяко множество $A$.
  \item 
    $A \cup \emptyset = A$, за всяко множество $A$.
  \item
    $\{1,2,\emptyset\} \cup \{1,2,\{\emptyset\}\} = \{1,2,\emptyset,\{\emptyset\}\}$.
  \item
    $\{1,2,\{1,2\}\} \cup \{1,\{1\}\} = \{1,2,\{1\},\{1,2\}\}$.
  \end{itemize}
\item
  \index{множества!разлика}
  {\bf Разлика}
  \[A\setminus B = \{x\ \mid\ x\in A\ \wedge\ x\not\in B\}.\]
  $A\setminus B$ е множеството, за което е изпълнено, че:
  \[(\forall x)[x \in A\setminus B \iff (x\in A\ \wedge\ x \not\in B)].\]
  Примери:
  \begin{itemize}
  \item
    $A \setminus A = \emptyset$, за всяко множество $A$.
  \item 
    $A \setminus \emptyset = A$, за всяко множество $A$.
  \item 
    $\emptyset \setminus A = \emptyset$, за всяко множество $A$.
  \item
    $\{1,2,\emptyset\} \setminus \{1,2,\{\emptyset\}\} = \{\emptyset\}$.
  \item
    $\{1,2,\{1,2\}\} \setminus \{1,\{1\}\} = \{2,\{1,2\}\}$.
  \end{itemize}
\item
  \index{множества!степенно множество}
  {\bf Степенно множество}
  \[\Ps(A) = \{x\mid x\subseteq A\}.\]
  \mynote{На англ. \emph{power set} }
  $\Ps(A)$ е множеството, за което е изпълнено, че:
  \[(\forall x)[x \in \Ps(A) \iff (\forall y)[y\in x\rightarrow y \in A]].\]
  \mynote{В литературата се среща също така и означението $2^A$ за степенното множество на $A$.}
  Примери:
  \begin{itemize}
  \item 
    $\Ps(\emptyset) = \{\emptyset\}$.
  \item
    $\Ps(\{\emptyset\}) = \{\emptyset,\{\emptyset\}\}$.
  \item
    $\Ps(\{\emptyset,\{\emptyset\}\}) = \{\emptyset,\{\emptyset\},\{\{\emptyset\}\},\{\emptyset,\{\emptyset\}\}\}$.
  \item
    $\Ps(\{1,2\}) = \{\emptyset,\{1\},\{2\},\{1,2\}\}$.
  \end{itemize}
\end{itemize}

\begin{problem}
  Проверете верни ли са свойствата:
  \begin{enumerate}[a)]
  \item
    $A\subseteq B \iff A\setminus B = \emptyset \iff A\cup B = B \iff A\cap B = A$;
  \item
    $A\setminus \emptyset = A$, $\emptyset\setminus A=\emptyset$, $A\setminus B = B\setminus A$.
  \item
    $A\cap (B\cup A) = A \cap B$;
  \item
    $A\cup(B\cap C) = (A\cup B)\cap(A\cup C)$ и $A \cap (B \cup C) = (A \cup B) \cap (A \cup C)$;
  % \item
  %   $C\subseteq A\ \&\ C\subseteq B \rightarrow C\subseteq A\cap B$;
  % \item
  %   $A\subseteq C\ \&\ B\subseteq C \rightarrow A\cup B\subseteq C$;
  \item
    $A\backslash B = A \iff A\cap B = \emptyset$;
  \item
    $A\backslash B = A\backslash (A\cap B)$ и $A\backslash B = A\backslash (A\cup B)$;
  \item
    $(A\cup B)\setminus C = (A\setminus C) \cup (B\setminus C)$;
  % \item
  %   \mynote{Не е вярно!}
  %   $A\setminus (B\setminus C) = (A\setminus B)\setminus C$;
  \item
    \index{Де Морган}
    \mynote{Закони на Де Морган}
    $C\setminus (A\cup B) = (C\backslash A)\cap(C\backslash B)$ и $C \backslash (A\cap B) = (C\backslash A)\cup(C\backslash B)$
  % \item
  %   $C\backslash(\bigcup^{n}_{i=1} A_i) = \bigcap^{n}_{i=1} (C\backslash A_i)$ и $C \backslash(\bigcap^{n}_{i=1} A_i) = \bigcup^{n}_{i=1} (C\backslash A_i)$;
  \item
    $(A\backslash B)\backslash C = (A\backslash C)\backslash(B \backslash C)$ и $A\backslash (B\backslash C) = (A\backslash B) \cup (A\cap C)$;
  \item
    $A\subseteq B \Rightarrow \Ps(A) \subseteq \Ps(B)$;
  \item
    \mynote{$X \subseteq A\cup B \stackrel{?}{\Rightarrow} X\subseteq A \vee X \subseteq B$}
    $\Ps(A\cap B) = \Ps(A) \cap \Ps(B)$ и $\Ps(A\cup B) = \Ps(A) \cup \Ps(B)$;
  \end{enumerate}
\end{problem}

За да дадем определение на понятието релация, трябва първо 
да въведем понятието декартово произведение на множества,
което пък от своя страна се основава на понятието наредена двойка.

\subsection*{Наредена двойка}
\index{наредена двойка}
За два елемента $a$ и $b$ въвеждаме опрецията {\bf наредена двойка} $\pair{a,b}$.
Наредената двойка $\pair{a,b}$ има следното характеристичното свойство:
\[a_1 = a_2\ \wedge\ b_1 = b_2\ \iff\ \pair{a_1,b_1} = \pair{a_2,b_2}.\]
Понятието наредена двойка може да се дефинира по много начини, стига да изпълнява харектеристичното свойство.
Ето примери как това може да стане:
\begin{enumerate}[1)]
\item
  \mynote{Norbert Wiener (1914)}
  Първото теоретико-множествено определение на понятието наредена двойка е
  дадено от Норберт Винер:
  \index{Винер}
  \[\pair{a,b} \df \{\{\{a\},\emptyset\},\{\{b\}\}\}.\]
\item
  \mynote{Kazimierz Kuratowski (1921)}
  \index{Куратовски}
  Определението на Куратовски се приема за ,,стандартно'' в наши дни:
  \[\pair{a,b} \df \{\{a\},\{a,b\}\}.\]
\end{enumerate}

\begin{problem}
  Докажете, че горните дефиниции наистина изпълняват харектеристичното свойство за наредени двойки.
\end{problem}

\begin{definition}
  \mynote{Пример за индуктивна (рекурсивна) дефиниция}
  Сега можем, за всяко естествено число $n \geq 1$,
  да въведем понятието наредена $n$-орка $\pair{a_1,\dots,a_n}$:
  \begin{align*}
    & \pair{a_1} \df a_1,\\
    & \pair{a_1,a_2,\dots,a_n} \df \pair{a_1,\pair{a_2,\dots,a_n}}.
  \end{align*}
\end{definition}

Оттук нататък ще считаме, че имаме дадено понятието наредена $n$-орка, без да се интересуваме от нейната формална дефиниция.
 
\subsection*{Декартово произведение}
\mynote{На англ. cartesian product. Считаме, че $(A\times B)\times C = A\times (B\times C) = A\times B \times C$.}
\index{декартово произведение}

За две множества $A$ и $B$, определяме тяхното декартово произведение като
\[A\times B = \{\pair{a,b}\mid a\in A\ \&\ b\in B\}.\]
За краен брой множества $A_1,A_2,\dots,A_n$, определяме
\[A_1\times A_2\times\cdots\times A_n = \{\pair{a_1,a_2,\dots,a_n}\mid a_1 \in A_1\ \&\ \dots\ \&\ a_n \in A_n\}.\]

\begin{problem}
  Проверете, че:
  \begin{enumerate}[a)]
  \item
    $A\times(B\cup C) = (A\times B) \cup (A\times C)$.
  \item
    $(A\cup B)\times C = (A\times C)\cup (B\times C)$.
  \item 
    $A\times(B\cap C) = (A\times B) \cap (A\times C)$.
  \item
    $(A \cap B)\times C = (A \times C)\cap(B\times C)$.
  \item 
    $A\times(B\setminus C) = (A\times B) \setminus (A\times C)$.
  \item
    $(A\setminus B)\times C = (A\times C)\setminus (B\times C)$.
  \end{enumerate}
\end{problem}


\subsection*{Видове функции}

Функцията $f:A \to B$ е:
\begin{itemize}
\item
  \mynote{\comment{или $f$ е {\bf обратима}}}
  {\bf инекция}\index{функция!инекция}, ако е изпълнено свойството:
  \[(\forall a_1,a_2\in A)[\ a_1\neq a_2\ \to\ f(a_1)\neq f(a_2)\ ],\]
  или еквивалентно,
  \[(\forall a_1,a_2\in A)[\ f(a_1) = f(a_2)\ \to\ a_1 = a_2\ ].\]
\item
  \mynote{\comment{или $f$ е {\bf върху} $B$ }}
  {\bf сюрекция}\index{функция!сюрекция}, ако е изпълнено свойството:
  \[(\forall b\in B)(\exists a\in A)[\ f(a) = b\ ].\]
\item
  {\bf биекция}\index{функция!биекция}, ако е инекция и сюрекция.
\end{itemize}

\begin{problem}
  \index{Кантор}
  \mynote{Канторово кодиране. Най-добре се вижда като се нарисува таблица}
  Докажете, че $f: \Nat \times \Nat\rightarrow \Nat$ е биекция, където
  \[f(x, y) = \frac{(x+y)(x+y+1)}{2} + x.\]
\end{problem}

%%% Local Variables:
%%% mode: latex
%%% TeX-master: "../eai"
%%% End:


\section{Доказателства на твърдения}

\subsection*{Допускане на противното}

Да приемем, че искаме да докажем, че свойството $P(x)$
е вярно за всяко естествено число.
Един начин да направим това е следният:
\begin{itemize}
\item 
  Допускаме, че съществува елемент $n$, за който $\neg P(n)$.
\item
  Използвайки, че $\neg P(n)$ правим извод, от който следва факт, за който знаем, че винаги е лъжа.
  Това означава, че доказваме следното твърдение
  \[\exists x \neg P(x) \rightarrow \mathbf{0}.\]
\item
  Тогава можем да заключим, че $\forall x P(x)$, защото имаме следния извод:
  \begin{prooftree}
    \AxiomC{$\exists x \neg P(x) \rightarrow \mathbf{0}$}
    \UnaryInfC{$\mathbf{1} \rightarrow \neg \exists x \neg P(x)$}
    \UnaryInfC{$\neg \exists x \neg P(x)$}
    \UnaryInfC{$\forall x P(x)$}
  \end{prooftree}
\end{itemize}

Ще илюстрираме този метод като решим няколко прости задачи.

\begin{problem}
  \label{prob:even-number-square}
  За всяко $a \in \Int$, ако $a^2$ е четно, то $a$ е четно.
\end{problem}
\begin{proof}
  Ние искаме да докажем твърдението $P$, където:
  \[P \equiv (\forall a\in\Int)[a^2\mbox{ е четно}\ \rightarrow\ a\mbox{ е четно}].\]
  \mynote{$\neg (\forall x)(A(x) \rightarrow B(x))$ е еквивалентно на $(\exists x)(A(x) \wedge \neg B(x))$}
  Да допуснем противното, т.е. изпълнено е $\neg P$. Лесно се вижда, че
  \[\neg P \iff (\exists a\in\Int)[a^2\mbox{ е четно}\ \land\ a\mbox{ не е четно}].\]
  Да вземем едно такова нечетно $a$, за което $a^2$ е четно.
  Това означава, че $a = 2k+1$, за някое $k \in \Int$,
  и \[a^2 = (2k+1)^2 = 4k^2 + 4k + 1,\]
  което очевидно е нечетно число.
  Но ние допуснахме, че $a^2$ е четно.
  Така достигаме до противоречие, следователно нашето допускане е грешно 
  и 
  \[(\forall a\in\Int)[a^2\mbox{ е четно}\ \rightarrow\ a\mbox{ е четно}].\]
\end{proof}

\begin{problem}
  Докажете, $\sqrt{2}$ {\bf не} е рационално число.
\end{problem}
\begin{proof}
  Да допуснем, че $\sqrt{2}$ е рационално число. Тогава  съществуват $a,b \in \Int$, такива че
  \[\sqrt{2} = \frac{a}{b}.\]
  Без ограничение, можем да приемем, че $a$ и $b$ са естествени числа,
  които нямат общи делители, т.е. не можем да съкратим дробта $\frac{a}{b}$.
  Получаваме, че \[2b^2 = a^2.\]
  Тогава $a^2$ е четно число и от Задача \ref{prob:even-number-square}, $a$ е четно число.
  Нека $a = 2k$, за някое естествено число $k$. Получаваме, че
  \[2b^2 = 4k^2,\]
  от което следва, че
  \[b^2 = 2k^2.\]
  Това означава, че $b$ също е четно число, $b = 2n$, за някое естествено число $n$.
  Следователно, $a$ и $b$ са четни числа и имат общ делител $2$,
  което е противоречие с нашето допускане, че $a$ и $b$ нямат общи делители.
  Така достигаме до противоречие.
  Накрая заключаваме, че $\sqrt{2}$ не е рационално число.
\end{proof}


\subsection*{Индукция върху естествените числа}
\index{индукция}

\mynote{Да напомним, че естествените числа са $\Nat = \{0,1,2,\dots\}$}
Доказателството с индукция по $\Nat$ представлява следната схема:
\begin{prooftree}
  \AxiomC{$P(0)$}
  \AxiomC{$(\forall x\in\Nat)[P(x)\rightarrow P(x+1)]$}
  \BinaryInfC{$(\forall x\in\Nat) P(x)$}
\end{prooftree}

Това означава, че ако искаме да докажем, че свойството $P(x)$ е вярно за всяко естествено число $x$,
то трябва да докажем първо, че е изпълнено $P(0)$ и след това, за произволно естествено число $x$, ако $P(x)$ вярно, то също така е вярно $P(x+1)$.

\begin{problem}
  \label{prob:number-prod-prime}  
  Всяко естествено число $n \geq 2$ може да се запише като произведение на прости числа.
\end{problem}
\begin{proof}
  Искаме да докажем, че $(\forall n \geq 2)P(n)$, където $P(n)$ казва, че $n$ може да се запише като произведение на прости числа, т.е.
  \[n = p^{m_1}_1p^{m_2}_2\cdots p^{m_k}_k,\]
  за някои прости числа $p_1,p_2,\dots,p_k$ и естествени числа $m_1,m_2,\dots,m_k$.
  
  Доказателството протича с индукция по $n \geq 2$.
  \begin{enumerate}[a)]
  \item 
    За $n = 2$ е ясно, защото $2$ е просто число. В този случай $n = p^{m_1}_1$ и $p_1 = 2$ и $m_1 = 1$.
  \item
    Да приемем, че $P(n)$ е изпълнено за някое естествено число $n > 2$.
  \item
    Да разгледаме следващото естествено число $n+1$.
    Ако $n+1$ е просто число, то всичко е ясно.
    Ако $n+1$ е съставно, то съществуват естествени числа $n_1,n_2 \geq 2$, за които
    \[n + 1 = n_1\cdot n_2.\]
    Тогава, понеже $n_1,n_2 \leq n$, от от И.П. следва, че $P(n_1)$ и $P(n_2)$, т.е.
    \[n_1 = p^{\ell_1}_1\cdots p^{\ell_k}_k\text{ и }n_2 = q^{m_1}_1\cdots q^{m_r}_r,\]
    където $p_1,\dots,p_k$ и $q_1,\dots,q_r$ са прости числа, а $\ell_1,\dots,\ell_k$ и $m_1,\dots,m_r$ са естествени числа.
    Тогава е ясно, че $n+1$ също е произведение на прости числа.
  \end{enumerate}
\end{proof}

\begin{problem}
  Докажете, че за всяко естествено число $n$, 
  \[\sum^n_{i=0} 2^i = 2^{n+1} - 1.\]
\end{problem}
\begin{proof}
  Да разгледаме свойството
  \[P(n) \df \sum^n_{i=0} 2^i = 2^{n+1} - 1.\]
  Ще докажем с индукция по $n$, че $(\forall n)P(n)$, т.е. ще докажем следния извод:
  \begin{prooftree}
    \AxiomC{$P(0)$}
    \AxiomC{$(\forall n)[P(n) \implies P(n+1)]$}
    \BinaryInfC{$(\forall n)P(n)$}
  \end{prooftree}
  \begin{itemize}
  \item
    \mynote{Това е базата на индукцията.}
    Нека първо $n = 0$. Oчевидно е, че $P(0)$ е изпълнено, защото
    \[\sum^0_{i=0}2^i = 1 = 2^{1} - 1.\]
  \item
    \mynote{$P(n)$ се нарича индукционно предположение, а $P(n+1)$ се нарича индукционна стъпка.}
    Да разгледаме сега произволно естествено число $n$, като
    приемем, че свойството $P(n)$ е изпълнено.
    Ще докажем, че $P(n+1)$ също е изпълнено.
    Но това е лесно защото имаме следната верига от равенства:
    \begin{align*}
      \sum^{n+1}_{i=0} 2^i & = \sum^{n}_{i=0}2^i + 2^{n+1}\\
                           & = 2^{n+1} - 1 + 2^{n+1} & \comment\text{защото $P(n)$ е изпълнено}\\
                           & = 2.2^{n+1} - 1 \\
                           & = 2^{1+(n+1)} - 1\\
                           & = 2^{n+2} - 1.
    \end{align*}
  \end{itemize}
\end{proof}

%\subsection*{Пълна индукция върху естествените числа}


%%% Local Variables:
%%% mode: latex
%%% TeX-master: "../eai"
%%% End:


\section{Азбуки, думи, езици}

\subsection*{Основни понятия}

\begin{itemize}
\item 
  \index{азбука}
  {\bf Азбука} ще наричаме всяко крайно множество,
  като обикновено ще я означаваме със $\Sigma$.
  \marginpar{Често ще използваме буквите $a$, $b$, $c$ за да означаваме букви.}
  Елементите на азбуката $\Sigma$ ще наричаме {\bf букви} или символи.
\item
  \index{дума}
  {\bf Дума} над азбуката $\Sigma$ е произволна крайна редица от елементи на $\Sigma$.
  Например, за $\Sigma = \{a,b\}$, $aababba$ е дума над $\Sigma$ с дължина $7$.
  С $\abs{\alpha}$ ще означаваме дължината на думата $\alpha$.
  \marginpar{Обикновено ще означаваме думите с $\alpha$, $\beta$, $\gamma$, $\omega$.}
\item
  Обърнете внимание, че имаме единствена дума с дължина $0$.
  Тази дума ще означаваме с $\varepsilon$ и ще я наричаме {\bf празната дума},
  т.е. $\abs{\varepsilon} = 0$.
\item
  С $a^n$ ще означаваме думата съставена от $n$ $a$-та.
  Формалната индуктивна дефиниция е следната:
  \begin{align*}
    a^0 & \df \varepsilon,\\
    a^{n+1} & \df a^na.
  \end{align*}
\item
  Множеството от всички думи над азбуката $\Sigma$ ще означаваме със $\Sigma^\star$.
  Например, за $\Sigma = \{a,b\}$,
  \[\Sigma^\star = \{\varepsilon,a,b,aa,ab,ba,bb,aaa,aab,\dots\}.\]
  Обърнете внимание, че $\emptyset^\star = \{\varepsilon\}$.
\item
  {\bf Език} над азбуката $\Sigma$ ще наричаме всяко подмножество на $\Sigma^\star$.
  Например, за $\Sigma = \{a, b\}$,
  \[L = \{\alpha \in \{a, b\}^\star \mid \alpha\mbox{ започва с }a\}\]
  е език над $\Sigma$.
% \item
%   {\bf Лексикографска наредба}
\end{itemize}

\subsection*{Операции върху думи}

\begin{itemize}
\item 
  \index{конкатенация}
  Операцията {\bf конкатенация} взима две думи $\alpha$ и $\beta$ и образува 
  новата дума $\alpha\cdot\beta$ като слепва двете думи.
  Например $aba\cdot bb = ababb$.
  Обърнете внимание, че в общия 
  случай $\alpha\cdot\beta \neq \beta\cdot\alpha$. 
  \marginpar{Често ще пишем $\alpha\beta$ вместо $\alpha\cdot\beta$}
  Можем да дадем формална индуктивна дефиниция на операцията конкатенация по
  дължината на думата $\beta$.
  \begin{itemize}
  \item 
    Ако $\abs{\beta} = 0$, то $\beta = \varepsilon$.
    Тогава $\alpha\cdot \varepsilon \df \alpha$.
  \item
    Ако $\abs{\beta} = n+1$, то $\beta = \gamma b$, $\abs{\gamma} = n$.
    Тогава $\alpha\cdot\beta \df (\alpha\cdot\gamma)b$.
  \end{itemize}
\item
  Друга често срещана операция върху думи е {\bf обръщането} на дума.
  Дефинираме думата $\alpha^R$ като обръщането на $\alpha$ по следния начин.
  \begin{itemize}
  \item 
    Ако $\abs{\alpha} = 0$, то $\alpha = \varepsilon$ и $\alpha^R \df \varepsilon$.
  \item
    Ако $\abs{\alpha} = n+1$, то $\alpha = a\beta$, където $\abs{\beta} = n$.
    Тогава $\alpha^R \df (\beta^R)a$.
  \end{itemize}
  Например, $reverse^R = esrever$.
\item
  \index{дума!префикс}
  \index{дума!суфикс}
  Казваме, че думата $\alpha$ е {\bf префикс} на думата $\beta$,
  ако съществува дума $\gamma$, такава че $\beta = \alpha\cdot\gamma$.
  $\alpha$ е {\bf суфикс} на $\beta$, ако $\beta = \gamma\cdot\alpha$, за някоя дума $\gamma$.
\item
  \marginpar{Обърнете внимание, че $\emptyset\cdot A = A\cdot\emptyset = \emptyset$}
  \marginpar{Също така, $\{\varepsilon\}\cdot A = A\cdot\{\varepsilon\} = A$}
  Нека $A$ и $B$ са множества от думи.
  Дефинираме конкатенацията на $A$ и $B$ като
  \[A\cdot B \df \{\alpha\cdot\beta \mid \alpha\in A\ \&\ \beta \in B\}.\]
\item
  Сега за едно множество от думи $A$, дефинираме $A^n$ индуктивно:
  \begin{align*}
    A^0 & \df \{\varepsilon\},\\
    A^{n+1} & \df A^n \cdot A.
  \end{align*}
  Ако $A = \{ab, ba\}$, то
  $A^0 = \{\varepsilon\}$, $A^1 = A$, $A^2 = \{abab, abba, baba, baab\}$.
  Ако $A = \{a,b\}$, то $A^n = \{\alpha \in \{a,b\}^\star \mid \abs{\alpha} = n\}$.
\item
  За едно множеството от думи $A$, дефинираме:
  \marginpar{Операцията $\star$ е известна като звезда на Клини}
  \begin{align*}
    A^\star & \df \bigcup_{n\geq 0} A^n\\
    & = A^0 \cup A^1 \cup A^2 \cup A^3 \cup \dots\\
    A^+ & \df A\cdot A^\star.
  \end{align*}
\end{itemize}

\begin{problem}
  Проверете:
  \begin{enumerate}[a)]
  \item 
    операцията конкатенация е {\em асоциативна}, т.е. за всеки три думи $\alpha$, $\beta$, $\gamma$,
    \[(\alpha\cdot\beta)\cdot\gamma = \alpha\cdot(\beta\cdot\gamma);\]
  \item
    за множествата от думи $A$, $B$ и $C$,
    \[(A\cdot B)\cdot C = A\cdot (B\cdot C);\]
  \item
    $\{\varepsilon\}^\star = \{\varepsilon\}$;
  \item
    за произволно множество от думи $A$,
    $A^\star = A^\star\cdot A^\star$ и $(A^\star)^\star = A^\star$;
  \item
    за произволни букви $a$ и $b$,
    $\{a,b\}^\star = \{a\}^\star\cdot(\{b\}\cdot\{a\}^\star)^\star$.
  \end{enumerate}
\end{problem}


\begin{problem}
  Докажете, че за всеки две думи $\alpha$ и $\beta$ е изпълено:
  \begin{enumerate}[a)]
  \item 
    $(\alpha\cdot\beta)^R = \beta^R\cdot\alpha^R$;
  \item
    $\alpha$ е префикс на $\beta$ точно тогава, когато $\alpha^R$ е суфикс на $\beta^R$;
  \item
    $(\alpha^R)^R = \alpha$;
  \item
    $(\alpha^n)^R = (\alpha^R)^n$, за всяко $n \geq 0$.
  \end{enumerate}
\end{problem}

\begin{problem}
  \marginpar{С други думи, ако $\varepsilon \not\in X$, то $Z = X^\star Y$ е най-малкото решение на уравнението $Z = XZ \cup Y$.}
  Нека $X, Y, Z \subseteq \Sigma^\star$ със свойството, че $Z = XZ \cup Y$.
  \begin{enumerate}[a)]
  \item 
    Докажете, че за всяко $n \in \Nat$, $X^nY \subseteq Z$.
    Заключете, че $X^\star Y \subseteq Z$.
  \item
    Да предположим, че $\varepsilon \not\in X$.
    Докажете, че за всяка дума $\omega \in Z$ е изпълнено, че $\omega \in X^\star Y$.
  \end{enumerate}
\end{problem}


%%% Local Variables:
%%% mode: latex
%%% TeX-master: "../eai"
%%% End:


\section*{Бележки}

Повечето книги в тази област започват с уводна глава, в която въвеждат понятията множества, релации и езици.
\begin{itemize}
\item 
  Глава 1 от \cite{rosen}.
\item
  Глава 1 от \cite{papadimitriou}.
\item
  За описанието на думи и азбуки следваме \cite[Глава 2]{kozen}.
\end{itemize}



%%% Local Variables:
%%% mode: latex
%%% TeX-master: "../eai"
%%% End:


% Add Problem section

%%% Local Variables:
%%% mode: latex
%%% TeX-master: "../eai"
%%% End:



\chapter{Регулярни езици и автомати}

\section{Автоматни езици}

% Един от източниците е втора и трета глава от книгата на Сипсер, \cite{sipser}.
% Друг основен източник е книгата на Пападимитриу и Люис, \cite{papadimitriou}.
%По Сипсер, стр. 35
\begin{dfn}
  Краен автомат е петорка $\A = \FA$, където
  \begin{enumerate}[1)]
  \item
    $Q$ е крайно множество от състояния;
  \item
    $\Sigma$ е азбука;
  \item
    % \marginpar{Тук нямаме $\varepsilon$-преходи}
    \marginpar{(Sipser разглежда тотални $\delta$ функции)}
    $\delta:Q\times\Sigma\to Q$ е (частична) функция на преходите;
  \item
    $s\in Q$ е начално състояние;
  \item
    $F\subseteq Q$ е множеството от финални състояния, $F \neq \emptyset$.
  \end{enumerate}
\end{dfn}

\index{автомат!детерминиран}\index{автомат!тотален детерминиран}
Ако функцията на преходите $\delta$ е тотална функция, то казваме, 
че автоматът $\A$ е {\bf тотален}. Това означава, че за всяка двойка $(a,q) \in \Sigma\times Q$,
същесествува $q' \in Q$, за което $\delta(a,q) = q'$.

Нека имаме една дума $\alpha \in \Sigma^\star$, $\alpha = a_1a_2\cdots a_n$.
Казваме, че $\alpha$ се {\bf разпознава} от автомата $\A$, ако
съществува редица от състояния $q_0,q_1,q_2,\dots,q_n$, такива че:
\begin{itemize}
\item
  $q_0 = s$, началното състояние на автомата;
\item
  $\delta(q_i,a_{i+1}) = q_{i+1}$, за всяко $i = 0, \dots, n-1$;
\item
  $q_n \in F$.
\end{itemize}

% Алтернативен запис е следния:
% $(q,a\beta) \vdash (p,\beta)$, ако $\delta(q,a) = p$.
% $(q,\alpha\beta) \vdash^\star (p,\beta)$, ако $\delta^\star(q,\alpha) = p$.
% Тогава една дума $\alpha$ се разпознава от автомата, ако $(s,\alpha) \vdash^\star (p,\varepsilon)$ и $p \in F$.

Казваме, че $\A$ {\bf разпознава} езика $L$, ако $\A$ разпознава точно думите от $L$, т.е.
$L = \{\alpha \in \Sigma^\star \mid \A\mbox{ разпознава }\alpha\}$.
Обикновено означаваме езика, който се разпознава от даден автомат $\A$ с $\L(\A)$.
\index{език!автоматен}
В такъв случай ще казваме, че езикът $L$ е {\bf автоматен}.

При дадена (частична) функция на преходите $\delta$,
често е удобно да разглеждаме (частичната) функция $\delta^\star:Q\times\Sigma^\star \to Q$, кято е дефинирана по следния начин:
\marginpar{Това е пример за индуктивна (рекурсивна) дефиниция по дължината на думата $\alpha$}
\begin{itemize}
\item 
  $\delta^\star(q,\varepsilon) = q$, за всяко $q\in Q$;
\item
  $\delta^\star(q,a\beta) = \delta^\star(\delta(q,a),\beta)$, за всяко $q\in Q$, всяко $a\in\Sigma$ и $\beta\in\Sigma^\star$.
\end{itemize}
Тогава една дума $\alpha$ се {\em разпознава} от автомата $\A$ точно тогава, когато $\delta^\star(s,\alpha) \in F$.
Оттук следва, че
\[\L(\A) = \{\alpha\in\Sigma^\star \mid \delta^\star(s,\alpha) \in F\}.\]

\begin{prop}
  $(\forall q\in Q)(\forall\alpha,\beta\in\Sigma^\star)[\delta^\star(q,\alpha\beta) = \delta^\star(\delta^\star(q,\alpha),\beta)]$.
\end{prop}
\begin{proof}
  \marginpar{\ding{45} Напише доказателството!}
  Индукция по дължината на $\alpha$.
\end{proof}

\index{моментно описание}
{\em Моментното описание} на изчисление с краен автомат представлява двойка от вида $(q,\alpha) \in Q\times\Sigma^\star$,
т.е. автоматът се намира в състояние $q$, а думата, която остава да се прочете е $\alpha$.
Удобно е да въведем бинарната релация $\vdash_\A$ над $Q\times\Sigma^\star$,
която ще ни казва как моментното описание на автомата $\A$ се променя след изпълнение на една стъпка:
\[(q,x\alpha) \vdash_\A (p,\alpha), \text{ ако } \delta(q,x) = p.\]
Рефлексивното и транзитивно затваряне на $\vdash_\A$ ще означаваме с $\vdash^\star_\A$.
Получаваме, че 
\[\L(\A) = \{\alpha\in\Sigma^\star \mid (s,\alpha) \vdash^\star_\A(p,\varepsilon)\ \&\ p \in F\}.\]

Нашата дефиниция на автомат позволява $\delta$ да бъде частична функция, т.е.
може да има $q\in Q$ и $a\in\Sigma$, за които $\delta(q,a)$ не е дефинирана.
Следващото твърдение ни казва, че ние съвсем спокойно можем да разглеждаме автомати
само с тотални функции на преходите  $\delta$.
\begin{prop}
  За всеки краен автомат $\A$, съществува {\em тотален} краен автомат $\A'$,
  за който $\L(\A) = \L(\A')$.
\end{prop}
\begin{proof}
  Нека $\A = \FA$.
  Дефинираме тоталния автомат 
  \[\A' = \pair{Q\cup\{q_e\}, \Sigma, \delta', s, F},\]
  като за всеки преход $(q,a)$, за който $\delta$ не е дефинирана, 
  дефинираме $\delta'$ да отива в новото състояние $q_e$.
  Ето и цялата дефиниция на новата функция на преходите $\delta'$:
  \begin{itemize}
  \item 
    \marginpar{$q_e$ - error състояние}
    $\delta'(q_e,a) = q_e$, за всяко $a\in\Sigma$;
  \item
    \marginpar{$\A'$ симулира $\A$}
    За всяко $q\in Q$, $a\in\Sigma$, ако $\delta(q,a) = p$, то
    $\delta'(q,a) = p$;
  \item
    За всяко $q\in Q$, $a\in\Sigma$, ако $\delta(q,a)$ не е дефинирано, то
    $\delta'(q,a) = q_e$.
  \end{itemize}
  \marginpar{\writedown Довършете доказателството!}
  Сега лесно може да се докаже, че $\L(\A) = \L(\A')$.
\end{proof}

\begin{prop}
  \label{pr:automata-union}
  Класът на автоматните езици е затворен относно операцията {\bf обединение}.
  Това означава, че ако $L_1$ и $L_2$ са два произволни автоматни езика над азбуката $\Sigma$, то $L_1\cup L_2$
  също е автоматен език.
\end{prop}
\begin{proof}
  \marginpar{Защо изискваме $\A_1$ и $\A_2$ да са тотални?}
  Нека $L_1 = L(\A_1)$ и $L_2 = L(\A_2)$, 
  където \[\A_1 = \FAn{1},\ \A_2 = \FAn{2}\] са {\bf тотални}.
  Определяме автомата $\A = \FA$, който разпознава $L_1\cup L_2$ по следния начин:
  \begin{itemize}
  \item
    $Q = Q_1\times Q_2$;
  \item
    \marginpar{Едновременно симулираме изчисление и по двата автомата}
    Определяме за всяко $\pair{r_1,r_2} \in Q$ и всяко $a \in \Sigma$,
    \[\delta(\pair{r_1,r_2},a) = \pair{\delta_1(r_1,a),\delta_2(r_2,a)};\]
  \item
    $s = \pair{s_1,s_2}$;
  \item
    $F = \{\pair{r_1,r_2}\mid r_1\in F_1\vee r_2 \in F_2\} = (F_1\times Q_2)\cup (Q_1\times F_2)$.
  \end{itemize}
  \marginpar{По-нататък ще дадем друга конструкция за обединение, която ще бъде по-ефективна.}
  \marginpar{\writedown Проверете, че $\L(\A) = \L(\A_1)\cup \L(\A_2)$}
\end{proof}

\begin{cor}
  Класът на автоматните езици е затворен относно операцията {\bf сечение}.
  Това означава, че ако $L_1$ и $L_2$ са два произволни автоматни езика над азбуката $\Sigma$, то $L_1\cap L_2$
  също е автоматен език.
\end{cor}
\begin{proof}
  \marginpar{\ding{45} Докажете, че така построения автомат $\A$ разпознава $L_1\cap L_2$!}
  Използвайте конструкцията на автомата $\A = \FA$ от \Prop{automata-union},
  с единствената разлика, че тук избираме финалните състояния да бъдат елементите на множеството
  \[F = \{\pair{q_1,q_2} \mid q_1 \in F_1\ \&\ q_2 \in F_2\} = F_1\times F_2.\]
\end{proof}

\begin{prop}
  Нека $L$ е автоматен език.
  Тогава $\Sigma^\star\setminus L$ също е автоматен език.
\end{prop}
\begin{proof}
  \marginpar{Защо искаме $\A$ да бъде тотален ?}
  Нека $L = L(\A)$, където $\A = \FA$ е {\bf тотален}.
  Да вземем автомата $\A' = \pair{Q,\Sigma,s,\delta,Q\setminus F}$,
  т.е. $\A'$ е същия като $\A$, с единствената разлика, че финалните състояния на $\A'$
  са тези състояния, които {\bf не} са финални в $\A$.
  \marginpar{\writedown Проверете, че $\Sigma^\star\setminus L = \L(\A')$}
\end{proof}

\begin{problem}
  За всеки от следните езици $L$, постройте автомат $\A$, който разпознава езика $L$.
  \begin{enumerate}[a)]
  \item 
    $L = \{a^nb\mid n \geq 0\}$;
  \item
    $L = \{\varepsilon, a,b\}$;
  \item
    $L = \emptyset$;
  \item
    $L = \{a,b\}^\star\setminus\{\varepsilon\}$;
  \item
    $L = \{a^nb^m\mid n,m \geq 0\}$;
  \item
    $L = \{a^nb^m\mid n,m \geq 1\}$;
  \item
    $L = \{a,b\}^\star \setminus \{a\}$;
  \item
    $L = \{w \in \{a,b\}^\star \mid \mbox{съдържа поне две }a\}$;
  \item
    $L = \{w \in \{a,b\}^\star \mid \mbox{съдържа поне две }a\mbox{ и поне едно }b\}$;
  \item
    $L = \{w \in \{a,b\}^\star \mid \mbox{на всяка нечетна позиция на }w\mbox{ е буквата }a\}$;
  \item
    $L = \{w \in \{a,b\}^\star \mid w\mbox{ съдържа четен брой }a\mbox{ и най-много едно }b\}$;
  \item
    $L = \{w \in \{a,b\}^\star\mid \abs{w} \leq 3\}$;
  \item
    $L = \{w \in \{a,b\}^\star \mid w \mbox{ не започва с }ab\}$;
  \item
    $L = \{w \in \{a,b\}^\star \mid w \mbox{ завършва с }ab\}$;
  \item
    $L = \{w \in \{a,b\}^\star \mid w \mbox{ съдържа }bab\}$;
  \item
    $L = \{w \in \{a,b\}^\star \mid w \mbox{ не съдържа }bab\}$;
  \item
    \marginpar{(решена е по-долу)}
    $L = \{w \in \{a,b\}^\star \mid w \mbox{ няма две последователни }a\}$;
  \item
    $L = \{w \in \{a,b\}^\star \mid w\mbox{ започва и завършва с буквата } a\}$;
  \item
    $L = \{w \in \{a,b\}^\star \mid w\mbox{ започва и завършва с една и съща буква}\}$;
  \item
    $L = \{\omega \in \{a,b\}^\star \mid \abs{\omega} \equiv 0\ (\bmod\ 2)\ \&\ \omega \mbox{ съдържа точно едно }a\}$;
  \item
    $L = \{w \in \{a,b\}^\star \mid \mbox{ всяко }a\mbox{ в }w\mbox{ се следва от поне едно }b\}$;
  \item
    $L = \{w \in \{a,b\}^\star \mid \abs{w} \equiv 0 \bmod 3\}$;
  \item
    \marginpar{$N_a(w)$ - броят на срещанията на буквата $a$ в думата $w$}
    $L = \{w \in \{a,b\}^\star \mid N_a(w) \equiv 1 \bmod 3\}$;
  \item
    $L = \{\omega \in \{a,b\}^\star \mid N_a(\omega) \equiv 0 \bmod 3\ \&\ N_b(\omega) \equiv 1 \bmod 2\}$;
  \item
    $L = \{\omega \in \{a,b\}^\star \mid N_a(\omega) \equiv 0 \bmod 2\ \vee\ \omega\mbox{ съдържа точно две }b\}$;
  \item
    $L = \{\omega \in \{a,b\}^\star \mid \omega \text{ съдържа равен брой срещания на }ab\text{ и на }ba\}$.
  \item
    $L = \{\omega_1 \sharp \omega_2 \sharp \omega_3 \mid \forall i \in [1,3](\omega_i \in \{a,b\}^\star\ \&\ |\omega_i| \geq i+1)\}$;
  \end{enumerate}
\end{problem}
  
\marginpar{\ding{45} За всички тези автомати, дефинирайте функцията на преходите им!}
\begin{figure}[H]
  \begin{subfigure}[b]{0.5\textwidth}
    \begin{tikzpicture}[->,>=stealth,thick,node distance=45pt]
      \tikzstyle{every state}=[circle,minimum size=20pt,auto]
      
      \node[initial below, state]   (0) {$s$};
      \node[state]            (1) [right of=0]{$q_1$};
      \node[state]            (2) [right of=1]{$q_2$};
      \node[state,accepting]  (3) [right of=2]{$q_3$};
      
      \path 
      (0) edge [loop above]   node [above] {$a$}    (0)
      (0) edge [bend left=15] node [above] {$b$}    (1)
      (1) edge [loop above]   node [above] {$b$}    (1)
      (1) edge [bend left=15] node [above] {$a$}    (2)
      (2) edge [bend left=30] node [below] {$a$}    (0)
      (2) edge [bend left=15] node [above] {$b$}    (3)
      (3) edge [loop above]   node [above] {$a,b$}  (3);
    \end{tikzpicture}
    \caption{$\{\omega \in \{a,b\}^\star \mid \omega\mbox{ съдържа }bab\}$}
  \end{subfigure}
  \quad
  \begin{subfigure}[b]{0.4\textwidth}
    \begin{tikzpicture}[->,>=stealth,thick,node distance=45pt]
      \tikzstyle{every state}=[circle,minimum size=20pt,auto]
      
      \node[initial below, state]   (0) {$s$};
      \node[state]            (1) [right of=0]{$q_1$};
      \node[state,accepting]  (2) [right of=1]{$q_2$};
      
      \path 
      (0) edge [loop above]   node [above] {$b$}    (0)
      (0) edge [bend left=15] node [above] {$a$}    (1)
      (1) edge [loop above]   node [above] {$b$}    (1)
      (1) edge [bend left=15] node [above] {$a$}    (2)
      (2) edge [loop above]   node [above] {$a,b$}  (2);
    \end{tikzpicture}
    \caption{$\{\omega \in \{a,b\}^\star \mid N_a(\omega) \geq 2\}$}
  \end{subfigure}
  \begin{subfigure}[b]{0.5\textwidth}
    \begin{tikzpicture}[->,>=stealth,thick,node distance=45pt]
      \tikzstyle{every state}=[circle,minimum size=20pt,auto]
      
      \node[initial below, accepting, state] (0) {$s$};
      \node[state]                     (1) [right of=0]{$q_1$};
      
      \path 
      (0) edge [loop above]   node [above] {$b$}   (0)
      (0) edge [bend left=15] node [above] {$a$}   (1)
      (1) edge [bend left=15] node [below] {$b$}   (0);
    \end{tikzpicture}
    \caption{$\{\omega \in \{a,b\}^\star \mid $ всяко $a$ в $\omega$ се следва от поне едно $b\}$ }
  \end{subfigure}
  \qquad
  \begin{subfigure}[b]{0.5\textwidth}
    \begin{tikzpicture}[->,>=stealth,thick,node distance=45pt]
      \tikzstyle{every state}=[circle,minimum size=20pt,auto]
      
      \node[initial below, state, accepting]   (0) {$s$};
      \node[state]                       (1) [right of=0]{$q_1$};
      \node[state]                       (2) [right of=1]{$q_2$};
      
      \path 
      (0) edge [loop above]   node   [above] {$b$}    (0)
      (0) edge [bend left=15] node   [above] {$a$}    (1)
      (1) edge [loop above]   node   [above] {$b$}    (1)
      (1) edge [bend left=15] node   [above] {$a$}    (2)
      (2) edge [loop above]   node   [above] {$b$}    (2)
      (2) edge [bend left=30] node   [below] {$a$}    (0);
    \end{tikzpicture}
    \caption{$\{\omega \in \{a,b\}^\star \mid N_a(\omega) \equiv 0\ (\bmod\ 3)\}$}
  \end{subfigure}
\end{figure}    

В повечето от горните задачи е лесно да се съобрази, че построения автомат разпознава желания език.
При по-сложни задачи обаче, ще се наложи да дадем доказателство, като обикновено се прилага 
{\em метода на математическата индукция} върху дължината на думите.
Ще разгледаме няколко такива примера.

\begin{problem}
  Докажете, че езикът $L$ е автоматен, където
  \[L = \{\alpha \in \{a,b\}^\star\ \mid\ \alpha\mbox{ не съдържа две поредни срещания на }a\}.\]
\end{problem}
\begin{proof}
  Да разгледаме $\A = \FA$ с функция на преходите
  \begin{figure}[H]
    \begin{center}
      \begin{tikzpicture}[->,>=stealth,thick,node distance=45pt]
        \tikzstyle{every state}=[circle,minimum size=20pt,auto]
        
        \node[initial, accepting, state] (0) {$s$};
        \node[accepting, state]   (1) [right of=0]{$q_1$};
        \node[state]   (2) [right of=1]{$q_2$};
        
        \path 
        (0) edge [loop above]   node [above] {$b$}   (0)
        (0) edge [bend left=15] node [above] {$a$}   (1)
        (1) edge [bend left=15] node [below] {$b$}   (0)
        (1) edge [bend left=15] node [above] {$a$}   (2)
        (2) edge [loop above]   node [above] {$a,b$} (2);
      \end{tikzpicture}
    \end{center}
 \end{figure}

 Ще докажем, че $L = \L(\A)$.
 Първо ще се концентрираме върху доказателството на $\L(\A) \subseteq L$.
 \marginpar{Озн. $\abs{\alpha}$ - дължината на думата $\alpha$}
 Ще докажем с индукция по дължината на думата $\alpha$, че:
 \begin{enumerate}[(1)]
 \item 
   ако $\delta^\star(s,\alpha) = s$, то
   $\alpha$ не съдържа две поредни срещания на $a$
   и ако $\abs{\alpha} > 0$, то $\alpha$ завършва на $b$;
 \item
   ако $\delta^\star(s,\alpha) = q_1$, то
   $\alpha$ не съдържа две поредни срещания на $a$
   и завършва на $a$.
 \end{enumerate}

 За $\abs{\alpha} = 0$, то твърденията (1) и (2) са ясни (Защо?).
 Да приемем, че твърденията $(1)$ и $(2)$ са верни за произволни думи $\alpha$ с дължина $n$.
 Нека $\abs{\alpha} = n+1$, т.е. $\alpha = \beta x$, където $\abs{\beta} = n$ и $x \in \Sigma$.
 Ще докажем (1) и (2) за $\alpha$.
 \begin{itemize}[-]
 \item 
   Нека $\delta^\star(s,\beta x) = s = \delta(\delta^\star(s,\beta),x)$.
   Според дефиницията на функцията $\delta$, $x = b$ и $\delta^\star(s,\beta) \in \{s,q_1\}$.
   Тогава по {\bf И.П.} за (1) и (2), $\beta$ не съдържа две поредни срещания на $a$.
   Тогава е очевидно, че $\beta x$ също не съдържа две поредни срещания на $a$.
 \item
   Нека $\delta^\star(s,\beta x) = q_1 = \delta(\delta^\star(s,\beta),x)$.
   Според дефиницията на $\delta$, $x = a$ и $\delta^\star(s,\beta) = s$.
   Тогава по {\bf И.П.} за (2), $\beta$ не съдържа две поредни срещания на $a$
   и завършва на $b$.
   Тогава е очевидно, че $\beta x$ също не съдържа две поредни срещания на $a$.
 \end{itemize}
 
 Така доказахме с индукция по дължината на думата, че за всяка дума $\alpha$
 са  изпълнени твърденията $(1)$ и $(2)$. По дефиниция, ако $\alpha \in \L(\A)$,
 то $\delta^\star(s,\alpha) \in \{s,q_1\}$ и от $(1)$ и $(2)$ следва, че и в двата случа
 $\alpha$ не съдържа две поредни срещания на буквата $a$, т.е. $\alpha \in L$.
 С други думи, доказахме, че 
 \[\L(\A) \subseteq L.\]

 Сега ще докажем другата посока, т.е. $L \subseteq \L(\A)$.
 Това означава да докажем, че
 \[(\forall \alpha \in \Sigma^\star)[\alpha \in L\ \Rightarrow\ \delta^\star(s,\alpha) \in F],\]
 \marginpar{Да напомним, че $p \Rightarrow q \equiv \neg q \Rightarrow \neg  p$}
 което е еквивалентно на
 \begin{equation}
   \label{eq:case2}
   (\forall \alpha \in \Sigma^\star)[\delta^\star(s,\alpha) \not\in F \ \Rightarrow\ \alpha\not\in L].
 \end{equation}
 Това е лесно да се съобрази.
 Щом $\delta^\star(s,\alpha) \not\in F$, то 
 $\delta^\star(s,\alpha) = q_2$ и думата $\alpha$ може да се представи по следния начин:
 \[\alpha = \beta a \gamma\ \&\ \delta^\star(s,\beta) = q_1.\]
 
 Използвайки свойство (2) от по-горе, понеже $\delta^\star(s,\beta) = q_1$, то
 $\beta$ не съдържа две поредни срещания на $a$, но завършва на $a$.
 Сега е очевидно, че $\beta a$ съдържа две поредни срещания на $a$ и 
 щом $\beta a$ е префикс на $\alpha$, то думата $\alpha \not\in L$.
 С това доказахме Свойство \ref{eq:case2}, а следователно и посоката $L\subseteq \L(\A)$.
\end{proof}

\begin{framed}
  За една дума $\alpha \in \{0,1\}^\star$, 
  нека с $\alpha_{(2)}$ да означим числото в десетична бройна система, което се представя в двоична бройна система като $\alpha$.
  Например, $1101_{(2)} = 1 \cdot 2^3+1\cdot 2^2+0\cdot 2^1+1\cdot 2^0 = 13$.
  Тогава имаме следните свойства:
  \begin{itemize}
  \item
    $\varepsilon_{(2)} = 0$,
  \item
    $(\alpha0)_{(2)} = 2\cdot(\alpha)_{(2)}$,
  \item
    $(\alpha1)_{(2)} = 2\cdot(\alpha)_{(2)} + 1$.
  \end{itemize}
\end{framed}
\marginpar{Да отбележим, че за всяко число $n$ има безкрайно много думи $\alpha$, за които $\alpha_{(2)} = n$. Например, $10_{(2)} = 010_{(2)} = 0010_{(2)} = \cdots$}

\begin{problem}
  Докажете, че $L = \{\omega \in \{0,1\}^\star \mid \omega_{(2)} \equiv 2\ (\bmod\ 3)\}$ е автоматен.
\end{problem}
\begin{proof}
  Нашият автомат ще има три състояния $\{q_0,q_1,q_2\}$, като началното състояние ще бъде $q_0$.
  Целта ни е да дефинираме така автомата, че да имаме следното свойство:
  \begin{equation}
    (\forall\alpha\in\Sigma^\star)(\forall i < 3)[\alpha_{(2)} \equiv i\ (\bmod\ 3)\ \Leftrightarrow\ \delta^\star(q_0,\alpha) = q_i],
  \end{equation}
  т.е. всяко състояние отговаря на определен остатък при деление на три.
  Понеже искаме нашия автомат да разпознава тези думи $\alpha$,
  за които $\alpha_{(2)} \equiv 2\mod 3$, финалното състояние ще бъде $q_2$.
  Дефинираме функцията $\delta$ следвайки следните свойства:
  \begin{itemize}
  \item
    \marginpar{$\delta(q_0,0) = q_0$}
    $\alpha_{(2)} \equiv 0 \bmod 3\ \Rightarrow\ (\alpha0)_{(2)} \equiv 0 \bmod 3$;
  \item 
    \marginpar{$\delta(q_0,1) = q_1$}
    $\alpha_{(2)} \equiv 0 \bmod 3\ \Rightarrow\ (\alpha1)_{(2)} \equiv 1 \bmod 3$;
  \item
    \marginpar{$\delta(q_1,0) = q_2$}
    $\alpha_{(2)} \equiv 1 \bmod 3\ \Rightarrow\ (\alpha0)_{(2)} \equiv 2 \bmod 3$;
  \item 
    \marginpar{$\delta(q_1,1) = q_0$}
    $\alpha_{(2)} \equiv 1 \bmod 3\ \Rightarrow\ (\alpha1)_{(2)} \equiv 0 \bmod 3$;
  \item
    \marginpar{$\delta(q_2,0) = q_1$}
    $\alpha_{(2)} \equiv 2 \bmod 3\ \Rightarrow\ (\alpha0)_{(2)} \equiv 1 \bmod 3$;
  \item 
    \marginpar{$\delta(q_2,1) = q_2$}
    $\alpha_{(2)} \equiv 2 \bmod 3\ \Rightarrow\ (\alpha1)_{(2)} \equiv 2 \bmod 3$.
  \end{itemize}
  Ето и картинка на автомата $\A$:
  \begin{figure}[H]
    % \begin{subfigure}[b]{0.3\textwidth}% [$L_1 = L(M_1)$]{
    \begin{center}
      \begin{tikzpicture}[->,>=stealth,thick,node distance=45pt]
        \tikzstyle{every state}=[circle,minimum size=15pt,auto]
        
        \node[initial,state]      (0) {$q_0$};
        \node[state]              (1) [right of=0]{$q_1$};
        \node[accepting, state]   (2) [right of=1]{$q_2$};
        
        \path 
        (0) edge  [loop above]    node [above]  {$0$} (0)
        (0) edge  [bend left=15]  node [above]  {$1$} (1)
        (2) edge  [bend left=15] node [below]  {$0$} (1)
        (1) edge  [bend left=15]  node [below]  {$1$} (0)
        (1) edge  [bend left=15] node [above]  {$0$} (2)
        (2) edge  [loop above]    node [above]  {$1$} (2);
      \end{tikzpicture}
      \end{center}
      \caption{$\L(\A) \stackrel{?}{=} \{\omega\in\{0,1\}^\star \mid \alpha_{(2)} \equiv 2\ (\bmod\ 3)\}$}
 %   \end{subfigure}
 \end{figure}
 \noindent Да разгледаме твърденията:
 \begin{enumerate}[(1)]
  \item 
    $\delta^\star(q_0,\alpha) = q_0\ \Rightarrow\ \alpha_{(2)} \equiv 0 \mod 3$;
  \item 
    $\delta^\star(q_0,\alpha) = q_1\ \Rightarrow\ \alpha_{(2)} \equiv 1 \mod 3$;
  \item 
    $\delta^\star(q_0,\alpha) = q_2\ \Rightarrow\ \alpha_{(2)} \equiv 2 \mod 3$.
  \end{enumerate}
  Ще докажем (1), (2) и (3) {\em едновременно} с индукция по дължината на думата $\alpha$.
  За $\abs{\alpha} = 0$, всички условия са изпълнени. (Защо?)
  Да приемем, че (1), (2) и (3) са изпълнени за думи с дължина $n$.
  Нека $\abs{\alpha} = n+1$, т.е. $\alpha = \beta x$, $\abs{\beta} = n$.
  За да приложим индукционното предположение, ще използваме следното свойство:
  \[\delta^\star(q_0,\beta x) = \delta(\delta^\star(q_0,\beta),x).\]
  
  Ще докажем подробно само (3) понеже другите твърдения се доказват по сходен начин.
  \marginpar{Обърнете внимание, че в доказателството на (3) използваме И.П. не само за (3), но и за (2)}
  Нека $\delta^\star(q_0,\beta x) = q_2$. 
  Имаме два случая:
  \begin{itemize}
  \item 
    $x = 0$. 
    Тогава, по дефиницията на $\delta$, 
    $\delta(q_1,0) = q_2$ и следователно, $\delta^\star(q_0,\beta) = q_1$.
    По {\bf И.П.} за (2) с $\beta$,
    \[\delta^\star(q_0,\beta) = q_1\ \Rightarrow\ \beta_{(2)} \equiv 1 \bmod 3\]
    Тогава, $(\beta0)_{(2)} \equiv 2 \mod 3$. Така доказахме, че
    \[\delta^\star(q_0,\beta 0) = q_2\ \Rightarrow\ (\beta 0)_{(2)} \equiv 2 \bmod 3.\]
  \item
    $x = 1$.
    Тогава, по дефиницията на $\delta$, $\delta(q_2,1) = q_2$ и следователно,
    $\delta^\star(q_0,\beta) = q_2$.
    По {\bf И.П.} за (3) с $\beta$,
    \[\delta^\star(q_0,\beta) = q_2\ \Rightarrow\ \beta_{(2)} \equiv 2 \mod 3.\]
    Тогава, $(\beta1)_{(2)} \equiv 2 \mod 3$. Така доказахме, че
    \[\delta^\star(q_0,\beta 1) = q_2\ \Rightarrow\ (\beta 1)_{(2)} \equiv 2 \mod 3.\]
  \end{itemize}
  
  За да докажем (1), нека $\delta^\star(q_0,\beta x) = q_0$. 
  \begin{itemize}
  \item 
    $x = 0$. Разсъжденията са аналогични, като използваме {\bf И.П.} за (1).
  \item
    $x = 1$. Разсъжденията са аналогични, като използваме {\bf И.П.} за (2).
  \end{itemize}
  
  По същия начин доказваме и (2). Нека $\delta^\star(q_0,\beta x) = q_1$. 
  \begin{itemize}
  \item 
    При $x = 0$, използваме {\bf И.П.} за (3).
  \item
    При $x = 1$, използваме {\bf И.П.} за (1).
  \end{itemize}

  От (1), (2) и (3) следва директно, че $\L(\A) \subseteq L$.
  
  За другата посока, нека $\alpha \in L$, т.е. $(\alpha)_{(2)} \equiv 2 \bmod 3$.
  Ако допуснем, че $\alpha \not\in \L(\A)$, то това означава, че $\delta^\star(q_0,\alpha) \in \{q_0,q_1\}$.
  Но в тези случаи получаваме от твърдения (1) и (2), че $(\alpha)_{(2)} \equiv 0 \bmod 3$ или $(\alpha)_{(2)} \equiv 1 \bmod 3$.
  Това е противоречие с избора на $\alpha \in L$. Следователно, ако $\alpha \in L$, то $\delta(q_0,\alpha) = q_2$.
  Така доказахме и посоката $L \subseteq \L(\A)$.
\end{proof}

\section{Регулярни изрази и езици}

Да фиксираме една непразна азбука $\Sigma$.
\index{регулярен израз}
{\bf Регулярните изрази} $\mathbf{r}$ могат да се опишат със следната абстрактна граматика
\[\mathbf{r} ::= \emptyset\ |\ \varepsilon\ |\ a\ |\ \mathbf{ r_1 \cdot r_2}\ |\ \mathbf{r_1 + r_2}\ |\ \mathbf{r^\star_1},\]
където $\varepsilon$ означава празната дума и $a$ е произволна буква от азбуката $\Sigma$.

Друг начин да се опишат регулярните изрази е по следния начин:
\marginpar{Това е пример за индуктивна дефиниция}
\begin{itemize}
\item 
  Символите $\emptyset$, $\varepsilon$ и всяко $a \in \Sigma$ са регулярен изрази;
\item
  Ако $r_1$ и $r_2$ са регулярни изрази, то $r_1 \cdot r_2$, $r_1 + r_2$ и $r^\star_1$
  също са регулярни изрази.
\end{itemize}

\index{език!регулярен}
\marginpar{Това е друг пример за индуктивна (рекурсивна) дефиниция.}
Сега ще дефинираме езиците, които се описват с регулярни изрази.
Тези езици се наричат {\bf регулярни}.
Това ще направим следвайки индуктивната дефиниция на регулярните изрази,
т.е. за всеки регулярен израз $\mathbf{r}$ ще определим език $\L(\mathbf{r})$.
\begin{itemize}
\item
  $\{\varepsilon\}$ е регулярен език,
  който се разпознава от регулярния израз $\varepsilon$.
  Означаваме $\L(\varepsilon) = \{\varepsilon\}$;
\item
  за всяка буква $a \in \Sigma$, $\{a\}$ е регулярен език,
  който се разпознава от регулярния израз $a$.
  Означаваме $\L(a) = \{a\}$;
\item
  $\emptyset$ е регулярен език,
  който се разпознава от регулярния израз $\emptyset$;
\item
  Нека $L_1$ и $L_2$ са регулярни езици, т.е. съществуват регулярни изрази $\mathbf{r_1}$
  и $\mathbf{r_2}$, за които $\L(\mathbf{r}_1) = L_1$ и $\L(\mathbf{r_2}) = L_2$.
  Тогава:
  \begin{itemize}
  \item 
    \index{обединение}
    $L_1 \cup L_2$ е регулярен език, който се описва с регулярния израз $\mathbf{r_1 + r_2}$.
    Това означава, че $\L(\mathbf{r_1}) \cup \L(\mathbf{r_2}) = \L(\mathbf{r_1+r_2})$.
  \item
    \index{конкатенация}
    \marginpar{Тази операция се наричка конкатенация. Обикновено изпускаме знака $\cdot$}
    $L_1 \cdot L_2$ е регулярен език, който се описва с регулярния израз $\mathbf{r_1 \cdot r_2}$.
    Това означава, че $\L(\mathbf{r_1}) \cdot \L(\mathbf{r_2}) = \L(\mathbf{r_1 \cdot r_2})$.
    % $L_1\cdot L_2 = \{uw\mid u \in L_1\ \&\ w \in L_2\}$, където $L_1$ и $L_2$ са регулярни езици,
    % който се разпознава от регулярния израз $(r_1\cdot r_2)$,
    % където $r_1$ и $r_2$ са регулярните изрази за $L_1$ и $L_2$.
    % Записваме, че $\L(r_1)\cdot\L(r_2) = \L(r_1 \cdot r_2)$.    
  \item
    \marginpar{Звезда на Клини}
    \index{звезда на Клини}
    $L^\star_1$ е регуларен език, който се описва с регулярния израз $\mathbf{r^\star_1}$.
    Това означава, че $L^\star_1 = \L(\mathbf{r^\star_1})$.
  \end{itemize}
\end{itemize}

\begin{remark}
  Ние знаем, че:
  \begin{itemize}
  \item 
    понеже разглеждаме само крайни азбуки, то $\Sigma^\star$ е изброимо безкрайно множество;
  \item
    всички регулярни изрази са изброимо много, откъдето следва, че всички регулярни езици са изброимо много;
  \item
    всички езици над азбуката $\Sigma$ са неизброимо много, защото един език представлява елемент на $\Ps(\Sigma^\star)$.
  \end{itemize}
  От всичко това следва, че има езици, които не са регулярни.
  По-нататък ще видим примери за такива езици.
\end{remark}


\begin{example}
  Нека да разгледаме няколко примера какво точно представлява прилагането
  на операцията звезда на Клини върху един език.
  \begin{itemize}
  \item 
    Нека $L = \{0,11\}$. Тогава:
    \begin{itemize}
    \item 
      $L^0 = \{\varepsilon\}$, $L^1 = L$,
    \item
      $L^2 = L^1\cdot L^1 = \{00,011,110,1111\}$,
    \item
      $L^3 = L^1\cdot L^2 = \{000,0011,0110,01111,1100,11011,11110,111111\}$.
    \end{itemize}
  \item
    Нека $L = \emptyset$.
    Тогава:
    \begin{itemize}
    \item 
      $L^0 = \{\varepsilon\}$,
    \item
      $L^1 = \emptyset$,
    \item
      $L^2 = L^1 \cdot L^1 = \emptyset$.
    \end{itemize}    
    Получаваме, че $L^\star = \{\varepsilon\}$, т.е. {\em краен} език
  \item
    Нека $L = \{0^i\mid i \in \Nat\} = \{\varepsilon, 0, 00, 000, \dots\}$.
    Тогава лесно може да се види, че $L = L^\star$.
  \end{itemize}
\end{example}

\begin{problem}
  За произволни регулярни изрази $r$ и $s$, 
  проверете:
  \begin{enumerate}[a)]
  \item 
    $r+s = s + r$;
  \item
    $(\varepsilon + r)^\star = r^\star$;
  \item
    $\emptyset^\star = \varepsilon$;
  \item
    $(r^\star s^\star) = (r+s)^\star$;
  \item
    $(r^\star)^\star = r^\star$;
  \item
    $(rs + r)^\star r = r(sr+r)^\star$;
  \item
    $s(rs+s)^\star r = rr^\star s(rr^\star s)^\star$;
  \item
    $(r+s)^\star = r^\star + s^\star$;
  \item
    $\emptyset^\star = \varepsilon^\star$;
  \end{enumerate}
\end{problem}

\begin{framed}
\begin{thm}[Клини]
  \index{Клини}
  Всеки автоматен език се описва с регулярен израз.
\end{thm}
\end{framed}
\begin{proof}
  \marginpar{стр. 79 от \cite{papadimitriou}, стр. 33 от \cite{hopcroft1}}
  Нека  $L = \L(\A)$, за някой краен детерминиран автомат $\A$.
  Да фиксираме едно изброяване на състоянията $Q = \{q_1,\dots,q_n\}$,
  като началното състояние е $q_1$.
  Ще означаваме с $L(i,j,k)$ множеството от тези думи, които
  могат да се разпознаят от автомата по път, който започва от $q_i$,
  завършва в $q_j$, и междинните състояния имат индекси $\leq k$.
  Например, за думата $\alpha = a_1a_2\cdots a_n$ имаме, че $\alpha \in L(i,j,k)$
  точно тогава, когато съществуват състояния $q_{l_1},\dots,q_{l_{n-1}}$, като $l_1,\dots,l_{n-1} \leq k$ и
  \[q_i\stackrel{a_1}{\rightarrow} q_{l_1} \stackrel{a_2}{\rightarrow} q_{l_2} \stackrel{a_3}{\rightarrow} \dots \stackrel{a_{n-1}}{\rightarrow} q_{l_{n-1}}\stackrel{a_n}{\rightarrow} q_j.\]
  Тогава за $n = \abs{Q}$, 
  \[L(i,j,n) = \{\alpha\in\Sigma^\star\mid \delta^\star(q_i,\alpha) = q_j\}.\]
  Така получаваме, че 
  \[\L(\A) = \bigcup\{L(1,j,n)\mid q_j \in F\} = \bigcup_{q_j\in F}L(1,j,n).\]
  Ще докажем с {\bf индукция по $k$}, че за всяко $i,j,k$, множествата от думи $L(i,j,k)$
  се описват с регулярен израз $r^k_{i,j}$
  \begin{enumerate}[a)]
  \item
    Нека $k = 0$. Ще докажем, че за всяко $i,j$, $L(i,j,0)$ се описва с регулярен израз.
    Имаме да разгледаме два случая.
    
    Ако $i = j$, то 
    \[L(i, j, 0) = \{\varepsilon\}\cup\{a\in\Sigma \mid \delta(q_i,a) = q_j\}.\]
    Ако $i \neq j$, то
    \[L(i, j, 0) = \{a\in\Sigma \mid \delta(q_i, a) = q_j\}.\]
  \item
    Да предположим, че $k > 0$ и за всяко $i$, $j$, можем да намерим регулярните изрази
    съответстващи на $L(i,j,k-1)$. Тогава
    \[L(i,j,k) = L(i,j,k-1)\ \cup\ L(i,k,k-1)\cdot (L(k,k,k-1)^\star) \cdot L(k,j,k-1).\]
    Тогава по {\bf И.П.} следва, че $L(i,j,k)$ може да се опише с регулярен израз, който е
    \[r^{k-1}_{i,j} + r^{k-1}_{i,k}\cdot (r^{k-1}_{k,k})^\star\cdot r^{k-1}_{k,j}.\]
  \end{enumerate}
  Заключаваме, че за всяко $i,j,k$, $L(i,j,k)$ може да се опише с регулярен израз $r^{k}_{i,j}$.
  Тогава ако $F = \{q_{i_1},\dots,q_{i_k}\}$, то $\L(\A)$ се описва с регулярния израз
  \[r^n_{1,i_1} + r^n_{1,i_2} + \dots + r^n_{1,i_k}.\]
\end{proof}

\begin{example}
  \label{fig:a1}
  Да разгледаме следния автомат:
  
  \begin{figure}[H]
    \begin{center}
      \begin{tikzpicture}[->,>=stealth,thick,node distance=45pt]
        \tikzstyle{every state}=[circle,minimum size=15pt,auto]
        
        \node[initial,state]      (1) {$q_1$};
        \node[accepting, state]   (2) [right of=1]{$q_2$};
        
        \path 
        (1) edge [loop above]  node [above] {$1$} (1)
        (1) edge  node [above] {$0$} (2)
        (2) edge [loop above] node [above] {$0,1$} (2);
      \end{tikzpicture}
      \end{center}
 \end{figure}

 За да намерим регулярния език за автомата от Пример \ref{fig:a1}, 
 трябва да намерим $r^2_{1,2}$, защото началното състояние е $q_1$, финалното е $q_2$ и 
 броят на състоянията в автомата е $2$.
 \begin{align*}
   r^0_{1,1} =\ & \varepsilon + 1,\\
   r^0_{1,2} =\ & 0,\\
   r^0_{2,1} =\ & \emptyset,\\
   r^0_{2,2} =\ & \varepsilon +  0 + 1,\\
    r^1_{1,2} =\ & r^0_{1,2} + r^0_{1,1}\cdot(r^0_{1,1})^\star \cdot r^0_{1,2} = 0 + (\varepsilon + 1)(\varepsilon + 1)^\star0 = 1^\star0,\\
    r^1_{2,2} =\ & r^0_{2,2} + r^0_{2,1} \cdot (r^0_{1,1})^\star\cdot r^0_{1,2} = \varepsilon + 0 + 1 + \emptyset(\varepsilon + 1)^\star0 = \varepsilon + 0 + 1\\
    r^2_{1,2} =\ & r^{1}_{1,2} + r^{1}_{1,2}(r^1_{2,2})^\star r^1_{2,2} \\
    =\ & 1^\star0 + 1^\star0 (\varepsilon + 0 + 1)^\star (\varepsilon + 0 + 1) = 1^\star 0 (0 + 1)^\star.
  \end{align*}
Ясно е, че $L_1$ се описва с регулярния израз $r^2_{1,2} = 1^\star 0 (0 + 1)^\star$.
\end{example}

Следващата ни цел е да видим, че имаме и обратната посока на горната лема.
Ще докажем, че всеки регулярен език е автоматен. За тази цел първо ще 
въведем едно обобщение на понятието краен детерминиран автомат.

\section{Недетерминирани крайни автомати}
\index{автомат!недетерминиран}
\begin{dfn}
  \marginpar{Въведени от Рабин и Скот \cite{rabin-scott}}
  \marginpar{За яснота, често ще означаваме с $\N$ недетерминирани автомати, а с $\A$ детерминирани автомати}
  Недетерминиран краен автомат представлява
  \[\N = \NFA,\]
  \begin{itemize}
  \item
    $Q$ е крайно множество от състояния;
  \item
    $\Sigma$ е крайна азбука;
  \item
    $\Delta: Q\times\Sigma \to \Ps(Q)$ е функцията на преходите.
    \marginpar{Да напомним, че $\Ps(Q) = \{R\mid R\subseteq Q\}$, $\abs{\Ps(Q)} = 2^{\abs{Q}}$}
    \marginpar{Sipser позволява $\epsilon$-преходи}
    Обърнете внимание, че тя е тотална.
  \item
    $s \in Q$ е началното състояние;
  \item
    $F\subseteq Q$ е множеството от финални състояния.
  \end{itemize}
\end{dfn}

Удобно е да разширим функцията на преходите $\Delta: Q\times\Sigma \to \Ps(Q)$ 
до функцията $\Delta^\star: Q\times\Sigma^\star \to \Ps(Q)$ по следния начин:
\begin{itemize}
\item 
  $\Delta^\star(q, \varepsilon) = \{q\}$;
\item
  \marginpar{Съобразете, че $\Delta^\star(q, \alpha b) = \bigcup_{p \in \Delta^\star(q,\alpha)} \Delta(p,b)$}
  $\Delta^\star(q, b\alpha) = \bigcup_{p \in \Delta(q,b)} \Delta^\star(p, \alpha)$;
\end{itemize}

\begin{framed}
\begin{thm}[Рабин-Скот \cite{rabin-scott}]
  За всеки НKА $\N$ съществува еквивалентен на него ДКА $\D$, т.е. $\L(\N) = \L(\D)$.
\end{thm}
\end{framed}
\begin{hint}
  Нека $\N = \NFA$. Ще построим детерминиран автомат
  \[\D = (Q',\Sigma,\delta,s',F'),\]
  за който $\L(\N) = \L(\D)$.
  Конструкцията е следната:
  \marginpar{Да отбележим, че детерминираният автомат $\D$ има не повече от $2^{\abs{Q}}$ на брой състояния $Q'$}
  \begin{itemize}
  \item
    $Q' = \Ps(Q)$;
  \item
    $\delta(R,a) = \{q\in Q\mid (\exists r\in R)[q\in\Delta(r,a)]\} = \bigcup_{r\in R}\Delta(r,a)$;
  \item
    $s' = \{s\}$;
  \item
    $F' = \{R \subseteq Q \mid R\cap F \neq \emptyset\}$.
  \end{itemize}
\end{hint}

% \begin{problem}
%   За дума $\alpha = a_1a_2\cdots a_n$, дефинираме $\alpha^R = a_na_{n-1}\cdots a_1$.
%   \marginpar{Индукция по $\abs{\beta}$.}
%   Докажете, че
%   \[(\forall \alpha,\beta\in\Sigma^\star)[(\alpha\beta)^R = \beta^R\alpha^R].\]
% \end{problem}

\begin{problem}
  За всеки НКА $\N$ съществува НКА $\N'$ с едно финално състояние, 
  за който $\L(\N) = \L(\N')$.
\end{problem}
\begin{hint}
  Вместо формална конструкция, да разгледаме един пример, който илюстрира идеята.
  \begin{figure}[H]
    \begin{subfigure}[b]{0.3\textwidth}
      \begin{tikzpicture}[framed,->,>=stealth,thick,node distance=45pt]
        \tikzstyle{every state}=[circle,minimum size=20pt,auto]
        \node[initial,state]      (1) {$s$};
        \node[state,accepting]     [above right of=1] (2) {$q_1$};
        \node[state,accepting]     [below right of=1] (3) {$q_2$};
        \path
        (1) edge [bend left=15] node  [above] {$a$} (2)
        (2) edge [bend left=15] node  [right] {$b$} (1)
        % (2) edge [loop above] node  [above] {$a$} (2)
        (2) edge [bend left=15] node  [right] {$a$} (3)
        (3) edge [bend left=15] node  [below] {$a$} (1)
        (3) edge [loop below] node  [right] {$b$} (3);
        % (1) edge [bend right=15] node [below] {$b$} (3);
      \end{tikzpicture}
      \caption{автомат $\N$}
    \end{subfigure}
    \qquad
    \qquad
    \begin{subfigure}[b]{0.4\textwidth}
      \begin{tikzpicture}[framed,->,>=stealth,thick,node distance=45pt]
        \tikzstyle{every state}=[circle,minimum size=20pt,auto]
        \node[initial,state]      (1) {$s$};
        \node[state]     [above right of=1] (2) {$q_1$};
        \node[state]     [below right of=1] (3) {$q_2$};
        \node[state,accepting]     [right=3cm of 1] (4) {$f$};
        \path
        (1) edge [bend left=15] node  [above] {$a$} (2)
        % (2) edge [loop above] node  [above] {$a$} (2)
        (2) edge [bend left=15] node  [right] {$b$} (1)
        (2) edge [bend left=15] node  [right] {$a$} (3)
        (3) edge [loop below] node  [right] {$b$} (3)
        (3) edge [bend left=15] node  [below] {$a$} (1)
        (1) edge [dashed,bend left=15] node  [above] {$a$} (4)
        (2) edge [dashed,bend left=15] node  [above] {$a$} (4)
        (3) edge [dashed,bend right=15] node  [below] {$b$} (4);
        % (1) edge [bend right=15] node [below] {$b$} (3);
      \end{tikzpicture}
    \caption{автомат $\N'$, $\L(\N') = \L(\N)$}
  \end{subfigure}
\end{figure}  
За произволен автомат $\N$, формулирайте точно конструкцията на $\N'$ с едно финално състояние и докажете, че наистина $\L(\N) = \L(\N')$.
Обърнете внимание, че примера показва, че е възможно $\N$ да е детерминиран автомат, но полученият $\N'$ да бъде недетерминиран.
\end{hint}

\begin{problem}
  \marginpar{Нека $\A$, $L = \L(\A)$, е само с едно финално състояние. }
  Докажете, че ако $L$ е автоматен език, то $L^R = \{\omega^R \mid \omega \in L\}$
  също е автоматен.
\end{problem}

\begin{lemma}
  Съществува НКА $\N = \NFA$, който разпознава езика $L(r)$, 
  където $r = \emptyset$, $r = \varepsilon$ или $r = a$, за $a\in \Sigma$.
\end{lemma}
\begin{hint}
  \begin{figure}[H]
    \begin{subfigure}[b]{0.2\textwidth}
      \label{subf:a1}
      \begin{tikzpicture}[->,>=stealth,thick,node distance=35pt]
        \tikzstyle{every state}=[circle,minimum size=15pt,auto]
        \node[initial,state]      (1) {$s$};
      \end{tikzpicture}
      \caption{$L(\emptyset)$}
    \end{subfigure}
    \qquad
    \begin{subfigure}[b]{0.2\textwidth}
      \begin{tikzpicture}[->,>=stealth,thick,node distance=35pt]
        \tikzstyle{every state}=[circle,minimum size=15pt,auto]
        \node[initial,state,accepting]      (1) {$s$};
      \end{tikzpicture}
      \caption{$L(\varepsilon)$}
    \end{subfigure}
    \qquad
    \begin{subfigure}[b]{0.3\textwidth}
      \begin{tikzpicture}[->,>=stealth,thick,node distance=35pt]
        \tikzstyle{every state}=[circle,minimum size=15pt,auto]
        \node[initial,state]      (1)              {$s$};
        \node[accepting,state]    (2) [right of=1] {$q$};
        \path 
        (1) edge  node [above] {$a$} (2);
      \end{tikzpicture}
      \caption{$L(a)$}
    \end{subfigure}
  \end{figure}
\end{hint}

\begin{lemma}
  Класът на автоматните езици е затворен относно операцията {\bf конкатенация}.
  Това означава, че ако $L_1$ и $L_2$ са два произволни автоматни езика, то $L_1\cdot L_2$
  също е автоматен език.
\end{lemma}
\begin{proof}
  Нека са дадени автоматите:
  \begin{itemize}
  \item
    $\N_1 = \NFAn{1}$, като $\L(\N_1) = L_1$;
  \item
    $\N_2 = \NFAn{2}$, като $\L(\N_2) = L_2$.
  \end{itemize}
  Ще дефинираме автомата $\N = \NFA$ като
  \[\L(\N) = L_1\cdot L_2 = \L(\N_1)\cdot\L(\N_2).\]
  \begin{itemize}
  \item
    $Q = Q_1 \cup Q_2$;
  \item
    $s = s_1$;
  \item
    $F = 
    \begin{cases}
      F_1 \cup F_2, & \text{ ако } s_2 \in F_2\\
      F_2,          & \text{ иначе}.
    \end{cases}$
  \item 
    $\Delta(q,a) = 
    \begin{cases}
      \Delta_1(q,a),                      & \text{ ако }q\in Q_1\setminus F_1\ \&\ a\in\Sigma\\
      \Delta_2(q,a),                      & \text{ ако }q\in Q_2\ \&\ a\in\Sigma\\
      \Delta_1(q,a) \cup \Delta_2(s_2,a), & \text{ ако }q \in F_1\ \&\ a\in\Sigma.
    \end{cases}$
  \end{itemize}
\end{proof}

\begin{figure}[H]
  \center
  \begin{subfigure}[b]{0.3\textwidth}
    \label{subf:a1}
    \begin{tikzpicture}[framed,->,>=stealth,thick,node distance=45pt]
      \tikzstyle{every state}=[circle,minimum size=15pt,auto]
      \node[initial,state,accepting]      (1) {$s_1$};
      \node[state]                        (2) [right of=1] {$q_1$};
      \node[state]                        (3) [above right of=2] {$q_2$};
      \node[state,accepting]              (4) [below right of=2] {$q_3$};
      \path
      (1) edge node [above] {$a$} (2)
      (2) edge node [above] {$a$} (3)
      (2) edge node [below] {$b$} (4)
      (3) edge [bend right=30] node [above] {$a$} (1)
      (4) edge [bend left=30] node [below] {$b$} (1);
    \end{tikzpicture}
    \caption{автомат $\N_1$}
  \end{subfigure}
  \qquad
  \qquad
  \qquad
  \begin{subfigure}[b]{0.3\textwidth}
    \begin{tikzpicture}[framed,->,>=stealth,thick,node distance=45pt]
      \tikzstyle{every state}=[circle,minimum size=15pt,auto]
      \node[initial,state]      (1) {$s_2$};
      \node[state]     [above right of=1] (2) {$q_4$};
      \node[state,accepting]     [below right of=1] (3) {$q_5$};
      \path
      (1) edge [bend left=15] node  [above] {$a$} (2)
      (2) edge [bend left=15] node  [right] {$a$} (3)
      (1) edge [bend right=15] node [below] {$b$} (3);
    \end{tikzpicture}
    \caption{автомат $\N_2$}
  \end{subfigure}
\end{figure}

\begin{example}
    За да построим автомат, който разпознава конкатенацията на $\L(\N_1)$ и $\L(\N_2)$,
    трябва да свържем финалните състояния на $\N_1$ с изходящите от $s_2$ състояния на $\N_2$.
    
    \begin{figure}[H]
      \center
      % \begin{subfigure}[b]{0.3\textwidth}
      \begin{tikzpicture}[framed,->,>=stealth,thick,node distance=2cm]
        \tikzstyle{every state}=[circle,minimum size=15pt,auto]
        \node[initial,state]                      (1) {$s_1$};
        \node[state] [right of=1]                 (2) {$q_1$};
        \node[state] [above right of=2]           (3) {$q_2$};
        \node[state] [below right of=2]           (4) {$q_3$};
        \node[state] [right=4cm of 1]             (5) {$s_2$};
        \node[state] [above right of=5]           (6) {$q_4$};
        \node[state,accepting] [below right of=5] (7) {$q_5$};
        \path
        (1) edge node [above]                         {$a$} (2)
        (2) edge node [above]                         {$a$} (3)
        (2) edge node [below]                         {$b$} (4)
        (3) edge [bend right=15] node [above]         {$a$} (1)
        (4) edge [bend left=15] node [below]          {$b$} (1)
        (5) edge [bend left=15] node [below]          {$a$} (6)
        (6) edge [bend left=15] node [right]          {$a$} (7)
        (5) edge [bend right=15] node [above]         {$b$} (7)
        (1) edge [dashed, bend left=45] node [above]  {$a$} (6)
        (1) edge [dashed, bend right=45] node [below] {$b$} (7)
        (4) edge [dashed, bend left=45] node [above]  {$a$} (6)
        (4) edge [dashed, bend left=10] node [above]  {$b$} (7);
      \end{tikzpicture}
      \caption{$\L(\N) = \L(\N_1)\cdot\L(\N_2)$}
  \end{figure}  
  Обърнете внимание, че $\N_1$ и $\N_2$ са детерминирани автомати, но $\N$ е недетерминиран.
  Също така, в този пример се оказва, че вече $s_2$ е недостижимо състояние, но в общия случай не можем да 
  го премахнем, защото може да има преходи влизащи в $s_2$.
\end{example}


\begin{lemma}
  Класът от автоматните езици е затворен относно операцията {\bf обединение}.
\end{lemma}
\begin{hint}
  Нека са дадени автоматите:
  \begin{itemize}
  \item 
    $\N_1 = \NFAn{1}$, като $L(\N_1) = L_1$;
  \item
    $\N_2=\NFAn{2}$, като $L(\N_2) = L_2$.
  \end{itemize}
  Ще дефинираме автомата $\N=\NFA$, така че
  \[L(\N) = L(\N_1) \cup L(\N_2).\]
  \begin{itemize}
  \item 
    $Q = Q_1 \cup Q_2 \cup \{s\}$;
  \item
    $F = 
    \begin{cases}
      F_1 \cup F_2 \cup \{s\}, & \text{ ако } s_1 \in F_1 \vee s_2 \in F_2\\
      F_1 \cup F_2,            & \text{ иначе } 
    \end{cases}$
  \item
    $
    \Delta(q,a) = 
    \begin{cases}
      \Delta_1(q,a),                       & \text{ ако } q\in Q_1\ \&\ a\in\Sigma\\
      \Delta_2(q,a),                       & \text{ ако } q\in Q_2\ \&\  a\in\Sigma\\
      \Delta_1(s_1,a) \cup \Delta_2(s_2,a), & \text{ ако } q = s\ \&\  a \in\Sigma.
    \end{cases}
    $
    % $\Delta(q,a) = \Delta_1(q,a)$, за всяко $q\in Q_1$, $a\in\Sigma$;
  % \item
  %   $\Delta(q,a) = \Delta_2(q,a)$, за всяко $q\in Q_2$, $a\in\Sigma$;
  % \item
  %   $\Delta(s,a) = \Delta_1(s_1,a) \cup \Delta_2(s_2,a)$, за всяко $a\in\Sigma$;
  \end{itemize}
\end{hint}
\begin{remark}
  В началното състояние на новопостроения автомат $\N$ не влизат ребра.
\end{remark}


\begin{example}
    За да построим автомат, който разпознава обединението на $\L(\N_1)$ и $\L(\N_2)$,
    трябва да свържем финалните състояния на $\N_1$ с изходящите от $s_2$ състояния на $\N_2$.
    
    \begin{figure}[H]
      \center
      % \begin{subfigure}[b]{0.3\textwidth}
      \begin{tikzpicture}[framed,->,>=stealth,thick,node distance=2cm]
        \tikzstyle{every state}=[circle,minimum size=20pt,auto]
        \node[initial,state,accepting]      (0) {$s$};
        \node[state,accepting]    [above right of=0]        (1) {$s_1$};
        \node[state]    [right of=1]        (2) {$q_1$};
        \node[state]                        (3) [above right of=2] {$q_2$};
        \node[state,accepting]                        (4) [below right of=2] {$q_3$};
        \node[state]    [below right=2cm of 0] (5) {$s_2$};
        \node[state]     [above right of=5] (6) {$q_4$};
        \node[state,accepting]     [below right of=5] (7) {$q_5$};
        \path
        (1) edge node [above]                  {$a$} (2)
        (2) edge node [above]                  {$a$} (3)
        (2) edge node [below]                  {$b$} (4)
        (3) edge [bend right=15] node [above]  {$a$} (1)
        (4) edge [bend left=15]  node [below]  {$b$} (1)
        (5) edge [bend left=15] node [below]   {$a$} (6)
        (6) edge [bend left=15] node  [right] {$a$} (7)
        (5) edge [bend right=15] node [above]  {$b$} (7)
        (0) edge [dashed, bend right=15] node [below]  {$a$} (2)
        (0) edge [dashed, bend right=15] node [below]  {$a$} (6)
        (0) edge [dashed, bend right=45] node [below]  {$b$} (7);
      \end{tikzpicture}
      \caption{$\L(\N) = \L(\N_1)\cup\L(\N_2)$}
  \end{figure}  
  Обърнете внимание, че $\N_1$ и $\N_2$ са детерминирани автомати, но $\N$ е недетерминиран.
  Освен това, новото състояние $s$ трябва да бъде маркирано като финално, защото $s_1$ е финално.
\end{example}

\begin{lemma}
  Класът от автоматните езици е затворен относно операцията {\bf звезда на Клини}.
\end{lemma}
\begin{proof}
  Нека е даден автомата $\N = \NFA$, за който е изпънено, че
  $L(\N) = L(r)$.
  Първата стъпка е да построим $\N_1 = \NFAn{1}$, такъв че 
  \[L(\N_1) = \bigcup_{n\geq 1} (L(\N))^n = \bigcup_{n\geq 1} (L(r))^n = L(r^+).\]
  \begin{itemize}
  \item
    $Q_1 = Q$;
  \item
    $s_1 = s$;
  \item
    $F_1 = F$;
  \item
    $
    \Delta_1(q,a) = 
    \begin{cases}
      \Delta(q,a), & \text{ ако } q\in Q\setminus F, a \in \Sigma\\
      \Delta(q,a) \cup \Delta(s,a), & \text{ ако } q\in F, a\in\Sigma.
    \end{cases}
    $
    % $\Delta_1(q,a) = \Delta(q,a)$, за всяко $q\in Q\setminus F$, $a\in\Sigma$;
  % \item
  %   $\Delta_1(q,a) = \Delta(q,a) \cup \Delta(s,a)$, за всяко $q\in F$, $a\in\Sigma$;
  \end{itemize}
  Накрая строим автомат $\N_2$, за който $L(\N_2) = \{\varepsilon\} \cup L(\N_1)$.
\end{proof}


\begin{example}
  Нека да приложим конструкцията за да намерим автомат разпознаващ $\L(\N_3)^\star$.
  
  \begin{figure}[H]
    % \center
    \begin{subfigure}[b]{0.3\textwidth}
      \begin{tikzpicture}[framed,->,>=stealth,thick,node distance=45pt]
        \tikzstyle{every state}=[circle,minimum size=15pt,auto]
        \node[initial above,state]      (1) {$s_1$};
        \node[state]              (2) [right of=1] {$q_1$};
        \node[state,accepting]    (3) [right of=2] {$q_2$};
        \path
        (1) edge node [above] {$a$} (2)
        (2) edge node [above] {$b$} (3)
        (3) edge [bend left=45] node [below] {$b$} (1);
      \end{tikzpicture}
      \caption{автомат $\N_3$}
    \end{subfigure}
    \hspace{2cm}
    \begin{subfigure}[b]{0.5\textwidth}
      \begin{tikzpicture}[framed,->,>=stealth,thick,node distance=45pt]
        \tikzstyle{every state}=[circle,minimum size=20pt,auto]
        \node[initial above,state,accepting] (0) {$s$};
        \node[state]                   (1) [below right of=0] {$s_1$};
        \node[state]                   (2) [right of=1] {$q_1$};
        \node[state,accepting]         (3) [right of=2] {$q_2$};
        \path
        (0) edge [dashed, bend left=15] node [above] {$a$} (2)
        (1) edge node [above] {$a$} (2)
        (2) edge node [below] {$b$} (3)
        (3) edge [bend left=45] node [below] {$b$} (1)
        (3) edge [dashed, bend right=45] node [above] {$a$} (2);        
      \end{tikzpicture}
      \caption{$\L(\N) = \L(\N_3)^\star = \L(\N_3)^+ \cup \{\varepsilon\}$}
    \end{subfigure}
  \end{figure}
    
  Лесно се вижда, че $\L(\N_1) = \{(abb)^nab\mid n\in\Nat\}$.
  Формално погледнато, след като построим автомат за езика $\L(\N_1)^+$, трябва да приложим
  конструкцията за обединение на автомата за езика $\L(\N_1)^+$ с автомата за езика $\{\varepsilon\}$.
  Защо трябва да добавим ново начално състояние $s$?
  Да допуснем, че вместо това сме направили $s_1$ финално.
  Тогава има опасност да разпознаем повече думи. Например, думата $abb$ би се разпознала от този автомат,
  но $abb \not\in\L(\N_1)^\star$.
  
\end{example}
% \begin{remark}
%   Запазваме свойството, че в началното състояние не влизат ребра.
% \end{remark}

\begin{problem}
  Да фиксираме една дума $\alpha$ над дадена азбука $\Sigma$.
  Опишете алгоритъм, който за вход произволен текстов файл, чието съдържание означаваме с $\beta$,
  отговаря дали думата $\alpha$ се среща в $\beta$.
  Каква е сложността на този алгоритъм относно дължините на $\alpha$ и $\beta$ ?
\end{problem}


\section{Езици, които не са регулярни}
\begin{lemma}[за покачването (регулярни езици)]
  \index{лема за покачването!регулярни езици}
  \label{lem:pumping-reg}
  \marginpar{На англ. се нарича \\ Pumping Lemma}
  \marginpar{Има подобна лема и за безконтекстни езици}
  \marginpar{Обърнете внимание, че $0 \in \Nat$ и $xy^0z =  xz$}
  Нека $L$ да бъде регулярен език.
  Съществува число $p\geq 1$, зависещо само от $L$, 
  за което за всяка дума $\alpha\in L, \abs{\alpha}\geq p$ може да 
  бъде записана във вида $\alpha = xyz$ и 
  \begin{enumerate}[1)]
  \item
    $|y|\geq 1$;
  \item
    $|xy|\leq p$;
  \item
    $(\forall i\in\Nat)[xy^iz \in L]$.
  \end{enumerate}
\end{lemma}
\begin{proof}
  \marginpar{стр. 88 от \cite{papadimitriou}, стр. 78 от \cite{sipser1}}
  Понеже $L$ е регулярен, той се разпознава от $\A = \FA$.
  Да положим $p = \abs{Q}$ и нека $\alpha = a_1a_2\cdots a_k$ е дума, за която $k \geq p$.
  Да разгледаме първите $p$ стъпки от изпълнението на $\alpha$ върху $\A$:
  \[q_0\stackrel{a_1}{\rightarrow} q_1 \stackrel{a_2}{\rightarrow} \dots \stackrel{a_p}{\rightarrow} q_p.\]
  Тъй като $\abs{Q} = p$, а по този път участват $n+1$ състояния $q_0,q_1,\dots,q_p$,
  то съществуват числа $i, j$, за които $0\leq i < j\leq p$ и $q_i = q_j$.
  Нека разделим думата $\alpha$ на три части по следния начин:
  \[x = a_1\cdots a_i,\quad y = a_{i+1}\cdots a_j,\quad z = a_{j+1}\cdots a_k.\]
  Ясно е, че $\abs{y} \geq 1$ и $\abs{xy} = j \leq p$.
  \marginpar{\ding{45} Докажете!}
  Освен това, лесно се съобразява, че за всяко $i \in\Nat$,
  $xy^iz \in L$. Да разгледаме само случая за $i = 0$.
  Думата $xy^0z = xz \in L$, защото имаме следното изчисление:
  \[q_0\stackrel{a_1}{\rightarrow} \cdots \stackrel{a_i}{\rightarrow} q_i\stackrel{a_{j+1}}{\rightarrow}q_{j+1}\cdots\stackrel{a_{p}}{\rightarrow}q_p\in F,\]
  защото $q_i = q_j$.
\end{proof}


% При фиксиран език $L$, условието на \Lem{pumping-reg} може формално да се запише така:
% {\scriptsize
% \[(\exists p \geq 1)(\forall \alpha \in L)[\abs{\alpha} \geq p \Rightarrow (\exists x,y,z\in\Sigma^\star)[\alpha = xyz\ \wedge\ \abs{y} \geq 1\ \wedge\ \abs{xy} \leq p\ \wedge\ (\forall i\in\Nat)[xy^iz \in L]]].\]}
% Отрицанието на горната формула може да се запише по следния начин:
% {\scriptsize  \[(\forall p \geq 1)(\exists \alpha \in L)[\abs{\alpha} \geq p\ \wedge (\forall x,y,z\in\Sigma^\star)[\alpha \neq xyz\ \vee\ \abs{y} \not\geq 1\ \vee\ \abs{xy} \not\leq p\ \vee\ (\exists i\in\Nat)[xy^iz \not\in L]]],\]}
% което е еквивалентно на:
% {\scriptsize
%   % \begin{equation}
%   %   \label{pump-neg}
%   \[(\forall p \geq 1)(\exists \alpha \in L)[\abs{\alpha} \geq p\ \wedge\ (\forall x,y,z\in\Sigma^\star)[(\alpha = xyz \wedge \abs{y} \geq 1\wedge \abs{xy} \leq p) \Rightarrow (\exists i\in\Nat)[xy^iz \not\in L]]].\]}
% % \end{equation}

% Това означава, че условието на \Lem{pumping-reg} може да се запише така:\\
% Ако условието \ref{pump-neg} е изпълнено, то $L$ не е регулярен.

% \begin{framed}
%   \Lem{pumping-reg} е полезна, когато искаме да докажем, че даден език $L$ {\bf не} е регулярен.
%   За да постигнем това, ние доказваме {\bf отрицанието} на условията от \Lem{pumping-reg} за $L$, т.е.
%   за всяка константа $p \geq 1$, намираме дума $\alpha \in L$, $\abs{\alpha}\geq p$, такава че за всяко разбиване на думата на три части, $\alpha = xyz$,
%   със свойствата $\abs{y} \geq 1$ и $\abs{xy} \leq p$, е изпълнено, че $(\exists i)[xy^iz \not\in L]$.
% \end{framed}

Практически е по-полезно да разглеждаме следната еквивалентна формулировка на лемата за покачването.
\marginpar{Контрапозиция на твърдението $p \to q$ е твърдението $\neg q \to \neg p$}
\begin{cor}[Контрапозиция на лемата за покачването]
  \label{cor:pumping-reg}
  \marginpar{Ясно е, че всеки краен език е регулярен. Нали?}
  Нека $L$ е произволен {\bf безкраен} език. Нека също така е изпълнено, че за всяко естествено число $p \geq 1$ можем да намерим дума $\alpha \in L$, $\abs{\alpha}\geq p$, такава че за всяко разбиване на думата на три части, $\alpha = xyz$,
  със свойствата $\abs{y} \geq 1$ и $\abs{xy} \leq p$, е изпълнено, че $(\exists i)[xy^iz \not\in L]$.
  Тогава $L$ {\bf не} е регулярен език.
\end{cor}
\begin{proof}
  \Lem{pumping-reg} гласи, че ако $L$ е регулярен език, то
  {\scriptsize
    \[(\exists p \geq 1)(\forall \alpha \in L)[\abs{\alpha} \geq p \Rightarrow (\exists x,y,z\in\Sigma^\star)[\alpha = xyz\ \wedge\ \abs{y} \geq 1\ \wedge\ \abs{xy} \leq p\ \wedge\ (\forall i\in\Nat)[xy^iz \in L]]].\]}
  Отрицанието на горното твърдение гласи, че ако 
  {\scriptsize  \[(\forall p \geq 1)(\exists \alpha \in L)[\abs{\alpha} \geq p\ \wedge (\forall x,y,z\in\Sigma^\star)[\alpha \neq xyz\ \vee\ \abs{y} \not\geq 1\ \vee\ \abs{xy} \not\leq p\ \vee\ (\exists i\in\Nat)[xy^iz \not\in L]]],\]}
  то $L$ {\bf не} е регулярен език.
  Горната формула е еквивалентна на:
  {\scriptsize
    % \begin{equation}
    %   \label{pump-neg}
    \[(\forall p \geq 1)(\exists \alpha \in L)[\abs{\alpha} \geq p\ \wedge\ (\forall x,y,z\in\Sigma^\star)[(\alpha = xyz \wedge \abs{y} \geq 1\wedge \abs{xy} \leq p) \Rightarrow (\exists i\in\Nat)[xy^iz \not\in L]]].\]}
\end{proof}


\begin{example}
  Езикът $L = \{a^nb^n \mid n\in \Nat\}$ {\bf не} ерегулярен.
\end{example}
\begin{proof}
  \marginpar{Това е важен пример. По-късно ще видим, че този език е безконтекстен}
  % Да допуснем, че $L$ е регулярен.
  % Ще достигнем до противоречие като докажем отрицанието на условието на \Lem{pumping-reg},
  Ще докажем, че
  {\scriptsize
    \[(\forall p \geq 1)(\exists \alpha \in L)[\abs{\alpha} \geq p\ \wedge\ (\forall x,y,z\in\Sigma^\star)[(\alpha = xyz \wedge \abs{y} \geq 1\wedge \abs{xy} \leq p) \Rightarrow (\exists i\in\Nat)[xy^iz \not\in L]].\]}
  Доказателството следва стъпките:
  \begin{itemize}
  \item 
    Разглеждаме произволно число $p \geq 1$ (нямаме власт над избора на $p$).
  \item
    \marginpar{Няма общо правило, което да ни казва как избираме думата $\alpha$. Нормално е пробаваме с няколко думи $\alpha$, докато намерим такава, която върши работа}
    Избираме дума $\alpha \in L$, за която $\abs{\alpha} \geq p$. Имаме свободата да изберем каквато дума $\alpha$
    си харесаме, стига тя да принадлежи на $L$ и да има дължина поне $p$.
    \marginpar{Обърнете внимание, че думата $\alpha$ зависи от константата $p$}
    Щом имаме тази свобода, нека да изберем думата $\alpha = a^pb^p \in L$.
    Очевидно е, че $\abs{\alpha} \geq p$.
  \item
    Разглеждаме произволно разбиване на $\alpha$ на три части, $\alpha = xyz$,
    за които изискваме свойствата $\abs{xy} \leq p$ и $\abs{y} \geq 1$ (не знаем нищо друго за $x$, $y$ и $z$ освен тези две свойства).
  \item
    Ще намерим $i\in\Nat$, за което $xy^iz \not\in L$.
    Понеже $\abs{xy} \leq p$, то $y = a^k$, за  $1\leq k \leq p$.
    Тогава ако вземем $i = 0$, получаваме $xy^0z = a^{p-k}b^p$.
    Ясно е, че $xz \not\in L$, защото $p-k < p$.
  \end{itemize}  
  % Доказахме, че ако $L$ е регулярен език, то свойствата от \Lem{pumping-reg} не са изпълнени. Следователно, езикът $L$
  % не е регулярен.  
  Тогава от \Cor{pumping-reg} следва, че $L$ не е регулярен език.
\end{proof}

\begin{example}
  Езикът $L = \{a^mb^n \mid m,n\in \Nat\ \&\ m < n\}$ {\bf не} е регулярен.
\end{example}
\begin{proof}
  % Да допуснем, че $L$ е регулярен.
  % Следваме същата процедура както в предишния пример.
  Доказателството следва стъпките:
  \begin{itemize}
  \item 
    Разглеждаме произволно число $p \geq 1$.
  \item
    Избираме дума $\alpha \in L$, за която $\abs{\alpha} \geq p$. Имаме свободата да изберем каквато дума $\alpha$
    си харесаме, стига тя да принадлежи на $L$ и да има дължина поне $p$.
    Щом имаме тази свобода, нека да изберем думата $\alpha = a^{p}b^{p+1} \in L$. Очевидно е, че $\abs{\alpha} \geq p$.
  \item
    Разглеждаме произволно разбиване на $\alpha$ на три части, $\alpha = xyz$,
    за които изискваме свойствата $\abs{xy} \leq p$ и $\abs{y} \geq 1$ (не знаем нищо друго за $x$, $y$ и $z$ освен тези две свойства).
  \item
    Ще намерим $i\in\Nat$, за което $xy^iz \not\in L$.
    Понеже $\abs{xy} \leq p$, то $y = a^k$, за  $1\leq k \leq p$.
    Тогава ако вземем $i = 2$, получаваме 
    \[xy^2z = a^{p-k}a^{2k}b^{p+1} = a^{p+k}b^{p+1}.\]
    Ясно е, че $xy^2z \not\in L$, защото $p+k \geq p+1$.
  \end{itemize}
  Тогава от \Cor{pumping-reg} следва, че $L$ не е регулярен език.
\end{proof}

\begin{example}
  Езикът $L = \{a^n\ \mid\ n\mbox{ е просто число}\}$ не е регулярен.
\end{example}
\begin{proof}
  % Да допуснем, че $L$ е регулярен език. Ще достигнем до противоречие като докажем отрицанието на условието на \Lem{pumping-reg},
  % т.е. ще докажем, че
  % % {\scriptsize
  % \begin{align*}
  %   (\forall p \geq 1)(\exists \alpha \in L)[\abs{\alpha} \geq p\ \wedge\ (\forall x,y,z\in\Sigma^\star)[ & (\alpha = xyz \wedge \abs{y} \geq 1\wedge \abs{xy} \leq p) \Rightarrow\\
  %   & (\exists i\in\Nat)[xy^iz \not\in L]].
  % \end{align*}
  % }
  Доказателството следва стъпките:
  \begin{itemize}
  \item 
    Разглеждаме произволно число $p \geq 1$.
  \item
    Избираме дума $w \in L$, за която $\abs{w} \geq p$. Можем да изберем каквото $w$ 
    си харесаме, стига то да принадлежи на $L$ и да има дължина поне $p$.
    Нека да изберем думата $w \in L$, такава че $\abs{w} > p+1$.
    Знаем, че такава дума съществува, защото $L$ е безкраен език. По-долу ще видим защо този избор е важен за нашите разсъждения.
  \item
    Разглеждаме произволно разбиване на $w$ на три части, $w = xyz$,
    за които изискваме свойствата $\abs{xy} \leq p$ и $\abs{y} \geq 1$.
  \item
    Ще намерим $i$, за което $xy^iz \not\in L$,
    т.е. ще намерим $i$, за което 
    $\abs{xy^iz} = \abs{xz} + i\cdot\abs{y}$ е {\em съставно число}.
    Понеже $\abs{xy} \leq p$ и $\abs{xyz} > p+1$, то $\abs{z} > 1$.
    Да изберем $i = \abs{xz} > 1$. Тогава:
    \[\abs{xy^iz} = \abs{xz} + i.\abs{y} = \abs{xz} + \abs{xz}.\abs{y} = (1 + \abs{y})\abs{xz}\] е съставно число, следователно 
    $xy^iz \not\in L$.
  \end{itemize}
  Тогава от \Cor{pumping-reg} следва, че $L$ не е регулярен език.
\end{proof}

\begin{problem}
  Докажете, че езикът $L = \{a^{n^2}\ \mid\ n\in\Nat\}$ не е регулярен.  
\end{problem}
\begin{proof}
  % Да допуснем, че $L$ е регулярен. Отново ще докажем отрицанието на свойството за покачване от \Lem{pumping-reg}.
  Доказателството следва стъпките:
  \begin{itemize}
  \item 
    Разглеждаме произволно число $p \geq 1$.
  \item
    Избираме достатъчно дълга дума, която принадлежи на езика $L$.
    Например, нека $w = a^{p^2}$.
  \item
    Разглеждаме произволно разбиване на $w$ на три части, $w = xyz$, 
    като $\abs{xy} \leq p$ и $\abs{y} \geq 1$.
  \item
    Ще намерим $i$, за което $xy^iz \not\in L$.
    В нашия случай това означава, че $\abs{xz} + i\cdot\abs{y}$ не е точен квадрат.
    Тогава за $i = 2$,
    \[p^2 = \abs{xyz} < \abs{xy^2z} = \abs{xz} + 2\abs{y} \leq p^2 + 2p < (p+1)^2 .\]
    Получаваме, че $p^2 < \abs{xy^2z} < (p+1)^2$,
    откъдето следва, че $\abs{xy^2z}$ не е точен квадрат.
    Следователно, $xy^2z \not\in L$.
  \end{itemize}
  Тогава от \Cor{pumping-reg} следва, че $L$ не е регулярен език.  
\end{proof}

\begin{problem}
  Докажете, че езикът $L = \{a^{n!}\ \mid\ n\in\Nat\}$ не е регулярен.  
\end{problem}
\begin{proof}
  Доказателството следва стъпките:
  \begin{itemize}
  \item 
    Разглеждаме произволно число $p \geq 1$.
  \item
    Избираме достатъчно дълга дума, която принадлежи на езика $L$. Например, нека $\omega = a^{(p+2)!}$.
  \item
    Разглеждаме произволно разбиване на $\omega$ на три части, $\omega = xyz$, 
    като $\abs{xy} \leq p$ и $\abs{y} \geq 1$.
    Да обърнем внимание, че $1 \leq \abs{y} \leq p$
  \item
    Ще намерим $i$, за което $xy^iz \not\in L$.
    В нашия случай това означава, че $\abs{xz} + i\cdot\abs{y}$ не е от вида $n!$.
    Възможно ли е $xy^0z \in L$?
    Понеже $\abs{xyz} = (p+2)!$, това означава, че $\abs{xz} = k!$, за някое $k \leq p+1$.
    Тогава 
    \[\abs{y} = \abs{xyz} - \abs{xz} = (p+2)! - k! \geq (p+2)! - (p+1)! = (p+1).(p+1)! > p.\]
    Достигнахме до противоречие.
  \end{itemize}
  Тогава от \Cor{pumping-reg} следва, че $L$ не е регулярен език.  
\end{proof}

\subsection*{Следствия от лемата за покачването}

\begin{prop}
  Нека е даден автомата $\A = \FA$.
  Езикът $\L(\A)$ е непразен е {\bf непразен} точно тогава, когато съдържа дума $\alpha, \abs{\alpha} < \abs{Q}$.
\end{prop}
\begin{proof}
  Ще разгледаме двете посоки на твърдението.
  \begin{description}
  \item[$(\Rightarrow)$]
    Нека $L$ е непразен език и нека $m = \min\{\abs{\alpha} \mid \alpha \in L\}$.
    Ще докажем, че $m < \abs{Q}$.    
    За целта, да допуснем, че $m \geq \abs{Q}$ и да изберем $\alpha \in L$, за която $\abs{\alpha} = m$.
    Според \Lem{pumping-reg}, съществува разбиване $xyz = \alpha$, 
    такова че $xz \in L$.
    При положение, че $\abs{y} \geq 1$, то $\abs{xz} < m$, което 
    е противоречие с минималността на $m$.
    Заключаваме, че нашето допускане е грешно. Тогава $m < \abs{Q}$, откъдето следва, че 
    съществува дума $\alpha \in L$ с $\abs{\alpha} < \abs{Q}$.
  \item[$(\Leftarrow)$]
    Тази посока е тривиална.
    Ако $L$ съдържа дума $\alpha$, за която $\abs{\alpha} < \abs{Q}$,
    то е очевидно, че $L$ е непразен език.
  \end{description}
\end{proof}

\begin{cor}
  \marginpar{$(L_1\setminus L_2) \cup (L_2 \setminus L_1) = \emptyset$?}
  Съществува алгоритъм, който определя дали два автомата $\A_1$ и $\A_2$ разпознават един и същ език.
\end{cor}

\begin{prop}
  Регулярният език $L$, 
  разпознаван от КДА $\A$, е {\bf безкраен} точно тогава, когато съдържа дума $\alpha, \abs{Q} \leq \abs{\alpha} < 2\abs{Q}$.
\end{prop}
\begin{proof}
  Да разгледаме двете посоки на твърдението.
  \begin{description}
  \item[$(\Leftarrow)$]
    Нека $L$ е регулярен език, за който съществува дума $\alpha$, такава че $\abs{Q} \leq \abs{\alpha} < 2\abs{Q}$.
    Тогава от \Lem{pumping-reg} следва, че съществува разбиване $\alpha = xyz$ със свойството, че
    за всяко $i \in \Nat$, $xy^iz \in L$. Следователно, $L$ е безкраен, защото $\abs{y} \geq 1$.
  \item[$(\Rightarrow)$]
    Нека $L$ е безкраен език и % да приемем, че няма думи $\alpha$ със
    % свойството $\abs{Q} \leq \abs{\alpha} <  2\abs{Q}$.
    да вземем {\em най-късата} дума $\alpha \in L$, за която $\abs{\alpha} \geq 2\abs{Q}$.
    Понеже $L$ е безкраен, знаем, че такава дума съществува.
    Тогава отново по \Lem{pumping-reg}, имаме следното разбиване на $\alpha$:
    \[\alpha = xyz,\ \abs{xy} \leq \abs{Q},\ 1\leq \abs{y},\ xz \in L.\]
    Но понеже $\abs{xyz} \geq 2\abs{Q}$, а $1 \leq \abs{y} \leq \abs{Q}$, то $\abs{xyz} > \abs{xz} \geq \abs{Q}$ и понеже избрахме $\alpha = xyz$
    да бъде най-късата дума с дължина поне $2\abs{Q}$, заключаваме, че $\abs{Q} \leq \abs{xz} < 2\abs{Q}$ и $xz \in L$.
  \end{description}
\end{proof}

\begin{cor}
  Съществува алгоритъм, който проверява дали даден регулярен език е безкраен.
\end{cor}


\subsection*{Примери, за които лемата не е  приложима}

\begin{problem}
  \marginpar{Например, $c^+\{a^nb^n\mid n\in\Nat\}\cup (a\vert b)^\star$}
  Да се даде пример за език $L$, който {\bf не} е регулярен, но удовлетворява
  условието на \Lem{pumping-reg}.
\end{problem}

\begin{example}
  Езикът $L = \{c^ka^nb^m\mid k,n,m \in \Nat\ \&\ k = 1\implies m = n\}$
  {\bf не} е регулярен, но условието за покачване от \Lem{pumping-reg} е изпълнено за него.
\end{example}
\begin{proof}
  Да допуснем, че $L$ е регулярен.
  Тогава ще следва, че 
  \[L_1 = L\cap ca^\star b^\star = \{ca^nb^n \mid n\in\Nat\}\]
  е регулярен,
  но с лемата за разрастването лесно се вижда, че $L_1$ не е.

  Сега да проверим, че условието за покачване от \Lem{pumping-reg} е изпълнено за $L$.
  Да изберем константа $p = 2$.
  Сега трябва да разгледаме всички думи $\alpha \in L$, $\abs{\alpha} \geq 2$
  и за всяка $\alpha$ да посочим разбиване $xyz = \alpha$, за което са изпълнени трите свойства от лемата.
  \marginpar{Условията за $x,y,z$ са:
    \begin{align*}
      & \abs{xy} \leq 2\\
      & \abs{y} \geq 1\\
      & (\forall i\in\Nat)(xy^iz \in L)
    \end{align*}}

  \begin{itemize}
  \item
    Ако $\alpha = a^n$ или $\alpha = b^n$, $n\geq 2$, то е  очевидно, че можем да
    намерим такова разбиване.
  \item
    $\alpha = a^nb^m$ и $n+m \geq 2$, $n \geq 1$.
    Избираме $x = \varepsilon$, $y = a$, $z = a^{n-1}b^m$.
  \item
    $\alpha = ca^nb^n$, $n\geq 1$.
    Избираме $x = \varepsilon$, $y = c$, $z = a^nb^n$.
  \item
    $\alpha = c^2a^nb^m$. 
    Избираме $x = \varepsilon$, $y = c^2$, $z = a^nb^m$.
  \item
    $\alpha = c^ka^nb^m$, $k \geq 3$.
    Избираме $x = \varepsilon$, $y = c$, $z = c^{k-1}a^nb^m$.
  \end{itemize}
\end{proof}

\section{Минимизация на ДКА}

%\marginpar{\href{http://en.wikipedia.org/wiki/DFA_minimization}{Уикипедия}}

\begin{itemize}
\item
  \index{Майхил-Нероуд!релация}
  \marginpar{$\approx_L$ е известна като релация на Майхил-Нероуд}
  Нека $L \subseteq \Sigma^\star$ е език и нека $x,y \in \Sigma^\star$.
  Казваме, че $x$ и $y$ са {\bf еквивалентни относно} $L$, което записваме 
  като $x \approx_L y$, ако е изпълнено:
  \[(\forall z \in \Sigma^\star)[xz \in L \iff yz \in L].\]
  Това означава, че $x\approx_L y$ ако или и две думи са в $L$ или и двете не са в $L$
  и освен това, като прибавим произволна дума на края на $x$ и $y$, новополучените
  думи са или и двете в $L$ или и двете не са в $L$.  
\item
  \marginpar{Трябва ли $\A$ да е тотален?}
  Нека $\A = \FA$ е ДКА.
  Казваме, че две думи $\alpha,\beta \in \Sigma^\star$ са {\bf еквивалентни относно $\A$},
  което означаваме с $\alpha \sim_\A \beta$, ако 
  \[\delta^\star(s,\alpha) = \delta^\star(s,\beta).\]
\item
  Проверете, че $\approx_L$ и $\sim_\A$ са {\bf релации на еквивалентност}, т.е.
  те са рефлексивни, транзитивни и симетрични.
\item
  Класът на еквивалентност на думата $\alpha$ относно релацията $\approx_L$ означаваме като
  \[[\alpha]_L = \{\beta \in \Sigma^\star \mid \alpha \approx_L \beta\}.\]
  С $\abs{\approx_L}$ ще означаваме броя на класовете на еквивалентност на релацията $\approx_L$.
\item
  Класът на еквивалентност на думата $\alpha$ относно релацията $\sim_\A$ означаваме като
  \[[\alpha]_\A = \{\beta \in \Sigma^\star \mid \alpha \sim_\A \beta\}.\]
  С $\abs{\sim_\A}$ ще означаваме броя на класовете на еквивалентност на релацията $\sim_\A$.
\item
  Съобразете, че всяко състояние на $\A$, което е достижимо от началното състояние, определя клас на еквивалентност относно 
  релацията $\sim_\A$. Това означава, че ако за всяка дума означим  $q_\alpha = \delta^\star_\A(s,\alpha)$, то
  $\alpha \sim_\A \beta$ точно тогава, когато $q_\alpha = q_\beta$. Заключаваме, че броят на класовете на еквивалентност
  на $\sim_\A$ е равен на броя на достижимите от $s$ състояния.
\item
  Релациите $\approx_\L$ и $\sim_\A$ са дясно-инвариантни, т.е. за всеки две думи $\alpha$ и $\beta$
  е изпълнено:
  \begin{align*}
    \alpha \sim_\A \beta  &\implies (\forall \gamma\in\Sigma^\star)[\alpha\gamma \sim_\A \beta\gamma],\\
    \alpha \approx_\L \beta & \implies (\forall \gamma\in\Sigma^\star)[\alpha\gamma \approx_\L \beta\gamma].
  \end{align*}
\end{itemize}

\begin{thm}
  \label{th:rel-finer}
  За всеки ДКА $\A = \FA$ е изпълнено:
  \[(\forall \alpha,\beta \in \Sigma^\star)[\alpha\sim_\A\beta \implies \alpha\approx_{\L(\A)}\beta].\]
  С други думи, 
  $[\alpha]_\A \subseteq [\alpha]_{\L(\A)}$, за всяка дума $\alpha \in \Sigma^\star$.
\end{thm}
\begin{proof}
%  \marginpar{стр. 95 от \cite{papadimitriou}}
  Да означим за всяка дума $\alpha$, $q_\alpha = \delta^\star_\A(s, \alpha)$.
  Лесно се съобразява, че за всеки две думи $\alpha$ и $\beta$ имаме 
  \begin{align*}
    \alpha \sim_\A \beta & \iff \delta^\star(s,\alpha) = \delta^\star(s,\beta) & (\text{по деф. на }\sim_\A)\\
    & \iff q_\alpha = q_\beta.
  \end{align*}
  Нека $\alpha \sim_\A \beta$. Ще проверим, че  $\alpha \approx_{\L(\A)} \beta$.
  За произволно $\gamma \in \Sigma^\star$ имаме:
  \begin{align*}
    \alpha\gamma \in \L(\A) & \iff \delta^\star(s,\alpha\gamma)\in F & (\text{по деф. на }\L(\A))\\
    & \iff \delta^\star(\delta^\star(s,\alpha),\gamma) \in F & (\text{по деф. на }\delta^\star)\\
    & \iff \delta^\star(q_\alpha, \gamma) \in F & (q_\alpha = \delta^\star(s,\alpha))\\
    & \iff \delta^\star(q_\beta, \gamma) \in F & (q_\alpha = q_\beta, \text{ защото }\alpha \sim_\A \beta)\\
    & \iff \delta^\star(\delta^\star(s,\beta),\gamma) \in F & (q_\beta = \delta^\star(s,\beta))\\
    & \iff \delta^\star(s,\beta\gamma) \in F & (\text{по деф. на }\delta^\star)\\
    & \iff \beta\gamma \in \L(\A) & (\text{по деф. на }\L(\A)).
  \end{align*}
  Заключаваме, че 
  \[(\forall \alpha,\beta \in \Sigma^\star)[\alpha\sim_\A\beta \implies \alpha\approx_{\L(\A)}\beta].\]
\end{proof}

\begin{cor}
  \label{cor:approx-less-sim}
  За всеки тотален ДКА $\A$ е изпълнено, че
  \[\abs{\approx_{\L(\A)}} \leq \abs{\sim_\A}.\]
\end{cor}
\begin{proof}
  Нека $A = \{[\alpha]_{\L(\A)} \mid \alpha\in\Sigma^\star\}$ и $B = \{[\alpha]_\A \mid \alpha\in\Sigma^\star\}$.
  Да разгледаме изображението $f:B\to A$, определено като $f([\alpha]_\A) = [\alpha]_{\L(\A)}$.
  \begin{itemize}
  \item 
    Първо ще проверим, че $f$ е {\bf функция}, т.е. трябва да проверим, че 
    \[(\forall\alpha,\beta\in\Sigma^\star)[\alpha \sim_\A \beta \implies f([\alpha]_\A) = f([\beta]_\A)].\]
    
    Да допуснем, че съществуват думи $\alpha$ и $\beta$, такива че
    $[\alpha]_{\A} = [\beta]_{\A}$, но $f([\alpha]_{\A}) = [\alpha]_{\L(\A)} \neq [\beta]_{\L(\A)} = f([\beta]_{\A})$.
    Понеже $\sim_\A$ релация на еквивалентност, от $[\alpha]_{\L(\A)} \neq [\beta]_{\L(\A)}$
    следва, че $[\alpha]_{\L(\A)} \cap [\beta]_{\L(\A)} = \emptyset$.
    От \Th{rel-finer} следва веднага, че това е невъзможно, защото
    \[\emptyset \neq [\alpha]_\A = [\beta]_\A \subseteq [\alpha]_{\L(\A)} \cap [\beta]_{\L(\A)}.\]
  \item
    \marginpar{$(\forall a\in A)(\exists b\in B)(f(b) = a)$}
    Очевидно е, че $f$ е {\bf сюрекция}, защото на всеки клас $[\alpha]_{\L(\A)}$ съответства класа $[\alpha]_\A$.
  \item
    \marginpar{Защо?\\ \ding{45} Обяснете!}
    От това, че $f:B\to A$ е сюрективна функция следва, че $\abs{B} \leq \abs{A}$.
  \end{itemize}
\end{proof}

\begin{cor}
  \label{cor:upper-bound}
  Нека $L$ е произволен регулярен език $L$.  
  Всеки тотален ДКА $\A$, който разпознава $L$ има свойството
  \[\abs{Q} \geq \abs{\approx_L}.\]
\end{cor}
\begin{proof}
  Да изберем $\A$, който разпознава $L$, бъде такъв, че да {\bf няма недостижими състояния}.
  Тъй като всяко достижимо състояние определя клас на еквивалентност относно $\sim_\A$,
  то получаваме, че $\abs{Q} = \abs{\sim_\A}$.
  Комбинирайки със \Cor{approx-less-sim},
  \[\abs{Q} = \abs{\sim_\A} \geq \abs{\approx_L}.\]
\end{proof}
Така получаваме {\em долна граница} за броя на състоянията в тотален автомат разпознаващ езика $L$.
Този брой е не по-малък от броя на класовете на еквивалентност на $\approx_L$.

\subsection*{Проверка за регулярност на език}

\begin{framed}
  \begin{prop}
    Езикът $L$ е регулярен точно тогава, когато релацията $\approx_L$ има {\em крайно много} класове на еквивалентност.
  \end{prop}
\end{framed}
\begin{proof}
  Ако $L$ е регулярен, то той се разпознава от някой ДКА $\A$, който има крайно много състояния 
  и следователно крайно много класове на еквивалентност относно $\sim_\A$.
  Релацията $\approx_L$ е по-груба от $\sim_\A$ и има по-малко класове на еквивалентност.
  Следователно, $\approx_L$ има крайно много класове на еквивалентност.
  
  За другата посока, ако $\approx_L$ има крайно много класове на еквивалентност, то можем да 
  построим ДКА $\A$ както в доказателството на \Th{myhill-nerode}, който разпознава $L$.
\end{proof}

Това следствие ни дава още един начин за проверка дали даден език е регулярен.
За разлика от \Lem{pumping-reg}, сега имаме {\bf необходимо и достатъчно условие}.
При даден език $L$, ние разглеждаме неговата релация $\approx_L$.
Ако тя има крайно много класове, то езикът $L$ е регулярен.
В противен случай, езикът $L$ не е регулярен.

\begin{example}
  За езика $L = \{a^nb^n\mid n \in \Nat\}$ имаме, че $\abs{\approx_L} = \infty$,
  защото \[(\forall k,j\in\Nat)[k \neq j \implies [a^kb]_L \neq [a^jb]_L].\]
  Проверете, че $[a^kb]_L = \{a^kb,a^{k+1}b^{2},\dots,a^{k+l}b^{l+1},\dots\}$.
  Така получаваме, че релацията $\approx_L$ има безкрайно много класове на еквивалентност.
  Заключаваме, че този език {\bf не} е регулярен.
\end{example}

\begin{example}
  За езика $L = \{a^{n^2} \mid n \in \Nat\}$ имаме, че $\abs{\approx_L} = \infty$,
  защото \[(\forall m,n\in\Nat)[m \neq n \implies [a^{n^2}]_L \neq [a^{m^2}]_L].\]
  
  Без ограничение на общността, да разгледаме $n < m$ и думата $\gamma = a^{2n+1}$.
  Тогава $a^{n^2}\gamma = a^{(n+1)^2} \in L$, но 
  $m^2 < m^2 + 2n + 1 < (m+1)^2$ и следователно $a^{m^2}\gamma = a^{m^2+2n+1}\not\in L$.
\end{example}

\begin{example}
  За езика $L = \{a^{n!} \mid n \in \Nat\}$ имаме, че $\abs{\approx_L} = \infty$,
  защото \[(\forall m,n\in\Nat)[m \neq n \implies [a^{n!}]_L \neq [a^{m!}]_L].\]
  
  Без ограничение на общността, да разгледаме $n < m$ и думата $\gamma = a^{(n!)n}$.
  Тогава $a^{n!}\gamma = a^{(n+1)!} \in L$, но 
  $m! < m! + (n!)n < m! + (m!)m = (m+1)!$ и следователно $a^{m!}\gamma = a^{m!+(n!)n}\not\in L$.
\end{example}

\begin{problem}
  Да разгледаме езика
  \[L = \{a^{f_n} \mid f_0 = f_1 = 1\ \&\ f_{n+2} = f_{n+1} + f_{n}\}.\]
  Докажете, че $\abs{\approx_L} = \infty$.
\end{problem}


\subsection*{Теорема за съществуване на МДКА}

\index{минимален автомат}
\begin{dfn}
  Нека $\A$ а тотален ДКА, за който $L = \L(\A)$.
  Казваме, че $\A$ е {\bf минимален} за езика $L$, ако $\abs{Q_\A} = \abs{\approx_L}$.
\end{dfn}

% Да приемем, че сме фиксирали азбуката $\Sigma$.
\begin{thm}[Майхил-Нероуд]
  \label{th:myhill-nerode}
  \index{Майхил-Нероуд!теорема}
  % \index{Майхил}
  % \index{Нероуд}
  \marginpar{на англ. Myhill-Nerode}
  Нека $L\subseteq \Sigma^\star$ е регулярен език.
  Тогава съществува ДКА $\A = \FA$, който разпознава $L$,
  с точно толкова състояния, колкото са класовете на еквивалентност на релацията $\approx_L$,
  т.е. $\abs{Q} = \abs{\approx_L}$.
\end{thm}
\begin{proof}
%  \marginpar{стр. 96 от \cite{papadimitriou}}
  Да фиксираме регулярния език $L$.
  Ще дефинираме тотален ДКА $\A = \FA$, разпознаващ $L$, като:
  \begin{itemize}
  \item
    $Q = \{[\alpha]_L\mid \alpha\in \Sigma^\star\}$;
  \item
    $s = [\varepsilon]_L$;
  \item
    $F = \{[\alpha]_L\mid \alpha\in L\} = \{[\alpha]_L \mid [\alpha]_L \cap L \neq \emptyset\}$;
  \item
    Определяме изображението $\delta$ като 
    за всяка буква $x \in \Sigma$ и всяко състояние $[\alpha]_L\in Q$, 
    \[\delta([\alpha]_L,x) = [\alpha x]_L.\]
  \end{itemize}
  
  Първо, трябва да се уверим, че множеството от състояния $Q$ е крайно, т.е.
  релацията $\approx_\L$ има крайно много класове на еквивалентност.
  И така, тъй като $\L$ е регулярен език, то той се разпознава от някой тотален ДКА $\A'$.
  От \Cor{upper-bound} имаме, че $\abs{Q^{\A'}} \geq \abs{\approx_L}$.
  Понеже $Q^{\A'}$ е крайно множество, то $\approx_L$ има крайно много класове и 
  следователно $Q$ също е крайно множество.

  Второ, трябва да се уверим, че изображението $\delta$ задава функция, т.е. 
  да проверим, че за всеки две думи $\alpha$, $\beta$ и всяка буква $x$,
  \[[\alpha]_L = [\beta]_L \implies \delta([\alpha]_L,x) = \delta([\beta]_L,x).\]
  Но това се вижда веднага, защото от определението на релацията $\approx_L$ следва, че
  ако $\alpha \approx_L \beta$, то за всяка буква $x$, $\alpha x \approx_L \beta x$,
  т.е. $[\alpha x]_L = [\beta x]_L$ и 
  \begin{align*}
    [\alpha]_L = [\beta]_L & \implies [\alpha x]_L = [\beta x]_L & (\text{свойство на }\approx_L)\\
    & \implies \delta([\alpha]_L,x) = [\alpha x]_L = [\beta x]_L = \delta([\beta]_L,x) & (\text{деф. на }\delta)
  \end{align*}
  
  Така вече сме показали, че $\A$ е коректно зададен тотален ДКА.
  Остава да покажем, че $\A$ разпознава езика $L$, т.е. $\L(\A) = L$.
  За целта, първо ще докажем едно помощно твърдение.
  \begin{prop}
    За всеки две думи $\alpha$ и $\beta$,
    $\delta^\star([\alpha]_L,\beta) = [\alpha\beta]_L$.
  \end{prop}
  \begin{proof}
    Ще докажем това свойство с индукция по дължината на $\beta$.
    \begin{itemize}
    \item
      За $\beta = \varepsilon$ свойството следва директно от дефиницията на $\delta^\star$ като рефлексивно и транзитивно затваряне на $\delta$,
      защото $\delta^\star([\alpha]_L,\varepsilon) = [\alpha]_L$.
    \item
      Нека $\abs{\beta} = n+1$ и да приемем, че сме доказали твърдението за думи с дължина $n$.
      Тогава $\beta = \gamma a$, където $\abs{\gamma} = n$. Свойството следва от следните равенства:
      \begin{align*}
        \delta^\star([\alpha]_L, \gamma a) & = \delta(\delta^\star([\alpha]_L,\gamma),a) & (\text{деф. на }\delta^\star)\\
                                          & = \delta([\alpha\gamma]_L,a) & (\text{от {\bf И.П.} за }\gamma)\\
                                          & = [\alpha\gamma a]_L & (\text{от деф. на }\delta)\\
                                          & = [\alpha\beta]_L & (\beta = \gamma a).
      \end{align*}
    \end{itemize}
  \end{proof}
  \noindent За да се убедим, че $L = \L(\A)$ е достатъчно да проследим еквивалентностите:
  \begin{align*}
    \alpha\in \L(\A) & \iff \delta^\star(s,\alpha) \in F & (\text{от деф. на }\L(\A))\\
                     & \iff \delta^\star([\varepsilon]_L,\alpha) \in F & (\text{по деф. }s = [\varepsilon]_L)\\
                     & \iff \delta^\star([\varepsilon]_L,\alpha) = [\alpha]_L\ \&\ \alpha\in L & (\text{от деф. на }F)\\
                     & \iff \alpha \in L & (\text{от последното твърдение}).
  \end{align*}
\end{proof}

\begin{dfn}
  \index{изоморфизъм}
  Нека $\A_1 = \FAn{1}$ и $\A_2 = \FAn{2}$.
  Казваме, че $\A_1$ и $\A_2$ са {\bf изоморфни}, което означаваме с $\A_1 \cong \A_2$, ако
  съществува биекция $f: Q_1\to Q_2$, за която:
  \begin{itemize}
  \item
    $f(s_1) = s_2$;
  \item
    $f[F_1] = \{f(q)\mid q\in F_1\} = F_2$;
  \item
    $(\forall a\in\Sigma)(\forall q\in Q_1)[f(\delta_1(q,a)) = \delta_2(f(q),a)]$.
  \end{itemize}
  Ще казваме, че $f$ задава изоморфизъм на $\A_1$ върху $\A_2$.
\end{dfn}

Това означава, че два автомата $\A_1$ и $\A_2$ са изоморфни, ако можем да получим $\A_2$
като преименуваме състоянията на $\A_1$.

\begin{cor}
  Нека е даден регулярния език $L$.
  Всички минимални автомати за $L$ са изоморфни на $\A_0$, автомата построен в теоремата на Нерод-Майхил.
\end{cor}
\begin{proof}
  Нека $\A = \FA$ е произволен тотален автомат, за който $\L(\A) = L$ и $\abs{Q} = \abs{\approx_L}$.
  Съобразете, че $\A$ е {\em свързан}, т.е. всяко състояние на $\A$ е достижимо от началното.
  Искаме да докажем, че $\A \cong \A_0$.
  Понеже $\A$ е свързан, за всяко състояние $q$ можем да намерим дума $\omega_q$,
  за която $\delta^\star(s,\omega_q) = q$.
  Да дефинираме изображението $f:Q\to [\approx_L]$ като $f(q) = [\omega_q]_L$.
  Ще докажем, че
  $f$ задава изоморфизъм на $\A$ върху $\A_0$. 
  \begin{itemize}
  \item
    Първо да съобразим, че ако $\delta^\star_\A(s,\alpha) = q$, то $[\omega_q]_L = [\alpha]_L$.
    Понеже $\delta^\star_\A(s,\alpha) = q = \delta^\star_\A(s,\omega_q)$, то $\omega_q \sim_\A \alpha$.
    От \Th{rel-finer} имаме, че
    \[\omega_q \sim_\A \alpha \implies \omega_q \approx_L \alpha.\]
    Това означава, $[\omega_q]_L = [\alpha]_L$ и следователно $f$ е определена коректно, т.е. $f$ е {\bf функция}.
  \item
    Ще проверим, че $f$ е {\bf инективна}, т.е.
    \[(\forall q_1,q_2 \in Q)[q_1\neq q_2 \implies f(q_1) \neq f(q_2)].\]
    Да допуснем, че има състояния $q_1 \neq q_2$, за които 
    \[f(q_1) = [\omega_{q_1}]_L = [\omega_{q_2}]_L = f(q_2).\]
    Тогава $\omega_{q_1} \not\sim_\A \omega_{q_2}$ и $\omega_{q_1} \approx_L \omega_{q_2}$.
    \marginpar{\writedown Обяснете!}
    Но тогава от \Cor{upper-bound} получаваме, че $\abs{\sim_\A} > \abs{\approx_L}$,
    което противоречи с минималността на $\A$.
  \item
    За да бъде $f$ {\bf сюрективна} трябва за всеки клас $[\beta]_L$ да съществува състояние $q$, за което $f(q) = [\beta]_L$.
    Понеже $\A$ е свързан, съществува състояние $q$, за което $\delta^\star_\A(s,\beta) = q$.
    Вече се убедихме, че в този случай $\beta \approx_L \omega_q$, защото $\beta \sim_\A \omega_q$.
    Тогава $f(q) = [\omega_q]_L = [\beta]_L$.
  \item
    За последно оставихме проверката, че $f$ наистина е {\bf изоморфизъм}:
    \begin{align*}
      f(\delta_\A(q,a)) & = f(\delta_\A(\delta^\star_\A(s,\omega_q),a)) & (\text{от избора на }\omega_q)\\
      & = f(\delta^\star_\A(s,\omega_qa)) & (\text{от деф. на }\delta^\star_\A)\\
      & = [\omega_qa]_L & (\text{от деф. на }f)\\
      & = \delta^\star_{\A_0}([\varepsilon]_L, \omega_qa) & (\text{от деф. на }\A_0)\\ 
      & = \delta_{\A_0}(\delta^\star_{\A_0}([\varepsilon]_L, \omega_q),a) & (\text{от деф. на }\delta^\star_{\A_0})\\
      & = \delta_{\A_0}([\omega_q]_L, a) & (\text{свойство на }\delta^\star_{\A_0})\\
      & = \delta_{\A_0}(f(q), a) & ( f(q) = [\omega_q]_L).
    \end{align*}
  \end{itemize}
\end{proof}

\subsection*{Алгоритъм за намиране на МДКА.}
\begin{itemize}
\item
  Да фиксираме произволен тотален ДКА $\A = \FA$.
\item
  Казваме, че две състояния $p,q$ на автомата  са {\bf еквивалентни}, означаваме $p\equiv_\A q$,
  ако \[p \equiv_\A q\ \iff\ (\forall \gamma\in \Sigma^\star)[\delta^\star(p,\gamma) \in F\ \iff\ \delta^\star(q,\gamma) \in F].\]
\item
  Релацията $\equiv_\A$ между състояния на автомата $\A$ е релация на еквивалентност. 
\item
  Нека $q_\alpha$ е състоянието, което съответства на думата $\alpha$ в $\A$, т.е.
  $\delta^\star_\A(s,\alpha) = q_\alpha$. Тогава:
  \[q_\alpha \equiv_\A q_\beta\ \iff\ \alpha\approx_{\L(\A)} \beta.\]
  Това означава, че ако в $\A$ няма недостижими състояния от началното състояние $s$, то $\abs{\equiv_\A} = \abs{\approx_{\L(\A)}}$.
\end{itemize}

При даден език $L$ и тотален ДКА $\A = \FA$, който го разпознава, нашата цел е да построим нов ДКА $\A_0$,
който има толкова състояния колкото са класовете на еквивалентност на релацията $\approx_\L$.
Това ще направим като ``слеем'' състоянията на $\A$, които са еквивалентни относно релацията $\equiv_\A$.
Това означава, че всяко състояние на $\A_0$ ще отговаря на един клас на еквивалентност на релацията $\equiv_\A$.
Проблемът с намирането на класовете на еквивалентност на релацията $\equiv_\A$ е кванторът $\forall \gamma \in \Sigma^\star$
в нейната дефиницията.

Алгоритъмът представлява намирането на релации $\equiv_n$, където
\[p\equiv_n q \iff (\forall\gamma\in\Sigma^\star)[\abs{\gamma}\leq n\ \rightarrow\ (\delta^\star(p,\gamma) \in F\ \iff\ \delta^\star(q,\gamma) \in F)].\]
Релациите $\equiv_n$ представляват апроксимации на релацията $\equiv_\A$.
Обърнете внимание, че за всяко $n$, $\equiv_n$ е {\em по-груба} релация от $\equiv_{n+1}$, 
която на свой ред е по-груба от $\equiv_\A$.
Алгоритъмът строи $\equiv_n$ докато не срещнем $n$, за което $\equiv_n\ =\ \equiv_{n+1}$.
Тъй като броят на класовете на еквивалентност на $\equiv_\A$ е краен ($\leq \abs{Q}$), то 
със сигурност ще намерим такова $n$, за което $\equiv_n\ =\ \equiv_{n+1}$.
Тогава заключаваме, че $\equiv_\A\ =\ \equiv_n$.

Понеже единствената дума с дължина $0$ e $\varepsilon$ и по определение $\delta^\star(p,\varepsilon) = p$, 
лесно се съобразява, че $\equiv_0$ има два класа на еквивалентност.
Единият е $F$, а другият е $Q\setminus F$.

\begin{prop}
  За всеки две състояния $p,q \in Q$, и всяко $n$, $p \equiv_{n+1} q$ точно тогава, когато
  \begin{enumerate}[a)]
  \item
    $p \equiv_{n} q$ и
  \item
    $(\forall a \in \Sigma)[\delta(q,a) \equiv_{n} \delta(p,a)]$.
  \end{enumerate}
\end{prop}
\begin{proof}
  \marginpar{(стр. 99 от \cite{papadimitriou})}
  \begin{align*}
    p \equiv_{n+1} q \iff & (\forall \gamma\in\Sigma^{\leq n+1})[\delta^\star(p,\gamma)\in F \iff \delta^\star(q,\gamma) \in F]\\
    \iff & (\forall \gamma\in\Sigma^{\leq n})[\delta^\star(p,\gamma)\in F \iff \delta^\star(q,\gamma) \in F]\ \wedge\ \\
    & (\forall a\in\Sigma)(\forall \gamma\in\Sigma^{\leq n})[\delta^\star(p, a\gamma)\in F \iff \delta^\star(q, a\gamma) \in F]\\
    \iff & p \equiv_n q\ \&\ (\forall a\in\Sigma)[\delta(p,a) \equiv_n \delta(q,a)].
  \end{align*}
\end{proof}

Нека е даден автомата $A = \FA$.
След като сме намерили релацията $\equiv_\A$ за $\A$, 
строим автомата $\A' = (Q',\Sigma,s',\delta',F')$, където:
\begin{enumerate}[1)]
\item
  $Q' = \{[q]_{\equiv_\A} \mid q\in Q\}$;
\item
  $s' = [s]_{\equiv_\A}$;
\item
  $\delta'([q]_{\equiv_\A}, a) = [\delta(q,a)]_{\equiv_\A}$;
\item
  $F' = \{[q]_{\equiv_\A}\mid F\cap [q]_{\equiv_\A} \neq \emptyset\}$;
\end{enumerate}

От всичко казано дотук знаем, че $\A'$ е минимален автомат разпознаващ езика $\L(\A)$.

\begin{example}
  Да разгледаме следния краен детерминиран автомат $\A$.
  \begin{figure}[H]
    \begin{subfigure}[b]{.4\textwidth}
      \begin{tikzpicture}[->,>=stealth,thick,node distance=55pt]
        \tikzstyle{every state}=[circle,minimum size=20pt,auto]
        
        \node[initial above, state]   (0) {$0$};
        \node[state]            (1) [above right of=0]{$1$};
        \node[state]            (2) [below right of=0]{$2$};
        \node[state,accepting]  (3) [right of=1]{$3$};
        \node[state,accepting]  (4) [right of=2]{$4$};
        \node[state,accepting]  (5) [below right of=3]{$5$};
        
        
        \path 
        (0) edge  node [above] {$a$}   (1)
        (0) edge  node [below] {$b$}   (2)
        (1) edge node [above] {$a$}    (3)
        (1) edge [bend left=15] node [below] {$b$}    (4)
        (2) edge [bend left=15] node [left] {$b$}    (3)
        (2) edge node [below] {$a$}   (4)
        (4) edge  node [below] {$a,b$} (5)
        (3) edge  node [left] {$a,b$}  (5)
        (5) edge [loop above]   node [above] {$a,b$}  (5);
      \end{tikzpicture}
      \caption{Ще построим минимален автомат разпознаващ $\L(\A)$}
    \end{subfigure}
    \qquad
    \qquad
    \begin{subfigure}[b]{0.5\textwidth}
      \begin{tikzpicture}[->,>=stealth,thick,node distance=45pt]
        \tikzstyle{every state}=[circle,minimum size=20pt,auto,scale=.9]
        
        \node[initial above, state]   (0) {$B_0$};
        \node[state]            (1) [right of=0]{$B_1$};
        \node[state,accepting]  (2) [right of=1]{$B_2$};
        
        \path 
        (0) edge [bend left=15] node [above] {$a,b$}   (1)
        (1) edge [bend left=15] node [above] {$a,b$}   (2)
        (2) edge [loop above] node [above] {$a,b$}   (2);
      \end{tikzpicture}
      \caption{Получаваме следния минимален автомат $\A_0$, $\L(\A_0) = \L(\A)$}
      \label{sub:min1}
    \end{subfigure}
  \end{figure}
  \marginpar{Съобразете, че $\L(\A) = \{\alpha \in \{a,b\}^\star \mid \abs{\alpha} \geq 2\}$.}

Ще приложим алгоритъма за минимизация за да получим минималния автомат за езика $L$.
За всяко $n = 0,1,2,\dots$, ще намерим класовете на еквивалентност на $\equiv_n$,
докато не намерим $n$, за което $\equiv_n\ =\ \equiv_{n+1}$.

\begin{itemize}
\item 
  Класовете на еквивалентност на $\equiv_0$ са два.
  Те са $A_0 = Q\setminus F = \{0,1,2\}$ и $A_1 = F = \{3,4,5\}$.
\item
  Сега да видим дали можем да разбием някои от класовете на еквивалентност на $\equiv_0$.
  
  \begin{tabular}{|c|c|c|c|c|c|c|}
    \hline
    $Q$ & $0$ & $1$ & $2$ & $3^\star$ & $4^\star$ & $5^\star$ \\
    \hline
    \hline
    $\equiv_0$ & $A_0$ & $A_0$ & $A_0$ & $A_1$ & $A_1$ & $A_1$\\
    \hline
    $a$ & $A_0$& $A_1$ & $A_1$ & $A_1$ & $A_1$ & $A_1$\\
    \hline
    $b$ & $A_0$& $A_1$ & $A_1$ & $A_1$ & $A_1$ & $A_1$\\
    \hline
  \end{tabular}

  Виждаме, че $0 \not\equiv_1 1$ и $1 \equiv_1 2$.
  Класовете на еквивалентност на $\equiv_1$ са 
  $B_0 = \{0\}$, $B_1 = \{1,2\}$, $B_2 = \{3,4,5\}$.
\item
  Сега да видим дали можем да разбием някои от класовете на еквивалентност на $\equiv_1$.
  
  \begin{tabular}{|c|c|c|c|c|c|c|}
    \hline
    $Q$ & $0$ & $1$ & $2$ & $3^\star$ & $4^\star$ & $5^\star$ \\
    \hline
    \hline
    $\equiv_1$ & $B_0$ & $B_1$ & $B_1$ & $B_2$ & $B_2$ & $B_2$\\
    \hline
    $a$ & $B_1$ & $B_2$ & $B_2$ & $B_2$ & $B_2$ & $B_2$\\
    \hline
    $b$ & $B_1$ & $B_2$ & $B_2$ & $B_2$ & $B_2$ & $B_2$\\
    \hline
  \end{tabular}

  Виждаме, че $\equiv_1\ =\ \equiv_2$.
  \marginpar{Получаваме, че $\equiv_\A\ =\ \equiv_1$}
  Следователно, минималният автомат има три състояния.
  Той е изобразен на Фигура \ref{sub:min1}.  
  Минималният автомат може да се представи и таблично:
  
  \begin{tabular}{|c|c|c|c|c|c|c|}
    % \hline
    % $Q$ & $0$ & $1$ & $2$ & $3^\star$ & $4^\star$ & $5^\star$ \\
    % \hline
    \hline
    $\delta$ & $B_0$ & $B_1$ & $B_2$ \\
    \hline
    $a$ & $B_1$ & $B_2$ & $B_2$ \\
    \hline
    $b$ & $B_1$ & $B_2$ & $B_2$ \\
    \hline
  \end{tabular}
\end{itemize}
\end{example}

\begin{example}
  Да разгледаме следния краен детерминиран автомат $\A$.
  \begin{figure}[H]
    % \begin{center}
    \begin{subfigure}[b]{0.4\textwidth}
      \begin{tikzpicture}[->,>=stealth,thick,node distance=55pt]
        \tikzstyle{every state}=[circle,minimum size=20pt,auto]
        
        \node[initial above, state]   (0) {$0$};
        \node[state,accepting]        (1) [above right of=0]{$1$};
        \node[state,accepting]        (2) [below right of=0]{$2$};
        \node[state]                  (3) [right of=1]{$3$};
        \node[state]                  (4) [right of=2]{$4$};
        \node[state,accepting]        (5) [below right of=3]{$5$};
        
        \path 
        (0) edge node [below] {$a$}   (1)
            edge node [below] {$b$}   (2)
        (1) edge node [above] {$a$}    (3)
            edge [bend left=15] node [below] {$b$}    (4)
        (2) edge [bend left=15] node [left] {$b$}    (3)
            edge node [below] {$a$}   (4)
        (4) edge node [below] {$a,b$} (5)
        (3) edge node [left] {$a,b$}  (5)
        (5) edge [loop above]   node [above] {$a,b$}  (5);
      \end{tikzpicture}
      \caption{Ще построим минимален автомат разпознаващ $\L(\A)$}
    \end{subfigure}
    \qquad
    \qquad
    \begin{subfigure}[b]{0.4\textwidth}
      \begin{tikzpicture}[->,>=stealth,thick,node distance=45pt]
        \tikzstyle{every state}=[circle,minimum size=20pt,auto,scale=.9]
        
        \node[initial above, state]   (0) {$C_0$};
        \node[state,accepting]  (1) [right of=0]{$C_1$};
        \node[state]            (2) [right of=1]{$C_2$};
        \node[state,accepting]  (3) [right of=2]{$C_3$};
                
        \path 
        (0) edge [bend left=15] node [above] {$a,b$}   (1)
        (1) edge [bend left=15] node [above] {$a,b$}   (2)
        (2) edge [bend left=15] node [above] {$a,b$}   (3)
        (3) edge [loop above]   node [above] {$a,b$}   (3);
      \end{tikzpicture}
      \caption{Получаваме следния минимален автомат $\A_0$, $\L(\A_0) = \L(\A)$}
      \label{sub:min2}
    \end{subfigure}
  \end{figure}

  \marginpar{Съобразете, че $\L(\A) = \{a,b\} \cup \{\alpha \in \{a,b\}^\star \mid \abs{\alpha} \geq 3\}$.}
  
  Отново следваме същата процедура за минимизация.
  Ще намерим класовете на еквивалентност на $\equiv_n$,
  докато не намерим $n$, за което $\equiv_n\ =\ \equiv_{n+1}$.
  \begin{itemize}
  \item
    Класовете на екиваленост на $\equiv_0$ са 
    $A_0 = Q\setminus F = \{0,3,4\}$ и $A_1 = F = \{1,2,5\}$.
  \item
    Разбиваме класовете на еквивалентност на $\equiv_0$.
    
    \begin{tabular}{|c|c|c|c|c|c|c|}
      \hline
      $Q$ & 0 & $1^\star$ & $2^\star$ & 3 & 4 & $5^\star$ \\
      \hline
      \hline
      $\equiv_0$ & $A_0$ & $A_1$ & $A_1$ & $A_0$ & $A_0$ & $A_1$\\
      \hline
      $a$ & $A_1$& $A_0$ & $A_0$ & $A_1$ & $A_1$ & $A_1$\\
      \hline
      $b$ & $A_1$& $A_0$ & $A_0$ & $A_1$ & $A_1$ & $A_1$\\
      \hline
    \end{tabular}
    
    Виждаме, че $1 \not\equiv_1 5$ и $1 \equiv_0 5$.
    Следователно, $\equiv_0\ \neq\ \equiv_1$.
    Класовете на еквивалентност на $\equiv_1$ са 
    $B_0 = \{0,3,4\}$, $B_1 = \{1,2\}$, $B_2 = \{5\}$.
  \item
    Сега се опитваме да разбием класовете на еквивалентност на $\equiv_1$.

    \begin{tabular}{|c|c|c|c|c|c|c|}
      \hline
      $Q$ & 0 & $1^\star$ & $2^\star$ & 3 & 4 & $5^\star$ \\
      \hline
      \hline
      $\equiv_1$ & $B_0$ & $B_1$ & $B_1$ & $B_0$ & $B_0$ & $B_2$\\
      \hline
      $a$ & $B_1$ & $B_0$ & $B_0$ & $B_2$ & $B_2$ & $B_2$\\
      \hline
      $b$ & $B_1$ & $B_0$ & $B_0$ & $B_2$ & $B_2$ & $B_2$\\
      \hline
    \end{tabular}
    
    Имаме, че $0 \equiv_1 3$, но $0 \not\equiv_2 3$. Следователно $\equiv_1\ \neq\ \equiv_2$.
    Класовете на еквивалентност на $\equiv_2$ са 
    $C_0 = \{0\}$, $C_1 = \{1,2\}$, $C_2 = \{3,4\}$, $C_3 = \{5\}$.
  \item
    Отново опитваме да разбием класовете на $\equiv_2$.

      \begin{tabular}{|c|c|c|c|c|c|c|}
        \hline
        $Q$ & 0 & $1^\star$ & $2^\star$ & 3 & 4 & $5^\star$ \\
        \hline
        \hline
        $\equiv_2$ & $C_0$ & $C_1$ & $C_1$ & $C_2$ & $C_2$ & $C_3$\\
        \hline
        $a$ & $C_1$ & $C_2$ & $C_2$ & $C_3$ & $C_3$ & $C_3$\\
        \hline
        $b$ & $C_1$ & $C_2$ & $C_2$ & $C_3$ & $C_3$ & $C_3$\\
        \hline
      \end{tabular}
      
      Виждаме, че не можем да разбием $C_1$ или $C_2$.
      \marginpar{Получаваме, че $\equiv_\A\ =\ \equiv_2$}
      Следователно, $\equiv_2\ =\ \equiv_3$ и минималният автомат разпознаващ езика $L$
      има четири състояния. Вижте Фигура \ref{sub:min2} за преходите на минималния автомат.
      Минималният автомат може да се представи и таблично:

      \begin{tabular}{|c|c|c|c|c|}
        \hline
        $\delta$ & $C_0$ & $C_1$ & $C_2$ & $C_3$ \\
        \hline
        $a$ & $C_1$ & $C_2$ & $C_3$ & $C_3$ \\
        \hline
        $b$ & $C_1$ & $C_2$ & $C_3$ & $C_3$ \\
        \hline
      \end{tabular}
      
  \end{itemize}
\end{example}

\section{Регулярни граматики}
\index{граматика!регулярна}
\section*{Библиография}

Основни източници в тази глава са:
\begin{itemize}
\item 
  глави 2 и 3 от \cite{hopcroft1}.
\item
  глави 2,3 и 4 от \cite{hopcroft2}.
\item
  Глава 1 от \cite{sipser1}.
\item
  глава 2 от \cite{papadimitriou}.
\item
  Първа част на \cite{kozen}. Въпросът за минимизация на автомат е разгледан подробно.
\end{itemize}

%\bibentry{sipser1}

% \newpage
% \cite{min-hopcroft}
% \section{Въпроси}

% Вярно ли е, че:
% \begin{itemize}
%   % \item
% %   \marginpar{Не}
% %   езикът $\{a^nb^n\mid n \in \Nat \}$ е регулярен?
% % \item
% %   \marginpar{Не}
% %   езикът $\{a^nb^k\mid n > k\}$ е регулярен?
% % \item
% %   \marginpar{Не}
% %   езикът $\{a^{n^2}\mid n \in \Nat\}$ е регулярен?
% \item
%   \marginpar{Да}
%   за всеки два регулярни езика $R_1, R_2$, то $R_1 \setminus R_2$ е регулярен ?
% \item
%   \marginpar{Да}
%   за всеки краен език $F$ и всеки регулярен $R$, то $R\setminus F$ е регулярен ?
% \item
%   \marginpar{Да}
%   за всеки краен език $F$ и всеки рег. $R$, то $R\cup (\Sigma^\star \setminus F)$ е регулярен ?
% \item
%   \marginpar{Да}
%   съществува регулярен език $R$ и нерегулярен $K$, за които $R\cap K$ не е регулярен ?
% \item
%   \marginpar{Да}
%   съществува регулярен език $R$ и нерегулярен $K$, за които $R\setminus K$ не е регулярен ?
% \item
%   \marginpar{Не}
%   за всеки регулярен език $R$ и всеки $K \subseteq R$, то $R\setminus K$ е регулярен ?
% \item
%   \marginpar{Не}
%   Езикът $L = \{\omega \in \{a,b\}^\star \mid n_a(\omega) \text{ не дели }n_b(\omega)\}$ е регулярен?
% \item
%   \marginpar{Да}
%   съществува алгоритъм, който може да провери дали за даден регулярен израз $r$
%   е изпълнено, че $\abs{\L(r)} = 0$.
% \item
%   \marginpar{Да}
%   съществува алгоритъм, който може да провери дали за даден регулярен израз $r$
%   е изпълнено, че $\abs{\L(r)} < \infty$.
% \item
%   \marginpar{Да}
%   съществува алгоритъм, който може да провери дали за даден регулярен израз $r$
%   е изпълнено, че $\abs{\L(r)} = \infty$.
% \item
%   \marginpar{Да}
%   съществува алгоритъм, който може да провери дали за дадени регулярни изрази $r_1$ и $r_2$
%   е изпълнено, че $\L(r_1) = \L(r_2)$.
% \item
%   съществува алгоритъм, който може да провери дали за дадени регулярни изрази $r_1$ и $r_2$
%   е изпълнено, че $\L(r_1) \neq \L(r_2)$.
% \item
%   съществува алгоритъм, който може да провери дали за дадени регулярни изрази $r_1$ и $r_2$
%   е изпълнено, че $\L(r_1) \subseteq \L(r_2)$.
% \item
%   съществува алгоритъм, който може да провери дали за дадени регулярни изрази $r_1$ и $r_2$
%   е изпълнено, че $\L(r_1) \subsetneq \L(r_2)$.
% \item
%   съществува алгоритъм, който може да провери дали за дадени регулярни изрази $r_1$ и $r_2$
%   е изпълнено, че $\L(r_1) \cap \L(r_2) = \emptyset$.
% \item
%   съществува алгоритъм, който може да провери дали за дадени регулярни изрази $r_1$ и $r_2$
%   е изпълнено, че $\L(r_1) \cap \L(r_2) \neq \emptyset$.
% \item
%   съществува алгоритъм, който може да провери дали за дадени регулярни изрази $r_1$ и $r_2$
%   е изпълнено, че $\L(r_1) \cup \L(r_2) = \emptyset$.
% \item
%   съществува алгоритъм, който може да провери дали за дадени регулярни изрази $r_1$ и $r_2$
%   е изпълнено, че $\L(r_1) \cup \L(r_2) \neq \emptyset$.
% \item
%   съществува алгоритъм, който може да провери дали за дадени регулярни изрази $r_1$ и $r_2$
%   е изпълнено, че $\L(r_1) \setminus \L(r_2) = \emptyset$.
% \item
%   съществува алгоритъм, който може да провери дали за дадени регулярни изрази $r_1$ и $r_2$
%   е изпълнено, че $\L(r_1) \setminus \L(r_2) \neq \emptyset$.
% \end{itemize}

% \section{Домашна работа}

% \begin{itemize}
% \item
%   Вход - файл, в който е записан регулярен израз
% \item
%   Преобразуване на регулярния израз в обратен полски запис.
%   (\href{http://en.wikipedia.org/wiki/Shunting-yard_algorithm}{тук} 
%   добре е обяснено как става за произволни аритмечни изрази)
% \item
%   Строене на краен детерминиран автомат по регулярния израз.
% \item
%   Извеждане на автомата във формат за програмата \href{http://graphviz.org}{graphviz}.
%   (вижте \href{http://sundarpillay.blogspot.com/2012/02/graphviz-and-finite-automata-diagrams_05.html}{пример})
% \end{itemize}


%%% Local Variables: 
%%% mode: latex
%%% TeX-master: "EAI"
%%% End: 


\section{Допълнителни свойства и задачи}

\begin{dfn}
  Хомоморфизъм
\end{dfn}

Някои свойства на регулярните езици:
\begin{itemize}
\item 
  те са затворени относно хомоморфизми, т.е.
  ако $L \subseteq \Sigma^\star_1$ е регулярен език и $h:\Sigma_1\to\Sigma^\star_2$ е хомоморфизъм, 
  то езикът 
  \[h(L) = \{h(\alpha) \in \Sigma^\star_2 \mid \alpha \in L\}\]
  е регулярен.
\item
  те са затворени относно обратни хомоморфизми, т.е.
  ако $L\subseteq \Sigma^\star_2$ е регулярен език и $h:\Sigma_1\to\Sigma^\star_2$ е хомоморфизъм, 
  то езикът
  \[h^{-1}(L) = \{\alpha \in \Sigma^\star_1 \mid h(\alpha) \in L\}\]
  е регулярен.
\end{itemize}

\begin{problem}
  \marginpar{\cite{papadimitriou} стр. 84}
  При дадени езици $L$, $L'$ над азбуката $\Sigma$, да разгледаме:
  \begin{enumerate}[a)]
  \item
    $\mbox{Pref}(L) = \{\alpha \in \Sigma^\star \mid (\exists \beta \in \Sigma^\star)[\alpha\beta \in L]\}$;
  \item
    $\mbox{Suf}(L) = \{\beta \in \Sigma^\star \mid (\exists \alpha \in \Sigma^\star)[\alpha\beta \in L]\}$;
  \item 
     $\mbox{Half}(L) = \{\omega \in \Sigma^\star \mid (\exists \alpha \in \Sigma^\star)[\omega\alpha \in L\ \&\ \abs{\omega} = \abs{\alpha}]\}$;
   \item
     $L/L' = \{\alpha \in \Sigma^\star \mid (\exists \beta \in L')[\alpha\beta \in L]\}$;
   \item
     $\mbox{Max}(L) = \{\alpha \in \Sigma^\star \mid (\forall \beta\in\Sigma^\star)[\beta \neq \varepsilon\implies \alpha\beta \not\in L]\}$.
  \end{enumerate}
  За всички тези езици, докажете, че са регулярни при условие, че $L$ и $L'$ са регулярни.
  Освен това, докажете, че $L/L'$ е регулярен и при условието, че $L$ е регулярен, но $L'$ е произволен език.
\end{problem}

\begin{problem}
  \marginpar{\cite{sipser1}, стр. 90}
  Да разгледаме езика
  \[L = \{\omega \in \{0,1\}^\star \mid \omega\text{ съдържа равен брой поднизове }01\text{ и }10\}.\]
  Например, $101 \in L$, защото съдържа по веднъж $10$ и $01$.
  $1010 \not\in  L$, защото съдържа два пъти $10$ и само веднъж $01$.
  Докажете, че $L$ е регулярен.
\end{problem}

\begin{problem}
  За момента ще разглеждаме езици над азбука само с един символ $\Sigma = \{a\}$.
  Да положим за всяко $p,q\in\Nat$, 
  \[L(p,q) = \{a^l \mid (\exists n\in\Nat)[l = p+q\cdot n]\},\]
  т.е. дължините на думите образуват аритметична прогресия.
  Такива езици ще наричаме {\em породени от аритметична прогресия}.
  Докажете, че $L \subseteq \{a\}^\star$ е регулярен точно тогава, когато $L$
  е обединение на крайно много езици породени от артиметична прогресия.
  
  За произволна азбука $\Sigma$, докажете, че ако $L \subseteq \Sigma^\star$ е регулярен език, 
  то множеството $\{\abs{\omega}\mid \omega \in L\}$ може да се представи като обединение 
  на крайно много аритметични прогресии.
\end{problem}
\begin{hint}
  \begin{itemize}
  \item 
    Докажете, че за всяко $p,q \in \Nat$, $L(p,q)$ е регулярен език.
  \item
    Докажете, че за крайно много $p_0,\dots,p_k$, $q_0,\dots,q_k$,
    $\bigcup_{i \leq k}L(p_i,q_i)$ е регулярен език.
  \item
    С индукция по построението на регулярните езици, 
    докажете, че ако $L$ е регулярен, то $L$ може да се представи
    като крайно обединение на езици породени от аритметични прогресии.
    Съществената част от доказателството се състои в следното:
    \begin{itemize}
    \item 
      \marginpar{$L(p_1,q_1)\cdot L(p_2,q_2) = L(p_1+p_2,\mbox{НОД}(q_1,q_2))\setminus F$, където $F$ е крайно м-во, и ако $q_1 = q_2$, то $F = \emptyset$}
      езикът $L(p_1,q_1) \cdot L(p_2,q_2)$ може да се представи като крайно обединение 
      на езици породени от артиметични прогресии.
    \item
      езикът $L(p,q)^\star$ може да се представи като крайно обединение 
      на езици породени от артиметични прогресии.
    \end{itemize}
  \end{itemize}
\end{hint}

\begin{problem}
  Да разгледаме азбуката:
  \[\Sigma_3 = \left\{\begin{bmatrix} 0\\0\\0\end{bmatrix},\begin{bmatrix} 0\\0\\1\end{bmatrix},\begin{bmatrix} 0\\1\\0\end{bmatrix},\begin{bmatrix} 0\\1\\1\end{bmatrix},\dots,\begin{bmatrix} 1\\1\\1\end{bmatrix}\right\}.\]
  Докажете, че 
  $L = \left\{\begin{bmatrix} \alpha\\\beta\\\gamma\end{bmatrix} \in \Sigma^\star_3 \mid \alpha_{(2)}+\beta_{(2)} = \gamma_{(2)}\right\}$
  е автоматен език.
\end{problem}
\begin{hint}
  Отново по-удобно е да построим автомат $\A$, $\L(\A) = L^R$.
  Да започнем с състоянието $q_{\scriptscriptstyle{=}}$, за което искаме да имаме свойството, че за произволно състояние $q$,
  \[\delta^\star(q, \tiny{ \begin{bmatrix} \alpha\\ \beta \\ \gamma\end{bmatrix} }) = q_{\scriptscriptstyle{=}}  \iff \alpha^R_{(2)} + \beta^R_{(2)} = \gamma^R_{(2)}.\]
  Понеже за $\varepsilon + \varepsilon = \varepsilon$, състоянието $q_{\scriptscriptstyle{=}}$ ще бъде начално и финално за $\A$.
  \begin{itemize}
  \item 
    Нека $\alpha_{(2)}+\beta_{(2)} = \gamma_{(2)}$. Тогава:
    \begin{itemize}
    \item 
      $0\alpha + 0\beta = 0\gamma$;
      \marginpar{$\delta(q_{\scriptscriptstyle{=}},\tiny{ \begin{bmatrix} 0\\ 0 \\ 0\end{bmatrix} }) = q_{\scriptscriptstyle{=}}$}
    \item
      $0\alpha + 1\beta = 1\gamma$;
      \marginpar{$\delta(q_{\scriptscriptstyle{=}},\tiny{ \begin{bmatrix} 0\\ 0 \\ 0\end{bmatrix} }) = q_{\scriptscriptstyle{=}}$}
    \item
      $1\alpha + 0\beta = 1\gamma$;
      \marginpar{$\delta(q_{\scriptscriptstyle{=}},\tiny{ \begin{bmatrix} 1\\ 0 \\ 1\end{bmatrix} }) = q_{\scriptscriptstyle{=}}$}
    \item
      $1\alpha + 1\beta = 10\gamma$. Този случай е по-специален и трябва да бъде разгледан отделно.
      Трябва да отидем в състояние $q_1$, в което ще помним, че третия ред трябва да започва с $1$-ца.
      \marginpar{$\delta(q_{\scriptscriptstyle{=}},\tiny{ \begin{bmatrix} 1\\ 1 \\ 0\end{bmatrix} }) = q_1$}
    \item
      За останалите $x \in \Sigma_3$, $\delta(q_{\scriptscriptstyle{=}},x) = q_{err}$,
      където $q_{err}$ е състоянието, от което не можем да излезем.
      % Остават $\delta(q_{\scriptscriptstyle{=}},\tiny{ \begin{bmatrix} 0\\ 1 \\ 0\end{bmatrix} }) = \delta(q_{\scriptscriptstyle{=}},\tiny{ \begin{bmatrix} 1\\ 0 \\ 0\end{bmatrix} }) = \delta(q_{\scriptscriptstyle{=}},\tiny{ \begin{bmatrix} 1\\ 1 \\ 1\end{bmatrix} }) = \delta(q_{\scriptscriptstyle{=}},\tiny{ \begin{bmatrix} 0\\ 0 \\ 1\end{bmatrix}}) = q_{err}$;
    \end{itemize}
  \item
    Горните разглеждания ни подсказват, че ще ни трябва и състояние $q_1$, за което искаме да е изпълнено свойството,
    че за произволно $q$,
    \[\delta^\star(q, \tiny{ \begin{bmatrix} \alpha\\ \beta \\ \gamma\end{bmatrix} }) = q_{\scriptscriptstyle{1}}  \iff \alpha^R_{(2)} + \beta^R_{(2)} = 1\gamma^R_{(2)}.\]
    Да разгледаме сега случая $\alpha + \beta = 1\gamma$. Тогава:
    \begin{itemize}
    \item 
      \marginpar{$\delta(q_1,\tiny{ \begin{bmatrix} 0\\ 0 \\ 1\end{bmatrix} }) = q_{\scriptscriptstyle{=}}$}
      Очевидно е, че $0\alpha + 0\beta = 1\gamma$;
    \item
      \marginpar{$\delta(q_1,\tiny{ \begin{bmatrix} 1\\ 1 \\ 1\end{bmatrix} }) = q_{1}$}
      $1\alpha + 1\beta = 11\gamma$;
    \item
      \marginpar{$\delta(q_1,\tiny{ \begin{bmatrix} 1\\ 0 \\ 0\end{bmatrix} }) = q_{1}$}
      $1\alpha + 0\beta = 10\gamma$;
    \item
      Аналогично, $0\alpha + 1\beta = 10\gamma$;
    \item
      За останалите $x \in \Sigma_3$, $\delta(q_{1},x) = q_{err}$.
    \end{itemize}    
    \marginpar{$\delta(q_1,\tiny{ \begin{bmatrix} 0\\ 1 \\ 0\end{bmatrix} }) = q_{1}$}
  \end{itemize}
\end{hint}

\begin{problem}
  Да разгледаме азбуката:
  \[\Sigma_2 = \left\{\begin{bmatrix} 0\\0\end{bmatrix},\begin{bmatrix} 0\\1\end{bmatrix},\begin{bmatrix} 1\\0\end{bmatrix},\begin{bmatrix} 1\\1\end{bmatrix}\right\}.\]
  Една дума над азбуката $\Sigma_2$ ни дава два реда от $0$-ли и $1$-ци, които ще разглеждаме като числа в двоична бройна система.
  Да разгледаме езиците:
  \begin{itemize}
  \item 
    $L_1 = \{\omega \in \Sigma^\star_2 \mid \text{долният ред на }\omega\text{ е по-голямо число от горния ред}\}$;
  \item
    $L_2 = \{\omega \in \Sigma^\star_2 \mid \text{долният ред на }\omega\text{ е три пъти по-голямо число от горния}\}$;
  \item
    $L_3 = \{\omega \in \Sigma^\star_2 \mid \text{долният ред на }\omega\text{ е обратния низ на горния ред}\}$.
  \end{itemize}
  Докажете, че  $L_1$ и $L_2$ са автоматни, а $L_3$ не е автоматен.
\end{problem}
\begin{hint}
  Ще построим автомат $\A = \FA$ за езика $L^R_1$.
  За улеснение, в рамките на тази задача ще пишем, че $\alpha \equiv \beta$, ако $(\alpha^R)_{(2)} = (\beta^R)_{(2)}$,
  $\alpha < \beta$, ако $(\alpha^R)_{(2)} < (\beta^R)_{(2)}$,
  $\alpha > \beta$, ако $(\alpha^R)_{(2)} > (\beta^R)_{(2)}$.

  Нека състоянията на автомата са $Q = \{q_{\scriptscriptstyle{=}},q_{\scriptscriptstyle{<}},q_{\scriptscriptstyle{>}}\}$.
  Искаме да е изпълнено свойствата:
  \begin{itemize}
  \item 
    % За всяко $q \in Q$,
    $\delta^\star(q_{\scriptscriptstyle{=}}, \scriptsize{\begin{bmatrix} \alpha\\ \beta\end{bmatrix}}) = q_{\scriptscriptstyle{=}}$ точно тогава, когато $\alpha \equiv \beta$;
  \item 
    % За всяко $q \in Q$,
    $\delta^\star(q_{\scriptscriptstyle{=}}, \scriptsize{\begin{bmatrix} \alpha\\ \beta\end{bmatrix}}) = q_{\scriptscriptstyle{<}}$ точно тогава, когато $\alpha < \beta$;
  \item 
    % За всяко $q \in Q$,
    $\delta^\star(q_{\scriptscriptstyle{=}}, \scriptsize{\begin{bmatrix} \alpha\\ \beta\end{bmatrix}}) = q_{\scriptscriptstyle{>}}$ точно тогава, когато $\alpha > \beta$.
  \end{itemize}
  Множеството от финални състояния ще бъде $F = \{q_{\scriptscriptstyle{<}}\}$, а началното състояние $s = q_{\scriptscriptstyle{=}}$.
  За да дефинираме функцията на преходите, трябва да разгледа няколко случая, в зависимост от това какво е отношението между $\alpha$ и $\beta$.
  \begin{itemize}
  \item
    Нека $\alpha \equiv \beta$. Тогава:  
    \begin{itemize}
    \item 
      \marginpar{$\delta(q_{\scriptscriptstyle{=}},\scriptsize{\begin{bmatrix} 0\\0\end{bmatrix}}) = \delta(q_{\scriptscriptstyle{=}},\scriptsize{\begin{bmatrix} 1\\1\end{bmatrix}}) = q_{\scriptscriptstyle{=}}$}
      $\alpha 0 \equiv \beta 0$, $\alpha 1 \equiv \beta 1$;
    \item
      \marginpar{$\delta(q_{\scriptscriptstyle{=}},\scriptsize{\begin{bmatrix} 0\\1\end{bmatrix}}) = q_{\scriptscriptstyle{>}}$}
      $\alpha 0 < \beta 1$;
    \item
      \marginpar{$\delta(q_{\scriptscriptstyle{=}},\scriptsize{\begin{bmatrix} 1\\0\end{bmatrix}}) = q_{\scriptscriptstyle{<}}$}
      $\alpha 1 < \beta 0$;
    \end{itemize}
  \item 
    Нека $\alpha < \beta$. Тогава:
    \begin{itemize}
    \item 
      \marginpar{$\delta(q_{\scriptscriptstyle{<}},\scriptsize{\begin{bmatrix} 0\\0\end{bmatrix}}) = \delta(q_{\scriptscriptstyle{<}},\scriptsize{\begin{bmatrix} 1\\1\end{bmatrix}}) = \delta(q_{\scriptscriptstyle{<}},\scriptsize{\begin{bmatrix} 0\\1\end{bmatrix}}) = q_{\scriptscriptstyle{<}}$}
      $\alpha 0 < \beta 0$, $\alpha 1 < \beta 1$, $\alpha 0 < \beta 1$;
    \item
      \marginpar{$\delta(q_{\scriptscriptstyle{<}},\scriptsize{\begin{bmatrix} 1\\0\end{bmatrix}}) = q_{\scriptscriptstyle{>}}$}
      $\alpha 1 > \beta 0$;
    \end{itemize}    
  \item
    Нека $\alpha > \beta$. Тогава:
    \begin{itemize}
    \item 
      \marginpar{$\delta(q_{\scriptscriptstyle{>}},\scriptsize{\begin{bmatrix} 0\\0\end{bmatrix}}) = \delta(q_{\scriptscriptstyle{>}},\scriptsize{\begin{bmatrix} 1\\1\end{bmatrix}}) = \delta(q_{\scriptscriptstyle{>}},\scriptsize{\begin{bmatrix} 1\\0\end{bmatrix}}) = q_{\scriptscriptstyle{>}}$}
      $\alpha 0 > \beta 0$, $\alpha 1 > \beta 1$, $\alpha 1 > \beta 0$;
    \item
      $\alpha 0 < \beta 1$;
    \end{itemize}
  \end{itemize}
  Докажете, че за така дефинирания автомат $\A$, $\L(\A) = L^R_1$.
  \marginpar{$\delta(q_{\scriptscriptstyle{>}},\scriptsize{\begin{bmatrix} 0\\1\end{bmatrix}}) = q_{\scriptscriptstyle{<}}$}
\end{hint}

%%% Local Variables: 
%%% mode: latex
%%% TeX-master: "EAI"
%%% End: 


\chapter{Безконтекстни езици и стекови автомати}


\section{Безконтекстни граматики}
\index{граматика!безконтекстна}
% От Сипсер, същото е в слайдовете на Сашка
% Малко е тъпо, че в Пападимитриу дефиницията е различна. Там \Sigma \subseteq V

\begin{dfn}
  \marginpar{На англ. {\bf context-free grammar}}
  \marginpar{Други срещани наименования на български са {\bf контекстно-свободна}, {\bf контекстно-независима}}
%  \marginpar{Според йерархията на Чомски, това са граматики тип }
  Безконтекстна граматика e четворка от вида
  \[G = (V,\Sigma,R,S),\]
  където
  \begin{itemize}
  \item
    \marginpar{Променливите се наричат също нетерминали}
    $V$ е крайно множество от {\em променливи};
  \item
    \marginpar{Буквите се наричат също терминали.}
    $\Sigma$ е крайно множество от {\em букви}, $\Sigma \cap V = \emptyset$;
  \item
    $R \subseteq V\times (V\cup\Sigma)^\star$, крайно множество от {\em правила};
  \item
    $S \in V$ е началната променлива. 
  \end{itemize}
\end{dfn}

При дадена граматика $G$, за правилата на граматиката обикновено ще пишем $A \rightarrow \alpha$ вместо $(A,\alpha) \in R$.
Ще въведем и релация между думи $\alpha,\beta\in (V \cup \Sigma)^\star$, която ще казва, че думата $\beta$
се получаава от $\alpha$ като приложим правло от граматиката.
За две думи $u,v\in (V\cup\Sigma)^\star$ ще пишем $u \rightarrow_G v$, ако съществуват думи $x,y\in (\Sigma\cup V)^\star$, $A\in V$,
правило $A\rightarrow \alpha$ и $u = xAy$, $v = x\alpha y$.
С $\rightarrow^\star_G$ ще означаваме рефлексивното и транзитивно затваряне на релацията $\rightarrow_G$.
% \marginpar{Да се дефинира $\rightarrow^\star_G$}

Езикът породен от граматиката $G$ е множеството от думи
\[\L(G) = \{\alpha\in\Sigma^\star\mid S \rightarrow^\star_G \alpha\}.\]
  
\begin{problem}
  Докажете, че езикът $L = \{a^mb^nc^k\mid m+n \geq k\}$ е безконтекстен.
\end{problem}
\begin{proof}
  Да разгледаме граматиката $G$ с правила
  \begin{align*}
    S& \rightarrow aSc\vert aS \vert B\\
    B& \rightarrow bBc\vert bB\vert\varepsilon.
  \end{align*}
  
  Лесно се вижда с индукция по $n$, че за всяко $n$ имаме свойствата:
  \marginpar{\ding{45} Докажете!}
  \begin{itemize}
  \item 
    $S \rightarrow^\star a^nSc^n$,
  \item
    $S \rightarrow^\star a^nS$,
  \item
    $B \rightarrow^\star a^nBc^n$,
  \item
    $B \rightarrow^\star b^nB$.
  \end{itemize}
  Комбинирайки горните свойства, можем да видим, че за всяко $n \geq k$,
  \begin{itemize}
  \item 
    $S \rightarrow^\star a^nSc^k$,
  \item
    $B \rightarrow^\star b^nBc^k$.
  \end{itemize}
  За да докажем, че $L \subseteq L(G)$, 
  да разгледаме една дума $\omega \in L$, т.е. $\omega = a^mb^nc^k$, където $m+n \geq k$.
  Имаме два случая:
  \begin{itemize}
  \item 
    $k \leq m$, т.е. $m = k+l$ и $m+n = k+l+n$.
    Тогава имаме изводите:
    \[S \rightarrow^\star a^kSc^k,\ S \rightarrow^\star a^lS,\ S \rightarrow B,\ B \rightarrow^\star b^nB,\ B \rightarrow \varepsilon.\]
    Обединявайки всичко това, получаваме:
    \[S \rightarrow^\star a^mb^nc^k.\]
  \item
    $k > m$, т.е. $k = m+l$, за някое $l > 0$, и $m+n = k+r = m+l+r$, за някое $r$.
    Тогава имаме изводите:
    \[S \rightarrow^\star a^mSc^m,\ S\rightarrow B,\ B\rightarrow^\star b^lBc^l,\ B\rightarrow b^rB,\ B\rightarrow\varepsilon,\]
    и отново получаваме $S \rightarrow^\star a^mb^nc^k$.
  \end{itemize}
  Така доказахме, че $\omega \in \L(G)$.
  
  Сега ще докажем, че $\L(G) \subseteq L$.
  С индукция по дължината на извода $l$,
  ще докажем, че ако $S \stackrel{l}{\rightarrow}\omega$, то $\omega \in M$, където
  \[M = \{a^nSc^k\mid n\geq k\}\cup\{a^nb^mBc^k\mid n+m\geq k\}\cup\{a^nb^mc^k\mid n+m\geq k\}.\]
  
  Ако $l = 0$, то е ясно, че $S \stackrel{0}{\rightarrow} S$ и $S \in M$.

  Нека $l > 0$ и $S \stackrel{l-1}{\rightarrow} \alpha \rightarrow \omega$.
  От {\bf И.П.} имаме, че $\alpha \in M$. Нека $\omega$ се получава от $\alpha$ с прилагане на правилото $C \rightarrow \gamma$.
  Разглеждаме всички варианти за думата $\alpha \in M$ и за правилото $C\rightarrow \gamma$ в граматиката $G$
  за да докажем, че  $\omega \in M$.
  Удобно е да представим всички случаи в таблица.
  \begin{center}
    \begin{tabular}{| c | c | c |}
      \hline
      $\alpha\in M$ & $C \rightarrow \gamma$ & $\omega \in M?$ \\ \hline
      $a^nSc^k$ & $S \rightarrow aSc$ & $a^{n+1}Sc^{k+1}$ \\ \hline
      $a^nSc^k$ & $S \rightarrow aS$ & $a^{n+1}Sc^{k}$ \\ \hline
      $a^nSc^k$ & $S \rightarrow B$ & $a^{n}Bc^{k}$ \\ \hline
      $a^nb^mBc^k$ & $B \rightarrow bBc$ & $a^nb^{m+1}Bc^{k+1}$\\ \hline
      $a^nb^mBc^k$ & $B \rightarrow bB$ & $a^nb^{m+1}Bc^{k}$\\ \hline
      $a^nb^mBc^k$ & $B \rightarrow \varepsilon$ & $a^nb^{m}c^{k}$\\ \hline
    \end{tabular}
  \end{center}
  Във всички случаи се установява, че $\omega \in M$.
  Сега, за всяка дума $\omega \in L(G)$ следва, че
  \[\omega \in \Sigma^\star \cap M = \{a^mb^nc^k\mid m+n \geq k\}.\]
\end{proof}


\begin{problem}
  \marginpar{
    $S \to aS \mid aSc \mid aB \mid bB$\\
    $B \to bB \mid bBc \mid \varepsilon$
}
  Докажете, че езикът $L = \{a^mb^nc^k\mid m+n \geq k + 1\}$ е безконтекстен.  
\end{problem}

\begin{problem}
  \label{pr:nanb}
  Нека $\omega$ е произволна дума над азбуката $\{a,b\}$. 
  Тогава:
  \begin{enumerate}[a)]
  \item 
    ако $n_a(\omega) = n_b(\omega) + 1$, то съществуват думи $\omega_1$, $\omega_2$, за които
    $\omega = \omega_1 a \omega_2$, $n_a(\omega_1) = n_b(\omega_1)$ и $n_a(\omega_2) = n_b(\omega_2)$.
  \item
    ако $n_b(\omega) = n_a(\omega) + 1$, то съществуват думи $\omega_1$, $\omega_2$, за които
    $\omega = \omega_1 b \omega_2$, $n_a(\omega_1) = n_b(\omega_1)$ и $n_a(\omega_2) = n_b(\omega_2)$.
  \end{enumerate}
\end{problem}
\begin{proof}
  Пълна индукция по дължината на думата $\omega$, за които $n_a(\omega) = n_b(\omega)+1$.
  \begin{itemize}
  \item 
    $\abs{\omega} = 1$. Тогава $\omega_1 = \omega_2 = \varepsilon$ и $\omega = a$.
  \item
    $\abs{\omega} = n+1$. Ще разгледаме два случая, в зависимост от първия символ на $\omega$.
    \begin{itemize}
    \item 
      Случаят $\omega = a\omega'$ е лесен. (Защо?)
    \item
      Интересният случай е $\omega = b\omega'$.    
      Тогава $\omega = b^{i+1}a\omega'$. Да разгледаме думата $\omega''$, която се получава от $\omega$
      като премахнем първото срещане на думата $ba$, т.е. 
      $\omega'' = b^i\omega'$ и $\abs{\omega''} = n-1$.
      Понеже от $\omega$ сме премахнали равен брой $a$-та и $b$-та, $n_a(\omega'') = n_b(\omega'')+1$.
      Според {\bf И.П.} за $\omega''$, можем да запишем думата като $\omega'' = \omega''_1a\omega''_2$
      и $n_a(\omega''_1) = n_b(\omega''_1)$, $n_a(\omega''_2) = n_b(\omega''_2)$.
      Понеже $b^i$ е префикс на $\omega''_1$, за да получим обратно $\omega$, трябва 
      да прибавим премахнатата част $ba$ веднага след $b^i$ в $\omega''_1$.
    \end{itemize}
  \end{itemize}
\end{proof}

\begin{problem}
  За произволна дума $\omega \in \{a,b\}^\star$, 
  докажете, че ако $n_a(\omega) > n_b(\omega)$, то съществуват думи $\omega_1$ и $\omega_2$,
  за които $\omega = \omega_1 a \omega_2$ и $n_a(\omega_1) \geq n_b(\omega_1)$, $n_a(\omega_2) \geq n_b(\omega_2)$.
\end{problem}

\begin{problem}
  Да се докаже, че езикът $L = \{\alpha \in \{a,b\}^\star\mid n_a(\alpha) = n_b(\alpha)\}$ 
  е безконтекстен.
\end{problem}
\begin{proof}
  \marginpar{  Алтернативна граматика за езика $L$ е
  \begin{align*}
    S& \rightarrow aB\vert bA\\
    A& \rightarrow a\vert aS\vert bAA\\
    B& \rightarrow b\vert bS\vert aBB
  \end{align*}}
  Една възможна граматика $G$ е следната: 
  \[S \rightarrow aSbS\vert bSaS \vert\varepsilon.\]
  Например, да разгледаме извода на думата $aabbba$ в тази граматика:
  \begin{align*}
    S & \to aSbS \to aaSbSbS \to aa\varepsilon bSbS \to aab\varepsilon bS \to aabbbSaS\\
    & \to aabbb\varepsilon a S \to aabbba.
  \end{align*}
  
  Като следствие от \Prob{nanb} може лесно да се изведе, че за думи $\omega$, за които $n_a(\omega) = n_b(\omega)$,
  е изпълнено следното:
  \begin{enumerate}[a)]
  \item 
    ако $\omega = a\omega'$, то
    $\omega = a\omega_1b\omega_2$ и $n_a(\omega_1) = n_b(\omega_1)$, $n_a(\omega_2) = n_b(\omega_2)$;
  \item
    ако $\omega = b\omega'$, то
    $\omega = b\omega_1a\omega_2$ и $n_a(\omega_1) = n_b(\omega_1)$, $n_a(\omega_2) = n_b(\omega_2)$.
  \end{enumerate}

  Сега първо ще проверим, че $L \subseteq L(G)$.
  За целта ще докажем с {\em пълна индукция} по дължината на думата $\omega$, че за всяка дума $\omega$ със свойството $n_a(\omega) = n_b(\omega)$ е изпълнено
  $S \rightarrow^\star \omega$.
  \begin{itemize}
  \item 
    Нека $\abs{\omega} = 0$. Тогава $S \rightarrow \varepsilon$.
  \item
    Нека $\abs{\omega} = k+1$. Имаме два случая.
    \begin{itemize}
    \item 
      $\omega = a\omega^\prime$, т.е. от свойство а), $\omega = a\omega_1b\omega_2$ и $n_a(\omega_1) = n_b(\omega_1)$, $n_a(\omega_2) = n_b(\omega_2)$.
      Тогава $\abs{\omega_1} \leq k$ и по И.П. $S \rightarrow^\star \omega_1$.
      Аналогично, $S \rightarrow^\star \omega_2$.
      Понеже имаме правило $S \rightarrow aSbS$, заключаваме че $S \rightarrow^\star a\omega_1b\omega_2$.
    \item
      $\omega = b\omega^\prime$, т.е. свойство б), $\omega = b\omega_1a\omega_2$ и $n_a(\omega_1) = n_b(\omega_1)$, $n_a(\omega_2) = n_b(\omega_2)$.
      Този случай се разглежда аналогично.
    \end{itemize}
  \end{itemize}
  
  Преминаваме към доказателството на другата посока, т.е. $L(G) \subseteq L$.
  Тук с индукция по дължината на извода $l$ ще докажем, че
  $S \stackrel{l}{\rightarrow} \omega$, то $\omega \in M$,
  където
  \[M = \{\omega \in \{a,b,S\}^\star \mid n_a(\omega) = n_b(\omega)\}.\]
  За $l = 0$  е ясно, че $S \stackrel{0}{\rightarrow^\star} S$.
  За $l = k+1$, то $S \stackrel{k}{\rightarrow^\star} \alpha \rightarrow \omega$.
  От {\bf И.П.} имаме, че $\alpha \in M$.
  Нека $\omega$ се получава от $\alpha$ с прилагане на правилото $C \rightarrow \gamma$.
  Разглеждаме всички варианти за думата $\alpha \in M$ и за правилото $C\rightarrow \gamma$ в граматиката $G$
  за да докажем, че  $\omega \in M$.
  Удобно е да представим всички случаи в таблица.
  \begin{center}
    \begin{tabular}{| c | c | c |}
      \hline
      $\alpha$ & $C \rightarrow \gamma$ & $\omega$ \\ \hline
      $\in M$ & $S \rightarrow aSbS$ & $\in M$ \\ \hline
      $\in M$ & $S \rightarrow bSaS$ & $\in M$ \\ \hline
      $\in M$ & $S \rightarrow \varepsilon$ & $\in M$ \\ \hline
    \end{tabular}
  \end{center}
  Във всички случаи лесно се установява, че $\omega \in M$.
  Така за всяка дума $\omega \in L(G)$ следва, че
  \[\omega \in \Sigma^\star \cap M = L.\]
\end{proof}

\begin{problem}
  Докажете, че следните езици са безконтекстни.
  \begin{enumerate}[a)]
  \item
    \marginpar{$S \rightarrow aSa\ \vert\ bSb\ \vert\ \varepsilon$}
    $L = \{ww^R \mid w \in \{a,b\}^\star\}$;
  \item
    \marginpar{$S \rightarrow aSa\ \vert\ bSb\ \vert\ a\vert\ b\ \vert\ \varepsilon$}
    $L = \{w \in \{a,b\}^\star \mid w = w^R\}$;
  \item
    $L = \{a^nb^{2m}c^{n} \mid m,n \in \Nat\}$;
  \item
    $L = \{a^nb^{m}c^{m}d^n \mid m,n \in \Nat\}$;
  \item
    $L = \{a^nb^{2k} \mid n,k \in \Nat\ \&\ n \neq k\}$;
  \item
    \marginpar{$S \rightarrow aSb | aS | a$}
    $L = \{a^nb^k \mid n > k\}$;
  \item
    $L = \{a^nb^k \mid n \geq 2k\}$;
  \item
    \marginpar{$S \rightarrow aSc | B,\ B \rightarrow bBc | \varepsilon$}
    $L = \{a^nb^mc^{n+m}\mid n,m \in \Nat\}$;
  \item
    \marginpar{$S \rightarrow aSc | aS | B$, $B\rightarrow bBc | bB | \varepsilon$}
    $L = \{a^nb^kc^m \mid n + k \geq m\}$;
  \item
    \marginpar{$S \rightarrow aSc | aS | aB | bB$,\\$B\rightarrow bBc | bB | \varepsilon$}
    $L = \{a^nb^kc^m \mid n + k \geq m+1\}$;
  \item
    $L = \{a^nb^kc^m \mid n + k \geq m+2\}$;
  \item
    \marginpar{$S \rightarrow aSc | aS | B | Bc$,\\$B\rightarrow bBc | bB | \varepsilon$}
    $L = \{a^nb^kc^m \mid n + k + 1 \geq m\}$;
  \item
    $L = \{a^nb^kc^m \mid n + k + 2 \geq m\}$;
  \item
    $L = \{a^nb^kc^m \mid n + k \leq m\}$;
  \item
    $L = \{a^nb^kc^m \mid n + k \leq m+1\}$;
  \item
    \marginpar{Обединение на три езика}
    $L = \{a^nb^mc^k \mid n, m, k \text{ не са страни на триъгълник}\}$.
  \item
    $L = \{a,b\}^\star \setminus \{a^{2n}b^n \mid n\in\Nat\}$;
  \item
    \marginpar{$S\to EaE$, $E \to aEbE | bEaE | \varepsilon$}
    $L = \{\alpha \in \{a,b\}^\star\mid n_a(\alpha) = n_b(\alpha) + 1\}$;
  \item
    \marginpar{$S\to E | SaS$, $E \to aEbE | bEaE | \varepsilon$}
    $L = \{\alpha \in \{a,b\}^\star\mid n_a(\alpha) \geq n_b(\alpha)\}$;
  \item
    $L = \{\alpha \in \{a,b\}^\star\mid n_a(\alpha) > n_b(\alpha)\}$;
  \item
    $L = \{\omega_1 a \omega_2 b \mid \omega_1,\omega_2 \in \{a,b\}^\star\ \&\ \abs{\omega_1} = \abs{\omega_2}\}$;
  \item
    $L = \{\alpha c \beta \mid \alpha,\beta \in \{a,b\}^\star\ \&\ \alpha^R\mbox{ е поддума на }\beta \}$.
  \end{enumerate}
\end{problem}

\begin{problem}
  \marginpar{от Владислав}
  Да разгледаме граматиката $G = \CFG$, където  $V = \{S,A,B\}$, $\Sigma = \{a,b\}$, а правилата $R$ са
  \[S \to AA | B, A \to B | bb, B \to aa | aB.\]
  Да се намери езика на тази граматика и да се докаже, че граматиката разпознава точно този език.
\end{problem}

\section{Езици, които не са безконтекстни}

\begin{lemma}[за покачването (безконтекстни езици)]
  \index{лема за покачването!безконтекстни езици}
  \label{lem:pumping-context} 
  \marginpar{(стр. 123 от \cite{sipser1}; стр. 125 от \cite{hopcroft1})}
  За всеки безконтекстен език $L$ съществува $p>0$, такова
  че ако $\alpha\in L, \abs{\alpha} \geq p$, то съществува разбиване на думата на пет части, $\alpha=xyuvw$,
  за което е изпълнено:
  \begin{enumerate}[1)]
  \item
    $\abs{yv}\geq 1$,
  \item
    $\abs{yuv}\leq p$, и
  \item
    $(\forall i\geq 0)[xy^iuv^iw\in L]$.
\end{enumerate}
\end{lemma}
\begin{proof}
  Нека $G$ е граматиката за езика $L$.
  \marginpar{За простота, можем да си мислим, че $G$ е в НФЧ. Тогава $b=2$.}
  Нека \[b = \max\{\abs{\beta} \mid A\rightarrow_G \beta\}.\]
  Можем да приемем, че $b \geq 2$.
  \marginpar{Възлите във вътрешността на дървото са променливи, а листата са букви или $\varepsilon$}
  Това означава, че във всяко дърво на извод, всеки възел има
  не повече от $b$ наследника.
  Нека $p = b^{\abs{V}}+1$. Ще покажем, че $p$ е константа на покачването за граматиката $G$.
  Това означава, че всяка дума с дължина поне $p$ в езика $L$ има дърво на извод с височина
  поне $\abs{V} + 1$.
  
  Нека $\abs{\alpha} \geq p$ и $T$ е дърво на извода за думата $\alpha$.
  Понеже думата $\alpha$ може да има много дървета на извод, нека $T$ също така да бъде {\em с минимален брой възли}. 
  От направените по-горе разсъждения е ясно, че височината на $T$ е поне $\abs{V} + 1$,
  Следователно, по най-дългия път $\pi$ в $T$ имаме поне $\abs{V}+2$ възела, от които
  поне $\abs{V}+1$ са променливи, защото само листата могат да не са променливи.
  Да разгледаме последните $\abs{V}+1$ променливи по пътя $\pi$.
  От принципа на Дирихле следва, че измежду тези $\abs{V}+1$ променливи има поне една повтаряща се.
  Нека $R$ да бъде една такава променлива.
  Последните две повтаряния на $R$ разделят думата $\alpha$ на пет части.
  Нека $\alpha = xyuvw$.
  \begin{enumerate}[1)]
  \item
    $\abs{yv}\geq 1$,
    защото ако допуснем, че $\abs{yv} = 0$,
    то ще достигнем до противоречие с минималността на $T$.
  \item
    $\abs{yuv} \leq p$, защото сме избрали най-долното $R$.
  \item
    $xy^iuv^iw \in L$, защото можем да заменим поддървото 
    с корен последното $R$ за поддървото с корен предпоследното $R$.
    В случая $i = 0$, правим обратното.
  \end{enumerate}
\end{proof}

\begin{cor}
  \marginpar{\writedown Докажете!}
  Нека $G$ е безконтекстна граматика и $p$ е константата на покачването за $G$, $L = \L(G)$.
  Тогава $\abs{L} = \infty$ точно тогава, когато съществува $\alpha \in L$, за която $p \leq \abs{\alpha} < 2p$.
\end{cor}
% \begin{proof}
%   Ако съществува дума $\alpha \in L$, за която $\abs{\alpha} \geq p$, то от \Lem{pumping-context} следва,
%   че $\abs{L} = \infty$, защото $\alpha = xyuvw$ и $xy^iuv^iw \in L$, за всяко $i\in\Nat$.

%   За другата посока, нека сега $\abs{L} = \infty$.
%   Да изберем най-късата дума $\alpha \in L$, за която $\abs{\alpha} \geq p$.
%   Ще докажем, че $p \leq \abs{\alpha} < 2p$. За целта да допуснем, че $\abs{\alpha} \geq 2p$.
%   Тогава от \Lem{pumping-context} следва, че $\alpha = xyuvw$, $\abs{yv} \geq 1$, $\abs{yuv} \leq p$, $xy^0uv^0w = xuw \in L$.
%   Ако $\abs{xuw} < p$, то $\abs{yv} > p$, защото $\abs{yv} + \abs{xuw} = \abs{\alpha} \geq 2p$, и следователно $\abs{yuv} > p$, което е противоречие.
%   Следва, че $\abs{\alpha} > \abs{xuw} \geq p$.
%   Получихме, че думата $xuw\in L$ и $\abs{xuw} \geq p$. Това е противоречие с минималността на $\alpha$.
% \end{proof}

\begin{framed}
  \Lem{pumping-context} е полезна, когато искаме да докажем, че даден език $L$ {\bf не} е безконтекстен.
  За целта, доказваме отрицанието на \Lem{pumping-context} за $L$, т.е.
  за всяка константа $p$, ние намираме дума $\alpha \in L$, $\abs{\alpha}\geq p$, такава че за всяко разбиване на думата на пет части, $\alpha = xyuvw$,
  със свойствата $\abs{yv} \geq 1$ и $\abs{yuv} \leq p$, е изпълнено, че $(\exists i)[xy^iuv^iw \not\in L]$.
\end{framed}

\begin{example}
  \label{example:anbncn}
  Езикът $L = \{a^nb^nc^n\ \mid\ n\in\Nat\}$ не е безконтекстен.
\end{example}
\begin{proof}
  \begin{itemize}
  \item 
    Разглеждаме произволна константа $p \geq 1$.
  \item
    Избираме дума $\alpha \in L$, $\abs{\alpha} \geq p$.
    В случая, нека $\alpha = a^pb^pc^p$.
  \item
    Разглеждаме произволно разбиване $xyuvw = \alpha$, за което $\abs{xyv} \leq p$ и $1 \leq \abs{yv}$.
  \item
    Трябва да изберем $i$, за което $xy^iuv^iw \not\in L$.
    Знаем, че поне едно от $y$ и $v$ не е празната дума.
    Имаме няколко случая за $y$ и $v$.
    \begin{itemize}
    \item
      $y$ и $v$ са думи съставени от една буква.
      В този случай получаваме, че $xy^2uv^2w$ има различен брой букви $a$, $b$ и $c$.
    \item
      $y$ или $v$ е съставена от две букви.
      Тогава е възможно да се окаже, че $xy^2uv^2w$ да има равен брой $a$, $b$ и $c$,
      но тогава редът на буквите е нарушен.
    \item
      понеже $\abs{yuv} \leq p$, то не е възможно в $y$ или $v$ да се срещат и трите букви.
    \end{itemize}  
    Оказа се, че във всички възможни случаи за $y$ и $v$, 
    $xy^2uv^2w \not\in L$.
  \end{itemize}
  Следователно, езикът $L$ не е безконтекстен.
\end{proof}

% \begin{problem}
%   Да се даде пример за език $L$, който {\bf не} е безконтекстен, но удовлетворява
%   лемата за разрастването.
% \end{problem}

\begin{example}
  Приложете лемата за разрастването за да докажете, че
  езикът $L$ не е безконтекстен, където:
  \begin{enumerate}[a)]
  \item
    $L = \{a^ib^jc^k\ \mid\ 0 \leq i \leq j \leq k\}$;
  \item
    $L = \{\beta\beta\mid \beta\in \{a,b\}^\star\}$;
  \item
    $L = \{a^{n^2}\mid n\in\Nat\}$.
  \end{enumerate}
\end{example}
\begin{proof}
  \begin{enumerate}[a)]
  \item
    Да фиксираме думата $\alpha = a^pb^pc^p$ и да разгледаме
    едно произволно нейно разбиване, $\alpha = xyuvw$, за което
    $\abs{yuv} \leq p$ и $1 \leq \abs{yv}$.
    Знаем, че поне една от $y$ и $v$ не е празната дума.
    \begin{itemize}
    \item
      $y$ и $v$ са съставени от една буква.
      Имаме три случая.
      \begin{enumerate}[i)]
      \item
        $a$ не се среща в $y$ и $v$.
        Тогава $xy^0vu^0w$ съдържа повече $a$ от $b$ или $c$.
      \item
        $b$ не се среща в $y$ и $v$.
        Ако $a$ се среща в $y$ или $v$, тогава $xy^2uv^2w$ съдържа повече $a$ от $b$
        Ако $c$ се среща в $y$ или $v$, тогава $xy^0uv^0w$ съдържа по-малко $c$ от $b$.
      \item
        $c$ не се среща в $y$ и $v$.
        Тогава $xy^2uv^2w$ съдържа повече $a$ или $b$ от $c$.
      \end{enumerate}      
    \item
      $y$ или $v$ е съставена от две букви.
      Тук разглеждаме $xy^2uv^2w$ и съобразяваме, че редът на буквите е нарушен.
    \end{itemize}
  \item
    \marginpar{Защо $\alpha = a^pba^pb$ не е добър кандидат?}
    Разгледайте $\alpha = a^pb^pa^pb^p$, т.е. $\beta = a^pb^p$ и $\alpha = \beta\beta$.
    Нека $xyuvw = \alpha$ е произволно разбиване на $\alpha$, за което е изпълнено, че
    $\abs{yuv} \leq p$ и $1\leq \abs{yv}$.
    \begin{itemize}
    \item
      Ако $yuv$ е в първата част на думата, то 
      $xy^0uv^0w = a^ib^ja^pb^p \not\in L$.
      Аналогично ако $yuv$ е във втората част на думата.
    \item
      Ако $yuv$ е в двете части на думата, то 
      $xy^0uv^0w = a^pb^ia^jb^p \not\in L$.
    \end{itemize}    
  \item
    Решава се аналогично както за регулярни езици.
  \end{enumerate}
\end{proof}


\begin{thm}
  Безконтекстните езици {\bf не} са затворени относно сечение и допълнение.
\end{thm}
\begin{proof}
  Да разгледаме езика
  \[L_0 = \{a^nb^nc^n\mid n\in\Nat\},\] за който вече знаем от Пример \ref{example:anbncn}, че не е безконтекстен.
  Да вземем също така и безконтекстните езици 
  \marginpar{\writedown Защо са безконтекстни?}
  \[L_1 = \{a^nb^nc^m\mid n,m\in\Nat\},\ L_2 = \{a^mb^nc^n\mid n,m\in\Nat\},\]
  \begin{itemize}
  \item 
    Понеже $L_0 = L_1\cap L_2$, то заключаваме, че безконтекстните езици не са затворени 
    относно операцията сечение.
  \item
    \marginpar{Озн. $\ov{L} = \Sigma^\star \setminus L$}
    Да допуснем, че безконтекстните езици са затворени относно операцията допълнение.
    Тогава  $\ov{L}_1$ и $\ov{L}_2$ са безконтекстни.
    Знаем, че безконтекстните езици са затворени относно обединение. 
    Следователно, езикът $L_3 = \ov{L}_1 \cup \ov{L}_2$ също е безконтекстен.
    Ние допуснахме, че безконтекстните са затворени относно допълнение, следователно $\ov{L}_3$
    също е безконтекстен.
    Но тогава получаваме, че езикът
    \[L_0 = L_1 \cap L_2 = \ov{\ov{L}_1 \cup \ov{L}_2} = \ov{L}_3\]
    е безконтекстен, което е противоречие.
  \end{itemize}
\end{proof}


\section{Алгоритми}

\subsection{Премахване на безполезните променливи}

Нека е дадена безконтекстната граматика $G = \CFG$.
Една променлива $A$ се нарича {\bf полезна}, ако съществува извод от следния вид:
\[S \to^\star \alpha A \beta \to^\star \gamma,\]
където $\gamma \in \Sigma^\star$, а $\alpha,\beta \in (V \cup \Sigma)^\star$.
Една променлива се нарича {\bf безполезна}, ако не е полезна.

\begin{lemma}
  \marginpar{\cite{hopcroft1} стр. 88}
  Нека е дадена безконтекстната граматика $G = \CFG$ и $L(G) \neq \emptyset$.
  Съществува алгоритъм, който намира граматика $G' = \pair{V',\Sigma,S,R'}$, за която 
  $\L(G) = \L(G')$, и за всяка променлива $A' \in V'$, съществува дума $\alpha \in \Sigma^\star$,
  за която $A' \to^\star \alpha$. С други думи, във $V'$ няма безполезни променливи.
\end{lemma}
\begin{proof}
  Строим множествата $V_i$, докато не стигнем стъпка $i+1$, за която $V_i = V_{i+1}$.
  Тръгваме от множеството
  \[V_0 = \{A \in V \mid (\exists a\in \Sigma)[A \to a]\}.\]
  Нека сме построили $V_i$. Тогава:
  \[V_{i+1} := V_i \cup \{A \in V \mid (\exists \alpha \in (\Sigma \cup V_i)^\star)[A \to_G \alpha]\}\]
\end{proof}


\subsection{Нормална Форма на Чомски}

\begin{dfn}
%[стр. 99 от \cite{sipser}]
\index{Нормална форма на Чомски}
Една безконтекстна граматика е в {\em нормална форма на Чомски}, ако
всяко правило е от вида
\[A \rightarrow BC\mbox{ и }A \rightarrow a,\]
като $B, C$ {\em не могат} да бъдат променливата за начало $S$.
Освен това, позволяваме правилото $S\to\varepsilon$.
\footnote{На стр. 151 в \cite{papadimitriou} дефиницията е малко по-различна.
Там дефинират $G$ да бъде в нормална форма на Чомски ако $R \subseteq V\times(V\cup\Sigma)^2$.
В този случай губим езиците $\{\varepsilon\}$ и $\{a\}$, за $a\in\Sigma$.}
\end{dfn}

\begin{thm}
  Всеки безконтекстен език $L$ е генериран от контекстно-свободна
  граматика в нормална форма на Чомски.
\end{thm}
\begin{proof}
%  \marginpar{Броят на правилата може да се увеличи експоненциално.}
  Нека имаме контекстно-свободна граматика $G$, за която $L = L(G)$.
  Ще построим контекстно-свободна граматика $G^\prime$ в нормална форма на Чомски, $L = L(G^\prime)$.
  % [стр. 99 от \cite{sipser}]
  Следваме следната процедура:
  \begin{itemize}
  \item
    Добавяме нов начален символ $S_0$ и правило $S_0 \to S$.
  \item
    \marginpar{Сложност $O(n)$}
    Съкращаваме дължината на правилата.
    Заменяме правилата от вида $A\to u_1\dots u_n$, $n\geq 3$, $u_i \in V\cup\Sigma$, с
    правилата \[A\to u_1A_1,\ A_1\to u_2A_2,\ \dots,\ A_{n-2} \to u_{n-1}u_n.\]
    където $A_i$ са нови променливи.
  \item
    \marginpar{Сложност $O(n)$}
    За всяка променлива $A \neq S_0$ премахваме правилата от вида $A\to\varepsilon$.
    Това правим по следния начин.
    
    Ако имаме правило от вида $R \to Au$ или $R\to u A$, $u \in V \cup \Sigma$,
    то добавяме правилото $R\to u$.
    %Правим това за всяко срещане на променливата $A$ в дясната страна на правило.
    Например, 
    \begin{itemize}
    \item 
      ако имаме правило $R\to aA$, то добавяме правилото $R \to a$;
    \item
      ако имаме правило $R\to AA$, то добавяме правилото $R \to A$.
    \end{itemize}
    Ако имаме правило от вида $R\to A$, то добавяме правилото $R\to\varepsilon$
    само ако променливата $R$ още не е преминала през процедурата за премахване на $\varepsilon$.
  \item
    \marginpar{Сложност $O(n^2)$}
    Премахваме преименуващите правила, т.е. правила от вида $A\to B$.
    Всяко правило от вида $B \to \beta$ го заменяме с $A\to \beta$,
    освен ако $A \to \beta$ е вече премахнато преименуващо правило.
  \item
    % Заменяме правилата от вида $A\to u_1\dots u_n$, $n\geq 3$, $u_i \in V\cup\Sigma$, с
    % правилата \[A\to u_1A_1,\ A_1\to u_2A_2,\ \dots,\ A_{n-2} \to u_{n-1}u_n.\]
    % където $A_i$ са нови променливи.
    За правила от вида $A\to u_1 u_2$, където $u_1, u_2 \in V \cup \Sigma$, 
    заменяме всяка буква $u_i$ с новата променлива $U_i$
    и добавяме правилото $U_i\to u_i$.
    Например, правилото $A \to aB$ се заменя с правилото $A \to XB$ и добавяме правилото $X \to a$,
    където $X$ е нова променлива.
  \end{itemize}
\end{proof}

\begin{thm}
  При дадена безконтекстна граматика $G$ с дължина $n$, можем да намерим еквивалентна
  на нея граматика $G'$ в нормална форма на Чомски за време $O(n^2)$,
  като получената граматика е с дължина $O(n^2)$.
\end{thm}


\begin{problem}
  Нека е дадена граматиката  $G = \pair{\{S,A,B,C,D,E\}, \{a,b\},S, R}$.
  \begin{enumerate}[a)]
  \item
    Намерете множеството $\{X \in V \mid X \rightarrow^\star_G \varepsilon\}$.
  \item
    Вярно ли е, че $\varepsilon \in L(G)$?
  \item
    Постройте граматика $G_1$ без $\varepsilon$-правила, за която $L(G_1)=L(G)\setminus\{\varepsilon\}$.
  \end{enumerate}
  Множеството от правила $R$ на граматиката $G$ е зададено като:
  \begin{enumerate}[a)]
  \item
    $R = \{S\rightarrow D,D\rightarrow AD|b,A\rightarrow ACB|BC|a, B\rightarrow ABCA|CEC,C\rightarrow \varepsilon|CA|a, E\rightarrow \varepsilon|aEb\}$;
  \item
    $R = \{S \rightarrow aD, D\rightarrow \varepsilon|ABBA|ADD,A\rightarrow DEB|a,B\rightarrow DDD|DC|b,C\rightarrow CCE|a, E\rightarrow \varepsilon|bEa\}$;
  \item
    $R = \{ S\rightarrow D,D\rightarrow AD|b,A\rightarrow AB|BC|a, B\rightarrow AB|CC, C\rightarrow \varepsilon|CA|a, E\rightarrow a|EB\}$;
  \item
    $R = \{ S \rightarrow AD|a, D\rightarrow \varepsilon|BB|AD,A\rightarrow DB|a,B\rightarrow DD|DC|b,C\rightarrow CE|a, E\rightarrow AB|b|EA\}$;
  \item
    $R =\{S\rightarrow AS|SB|SS,B\rightarrow CA|b, C\rightarrow AA|a|BA,A\rightarrow \varepsilon|BS\}$;
  % \item
  %   $R = \{S\rightarrow AB|AC,B\rightarrow \varepsilon |BC|b,A\rightarrow BB|CC|a,C\rightarrow CS|a\}$;
  % \item
  %   $R = \{S\rightarrow AS|SB|SS,B\rightarrow AC|b, C\rightarrow A|a|AB,A\rightarrow \varepsilon|BS\}$;
  \item
    $R = \{S\rightarrow BA|CA,B\rightarrow \varepsilon |BC|b,A\rightarrow BB|CC|a, C\rightarrow CS|a\}$;
  \item
    $R = \{S\rightarrow AS|b,A\rightarrow AC|BC|a, B\rightarrow BC|CC,C\rightarrow \varepsilon|CA|a\}$;
  \item
    $R = \{S\rightarrow \varepsilon|BA|AS,A\rightarrow SB|a,B\rightarrow SS|SC|b,
    C\rightarrow CC|a\}$; 
  \end{enumerate}
\end{problem}

\begin{problem}
  Нека е дадена граматиката  $G = \pair{\{S,A,B,C\}, \{a,b\}, S, R}$.
  Използвайте обща конструкция, за да премахнете ,,дългите'' правила 
  (т.е. правила с дължина поне 2, които не са в н.ф. на Чомски) от $ G$ като при това получите 
  безконтестна граматика $G_1$ с език $L(G)=L(G_1)$, където:
  \begin{enumerate}[a)]
  \item
    $R = \{S \rightarrow \varepsilon|ab|aAba, A\rightarrow aBCb, B\rightarrow bbb, C\rightarrow aC\vert aCaC\}\rangle$;
  \item
    $R = \{S \rightarrow \varepsilon|ab|baAb, A\rightarrow BaBb,B\rightarrow b,C\rightarrow AbA\vert aCCa\}$;
  \item
    $R = \{A\rightarrow BSB|a,B\rightarrow ba|BC,C\rightarrow BaSA|a|b,S\rightarrow CC|b\}$;
  \item
    $R = \{A\rightarrow BAS,B\rightarrow CB,C\rightarrow ab|ABbS,S\rightarrow CC|b\}$;
  \end{enumerate}
\end{problem}

\begin{problem}
  Използвайте обща конструкция, за да премахнете преименуващите правила от граматиката $G$ като при това запазите езика,
  където $G = \pair{\{A,B,C,S\},\{a,b\}, S, R}$ и
  \begin{enumerate}[a)]
  \item
    $R = \{A\rightarrow B|S,B\rightarrow C|BC,C\rightarrow AB|a|b,S\rightarrow B|CC|b\}$;
  \item
    $R = \{A\rightarrow B,B\rightarrow S|C|BC,C\rightarrow a|AB,S\rightarrow C|CC|b\}$;
  \item
    $R = \{A\rightarrow B|CC|a,B\rightarrow S|AB,C\rightarrow SC|b,S\rightarrow A|CC|b\}$;
  \item
    $R = \{A\rightarrow BB|b,B\rightarrow S|SS|b,C\rightarrow B|a,S\rightarrow C|AB|a\}$;
  \item
    $R = \{S\rightarrow A|a,A\rightarrow B|C|b, B\rightarrow AB, C\rightarrow CC|a\}$;
  \item
    $R = \{S\rightarrow A|B, A\rightarrow a|C|AB, B\rightarrow b|C, C\rightarrow CS|a|b\}$;
  \end{enumerate}
\end{problem}

\begin{problem}
  Намерете безконтекстна граматика в нормална форма на Чомски за езиците от задача 6.
\end{problem}


\subsection{Проблемът за принадлежност}

\begin{thm}
  Съществува {\em полиномиален} алгоритъм , който проверява дали дадена дума принадлежни на граматиката $G$.
  \marginpar{За дума $\alpha$, алгоритъмът работи за време $O(\abs{\alpha}^3)$}
\end{thm}
% \begin{proof}[стр. 154 от \cite{papadimitriou}]
Можем да приемем, че $G = \CFG$ е граматика в нормална форма на Чомски.
Нека $\alpha = a_1a_2\dots a_n$ е дума, за която искаме да проверим дали $\alpha \in L(G)$.
\marginpar{Това е алгоритъм на Cocke, Younger и Kasami (CYK), който е пример за динамично програмиране (стр. 195 от \cite{kozen})}
\begin{algorithm}[H]
  \caption{Проверка за $\alpha \in L(G)$}
  \label{alg:belongs-to-grammar}
  \begin{algorithmic}[1]
    \State $n := \abs{\alpha}$ \Comment{Вход дума $\alpha = a_1\cdots a_n$}
    \ForAll{$i\in [1,n]$}
    \State $V[i,i] = \{A \in V \mid A\rightarrow a_i\}$
    \EndFor
    \ForAll{$i,j \in [1,n]\ \&\ i \neq j$}
    \State $V[i,j] = \emptyset$
    \EndFor      
    \ForAll{$s \in [1, n)$} \Comment{Дължина на интервала}
    \ForAll{$i \in [1, n-s]$}\Comment{Начало на интервала}
    \ForAll{$k \in [i, i + s)$}\Comment{Разделяне на интервала}
    \If{$\exists A\to BC \in R\ \&\ B \in V[i,k]\ \&\ C\in V[k+1,i+s]$}
    \State $V[i,i+s] := V[i,i+s] \cup \{A\}$
    \EndIf
    \EndFor
    \EndFor
    \EndFor
    \If{$S \in V[1,n]$}
    \State \Return \texttt{True}\Comment{Има извод на думата от $S$}
    \Else
    \State \Return \texttt{False}
    \EndIf
  \end{algorithmic}
\end{algorithm}

\begin{lemma}
  За дадена граматика в нормална форма на Чомски и дума $\alpha$, 
  за всяко $0 \leq s < \abs{\alpha}$, след $s$-тата итерация на алгоритъма (редове 6 - 10), за всяка позиция $i = 1,\dots,n-s$,
  \[V[i,i+s] = \{A \in V \mid A \rightarrow^\star_G a_i\dots a_{i+s}\}.\]
\end{lemma}
\begin{proof}
  Пълна индукция по $s$.
  За $s = 0$  е ясно. (Защо?)

  Нека твърдението е вярно за $s < n$. Ще докажем твърдението за $s+1$, т.е. за всяко $i = 1,\dots,n-s-1$,
  \[V[i,i+s+1] = \{A \in V \mid A \rightarrow^\star_G a_i\dots a_{i+s+1}\}.\]
  % Да разгледаме $A \in V[i,i+s+1]$.
  За едната посока, да разгледаме първoто правило в извода $A \to^\star_G a_i\cdots a_{i+s+1}$.
  Понеже $G$ е в НФЧ, то е от вида $A \to BC$ и тогава съществува някое $t$, за което 
  $B \to^\star a_i\cdots a_{i+t}$ и $C \to^\star a_{i+t+1}\cdots a_{i+s+1}$.
  От И.П. получаваме, че $B \in V[i,i+t]$ и $C \in V[i+t+1,i+s+1]$.
  Тогава от ред 10 на алгоритъма е ясно, че $A \in V[i,i+s+1]$.
  
  За другата посока, нека $A \in V[i,i+s+1]$.
  Единствената стъпка на алгоритъма, при която може да сме добавили $A$ към множеството $V[i,i+s+1]$ е ред 10.
  Тогава имаме, че съществува $k$, за което $B \in V[i,k]$, $C \in V[k+1,i+s+1]$, и $A\to BC$ е правило в граматиката $G$.
  От И.П. имаме, че $B \to^\star_G a_i\cdots a_k$ и $C \to^\star_G a_{k+1}\cdots a_{i+s+1}$.
  Заключаваме веднага, че $A \to^\star_G a_i\cdots a_{i+s+1}$.
\end{proof}

\begin{problem}
  Нека е дадена граматиката $G = \pair{\{a,b\}, \{S,A,B,C\},S,R}$.
  Използвайте CYK-алгоритъма, за да проверите дали
  думата $\alpha$ принадлежи на $L(G)$, където правилата на граматиката $R$ и думата $\alpha$
  са зададени като:
  \begin{enumerate}[a)]
  \item
    $R =\{S\rightarrow a| AB|AC, C\rightarrow SB|AS,A\rightarrow a, B\rightarrow b\}$, $\alpha=aaabb$;
  \item
    $R = \{S\rightarrow BA| CA|a, C\rightarrow BS|SA,A\rightarrow a, B\rightarrow b\}$, $\alpha=bbaaa$;
  \item
    $R =\{S\rightarrow AB|BC, A\rightarrow BA|a,B\rightarrow CC|b, C\rightarrow AB|a\}$, $\alpha=baaba$;
  \item
    $R = \{S\rightarrow AB, A\rightarrow AC|a|b,B\rightarrow CB|a, C\rightarrow a\}$, $\alpha=babaa$;
  % \item
  %   $R = \{S\rightarrow BA|SS|b, A\rightarrow SA|a,B\rightarrow BS|b\}$, $\alpha = bbbaa$;
  % \item
  %   $R = \{S\rightarrow AB| BS|b, A\rightarrow SS|a,B\rightarrow BA|b\}$, $\alpha = babab$;
  % \item
  %   $R = \{S\rightarrow BA| AS|a, A\rightarrow AB|a,B\rightarrow SS|b\}$, $\alpha = ababa$;
  % \item
  %   $R = \{S\rightarrow AB|a, A\rightarrow BA|SS|a,B\rightarrow SS|b\}$, $\alpha = aabba$.
  \end{enumerate}
\end{problem}

\begin{thm}
  \marginpar{\cite{hopcroft1}, стр. 137}
  Съществуват алгоритми, които определят по дадена безконтекстна граматика $G$ дали:
  \begin{enumerate}[a)]
  \item 
    $\abs{\L(G)} = 0$;
  \item
    $\abs{\L(G)} < \infty$;
  \item
    $\abs{\L(G)} = \infty$.
  \end{enumerate}
\end{thm}
\begin{proof}
  Нека е дадена една безконтекстна граматика $G$.
  \begin{description}
  \item[($\L(G) = \emptyset?$)]
    Прилагаме алгоритъма за премахване на безполезните променливи.
    Ако открием, че $S$ е безполезна променлива, то $\L(G) = \emptyset$.
  \item[($\abs{\L(G)} < \infty?$ или $\abs{\L(G)} = \infty?$)]
    Нека да разгледаме граматиката $G'$ в НФЧ без безполезни променливи, за която $\L(G) = \L(G')$.
    От граматиката $G' = \pair{V',\Sigma,S,R'}$ строим граф с възли променливите от $V'$ като
    за $A,B \in V'$ имаме ребро $A \to B$ точно тогава, когато съществува $C \in V'$,
    за което $A \to BC$ или $A \to CB$ е правило в $R'$.
    
    Ако в получения граф имаме цикъл, то $\L(G') = \infty$.
  \end{description}
\end{proof}

\section{Недетерминирани стекови автомати}

\index{автомат!недетерминиран стеков}
\marginpar{На англ. {\bf Push-down automaton} (стр. 157 от \cite{kozen})}
%Sipser p.102
\begin{dfn}
  Недетерминиран стеков автомат е 7-орка от вида
  \[P = \PDA,\] където:
  \begin{itemize}
  \item
    $Q$ е крайно множество от състояния;
  \item  
    $\Sigma$ е крайна входна азбука;
  \item
    $\Gamma$ е крайна стекова азбука;
  \item
    $\# \in \Gamma$ е символ за дъно на стека;
  \item
    $s\in Q$ е начално състояние;
  \item
    \marginpar{Озн. $\Ps_{fin}(A)$ - крайните подмножества на $A$}
    $\Delta:Q\times(\Sigma \cup \{\varepsilon\})\times\Gamma\rightarrow \Ps_{fin}(Q\times\Gamma^\star)$ 
    е функция на преходите;    
  \item
    $F\subseteq Q$ е множество от заключителни състояния.
  \end{itemize}
\end{dfn}

\marginpar{Instanteneous description}
{\em Моментно описание} (или конфигурация) на изчислението със стеков автомат представлява тройка от вида $(q,\alpha,\gamma) \in Q\times\Sigma^\star\times\Gamma^\star$,
т.е. автоматът се намира в състояние $q$, думата, която остава да се прочете е $\alpha$,
а съдържанието на стека е думата $\gamma$.
Удобно е да въведем бинарната релация $\vdash_P$ над $Q\times\Sigma^\star\times\Gamma^\star$,
която ще ни казва как моментното описание на автомата $P$ се променя след изпълнение на една стъпка:
\[(q,x\alpha,Y\gamma) \vdash_P (p,\alpha,\beta\gamma), \text{ ако } \Delta(q,x,Y) \ni (p,\beta),\]
\[(q,\alpha,Y\gamma) \vdash_P (p,\alpha,\beta\gamma), \text{ ако } \Delta(q,\varepsilon,Y) \ni (p,\beta).\]
Рефлексивното и транзитивно затваряне на $\vdash_P$ ще означаваме с $\vdash^\star_P$.
Сега вече можем да дадем дефиниция на език, разпознаван от стеков автомат $P$.
\begin{itemize}
\item
  $\L_F(P)$ е езика, който се разпознава от $P$ {\bfс финално състояние},
  \[\L_F(P) = \{\omega \mid (q_0,\omega,\#) \vdash^\star_P (q,\varepsilon,\alpha)\ \&\ q \in F\}.\]    
\item
  $\L_S(P)$ е езика, който се разпознава от $P$  {\bf с празен стек},
  \[\L_S(P) = \{w\mid (q_0,w,\#) \vdash^\star_P (q,\varepsilon,\varepsilon)\}.\]    
\end{itemize}

\begin{example}
  \label{ex:anbn}
  За езика $L = \{a^nb^n\mid n\in\Nat\}$ съществува стеков автомат $P$, такъв че
  $L = \L_S(P)$.
  Да разгледаме $P = \PDA$, където
  \begin{itemize}
  \item
    $Q = \{q\}$;
  \item
    $\Sigma = \{a,b\}$;
  \item
    $\Gamma = \{\#,A\}$;
  \item
    $F = \emptyset$;
  \item 
    $\Delta(q,a,\#) = \{(q, A\#)\}$;
  \item 
    $\Delta(q,\varepsilon,\#) = \{(q,\varepsilon)\}$;
  \item 
    $\Delta(q,b,A) = \{(q,\varepsilon)\}$.
  \end{itemize}
  Вместо доказтелство, да видим как думата $a^2b^2$ се разпознава от автомата с празен стек:
  \marginpar{\writedown Докажете, че $L = \L_S(P)$!}
  \begin{align*}
    (q,a^2b^2,\#) & \vdash_P (q,ab^2,A\#) \\
    & \vdash_P (q,b^2, AA\#)\\
    & \vdash_P (q,b,A\#)\\
    & \vdash_P (q,\varepsilon,\#)\\
    & \vdash_P (q,\varepsilon,\varepsilon).
  \end{align*}
\end{example}

\begin{example}
  За езика $L = \{\omega\omega^R \mid \omega \in \{a,b\}^\star\}$ съществува стеков автомат $P$, такъв че
  $L = \L_S(P)$.
  Нека $P = \PDA$, където:
  \begin{itemize}
  \item 
    $\Delta(q, a, \#) = \{(q, A\#)\}$;
  \item 
    $\Delta(q, b, \#) = \{(q, B\#)\}$;
  \item
    $\Delta(q, a, A) = \{(q, AA), (p, \varepsilon)\}$;
  \item
    $\Delta(q, a, B) = \{(q, AB)\}$;
  \item
    $\Delta(q, b, B) = \{(q, BB), (p, \varepsilon)\}$;
  \item
    $\Delta(q, b, A) = \{(q, BA)\}$;
  \item
    $\Delta(p, a, A) = \{(p,\varepsilon)\}$;
  \item
    $\Delta(p, b, B) = \{(p,\varepsilon)\}$;
  \item
    $\Delta(q, \varepsilon, \#) = \{(q,\varepsilon)\}$;
  \item
    $\Delta(p, \varepsilon, \#) = \{(p,\varepsilon)\}$;
  \end{itemize}
  Основното наблюдение, което трябва да направим за да разберем конструкцията на автомата е, че
  всяка дума от вида $\omega\omega^R$ може да се запише като $\omega_1aa\omega^R_1$ или $\omega_1bb\omega^R_1$.
  Да видим защо $P$ разпознава думата $abaaba$.
  Започваме по следния начин:
  \begin{align*}
    (q,abaaba,\#) & \vdash_P (q,baaba,A\#)\\
    & \vdash_P (q, aaba, BA\#) \\
    & \vdash_P (q, aba, ABA\#).
  \end{align*}
  Сега можем да направим два избора как да продължим. Състоянието $p$ служи за маркер, което ни казва, че вече сме започнали 
  да четем $\omega^R$. Поради тази причина, продължаваме така:
  \begin{align*}
    (q, aba, ABA\#) & \vdash_P (p, ba, BA\#)\\
    & \vdash_P (p, a, A\#)\\
    & \vdash_P (p, \varepsilon, \#) \\
    & \vdash_P (p,\varepsilon,\varepsilon).
  \end{align*}
  Да проиграем още един пример. Да видим защо думата $aba$ не се извежда от автомата.
  \begin{align*}
    (q,aba,\#) & \vdash_P (q, ba,A\#)\\
    & \vdash_P (q, a, BA\#)\\
    & \vdash_P (q, \varepsilon, ABA\#).
  \end{align*}
  От последното моментно описание на автомата нямаме нито един преход, следователно
  думата $aba$ не се разпознава от $P$ с празен стек.
\end{example}


\begin{thm}
  \marginpar{(\cite{hopcroft1}, стр. 114) }
  Нека $L$ е произволен език над азбука $\Sigma$.
  \begin{enumerate}[1)]
  \item 
    Ако съществува НСА $P$, за който $L = \L_F(P)$, то съществува НСА $P^\prime$, за който $L = \L_S(P^\prime)$.
  \item
    Ако съществува НСА $P$, за който $L = \L_S(P)$, то съществува НСА $P^\prime$, за който $L = \L_F(P^\prime)$.
  \end{enumerate}
  С други думи, езиците разпознавани от НСА с празен стек са точно езиците разпознавани от НСА с финално състояние.
\end{thm}
\begin{proof}
  \begin{enumerate}[1)]
  \item 
    Нека $L = \L_F(P)$, където $P = \PDA$.
    Ще построим $P^\prime$, така че да симулира $P$ и като отидем във финално състояние ще изпразним стека.
    Нека
    \[P^\prime = \langle{Q\cup\{q_e,s^\prime\},\Sigma,\Gamma \cup \{\$\},\$,s^\prime,\Delta^\prime,\emptyset}\rangle\]
    Важно е $P^\prime$ да има собствен нов символ за дъно на стека, защото е възможно за някоя дума $\alpha \not\in \L_F(P)$
    стековият автомат $P$ да си изчисти стека и така да разпознаем повече думи.
    \begin{itemize}
    \item 
      \marginpar{- започваме симулацията}
      $\Delta^\prime(s^\prime,\varepsilon,\$) = \{(s,\#\$)\}$;
    \item
      \marginpar{- симулираме $P$}
      $\Delta^\prime(q,a,X)$ включва множеството $\Delta(q,a,X)$, за всяко $q\in Q$, $a\in\Sigma_\varepsilon$, $X\in\Gamma$;
    \item
      \marginpar{- ако сме във финално, започваме да чистим стека}
      $\Delta^\prime(q,\varepsilon,X)$ съдържа също и елемента $(q_e,\varepsilon)$, за всяко $q\in F$, $X \in \Gamma \cup \{\$\}$;
    \item
      \marginpar{- изчистваме стека}
      $\Delta^\prime(q_e,\varepsilon,X) = \{(q_e,\varepsilon)\}$, за всяко $X \in \Gamma \cup \{\$\}$.
    \end{itemize}
  \item
    Сега имаме $L = \L_S(P)$, където $P = \langle{Q,\Sigma,\Gamma,\#,s,\Delta,\emptyset}\rangle$. 
    Да положим
    \[P^\prime = \langle{Q\cup\{s^\prime,q_f\}, \Sigma, \Gamma \cup \{\$\}, \Delta^\prime, \$, \{q_f\}}\rangle.\]
    $P^\prime$ ще симулира $P$ като ще внимаваме кога $P$ изчиства символа $\#$. Тогава ще искаме да отидем във финалното състояние $q_f$.
    \begin{itemize}
    \item 
      \marginpar{- започваме симулацията}
      $\Delta^\prime(s^\prime,\varepsilon,\$) = \{(s, \#\$)\}$;
    \item
      \marginpar{- симулираме $P$}
      $\Delta^\prime(q,a,X) = \Delta(q,a,X)$, за всяко $q \in Q$, $a \in \Sigma_\varepsilon$, $X \in \Gamma$;
    \item
      \marginpar{- щом сме стигнали до $\$$, значи $P$ е изчистил стека си}
      $\Delta^\prime(q,\varepsilon,\$) = \{(q_f,\varepsilon)\}$.
    \end{itemize}
  \end{enumerate}
\end{proof}

\begin{problem}
  Като използвате стековия автомат от Пример \ref{ex:anbn}, дефинирайте автомат $P'$, за който
  $\L_F(P') = \{a^nb^n \mid n\in\Nat\}$.
\end{problem}

\begin{framed}
\begin{thm}
  Класът на езиците, които се разпознават от краен стеков автомат съвпада с
  класа на безконтекстните езици.
\end{thm}
\end{framed}
%\marginpar{(\cite{hopcroft1}, стр. 117)}
\begin{proof}
  \begin{enumerate}[1)]
  \item 
    Нека е дадена безконтекстна граматика $G = \CFG$.
    Нашата цел е да построим стеков автомат $P$, така че $\L_S(P) = \L(G)$.
    Нека  \[P = \langle{\{q\},\Sigma,\Sigma\cup V,S,q,\Delta,\emptyset}\rangle,\]
    където функцията на преходите е:
    \begin{align*}
      & \Delta(q,\varepsilon,A) = \{(q,\alpha)\mid A\to\alpha\mbox{ е правило в граматиката }G\}\\
      & \Delta(q,a,a) = \{(q,\varepsilon)\}
    \end{align*}
  \item
    Нека имаме $P = \langle{Q, \Sigma, \Gamma, \Delta, s, \#, \emptyset}\rangle$.
    Ще дефинираме безконтекстна граматика $G$, за която $\L_S(P) = \L(G)$.
    Променливите на граматика са 
    \[V = \{[q,A,p] \mid q,p \in Q, A \in \Gamma\}.\]
    Правилата на $G$ са следните:
    \begin{itemize}
    \item
      $S \to [s,\#,q]$, за всяко $q \in Q$;
    \item
      $[q,A,q_{m+1}] \to a[q_1,B_1,q_2][q_2,B_2,q_3]\dots [q_m,B_m,q_{m+1}]$,
      където 
      \[(q_1,B_1\dots B_m) \in \Delta(q, a, A)\]
      и произволни $q,q_1,\dots,q_{m+1} \in Q$,
      $a \in \Sigma_\varepsilon$.
    \item
      Ако $m = 0$, т.е. $(q_1,\varepsilon) \in \Delta(q, a, A)$,  то имаме правилото $[q,A,q_{1}] \to a$, където $a \in \Sigma_\varepsilon$.
    \end{itemize}
    Трябва да докажем, че:
    \[[q,A,p] \rightarrow^\star_G \alpha\ \iff\ (q,\alpha,A) \vdash^\star_P (p,\varepsilon,\varepsilon).\]
    \begin{description}
    \item[$(\Rightarrow)$]
      С пълна индукция по $i$, ще докажем, че 
      \[(q,\alpha,A) \vdash^i_P (p,\varepsilon,\varepsilon)\ \implies\ [q,A,p] \Rightarrow^\star_G \alpha.\]
      Ако $i = 1$, то е лесно, защото $\alpha = a$ или $\alpha = \varepsilon$, в случай, че $m = 0$.
      Ако $i > 1$, нека $\alpha = a\beta$. Тогава:
      \[(q,a\beta,A) \vdash_P (q_1,\beta,B_1\dots B_n) \vdash^{i-1}_P (p, \varepsilon, \varepsilon)\]
      Да разбием думата $\beta$ на $n$ части, $\beta = \beta_1\cdots \beta_n$, със свойството, че след като прочетем $\beta_i$, то 
      сме премахнали променливата $B_i$ от върха на стека. Това означава, че :
      \begin{align*}
        & (q_j, \beta_j, B_j) \vdash^{i_j}_P (q_{j+1},\varepsilon,\varepsilon), \text{ за }j = 1,\dots,n-1,\\
        & (q_n, \beta_n, B_n) \vdash^{i_n}_P (p,\varepsilon,\varepsilon),
      \end{align*}
      където $i_1+i_2+\cdots+i_n = i-1$.
      Сега по {\bf И.П.}, 
      \begin{align*}
        & (q_j, \beta_j, B_j) \vdash^{i_j}_P (q_{j+1},\varepsilon,\varepsilon) \implies [q_j,B_j, q_{j+1}] \rightarrow^\star_G \beta_j, \text{ за }за j = 1,\dots,n-1,\\
        & (q_n, \beta_n, B_n) \vdash^{i_n}_P (p,\varepsilon,\varepsilon) \implies [q_n,B_n, p] \rightarrow^\star_G \beta_n.
      \end{align*}
      Обединявайки тези изводи с правилото
      \[[q,A,p] \rightarrow_G a[q_1,B_1,q_2]\dots[q_n,B_n,p],\]
      получаваме извода
      \[[q,A,p] \rightarrow^\star_G a\beta.\]
    \item[$(\Leftarrow)$]
      Отново с пълна индукция по $i$ ще докажем, че
      \[[q,A,p] \rightarrow^i_G \alpha \implies (q,\alpha,A) \vdash^\star_P (p,\varepsilon,\varepsilon).\]
      Ако $i = 1$, то имаме $[q,A,p] \Rightarrow \alpha$, където $\alpha = a$ или $\alpha = \varepsilon$.
      Ако $i > 1$, то имаме, че $\alpha = a\beta$ и за някое $n$, 
      \[[q,A,p] \rightarrow_G a[q_1,B_1,q_2][q_2,B_2,q_3]\dots[q_n,B_n,p] \rightarrow^{i-1}_G \beta.\]
      Отново нека $\beta = \beta_1\dots \beta_n$, където 
      \begin{align*}
        & [q_j,B_j,q_{j+1}] \rightarrow^{i_j}_G \beta_j, \text{ за } j = 1,\dots,n-1,\\
        & [q_{n},B_n,p ] \rightarrow^{i_n}_G \beta_n,
      \end{align*}
      където $i_1 + i_2 + \cdots + i_n = i-1$.
      От {\bf И.П.} получаваме, че 
      \begin{align*}
        & [q_j,B_j,q_{j+1}] \rightarrow^{i_j}_G \beta_j \implies (q_j,\beta_j,B_j) \vdash^\star_P (q_{j+1},\varepsilon,\varepsilon),\ j = 1,\dots,n-1\\
        & [q_n,B_n,p] \rightarrow^{i_n}_G \beta_n \implies (q_n,\beta_n,B_n) \vdash^\star_P (p,\varepsilon,\varepsilon),
      \end{align*}
      Обединявайки всичко, което знаем, получаваме:
      \begin{align*}
        (q, a\beta, A) & \vdash_P (q_1, \beta_1\cdots\beta_n, B_1\cdots B_n)\\
        & \vdash^\star_P (q_2, \beta_{2}\cdots\beta_n, B_2\cdots B_n)\\
        & \dots\\
        & \vdash^\star_P (q_n, \beta_n, B_n)\\
        & \vdash^\star_P (p, \varepsilon, \varepsilon)
      \end{align*}
    \end{description}
  \end{enumerate}
\end{proof}

\begin{problem}
  Нека е дадена граматиката $G = \pair{\{S,A,B\},\{a,b\},S,R\}}$.
  Постройте стеков автомат $P = \PDA$, такъв че $\L_S(P) = \L(G)$, където правилата $R$ на граматиката $G$ са зададени като:
  \begin{enumerate}[a)]
    % За едно тези двете да се даде пример как става 
  \item
    $R = \{S\rightarrow ASB\vert \varepsilon, A\rightarrow aAa\vert a, B\rightarrow bBb\vert b\}$;
  \item
    $R = \{S\rightarrow ASB\vert \varepsilon, A\rightarrow aA\vert a, B\rightarrow Bb\vert b\}$;
  \item
    $R =\{S\rightarrow SA|\varepsilon,A\rightarrow BSa|B, B\rightarrow b|BS|ab\}$;
  \item
    $R = \{S\rightarrow AS|\varepsilon,A\rightarrow SaBB|A, B\rightarrow b|BBbS|AA\}$;
  \end{enumerate}
\end{problem}

\begin{thm}
  \marginpar{(стр. 144 от \cite{papadimitriou})}
  Нека $L$ e безконтекстен език и $R$ е регулярен език.
  Тогава тяхното сечение $L \cap R$ е безконтекстен език.
\end{thm}
\begin{proof}
  Нека имаме стеков автомат
  \[M_1 = \PDAn{1}, \text{ където } \L_F(M_1) = L,\]
  \marginpar{всъщност няма нужда да е детерминиран}
  и краен детерминиран автомат 
  \[M_2 = \FAn{2}, \text{ където } \L(M_2) = R.\]
  Ще определим нов стеков автомат $M = \PDA$, където
  \begin{itemize}
  \item 
    $Q = Q_1 \times Q_2$;
  \item
    $s = \pair{s_1,s_2}$;
  \item
    $F = F_1 \times F_2$;
  \item 
    Функцията на преходите $\Delta$ е дефинирана както следва:
    \begin{itemize}
    \item 
      \marginpar{симулираме едновременно изчислението и на двата автомата}
      Ако $\Delta_1(q_1, a, b) \ni \pair{r_1,c}$
      и $\delta_2(q_2,a) = r_2$, то
      \[\Delta(\pair{q_1,q_2},a,b) \ni \pair{\pair{r_1,r_2}, c}.\]
    \item
      \marginpar{празен ход на автомата $M_2$}
      Ако $\Delta_1(q_1,\varepsilon,b) \ni \pair{r_1,c}$,
      то за всяко $q_2 \in Q_2$,
      \[\Delta(\pair{q_1,q_2},\varepsilon,b) \ni \pair{\pair{r_1,q_2},c}.\]    
    \end{itemize}   
  \end{itemize}
  На практика $M$ симулира едновременно и двата автомата $M_1$ и $M_2$.
\end{proof}

\begin{example}
  Езикът $L = \{w \in \{a,b,c\}^\star \mid n_a(w) = n_b(w) = n_c(w)\}$ не е безконтекстен.
\end{example}
\begin{proof}
  Да допуснем, че $L$ е безконтекстен език.
  Тогава \[L^\prime = L \cap \L(a^\star b^\star c^\star)\] също е безконтекстен език.
  Но $L^\prime = \{a^nb^nc^n \mid n \in \Nat\}$, за който знаем от Пример \ref{example:anbncn}, че {\em не} е безконтекстен.
  Достигнахме до противоречие. Следователно, $L$ не е безконтекстни език.
\end{proof}

\section{Свойства}

Безконтекстните езици са:
\begin{itemize}
\item 
  затворени относно операцията {\em обединение}, т.е.
  ако $L_1, L_2$ са безконтекстни, то езикът $L_1 \cup L_2$ е безконтекстен; 
\item
  затворени относно операцията {\em конкатенация}, т.е.
  ако $L_1, L_2$ са безконтекстни, то езикът $L_1 \cdot L_2$ е безконтекстен; 
\item
  затворени относно операцията {\em звезда} на Клини, т.е.
  ако $L$ е безконтекстен, то езикът $L^\star = \bigcup_{n\in\Nat} L^n$ е безконтекстен; 
\item
  затворени относно {\em сечение} с регулярен език, т.е.
  ако $L$ е безконтекстен език и $R$ е регулярен език, то езикът $L = L \cap R$ е безконтекстен; 
\item
  {\bf не} са затворени относно операцията {\em сечение}, т.е.
  съществуват контекстно-свободни езици $L_1, L_2$, за които езикът $L_1 \cap L_2$ {\bf не} е безконтекстен; 
\item
  {\bf не} са затворени относно операцията {\em допълнение}, т.е.
  съществува безконтекстен език $L$, за който езикът $\Sigma^\star\setminus L$ {\bf не} е безконтекстен; 
% \item
%   те са затворени относно хомоморфизми, т.е.
%   ако $L \subseteq \Sigma^\star_1$ е безконтекстен език и $h:\Sigma_1\to\Sigma^\star_2$ е хомоморфизъм, 
%   то езикът $h(L) = \{h(\alpha) \in \Sigma^\star_2 \mid \alpha \in L\}$
%   е безконтекстен.
% \item
%   те са затворени относно обратни хомоморфизми, т.е.
%   ако $L\subseteq \Sigma^\star_2$ е безконтекстен език и $h:\Sigma_1\to\Sigma^\star_2$ е хомоморфизъм, 
%   то езикът $h^{-1}(L) = \{\alpha \in \Sigma^\star_1 \mid h(\alpha) \in L\}$
%   е безконтекстен.
\end{itemize}

\section*{Библиография}

Основни източници в тази глава са:
\begin{itemize}
\item 
  глава 4 от \cite{hopcroft1}, глави 5, 6 и 7 от \cite{hopcroft2};
\item
  глава 2 от \cite{sipser1};
\item
  глава 3 от \cite{papadimitriou}.
\end{itemize}


% \section{Въпроси}
% Вярно ли е, че:
% \begin{itemize}
% \item
% %  \marginpar{Да} 
%   ако $L$ е безконтекстен език, то езикът $L \cap \{a^{2n}b^{2k}\mid n,k\in\Nat\}$ е безконтекстен ?
% \item
%  % \marginpar{Да}
%   ако $L$ е безкраен безконтекстен език, то съществува безкрайна редица от регулярни езици $L_1,L_2,\dots$,
%   за които $L = \bigcup_{i\in\Nat}L_i$ ?
% \item
%   \marginpar{Не}
%   за всяка безкрайна редица от регулярни езици $L_1,L_2,\dots$, то 
%   езикът $L = \bigcup_{i\in\Nat}L_i$ е безконтекстен ?
% \item
%   %\marginpar{Да}
%   за всеки регулярен език $R$ и всеки безконтекстен език $L$, то $L \cap R$ е безконтекстен ?
% \item
%   за всеки регулярен език $R$ и всеки безконтекстен език $L$, то $L \cup R$ е безконтекстен ?
% \item
%   за всеки регулярен език $R$ и всеки безконтекстен език $L$, то $L \setminus R$ е безконтекстен ?
% \item
%   за всеки регулярен език $R$ и всеки безконтекстен език $L$, то $R \setminus L$ е безконтекстен ?
% \item
%   съществува регулярен език $R$ и безконтекстен език $L$, за които $L \cap R$ не е безконтекстен ?
% \item
%   съществува регулярен език $R$ и нерегулярен, но безконтекстен език $L$, за които $L \cap R$ е регулярен ?
% \item
%   за всеки два нерегулярни, но контекстно-свободни езика $L_1,L_2$, то $L_1\cup L_2$ е регулярен ?
% \item
%   съществуват два нерегулярни, но безконтекстни езика $L_1,L_2$, за които $L_1\setminus L_2$ е регулярен ?
% \item
%   съществуват два нерегулярни, но безконтекстни езика $L_1,L_2$, за които $L_1\cap L_2$ е регулярен ?
% \item
%   съществуват два нерегулярни, но безконтекстни езика $L_1,L_2$, за които $L_1\cup L_2$ е регулярен ?
% \item
%   съществува регулярен език $R$, който може да се представи като $R = L_1 \cup L_2$, където
%   $L_1 \cap L_2 = \emptyset$, $L_1,L_2$ са нерегулярни, но контекстно-свободни ?
% \item
%   езикът $\{a,b\}^\star \setminus \{a^nb^n \mid n\in\Nat\}$ е регулярен ?
% \item
%   езикът $\{a,b\}^\star \setminus \{a^nb^n \mid n\in\Nat\}$ е безконтекстен ?
% \item
%   езикът $\{a,b\}^\star \setminus \{a^nb^{2k+1} \mid n,k\in\Nat\}$ е регулярен ?
% \item
%   езикът $\{a,b\}^\star \setminus \{a^nb^{k} \mid n > k\}$ е регулярен ?
% \item
%   езикът $\{a,b\}^\star \setminus \{a^nbba^{n} \mid n \in \Nat\}$ е регулярен ?
% \item
%   езикът $\{a,b\}^\star \setminus \{a^nb^n \mid n\in\Nat\}$ е безконтекстен ?
% \item
%   езикът $\{a,b,c\}^\star \setminus \{a^nb^mc^k \mid m < n\ \&\ m < k\}$ е безконтекстен ?
% \item
%   \marginpar{Не. $\alpha = b^pa^pbba^p$.}
%   езикът $L = \{uvv^R \mid u,v \in \{a,b\}^\star\ \&\ \abs{u} \leq \abs{v}\}$ е регулярен ?
% \item
%   \marginpar{Да.}
%   езикът $L = \{uvv^R \mid u,v \in \{a,b\}^\star\ \&\ \abs{u} \leq \abs{v}\}$ е безконтекстен ?
% \item
%   съществува алгоритъм, който за даден вход регулярен израз $r$ и безконтекстна граматика $G$
%   проверява дали $\L(r) = \L(G)$?
% \item
%   съществува алгоритъм, който за даден вход регулярен израз $r$ и безконтекстна граматика $G$
%   проверява дали $\L(r) \cap \L(G) = \emptyset$?
% \item
%   съществува алгоритъм, който за даден вход регулярен израз $r$ и безконтекстна граматика $G$
%   проверява дали $\abs{\L(r) \cap \L(G)} < \infty$?
% \item
%   съществува алгоритъм, който за даден вход регулярен израз $r$ и безконтекстна граматика $G$
%   проверява дали $\abs{\L(r) \cap \L(G)} = \infty$?
% \item
%   съществува алгоритъм, който за даден вход регулярен израз $r$, безконтекстна граматика $G$
%   и число $k$, проверява дали $\abs{\L(r) \cap \L(G)} = k$?
% \item
%   съществува алгоритъм, който за даден вход регулярен израз $r$ и безконтекстна граматика $G$
%   проверява дали $\L(r) \setminus \L(G) = \emptyset$?
% \item
%   съществува алгоритъм, който за даден вход регулярен израз $r$ и безконтекстна граматика $G$
%   проверява дали $\L(G) \setminus \L(r) = \emptyset$?
% \item
%   съществува алгоритъм, който за даден вход регулярен израз $r$ и безконтекстна граматика $G$
%   проверява дали $\abs{\L(r) \setminus \L(G)} < \infty$?
% \item
%   съществува алгоритъм, който за даден вход регулярен израз $r$ и безконтекстна граматика $G$
%   проверява дали $\abs{\L(G) \setminus \L(r)} < \infty$?
% \item
%   съществува алгоритъм, който за даден вход регулярен израз $r$ и безконтекстна граматика $G$
%   проверява дали $\abs{\L(r) \setminus \L(G)} = \infty$?
% \item
%   съществува алгоритъм, който за даден вход регулярен израз $r$ и безконтекстна граматика $G$
%   проверява дали $\abs{\L(G) \setminus \L(r)} = \infty$?
% \item
%   съществува алгоритъм, който за даден вход регулярен израз $r$, безконтекстна граматика $G$
%   и число $k$, проверява дали $\abs{\L(r) \setminus \L(G)} = k$?
% \item
%   съществува алгоритъм, който за даден вход регулярен израз $r$, безконтекстна граматика $G$
%   и число $k$, проверява дали $\abs{\L(G) \setminus \L(r)} = k$?
% \end{itemize}

% Нека е дадена безконтекстна граматика $G$ с правила \[S\rightarrow a\vert AB \vert AC, A \rightarrow a, B\rightarrow b, C\rightarrow SB.\]
% Вярно ли е, че ако приложим CYK алгоритъма върху думата $\alpha$, където
% \begin{itemize}
% \item 
%   $\alpha = aabb$, то $N[1,1] = \{S\}$.
% \item 
%   $\alpha = aabb$, то $N[3,3] = \{B\}$.
% \item 
%   $\alpha = aabb$, то $N[1,4] = \{\}$.
% \item
%   $\alpha = baab$, то $N[2,4] = \{\}$.
% \item
%   $\alpha = baab$, то $N[1,3] = \{\}$.
% \end{itemize}



%%% Local Variables: 
%%% mode: latex
%%% TeX-master: "EAI"
%%% End: 


\section{Допълнителни задачи}

\begin{problem}
  Докажете, че следните езици са безконтекстни.
  \begin{enumerate}[a)]
  \item
    \marginpar{$S \rightarrow aSa\ \vert\ bSb\ \vert\ \varepsilon$}
    $L = \{ww^R \mid w \in \{a,b\}^\star\}$;
  \item
    \marginpar{$S \rightarrow aSa\ \vert\ bSb\ \vert\ a\vert\ b\ \vert\ \varepsilon$}
    $L = \{w \in \{a,b\}^\star \mid w = w^R\}$;
  \item
    $L = \{a^nb^{2m}c^{n} \mid m,n \in \Nat\}$;
  \item
    $L = \{a^nb^{m}c^{m}d^n \mid m,n \in \Nat\}$;
  \item
    \marginpar{Обединение на два езика}
    $L = \{a^nb^{2k} \mid n,k \in \Nat\ \&\ n \neq k\}$;
  \item
    \marginpar{$S \rightarrow aSb | aS | a$}
    $L = \{a^nb^k \mid n > k\}$;
  \item
    $L = \{a^nb^k \mid n \geq 2k\}$;
  % \item
  %   \marginpar{$S \rightarrow aSc | B,\ B \rightarrow bBc | \varepsilon$}
  %   $L = \{a^nb^mc^{n+m}\mid n,m \in \Nat\}$;
  % \item
  %   \marginpar{$S \rightarrow aSc | aS | B$, $B\rightarrow bBc | bB | \varepsilon$}
  %   $L = \{a^nb^kc^m \mid n + k \geq m\}$;
  \item
    \marginpar{$S \rightarrow aSc | aS | aB | bB$,\\$B\rightarrow bBc | bB | \varepsilon$}
    $L = \{a^nb^kc^m \mid n + k \geq m+1\}$;
  \item
    $L = \{a^nb^kc^m \mid n + k \geq m+2\}$;
  \item
    \marginpar{$S \rightarrow aSc | aS | B | Bc$,\\$B\rightarrow bBc | bB | \varepsilon$}
    $L = \{a^nb^kc^m \mid n + k + 1 \geq m\}$;
  \item
    $L = \{a^nb^kc^m \mid n + k + 2 \geq m\}$;
  \item
    $L = \{a^nb^kc^m \mid n + k \leq m\}$;
  \item
    $L = \{a^nb^kc^m \mid n + k \leq m+1\}$;
  \item
    \marginpar{Обединение на три езика}
    $L = \{a^nb^mc^k \mid n, m, k \text{ не са страни на триъгълник}\}$.
  \item
    $L = \{a,b\}^\star \setminus \{a^{2n}b^n \mid n\in\Nat\}$;
  \item
    \marginpar{$S\to EaE$, $E \to aEbE | bEaE | \varepsilon$}
    $L = \{\alpha \in \{a,b\}^\star\mid N_a(\alpha) = N_b(\alpha) + 1\}$;
  \item
    \marginpar{$S\to E | SaS$, $E \to aEbE | bEaE | \varepsilon$}
    $L = \{\alpha \in \{a,b\}^\star\mid N_a(\alpha) \geq N_b(\alpha)\}$;
  \item
    $L = \{\alpha \in \{a,b\}^\star\mid N_a(\alpha) > N_b(\alpha)\}$;
  \item
    $L = \{\omega_1 a \omega_2 b \mid \omega_1,\omega_2 \in \{a,b\}^\star\ \&\ \abs{\omega_1} = \abs{\omega_2}\}$;
  \item
    $L = \{\alpha \sharp \beta \mid \alpha,\beta \in \{a,b\}^\star\ \&\ \alpha^R\mbox{ е поддума на }\beta \}$.
  \item 
    $L = \{\omega_1\sharp\omega_2 \mid \omega_1,\omega_2 \in \{a,b\}^\star\ \&\ \abs{\omega_1} = \abs{\omega_2}\}$;
  \item
    $L = \{\omega_1 \sharp \omega_2 \sharp \cdots \sharp \omega_n \mid n\geq 2\ \&\ \omega_1,\omega_2,\dots,\omega_n \in \{a,b\}^\star\ \&\ \abs{\omega_1} = \abs{\omega_2}\}$;
  \item
    $L = \{\omega_1 \sharp \omega_2 \sharp \cdots \sharp \omega_n \mid n\geq 2\ \&\ \omega_1,\dots,\omega_n \in \{a,b\}^\star\ \&\ (\exists i \neq j)[\abs{\omega_i} = \abs{\omega_j}]\}$;
  \item
    $L = \{\omega_1 \sharp \omega_2 \sharp \cdots \sharp \omega_n \mid n\geq 2\ \&\ (\forall i\in[1,n])[\omega_i \in \{a,b\}^\star\ \&\ \abs{\omega_i} = \abs{\omega_{n+1-i}}]\}$.
  \end{enumerate}
\end{problem}


\begin{problem}
  Проверете дали следните езици са безконтекстни:
  \begin{enumerate}[a)]
  \item
    $\{a^nb^{2n}c^{3n}\ \mid\ n\in\Nat\}$;
  \item
    $\{a^nb^{2n}c^{n}\ \mid\ n\in\Nat\}$;
  \item
    $\{a^mb^n\mid\ m \neq n\}$;
  \item
    $\{a^nb^mc^k\mid n < m < k\}$;
  \item
    $\{a^nb^mc^k\mid k = \min\{n,m\}\}$;
  \item
    $\{a^nb^nc^m\mid m \leq n\}$;
  \item
    $\{a^nb^mc^k\mid k = n\cdot m\}$;
  \item
    $L^\star$, където
    $L = \{\alpha\alpha^R \mid \alpha \in \{a,b\}^\star\}$;
  \item
    $\{www\mid w\in \{a,b\}^\star\}$;
  \item
    $\{ww^R\mid w\in \{a,b\}^\star\}$;
  \item
    $\{a^{n^2}b^n\ \mid n \in \Nat\}$;
  \item
    $\{a^p\ \mid\ p\mbox{ е просто }\}$;
  \item
    $\{\omega \in \{a,b\}^\star \mid \omega = \omega^R\}$;
  \item
    $\{\omega^n \mid \omega \in \{a,b\}^\star\ \&\ n \in \Nat\}$;
  \item
    $\{a^{n^3 + 2n^2} \mid n \in \Nat\}$;
  \item
    % Дефиниция на подниз
    $L = \{w c x\mid w,x\in \{a,b\}^\star\ \&\ w\mbox{ е подниз на }x\}$;
  \item
    $L = \{x_1 c x_2 c \dots c x_k\mid k\geq 2\ \&\ x_i\in\{a,b\}^\star\ \&\ (\exists i,j)[i \neq j\ \&\ x_i = x_j]\}$;
  \item
    $L = \{a^ib^jc^k\mid i,j,k\geq 0\ \&\ (i = j \vee j = k)\}$;
  \item
    % \marginpar{Разгл. $L' = L \cap L(a^*b^*c^*)$.}
    $L = \{\alpha \in \{a,b,c\}^\star\mid N_a(\alpha) > N_b(\alpha) > n_c(\alpha)\}$;
  \item
    $L = \{a,b\}^\star \setminus \{a^nb^n\mid n\in \Nat\}$;
  \item
%    \marginpar{Разгл. $L' = L \cap L(a^*b^*a^*)$.}
    $L = \{\omega \in \{a,b\}^\star \mid N_a(\omega) = 2N_b(\omega)\}$;
  \item
    $L = \{a^nb^mc^ma^n \mid m,n\in\Nat\ \&\ n = m+42\}$;
  \item
    $L = \{babaabaaab\cdots ba^{n-1}ba^nb \mid n \geq 1\}$;
%   \end{enumerate}
% \end{problem}

% \begin{problem}
%   Проверете кои от следните езици са безконтекстен:
%   \begin{enumerate}[a)]
  \item
    $\{a^mb^nc^k\mid m = n \vee n = k \vee m = k\}$;
  \item
    $\{a^mb^nc^k\mid m \neq n \vee n \neq k \vee m \neq k\}$;
  \item
    $\{a^mb^nc^k\mid m = n \wedge n = k \wedge m = k\}$;
  \item
    $\{w \in \{a,b,c\}^\star\mid N_a(w) \neq N_b(w) \vee N_a(w) \neq n_c(w) \vee N_b(w) \neq n_c(w)\}$.
  \end{enumerate}
\end{problem}

\begin{problem}
  Докажете, че ако $L$ е безконтекстен език, то $L^R = \{\omega^R \mid \omega \in L\}$ 
  също е безконтекстен.
\end{problem}

\begin{problem}
  \marginpar{от Владислав}
  Нека $\Sigma = \{a,b,c,d,f,e\}$.
  Докажете, че езикът $L$ е безконтекстен, където за думите $\omega \in L$ са изпълнени свойствата:
  \begin{itemize}[-]
  \item 
    за всяко $n\in\Nat$, след всяко срещане на $n$ последнователни $a$-та
    следват $n$ последователни $b$-та, и $b$-та не се срещат по друг повод в $\omega$, и
  \item
    за всяко $m\in\Nat$, след всяко срещане на $m$ последнователни $c$-та
    следват $m$ последователни $d$-та, и $d$-та не се срещат по друг повод в $\omega$, и
  \item
    за всяко $k\in\Nat$, след всяко срещане на $k$ последнователни $f$-а
    следват $k$ последователни $e$-та, и $e$-та не се срещат по друг повод в $\omega$.
  \end{itemize}
\end{problem}

\begin{problem}
  \marginpar{от Владислав}
  Да разгледаме езиците:
  \begin{align*}
    & P = \{\alpha\in\{a,b,c\}^*\,|\, \alpha \text{ е палиндром с четна дължина}\} \\
    & L =  \{\beta b^n\,|\, n\in\mathbb{N}, \beta\in P^n\}.
  \end{align*}
  Да се докаже, че:
  \begin{enumerate}[a)]
  \item 
    $L$ не е регулярен;
  \item 
    $L$ е безконтекстен.
  \end{enumerate}
\end{problem}

\begin{problem}
  \marginpar{от Владислав}
  Нека $L_1$ е произволен регулярен език над азбуката $\Sigma$, 
  а $L_2$ е езика от всички думи палиндроми над $\Sigma$.
  Докажете, че $L$ е безконтекстен език, където:
  \[L = \{\alpha_1\alpha_2\cdots\alpha_{3n}\beta_1\cdots\beta_m\gamma_1\cdots\gamma_n \mid \alpha_i,\gamma_j \in L_1, \beta_k\in L_2, m,n \in \Nat\}.\]
\end{problem}

\begin{problem}
  \marginpar{от Владислав}
  Нека $L = \{\omega\in\{a,b\}^\star \mid N_a(\omega) = 2\}$.
  Да се докаже, че езикът $L' = \{\alpha^n \mid \alpha\in L, n \geq 0\}$ не е безконтекстен.
\end{problem}


\begin{problem}
  \marginpar{от Владислав}
  Нека $\Sigma = \{a,b,c\}$ и $L \subseteq \Sigma^\star$ е безконтестен език. Ако имаме дума 
  $\alpha \in \Sigma^\star$, тогава \emph{L-вариант} на $\alpha$ ще наричаме думата, която се получава като в $\alpha$ всяко едно 
  срещане на символа $a$ заменим с (евентуално различна) дума от $L$.
  Тогава, ако $M \subseteq \Sigma^*$ е произволен безконтестен език, да се докаже че езикът
  \begin{equation*}
    M' = \{\beta\in\Sigma^\star |\ \beta \text{ е $L$-вариант на } \alpha \in M \}
  \end{equation*}
  също е безконтекстен.
\end{problem}

\begin{problem}
  Докажете, че всеки безконтекстен език над азбуката $\Sigma = \{a\}$
  е регулярен.
\end{problem}

\begin{problem}
%  \marginpar{\cite{papadimitriou} стр. 149}
  Да фиксираме азбуката $\Sigma$.
  Нека $L$ е безконтекстен език, а $R$ е регулярен език.
  Докажете, че езикът
  $L/R = \{\omega \in \Sigma^\star \mid (\exists u \in R)[\omega u \in L]\}$
  е безконтекстен.
\end{problem}


\begin{problem}
  Нека е дадена граматиката $G = \pair{\{a,b\}, \{S,A,B,C\},S,R}$.
  Използвайте CYK-алгоритъма, за да проверите дали
  думата $\alpha$ принадлежи на $\L(G)$, където правилата на граматиката $R$ и думата $\alpha$
  са зададени като:
  \begin{enumerate}[a)]
  \item
    $R = \{S\rightarrow BA| CA|a, C\rightarrow BS|SA,A\rightarrow a, B\rightarrow b\}$, $\alpha=bbaaa$;
  \item
    $R =\{S\rightarrow AB|BC, A\rightarrow BA|a,B\rightarrow CC|b, C\rightarrow AB|a\}$, $\alpha=baaba$;
  \item
    $R = \{S\rightarrow AB, A\rightarrow AC|a|b,B\rightarrow CB|a, C\rightarrow a\}$, $\alpha=babaa$;
  \end{enumerate}
\end{problem}

\begin{problem}
  \marginpar{Интересно е също да се направи и безконтекстна граматика за $L$}
  Постройте стеков автомат за езика над азбуката $\{a,b,\sharp\}$:
  \[L = \{\omega_1 \sharp \omega_2 \sharp \cdots \sharp \omega_{2n} \mid n\in\Nat\ \&\ \sum^n_{i=1}\abs{\omega_{2i}} = \sum^{n}_{i=1}\abs{\omega_{2i-1}}\}.\]
\end{problem}


%%% Local Variables: 
%%% mode: latex
%%% TeX-master: "EAI"
%%% End: 


\chapter{Машини на Тюринг}

\setlength{\epigraphwidth}{0.65\textwidth}\epigraph{Turing’s ‘Machines’. These machines are humans who calculate. \cite[§ 1096]{rpp1}.}




% \begin{framed}
%   {\bf Теза на Чърч-Тюринг:} Всеки алгоритъм може да се осъществи като машина на Тюринг.
% \end{framed}

% \section{Основни понятия}
\index{Тюринг}
\index{машина на Тюринг!детерминистична}
\mynote{Тук до голяма степен следваме \cite[Глава 3]{sipser3}. Понятието за машина на Тюринг има много еквивалентни дефиниции. }
{\em Детерминистична} машина на Тюринг ще наричаме осморка от вида 
\[\M = \TM,\] където:
\begin{itemize}
\item 
  $Q$ - крайно множество от състояния;
\item
  $\Sigma$ - крайна азбука за входа;
\item
  $\Gamma$ - крайна азбука за лентата, $\Sigma \subseteq \Gamma$;

\item
  $\blank$ - символ за празна клетка на лентата,  $\blank \in \Gamma \setminus \Sigma$;
\item
  $\qstart \in Q$ - начално състояние;
\item
  \mynote{Тези две състояния ще наричаме заключителни}
  $\qaccept \in Q$ - приемащо състояние;
\item
  $\qreject \in Q$ - отхвърлящо състояние, където $\qaccept \neq \qreject$;
\item
  % \mynote{Няма нужда да изискваме главата да остава върху същата клетка от лентата}
  \mynote{Това означава, че веднъж достигнем ли заключително състояние, не можем да правим повече преходи. Тук следваме \cite[стр. 169]{sipser3} и \cite[стр. 327]{hopcroft2}.}
  $\delta:Q'\times\Gamma \to Q\times \Gamma \times \{\goleft,\goright,\stay\}$ - тотална функция на преходите, където\\
  $Q'~=~Q~\setminus~\{\qaccept, \qreject\}$.
\end{itemize}

Всяка машина на Тюринг разполага с неограничено количество памет, която е представена като безкрайна (и в двете посоки) лента, разделена на клетки.
Всяка клетка съдържа елемент на $\Gamma$.
Сега ще опишем как $\M$ работи върху вход думата $\alpha \in \Sigma^\star$.
Първоначално безкрайната лента съдържа само думата $\alpha$. Останалите клетки на лентата съдържат символа $\blank$.

Освен това, $\M$ се намира в началното състояние $\qstart$ и главата за четене е върху най-левия символ на $\alpha$.
Работата на $\M$ е описана от функцията на преходите $\delta$.
  
\begin{itemize}
\item
  \index{машина на Тюринг!конфигурация}
  \index{машина на Тюринг!моментно описание}
  \mynote{На англ. instanteneous description.\\
    Понякога за удобство ще означаваме моментната конфигурация като $(q,\alpha\underline{x}\beta)$ вместо по-неудобното $(\alpha,q,x\beta)$.}
  Формално, {\bf моментната конфигурация} (или описание) на едно изчисление на машина на Тюринг
  е тройка от вида 
  \[(\alpha, q, \beta) \in \Gamma^\star\times Q \times \Gamma^+,\]
  като интерпретацията на тази тройка е, че машината се намира в състояние $q$ и лентата има вида
  \[\tape{\alpha\underline{x}\beta'},\]
  където $\beta = x\beta'$ и четящата глава на машината е поставена върху $x$.
\item
  Макар и да имаме безкрайна лента, моментната конфигурация, която може да се представи като {\em крайна} дума,
  описва цялото моментно състояние на машината на Тюринг.
\item
  \index{машина на Тюринг!начална конфигурация}
  {\bf Началната конфигурация} за входната дума $\alpha \in \Sigma^\star$ представлява тройката
  \[(\varepsilon, \qstart, \alpha\blank).\]
\item
  \index{машина на Тюринг!приемаща конфигурация}
  {\bf Приемаща конфигурация} представлява тройка от вида
  \[(\beta, \qaccept, \gamma).\]
\item
  \index{машина на Тюринг!отхвърляща конфигурация}
  {\bf Отхвърляща конфигурация} представлява тройка от вида
  \[(\beta, \qreject, \gamma).\]
\item
  \index{машина на Тюринг!заключителна конфигурация}
  Една конфигурация ще наричаме {\bf заключителна}, ако тя е или приемаща или отхвърляща.
\end{itemize}

Както за автомати, удобно е да дефинираме бинарна релация $\vdash_\M$ над $\Gamma^\star~\times~Q~\times~\Gamma^+$,
която ще казва как моментната конфигурация на машината $\M$ се променя при изпълнение на една стъпка.


За да направим това, удобно е първо да дефинираме бинарната релация $\vdash_{y,d}$ над $\Gamma^\star~\times~Q~\times~\Gamma^+$, която показва как една моментна конфигурация се променя, когато заменим символа на главата с $y$ и се придвижим на посока $d \in \{\goleft, \goright, \stay\}$.

\mynote{Дефиницията на релацията $\vdash_{y,d}$ не зависи от конкретна машина на Тюринг!}
\begin{important}
  \begin{figure}[H]
    \begin{subfigure}[b]{0.5\textwidth}
      \begin{prooftree}
        \AxiomC{}
        \UnaryInfC{$(\lambda, q, xz\rho) \vdash_{y,\goright} (\lambda y, p, z\rho)$}
      \end{prooftree}
      \vspace*{2mm}
    \end{subfigure}
    ~
    \begin{subfigure}[b]{0.5\textwidth}
      \begin{prooftree}
        \AxiomC{}
        \UnaryInfC{$(\lambda, q, x) \vdash_{y,\goright} (\lambda y, p, \blank)$}
      \end{prooftree}
      \vspace*{2mm}
    \end{subfigure}
    
    \begin{subfigure}[b]{0.5\textwidth}
      \begin{prooftree}
        \AxiomC{}
        \UnaryInfC{$(\lambda z, q, x\rho) \vdash_{y,\goleft} (\lambda, p, z y\rho)$}
      \end{prooftree}
    \end{subfigure}
    ~
    \begin{subfigure}[b]{0.5\textwidth}
      \begin{prooftree}
        \AxiomC{}
        \UnaryInfC{$(\varepsilon, q, x\rho) \vdash_{y,\goleft} (\varepsilon, p, \blank y \rho)$}
      \end{prooftree}
    \end{subfigure}
    
    \begin{prooftree}
      \AxiomC{}
      \UnaryInfC{$(\lambda, q, x\rho) \vdash_{y,\stay} (\lambda, p, y \rho)$}
    \end{prooftree}
    \caption{Дефиниция на релацията $\vdash_{y,d}$.}
  \end{figure}
\end{important}



\mynote{Ако няма опасност да се заблудим за коя точно машина на Тюринг $\M$ говорим, то е възможно да пишем просто $\vdash$ вместо $\vdash_\M$.}

Сега вече сме готови да дефинираме релацията $\vdash_\M$.

\begin{important}
  \begin{figure}[H]
    \centering
    \begin{prooftree}
      \AxiomC{$\delta(q,x) = (q',y,d)$}
      \AxiomC{$(\lambda, q, x\rho) \vdash_{y,d} (\lambda', q', \rho')$}
      \BinaryInfC{$(\lambda, q, x\rho) \vdash_{\M} (\lambda', q', \rho')$}
    \end{prooftree}
    \caption{Едностъпков преход в еднолентова детерминистична машина на Тюринг $\M$}
  \end{figure}
\end{important}

Сега за всяко естествено число $\ell$, ще дефинираме релацията $\vdash^{\ell}$,
която ще казва, че от конфигурацията $\kappa$ можем да достигнем до конфигурацията $\kappa'$ за $\ell$ на брой стъпки.

\begin{figure}[H]
  \begin{subfigure}[b]{0.5\textwidth}
    \begin{prooftree}
      \AxiomC{}
      \RightLabel{\scriptsize{(рефлексивност)}}
      \UnaryInfC{$\kappa \vdash^0 \kappa$}
    \end{prooftree}
  \end{subfigure}
  ~
  \begin{subfigure}[b]{0.5\textwidth}
    \begin{prooftree}
      \AxiomC{$\kappa \vdash \kappa''$}
      \AxiomC{$\kappa'' \vdash^{\ell} \kappa'$}
      \RightLabel{\scriptsize{(транзитивност)}}
      \BinaryInfC{$\kappa \vdash^{\ell+1}\kappa'$}
    \end{prooftree}
  \end{subfigure}
\end{figure}

\begin{itemize}
% \item
%   $(\lambda, q, \rho) \vdash^\star (\lambda',q',\rho') \dff (\exists \ell)[(\lambda, q, \rho) \vdash^\ell (\lambda', q',\rho')]$.
\item
  С $\vdash^\star$ ще означаваме рефлексивното и транзитивно затваряне на релацията $\vdash$ или с други думи,
  \[\kappa \vdash^\star \kappa' \dff (\exists \ell \in \Nat)[\kappa \vdash^\ell \kappa'].\]
\item
  Макар и една конфигурация $\kappa$ да преставлява тройка, то често ще бъде удобно да гледаме на $\kappa$ като на дума от $\Gamma^\star Q \Gamma^+$.
\item
  Важно свойство е, че ако $\kappa \vdash^\star \kappa'$, то $\abs{\kappa} \leq \abs{\kappa'}$.
\item
  % \mynote{Аналогично както на си думата \texttt{abc} се представя като \texttt{a,b,c,0}, тук
  % представяме думата $\alpha$ като $\alpha\blank$.}
  \mynote{Важно е да имаме $\blank$ след думата $\alpha$, защото е възможно $\alpha$ да е празната дума.}
  машината на Тюринг $\M$ {\bf приема} думата $\alpha$, ако за някои $\lambda, \rho \in \Gamma^\star$,
  \[(\varepsilon, \qstart, \alpha\blank) \vdash^\star_\M (\lambda, \qaccept, \rho).\]
\item
  Машината на Тюринг $\M$ {\bf отхвърля} думата $\alpha$, ако за  някои $\lambda, \rho \in \Gamma^\star$,
  \[(\varepsilon, \qstart, \alpha\blank) \vdash^\star_\M (\lambda, \qreject, \rho).\]
  % за някои $\gamma_1, \gamma_2 \in \Gamma^\star$.
\item
  Машината на Тюринг $\M$ {\bf не приема} думата $\alpha$, 
  ако $\M$ отхвърля $\alpha$ или $\M$ никога не завършва при начална конфигурация $(\varepsilon,\qstart,\alpha)$.
\item
  \index{машина на Тюринг!разрешител}
  \mynote{На англ. такава машина на Тюринг се нарича {\bf decider} \cite[стр. 170]{sipser3}. Може такива машини на Тюринг да се наричат и тотални \cite[стр. 213]{kozen}.
    Да се внимава, че в Манев понятията са различни.}
  Една машина на Тюринг се нарича {\bf разрешител}, ако при всеки вход достига до заключително състояние,
  т.е. достига до $\qaccept$ или $\qreject$.
\item 
  Езикът, който се {\bf разпознава} от машината $\M$ е:
  \[\L(\M) \df \{\alpha\in\Sigma^\star \mid (\varepsilon, \qstart, \alpha\blank) \vdash^\star_\M (\lambda, \qaccept, \rho), \text{ за някои }\lambda,\rho\in\Gamma^\star\}.\]
\item
  \index{език!полуразрешим}
  \mynote{На англ. {\bf semidecidable language}. В литературата се използва и названието {\bf рекурсивно номеруем език}.}
  Езикът $L$ се нарича {\bf полуразрешим}, ако съществува машина на Тюринг $\M$, за която
  $L = \L(\M)$.
  В този случай се казва, че $\M$ разпознава езика $L$.
  Ако една дума $\alpha \in L$, то след крайно много стъпки ще достигнем до състоянието $\qaccept$.
  Ако $\alpha \not\in L$, то не е ясно дали какво се случва с изчислението на $\M$ върху $\alpha$. Възможно е да достигнем до състоянието $\qreject$, но може да попаднем в безкрайно изчисление.
\item
  \index{език!разрешим}
  \mynote{На англ. {\bf decidable language}. В литературата се използва и названието {\bf рекурсивен език}.}
  Един език $L$ се нарича {\bf разрешим}, ако за него съществува {\em разрешител} $\M$, за която
  $L = \L(\M)$.
  В този случай се казва, че $\M$ разрешава езика $L$.
\end{itemize}

\begin{framed}
  \begin{proposition}
    Ако $L$ е разрешим език над азбуката $\Sigma$, то $\Sigma^\star \setminus L$ също е разрешим език.
  \end{proposition}
\end{framed}

От дефинициите е ясно, че всеки разрешим език е полуразрешим.
По-късно, ще видим, че съществуват полуразрешими езици, чиито допълнения не са полуразрешими,
т.е. не всеки полуразрешим език е разрешим.
Една от основните ни задачи ще бъде да класифицираме различни езици като (не)раз\-ре\-ши\-ми и (не)полуразрешими.
За да придобием по-добра интуиция за тези нови понятия, ще разгледаме подробно няколко примера.
Ще видим също как можем да изобразяваме функцията на преходите на $\M$ графично.


%%% Local Variables:
%%% mode: latex
%%% TeX-master: "../eai"
%%% End:


\section{Примери за разрешими езици}

\begin{example}
  \marginpar{Знаем, че $L$ не е безконтекстен}
  Да разгледаме езика $L = \{a^nb^nc^n \mid n\in\Nat\}$.
 
  Нека да въведем нов символ $d$, с който ще маркираме обработените символи $a$, $b$, $c$.
  Идеята на алгоритъма, който ще разгледаме е да маркира на всяка итерация по едно $a$, $b$, и $c$.
  Той завършва успешно ако всички символи на думата са маркирани.
  Нека първоначално думата е копирана върху лентата и четящата глава е върху първия символ на думата.
  \begin{enumerate}[(1)]
  \item 
    Чете $d$-та надясно по лентата докато срещне първото $a$ и го замества с $d$. Отива на стъпка (2).
    Ако символите свършат (т.е. достигне се $\blank$) преди да се достигне $a$,
    то алгоритъмът завършва успешно.
  \item
    Чете $d$-та надясно по лентата докато срещне първото $b$ и го замества с $d$.
    Отива на стъпка (3).
  \item
    Чете $d$-та надясно по лентата докато срещне първото $c$ и го замества с $d$.
  \item
    Връща четящата глава в началото на лентата, т.е. чете наляво докато не срещне символа $\blank$.
    Връща се в стъпка (1). 
  \end{enumerate}

  Нека сега да видим, че този алгоритъм може да се опише съвсем формално с машина на Тюринг.
  Ще построим машина на Тюринг $\M$, за която $L = \L(\M)$, където
  \begin{itemize}
  \item 
    $\Sigma = \{a,b,c\}$;
  \item
    $\Gamma = \{a,b,c,d,\blank\}$, за някой нов символ $d$;
  \item
    $Q = \{q_1,q_2,\dots,q_5\}$;
  \item
    $q_{accept} = q_5$;
  \item
    частичната функция на преходите $\delta:Q\times\Gamma \to Q\times\Gamma\times\{\goleft,\goright,\stay\}$
    е описана на схемата отдолу.
  \end{itemize}

  \begin{framed}
  \begin{figure}[H]
    \begin{center}
      \begin{tikzpicture}[->,>=stealth,thick,node distance=50pt]
        \tikzstyle{every state}=[circle,minimum size=10pt,auto]
        
        \node[state,initial]    (1) {$q_1$};
        \node[state]            (2) [right of=1]{$q_2$};
        \node[state]            (3) [right of=2]{$q_3$};
        \node[state]            (4) [right of=3]{$q_4$};
        \node[state,accepting]  (5) [below of=1]{$q_5$};
        % \node[state,accepting]  (6) [right of=5]{$6$};
        
        \begin{scope}[every node/.style={scale=.8}]
          \path
          (1) edge [loop above] node [above] {$d;\goright$} (1)
          (1) edge [bend right=15] node [left] {$\blank;\goright$} (5)
          % (1) edge [bend left=15] node [left] {$\{b,c\}$} (6)
          (1) edge [bend left=15] node [above] {$a/d;\goright$} (2)
          (2) edge [bend left=15] node [above] {$b/d;\goright$} (3)
          (2) edge [loop above] node   [above] {$\{a,d\};\goright$} (2)
          (3) edge [bend left=15] node [above] {$c/d;\goleft$} (4)
          (3) edge [loop above] node   [above] {$\{b,d\};\goright$} (3)
          (4) edge [loop right] node   [below right] {$\{a,b,d\};\goleft$} (4)
          (4) edge [in=65,out=115,above] node [above] {$\blank;\goright$} (1);
        \end{scope}
      \end{tikzpicture}
    \end{center}
    \caption{детерминистична частична машина на Тюринг $\M$, за която $\L(\M) = \{a^nb^nc^n \mid n \in \Nat\}$}
  \end{figure}
  \end{framed}

  Например,
  \begin{align*}
    & \delta(q_1, a) = (q_2, d, \goright)\\
    & \delta(q_4, \blank) = (q_1, \blank, \goright)\\
    & \delta(q_2, a) = (q_2, a, \goright).
  \end{align*}

  % Да проследим изчислението на думата $aabbcc$:
  
  % \[_1aabbcc \vdash d_2abbcc \vdash da_2bbcc \vdash dad_3bcc \vdash dadb_3cc \vdash dad_4bdc \vdash da_4dbdc \vdash \cdots \vdash\]
  % \[_4dadbdc \vdash\ _4\blank dadbdc \vdash\ _1dadbdc \vdash d_1adbdc \vdash dd_2dbdc \vdash ddd_2bdc \vdash dddd_3dc \vdash \]
  % \[ ddddd_3c \vdash dddddd_4 \vdash \cdots \vdash\ _4\blank dddddd \vdash\ _1dddddd \vdash \cdots \vdash dddddd_1\blank \vdash dddddd_5\blank.\]

  Съобразете, че тази машина на Тюринг може да се направи тотална като се добави ново състояние $\qreject$
  и за всяка двойка $(q,x)$, за която функцията на преходите не е дефинирана, да сочи към $\qreject = q_6$.
  Така можем да получим {\em тотална} машина на Тюринг за езика $L$, което означава, че 
  $L$ е не само полуразрешим, но {\em разрешим} език.

\begin{framed}
  \begin{figure}[H]
    \begin{center}
      \begin{tikzpicture}[->,>=stealth,thick,node distance=50pt]
        \tikzstyle{every state}=[circle,minimum size=10pt,auto]
        
        \node[state,initial]    (1) {$q_1$};
        \node[state]            (2) [right of=1]{$q_2$};
        \node[state]            (3) [right of=2]{$q_3$};
        \node[state]            (4) [right of=3]{$q_4$};
        \node[state,accepting]  (5) [below of=1]{$q_5$};
        \node[state]            (6) [below of=3]{$q_6$};
        
        \begin{scope}[every node/.style={scale=.8}]
          \path
          (1) edge [loop above] node [above] {$d;\goright$} (1)
          (1) edge [bend right=15] node [left] {$\blank;\stay$} (5)
          % (1) edge [bend left=15] node [left] {$\{b,c\}$} (6)
          (1) edge [bend left=15] node [above] {$a/d;\goright$} (2)
          (2) edge [bend left=15] node [above] {$b/d;\goright$} (3)
          (2) edge [loop above] node [above] {$\{a,d\};\goright$} (2)
          (3) edge [bend left=15] node [above] {$c/d;\goleft$} (4)
          (3) edge [loop above] node [above] {$\{b,d\};\goright$} (3)
          (4) edge [loop right] node [below right] {$\{a,b,d\};\goleft$} (4)
          (4) edge [in=65,out=115,above] node [above] {$\blank;\goright$} (1);

          \path
          (1) edge [dashed, bend right=15] node [left] {$\{b,c\};\stay$} (6)
          (2) edge [dashed, bend left=30] node [left] {$\{c,\blank\};\stay$} (6)
          (3) edge [dashed, bend left=25] node [right] {$\{a,\blank\};\stay$} (6)
          (5) edge [dashed, loop below] node [below] {$\{a,b,c,d,\blank\};\stay$} (5)
          (6) edge [dashed, loop right] node [right] {$\{a,b,c,d,\blank\};\stay$} (6);
        \end{scope}
      \end{tikzpicture}
    \end{center}
  \end{figure}
\end{framed}

\end{example}

\begin{example}
  \marginpar{Да напомним, че този език не е безконтекстен}
  \marginpar{В \cite[стр. 155]{hopcroft1} е дадено по-различно решение. Тук следваме \cite[стр. 173]{sipser3}. Там има малка грешка}
  Да разгледаме езика $L = \{\omega \sharp \omega \mid \omega\in\{a,b\}^\star\}$.
  Нека първо да видим, че можем неформално да опишем алгоритъм, който да разпознава думите на езика $L$.
  Нека една дума е копирана върху лентата и четящата глава е поставена върху първия символ от думата.
  \begin{enumerate}[(1)]
  \item 
    Чете $x$-ове надясно по лентата докато не срещне $a$ или $b$ и го замества с $x$.
    Запомня дали сме срещнали $a$ или $b$.
    Ако вместо $a$ или $b$ срещне $\sharp$, то отива на стъпка $(6)$.
  \item
    Чете $a$-та и $b$-та надясно по лентата докато не стигне $\sharp$. 
  \item
    Чете $c$-то надясно по лентата и всички следващи $x$-ове докато не срещне символа $a$ или $b$.
    Той трябва да е същия символ, който сме запаметили на стъпка $(1)$.
    Заместваме този символ с $x$.
  \item
    Чете $x$-ове наляво по лентата докато не стигне $\sharp$.
  \item
    Чете $a$-та и $b$-та по лентата докато не стигне $x$.
    Поставя четящата глава върху символа точно след първия $x$.
    Отива на стъпка $(1)$.
  \item
    Прочита $\sharp$ надясно по лентата и чете надясно $x$-ове докато не срещне $\blank$.
    Алгоритъмът завършва успешно.
  \end{enumerate}

  Ще построим машина на Тюринг $\M$, за която $L = \L(\M)$.
  \begin{itemize}
  \item 
    $\Sigma = \{a,b,\sharp\}$;
  \item
    $\Gamma = \{a,b,\sharp,x,\blank\}$;
  \item
    $Q = \{q_1,q_2,\dots,q_9\}$;
  \item
    $\qaccept = q_9$;
  \end{itemize}

  \begin{framed}
  \begin{figure}[H]
    \begin{center}
      \begin{tikzpicture}[->,>=stealth,thick,node distance=50pt]
        \tikzstyle{every state}=[circle,minimum size=10pt,auto,scale=.9]
        
        \node[state,initial]    (1) {$q_1$};
        \node[state]            (2) [above right of=1]{$q_2$};
        \node[state]            (3) [below right of=1]{$q_3$};
        \node[state]            (4) [right of=2]{$q_4$};
        \node[state]            (5) [right of=3]{$q_5$};
        \node[state]            (6) [below right of=4]{$q_6$};
        \node[state]            (7) [above of=6]{$q_7$};
        \node[state]            (8) [left of=3]{$q_8$};
        \node[state,accepting]  (9) [below left of=3]{$q_9$};
        
        \begin{scope}[every node/.style={scale=.8}]
          \path
          (1) edge [bend left=15] node [below right] {$a/x;\goright$} (2)
              edge [bend right=15] node [above right] {$b/x;\goright$} (3)
              edge [bend right=15] node [left] {$\sharp;\goright$} (8)
          (2) edge [loop above] node [above] {$\{a,b\};\goright$} (2)
              edge [bend left=15] node [above] {$\sharp;\goright$} (4)
          (3) edge [loop below] node [below] {$\{a,b\};\goright$} (3)
              edge [bend right=15] node [below] {$\sharp;\goright$} (5)
          (4) edge [loop above] node [above] {$x;\goright$} (4)
              edge [bend left=15] node [below left] {$a/x;\goleft$} (6)
          (5) edge [loop below] node [below] {$x;\goright$} (5)
              edge [bend right=15] node [above left] {$b/x;\goleft$} (6)
          (6) edge [loop right] node [right] {$x;\goleft$} (6)
              edge [bend right=15] node [right] {$\sharp;\goleft$} (7)
          (7) edge [loop right] node [right] {$\{a,b\};\goleft$} (7)
              edge [out=130,in=120,above,distance=2.5cm] node [above] {$x;\goright$} (1)
          (8) edge [loop left] node [left] {$x;\goright$} (8)
              edge [bend right=15] node [left] {$\blank;\stay$} (9);
        \end{scope}
      \end{tikzpicture}
    \end{center}
    \caption{детерминистична частична машина на г-н Тюринг $\M$, за която $\L(\M) = \{\omega\sharp\omega \mid \omega \in \{a,b\}^\star\}$}
  \end{figure}
\end{framed}

  Да проследим изчислението на думата $ab\sharp ab$.
  
  \begin{align*}
    (q_1, \underline{a}b\sharp ab) & \to (q_2, x\underline{b}\sharp ab) \to (q_2,xb\underline{\sharp}ab) \to (q_4, xb\sharp\underline{a}b) \to (q_6, xb\underline{\sharp}xb)\\
    & \to (q_7, x\underline{b}\sharp xb) \to (q_7, \underline{x}b\sharp xb) \to (q_1, x\underline{b}\sharp xb) \to (q_3, xx\underline{\sharp}xb)\\
    & \to (q_5, xx\sharp\underline{x}b) \to (q_5, xx\sharp x\underline{b}) \to (q_6, xx\sharp\underline{x}x) \to (q_6, xx\underline{\sharp} xx)\\
    & \to (q_7, x\underline{x}\sharp xx) \to (q_1, xx\underline{\sharp} xx) \to (q_8, xx\sharp\underline{x}x) \to (q_8, xx\sharp x\underline{x})\\
    & \to (q_8, xx\sharp xx\underline{\blank}) \to (q_9,xx\sharp xx\underline{\blank}).
  \end{align*}

  Може лесно да се съобрази, че тази машина на Тюринг може да се допълни до {\em тотална}.
  
\end{example}

%%% Local Variables:
%%% mode: latex
%%% TeX-master: "../eai"
%%% End:


\subsection*{Многолентови машини на Тюринг}
\index{машина на Тюринг!многолентова}
%Това е просто като имаш shift.
%Използват се при недет. машини

Машина на Тюринг с $k$ ленти има същата дефиниция като еднолентова машина на Тюринг
с единствената разлика, че
\[\delta: Q \times \Gamma^k\to Q \times \Gamma^k \times \{L,R,S\}^k.\]
Тук добавяме и възможността главата върху някои от лентите да стои на място.
\marginpar{В \cite[стр. 177]{sipser3} конструкцията е малко по-различна. Там съдържанието на всяка лента се поставя последователно върху една лента, като се разделят със специален символ}
\begin{prop}
  За всяка $k$-лентова машина на Тюринг $\M$ съществува еднолентова машина на Тюринг $\M'$,
  такава че $\L(\M) = \L(\M')$.
\end{prop}
\begin{proof}
  \marginpar{Тук на практика следваме \cite[стр. 162]{hopcroft1}}
  Нека $\M$ е $k$-лентова машина на Тюринг.
  Ще построим еднолентова машина на Тюринг $\M'$, за която $\L(\M) = \L(\M')$.
  Да означим $\hat\Gamma = \{\hat X \mid X \in \Gamma\}$.
  Тогава азбуката на лентата на $\M'$ ще бъде $\Gamma' = (\hat\Gamma \cup \Gamma)^{k}$.
  Сега вместо да имаме $k$ ленти ще имаме една лента, която представлява $k$-орка.
  За да симулираме $\M$, използваме символите $\hat X$ за да маркират позицията на главите на $\M$,
  като във всяка координата на лентата има точно по един символ от вида $\hat X$.
  % С $\$$ ще отблезяваме границите на всяка лента, в която можем да търсим маркера.
  За да определим следващия ход на машината $\M'$, ние трябва да сканираме лентата докато не 
  открием разположението на всичките $k$ на брой маркирани клетки. Тогава симулираме ход на $\M$
  и отново трябва да променим маркираните клетки.
\end{proof}

%%% Local Variables:
%%% mode: latex
%%% TeX-master: "../eai"
%%% End:


\section{Изчислими функции}

Една {\em тотална} функция $f:\Sigma^\star \to \Sigma^\star$ се нарича изчислима с машина на Тюринг $\M$, ако 
за всяка дума $\alpha \in \Sigma^\star$,
\[(\varepsilon, \qstart, \alpha) \vdash^\star_\M (\varepsilon, \qaccept, f(\alpha)).\]
Това означава, че машината на Тюринг $\M$ е тотална.

Лесно може да се съобрази, че тогава езикът
\[Graph(f) = \{\alpha\sharp f(\alpha) \mid \alpha \in \Sigma^\star\}\]
е разрешим.

\begin{problem}
  Докажете, че съществуват функции от вида $f:\Sigma^\star\to\Sigma^\star$, които не са изчислими с машина на Тюринг.
\end{problem}
\begin{hint}
  Всяка машина на Тюринг може да се кодира с естествено число.
  Това означава, че съществуват изброимо безкрайно много машини на Тюринг.
  От друга страна, съществуват неизброимо много функции от вида $f:\Sigma^\star \to \Sigma^\star$.
\end{hint}

% Нека е дадена функцията $f:\Nat^k \to \Nat$.
% Ще казваме, че $f$ е изчислима с машината на Тюринг $\M$,
% ако за всяко $n_1,\dots,n_k$ е изпълнено:
% \begin{itemize}
% \item 
%   Представяме всяко от числата $n_1,\dots,n_k$ в монадична бройна система
%   като лентата на $\M$ има вида:  
%   \[\dots \blank \blank \underbrace{1111\dots 11}_{n} \blank\blank\dots,\]
%   като изискваме главата на $\M$ да е позиционирана върху най-лявата единица.
%   Такава конфигурация ще наричаме {\bf стандартна начална конфигурация}.
% \item
%   Ако $f(n_1,\dots,n_k) = m$, то $\M$ завършва с резултат върху лентата
%   \[\dots \blank \blank \underbrace{1111\dots 11}_{m} \blank\blank\dots,\]
%   като главата на $\M$ е върху най-лявата 1-ца.
%   Такава конфигурация се нарича {\bf стандартна финална конфигурация}.
% \item
%   Ако $f(n_1,\dots,n_k)$ е недефенирана, то $\M$ няма да завърши в стандартна конфигурация, т.е.
%   или $\M$ ще работи безкрайно време, или ще завърши в конфигурация, която не е стандартна.
% \end{itemize}

\begin{example}
  Да разгледаме функцията $f:\{1\}^\star \to \{1\}^\star$
  дефинирана като $|f(\alpha)| = 2|\alpha|$.
  
\begin{framed}
\begin{figure}[H]
  \begin{center}
    \begin{tikzpicture}[->,>=stealth,thick,node distance=50pt]
      \tikzstyle{every state}=[circle,minimum size=10pt,auto,scale=.7]
      
      \node[state,initial below]    (1) {$q_0$};
      \node[state]            (2) [right of=1]{$q_1$};
      \node[state]            (3) [right of=2]{$q_2$};
      \node[state]            (4) [right of=3]{$q_3$};
      \node[state]            (5) [right of=4]{$q_4$};
      \node[state]            (6) [right of=5]{$q_5$};
      \node[state]            (7) [right of=6]{$q_6$};
      \node[state]            (8) [right of=7]{$q_7$};
      \node[state]            (9) [right of=8]{$q_8$};
      \node[state]            (10) [right of=9]{$q_9$};
      \node[state]            (11) [below of=8]{$q_{10}$};
      \node[state,accepting]  (12) [below of=11]{$q_{11}$};
      
      \begin{scope}[every node/.style={scale=.8}]
      \path
      (1) edge [bend left=15] node [above] {$1;\goleft$} (2)
      (1) edge [bend right=30] node [above] {$\blank;\stay$} (12)
      (2) edge [bend left=15] node [above] {$1;\goleft$} (3)
      (2) edge [bend right=15] node [below] {$\blank;\goleft$} (3)
      (3) edge [bend left=15] node [above] {$\blank/1;\goleft$} (4)
      (4) edge [bend left=15] node [above] {$\blank/1;\goright$} (5)
      (5) edge [loop below] node [below] {$1;\goright$} (5)
      (5) edge [bend left=15] node [above] {$\blank;\goright$} (6)
      (6) edge [loop below] node [below] {$1;\goright$} (6)
      (6) edge [bend left=15] node [above] {$\blank;\goleft$} (7)
      (7) edge [bend left=15] node [above] {$1/\blank;\goleft$} (8)
      (8) edge [bend left=15] node [above] {$1;\goleft$} (9)
      (9) edge [loop below] node [below] {$1;\goleft$} (9)
      (9) edge [bend right=15] node [below] {$\blank;\goleft$} (10)
      (10) edge [loop below] node [below] {$1;\goleft$} (10)
      (10) edge [out=140,in=60, above] node [below] {$\blank;\goright$} (2)
      (8) edge [] node [right] {$\blank;\goleft$} (11)
      (11) edge [loop left] node [left] {$1;\goleft$} (11)
      (11) edge [] node [right] {$\blank;\goright$} (12);
      \end{scope}
    \end{tikzpicture}
  \end{center}
\end{figure}
\end{framed}

\begin{align*}
  (q_0, \underline{1}1) & \vdash (q_1, \underline{\blank}11) \vdash  (q_2, \underline{\blank} \blank 11) \vdash  (q_3, \underline{\blank} 1 \blank 11)\\
                        & \vdash (q_4, 1\underline{1}\blank 11) \vdash (q_4, 11 \underline{\blank} 11) \vdash (q_5, 11\blank \underline{1}1)\\
                        & \vdash \cdots \vdash (q_7, 11 \blank \underline{1}\blank) \vdash \cdots
\end{align*}

\end{example}


\begin{example}
  Да видим защо тоталната функция $f:\{a,b\}^\star \to \{a,b\}^\star$, дефинирана като
  $f(\alpha) = \alpha\cdot\alpha$ е изчислима с машина на Тюринг.
  
  \begin{itemize}
  \item
    $\Sigma = \{a,b\}$;
  \item 
    $\Gamma = \{a,b,x,A,B\}$;
  \item
    $\qstart = q_0$;
  \item
    $\qaccept = q_6$
  \end{itemize}
  
  \begin{framed}
  \begin{figure}[H]
    \begin{center}
      \begin{tikzpicture}[->,>=stealth,thick,node distance=70pt]
        \tikzstyle{every state}=[circle,minimum size=10pt,auto,scale=.9]
        
        \node[state]            (1) {$q_0$};
        \node[state]            (2) [above of=1]{$q_1$};
        \node[state]            (3) [right of=2]{$q_2$};
        \node[state]            (4) [below of=1]{$q_3$};
        \node[state]            (5) [right of=4]{$q_4$};
        \node[state]            (6) [right of=1]{$q_5$};
        \node[state,accepting]  (7) [right of=6]{$q_6$};
        
        \begin{scope}[every node/.style={scale=.8}]
          \path
          (1) edge [bend left=15] node [left] {$a/x;\goright$} (2)
          (2) edge [loop above] node [above] {$\{a,b,A,B\};\goright$} (2)
          (2) edge [bend left=15] node [above] {$\blank/A;\goleft$} (3)
          (3) edge [loop right] node [right] {$\{a,b,A,B\};\goleft$} (3)
          (3) edge [bend right=15] node [right] {$x/a;\goright$} (1)
          (1) edge [bend right=15] node [left] {$b/x;\goright$} (4)
          (4) edge [loop below] node [below] {$\{a,b,A,B\};\goright$} (4)
          (4) edge [bend right=15] node [below] {$\blank/B;\goleft$} (5)
          (5) edge [loop right] node [right] {$\{a,b,A,B\};\goleft$} (5)
          (5) edge [bend left=15] node [right] {$x/b;\goright$} (1)
          (1) edge [loop left] node [left] {$A/a,B/b;\goright$} (1)
          (1) edge [bend left=15] node [above] {$\blank;\goleft$} (6)
          (6) edge [loop below] node [right] {$\{a,b\};\goleft$} (6)
          (6) edge [bend left=15] node [above] {$\blank;\goright$} (7);
        \end{scope}
      \end{tikzpicture}
    \end{center}
  \end{figure}
\end{framed}

Да проследим работата на $\M$ върху думата $ab$:

\begin{align*}
  (q_0, \underline{a}b) & \vdash (q_1, x\underline{b}) \vdash (q_1,xb\underline{\blank}) \vdash (q_2, x\underline{b}A) \vdash (q_2, \underline{x}bA)\\
                        & \vdash (q_0, a\underline{b}A) \vdash (q_3, ax\underline{A}) \vdash (q_3, axA\underline{\blank}) \vdash (q_4, ax\underline{A}B)\\
                        & \vdash (q_4, a\underline{x}AB) \vdash (q_0, ab\underline{A}B) \vdash (q_0,aba\underline{B}) \vdash (q_0, abab\underline{\blank})\\
                        & \vdash (q_5, aba\underline{b}) \vdash (q_5, ab\underline{a}b) \vdash (q_5, a\underline{b}ab) \vdash (q_5, \underline{a}bab)\\
                        & \vdash (q_5, \underline{\blank}abab) \vdash (q_6, \underline{a}bab).
\end{align*}
\end{example}

\begin{example}
  \marginpar{Изискваме $f(\alpha)$ да започва с $1$ за да може $f$ да бъде функция}
  Да разгледаме тоталната функция 
  \[f:\{0,1\}^\star \to 1\cdot\{0,1\}^\star,\]
  дефинирана като
  \[\ov{f(\alpha)}_{(2)} = \ov{\alpha}_{(2)} + 1.\]
  Нека да видим, че тази функция е изчислима с машина на Тюринг.

  \begin{itemize}
  \item 
    $\Sigma = \{0,1\}$;
  \item
    $\Gamma = \{0,1,\blank\}$;
  \item
    $\qstart = q_0$;
  \item
    $\qaccept = q_4$.
  \end{itemize}

  \begin{framed}
  \begin{figure}[H]
    \begin{center}
      \begin{tikzpicture}[->,>=stealth,thick,node distance=70pt]
        \tikzstyle{every state}=[circle,minimum size=10pt,auto,scale=.9]
        
        \node[state,initial below]    (0) {$q_0$};
        \node[state]            (1) [right of=0]{$q_1$};
        \node[state]            (2) [right of=1]{$q_2$};
        \node[state]            (3) [right of=2]{$q_3$};
        \node[state,accepting]  (4) [right of=3]{$q_4$};
        
        \begin{scope}[every node/.style={scale=.8}]
          \path
          (0) edge [loop above] node [above] {$0/\blank;\goright$} (0)
          (0) edge [bend left=15] node [above] {$1;\goright$} (1)
          (0) edge [bend right=30] node [below] {$\blank;\stay$} (2)
          (1) edge [loop above] node [above] {$\{0,1\};\goright$} (1)
          (1) edge [bend left=15] node [above] {$\blank;\goleft$} (2)
          (2) edge [loop above] node [above] {$1/0;\goleft$} (2)
          (2) edge [bend left=15] node [above] {$0/1;\goleft$} (3)
          (2) edge [bend right=30] node [below] {$\blank/1;\stay$} (4)
          (3) edge [loop above] node [above] {$\{0,1\};\goleft$} (3)
          (3) edge [bend left=15] node [above] {$\blank;\goright$} (4);
        \end{scope}
      \end{tikzpicture}
    \end{center}
  \end{figure}
\end{framed}

Да проследим изчислението на $\M$ върху вход $01011$.

\begin{align*}
  (q_0, 0\underline{1}011) & \vdash (q_0,\underline{1}011) \vdash (q_1, 1\underline{0}11) \vdash (q_1, 10\underline{1}1)\\
                           & \vdash (q_1, 101\underline{1}) \vdash (q_1, 1011\underline{\blank}) \vdash (q_2, 101\underline{1})\\
                           & \vdash (q_2, 10\underline{1}0) \vdash (q_2, 1\underline{0}00) \vdash (q_3, \underline{1}100)\\
                           & \vdash (q_3, \underline{\blank}1100) \vdash (q_4, \underline{1}100).
\end{align*}
\end{example}


\begin{problem}
  Да разгледаме азбуката $\Sigma = \{0,1,\dots,k-1\}$, където $k > 2$.
  Да разгледаме тоталната функция 
  \[f:\Sigma^\star \to (\Sigma\setminus\{0\})\cdot\Sigma^\star,\]
  дефинирана като
  \[\ov{f(\alpha)}_k = \ov{\alpha}_k + 1.\]
  Дефинирайте машина на Тюринг $\M$, която изчислява функцията $f$.
\end{problem}


%%% Local Variables:
%%% mode: latex
%%% TeX-master: "../eai"
%%% End:


\newpage
\section{Недетерминистични машини на Тюринг}
\index{машина на Тюринг!недетерминистична}

Една машина на Тюринг $\N$ се нарича недетерминистична, ако функцията на преходите има вида
\[\Delta_\N: Q\times \Gamma \to \Ps(Q \times \Gamma\times \{\goleft,\goright,\stay\}). \]

Отново можем да дефинираме бинарна релация $\vdash_\N$ над $\Gamma^\star \times Q \times \Gamma^\star$,
която ще казва как моментното описание на машината $\N$ се променя при изпълнение на една стъпка.
\begin{itemize}
\item
  Ако $\Delta_\N(q,z) \ni (p,y,\goright)$, то дефинираме $(\alpha, q, z\beta) \vdash_\N (\alpha y, p, \beta)$.
\item 
  Ако $\Delta_\N(q,z) \ni (p,y,\goleft)$, то дефинираме $(\alpha x, q, z\beta) \vdash_\N (\alpha , p, xy\beta)$.
\item 
  Ако $\Delta_\N(q,z) \ni (p,y,\stay)$, то дефинираме $(\alpha, q, z\beta) \vdash_\N (\alpha, p, y\beta)$.
\end{itemize}
С $\vdash^\star_\N$ ще означаваме рефлексивното и транзитивно затваряне на $\vdash_\N$.

Тогава за недетерминистична машина на Тюринг $\N$, 
\[\L(\N) = \{\alpha\in\Sigma^\star \mid (\varepsilon, \qstart, \alpha) \vdash^\star_\N (\beta, \qaccept, \gamma), \text{ за някои }\beta,\gamma\in\Gamma^\star\}.\]

\begin{remark}
  Върху дадена дума $\omega$, недетерминистичната машина на Тюринг $\N$ може да има много различни изчисления.
  Думата $\omega$ принадлежи на $\L(\N)$ ако съществува {\em поне едно} изчисление, което завършва в състоянието $\qaccept$.
  Възможно е много други изчисления за $\omega$ да завършват в $\qreject$ или да зациклят.
\end{remark}

Аналогично, дефинираме една недетерминистична машина на Тюринг $\N$ да бъде {\bf тотална}, ако за всяка дума и 
всяко изчисление на $\N$ върху $\omega$ завършва в $\qaccept$ или $\qreject$.

\begin{problem}
  \marginpar{\cite{hopcroft2}}
  $\N = (\{q_0,q_1,q_2,q_f\}, \{0,1\}, \{0,1,\blank\}, \Delta, q_0, \{q_f\})$,
  \begin{itemize}
  \item 
    $\Delta(q_0,0) = \{(q_0,1,\goright),(q_1,1,\goright)\}$;
  \item
    $\Delta(q_1,1) = \{(q_2,0,\goleft)\}$;
  \item
    $\Delta(q_2,1) = \{(q_0,1,\goright)\}$;
  \item
    $\Delta(q_1,\blank) = \{(q_f,\blank,\goright)\}$.
  \end{itemize}
  \marginpar{$\{0^{n+1}1^k \mid n,k\in\Nat\}$}
  Опишете $\L(\N)$.
\end{problem}

\begin{example}
  Да разгледаме езика  
  \[L = \{\alpha\sharp\beta \mid \alpha,\beta \in \{a,b\}^\star\ \&\ \alpha\text{ е подниз на }\beta\}.\]
  Ще видим, че този език е разрешим като построим недетерминистична машина на Тюринг $\N$,
  която разрешава този език.
  \begin{framed}
    \begin{figure}[H]
      \begin{center}
        \begin{tikzpicture}[->,>=stealth,thick,node distance=50pt]
          \tikzstyle{every state}=[circle,minimum size=10pt,scale=.9]
          
          \node[state,initial]    (1) {$q_0$};
          \node[state]            (2) [right of=1]{$q_1$};
          \node[state]            (3) [right of=2,node distance=70pt]{$q_2$};
          \node[state]            (4) [below of=3]{$q_3$};
          \node[state]            (5) [below right of=4,node distance=70pt]{$q_4$};
          \node[state]            (6) [right of=4]{$q_5$};
          \node[state]            (7) [above of=6]{$q_6$};
          \node[state]            (8) [right of=6,node distance=80pt]{$q_7$};
          \node[state]            (9) [right of=7,node distance=70pt]{$q_8$};
          \node[state,accepting]  (10)[below right of=5]{$q_{9}$};
          
          \begin{scope}[every node/.style={scale=.8}]
            \path
            (1) edge [loop above] node [above] {$\{a,b\};\goright$} (1)
            (1) edge [bend left=15] node [above] {$\sharp;\goright$} (2)
            (2) edge [loop above] node [above] {$a/\blank,b/\blank;\goright$} (2)
            (2) edge [bend left=15] node [above] {$\{a,b,\blank\};\goleft$} (3)
            (3) edge [loop above] node [above] {$\blank;\goleft$} (3)
            (3) edge [bend right=15] node [left] {$\sharp;\goleft$} (4)
            (4) edge [loop left] node [left] {$\{a,b\};\goleft$} (4)
            (4) edge [bend right=30] node [left] {$\blank;\goright$} (5)
            (5) edge [bend right=15] node [right] {$a/\blank;\goright$} (6)
            (6) edge [loop right] node [right] {$\{a,b\};\goright$} (6)
            (6) edge [bend right=15] node [right] {$\sharp;\goright$} (7)
            (7) edge [loop right] node [right] {$\blank;\goright$} (7)
            (7) edge [bend left=15] node [below] {$a/\blank;\goleft$} (3)
            (8) edge [loop right] node [right] {$\{a,b\};\goright$} (8)
            (5) edge [bend right=30] node [right] {$b/\blank;\goright$} (8)
            (8) edge [bend right=15] node [right] {$\sharp;\goright$} (9)
            (9) edge [loop right] node [right] {$\blank;\goright$} (9)
            (9) edge [bend right=45] node [above] {$b/\blank;\goleft$} (3)
            (5) edge [bend right=15] node [left] {$\sharp;\stay$} (10);
          \end{scope}
        \end{tikzpicture}
      \end{center}
    \end{figure}
  \end{framed}
  Да видим, че $\M$ успешно разпознава, че думата $ab\sharp aabb$ принадлежи на езика $L$.

  \begin{align*}
    (q_0, \underline{a}b\sharp aabb) & \vdash (q_0, a\underline{b}\sharp aabb) \vdash (q_0, ab\underline{\sharp} aabb) \vdash (q_1, ab\sharp\underline{a}abb) \vdash (q_1, ab\sharp\blank\underline{a}bb)\\
                                     & \vdash (q_2, ab\sharp\underline{\blank}abb) \vdash (q_2, ab\underline{\sharp}\blank abb) \vdash (q_3, a\underline{b}\sharp\blank abb) \vdash (q_3, \underline{a}b\sharp\blank abb)\\
                                     & \vdash (q_3, \underline{\blank}ab\sharp\blank abb) \vdash (q_4, \underline{a}b\sharp\blank abb) \vdash (q_5, \blank\underline{b}\sharp \blank abb) \vdash (q_5, \blank b\underline{\sharp} \blank abb)\\
                                     & \vdash (q_6, \blank b \sharp \underline{\blank} abb) \vdash (q_6, \blank b \sharp \blank \underline{a}bb) \vdash (q_2, \blank b \sharp \underline{\blank}\blank bb) \vdash (q_2, \blank b \underline{\sharp} \blank\blank bb)\\
                                     & \vdash (q_3, \blank \underline{b} \sharp \blank\blank bb) \vdash (q_3, \underline{\blank} b \sharp \blank\blank bb) \vdash (q_4,  \blank \underline{b} \sharp \blank\blank bb) \vdash (q_7, \blank \blank \underline{\sharp} \blank\blank bb)\\
                                     & \vdash (q_8, \blank \blank \sharp \underline{\blank}\blank bb)\vdash (q_8, \blank \blank \sharp \blank \underline{\blank} bb) \vdash (q_8, \blank \blank \sharp \blank \blank \underline{b}b)\\
    & \vdash (q_2, \blank \blank \sharp \blank \underline{\blank} \blank b) \vdash \cdots \vdash (q_4, \blank\blank\underline{\sharp}\blank\blank\blank b) \vdash (q_9, \blank\blank\underline{\sharp}\blank\blank\blank b)
  \end{align*}
\end{example}

\subsection*{Канонична подредба на $\Sigma^\star$}

\marginpar{За доказателството, че всяка НМТ е еквивалентна на ДМТ, е необходимо да фиксираме канонична подредба на думите над дадена азбука}
Нека $\Sigma = \{a_0,a_1,\dots,a_{k-1}\}$.
Подреждаме думите по ред на тяхната дължина.
Думите с еднаква дължина подреждаме по техния числов ред, т.е.
гледаме на буквите $a_i$ като числото $i$ в $k$-ична бройна система.
Тогава думите с дължина $n$ са числата от $0$ до $k^n-1$ записани в $k$-ична бройна система.
Ще означаваме с $\omega_i$ $i$-тата дума в $\Sigma^\star$ при тази подредба.

\begin{example}
  Ако $\Sigma = \{0,1\}$, то наредбата започва така:
  \[\varepsilon, 0, 1, \underbrace{00, 01, 10, 11}_{\text{от $0$ до $3$}}, \underbrace{000, 001, 010, 011, 100, 101, 110, 111}_{\text{от $0$ до $7$}}, 0000, 0001, \dots\]
  В този случай, $\omega_0 = \varepsilon$, $\omega_7 = 000$, $\omega_{13} = 110$.
\end{example}

\begin{problem}
  Нека $\Sigma = \{a_0,\dots,a_{k-1}\}$.
  Да разгледаме функцията $f:\Sigma^\star \to \Sigma^\star$, за която 
  $f(\alpha)$ е думата веднага след $\alpha$ в каноничната подредба на $\Sigma^\star$.
  Докажете, че $f$ е изчислима с машина на Тюринг.
\end{problem}
\begin{hint}
  Ако $\Sigma = \{0,1\}$, то машината на Тюринг има следния вид:
  \begin{framed}
    \begin{figure}[H]
      \begin{center}
        \begin{tikzpicture}[->,>=stealth,thick,node distance=70pt]
          \tikzstyle{every state}=[circle,minimum size=10pt,auto,scale=.9]
          
          \node[state,initial]    (1) [right of=0]{$q_1$};
          \node[state]            (2) [right of=1]{$q_2$};
          \node[state]            (3) [right of=2]{$q_3$};
          \node[state,accepting]  (4) [right of=3]{$q_4$};
          
          \begin{scope}[every node/.style={scale=.8}]
            \path
            (1) edge [loop above] node [above] {$\{0,1\};\goright$} (1)
            (1) edge [bend left=15] node [above] {$\blank;\goleft$} (2)
            (2) edge [loop above] node [above] {$1/0;\goleft$} (2)
            (2) edge [bend left=15] node [above] {$0/1;\goleft$} (3)
            (2) edge [bend right=30] node [below] {$\blank/0;\stay$} (4)
            (3) edge [loop above] node [above] {$\{0,1\};\goleft$} (3)
            (3) edge [bend left=15] node [above] {$\blank;\goright$} (4);
          \end{scope}
        \end{tikzpicture}
      \end{center}
    \end{figure}
  \end{framed}
\end{hint}


\begin{framed}
  \begin{thm}
    Ако $L$ се разпознава от {\em недетерминистична} машина на Тюринг $\N$, то $L$
    е разпознава и от {\em детерминистична} машина на Тюринг $\D$.
  \end{thm}
\end{framed}
\begin{proof}
  \marginpar{В \cite[стр. 164]{hopcroft1} не е добре обяснено.}
  Нека имаме недетерминистичната машина на Тюринг $\N$, за която $L = \L(\N)$.
  Една дума $\alpha$ принадлежи на $\L(\N)$ точно тогава, когато съществува изчисление,
  което започва с думата $\alpha$ върху лентата и след краен брой стъпки, следвайки функцията на преходите $\Delta_\N$,
  достига до състоянието $\qaccept$.
  Сложността идва от факта, че за думата $\alpha$ може да имаме много различни изчисления, 
  като само някои от тях завършват в $\qaccept$. Ще построим детерминистична машина на Тюринг,
  която последователно ще симулира всички възможни {\em крайни} изчисления за думата $\alpha$, докато 
  намери такова, което завършва в състоянието $\qaccept$.
  \marginpar{На практика това, което е правим е да представим всички възможни изчисления на $\N$ като $r$-разклонено дърво и да го обходим в широчина, докато не достигнем до $\qaccept$}
  
  Лесно се съобразява, че всяко изчисление на $\N$ може да се представи като 
  крайна редица от елементи на $Q \times \Gamma \times \{\goleft,\goright,\stay\}$.
  Понеже това множество е крайно, то можем на всяка такава тройка да
  съпоставим число в интервала $[1,r]$, където 
  \[r = 3 \cdot |Q| \cdot |\Gamma|.\]
  Оттук следва, че всяко изчисление на $\N$ може да се представи като крайна 
  редица от числа, всяко принадлежащо на интервала $[1,r]$.

  Детерминистичната машина на Тюринг $\D$ има три ленти.
  
  \begin{itemize}
  \item 
    На първата лента съхраняваме входящия низ и {\em тя никога не се променя}.
  \item
    На втората лента ще записваме последователно низове следвайки каноничната подредба на 
    думите над азбуката $\{1,2,\dots,r\}$.
  \item
    На третата лента симулираме изчислението на $\N$ върху думата от първата лента, използвайки изчислението, 
    което е описано на втората лента. Например, ако съдържанието на втората лента е $4,1,2$,
    това означава, че симулираме изчисление от три стъпки като на първата стъпка избираме четвъртата
    възможна тройка, на втората стъпка избираме първата възможна тройка, на третата стъпка избираме втората възможна тройка.
    
    Ако симулацията завърши в състоянието $\qaccept$ на $\N$, то машината $\D$ завършва успешно.
    В противен случай, на втората лента записваме следващия низ; изтриваме третата лента и започваме нова симулация.
  \end{itemize}
\end{proof}

\begin{prop}[Лема на Кьониг]
  \index{Кьониг}
  Ако $T$ е безкрайно дърво с крайно разклонение, то $T$ съдържа безкраен път.
\end{prop}
\begin{hint}
  Дефинираме безкрайния път на стъпки.
  На всяка стъпка избираме този наследник, който е корен на безкрайно дърво.
  Понеже $T$ е безкрайно дърво с крайно разклонение, на всяка стъпка можем да изберем такъв наследник.
\end{hint}

\begin{cor}
  Ако $L$ се разпознава от {\em тотална недетерминистична} машина на Тюринг $\N$, то $L$
  също се разпознава и от {\em тотална детерминистична} машина на Тюринг $\D$.
\end{cor}
\begin{proof}
  Да разгледаме дървото $T$, което представя всички изчисления на тоталната $\N$ при вход думата $\omega$.
  От лемата на Кьониг следва, че $T$ е крайно дърво, защото ако допуснем, че $T$ е безкрайно, то ще има безкрайно дълго изчисление на $\N$,
  което е невъзможно, понеже $\N$ винаги достига до финално състояние.
  \begin{itemize}
  \item 
    Ако $\N$ приема дадена дума $\omega$, то детерминистичната ни симулация на $\N$ ще достигне до изчисление, кодирано като път в $T$, 
    което завършва в състояние $\qaccept$.
  \item
    Ако $\N$ не приема дадена дума $\omega$, то детерминистичната ни симулация на $\N$ ще покаже, че всяко изчисление, кодирано като път в $T$, завършва в състояние $\qreject$.
  \end{itemize}
\end{proof}



%%% Local Variables:
%%% mode: latex
%%% TeX-master: "../eai"
%%% End:



\section{Основни свойства}

\begin{proposition}
  Ако $L$ е разрешим език над азбуката $\Sigma$, то $\Sigma^\star \setminus L$ също е разрешим език.
  \mynote{С други думи, разрешимите езици са затворени относно операцията допълнение.
    След малко в \Proposition{diagonal:accept} ще видим, че това твърдение не е изпълнено за полуразрешими езици.}
\end{proposition}
\begin{hint}
  Нека $L = \L(\M)$, където $\M$ е разрешител.
  Нека $\M'$ е същата като $\M$, само със сменени $\qaccept$ и $\qreject$ състояния.
  Тогава $\M'$ също е разрешител и $\ov{L} = \L(\M')$.
\end{hint}

\begin{proposition}
  Ако $L_1$ и $L_2$ са разрешими езици, то $L_1 \cup L_2$ е разрешим език.
  \mynote{С други думи, разрешимите езици са затворени относно операцията обединение.
    Като следствие получаваме, че всяко \emph{крайно} обединение на разрешими езици е разрешим език.

    \writedown Съобразете, че това твърдение е изпъленено и за полуразрешими езици.}
\end{proposition}
\begin{hint}
  Нека $L_1 = \L(\M_1)$ и $L_2 = \L(\M_2)$.
  Строим нова машина на Тюринг $\M$, която при вход думата $\alpha$
  симулира едновременно изчисленията на $\M_1$ и $\M_2$ върху $\alpha$.
  Това можем да направим като приемем, че $\M$ има две ленти - една за лентата на $\M_1$ и една за лентата на $\M_2$,
  като състоянията на $\M$ ще бъдат елементи на $Q_1 \times Q_2$.
  Ако една от двете машини достигне своето приемащо състояние, то $\M$ приема думата $\alpha$.
  Ако и двете машини достигнат своите отхвърлящи състояния, то $\M$ отхвърля думата $\alpha$.
\end{hint}

% \begin{proposition}
%   Ако $L_1$ и $L_2$ са полуразрешими езици, то $L_1 \cup L_2$ е полуразрешим език.
% \end{proposition}

\begin{proposition}
  Ако $L_1$ и $L_2$ са разрешими езици, то $L_1 \cap L_2$ е разрешим език.
  \mynote{С други думи, разрешимите езици са затворени относно операцията сечение.
    Като следствие получаваме, че всяко \emph{крайно} сечение на разрешими езици е разрешим език.

    \writedown Съобразете, че това твърдение е изпъленено и за полуразрешими езици.}
\end{proposition}
\begin{hint}
  Нека $L_1 = \L(\M_1)$ и $L_2 = \L(\M_2)$.
  Строим нова машина на Тюринг $\M$, която при вход думата $\alpha$
  симулира едновременно изчисленията на $\M_1$ и $\M_2$ върху $\alpha$.
  Ако и двете машини достигнат до приемащите си състояния, то $\M$ приема думата $\alpha$.
  Ако поне една от двете машини достигне до отхвърлящо състояние, то $\M$ отхвърля думата $\alpha$.
\end{hint}

% \begin{corollary}
%   Всяко крайно сечение на разрешими езици е разрешим език.
% \end{corollary}

% \begin{important}
%   \begin{theorem}
%     Разрешимите езици са затворени относно операциите обединение, сечение, допълнение.
%   \end{theorem}
% \end{important}


\index{Клини-Пост}
\begin{important}
  \begin{theorem}[Клини-Пост]
    $L$ и $\ov{L}$ са полуразрешими езици точно тогава, когато $L$ е разрешим език.
  \end{theorem}
\end{important}
\begin{hint}
  Посоката $(\Leftarrow)$ е ясна.
  За посоката $(\Rightarrow)$, нека $L = \L(\M_1)$ и $\ov{L} = \L(\M_2)$.
  Строим разрешител $\M$, която при вход думата $\alpha$ симулира едновременно изчисленията на $\M_1$ и $\M_2$ върху $\alpha$.
  Например, може $\M$ да има две ленти за симулацията на $\M_1$ и $\M_2$.
  Знаем със сигурност, че точно едно от двете симулирани изчисления ще завърши в приемащо състояние.
  Ако това е $\M_1$, то $\M$ приема $\alpha$.
  Ако това е $\M_2$, то $\M$ отхвърля $\alpha$.
\end{hint}


%%% Local Variables:
%%% mode: latex
%%% TeX-master: "../eai"
%%% End:


% \section{Машини на Тюринг като генератори}

% \mynote{\cite[стр. 168]{hopcroft1}}
% \mynote{\cite[стр. 180]{sipser3}}
% \mynote{На англ. се наричат {\em enumerators}}

% Нека да разгледаме един вариант на многолентовите машини на Тюринг, които ще наричаме {\bf генератори}.
% Нека машината на Тюринг да има две ленти, като в началото и двете ленти са празни.
% \begin{itemize}
% \item 
%   Първата лента ще служи за работна лента - върху нея можем да пишем и четем;
% \item
%   Втората лента служи единствено за изход - върху нея можем само да пишем пишем думи; не можем да четем какво вече сме написали върху нея и не можем да пишем върху вече записана клетка. Думите са разделени със специален символ - $\#$.
%   Това означава, че втората лента има вида
%   \[\omega_1\#\omega_2\#\cdots\#\omega_n\#\blank\blank\cdots\]
% \item
%   Езикът, които се извежда от такъв генератор е съставен от думите, които са изписани на изходната лента.
%   Такива езици ще наричаме {\bf изчислимо изброими}.
%   Обърнете внимание, че измежду думите на изходната лента е възможно да има повторения.
%   Ако езикът е безкраен, то машината ще работи безкрайно много време.
% \end{itemize}

% \begin{framed}
%   \begin{thm}
%     Един език $L$ е полуразрешим точно тогава, когато $L$ е изчислимо номеруем.
%   \end{thm}
% \end{framed}
% \begin{proof}
%   $(\Leftarrow)$ Нека $L$ да се номерира от генераторът $E$.
%   Машината на Тюринг $\M$, за която $L = \L(\M)$ ще работи по следния начин:
%   \begin{enumerate}[1)]
%   \item 
%     При вход думата $\omega$, $\M$ започва да симулира $E$;
%   \item
%     Когато се появи дума $\gamma$ върху изходната лента на $E$, сравняваме $\omega$ с $\gamma$;
%   \item
%     Ако $\omega = \gamma$, то отиваме в състоянието $q_{accept}$ на $\M$ и завършваме;
%   \item
%     В противен случай, отиваме обратно на стъпка $2)$.
%   \end{enumerate}

%   $(\Rightarrow)$ Нека сега $L = \L(\M)$. Целта ни е да изведем всички думи на $L$ върху изходната лента.
%   Основният проблем е, че за дадена дума $\omega$, не знаем за колко стъпки трябва да симулираме $\M$ за да сме сигурни дали думата $\omega \in \L(\M)$ или не. Оказва се, че можем да разрешим този проблем като позволяме да извеждаме повторящи се думи.
%   За целта, да подредим всички думи $\omega_1, \omega_2, \dots $ над азбуката $\Sigma$ спрямо каноничната наредба.
%   \begin{enumerate}[1)]
%   \item
%     Нека $s = 1$;
%   \item 
%     Симулираме $\M$ върху думите $\omega_1,\dots,\omega_s$ за $s$ стъпки;
%   \item
%     За всяка от тези думи $\omega_i$, които се приемат от $\M$, записваме ги върху изходната лента.
%   \item
%     Нека $s = s+1$; Отиваме обратно на стъпка $2)$.
%   \end{enumerate}
% \end{proof}

% \begin{remark}
%   В последната конструкция позволяваме думите на един полуразрешим език $L$ да се 
%   извеждат върху изходната лента многократно. Можем лесно да осигурим условието всяка дума на $L$
%   да се извежда точно по веднъж.
%   На стъпка $s = \pair{i,j}$, то проверяваме дали думата $\omega_i$ се приема успешно от $\M$
%   за {\em точно} $j$ на брой стъпки. Само тогава думата се записва на изходната лента.
  
%   Обърнете внимание, че не можем да осигурим условието думите да се извеждат във възходящ ред
%   относно каноничната наредба.
% \end{remark}

% \begin{framed}
%   \begin{thm}
%     Един език $L$ е разрешим точно тогава, когато съществува генератор за $L$, 
%     който изписва думите на $L$ във възходящ ред относно каноничната наредба.
%   \end{thm}
% \end{framed}
% \begin{proof}
%   $(\Rightarrow)$ Нека $L = \L(\M)$. Тази посока е лесна, защото $\M$ е тотална машина,
%   т.е. за всеки вход $\M$ завършва или в $q_{accept}$ или в $q_{reject}$.
%   \begin{enumerate}[1)]
%   \item 
%     Нека $s = 1$;
%   \item
%     Симулираме $\M$ върху думата $\omega_s$.
%   \item
%     Ако симулацията завърши в състояние $q_{accept}$, то записваме $\omega_s$
%     върху изходната лента. 
%   \item
%     Иначе ако симулацията завърши в състояние $q_{reject}$, то нищо не записваме върху изходната лента. 
%   \item
%     Нека $s = s+1$. Отиваме на стъпка $2)$.
%   \end{enumerate}

%   \mynote{Ако имам генератор $G$ за $L$ няма алгоритъм, който да ми каже дали $L$ е безкраен език или не. Това означава, че по код на $G$ няма как ефективно да получа код на $\M$}
%   $(\Leftarrow)$ Ако $L$ е краен, то е ясно, че мога да разпозная езика с краен автомат, което е частен случай на тотална машина на Тюринг.
%   По-интересният случай е когато $L$ е безкраен език.
%   Нека $L$ се генерира от машината на Тюринг $G$ като извежда думите на $L$ във възходящ ред.
%   \begin{itemize}
%   \item 
%     Вход дума $\omega$;
%   \item
%     Симулираме $G$ като гледаме думите, които се извеждат на изходната лента.
%     Ако срещнем думата $\omega$, то завършваме в състояние $q_{accept}$.
%   \item
%     Ако срещнем думата $\gamma$, която е по-голяма от $\omega$ относно каноничната наредба, 
%     то завършваме в състояние $q_{reject}$.
%   \end{itemize}
% \end{proof}

\subsection{Кодиране на машина на Тюринг}

\subsection*{Кодиране на преход}
Да разгледаме прехода $\delta(q_i,X_j) = (q_k,X_l,D_m)$.
Кодираме този преход по следния начин:
\[0^i10^j10^k10^l10^m\]
Да обърнем внимание, че в този двоичен код няма последователни единици и той 
започва и завършва с нула.


За да кодираме една машина на Тюринг $\M$ е достатъчно да кодираме функцията на преходите $\delta$.
Понеже $\delta$ е крайна функция, нека с числото $r$ да означим броя на всички възможни преходи.
По описания по-горе начин, нека $code_i$ е числото в двоичен запис, получено за $i$-тия преход на $\delta$.
Тогава кодът на $\M$ е следното число в двоичен запис:
\[\code{\M} \df 111\ \texttt{code}_1\ 11\ \texttt{code}_2\ 11\ \cdots\ 11\ \texttt{code}_r\ 111.\]
\begin{itemize}
\item
  Лесно се съобразява, че за две МТ $\M$ и $\M'$ с различни функции на преходите, имаме $\code{\M} \neq \code{\M'}$.
% \item
%   Ще казваме, че числото $r\in\Nat$ е {\bf код на} $\M$, ако $r$, записано в двоичен запис представлява думата $\code{\M}$.
%   Оттук нататък, когато пишем $\M_r$, ще имаме предвид машината на Тюринг с код $r$.
% \item
%   Ясно е, че не всяко естествено число е код на машина на Тюринг, но по дадено число $n$
%   има ефективна процедура, която ни казва дали $n$ е код на машина на Тюринг или не.
% \item
  % С $\pair{\M,\omega}$ ще означаваме кода на $\M$ при вход $\omega$ е числото с двоичен запис описанието на $\M$ и след това прикрепена думата $\omega$.
  % При едно число $r = \pair{M,\omega}$, лесно се намира кода на $\M$.
  % Просто започваме да четем двоичния запис на $r$ докато не срещнем за втори път $111$.
  % След това започва думата $\omega$.
% \item
%   Да въведем означението $\M_i$ за произволно ествестено число $i$.
%   Ако $i$ е код на машина на Тюринг $\M$, то $\M_i \df \M$.
%   Ако $i$ не е код на машина на Тюринг, то $\M_i$ е машина на Тюринг с празна функция на преходите.
\end{itemize}

\begin{example}
  Да се даде пример за кода на конкретна машина на Тюринг.
\end{example}


\begin{prop}
  Следните езици са разрешими:
  \begin{itemize}
  \item 
    $L = \{\code{\M} \mid \M\text{ е машина на Тюринг}\}$;
  \item
    $L = \{\code{\M} \mid \M\text{ е детерминистична машина на Тюринг}\}$.
  \end{itemize}
\end{prop}

\begin{remark}
  Следният език {\bf не} е разрешим:
  \[L_{\texttt{tot}} = \{\code{\M} \mid \M\text{ е тотална машина на Тюринг}\}.\]
\end{remark}


%%% Local Variables:
%%% mode: latex
%%% TeX-master: "../eai"
%%% End:


\subsection{Диагоналният език $L_{\texttt{diag}}$}

\newcommand{\Luniv}{L_{\texttt{univ}}}
\newcommand{\Lhalt}{L_{\texttt{halt}}}
\newcommand{\Laccept}{L_{\texttt{accept}}}

\index{език!неполуразрешим}
\begin{important}
  \begin{theorem}
    Езикът 
    \[L_{\texttt{diag}} \df \{\ \code{\M} \mid \M \text{ е машина на Тюринг и }\code{\M} \not\in L(\M)\ \}\]
    не се разпознава от машина на Тюринг, т.е. $L_{\texttt{diag}}$ {\bf не} е полуразрешим език.
  \end{theorem}
\end{important}
\begin{proof}
  Да допуснем, че $L_{\texttt{diag}}$ се разпознава от машина на Тюринг $\M_0$, т.е. 
  \[L_{\texttt{diag}} = \L(\M_0).\]
  Тогава да видим какво имаме за думата $\code{\M_0}$:
  \begin{align*}
    & \code{\M_0} \in L_{\texttt{diag}} \implies \code{\M_0} \in \L(\M_0) \implies \code{\M_0} \not\in L_{\texttt{diag}},\\
    & \code{\M_0} \not\in L_{\texttt{diag}} \implies \code{\M_0} \not\in \L(\code{\M_0}) \implies \code{\M_0} \in L_{\texttt{diag}}.
  \end{align*}
  Достигаме до противоречие.
\end{proof}

\index{език!полуразрешим}
\begin{important}
  \begin{proposition}
    Езикът 
    \[\Laccept \df \{\ \code{\M} \mid \text{$\M$ е машина на Тюринг и }\code{\M} \in \L(\M)\ \}\]
    е полуразрешим, но не е разрешим.
  \end{proposition}  
\end{important}
\begin{hint}
  Лесно се съобразява, че $\Laccept$ е полуразрешим.
  Дефинираме (многолентова) машина на Тюринг $\M'$, която работи по следния начин:
  \begin{itemize}
  \item
    вход дума $\alpha$;
  \item 
    $\M'$ проверява дали $\alpha$ има вида $\code{\M}$,
    за някоя машина на Тюринг $\M$;
  \item
    Ако $\alpha$ няма вида $\code{\M}$,
    то $\M'$ завършва като отхвърля думата $\alpha$.
  \item
    Ако $\alpha = \code{\M}$, 
    то $\M'$ симулира работата на $\M$ върху $\alpha$. Тогава:
    \begin{itemize}
    \item 
      Ако $\M$ завърши след краен брой стъпки като приема $\alpha$,
      то $\M'$ приема $\alpha$.
    \item
      Ако $\M$ завърши след краен брой стъпки като отхвърля $\alpha$,
      то $\M'$ отхвърля $\alpha$.
    \item
      Ако $\M$ никога не завършва върху $\alpha$,
      то $\M'$ също никога не завършва върху $\alpha$.
    \end{itemize}
  \end{itemize}
  Получаваме, че
  \[\alpha \in \Laccept \iff \alpha \in \L(\M'),\]
  откъдето следва, че $\Laccept$ е полуразрешим език.

  Ако допуснем, че $\Laccept$ е разрешим,
  то езикът $\{0,1\}^\star \setminus \Laccept$ би бил разрешим и тогава 
  \[L_{\texttt{diag}} = (\{0,1\}^\star \setminus \Laccept) \cap \{\code{\M} \mid \text{$\M$ е машина на Тюринг}\}\]
  ще е разрешим език, което е противоречие, защото ще следва, че $L_{\texttt{diag}}$ е разрешим език, а той не е е дори полуразрешим.
\end{hint}

\begin{corollary}
  Съществуват езици, които са полуразрешими, но не са разрешими.
\end{corollary}

\begin{corollary}
  Съществуват полуразрешими езици $L$, за които $\ov{L}$ не са полуразрешими.
\end{corollary}

\begin{problem}
  Докажете, че езикът
  \[\Lhalt = \{\code{\M} \mid \M\text{ е машина на Тюринг и }\M\text{ спира при вход }\code{\M}\}\]
  е полуразрешим, но не е разрешим.
\end{problem}



%%% Local Variables:
%%% mode: latex
%%% TeX-master: "../eai"
%%% End:


\subsection{Универсалният език $L_{\texttt{univ}}$}

\begin{framed}
  \begin{thm}
    Езикът 
    \[\Luniv \df \{\ \code{\M} \cdot \omega \mid \text{$\M$ е машина на Тюринг и }\omega\in \L(\M)\ \}\]
    е полуразрешим, но {\bf не} е разрешим.
  \end{thm}
\end{framed}
\begin{hint}
  \marginpar{Разсъждението е много сходно с това защо $\Laccept$ полуразрешим.}
  Първо да съобразим защо $\Luniv$ е полуразрешим език.
  Дефинираме машина на Тюринг $\M'$, която работи по следния начин:
  \begin{itemize}
  \item
    вход дума $\alpha$;
  \item 
    $\M'$ проверява дали $\alpha$ има вида $\code{\M} \cdot \omega$,
    за някоя машина на Тюринг $\M$ и дума $\omega$. Това става лесно, защото $\omega$
    започва веднага след второ срещане на $111$ в $\alpha$.
  \item
    Ако $\alpha = \code{\M} \cdot \omega$, 
    то $\M'$ симулира работата на $\M$ върху $\omega$.
    \begin{itemize}
    \item 
      Ако $\M$ завърши след краен брой стъпки като приеме $\omega$,
      то $\M'$ приема $\alpha$.
    \item
      Ако $\M$ завърши след краен брой стъпки като отхвърли $\omega$,
      то $\M'$ отхвърля $\alpha$.
    \item
      Ако $\M$ никога не завършва върху $\omega$,
      то очевидно $\M'$ също никога не завършва върху $\alpha$.
    \end{itemize}
  \item
    Ако $\alpha$ няма вида $\code{\M} \cdot \omega$,
    то $\M'$ завършва веднага като отхвърля думата $\alpha$.
  \end{itemize}
  Получаваме, че
  \[\alpha \in \Luniv \iff \alpha \in \L(\M').\]
  
  Сега да съобразим защо $\Luniv$ не е разрешим език.
  Имаме, че за произволна дума $\omega$,
  \begin{align*}
    \omega \in \Laccept & \iff (\exists \M)[\M\text{ е М.Т.}\ \&\ \omega = \code{\M}\ \&\ \omega \in \L(\M)]\\
                                       & \iff \omega \cdot \omega \in \Luniv.
  \end{align*}
  Ако допуснем, че $\Luniv$ е разрешим, то тогава $\Laccept$ е разрешим език, което е противоречие.
\end{hint}

\begin{cor}
  Езикът
  \[\ov{\Luniv} \df \{\code{\M} \cdot \omega \mid \code{\M} \text{ е машини на Тюринг и }\omega\not\in \L(\M)\}\]
  {\bf не} е полуразрешим.
\end{cor}





%%% Local Variables:
%%% mode: latex
%%% TeX-master: "../eai"
%%% End:


\section{Критерий за разрешимост}

\mynote{Сипсър нарича $\leq_m$ \emph{mapping reducibility} \cite[235]{sipser3}.}

\begin{important}
  Доказателството, че $\Luniv$ не е разрешим е пример за една обща схема, с която можем да докажем, че даден език не е разрешим:
  \begin{itemize}
  \item 
    Нека имаме езика $K$, за който вече знаем, че не е разрешим.
    В нашия пример, $K = \Laccept$.
  \item
    Питаме се дали някой друг език $L$ е разрешим.
  \item
    Намираме изчислима тотална функция $f$, за която е изпълнено, че:
    \[\omega \in K \iff f(\omega) \in L.\]
    В \Theorem{universal}, това е функцията $f(\omega) = \omega \sharp \omega$.
  \item
    В този случай ще означаваме $K \leq_m L$.
  \item
    Тогава, ако $L$ е разрешим ще следва, че $K$ е разрешим, което е противоречие.
  \end{itemize}
\end{important}

Сега искаме да разгледаме един критерий, който ще ни казва кога един език съставен от кодове на машини на Тюринг е разрешим. С негова помощ ще можем директно да решаваме наглед трудни задачи. Например,
в момента не е очевидно защо следния език не е разрешим:
\begin{align*}
  L_{\texttt{palin}}\{\omega \in \{0,1\}^\star \mid & \ \omega\text{ е код на машина на Тюринг и }\L(\M)\\
                                                    & \text{ съдържа само думи палиндроми}\}.
\end{align*}
След малко ще видим, че според критерия, който ще разгледаме, директно ще можем да заключим, че $L_{\texttt{palin}}$ не е разрешим. Да започнем с няколко примера.

\begin{proposition}
  Докажете, че езикът
  \[L_{\Sigma^\star} \df \{\omega \in \{0,1\}^\star \mid \omega\text{ е код на машина на Тюринг и }\L(\M_\omega) = \Sigma^\star\}\]
  не е разрешим.
\end{proposition}
\begin{proof}
  \mynote{$L_{\Sigma^\star}$ не е дори полуразрешим, но за момента не знаем как да докажем това.}
  Ще покажем, че съществува тотална изчислима функция $f$, за която:
  \[\alpha \in \Laccept \iff f(\alpha) \in L_{\Sigma^\star}.\]

  Ще опишем алгоритъм (формално машина на Тюринг), за който при входна думата $\omega$ прави следното:
  \begin{itemize}
  \item
    Ако $\omega$ не е код на машина на Тюринг, то връщаме $\omega$.
  \item
    Ако $\omega$ е код на машина на Тюринг $\M$, то тук става интересно.
    Връщаме код на друга машина на Тюринг $\M'$, която работи по следния начин:
    \mynote{За различни $\M$ получаваме различни $\M'$.}
    \begin{itemize}
    \item 
      Вход дума $\alpha$;
    \item
      Първоначално $\M'$ не обръща внимание на $\alpha$.
    \item
      $\M'$ симулира работата на $\M$ върху думата $\code{\M}$;
      \begin{itemize}
      \item 
        Ако след краен брой стъпки $\M$ завърши като приеме думата $\code{\M}$,
        то $\M'$ приема думата $\alpha$, т.е. $\M'$ завършва в състоянието $q_{\texttt{accept}}$.
      \item
        Ако след краен брой стъпки $\M$ завърши като отхрърли думата $\code{\M}$,
        то $\M'$ отхвърля думата $\alpha$, т.е. $\M'$ завършва в състоянието $q_{\texttt{reject}}$.
      \item
        В противен случай, $\M$ никога не завършва върху $\code{\M}$.
        Това означава, че $\M'$ никога не завършва върху входа $\alpha$
        и следователно $\M'$ не приема думата $\alpha$.
      \end{itemize}
    \end{itemize}    
  \end{itemize}
  Получаваме, че:
  \begin{align*}
    & \code{\M} \in \Laccept \implies \L(\M') = \Sigma^\star \implies \code{\M'} \in L_{\Sigma^\star},\\
    & \code{\M} \not\in \Laccept \implies \L(\M') = \emptyset \implies \code{\M'} \not\in L_{\Sigma^\star}.
  \end{align*}
  На практика гореописаният алгоритъм дефинира тоталната изчислима функция
  \begin{align*}
    f(\omega) =
    \begin{cases}
      \code{\M'}, & \text{ако }\omega = \code{\M}\\
      \omega, & \text{иначе}.
    \end{cases}
  \end{align*}
  Тогава получаваме, че
  \[\omega \in \Laccept \iff f(\omega) \in L_{\Sigma^\star}\]
  и ако допуснем, че $L_{\Sigma^\star}$ е разрешим език, то $\Laccept$ също ще е разрешим, което е противоречие.
\end{proof}

\begin{corollary}
  Езикът
  \[\ov{L}_{\emptyset} \df \{\omega \in \{0,1\}^\star \mid \omega \text{ е код на машина на Тюринг и }\L(\M_\omega) \neq \emptyset\}\]
  е полуразрешим, но не е разрешим.
\end{corollary}

\begin{corollary}
  Езикът
  \[L_{\emptyset} \df \{\omega \in \{0,1\}^\star \mid \omega\text{ е код на машина на Тюринг и }\L(\M_\omega) = \emptyset\}\]
  не е полуразрешим.
\end{corollary}
\begin{hint}
  Ако $L_{\emptyset}$ беше разрешим, то неговото допълнение
  \[\ov{L}_{\emptyset} = L_{\texttt{code}} \setminus L_{\texttt{Empty}}\]
  щеше да е разрешим език, което е противоречие.

  Ако $L_{\emptyset}$ беше полуразрешим, тогава, използвайки, че $\ov{L}_{\emptyset}$ е полуразрешим, от теоремата на Клини-Пост щеше да следва, че
  $L_{\emptyset}$ е разрешим, което е противоречие
\end{hint}

\begin{problem}
  Докажете, че езикът
  \[L_{\texttt{Dec}} = \{\omega \in \{0,1\}^\star \mid \omega \text{ е код на машина на Тюринг, която е разрешител }\}\]
  не е разрешим.
\end{problem}

\begin{important}
  \begin{proposition}
    Езикът
    \[L_{\texttt{reg}} \df \{\ \omega \mid \omega\text{ е код на машина на Тюринг и }\L(\M_\omega) \text{ е регулярен език}\ \}\]
    не е разрешим.
  \end{proposition}
\end{important}
\begin{proof}
  \mynote{\cite[стр. 219]{sipser3}}
  Да фиксираме един език, за който знаем, че не е регулярен, например, 
  $\{0^n1^n \mid n \in \Nat\}$.
  Дефинираме алгоритъм, за който по вход $\code{\M}$ 
  връща код на машината на Тюринг $\M'$, която работи по следния начин:
  \begin{itemize}
  \item 
    Вход думата $\alpha$;
  \item
    Ако $\alpha = 0^n1^n$, за някое $n$, то $\M'$ приема думата $\alpha$.
  \item
    Ако $\alpha$ не е от вида $0^n1^n$, тогава $\M'$ симулира $\M$ върху думата $\code{\M}$.
    \begin{itemize}
    \item 
      Ако след краен брой стъпки $\M$ завърши като приеме думата $\code{\M}$, то $\M'$ приема $\alpha$.
    \item
      Ако след краен брой стъпки $\M$ завърши като отхвърли думата $\code{\M}$, то $\M'$ отхвърля думата $\alpha$.
    \item
      В противен случай, $\M$ никога не завършва върху $\code{\M}$.
      Това означава, че $\M'$ никога не завършва върху входа $\alpha$
      и следователно $\M'$ не приема думата $\alpha$.
    \end{itemize}
  \end{itemize}
  \mynote{Използваме наготово, че $\Sigma^\star$ е регулярен език.}
  Получаваме, че:
  \begin{align*}
    & \code{\M} \in \Laccept \implies \L(\M') = \Sigma^\star \implies \code{\M'} \in L_{\texttt{reg}},\\
    & \code{\M} \not\in \Laccept \implies \L(\M') = \{0^n1^n \mid n \in \Nat\} \implies \code{\M'} \not\in L_{\texttt{reg}}.
  \end{align*}
  Сега вече трябва да е ясно, че следната тотална функция е изчислима:
  \begin{align*}
    f(\omega) =
    \begin{cases}
      \code{\M'}, & \text{ако }\omega = \code{\M}\\
      \omega, & \text{иначе}.
    \end{cases}
  \end{align*}
  Тогава получаваме, че
  \[\omega \in \Laccept \iff f(\omega) \in L_{\texttt{reg}}\]
  и ако допуснем, че $L_{\texttt{reg}}$ е разрешим език, то $\Laccept$ също ще е разрешим, което е противоречие.  
\end{proof}

Сега ще видим, че идеята, която следвахме в горните доказателства може да се обобщи.
Нека $\Ss$ е множество от полуразрешими езици над фиксирана азбука $\Sigma$.
% \mynote{$\Ss = \{L \mid L\text{ се разпознава от М.Т. с}\\\text{по-малко от 10 състояния}$. Това защо не върши работа?}
Например, 
\[\Ss = \{L \subseteq \Sigma^\star \mid L\text{ е регулярен език}\}.\]
Ще казваме, че $\Ss$ е свойство на полуразрешимите езици.
$\Ss$ е {\bf тривиално свойство}, ако $\Ss = \emptyset$ или $\Ss$ съдържа точно всички полуразрешими езици.
Нека разгледаме изброимото множество от всички машини на Тюринг, които разпознават езиците от $\Ss$.
Ще представим това множество като език от кодовете на тези машини на Тюринг, т.е.
\index{$\texttt{Code}(\Ss)$}
\[\texttt{Code}(\Ss) \df \{\omega \mid \text{$\omega$ е код на машина на Тюринг и } \L(\M_\omega) \in \Ss\}.\]
\index{$\texttt{Code}(L)$}
\mynote{Можем да дефинираме и $\texttt{Code}(L)$, което е безкрайно изброимо множество, ако $L$ е полуразрешим език.}

\begin{important}
  \begin{theorem}[Райс \cite{rice}]
    \index{Райс}
    \mynote{\cite[стр. 188]{hopcroft1}}
    За всяко нетривиално свойство $\Ss$ на полуразрешимите езици,
    $\texttt{Code}(\Ss)$ е неразрешим.
  \end{theorem}
\end{important}
\begin{proof}
  \mynote{Цел: да сведем ефективно $\Laccept$ към $L_\Ss$}
  Без ограничение на общността, нека $\emptyset \not\in \Ss$.
  Понеже $\Ss$ е нетривиално свойство, да разгледаме езика $L \in \Ss$,
  като $\M_L$ е машина на Тюринг, за която $\L(\M_L) = L$.
  Да разгледаме алгоритъм, който по дадена дума $\code{\M}$
  връща код на машина на Тюринг $\M'$, която зависи от $\M$ и от $\M_L$.
  Тя работи по следния начин:
  \begin{itemize}
  \item
    \mynote{Неформално описваме функцията $\delta$ за $\M'$}
    вход думата $\alpha$;
  \item
    първоначално $\M'$ не обръща внимание на $\alpha$;
  \item
    $\M'$ симулира $\M$ върху думата $\code{\M}$.
    \begin{itemize}
    \item
      \mynote{в този случай ще получим, че $\L(\M') = L$}
      ако след краен брой стъпки $\M$ завърши като приеме думата $\code{\M}$, то 
      $\M'$ симулира $\M_L$ върху входната дума $\alpha$;
      \begin{itemize}
      \item
        ако след краен брой стъпки $\M_L$ завърши като приеме думата $\alpha$, то 
        $\M'$ приема $\alpha$;
      \item 
        ако след краен брой стъпки $\M_L$ завърши като отхвърли думата $\alpha$, то 
        $\M'$ отхвърля $\alpha$;
      \item
        ако $\M_L$ никога не завършва върху $\alpha$, то 
        $\M'$ никога няма да завърши върху $\alpha$ и следователно $\M'$
        не приема $\alpha$.
      \end{itemize}
    \item
      \mynote{при тези два случая ще получим, че $\L(\M') = \emptyset$}
      ако след краен брой стъпки $\M$ завърши като отхвърли думата $\code{\M}$, то 
      $\M'$ отхвърля $\alpha$;
    \item
      Ако $\M$ никога не свършва върху $\code{\M}$, то $\M'$ никога няма да свърши върху $\alpha$,
      което означава, че $\M'$ не приема $\alpha$.      
    \end{itemize}
  \end{itemize}
  От всичко това следва, че така описаната машина на Тюринг $\M'$ има свойствата:
  \begin{align*}
    & \code{\M} \in \Laccept \implies \L(\M') = L \implies \L(\M') \in \Ss,\\
    & \code{\M} \not\in \Laccept \implies \L(\M') = \emptyset \implies \L(\M') \not\in \Ss.
  \end{align*}
  Сега вече трябва да е ясно, че гореописаният алгоритъм дефинира тоталната изчислима функция
  \begin{align*}
    f(\omega) =
    \begin{cases}
      \code{\M'}, & \text{ако }\omega = \code{\M}\\
      \omega, & \text{иначе}.
    \end{cases}
  \end{align*}
  Тогава получаваме, че
  \[\omega \in \Laccept \iff f(\omega) \in \texttt{Code}(\Ss)\]
  и ако допуснем, че $\texttt{Code}(\Ss)$ е разрешимо множество, то ще следва, че $\Laccept$ е разрешимо, което е противоречие.

  Ако $\emptyset \in \Ss$, то правим горните разсъждения за класа от езици
  \[\ov{\Ss} = \{ L \subseteq \Sigma^\star \mid L\text{ е полуразрешим език и } L \not\in\Ss\ \}.\]
  По аналогичен начин доказваме, че $\texttt{Code}(\ov{\Ss})$ не е разрешим език.
  Понеже 
  \[\texttt{Code}(\ov{\Ss}) = L_{\texttt{code}} \setminus \texttt{Code}(\Ss),\]
  то $\texttt{Code}(\Ss)$ също не е разрешим език.
\end{proof}

\begin{corollary}
  За всяко от следните свойства $\Ss$ на полуразрешимите езици, 
  $\texttt{Code}(\Ss)$ {\bf не} е разрешим език, където:
  \mynote{Тук няма нужда нищо да доказваме. Просто съобразяваме, че всяко от тези свойства на полуразрешимите езици е нетривиално.}
  \begin{enumerate}[a)]
  \item 
    $\Ss$ е свойството празнота, т.е. езикът
    \[\texttt{Code}(\Ss) = \{\code{\M} \mid \text{$\M$ е машина на Тюринг и } \L(\M) = \emptyset\}\]
    не е разрешим;
  \item 
    $\Ss$ е свойството за пълнота, т.е. езикът
    \[\texttt{Code}(\Ss) = \{\code{\M} \mid \text{$\M$ е машина на Тюринг и } \L(\M) = \Sigma^\star\}\]
    не е разрешим;
  \item
    $\Ss$ е свойството крайност, т.е. езикът
    \[\texttt{Code}(\Ss) = \{\code{\M} \mid \text{$\M$ е машина на Тюринг и }|\L(\M)| < \infty\}\]
    не е разрешим;
  \item
    $\Ss$ е свойството безкрайност, т.е. езикът
    \[\texttt{Code}(\Ss) = \{\code{\M} \mid \text{$\M$ е машина на Тюринг и }|\L(\M)| = \infty\}\]
    не е разрешим;
  \item
    $\Ss$ е свойството регулярност, т.е. езикът
    \[\texttt{Code}(\Ss) = \{\code{\M} \mid \text{$\M$ е машина на Тюринг и $\L(\M)$ е регулярен език}\}\]
    не е разрешим;
  \item
    \mynote{Това свойство е нетривиално, защото вече показахме, че $\{a^nb^nc^n \mid n \in \Nat\}$ е полуразрешим (дори разрешим) език, а знаем отдавна, че този език не е безконтекстен.}
    $\Ss$ е свойството безконтекстност, т.е. езикът
    \[\texttt{Code}(\Ss) = \{\code{\M} \mid \text{$\M$ е машина на Тюринг и $\L(\M)$ е безконтекстен}\}\]
    не е разрешим;
  \item
    \mynote{Тук също - вече сме разгледали примери за полуразрешими езици, които не са разрешими.}
    $\Ss$ е свойството разрешимост, т.е. езикът
    \[\texttt{Code}(\Ss) = \{\code{\M} \mid \text{$\M$ е машина на Тюринг и $\L(\M)$ е разрешим}\}\]
    не е разрешим.
  \end{enumerate}
\end{corollary}


%%% Local Variables:
%%% mode: latex
%%% TeX-master: "../eai"
%%% End:


\section{Критерии за полуразрешимост}

\begin{lemma}
  \mynote{Това означава, че ако $\texttt{Code}(\Ss)$ е полуразрешим език, то всеки език $L_0 \in \Ss$ притежава краен подезик, който също принадлежи на $\Ss$.}
  Нека $\Ss$ е свойство на полуразрешимите езици.
  Ако съществува безкраен език $L_0 \in \Ss$, който няма крайно подмножество в $\Ss$,
  то $\texttt{Code}(\Ss)$ не е полуразрешим език.  
\end{lemma}
\begin{hint}
  Нека $L_0 = \L(\M_0)$ като $L_0 \in \Ss$, но всеки краен подезик на $L_0$ не принадлежи на $\Ss$.
  Сега ще дефинираме тотална изчислима функция $f$, която при вход думата $\omega \in \{0,1\}^\star$ работи по следния начин:
  \begin{itemize}
  \item
    Ако $\omega$ не е код на машина на Тюринг, то $f(\omega) \df \omega$.
  \item
    Ако $\omega$ е код на машината на Тюринг $\M_\omega$, то тогава $f(\omega) = \code{\M'}$,
    където $\M'$ работи така:
    \begin{itemize}
    \item 
      вход думата $\alpha$;
    \item
      $\M'$ симулира работата на $\M_\omega$ върху думата $\omega$:
      \begin{itemize}
      \item
        ако $\M_\omega$ завърши за по-малко от $|\alpha|$ на брой стъпки като \emph{приеме} $\omega$, 
        то $\M'$ завършва веднага като \emph{отхвърля} $\alpha$;
      \item
        в противен случай, $\M'$ симулира работата на $\M_0$ върху $\alpha$.
      \end{itemize}
    \end{itemize}
    Така получаваме, че 
    \begin{align*}
      \L(\M') = 
      \begin{cases}
        \{\alpha \in L_0 \mid \abs{\alpha} < s_0\}, & \text{ако } \M_\omega \text{ приема }\omega\\
        L_0, & \text{ако }\M_\omega \text{ не приема }\omega,
      \end{cases}
    \end{align*}
    където $s_0$ е минималният брой стъпки необходими на $\M_\omega$ за да приеме думата $\omega$.
  \end{itemize}
  
  Заключаваме, че 
  \begin{align*}
    & \M_\omega \text{ не приема }\omega \implies \omega \in L_{\texttt{diag}} \implies \code{\M'} \in \texttt{Code}(\Ss)\\
    & \M_\omega \text{ приема }\omega \implies \omega \not\in L_{\texttt{diag}} \implies \code{\M'} \not\in \texttt{Code}(\Ss).
  \end{align*}
  Понеже
  \[\omega \in L_{\texttt{diag}} \iff f(\omega) \in \texttt{Code}(\Ss),\]
  то $\texttt{Code}(\Ss)$ не е полуразрешим, защото ние знаем, че $L_{\texttt{diag}}$ не е полуразрешим.
\end{hint}

\begin{cor}
  Следните езици {\bf не} са полуразрешими:
  \begin{itemize}
  \item 
    $L = \{\code{\M} \mid \abs{\L(\M)} = \infty\}$;
  \item
    $L = \{\code{\M} \mid \L(\M) = \Sigma^\star\}$;
  \item
    $L = \{\code{\M} \mid \L(\M)\text{ не е разрешим}\}$;
  \item
    $L = \{\code{\M} \mid \L(\M)\text{ не е полуразрешим}\}$;
  \item
    $L = \{\code{\M} \mid \L(\M)\text{ не е регулярен}\}$.
  \end{itemize}
\end{cor}

\begin{lemma}
  \mynote{Това означава, че ако $\texttt{Code}(\Ss)$ е полуразрешим език, ако $L_0 \in \Ss$ и $L_0 \subseteq L_1$, като $L_1$ е полуразрешим, то $L_1 \in \Ss$.}
  Нека $L_1$ е език в $\Ss$ и нека $L_2$ е полуразрешим език, като $L_1 \subset L_2$ и $L_2 \not\in\Ss$.
  Тогава $\texttt{Code}(\Ss)$ не е полуразрешим език.
\end{lemma}
\begin{hint}
  Нека $L_1 = \L(\M_1)$ и $L_2 = \L(\M_2)$.
  Ще дефинираме тотална изчислима функция $f$, която при вход думата $\omega \in \{0,1\}^\star$ работи по следния начин:
  \begin{itemize}
  \item
    Ако $\omega$ не е код на машина на Тюринг, то $f(\omega) = \omega$.
  \item
    Ако $\omega$ е код на машината на Тюринг $\M_\omega$, тогава $f(\omega)$ ще бъде код на машината на Тюринг $\hat{\M}$,
    която работи по следния начин:
    \begin{itemize}
    \item 
      вход думата $\alpha$;
    \item
      $\hat{\M}$ симулира едновременно две изчисления - $\M_1$ върху $\alpha$ и $\M_\omega$ върху $\omega$
      докато намери стъпка $s$, такава че:    
      \begin{itemize}
      \item 
        ако $\M_1$ завършва за $s$ на брой стъпки като приема думата $\alpha$, то $\hat{\M}$ завършва като приема думата $\alpha$;
      \item
        ако $\M_\omega$ завършва за $s$ на брой стъпки като приема думата $\omega$, 
        то $\hat{\M}$ симулира работата $\M_2$ върху $\alpha$.
      \item
        ако $\hat{\M}$ не намери такава стъпка, то е ясно, че $\hat{\M}$ никога не завършва върху $\alpha$.
      \end{itemize}
    \end{itemize}
    Получаваме, че:
    \begin{align*}
      \L(\hat{\M}) = 
      \begin{cases}
        L_2, & \text{ако $\M_\omega$ приема }\omega\\
        L_1, & \text{ако $\M_\omega$ не приема }\omega.
      \end{cases}
    \end{align*}
  \end{itemize}
  Заключаваме, че:
  \[\omega \in L_{\texttt{diag}} \iff f(\omega) \in \texttt{Code}(\Ss),\]
  защото $L_2 \not\in \Ss$, а $L_1 \in \Ss$.
  Това означава, че ефективно можем да сведем въпрос за принадлежност в $L_{\texttt{diag}}$
  към въпрос за принадлежност в $\texttt{Code}(\Ss)$.
  Следователно, ако $\texttt{Code}(\Ss)$ е полуразрешим език, то $L_{\texttt{diag}}$ е полуразрешим език, което е противоречие.  
\end{hint}

\begin{cor}
  Следните езици {\bf не} са полуразрешими:
  \begin{itemize}
  \item 
    $L = \{\code{\M} \mid \L(\M) \text{ е регулярен} \}$;
  \item
    $L = \{\code{\M} \mid \L(\M) \text{ е безконтекстен} \}$;
  \item
    $L = \{\code{\M} \mid \L(\M) \text{ е разрешим} \}$;
  \item
    $L = \{\code{\M} \mid \abs{\L(\M)} = 42\}$.
  \end{itemize}
\end{cor}


\begin{framed}
  \begin{theorem}[Райс-Шапиро]
    Нека $\texttt{Code}(\Ss)$ е полуразрешим език. Тогава е изпълнено, че:
    \[L \in \Ss \iff (\exists L_0 \subseteq \Sigma^\star )[L_0\text{ е краен и }L_0 \subseteq L \implies L_0 \in \Ss].\]
  \end{theorem}
\end{framed}


% % \section{Проблеми за безконтекстни езици}

% % \begin{lemma}
% %   Нека е дадена $\M = \TM$.
% %   Тогава езикът 
% %   \[L = \{\alpha\sharp\beta^R \mid \alpha,\beta \in \Gamma^\star Q \Gamma^\star\ \&\  \alpha \vdash_\M \beta\}\]
% %   е безконтекстен.
% % \end{lemma}
% % \begin{proof}
% %   Ще покажем, че съществува стеков автомат $P$, за който $\L_S(P) = L$.
% %   Четем буквата $X$. Тогава:
% %   \begin{itemize}
% %   \item 
% %     ако $\delta_\M(q,X) =(p,Y,R)$, то слагаме $Yp$ на върха на стека;
% %   \item
% %     ако $\delta_\M(q,X) =(p,Y,L)$, то ако $Z$ е върха на стека, заменяме $Z$ с $pZY$;
% %   \end{itemize}
% % \end{proof}



% % \begin{thm}
% %   Неразрешим е проблемът за проверка дали при дадени две произволни безконтекстни граматики $G_1$ и $G_2$,
% %   $\L(G_1) \cap \L(G_2) = \emptyset$.  
% % \end{thm}

% % \begin{thm}
% %   Неразрешим е проблемът за проверка дали при дадена произволна безконтекстна граматика $G$,
% %   $\L(G) = \Sigma^\star$.  
% % \end{thm}


% % \section{Въпроси}

% % Вярно ли е, че следният проблем е {\em разрешим}:
% % \begin{itemize}
% % \item
% %   за произволна безконтекстна граматика $G$, проверява дали $\L(G) = \emptyset$?
% % \item
% %   за произволна безконтекстна граматика $G$, проверява дали $\L(G) = \Sigma^\star$?
% % \item
% %   за произволни безконтекстни граматики $G_1$ и $G_2$, проверява дали $\L(G_1) \cap \L(G_2) = \emptyset$?
% % \item
% %   за произволни безконтекстни граматики $G_1$ и $G_2$, проверява дали $\L(G_1) \cap \L(G_2) = \Sigma^\star$?
% % \item
% %   за произволни безконтекстни граматики $G_1$ и $G_2$, проверява дали $\L(G_1) = \L(G_2)$?
% % \item
% %   за произволни безконтекстни граматики $G_1$ и $G_2$, проверява дали $\L(G_1) \subseteq \L(G_2)$?
% % \item
% %   за произволна безконтекстна граматика $G$ и произволен регулярен израз $r$,
% %   проверява дали $\L(G) = \L(r)$?
% % \item
% %   за произволна безконтекстна граматика $G$ и произволен регулярен израз $r$,
% %   проверява дали $\L(G) \subseteq \L(r)$?
% % \item
% %   за произволна безконтекстна граматика $G$ и произволен регулярен израз $r$,
% %   проверява дали $\L(r) \subseteq \L(G)$?
% % \item
% %   за произволни безконтекстни граматики $G_1$ и $G_2$, проверява дали $\L(G_1) \subseteq \L(G_2)$ 
% %   е безконтекстен език ?
% % \item
% %   за произволна безконтекстна граматика $G$, проверява дали $\Sigma^\star \setminus \L(G)$
% %   е безконтекстен език ?
% % \item
% %   за произволна безконтекстна граматика $G$, проверява дали $\L(G)$ е регулярен език?
% % \end{itemize}


%%% Local Variables:
%%% mode: latex
%%% TeX-master: "../eai"
%%% End:


% 
\subsection*{Безконтекстен език за преходите в машина на Тюринг}

% \subsection*{Валидни и невалидни изчисления на машини на Тюринг}
\marginpar{\cite{hopcroft1} стр. 201}
Да разгледаме машината на Тюринг $\M$.

Една дума $\omega$ описва конфигурация на машина на Тюринг,
ако $\omega \in \Gamma^\star Q \Gamma^\star$.

\begin{framed}
  \begin{prop}
    Да фиксираме една машина на Тюринг $\M$. 
    Тогава следните езици за безконтекстни:
    \begin{itemize}
    \item 
      $\texttt{ValidStep}(\M) \df \{\ \alpha\#\beta^{rev} \mid \alpha,\beta \in \Gamma^\star Q \Gamma^\star\ \&\ \alpha \vdash_\M \beta\ \}$;
    \item
      $\texttt{ValidStep}'(\M)\df \{\ \alpha^{rev}\#\beta \mid \alpha,\beta \in \Gamma^\star Q \Gamma^\star\ \&\ \alpha \vdash_\M \beta\ \}$;
    \item
      $\texttt{InvalidStep}(\M) \df \{\ \alpha\#\beta^{rev} \mid \alpha,\beta \in \Gamma^\star Q \Gamma^\star\ \&\  \alpha \not\vdash_\M \beta\ \}$;
    \item
      $\texttt{InvalidStep}'(\M) \df \{\ \alpha^{rev}\#\beta \mid \alpha,\beta \in \Gamma^\star Q \Gamma^\star\ \&\ \alpha \not\vdash_\M \beta\ \}$.
    \end{itemize}
  \end{prop}  
\end{framed}

\begin{hint}

  Да напомним първо как дефинираме релацията $\vdash_\M$:
  \begin{align*}
    & (\alpha_1z, q, x\alpha_2) \vdash_\M  (\alpha_1 zy, p, \alpha_2) & \comment{\text{ ако } q \overset{x/y;\goright}{\longrightarrow} p} \\
    & (\alpha_1z, q, x\alpha_2) \vdash_\M (\alpha_1, p ,zy\alpha_2) & \comment{\text{ ако } q \overset{x/y;\goleft}{\longrightarrow} p} \\
    & (\alpha_1z, q, x\alpha_2) \vdash_\M (\alpha_1z, p, y\alpha_2) & \comment{\text{ ако } q \overset{x/y;\stay}{\longrightarrow} p}.
  \end{align*}

  Думите в езика $\texttt{ValidStep}(\M)$ кодират релацията $\vdash_\M$. Това означава, че всяка дума на 
  $\texttt{ValidStep}(\M)$ има някое от следните представяния:
  \begin{align*}
    & \alpha_1zqx\alpha_2 \sharp \alpha^{rev}_2 p y z \alpha^{rev}_1 & \comment{\text{ ако } q \overset{x/y;\goright}{\longrightarrow} p} \\
    & \alpha_1zqx\alpha_2 \sharp \alpha^{rev}_2 y z p \alpha^{rev}_1 & \comment{\text{ ако } q \overset{x/y;\goleft}{\longrightarrow} p} \\
    & \alpha_1zqx\alpha_2 \sharp \alpha^{rev}_2 y p z\alpha^{rev}_1 & \comment{\text{ ако } q \overset{x/y;\stay}{\longrightarrow} p}
  \end{align*}

  Ще опишем неформално стеков автомат $P$ за езика $\texttt{ValidStep}(\M)$.
  Нека 
  \[Q^{P} \df \{r_q \mid q \in Q^\M\} \cup \{r, \hat{r}\}.\]

  \begin{itemize}
  \item
    Първо четем $\alpha_1$ и я записваме в стека като $\alpha^{rev}_1$.
    Това правим като дефинираме функцията на преходите като 
    \[(\forall a,z \in \Sigma)[\ \delta_{P}(r,a,z) \df \{(r,az)\}\ ].\]
  \item 
    Правим това докато не срещнем някое $q \in Q^\M$. Тогава трябва да направим преход на $\M$.
    Тук трябва да внимаваме, защото за да направим преход, трябва да знаем състоянието $q$ и да прочетем следващия символ.
    Един начин да разрешим този проблем е като запомним кое състояние сме прочели на машината на Тюринг в състоянията на стековия автомат:
    \[(\forall q \in Q^\M)(\forall z \in \Sigma)[\ \delta_{P}(r,q,z) = \{(r_q,z)\}\ ].\]
    \begin{itemize}
    \item 
      \marginpar{Стекът представлява $z\alpha^{rev}_1$}
      ако $\delta_\M(q,x) = (p,y,\goright)$, то слагаме $yp$ на върха на стека, т.е.
      \[\delta_{P}(r_q,x,z) = \{(\hat{r}, ypz)\}.\]
    \item
      ако $\delta_\M(q,x) =(p,y,\goleft)$, то ако $z$ е върха на стека, заменяме $z$ с $pzy$, т.е.
      \[\delta_{P}(r_q,x,z) = \{(\hat{r}, pzy)\}.\]
    \item
      ако $\delta_\M(q,x) =(p,y,\stay)$, то ако $z$ е върха на стека, заменяме $z$ с $ypz$, т.е.
      \[\delta_{P}(r_q,x,z) = \{(\hat{r}, ypz)\}.\]
    \end{itemize}
  \item
    Сега вече сме в състояние $\hat{r}$ и остава да прочетем $\alpha_2$ и да я запишем в стека като $\alpha^{rev}_2$:
    \[\delta_{P}(\hat{r},x,z) = \{(\hat{r}, xz)\}.\]
  \end{itemize}
\end{hint}

\begin{remark}
  Да обърнем внимание, че горната конструкция на стековия автомат $P$ е {\bf ефективна}, т.е.
  съществува алгоритъм, който при вход машина на Тюринг $\M$ връща като изход стеков автомат $P$ за езика $\texttt{ValidStep}(\M)$.
  С други думи, езикът 
  \[\{\code{\M} \cdot \code{P} \mid \L(P) = \texttt{ValidStep}(\M)\}\]
  е разрешим.
\end{remark}

\subsection*{История на машина на Тюринг}
\marginpar{история на приемащо изчисление}

Дума от вида  $\omega_1 \sharp \omega^{rev}_2 \sharp \omega_3 \sharp \omega^{rev}_4\sharp\omega_5\cdots$
се нарича {\bf история на приемащо изчисление} на машината на Тюринг $\M$, ако
\begin{itemize}
\item
  $\omega_i \in \Gamma^\star Q \Gamma^\star$, т.е. $\omega_i$ описва моментна конфигурация
  и $\omega_i$ не започва и не завършва на $\blank$.
\item
  $\omega_1 \in \qstart\Sigma^\star$ описва начална конфигурация.
\item
  $\omega_n \in \Gamma^\star \cdot\{\qaccept\} \cdot \Gamma^\star$ описва приемаща конфигурация.
\item 
  $\omega_i \vdash_\M \omega_{i+1}$ за $i = 1,\dots,n-1$.
\end{itemize}

\begin{lemma}
  \marginpar{\cite{hopcroft1}, стр. 201}
  Нека да означим с $\texttt{Accept}(\M)$ езикът от историите на всички приемащи изчисления за машината на Тюринг $\M$.
  Тогава 
  \[\texttt{Accept}(\M) = L_1 \cap L_2,\]
  където $L_1$ и $L_2$ са безконтекстни езици.
  Освен това, граматиките на $L_1$ и $L_2$ могат ефективно да бъдат построени от $\M$.
\end{lemma}
\begin{hint}
  Да разгледаме езиците:
  \begin{align*}
    & L_1 \df (\texttt{ValidStep}(\M)\sharp)^\star(\{\varepsilon\}\cup \Gamma^\star \cdot \{\qaccept\} \cdot \Gamma^\star\sharp)\\
    & L_2 \df \qstart\Sigma^\star \sharp (\texttt{ValidStep}(\M)\sharp)^\star(\{\varepsilon\}\cup \Gamma^\star \cdot \{\qaccept\} \cdot \Gamma^\star\sharp),
  \end{align*}
  за които е ясно, че са безконтекстни.
\end{hint}

\begin{thm}
  Езикът
  \[L = \{\code{G_1}\cdot\code{G_2} \mid \text{$G_1$ и $G_2$ са безконт. грам. и }\L(G_1) \cap \L(G_2) = \emptyset\}\]
  е неразрешим.
\end{thm}
\begin{hint}
  По дадена дума $\code{\M}$, можем ефективно да намерим $G_1$ и $G_2$, за които
  $\L(G_1) \cap \L(G_2)$ са точно валидните изчисления на $\M$.
  Тогава ако $L$ е разрешим език, то $L_{\texttt{Empty}}$ е разрешим език, което е противоречие.
\end{hint}

\begin{lemma}
  За всяка машина на Тюринг $\M$, $\overline{\texttt{Accept}(\M)}$ е безконтекстен език.
\end{lemma}
\begin{hint}
  Една дума $\alpha$ не е история на приемащо изчисление, ако е изпълнено някое от следните условия:
  \begin{itemize}
  \item 
    \marginpar{Можем да опишем това свойство с регулярен език}
    $\alpha$ не е от вида $\omega_1 \sharp \omega_2 \sharp \cdots \sharp \omega_n$,
    където $\omega_i \in \Gamma^\star Q \Gamma^\star$, или
  \item
    ако $\alpha$ е от вида $\omega_1 \sharp \omega_2 \sharp \cdots \sharp \omega_n$,
    където $\omega_i \in \Gamma^\star Q \Gamma^\star$, тогава:
    \begin{itemize}
    \item 
      $\omega_1 \not\in \qstart \Gamma^\star$, или
    \item
      $\omega_n \not\in \Gamma^\star \cdot \{\qaccept\} \cdot \Gamma^\star$, или
    \item
      $\omega_i \not\vdash_\M \omega^{rev}_{i+1}$, за някое нечетно $i$, или
    \item
      $\omega^{rev}_i \not\vdash_\M \omega_{i+1}$, за някое четно $i$.
    \end{itemize}
  \end{itemize}
  Думите притежаващи някое от тези свойства могат да се опишат като обединение на три регулярни езика и двата безконтекстни езика.
\end{hint}

\begin{framed}
  \begin{thm}
    За дадена азбука $\Sigma$, 
    езикът 
    \[\texttt{All}_{\texttt{CFG}} = \{\code{G} \mid G\text{ е безконтекстна граматика и }\L(G) = \Sigma^\star\}\]
    е неразрешим.
  \end{thm}
\end{framed}
\begin{hint}
  По дадена дума $\code{\M}$, можем ефективно да намерим $G$, за която
  $\L(G)$ са точно невалидните изчисления на $\M$.
  Тогава ако допуснем, че $L$ е разрешим език, то $L_{\texttt{Empty}}$ е разрешим, което е противоречие.
\end{hint}

\begin{cor}
  Следните езици не са разрешими:
  \begin{enumerate}[a)]
  \item
    $L = \{\code{G_1}\cdot\code{G_2} \mid \text{$G_1$ и $G_2$ са безконт. грам. и }\L(G_1) = \L(G_2)\}$;
  \item
    $L = \{\code{G_1}\cdot\code{G_2} \mid \text{$G_1$ и $G_2$ са безконт. грам. и }\L(G_1) \subseteq \L(G_2)\}$;
  \item 
    $L = \{\code{G}\cdot r \mid \text{$G$ е безконт. грам. и $r$ е рег. израз и }\L(G) = \L(r)\}$;
  \item
    $L = \{\code{G}\cdot \code{\A} \mid \text{$G$ е безконт. грам. и $\A$ е ДКА и }\L(G) = \L(\A)\}$;
  \item 
    $L = \{\code{G}\cdot r \mid \text{$G$ е безконт. грам. и $r$ е рег. израз и }\L(r) \subseteq \L(G)\}$;
  \item
    $L = \{\code{G}\cdot \code{\A} \mid \text{$G$ е безконт. грам. и $\A$ е ДКА и }\L(\A) \subseteq \L(G)\}$;
  \end{enumerate}
\end{cor}

\begin{remark}
  Добре е да обърнем внимание, че езикът 
  \[L = \{\code{G}\cdot \code{\A} \mid \text{$G$ е безконт. грам. и $\A$ е ДКА и }\L(G) \subseteq \L(\A)\}\]
  е разрешим.
  Това е така, защото $\L(G) \subseteq \L(\A) \iff \L(G) \cap \L(\ov{\A}) = \emptyset$,
  защото сечението на безконтекстен и регулярен език е безконтекстен език.
\end{remark}

\newpage

\begin{framed}
  \begin{prop}
    Езикът 
    \[\texttt{Reg} = \{\code{G} \mid G\text{ е безконтекстна граматика и $\L(G)$ е регулярен}\}\]
    не е разрешим.
  \end{prop}
\end{framed}
\begin{hint}
  Да фиксираме език $L_0$, който е безконтекстен, но не е регулярен.
  За произволен език $L$, да разгледаме езика
  \[\hat{L} \df L_0 \sharp \Sigma^\star\ \cup\ \Sigma^\star \sharp L.\]
  Първо ще докажем, че: 
  \begin{equation}
    \label{eq:2}
    L = \Sigma^\star\ \iff\ \hat{L}\text{ е регулярен}.
  \end{equation}
  Да отбележим, че можем ефективно да получим от безконтекстна граматика $G$ за $L$
  безконтекстна граматика $\hat{G}$ за $\hat{L}$.
  Нека да означим с $\texttt{conv}$ изчислимата функция, за която
  $\texttt{conv}(\code{G}) = \code{\hat{G}}$.

  \begin{itemize}
  \item 
    Ако $L = \Sigma^\star$, то $\hat{L}$ е регулярен, защото тогава
    $\hat{L} = \Sigma^\star \sharp \Sigma^\star$ е очевидно регулярен.
  \item
    \marginpar{Ако $L$ е регулярен, то $L/_\beta \df \{\alpha \mid \alpha\beta \in L\}$ е регулярен}  
    Ако $L \neq \Sigma^\star$, то нека да фиксираме дума $\omega \not\in L$.
    Ако допуснем, че $\hat{L}$ е регулярен, то езикът
    $\hat{L}/_{\sharp\omega} = L_0$ ще е регулярен, което е противоречие с избора на $L_0$.
  \end{itemize}
  
  Нека да означим
  \[\texttt{Full} \df \{\code{G} \mid G\text{ е безконтекстна граматика и }\L(G) = \Sigma^\star\}.\]
  От (\ref{eq:2}) имаме, че 
  \[\code{G} \in \texttt{Full}\ \iff\ \texttt{conv}(\code{G}) \in \texttt{Reg}.\]
  
  Ако допуснем, че $\texttt{Reg}$ е разрешим език, то тогава ще следва, че
  $\texttt{Full}$ е разрешим език, за което вече знаем, че не е вярно.
\end{hint}

% \begin{cor}
%   Нека $G_1$ и $G_2$ са произволни безконтекстни граматики, а $r$ е произволен регулярен израз.
%   Следните проблеми са неразрешими:
%   \begin{enumerate}
%   \item 
%     $\L(G_1) = \L(G_2)$;
%   \item
%     $\L(G_2) \subseteq \L(G_1)$;
%   \item
%     $\L(G_1) = \L(r)$;
%   \item
%     $\L(r) \subseteq \L(G_1)$.
%   \end{enumerate}
% \end{cor}

%%% Local Variables:
%%% mode: latex
%%% TeX-master: "../eai"
%%% End:

% \begin{theorem}[Грейбах 1963]
  \index{Грейбах}
  \mynote{\cite[стр. 205]{hopcroft1}}
  Нека $\mathcal{C}$ е клас от езици, за който съществува ефективно кодиране $\code{L}$ на езиците в $\mathcal{C}$ и който е:
  \mynote{По дадена дума $\omega$ можем ефективно да проверим дали тя кодира език от $\mathcal{C}$ или не.}
  \begin{itemize}
  \item 
    ефективно затворен относно обединение;
  \item
    ефективно затворен относно конкатенация с регулярен език;
  \item
    "$= \Sigma^\star$" е неразрешим за достатъчно голяма $\Sigma$.
  \end{itemize}
  \mynote{Съществуват езици от $\mathcal{C}$, които не притежават свойството $P$ и такива, които го притежават.}
  Нека $P$ е нетривиално свойство на $\mathcal{C}$, което е изпълнено за всеки регулярен език и ако $L \in P$,
  то $L/_a \in P$, където
  \[L/_a = \{\omega \mid \omega a \in L\}.\]
  Тогава езикът $\{\code{L} \mid P(L)\ \&\ L \in \mathcal{C}\}$ е неразрешим.
\end{theorem}
\begin{hint}
  Да фиксираме език $L_0 \in \Cs$, за който {\em не е изпълнено} свойството $P$.
  Нека да приемем, че $L_0 \subseteq \Sigma^\star$, която е достатъчно голяма азбука, за която
  въпроса ``$= \Sigma^\star$'' е неразрешим.
  За произволен език $L \in \mathcal{C}$, да разгледаме езика
  \[\hat{L} \df L_0 \sharp \Sigma^\star\ \cup\ \Sigma^\star \sharp L.\]
  Ясно е, че $\hat{L}\in \mathcal{C}$, защото $\mathcal{C}$ е ефективно затворен относно конкатенация с регулярен език и относно обединение. 
  Първо ще докажем, че: 
  \begin{equation}
    \label{eq:2}
    L = \Sigma^\star\ \iff\ \code{\hat{L}} \in P.
  \end{equation}

  \begin{itemize}
  \item 
    Ако $L = \Sigma^\star$, то $\hat{L}$ е регулярен, защото тогава
    $\hat{L} = \Sigma^\star \sharp \Sigma^\star$ е очевидно регулярен и от избора на $P$, $\code{\hat{L}} \in \mathcal{C}$.
  \item
    \mynote{Ако $\code{L} \in P$ , то за $L/_\beta \df \{\alpha \mid \alpha\beta \in L\}$ е изпълнено $P$.}  
    Ако $L \neq \Sigma^\star$, то нека да фиксираме дума $\omega \not\in L$.
    Ако допуснем, че $\code{\hat{L}} \in P$, то езикът
    за езикът $\hat{L}/_{\sharp\omega} = L_0$ също ще е изпълнено свойството $P$, което е противоречие с избора на $L_0$.
  \end{itemize}

  От (\ref{eq:2}) следва, че $P$ е разрешимо свойство точно тогава, когато въпросът ''$=\Sigma^\star$'' за езиците от $\mathcal{C}$ е разрешим, което е противоречие.
\end{hint}

\begin{corollary}
  Въпросът дали една безконтекстна граматика описва регулярен език е неразрешим.
  По-точно, езикът
  \[\texttt{Reg} = \{\code{G} \mid G\text{ е безконтекстна граматика и }\L(G)\text{ е регулярен език}\}\]
  е неразрешим.
\end{corollary}
\begin{proof}
  Ясно е, че имаме ефективно кодиране на безконтекстните граматики $\code{G}$ и освен това те са
  ефективно затворени относно конкатенация с регулярен език и относно обединение.
  Вече знаем от \Theorem{computations:all-cfg}, че $= \Sigma^\star$ за безконтекстни граматики е неразрешим за достатъчно голяма азбука $\Sigma$.
  Тогава от теоремата на Грейбах следва, че $\texttt{Reg}$ е неразрешим език.
\end{proof}


%%% Local Variables:
%%% mode: latex
%%% TeX-master: "../eai"
%%% End:

% \newpage
% \section{Неограничени граматики}
\index{граматика!неограничена}

\begin{definition}
  \mynote{\cite[стр. 220]{hopcroft1}}
  \mynote{На англ. unrestricted grammar}
  \mynote{Според йерархията на Чомски, това е граматика от тип 0}
  Граматиката $G = (V,\Sigma,R,S)$
  се нарича неограничена граматика, 
  ако правилата $R$ са от вида $\alpha \to \beta$,
  където $\alpha,\beta \in (V\cup\Sigma)^\star$.
\end{definition}

\begin{lemma}
  За всеки полуразрешим език $L$, $L = \L(G)$, за някоя неограничена граматика $G$.  
\end{lemma}
\begin{proof}
  Нека $L = \L(\M)$, където 
  \[\M = \TM\] е детерминистична машина на Тюринг,
  като искаме лентата да е безкрайна само отдясно и входната дума $\alpha$ е
  поставена в началото на лентата.
  Ще построим граматика $G = \CFG$, където 
  \[V = ((\Sigma\cup\{\varepsilon\})\times\Gamma) \cup \{A_1,A_2,A_3\}.\]
  Правилата на $G$ са следните:
  \begin{enumerate}[1)]
  \item 
    $A_1 \to sA_2$;
  \item
    $A_2 \to [a,a]A_2$, за всяка $a\in\Sigma$;
  \item
    $A_2 \to A_3$;
  \item
    $A_3 \to [\varepsilon,\blank]A_3$;
  \item
    $A_3 \to \varepsilon$;
  \item
    $q[a,X] \to [a,Y]p$, за всяка $a \in \Sigma\cup\{\varepsilon\}$, всяко $q\in Q$, $X,Y \in\Gamma$, 
    за които $\delta(q,X) = (p,Y,R)$;
  % \item
  %   $q[a,X] \to p[a,Y]$, за всяка $a \in \Sigma\cup\{\varepsilon\}$, всяко $q\in Q$, $X,Y \in\Gamma$, 
  %   за които $\delta(q,X) = (p,Y,N)$;
  \item
    $[b,Z]q[a,X] \to p[b,Z][z,Y]$, за всяко $X,Y,Z \in \Gamma$, $a,b\in\Sigma\cup\{\varepsilon\}$, $q\in Q$,
    за които $\delta(q,X) = (p,Y,L)$;
  \item
    $[a,X]q \to qaq$, $q[a,X] \to qaq$, $q \to \varepsilon$, за всяко $a\in\Sigma\cup\{\varepsilon\}$, $X\in\Gamma$,
    и $q \in F$.
  \end{enumerate}
  
  Лесно се вижда, че, използвайки правилата 1) и 2), за всяко $n$, имаме
  \[A_1 \to^\star s[a_1,a_1]\cdots[a_n,a_n]A_2,\]
  където $a_i \in \Sigma$.

  Нека $\M$ приема думата $\alpha = a_1\cdots a_n$.
  Това означава, че за някое $m$, $\M$ използва не повече от $m$ клетки от лентата отдясно на входната дума.
  Ясно е, че имаме
  \[A_1 \to^\star s[a_1,a_1]\cdots[a_n,a_n][\varepsilon,\blank]^m.\]
  Оттук нататък, можем да използваме само правилата 6), 7), 8), докато не срещнем финално състояние.
  С индукция по броя на стъпки в $\M$, можем да докаже, че ако е изпълнено
  $(\varepsilon,s,a_1\cdots a_n) \vdash^\star_\M (X_1\cdots X_{r-1},q,X_r\cdots X_l)$, 
  то \[s[a_1,a_1]\dots[a_n,a_n][\varepsilon,\blank]^m \rightarrow^\star_G [a_1,X_1]\cdots[a_{r-1},X_{r-1}]q[a_r,X_r]\cdots[a_{n+m},X_{n+m}],\]
  където $a_1,\dots,a_n \in \Sigma$, $a_{n+1},\dots,a_{n+m} = \varepsilon$, $X_1,\dots,X_{n+m} \in \Gamma$ и
  $X_{l+1} = X_{l+2} = \dots = X_{n+m} = \blank$.
  
  Най-накрая, ако $q \in F$, то можем да използваме правилата от 9) и да докажем, че
  \[[a_1,X_1]\cdots[a_{t-1},X_{t-1}]q[a_t,X_t]\cdots[a_{n+m},X_{n+m}] \rightarrow^\star_G a_1\cdots a_n.\]
  
  Така доказахме, че ако $\alpha \in \L(\M)$, то $\alpha \in \L(G)$, т.е. $\L(\M) \subseteq \L(G)$.
  За да докажем обратната посока, трябва да направим подобни разсъждения.
\end{proof}

\begin{lemma}
  Ако $L = \L(G)$, където $G$ е неограничена граматика, то $L$ е полуразрешим език.
\end{lemma}
\mynote{Доказателствата в \cite{hopcroft1} и \cite{papadimitriou} са различни}
\begin{proof}
  $\M$ ще бъде недетерминистична машина с три ленти.
  \begin{enumerate}[1)]
  \item
    Записваме входната дума $\omega$ на първата лента на $\M$.
    Тя никога не се променя.
  \item
    На втората лента ще имаме думата $\gamma \in (V\cup\Sigma)^\star$.
    В началото $\gamma := S$.
  \item 
    Недетерминистично избираме правило $\alpha \to \beta$ от граматиката $G$.
  \item
    Недетерминистично избираме $\gamma_0,\gamma_1 \in (V\cup\Sigma)^\star$, за които 
    $\gamma = \gamma_0\alpha\gamma_1$.
    Тогава $\gamma := \gamma_0\beta\gamma_1$.
    Ако няма такива $\gamma_0$ и $\gamma_1$, то $\M$ ,,зацикля'' - текущият опит за извеждане на $\omega$ пропада.
  \item
    Сравняваме съдържанието на първите две ленти, т.е. проверяваме дали $\omega = \gamma$.
    Ако $\omega = \gamma$, то спираме и казваме, че $\M$ разпознава думата $\omega$.
    Ако $\omega \neq \gamma$, то се връщаме на стъпка 3).
  \end{enumerate}

  \begin{algorithm}[H]
  \caption{}
%  \label{alg:}
  \begin{algorithmic}[1]
    \State $\gamma:= S$
    \ForAll{$\alpha\to\beta \in R$}
    \If{$(\exists \gamma_0,\gamma_1\in (V\cup\Sigma)^\star)[\gamma = \gamma_0\alpha\gamma_1]$}
    \State $\gamma := \gamma_0\beta\gamma_1$
    \Else ...
    \EndIf
    \EndFor
  \end{algorithmic}
\end{algorithm}

\end{proof}

\begin{example}
  Граматика за $L = \{a^nb^nc^n \mid n\in\Nat\}$.
\end{example}

%%% Local Variables:
%%% mode: latex
%%% TeX-master: "../eai"
%%% End:


% \section{Контекстни граматики}

\index{граматика!контекстна}
Казваме, че $G = (V,\Sigma,R,S)$ е {\bf контекстна граматика}, ако правилата на $G$ са от вида
\[\alpha_1 A \alpha_2 \to \alpha_1 \beta \alpha_2,\]
където $\alpha_1,\alpha_2 \in (V\cup\Sigma)^\star$ и $\beta \in (V\cup\Sigma)^+$.

\begin{problem}
  Езикът $L = \{a^nb^nc^n \mid n \in \Nat\ \&\ n > 0\}$ е контекстен.
\end{problem}
\begin{hint}
  Разгледайте контекстната граматика $G$ зададена със следните правила:
  \begin{align*}
    & S \to aSBC\ |\ aBC,\hspace*{0.2cm} CB \to BC\\
    & aB \to ab,\ bB \to bb,\ bC \to bc,\ cC \to cc.
  \end{align*}

  Докажете, че
  \begin{itemize}
  \item
    $S \to^\star_Ga^n(BC)^n$;
  \item
    $(BC)^n \to^\star_G B^nC^n$;
  \item
    $aB^n \to^\star_G ab^n$;
  \item
    $bC^n \to^\star_G bc^n$.
  \end{itemize}
\end{hint}

\begin{proposition}
  Класът на безконтекстните езици строго се включва в класа на контекстните езици.
\end{proposition}

\begin{proposition}
  Всеки контекстен език е разрешим.
\end{proposition}

\begin{proposition}
  Съществува разрешим език, който не е контекстен.
\end{proposition}


%%% Local Variables:
%%% mode: latex
%%% TeX-master: "../eai"
%%% End:


\section{Сложност}

\begin{itemize}
\item
  \index{машина на Тюринг!детерминистично полиномиално ограничена}
  Казваме, че детерминистичната машина на Тюринг $\M$ е {\bf полиномиално ограничена}, ако 
  същестува полином $p(x)$, такъв че няма дума $\omega$,
  за която $\M$ да извършва при вход $\omega$ повече от $p(|\omega|)$ стъпки.
\item
  \index{език!детерминистично полиномиално разрешим}
  Езикът $L$ се нарича {\bf детерминистично полиномиално разрешим},
  ако съществува полиномиално ограниченен детерминистичен разрешител $\M$, за който $L = \L(\M)$.
  Нека
  \[\mathcal{P} \df \{L \subseteq \Sigma^\star \mid L\text{ е полиномиално разрешим с ДМТ}\}.\]
\item
  \index{машина на Тюринг!полиномиално ограничена}
  Казваме, че детерминистичната машина на Тюринг $\M$ е {\bf експоненциално ограничена}, ако 
  същестува полином $p(x)$, такъв че няма дума $\omega$,
  за която $\M$ да извършва при вход $\omega$ повече от $2^{p(|\omega|)}$ стъпки.
\item
  \index{език!детерминистично експоненциално разрешим}
  Езикът $L$ се нарича {\bf детерминистично експоненциално разрешим},
  ако съществува експоненциално ограниченен детерминистичен разрешител $\M$, за който $L = \L(\M)$.
  Нека
  \[\mathcal{EXP} \df \{L \subseteq \Sigma^\star \mid L\text{ е експоненциално разрешим с ДМТ}\}.\]
\item
  \index{машина на Тюринг!недетерминистично полиномиално ограничена}
  Казваме, че недетерминистичната машина на Тюринг $\N$ е {\bf полиномиално ограничена}, ако 
  същестува полином $p(x)$, такъв че няма дума $\omega$,
  за която $\N$ да извършва при вход $\omega$ повече от $p(|\omega|)$ стъпки.
\item
  \index{език!недетерминистично полиномиално разрешим}
  Езикът $L$ се нарича {\bf недетерминистично полиномиално разрешим},
  ако съществува полиномиално ограничена недетерминистичен разрешител $\N$,
  за който $L = \L(\N)$. Нека
  \[\mathcal{NP} \df \{L \subseteq \Sigma^\star \mid L\text{ е полиномиално разрешим с НМТ}\}.\]
\end{itemize}

% \begin{framed}
%   \begin{dfn}
%     \begin{align*}
%       & \mathcal{P} \df \{L \subseteq \Sigma^\star \mid L\text{ е полиномиално разрешим с ДМТ}\};\\
%       & \mathcal{EXP} \df \{L \subseteq \Sigma^\star \mid L\text{ е експоненциално разрешим с ДМТ}\};\\
%       & \mathcal{NP} \df \{L \subseteq \Sigma^\star \mid L\text{ е полиномиално разрешим с НМТ}\}.
%     \end{align*}
%   \end{dfn}
% \end{framed}

\begin{problem}
  Докажете, че класът $\mathcal{P}$ е затворен относно допълнение, обединение, сечение и конкатенация.
\end{problem}

\begin{problem}
  Докажете, че класът $\mathcal{P}$ е затворен относно операцията звезда на Клини.
\end{problem}




\begin{thm}
  $\mathcal{NP} \subseteq \mathcal{EXP}$.
\end{thm}

\begin{proposition}
  За азбука $\Sigma$ от поне две букви, можем да обобщим някои от резултатите от предишните глави:
  \[\texttt{REG} \subsetneqq \texttt{CFG} \subsetneqq \mathcal{P}.\]
\end{proposition}
\begin{hint}
  Езикът $\{a^nb^nc^n \mid n \in \Nat\} \in \mathcal{P}$,
  но не е безконтекстен.
\end{hint}


%%% Local Variables:
%%% mode: latex
%%% TeX-master: "../eai"
%%% End:


\section{Задачи}

\begin{problem}
  Вярно ли е, че следните езици са разрешими?
  \begin{enumerate}[a)]
  \item
    $\{\code{\A}\cdot \omega \mid \A \text{ е ДКА и } \omega \in \L(\A)\}$;
  \item
    $\{\code{\A} \mid \A \text{ е ДКА и } \L(\A)\text{ е безкраен език}\}$;
  \item
    $\{\code{\A} \mid \A \text{ е ДКА и }\L(\A) = \{0,1\}^\star\}$;
  \item 
    $\{\code{\A} \mid \A \text{ е ДКА и }\L(\A)\text{ съдържа поне една дума с равен брой нули и единици}\}$;
  \item
    $\{\code{\A} \mid \A \text{ е ДКА и }\L(\A)\text{ съдържа поне една дума палиндром}\}$;
  \item
    $\{\code{\A} \mid \A \text{ е ДКА и }\L(\A)\text{ не съдържа дума с нечетен брой единици}\}$;
  \item
    $\{\code{\A}\cdot\code{\B} \mid \A\text{ и }\B \text{ са ДКА и }\L(\A) \subseteq \L(\B)\}$;
  \item
    $\{\code{\A}\cdot\code{\B} \mid \A\text{ и }\B \text{ са ДКА и }\L(\A) = \L(\B)\}$;
  \end{enumerate}
\end{problem}

\begin{problem}
  Вярно ли е, че следните езици са разрешими?
  \begin{enumerate}[a)]
  \item
    $\{\code{G} \cdot \omega \mid G \text{ е безконтекстна граматика и } \omega \in \L(G)\}$;
  \item
    $\{\code{G} \mid G \text{ е безконтекстна граматика и } \L(G) = \emptyset\}$;
  \item 
    $\{\code{G} \mid G \text{ е безконтекстна граматика над }\{0,1\}^\star\text{ и }\L(1^\star) \subseteq \L(G)\}$;
  \item 
    $\{\code{G} \mid G \text{ е безконтекстна граматика над }\{0,1\}^\star\text{ и }\L(1^\star) \subseteq \L(G)\}$;
  \item 
    $\{\code{G} \mid G \text{ е безконтекстна граматика над }\{0,1\}^\star\text{ и }\varepsilon \in \L(G)\}$;
  \item
    $\{\code{G}\cdot 0^k \mid G \text{ е безконтекстна граматика над }\{0,1\}^\star\text{ и }|\L(G)| \leq k\}$;
  \item
    $\{\code{G} \mid G \text{ е безконтекстна граматика над }\{0,1\}^\star\text{ и }|\L(G)| = \infty\}$;
  \end{enumerate}
\end{problem}


\begin{problem}
  Докажете, че езикът
  \[L = \{\code{\M}\sharp\omega \mid \M\text{ прави движение наляво при работата си върху вход }\omega\}\]
  е разрешим.
\end{problem}
\begin{hint}
  Нужно е да симулирате работата на $\M$ върху $\omega$ само за $|\omega| + |Q^\M| + 1$ на брой стъпки.
\end{hint}

\begin{problem}
  Докажете, че езикът
  \[L = \{\code{\M}\sharp\omega \mid \M\text{ прави опит за движение наляво от най-лявата клетка при работата си върху вход }\omega\}\]
  е разрешим.
\end{problem}


%%% Local Variables:
%%% mode: latex
%%% TeX-master: "../eai"
%%% End:


% \section{Варианти на машини на Тюринг}

\begin{itemize}
\item
  Четящата глава не отива наляво от началната позиция;
\item
  Никога на пише $\blank$.
\end{itemize}

Така имаме по-лесен начин за представянето на моментното описание на едно изчисление на машина на Тюринг като дума:
\[\hat{\Gamma}^\star \cdot Q \cdot \hat{\Gamma}^\star \cup \hat{\Gamma}^\star \cdot \{\blank\}^\star \cdot Q \cdot \{\blank\},\]
където $\hat\Gamma = \Gamma \setminus\{\blank\}$.


%%% Local Variables:
%%% mode: latex
%%% TeX-master: "../eai"
%%% End:


% \section{Линейни автомати}
\mynote{На англ. linear bounded automaton}
\index{линеен автомат}

{\bf Линеен автомат} е машина на Тюринг, на която не се позволява четящата глава да излиза извън частта от лентата, върху която първоначално е записана входната дума.

\begin{theorem}
  Езикът
  \[L = \{\code{\M}\sharp \omega \mid \M\text{ е линеен автомат и } \omega \in \L(\M)\}\]
  е разрешим.
\end{theorem}
\begin{proof}
  Това е лесно, защото изчислението на $\M$ върху входната дума $\omega$
  може да се намира в една от $|Q|\cdot|\Gamma|^{|\omega|}\cdot |\omega|$ конфигурации.
\end{proof}


\begin{theorem}
  Езикът
  \[L = \{\code{\M} \mid \M\text{ е линеен автомат и } \L(\M) = \emptyset\}\]
  е неразрешим.
\end{theorem}
\begin{proof}
  
\end{proof}


%%% Local Variables:
%%% mode: latex
%%% TeX-master: "../eai"
%%% End:


% \section{Проблемът за съответствието на Пост}\label{sect:turing:pcp}

\mynote{На англ. Post's correspondence problem \cite[стр. 392]{hopcroft2}, но по-добре е обяснено в \cite[стр. 227]{sipser3}. Тези двойки от думи се наричат домино.}

Пример за проблема за съответствието на Пост се нарича всяка крайна редица от елементи на $\Sigma^\star \times \Sigma^\star$,
които ние ще означаваме така:
\[\begin{bmatrix} \alpha_1\\ \beta_1\end{bmatrix},\begin{bmatrix} \alpha_2\\ \beta_2\end{bmatrix},\dots,\begin{bmatrix} \alpha_n\\ \beta_n\end{bmatrix}.\]
Всяка една редица от този вид се нарича {\em пример} за \PCP.
\mynote{Ако $|\alpha_i| = |\beta_i|$ за всяко $i$, то задачата е тривиална.}
Една непразна редица от индекси $i_1,i_2,\dots,i_n$ се нарича {\em решение} на \PCP примера, ако е изпълнено, че:
\[\alpha_{i_1}\alpha_{i_2}\cdots\alpha_{i_n} = \beta_{i_1}\beta_{i_2}\cdots\beta_{i_n}.\]

\begin{problem}
  \mynote{\cite[стр. 239]{sipser3}}
  Намерете решение на следния пример за \PCP:
  \[\begin{bmatrix}ab\\ abab\end{bmatrix},\begin{bmatrix} b\\ a\end{bmatrix},\begin{bmatrix} aba\\ b\end{bmatrix},\begin{bmatrix} aa\\ a\end{bmatrix}.\]
\end{problem}
\begin{solution}
  \[\begin{bmatrix}ab\cdot ab \cdot aba \cdot b \cdot b \cdot aa \cdot aa\\abab \cdot abab \cdot b \cdot a \cdot a \cdot a \cdot a\end{bmatrix}\]
\end{solution}


\subsection*{Модифициран проблем за съответствието }

\mynote{Тук искаме винаги да започваме с първото домино.}
Казваме, че \MPCP има решение, ако съществува произволна редица от индекси $i_1,\dots,i_n$ (може и празна), такава че:
\[\alpha_1\alpha_{i_1}\cdots\alpha_{i_n} = \beta_1\beta_{i_1}\cdots\beta_{i_n}.\]

\begin{lemma}
  Съществува алгоритъм, който свежда \MPCP към \PCP.
\end{lemma}
\begin{proof}
  Нека имаме пример за \MPCP:
  \[\begin{bmatrix} \alpha_1\\ \beta_1\end{bmatrix},\begin{bmatrix} \alpha_2\\ \beta_2\end{bmatrix},\dots,\begin{bmatrix} \alpha_k\\ \beta_k\end{bmatrix} .\]
  Нека символите $\star,\$$ не са от $\Sigma$.
  Нека за думата $\alpha = a_1\cdots a_n$ да дефинираме следните операции:
  \begin{align*}
    & \star\alpha = \star a_1 \star a_2\cdots \star a_n\\
    & \alpha\star = a_1\star a_2\star\cdots a_n \star\\
    & \star\alpha\star = \star a_1\star a_2 \star \cdots a_n\star.
  \end{align*}
  Тогава на базата на горния пример за \MPCP, строим пример за \PCP:
  \[\begin{bmatrix*}[l] \star\alpha_1\star\\ \star\beta_1\end{bmatrix*},\begin{bmatrix} \alpha_1\star\\ \star \beta_1\end{bmatrix},\dots,\begin{bmatrix} \alpha_k\star\\ \star\beta_k\end{bmatrix},\begin{bmatrix*}[r] \$\\ \star\$\end{bmatrix*}.\]
  Така ние показахме, че
  \[\MPCP \leq_m \PCP.\]
\end{proof}

\begin{corollary}
  Ако $\PCP$ е разрешим, то $\MPCP$ също е разрешим.
\end{corollary}

Ясно е, че проблемът на Пост е полуразрешим. Сега ще видим, че той не е разрешим.

\begin{framed}
  \begin{theorem}[Е. Пост \cite{pcp}]\index{Пост}
    Проблемът за съответствието на Пост е неразрешим при азбука $\Sigma$ с поне два символа.
  \end{theorem}
\end{framed}
\begin{hint}
  \mynote{Лесно се съобразява, че за азбука $\Sigma$ само с една буква проблемът е разрешим.}
  Нека приемем, че работим с машини на Тюринг, които не движат главата си наляво от левия край на лентата.
  Ще докажем, че $\Luniv \leq_m \MPCP$. Вече знаем, че \MPCP се свежда алгоритмично към \PCP, т.е. $\MPCP \leq_m \PCP$.
  Това означава, че ще опишем работата на тотална изчислима функция $f$, за която
  \[\gamma \in \Luniv \iff f(\gamma) \in \MPCP.\]
  Сега неформално ще опишем работата на функцията $f$.
  Нека фиксираме символа $\sharp \not \in \Gamma$.
  \begin{enumerate}[1)]
  \item
    Нека имаме като вход дума $\gamma = \code{\M}\sharp \alpha$.
  \item
    \mynote{Горната част на доминото се опитва да настигне долната част.}
    Започваме като добавяме за думата $\alpha = a_1\cdots a_n$ над азбуката $\Sigma$ следната двойка
    $\begin{bmatrix*}[l] \sharp\\ \sharp qa_1\cdots a_n\sharp\end{bmatrix*}$.
  \item
    Ако $\delta(q,a) = (p,b,\goright)$, то добавяме двойката
    $\begin{bmatrix*}[l] qa\\ bp\end{bmatrix*}$.
  \item
    Ако $\delta(q,\blank) = (p,b,\goright)$, то добавяме и двойката $\begin{bmatrix*}[l] q\sharp\\ bp\sharp\end{bmatrix*}$.
  \item
    \mynote{Тук е важно, че не позволяваме четящата глава да се мести по-наляво от първата клетка върху която е четящата глава при стартиране на изчислението.}
    Ако $\delta(q,a) = (p,b,\goleft)$, то добавяме двойките
    $\begin{bmatrix*}[l] xqa\\ pxb\end{bmatrix*}$.
  \item
    Ако $\delta(q,\blank) = (p,b,\goleft)$, то добавяме двойките
    $\begin{bmatrix*}[l] xq\sharp\\ pxb\sharp\end{bmatrix*}$.
  \item
    Ако $\delta(q,a) = (p,b,\stay)$, то добавяме двойката
    $\begin{bmatrix*}[l] qa\\ pb\end{bmatrix*}$.
  \item
    за всеки $x \in \Gamma$, добавяме $\begin{bmatrix} x\\ x\end{bmatrix}$.
    Освен това, добавяме и двойката $\begin{bmatrix} \sharp\\ \sharp\end{bmatrix}$.
  \item
    \mynote{Когато достигнем до приемащо състояние, то започваме да трием съдържанието на доминото за да можем да изравним двете части на доминото.}
    За всеки $x \in \Gamma$, добавяме двойката
    $\begin{bmatrix*}[l] x\qaccept\\ \qaccept\end{bmatrix*}$ и $\begin{bmatrix*}[l] \qaccept x\\ \qaccept\end{bmatrix*}$.
  \item
    За да завършим, добавяме двойката
    $\begin{bmatrix*}[r] \qaccept\sharp\sharp\\ \sharp\end{bmatrix*}$.
  \end{enumerate}
\end{hint}

\begin{corollary}
  Проблемът за еднозначност на безконтекстна граматика е неразрешим.
\end{corollary}
\begin{hint}
  Нека да означим
  \[\texttt{AMBIG} = \{\code{G} \mid G \text{ е нееднозначна безконтекстна граматика}\}.\]
  Да разгледаме един пример за $\PCP$ над азбуката $\Sigma$:
  \[\begin{bmatrix} \alpha_1\\ \beta_1\end{bmatrix},\begin{bmatrix} \alpha_2\\ \beta_2\end{bmatrix},\dots,\begin{bmatrix} \alpha_n\\ \beta_n\end{bmatrix}.\]
  По него можем ефективно да построим следната безконтекстна граматика:
  \begin{align*}
    & S \to A\ |\ B\\
    & A \to \alpha_1A c_1\ |\ \alpha_2 A c_2\ |\ \cdots\ |\ \alpha_n A c_n\ |\ \alpha_1c_1\ |\ \alpha_2c_2\ |\ \cdots\ |\ \alpha_nc_n\\
    & B \to \beta_1B c_1\ |\ \beta_2 B c_2\ |\ \cdots\ |\ \beta_n B c_n\ |\ \beta_1c_1\ |\ \beta_2c_2\ |\ \cdots\ |\ \beta_nc_n,
  \end{align*}
  където $c_1,\dots,c_n \not \in \Sigma$.
  Лесно се съобразява, че горния пример за $\PCP$ има решение точно тогава, когато безконтекстната граматика е нееднозначна.
  С други думи, показахме, че
  \[\PCP \leq_m \texttt{AMBIG}.\]
\end{hint}

\begin{corollary}\label{cor:pcp:grammar-intersect}
  Проблемът за сечение на две безконтекстни граматики е неразрешим.
\end{corollary}
\begin{hint}
  Нека да означим
  \[\INTERSECT = \{\code{G_1}\sharp\code{G_2} \mid \L(G_1) \cap \L(G_2) \neq \emptyset \}.\]
  Да разгледаме един пример за $\PCP$ над азбуката $\Sigma$:
  \[\begin{bmatrix} \alpha_1\\ \beta_1\end{bmatrix},\begin{bmatrix} \alpha_2\\ \beta_2\end{bmatrix},\dots,\begin{bmatrix} \alpha_n\\ \beta_n\end{bmatrix}.\]
  По него можем ефективно да построим следните две безконтекстни граматика с правила:
  \begin{align*}
    & S_1 \to \alpha_1S_1 c_1\ |\ \alpha_2 S_1 c_2\ |\ \cdots\ |\ \alpha_n S_1 c_n\ |\ \alpha_1c_1\ |\ \alpha_2c_2\ |\ \cdots\ |\ \alpha_nc_n\\
    & S_2 \to \beta_1S_2 c_1\ |\ \beta_2 S_2 c_2\ |\ \cdots\ |\ \beta_n S_2 c_n\ |\ \beta_1c_1\ |\ \beta_2c_2\ |\ \cdots\ |\ \beta_nc_n,
  \end{align*}
  където $c_1,\dots,c_n \not \in \Sigma$.
  Така показахме, че
  \[\PCP \leq_m \INTERSECT.\]
\end{hint}



%%% Local Variables:
%%% mode: latex
%%% TeX-master: "../eai"
%%% End:


% \section*{Бележки}

% \begin{itemize}
% \item
%   За основните дефиниции следваме основно Глава 3 от \cite{sipser3}.
% \item 
%   За въпросите за неразрешимост следваме основно Глава 8 от \cite{hopcroft1}.
% % \item
% %   За по-задълбочено запознаване с теория на изчислимостта, добри уводни книги са
% %   \cite{ditchev-soskov} и \cite{nikolova-soskova}.
% \end{itemize}


%%% Local Variables:
%%% mode: latex
%%% TeX-master: "../eai"
%%% End:



\bibliographystyle{amsalpha}
\bibliography{EAI}

\printindex

\end{document}

%%% Local Variables: 
%%% mode: latex
%%% TeX-master: t
%%% End: 
