\subsection{Всеки автоматен език е регулярен}
\begin{framed}
\begin{thm}[Клини \cite{kleene-regular}]
  \label{th:regular:kleene}
  \index{Клини}
  Всеки автоматен език се описва с регулярен израз.
\end{thm}
\end{framed}
\begin{proof}
  \mynote{\cite[стр. 79]{papadimitriou}; \cite[стр. 33]{hopcroft1}}
  Нека  $L = \L(\A)$, за някой детерминиран краен автомат $\A$.
  Да фиксираме едно изброяване на състоянията
  \[Q = \{q_0,\dots,q_{n-1}\},\]
  като $\qstart = q_0$.
  Ще означаваме с $L(i,j,k)$ множеството от тези думи, които
  могат да се разпознаят от автомата по път, който започва от $q_i$,
  завършва в $q_j$, и междинните състояния имат индекси $< k$.
  Например, за думата $\alpha = a_1a_2\cdots a_s$ имаме, че $\alpha \in L(i,j,k)$
  точно тогава, когато съществуват състояния $q_{\ell_1},\dots,q_{\ell_{s-1}}$, като $\ell_1,\dots,\ell_{s-1} < k$ и
  \[q_i\stackrel{a_1}{\rightarrow} q_{\ell_1} \stackrel{a_2}{\rightarrow} q_{\ell_2} \stackrel{a_3}{\rightarrow} \dots \stackrel{a_{s-1}}{\rightarrow} q_{\ell_{s-1}}\stackrel{a_s}{\rightarrow} q_j.\]
  Тогава за $n = \abs{Q}$, 
  \[L(i,j,n) = \{\alpha\in\Sigma^\star\mid \delta^\star(q_i,\alpha) = q_j\}.\]
  Така получаваме, че 
  \[\L(\A) = \bigcup\{L(0,j,n)\mid q_j \in F\} = \bigcup_{q_j\in F}L(0,j,n).\]
  Ще докажем с {\em индукция по $k$}, че за всеки три състояния $q_i,q_j,q_k \in Q$, множествата от думи $L(i,j,k)$
  се описват с регулярен израз $\mathbf{r}^k_{i,j}$.
  \begin{enumerate}[a)]
  \item
    Нека $k = 0$. Ще докажем, че за всеки две състояния $q_i,q_j \in Q$, езикът $L(i,j,0)$ се описва с регулярен израз.
    Имаме да разгледаме два случая.
    \begin{itemize}
    \item
      Ако $i = j$, то 
      \begin{equation}
        \label{eq:kleene-equal}
        L(i, j, 0) = \{\varepsilon\}\cup\{a\in\Sigma \mid \delta(q_i,a) = q_j\}.
      \end{equation}
    \item
      Ако $i \neq j$, то
      \[L(i, j, 0) = \{a\in\Sigma \mid \delta(q_i, a) = q_j\}.\]      
    \end{itemize}
    И в двата случая, понеже $L(i,j,0)$ е краен език, то е ясно, че той се описва с регулярен израз.
  \item
    Да предположим, че за всяко $q_i,q_j\in Q$ имаме регулярните изрази $\mathbf{r}^{k}_{i,j}$, които
    описват езиците $L(i,j,k)$, т.е. имаме индукционното предположение, че
    \[(\forall i,j < n)[L(i,j,k) = \L(\mathbf{r}^{k}_{i,j})].\] 
    Ще докажем, че съществуват регулярни изрази $\mathbf{r}^{k+1}_{i,j}$, такива че
    \[(\forall i,j < n)[L(i,j,k+1) = \L(\mathbf{r}^{k+1}_{i,j})].\] 
    За всяка дума $\alpha \in L(i,j,k+1)$ имаме един от следните случаи:
    \begin{itemize}
    \item
      Състоянието $q_k$ не е измежду вътрешните състояния, които участват в изчислението на автомата $\A$ върху думата $\alpha$, т.е. имаме следната ситуация:
      \[q_i\stackrel{a_1}{\rightarrow} q_{\ell_1} \stackrel{a_2}{\rightarrow} q_{\ell_2} \stackrel{a_3}{\rightarrow} \dots \stackrel{a_{s-1}}{\rightarrow} q_{\ell_{s-1}}\stackrel{a_s}{\rightarrow} q_j,\]
      където $\ell_1,\dots,\ell_{s-1} < k$.
      Това означава, че $\alpha \in L(i,j,k)$.
    \item
      Състоянието $q_k$ е измежду вътрешните състояния, които участват в изчислението на автомата $\A$ върху думата $\alpha$.
      Ако състоянието $q_k$ се среща да кажем $m$ пъти, то можем да представим думата $\alpha$ като $\alpha = \alpha_1\alpha_2\cdots\alpha_{m+1}$ и имаме следната ситуация:
      \[q_i\stackrel{\alpha_1}{\rightarrow\cdots\rightarrow} q_{k} \stackrel{\alpha_2}{\rightarrow\cdots\rightarrow} q_k \rightarrow\cdots\rightarrow q_{k} \stackrel{\alpha_{m+1}}{\rightarrow\cdots\rightarrow} q_j,\]
      За всяко $\ell=1,\dots,m+1$, състоянието $q_k$ не е измежду вътрешните състояния, които участват в изчислението на автомата $\A$ върху $\alpha_\ell$, т.е. индексите на всички вътрешни състояния са $<k$.
      Това означава, че $\alpha_1 \in L(i,k,k)$, $\alpha_\ell \in L(k,k,k)$ за $\ell=2,\dots,m$, и $\alpha_{m+1} \in L(k,j,k)$, т.е.
      \mynote{Възможно е $m = 1$. Тогава $L(k,k,k)^{m-1} = \{\varepsilon\}$.}
      \[\alpha \in L(i,k,k) \cdot L(k,k,k)^{m-1} \cdot L(k,j,k).\]
      Понеже направихме това разсъждение за произволно $m$,
      можем да заключим, че ако $q_k$ се среща измежду вътрешните състояния, които участва в изчислението на автомата $\A$ върху думата $\alpha$,
      то \[\alpha \in L(i,k,k) \cdot L(k,k,k)^\star \cdot L(k,j,k).\]
    \end{itemize}
    Така доказахме, че 
    \[L(i,j,k+1) \subseteq L(i,j,k) \cup L(i,k,k) \cdot L(k,k,k)^\star \cdot L(k,j,k).\]
    \mynote{\writedown Защо?}
    Включването в другата посока следва директно от дефиницията на езиците $L(i,j,k)$.
    
    От направените по-горе разсъждения следва, че можем да изразим езика $L(i,j,k+1)$ по следния начин:
    \[L(i,j,k+1) = \underbrace{L(i,j,k)}_{\L(\mathbf{r}^{k}_{i,j})}\ \cup\ \underbrace{L(i,k,k)}_{\L(\mathbf{r}^{k}_{i,k})}\cdot (\underbrace{L(k,k,k)}_{\L(\mathbf{r}^{k}_{k,k})})^\star \cdot \underbrace{L(k,j,k)}_{\L(\mathbf{r}^{k}_{k,j})}.\]
    Тогава по {\bf И.П.} следва, че $L(i,j,k)$ може да се опише с регулярния израз
    \begin{equation}
      \label{eq:kleene}
      \mathbf{r}^{k+1}_{i,j} = \mathbf{r}^{k}_{i,j} + \mathbf{r}^{k}_{i,k}\cdot (\mathbf{r}^{k}_{k,k})^\star\cdot \mathbf{r}^{k}_{k,j}.
    \end{equation}
  \end{enumerate}
  Заключаваме, че за всеки три състояния $q_i,q_j,q_k \in Q$, $L(i,j,k)$ може да се опише с регулярен израз $\mathbf{r}^{k}_{i,j}$.
  Тогава ако $F = \{q_{i_1},\dots,q_{i_m}\}$, то $\L(\A)$ се описва с регулярния израз
  \[\mathbf{r}^n_{0,i_1} + \mathbf{r}^n_{0,i_2} + \dots + \mathbf{r}^n_{0,i_m}.\]
\end{proof}

\begin{proposition}
  \mynote{Естествено, можем да формулираме това твърдение и за краен недетерминиран автомат.}
  Съществува алгоритъм, за който при вход краен детерминиран автомат $\A$,
  извежда като изход регулярен израз $\mathbf{r}$, такъв че $\L(\A) = \L(\mathbf{r})$.
\end{proposition}

Доказателството на това твърдение използва идеята от доказателството на \hyperref[th:regular:kleene]{теоремата на Клини}
но излиза извън рамките на този курс. За подробно изложение на този въпрос, вижте \cite[стр. 69]{sipser3}.
Възможно е по-късно да използваме този резултат наготово.
Нека поне да разгледаме един пример, чиято цел е да ни убеди, 
че наистина може да се извлече алгоритъм от доказателството на \hyperref[th:regular:kleene]{теоремата на Клини}.


%%% Local Variables:
%%% mode: latex
%%% TeX-master: "../eai"
%%% End:
