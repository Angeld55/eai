\subsection{Всеки автоматен език е регулярен}
\begin{framed}
\begin{thm}[Клини \cite{kleene-regular}]
  \label{th:regular-kleene}
  \index{Клини}
  Всеки автоматен език се описва с регулярен израз.
\end{thm}
\end{framed}
\begin{proof}
  \marginpar{\cite[стр. 79]{papadimitriou}; \cite[стр. 33]{hopcroft1}}
  Нека  $L = \L(\A)$, за някой краен детерминиран автомат $\A$.
  Да фиксираме едно изброяване на състоянията
  \[Q = \{q_0,\dots,q_{n-1}\},\]
  като $\qstart = q_0$.
  Ще означаваме с $L(i,j,k)$ множеството от тези думи, които
  могат да се разпознаят от автомата по път, който започва от $q_i$,
  завършва в $q_j$, и междинните състояния имат индекси $< k$.
  Например, за думата $\alpha = a_1a_2\cdots a_s$ имаме, че $\alpha \in L(i,j,k)$
  точно тогава, когато съществуват състояния $q_{l_1},\dots,q_{l_{s-1}}$, като $l_1,\dots,l_{s-1} < k$ и
  \[q_i\stackrel{a_1}{\rightarrow} q_{l_1} \stackrel{a_2}{\rightarrow} q_{l_2} \stackrel{a_3}{\rightarrow} \dots \stackrel{a_{s-1}}{\rightarrow} q_{l_{s-1}}\stackrel{a_s}{\rightarrow} q_j.\]
  Тогава за $n = \abs{Q}$, 
  \[L(i,j,n) = \{\alpha\in\Sigma^\star\mid \delta^\star(q_i,\alpha) = q_j\}.\]
  Така получаваме, че 
  \[\L(\A) = \bigcup\{L(0,j,n)\mid q_j \in F\} = \bigcup_{q_j\in F}L(0,j,n).\]
  Ще докажем с {\em индукция по $k$}, че за всеки три състояния $q_i,q_j,q_k \in Q$, множествата от думи $L(i,j,k)$
  се описват с регулярен израз $\mathbf{r}^k_{i,j}$.
  \begin{enumerate}[a)]
  \item
    Нека $k = 0$. Ще докажем, че за всеки две състояния $q_i,q_j \in Q$, езикът $L(i,j,0)$ се описва с регулярен израз.
    Имаме да разгледаме два случая.
    \begin{itemize}
    \item
      Ако $i = j$, то 
      \begin{equation}
        \label{eq:kleene-equal}
        L(i, j, 0) = \{\varepsilon\}\cup\{a\in\Sigma \mid \delta(q_i,a) = q_j\}.
      \end{equation}
    \item
      Ако $i \neq j$, то
      \[L(i, j, 0) = \{a\in\Sigma \mid \delta(q_i, a) = q_j\}.\]      
    \end{itemize}
    И в двата случая, понеже $L(i,j,0)$ е краен език, то е ясно, че той се описва с регулярен израз.
  \item
    Да предположим, че за всяко $q_i,q_j\in Q$ имаме регулярните изрази $\mathbf{r}^{k}_{i,j}$, които
    описват езиците $L(i,j,k)$, т.е. имаме индукционното предположение, че
    \[(\forall i,j < n)[L(i,j,k) = \L(\mathbf{r}^{k}_{i,j})].\] 
    Ще докажем, че съществуват регулярни изрази $\mathbf{r}^{k+1}_{i,j}$, такива че
    \[(\forall i,j < n)[L(i,j,k+1) = \L(\mathbf{r}^{k+1}_{i,j})].\] 
    За всяка дума $\alpha \in L(i,j,k+1)$ имаме един от следните случаи:
    \begin{itemize}
    \item
      Състоянието $q_k$ не е измежду вътрешните състояния, които участват в изчислението на автомата $\A$ върху думата $\alpha$, т.е. имаме следната ситуация:
      \[q_i\stackrel{a_1}{\rightarrow} q_{l_1} \stackrel{a_2}{\rightarrow} q_{l_2} \stackrel{a_3}{\rightarrow} \dots \stackrel{a_{s-1}}{\rightarrow} q_{l_{s-1}}\stackrel{a_s}{\rightarrow} q_j,\]
      където $l_1,\dots,l_{s-1} < k$.
      Това означава, че $\alpha \in L(i,j,k)$.
    \item
      Състоянието $q_k$ е измежду вътрешните състояния, които участват в изчислението на автомата $\A$ върху думата $\alpha$.
      Ако състоянието $q_k$ се среща да кажем $m$ пъти, то можем да представим думата $\alpha$ като $\alpha = \alpha_1\alpha_2\cdots\alpha_{m+1}$ и имаме следната ситуация:
      \[q_i\stackrel{\alpha_1}{\rightarrow\cdots\rightarrow} q_{k} \stackrel{\alpha_2}{\rightarrow\cdots\rightarrow} q_k \rightarrow\cdots\rightarrow q_{k} \stackrel{\alpha_{m+1}}{\rightarrow\cdots\rightarrow} q_j,\]
      За всяко $l=1,\dots,m+1$, състоянието $q_k$ не е измежду вътрешните състояния, които участват в изчислението на автомата $\A$ върху $\alpha_l$, т.е. индексите на всички вътрешни състояния са $<k$.
      Това означава, че $\alpha_1 \in L(i,k,k)$, $\alpha_l \in L(k,k,k)$ за $l=2,\dots,m$, и $\alpha_{m+1} \in L(k,j,k)$, т.е.
      \marginpar{Възможно е $m = 1$. Тогава $L(k,k,k)^{m-1} = \{\varepsilon\}$.}
      \[\alpha \in L(i,j,k) \cdot L(k,k,k)^{m-1} \cdot L(k,j,k).\]
      Понеже направихме това разсъждение за произволно $m$,
      можем да заключим, че ако $q_k$ се среща измежду вътрешните състояния, които участва в изчислението на автомата $\A$ върху думата $\alpha$,
      то \[\alpha \in L(i,k,k) \cdot L(k,k,k)^\star \cdot L(k,j,k).\]
    \end{itemize}
    Така доказахме, че 
    \[L(i,j,k+1) \subseteq L(i,j,k) \cup L(i,k,k) \cdot L(k,k,k)^\star \cdot L(k,j,k).\]
    \marginpar{\writedown Защо?}
    Включването в другата посока следва директно от дефиницията на езиците $L(i,j,k)$.
    
    От направените по-горе разсъждения следва, че можем да изразим езика $L(i,j,k+1)$ по следния начин:
    \[L(i,j,k+1) = \underbrace{L(i,j,k)}_{\L(\mathbf{r}^{k}_{i,j})}\ \cup\ \underbrace{L(i,k,k)}_{\L(\mathbf{r}^{k}_{i,k})}\cdot (\underbrace{L(k,k,k)}_{\L(\mathbf{r}^{k}_{k,k})})^\star \cdot \underbrace{L(k,j,k)}_{\L(\mathbf{r}^{k}_{k,j})}.\]
    Тогава по {\bf И.П.} следва, че $L(i,j,k)$ може да се опише с регулярния израз
    \begin{equation}
      \label{eq:kleene}
      \mathbf{r}^{k+1}_{i,j} = \mathbf{r}^{k}_{i,j} + \mathbf{r}^{k}_{i,k}\cdot (\mathbf{r}^{k}_{k,k})^\star\cdot \mathbf{r}^{k}_{k,j}.
    \end{equation}
  \end{enumerate}
  Заключаваме, че за всеки три състояния $q_i,q_j,q_k \in Q$, $L(i,j,k)$ може да се опише с регулярен израз $\mathbf{r}^{k}_{i,j}$.
  Тогава ако $F = \{q_{i_1},\dots,q_{i_m}\}$, то $\L(\A)$ се описва с регулярния израз
  \[\mathbf{r}^n_{0,i_1} + \mathbf{r}^n_{0,i_2} + \dots + \mathbf{r}^n_{0,i_m}.\]
\end{proof}

\begin{prop}
  \marginpar{Естествено, можем да формулираме това твърдение и за краен недетерминиран автомат.}
  Съществува алгоритъм, за който при вход краен детерминиран автомат $\A$,
  извежда като изход регулярен израз $\mathbf{r}$, такъв че $\L(\A) = \L(\mathbf{r})$.
\end{prop}

Доказателството на това твърдение използва идеята от доказателството на \Th{regular-kleene},
но излиза извън рамките на този курс. За подробно изложение на този въпрос, вижте \cite[стр. 69]{sipser3}.
Възможно е по-късно да използваме този резултат наготово.
Нека поне да разгледаме един пример, чиято цел е да ни убеди, 
че наистина може да се извлече алгоритъм от доказателството на \Th{regular-kleene}.

\begin{example}
  Да разгледаме следния автомат:

  \begin{framed}
    \begin{figure}[H]
      \begin{center}
        \begin{tikzpicture}[->,>=stealth,thick,node distance=45pt]
          \tikzstyle{every state}=[circle,minimum size=15pt,auto]
          
          \node[initial,state]      (1) {$q_0$};
          \node[accepting, state]   (2) [right of=1]{$q_1$};
          
          \path 
          (1) edge [loop above]  node [above] {$1$} (1)
          (1) edge  node [above] {$0$} (2)
          (2) edge [loop above] node [above] {$0,1$} (2);
        \end{tikzpicture}
      \end{center}
      \caption{Автомат разпознаващ $\L(\mathbf{1^\star 0 (0 + 1)^\star)}$}
      \label{fig:a1}
    \end{figure}
\end{framed}

Лесно се съобразява, че езикът на автомата от Фигура \ref{fig:a1} се описва с регулярния израз $\mathbf{1^\star 0 (0 + 1)^\star}$.
Следвайки означенията и конструкцията от доказателството на \Th{regular-kleene},
езикът на този автомат се описва с регулярния израз $\mathbf{r}^2_{0,1}$, защото началното състояние е $q_0$, финалното е $q_1$ и 
броят на състоянията в автомата е $2$.
\begin{align*}
  \bm{r}^2_{0,1} & = \underbrace{\bm{r}^{1}_{0,1}}_{\bm{1^\star 0}} + \underbrace{\bm{r}^{1}_{0,1}}_{\bm{1^\star 0}}\cdot \underbrace{\bm{(r^1_{1,1})^\star}}_{\mathbf{(\bm{\varepsilon}+0+1)^\star}} \cdot \underbrace{\mathbf{r}^1_{1,1}}_{\mathbf{\bm{\varepsilon}+0+1}} & \comment \text{според (\ref{eq:kleene}}) \\
                     &  = \mathbf{1^\star0 + 1^\star 0 (\bm{\varepsilon} + 0 + 1)^\star (\bm{\varepsilon} + 0 + 1)} & \comment{\text{просто заместваме}}\\
                     & = \mathbf{1^\star0 + 1^\star 0 (\bm{\varepsilon} + 0 + 1)^+} & \comment \mathbf{r^+ \df r^\star r}\\
                     & = \mathbf{1^\star0 + 1^\star 0 (0 + 1)^\star} & \comment \mathbf{r^\star = (\bm{\varepsilon} + r)^+}\\
                     & = \mathbf{1^\star 0 (\bm{\varepsilon} + (0 + 1)^\star)} & \comment \mathbf{r + rq = r(\bm{\varepsilon} + q)}\\
                     & = \mathbf{1^\star 0 (0 + 1)^\star} & \comment \mathbf{r^\star = \bm{\varepsilon} + r^\star}
\end{align*}

Тук използвахме, че:
\begin{align*}
  \mathbf{r^1_{0,1}} & = \underbrace{\mathbf{r}^0_{0,1}}_{\mathbf{0}} + \underbrace{\mathbf{r}^0_{0,0}}_{\bm{\varepsilon + 1}}\cdot\underbrace{\mathbf{(r^0_{0,0})^\star}}_{\bm{(\varepsilon+1)^\star}} \cdot \underbrace{\mathbf{r^0_{0,1}}}_{\mathbf{0}}\\
                     & = \bm{0 + (\varepsilon + 1)(\varepsilon + 1)^\star0} & \comment{\text{просто заместваме}}\\
                     & = \mathbf{0 + 1^\star 0}  & \comment \mathbf{r}^\star = \bm{\varepsilon} + \mathbf{r}^\star \\
                     & = \mathbf{1^\star0} \\
  \mathbf{r}^1_{1,1} & = \underbrace{\mathbf{r}^0_{1,1}}_{\bm{\varepsilon+0+1}} + \underbrace{\mathbf{r}^0_{1,0}}_{\bm{\emptyset}} \cdot \underbrace{(\mathbf{r}^0_{0,0})^\star}_{\bm{\varepsilon+1}} \cdot \underbrace{\mathbf{r}^0_{0,1}}_{\bm{0}}\\
                     & = \bm{\varepsilon + 0 + 1 + \emptyset(\varepsilon + 1)^\star0} & \comment{\text{просто заместваме}}\\
                     & = \bm{\varepsilon + 0 + 1} & \comment \text{защото }\bm{\emptyset \cdot r = \emptyset}
\end{align*}
\end{example}


\begin{problem}
  Приложете конструкцията от Теоремата на Клини за да получите регулярен израз, който описва езика на автомата $\A$.
  
  \begin{framed}
    \begin{figure}[H]
      \begin{center}
        \begin{tikzpicture}[->,>=stealth,thick,node distance=45pt]
          \tikzstyle{every state}=[circle,minimum size=15pt,auto]
          
          \node[initial above,state]      (1) {$q_0$};
          \node[state]              (2) [right of=1]{$q_1$};
          \node[accepting, state]   (3) [right of=2]{$q_2$};
          
          \path 
          (2) edge [loop above]    node [above] {$a$} (2)
          (1) edge [bend left=15]  node [above] {$a$} (2)
          (2) edge [bend left=15]  node [above] {$b$} (3)
          (1) edge [bend right=45] node [below] {$b$} (3)
          (3) edge [loop above]    node [above] {$a,b$} (3);
        \end{tikzpicture}
      \end{center}
      \caption{Автомат разпознаващ $\L(\mathbf{a^\star b(a+b)^\star})$.}
      \label{fig:a1}
    \end{figure}
  \end{framed}
\end{problem}
\begin{solution}
  $\L(\A) = \L(\mathbf{r}^3_{0,2})$, където:
  \begin{align*}
    \mathbf{r}^3_{0,2} & = \underbrace{\mathbf{r}^2_{0,2}}_{\bm{a^\star b}} + \underbrace{\mathbf{r}^2_{0,2}}_{\bm{a^\star b}} \cdot \underbrace{(\mathbf{r}^2_{2,2})^\star}_{\bm{(\varepsilon+a+b)^\star}} \cdot \underbrace{\mathbf{r}^2_{2,2}}_{\bm{\varepsilon+a+b}}\\
                       & = \bm{a^\star b + a^\star b \cdot (a+b)^\star \cdot (\varepsilon+a+b)} \\
                       & = \bm{a^\star b + a^\star b \cdot (a+b)^\star} & \comment \bm{r^\star} = \bm{r^\star \cdot (\varepsilon + r)}\\
                       & = \bm{a^\star b (\varepsilon + (a+b)^\star)} & \comment\bm{r_1 + r_1\cdot r_2 = r_1 \cdot (\varepsilon+r_2)}\\
                       & = \bm{a^\star b (a+b)^\star}. & \comment \bm{r^\star} = \bm{\varepsilon + r^\star}
  \end{align*}
  
  Тук използвахме, че:
  \begin{align*}
    \mathbf{r}^2_{0,2} & = \mathbf{r}^1_{0,2} + \mathbf{r}^1_{0,1} \cdot (\mathbf{r}^1_{1,1})^\star \cdot \mathbf{r}^1_{1,2}\\
                       & = \bm{b + a \cdot a^\star \cdot b}\\
                       & = \bm{(\varepsilon + a^+)\cdot b} & \comment \mathbf{r}^+ \df \mathbf{r} \cdot \mathbf{r}^\star\\
                       & = \bm{a^\star b}. & \comment \mathbf{r}^\star = \bm{\varepsilon + r^+.г}
  \end{align*}
\end{solution}

%%% Local Variables:
%%% mode: latex
%%% TeX-master: "../eai"
%%% End:
