\begin{extra}
  Нека да вземем назаем от \texttt{python} нотацията за \texttt{slices} на масив и
  ако $\alpha = a_0 \cdot a_1 \cdots a_{n-1}$, то нека 
  \begin{align*}
    \alpha[i:j] & \df
                  \begin{cases}
                    a_i \cdots a_{m-1}, & \text{ ако }i < j\ \&\ m = \min\{j,|\alpha|\}\\
                    \varepsilon, & \text{ иначе}
                  \end{cases}\\
    \alpha[i:] & \df \alpha[i:|\alpha|]\\
    \alpha[:i] & \df \alpha[0:i].
  \end{align*}
Нека също така $\Sigma^{\geq k} \df \{ \alpha \in \Sigma^\star \mid |\alpha| \geq k\}$.

\begin{problem}
  Нека $L$ е регулярен език. Тогава езиците
  \begin{itemize}
  \item
    $L_1 = \{ \alpha \in \Sigma^\star \mid (\exists i)(\exists j)[\ \alpha[i:j] \in L\ ]\}$;
  \item
    $L_2 = \{ \alpha \in \Sigma^\star \mid (\forall i)(\forall j)[\ i < j \implies \alpha[i:j] \in L\ ]\}$;
  \item 
    $L_3 = \{ \alpha \in \Sigma^\star \mid (\forall i)(\exists j)[\ i < j\ \&\ \alpha[i:j] \in L\ ]\}$;
  \item
    $L_4 = \{ \alpha \in \Sigma^\star \mid (\exists i)(\forall j)[\ i < j \implies \alpha[i:j] \in L\ ]\}$.
  \end{itemize}
  също са регулярни.
\end{problem}
\begin{hint}
  Лесно се вижда, че
  \[L_1 = \Sigma^\star \cdot L \cdot \Sigma^\star,\]
  както и следното
  \[L_2 = \ov{\Sigma^\star \cdot \ov{L} \cdot \Sigma^\star}.\]
  Да обърнем внимание, че
  \begin{align*}
    \alpha \in \Sigma^\star \cdot L & \iff (\exists \beta \in \Sigma^\star)(\exists \gamma \in L)[ \alpha = \beta \cdot \gamma]\\
                                    & \iff (\exists j)[\ \alpha[j:] \in L\ ].
  \end{align*}
  Аналогично получаваме, че
    \begin{align*}
      \alpha \in L\cdot \Sigma^\star & \iff (\exists \gamma \in L)(\exists \beta \in \Sigma^\star)[ \alpha = \gamma \cdot \beta ]\\
                                     & \iff (\exists j)[\ \alpha[:j] \in L\ ].
  \end{align*}
  Нека да разгледаме по-подробно следния език:
  \begin{align*}
    \ov{L}_3 & = \{\alpha \in \Sigma^\star \mid (\exists i)(\forall j)[\ i < j \implies \alpha[i:j] \not\in L\ ]\}\\
             & = \{\alpha \in \Sigma^\star \mid (\exists i)(\forall j)[\ i < j \implies \alpha[i:j] \in \ov{L}\ ]\}\\
             & = \{\alpha \in \Sigma^\star \mid (\forall j)[\ \alpha[:j] \in \Sigma^\star \cdot \ov{L}\ ]\}
  \end{align*}
  Така получаваме, че:
  \begin{align*}
    L_3 & = \{\alpha \in \Sigma^\star \mid (\exists j)[\ \alpha[:j] \not\in \Sigma^\star \cdot \ov{L}\ ]\}\\
        & = \{\alpha \in \Sigma^\star \mid (\exists j)[\ \alpha[:j] \in \ov{\Sigma^\star \cdot \ov{L}}\ ]\}\\
        & = \{\alpha \in \Sigma^\star \mid \alpha \in (\ov{\Sigma^\star \cdot \ov{L}}) \cdot \Sigma^\star \ ]\}.
  \end{align*}
  Оттук заключваме, че
  \[L_3 = (\ov{\Sigma^\star \cdot \ov{L}}) \cdot \Sigma^\star.\]
  Сега лесно можем да съобразим, че
  \[ L_4 = \ov{ \ov{(\Sigma^\star \cdot L)} \cdot \Sigma^\star}.\]
\end{hint}

\begin{problem}
  Нека $L$ е регулярен език над азбуката $\Sigma$.
  За произволно естествено число $k$, докажете, че езикът
  \[L_k = \{ \alpha \in \Sigma^\star \mid (\forall i)(\exists j)[\ |\alpha[i:]| \geq k \implies \alpha[i:j] \in L\ ]\}\]
  е регулярен.
  С други думи, $L_k$ съдържа тези думи, за които от всяка позиция, с изключение на последните $k$, започва дума в езика $L$.
\end{problem}
\begin{hint}
  Да разпишем по-подробно дефиницията на езика $L_k$.
  \begin{align*}
    L_k & = \{ \alpha \in \Sigma^\star \mid (\forall i)(\exists j)[\ |\alpha[i:]| \geq k \implies \alpha[i:j] \in L\ ]\}\\
        & = \{ \alpha \in \Sigma^\star \mid (\forall i)[\ |\alpha[i:]| \geq k \implies (\exists j)[\alpha[i:j] \in L]\ ]\}\\
        & = \{ \alpha \in \Sigma^\star \mid (\forall i)[\ \alpha[i:] \in \Sigma^{\geq k} \implies \alpha[i:] \in L \cdot \Sigma^\star\ ]\}\\
        & = \{ \alpha \in \Sigma^\star \mid (\forall i)[\ \alpha[i:] \not\in \Sigma^{\geq k}\ \lor\ \alpha[i:] \in L \cdot \Sigma^\star\ ]\}\\
        & = \{ \alpha \in \Sigma^\star \mid \neg (\exists i)[\ \alpha[i:] \in \Sigma^{\geq k}\ \&\ \alpha[i:] \not\in L \cdot \Sigma^\star\ ]\}.
  \end{align*}
  Сега е ясно, че:
  \begin{align*}
    \ov{L}_k & = \{ \alpha \in \Sigma^\star \mid (\exists i)[\ \alpha[i:] \in \Sigma^{\geq k}\ \&\ \alpha[i..] \not\in L \cdot \Sigma^\star\ ]\}\\
             & = \{ \alpha \in \Sigma^\star \mid (\exists i)[\ \alpha[i:] \in \Sigma^{\geq k}\ \&\ \alpha[i..] \in \ov{L \cdot \Sigma^\star}\ ]\}\\
             & = \{ \alpha \in \Sigma^\star \mid (\exists i)[\ \alpha[i:] \in \Sigma^{\geq k} \cap \ov{L \cdot \Sigma^\star}\ ]\}\\
             & = \{ \alpha \in \Sigma^\star \mid \alpha \in \Sigma^\star \cdot (\Sigma^{\geq k} \cap \ov{L \cdot \Sigma^\star}) \}\\
             & = \Sigma^\star \cdot (\Sigma^{\geq k} \cap \ov{L \cdot \Sigma^\star}).
  \end{align*}
  Тогава
  \[L_k = \ov{\Sigma^\star \cdot (\Sigma^{\geq k} \cap \ov{L \cdot \Sigma^\star})}.\]
\end{hint}

\end{extra}


%%% Local Variables:
%%% mode: latex
%%% TeX-master: "../eai"
%%% End:
