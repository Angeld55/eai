\section{Изоморфни автомати}\label{sect:isomorphic}
\index{изоморфизъм}
\mynote{\cite[стр. 89]{kozen}}

Нека са дадени автоматите
$\A_1 = (\Sigma,Q_1,\qstart',\delta_1,F_1)$ и $\A_2 = (\Sigma, Q_2, \qstart'', \delta_2, F_2)$.
Казваме, че $\A_1$ и $\A_2$ са {\bf изоморфни}, което означаваме с $\A_1 \cong \A_2$, ако
съществува функция $f: Q_1\to Q_2$, за която:
\begin{enumerate}[(1)]
\item
  $f$ е биекция;
\item
  $f(\qstart') = \qstart''$;
\item
  $q \in F_1 \iff f(q) \in F_2$;
\item
  $(\forall a\in\Sigma)(\forall q\in Q_1)[\ f(\delta_1(q,a)) = \delta_2(f(q),a)\ ]$.
\end{enumerate}
Ще казваме, че $f$ задава изоморфизъм на $\A_1$ върху $\A_2$ и ще означаваме $\A_1 \cong_f \A_2$ или $f:\A_1 \cong \A_2$.

\begin{proposition}
  Нека $\A_1 \cong_f \A_2$. Тогава за всяка дума $\alpha$ и произволно състояние $q \in Q_1$ е изпъленено, че
  \begin{equation}
    \label{eq:3}
    f(\delta^\star_1(q,\alpha)) = \delta^\star_2(f(q), \alpha).
  \end{equation}
\end{proposition}
\begin{proof}
  Както винаги, ще докажем Свойство (\ref{eq:3}) с индукция по дължината на думата $\alpha$.
  \begin{itemize}
  \item 
    Нека $|\alpha| = 0$, т.е. $\alpha = \varepsilon$. Тогава:
    \begin{align*}
      f(\delta^\star_1(q,\varepsilon)) & = f(q) & \comment{\text{от деф. на }\delta^\star_1}\\
                                           & = \delta^\star_2(f(q), \varepsilon). & \comment{\text{от деф. на }\delta^\star_2}
    \end{align*}
  \item
    Да приемем, че (\ref{eq:3}) е изпълнено за думи с дължина $n$.
  \item
    Да разгледаме произволна дума $\alpha$ с дължина $n+1$, т.е. $\alpha = \beta c$ и $|\beta| = n$. Тогава:
    \begin{align*}
      f(\delta^\star_1(q,\beta c)) & = f(\delta_1(\delta^\star_1(q,\beta), c)) & \comment{\text{от деф. на }\delta^\star_1}\\
                                   & = \delta_2( f(\delta^\star_1(q,\beta)), c) & \comment{f\text{ е изоморфизъм}}\\
                                   & = \delta_2( \delta^\star_2(f(q),\beta), c) & \comment{\text{от И.П. за }\beta}\\
                                   & = \delta^\star_2(f(q), \beta c) & \comment{\text{от деф. на }\delta^\star_2}.
    \end{align*}
  \end{itemize}
\end{proof}

\begin{framed}
  \begin{proposition}
    Ако $\A_1 \cong \A_2$, то $\L(\A_1) = \L(\A_2)$.
  \end{proposition}  
\end{framed}
\begin{hint}
  \mynote{Лесно можем да съобразим, че в общия случай нямаме обратната посока на това твърдение.}
  Нека $\A_1 \cong_f \A_2$. Тогава имаме следните еквивалентности:
  \begin{align*}
    \alpha \in \L(\A_1) & \iff \delta^\star_1(\qstart',\alpha) \in F_1 & \comment\text{деф. на }\L(\A_1)\\
                       & \iff f(\delta^\star_1(\qstart',\alpha)) \in F_2 & \comment{f\text{ е изоморфизъм}}\\
                       & \iff \delta^\star_2(f(\qstart'),\alpha) \in F_2 & \comment{\text{от (\ref{eq:3})}}\\
                       & \iff \delta^\star_2(\qstart'',\alpha) \in F_2 & \comment{f(\qstart') \df \qstart''}\\
                       & \iff \alpha \in \L(\A_2). & \comment\text{деф. на }\L(\A_2)
  \end{align*}
\end{hint}

%%% Local Variables:
%%% mode: latex
%%% TeX-master: "../eai"
%%% End:
