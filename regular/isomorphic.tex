\section{Изоморфни автомати}\label{sect:isomorphic}
\index{изоморфизъм}
\mynote{\cite[стр. 89]{kozen}}

Нека са дадени автоматите
$\A_1 = (\Sigma,Q_1,\qstart',\delta_1,F_1)$ и $\A_2 = (\Sigma, Q_2, \qstart'', \delta_2, F_2)$.
Казваме, че $\A_1$ и $\A_2$ са {\bf изоморфни}, което означаваме с $\A_1 \cong \A_2$, ако
съществува {\em биективна} функция $f: Q_1\to Q_2$, за която:
% \mynote{С други думи, автоматът $\A_2$ може да се получи от автомата $\A_1$ с преименуване на състоянията, като внимаваме да запазим свойствата на автомата, т.е. начално отива в начално, финални отиват във финални и }
\mynote{
  % \begin{figure}[H]
  Условие (3) може да се представи графично така:
  
      \begin{tikzpicture}[->,>=stealth,thick,node distance=60pt,scale=0.8,every node/.style={scale=0.8}]
        \tikzstyle{every state}=[circle,minimum size=20pt,auto]
        
        \node[state]                  (1)              {$q$};
        \node[state]                  (2) [right of=1] {$p$};

        \node[state,inner sep=1pt]    (3) [below of=1] {\scriptsize{$f(q)$}};
        \node[state,inner sep=1pt]    (4) [below of=2] {\scriptsize{$f(p)$}};
        
        \path 
        (1) edge [bend left=15]  node [above] {$a$} (2)
        (3) edge [bend right=15] node [below] {$a$} (4)
        (1) edge [dashed, bend right=15] node [left]  {$f$} (3)
        (2) edge [dashed, bend left=15]  node [right] {$f$} (4);
      \end{tikzpicture}
      % \caption{Свойство (3)}
      % \end{figure}
}
\begin{enumerate}[(1)]
\item
  $f(\qstart') = \qstart''$;
\item
  $q \in F_1 \iff f(q) \in F_2$;
\item
  $\delta_1(q,a) = p\ \iff\ \delta_2(f(q),a) = f(p)$.
\end{enumerate}
С други думи, два автомата са изоморфни точно тогава, когато те са идентични с точност до
преименуване на състоянията.  

Ще казваме, че $f$ задава изоморфизъм на $\A_1$ върху $\A_2$ и ще означаваме $\A_1 \cong_f \A_2$ или $f:\A_1 \cong \A_2$.

\begin{proposition}
  Нека $\A_1 \cong_f \A_2$. Тогава за всяка дума $\alpha$ и състояние $q \in Q_1$
  е изпъленена импликацията:
  \begin{equation}
    \label{eq:3}
    \delta^\star_1(q,\alpha) = p\ \iff\ \delta^\star_2(f(q), \alpha) = f(p).
  \end{equation}
\end{proposition}
\mynote{
  Свойство (\ref{eq:3}) може да се представи графично така:
     \begin{tikzpicture}[->,>=stealth,thick,node distance=56pt,initial text=начало,scale=0.8,every node/.style={scale=0.8}]
        \tikzstyle{every state}=[circle,minimum size=20pt,auto]
        
        \node[state]          (1)              {$q$};
        \node[state]          (2) [right of=1] {$p$};

        \node[state, inner sep=1pt]   (3) [below of=1] {$f(q)$};
        \node[state, inner sep=1pt]   (4) [below of=2] {$f(p)$};
        
        \path
        (1) edge [dashed, bend right=15] node [left]  {$f$} (3)
        (2) edge [dashed, bend left=15] node [right] {$f$} (4);

        \draw [photon] (1) -- node [above] {$\alpha$} (2);
        \draw [photon] (3) -- node [above] {$\alpha$} (4);
      \end{tikzpicture}
    }
\begin{proof}
  Както винаги, ще докажем \Property{eq:3} с индукция по дължината на думата $\alpha$.
  \begin{itemize}
  \item 
    Нека $|\alpha| = 0$, т.е. $\alpha = \varepsilon$. Ясно е, че \Property{eq:3}
    е изпълнено за $\varepsilon$ защото $\delta^\star_1(q,\varepsilon) = q$ и съответно
    $\delta^\star_2(f(q),\varepsilon) = f(q)$ за произволно състояние $q \in Q_1$.
    % \[\delta^\star_1(q,\varepsilon) = q \]
    % \begin{align*}
    %   \delta^\star_1(q,\varepsilon)) & = f(q) & \comment{\text{от деф. на }\delta^\star_1}\\
    %                                        & = \delta^\star_2(f(q), \varepsilon). & \comment{\text{от деф. на }\delta^\star_2}
    % \end{align*}
  \item
    Да приемем, че \Property{eq:3} е изпълнено за думи с дължина $n$.
  \item
    \mynote{Индукционната стъпка може да се представи така:
     \begin{tikzpicture}[->,>=stealth,thick,node distance=45pt,initial text=начало,scale=0.8,every node/.style={scale=0.8}]
        \tikzstyle{every state}=[circle,minimum size=20pt,auto]
        
        \node[state]          (1)              {$q$};
        \node[state]          (2) [right of=1] {$r$};
        \node[state]          (3) [right of=2] {$p$};

        \node[state, inner sep=1pt]   (4) [below of=1] {$f(q)$};
        \node[state, inner sep=1pt]   (5) [below of=2] {$f(r)$};
        \node[state, inner sep=1pt]   (6) [below of=3] {$f(p)$};
        
        \path
        (1) edge [dashed, bend right=15] node [left]  {$f$} (4)
        (2) edge [dashed, bend right=15] node [left]  {$f$} (5)
        (3) edge [dashed, bend left=15] node [right] {$f$} (6)
        (2) edge [bend left=15] node [above] {$c$} (3)
        (5) edge [bend right=15] node [below] {$c$} (6);
        

        \draw [photon] (1) -- node [above] {$\beta$} (2);
        \draw [photon] (4) -- node [above] {$\beta$} (5);
      \end{tikzpicture}      
    }
    Да разгледаме произволна дума $\alpha$ с дължина $n+1$, т.е. $\alpha = \beta c$ и $|\beta| = n$. Тогава:
    \[\delta^\star_1(q,\beta c) = p \iff \delta_1(\underbrace{\delta^\star_1(q,\beta)}_{r}, c) = p.\]
    Понеже $f$ е изоморфизъм, то
    \[\delta_1(r,c) = p\ \iff\ \delta_2(f(r),c) = f(p).\]
    От \IndHyp за \Property{eq:3} имаме, че:
    \[\delta^\star_1(q,\beta) = r \iff \delta^\star_2(f(q),\beta) = f(r).\]
    Комбинираме тези две еквивалентности и получаваме следното:
    \begin{align*}
      \delta^\star_1(q,\beta c) = p & \iff \delta_1(\underbrace{\delta^\star_1(q,\beta)}_{r}, c) = p\\
                                    & \iff \delta_1(r,c) = p\\
                                    & \iff \delta_2(f(r),c) = f(p)\\
                                    & \iff \delta_2(\delta^\star_2(f(q),\beta),c) = f(p)\\
                                    & \iff \delta^\star(f(q),\beta c) = f(p).
    \end{align*}
    
    % \begin{align*}
    %   \delta^\star_1(q,\beta c) = p & \iff \delta_1(\delta^\star_1(q,\beta), c) = p\\
    %                                 & \iff 
    % \end{align*}

    
    % \begin{align*}
    %   f(\delta^\star_1(q,\beta c)) & = f(\delta_1(\delta^\star_1(q,\beta), c)) & \comment{\text{от деф. на }\delta^\star_1}\\
    %                                & = \delta_2( f(\delta^\star_1(q,\beta)), c) & \comment{f\text{ е изоморфизъм}}\\
    %                                & = \delta_2( \delta^\star_2(f(q),\beta), c) & \comment{\text{от И.П. за }\beta}\\
    %                                & = \delta^\star_2(f(q), \beta c) & \comment{\text{от деф. на }\delta^\star_2}.
    % \end{align*}
  \end{itemize}
\end{proof}

\begin{framed}
  \begin{proposition}
    Ако $\A_1 \cong \A_2$, то $\L(\A_1) = \L(\A_2)$.
  \end{proposition}  
\end{framed}
\begin{hint}
  \mynote{Лесно можем да съобразим, че в общия случай нямаме обратната посока на това твърдение.}
  Нека $\A_1 \cong_f \A_2$. Тогава имаме следните еквивалентности:
  \begin{align*}
    \alpha \in \L(\A_1) & \iff \delta^\star_1(\qstart',\alpha) = p \in F_1 & \comment\text{деф. на }\L(\A_1)\\
                       & \iff \delta^\star_2(f(\qstart'),\alpha)) = f(p) \in F_2 & \comment{\text{\Property{eq:3}}}\\
                       % & \iff \delta^\star_2(f(\qstart'),\alpha) \in F_2 & \comment{\text{от (\ref{eq:3})}}\\
                       & \iff \delta^\star_2(\qstart'',\alpha) \in F_2 & \comment{f(\qstart') \df \qstart''}\\
                       & \iff \alpha \in \L(\A_2). & \comment\text{деф. на }\L(\A_2)
  \end{align*}
\end{hint}

%%% Local Variables:
%%% mode: latex
%%% TeX-master: "../eai"
%%% End:
