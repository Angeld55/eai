\section{Изоморфни автомати}
\label{sect:isomorphic}

\index{изоморфизъм}
Нека са дадени автоматите
$\A' = (\Sigma,Q',\qstart',\delta',F')$ и $\A'' = (\Sigma, Q'', \qstart'', \delta'', F'')$.
Казваме, че $\A'$ и $\A''$ са {\bf изоморфни}, което означаваме с $\A' \cong \A''$, ако
съществува функция $f: Q'\to Q''$, за която:
\begin{enumerate}[(1)]
\item
  $f$ е биекция;
\item
  $f(\qstart') = \qstart''$;
\item
  $q \in F' \iff f(q) \in F''$;
\item
  $(\forall a\in\Sigma)(\forall q\in Q')[f(\delta'(q,a)) = \delta''(f(q),a)]$.
\end{enumerate}
Ще казваме, че $f$ задава изоморфизъм на $\A'$ върху $\A''$.

\begin{framed}
  \begin{thm}
    Ако $\A' \cong_f \A''$, то $\L(\A') = \L(\A'')$.
  \end{thm}  
\end{framed}
\begin{hint}
  Нека $\A' \cong_f \A''$. Първо с индукция по дължината на думата $\alpha$ ще докажем, че за произволно състояние $q \in Q'$,
  \begin{equation}
    \label{eq:3}
    f(\delta^\star_{\A'}(q,\alpha)) = \delta^\star_{\A''}(f(q), \alpha).
  \end{equation}
  \begin{itemize}
  \item 
    Нека $|\alpha| = 0$, т.е. $\alpha = \varepsilon$. Тогава:
    \begin{align*}
      f(\delta^\star_{\A'}(q,\varepsilon)) & = f(q) & \comment{\text{от деф. на }\delta^\star_{\A'}}\\
                                           & = \delta^\star_{\A''}(f(q), \varepsilon).
    \end{align*}
  \item
    Да приемем, че (\ref{eq:3}) е изпълнено за думи с дължина $n$.
    Да разгледаме произволна дума $\alpha$ с дължина $n+1$, т.е. $\alpha = \beta x$. Тогава:
    \begin{align*}
      f(\delta^\star_{\A'}(q,\beta x)) & = f(\delta_{\A'}(\delta^\star_{\A'}(q,\beta), x)) & \comment{\text{от деф. на }\delta^\star_{\A'}}\\
                                       & = \delta_{\A''}( f(\delta^\star_{\A'}(q,\beta), x)) & \comment{\text{от деф. на }f}\\
                                       & = \delta_{\A''}( \delta^\star_{\A''}(f(q),\beta), x)) & \comment{\text{от И.П.}}\\
                                       & = \delta^\star_{\A''}(f(q), \beta x) & \comment{\text{от деф. на }\delta^\star_{\A''}}.
    \end{align*}
  \end{itemize}
  Сега вече е лесно:
  \begin{align*}
    \alpha \in \L(\A') & \dff \delta^\star_{\A'}(\qstart',\alpha) \in F' \\
                       & \iff f(\delta^\star_{\A'}(\qstart',\alpha)) \in F'' & \comment{\text{от деф. на изоморфизъм}}\\
                       & \iff \delta^\star_{\A''}(f(\qstart'),\alpha) \in F'' & \comment{\text{от (\ref{eq:3})}}\\
                       & \iff \delta^\star_{\A''}(\qstart'',\alpha) \in F'' & \comment{f(\qstart') \df \qstart''}\\
                       & \dff \alpha \in \L(\A'').
  \end{align*}
\end{hint}

\index{Бжозовски}
Нека е даден регулярен език $L$ над азбуката $\Sigma$.
Конструкцията на автомата $\B$ по метода на Бжозовски е следната:
\begin{itemize}
\item 
  $Q^\B \df \{\alpha^{-1}(L) \mid \alpha \in \Sigma^\star\}$;
\item
  $\qstart^\B \df L$;
\item
  $F^\B \df \{N \in Q^\B \mid \varepsilon \in N\}$;
\item
  $\delta_\B(N,x) \df x^{-1}(N)$, за произволни $x \in \Sigma$ и $N \in Q^\B$.
\end{itemize}

Да припомним конструкцията на минималния автомат $\M$ според \hyperref[th:myhill-nerode]{Теоремата на Майхил-Нероуд}.
\begin{itemize}
\item 
  $Q^\M \df \{[\alpha]_L \mid \alpha \in \Sigma^\star\}$;
\item
  $\qstart^\M \df [\varepsilon]_L$;
\item
  $F^\M \df \{[\alpha]_L \mid [\alpha]_L \subseteq L\}$;
\item
  $\delta_\M([\alpha]_L,x) \df [\alpha\cdot x]_L$, за произволни $x \in \Sigma$ и $\alpha \in \Sigma^\star$.
\end{itemize}

\begin{framed}
  \begin{thm}
    $\B \cong \M$.
  \end{thm}  
\end{framed}
\begin{hint}
  Нека да дефинираме $f:Q^\M \to Q^\B$ по следния начин:
  \[f([\alpha]_L) \df \alpha^{-1}(L).\] 
  Ще докажем, че $f$ изпълнява свойствата за изоморфизъм.

  \begin{enumerate}[(1)]
  \item 
    Трябва първо да проверим, че $f$ е биектвна, т.е. $f$
    е инективна и сюрективна.
    \begin{itemize}
    \item
      Дефиницията на $f$ е зададена спрямо представител $\alpha$ на класа на еквивалентност на релацията $\approx_L$.
      \marginpar{Това е важно да се провери, защото дясната страна е дефинирана спрямо произволен представител на класа $[\alpha]_L$}
      Първо да проверим, че $f$ е функция.
      Нека $[\alpha]_L = [\beta]_L$, т.е. $\alpha^{-1}(L) = \beta^{-1}(L)$. Тогава 
      \begin{align*}
        f([\alpha]_L) & \df \alpha^{-1}(L)\\
                      & = \beta^{-1}(L) & \comment{\text{от } [\alpha]_L = [\beta]_L}\\
                      & \df f([\beta]_L).
      \end{align*}
    \item 
      Нека $[\alpha]_L \neq [\beta]_L$.
      Тогава:
      \begin{align*}
        f([\alpha]_L) & \df \alpha^{-1}(L)\\
                      & \neq \beta^{-1}(L) & \comment{\text{от }[\alpha]_L \neq [\beta]_L}\\
                      & \df f([\beta]_L).
      \end{align*}
      Оттук следва, че $f$ е {\em инективна}.
    \item
      Да разгледаме произволен елемент $N \in Q^\B$, т.е. $N$ е множество от думи и $N = \alpha^{-1}(L)$, за някоя дума $\alpha \in \Sigma^\star$.
      Понеже $f([\alpha]_L) \df \alpha^{-1}(L)$, то това означава, че $f$ е {\em сюрективна}.      
    \end{itemize}
  \item
    Лесно се съобразява, че
    \begin{align*}
      f(\qstart^\M) & = f([\varepsilon]_L) & \comment{\qstart^\M \df [\varepsilon]_L}\\
                    & \df \varepsilon^{-1}(L)\\
                    & = L \\
                    & \df \qstart^\B. & \comment{\qstart^\B \df L}
    \end{align*}
  \item
    Също не е трудно да се съобрази, че
    \begin{align*}
      [\alpha]_L \in F^\M & \dff [\alpha]_L \subseteq L\\
                          & \iff \varepsilon \in \alpha^{-1}(L)\\
                          & \iff f([\alpha]_L) \df \alpha^{-1}(L) \in F^\B.
    \end{align*}
  \item
    Имаме и свойството за произволна дума $\alpha \in \Sigma^\star$ и произволна буква $x \in \Sigma$:
    \begin{align*}
      f(\delta_\M([\alpha]_L,x)) & = f([\alpha\cdot x]_L) & \comment{\text{деф. на }\delta_\M}\\
                                 & = (\alpha\cdot x)^{-1}(L) & \comment{\text{деф. на }f}\\
                                 & = x^{-1}(\alpha^{-1}(L)) & \comment{\text{от \Prob{pullback}}}\\
                                 & = \delta_\B(\alpha^{-1}(L), x) & \comment{\text{деф. на }\delta_\B}\\
                                 & = \delta_\B(f([\alpha]_L), x), & \comment{\text{деф. на }f}
    \end{align*}
    от което следва, че $f$ е {\em изоморфизъм}.
  \end{enumerate}
\end{hint}

\begin{cor}
  Автоматът $\B$, построен по метода на Бжожовски, за регулярния език $L$ е минимален.
\end{cor}


\begin{framed}
  \begin{thm}
    \label{th:regular:isomorphic:minimal}
    Нека е даден регулярния език $L$.
    Нека $\A = \FA$ е произволен тотален автомат, за който $\L(\A) = L$ и $\abs{Q} = \abs{\Sigma^\star/_{\approx_L}}$.
    Тогава $\A \cong \M$, където $\M$ е автоматът построен според \hyperref[th:myhill-nerode]{Теоремата на Майхил-Нероуд} за езика $L$.
  \end{thm}  
\end{framed}
\begin{proof}
  Съобразете, че $\A$ е {\em свързан}, т.е. всяко състояние на $\A$ е достижимо от началното.
  Искаме да докажем, че $\A \cong \M$.
  Понеже $\A$ е свързан, за всяко състояние $q$ можем да намерим дума $\omega_q$,
  за която $\delta^\star(\qstart,\omega_q) = q$.
  Да дефинираме изображението $f:Q^\A\to Q^\M$ като 
  \[f(q) \df [\omega_q]_L.\]
  Ще докажем, че $f$ задава изоморфизъм на $\A$ върху $\M$. 
  \begin{enumerate}[(1)]
  \item
    Първо да съобразим, че $f:Q^\A \to Q^\M$ е биекция.
    \begin{itemize}
    \item
      Първо да съобразим, че ако $\delta^\star_\A(\qstart,\alpha) = q$, то $[\omega_q]_L = [\alpha]_L$.
      Понеже $\delta^\star_\A(\qstart,\alpha) = q = \delta^\star_\A(\qstart,\omega_q)$, то $\omega_q \sim_\A \alpha$.
      От \Prop{rel-finer} имаме, че
      \[\omega_q \sim_\A \alpha \implies \omega_q \approx_L \alpha.\]
      Това означава, че $[\omega_q]_L = [\alpha]_L$ и следователно $f$ е определена коректно, т.е. $f$ е {\bf функция}.
    \item
      Ще проверим, че $f$ е {\bf инективна}, т.е.
      \[(\forall q_1,q_2 \in Q)[q_1\neq q_2 \implies f(q_1) \neq f(q_2)].\]
      Да допуснем, че има състояния $q_1 \neq q_2$, за които 
      \[f(q_1) = [\omega_{q_1}]_L = [\omega_{q_2}]_L = f(q_2).\]
      Тогава $\omega_{q_1} \not\sim_\A \omega_{q_2}$ и $\omega_{q_1} \approx_L \omega_{q_2}$.
      \marginpar{\writedown Обяснете!}
      Но тогава от \Cor{upper-bound} получаваме, че $\abs{Q} = \abs{\Sigma^\star/_{\sim_\A}} > \abs{\Sigma^\star/_{\approx_L}}$,
      което противоречи с минималността на $\A$.
    \item
      За да бъде $f$ {\bf сюрективна} трябва за всеки клас $[\beta]_L$ да съществува състояние $q$, за което $f(q) = [\beta]_L$.
      Понеже $\A$ е свързан, съществува състояние $q$, за което $\delta^\star_\A(\qstart,\beta) = q$.
      Вече се убедихме, че в този случай $\beta \approx_L \omega_q$, защото $\beta \sim_\A \omega_q$.
      Тогава $f(q) = [\omega_q]_L = [\beta]_L$.
    \end{itemize}
  \item
    Понеже $\delta^\star_\A(\qstart,\varepsilon) = \qstart$,
    то е ясно, че $f(\qstart) = [\varepsilon]_L$.
  \item
    Също лесно се съобразява, че
    $q \in F^\A \iff f(q) \in F^\M$.
  \item
    За последно оставихме проверката, че $f$ наистина е {\bf изоморфизъм}:
    \begin{align*}
      f(\delta_\A(q,a)) & = f(\delta_\A(\delta^\star_\A(\qstart,\omega_q),a)) & \comment{\text{от избора на }\omega_q}\\
      & = f(\delta^\star_\A(\qstart,\omega_qa)) & \comment{\text{от деф. на }\delta^\star_\A}\\
      & = [\omega_qa]_L & \comment{\text{от деф. на }f}\\
      & = \delta^\star_{\M}([\varepsilon]_L, \omega_qa) & \comment{\text{от деф. на }\M}\\ 
      & = \delta_{\M}(\delta^\star_{\M}([\varepsilon]_L, \omega_q),a) & \comment{\text{от деф. на }\delta^\star_{\M}}\\
      & = \delta_{\M}([\omega_q]_L, a) & \comment{\text{свойство на }\delta^\star_{\M}}\\
      & = \delta_{\M}(f(q), a) & \comment{f(q) \df [\omega_q]_L}.
    \end{align*}
  \end{enumerate}
\end{proof}




%%% Local Variables:
%%% mode: latex
%%% TeX-master: "../eai"
%%% End:
