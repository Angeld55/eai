\section{Изоморфни автомати}
\label{sect:isomorphic}
\index{изоморфизъм}

Нека са дадени автоматите
$\A' = (\Sigma,Q',\qstart',\delta',F')$ и $\A'' = (\Sigma, Q'', \qstart'', \delta'', F'')$.
Казваме, че $\A'$ и $\A''$ са {\bf изоморфни}, което означаваме с $\A' \cong \A''$, ако
съществува функция $f: Q'\to Q''$, за която:
\begin{enumerate}[(1)]
\item
  $f$ е биекция;
\item
  $f(\qstart') = \qstart''$;
\item
  $q \in F' \iff f(q) \in F''$;
\item
  $(\forall a\in\Sigma)(\forall q\in Q')[\ f(\delta'(q,a)) = \delta''(f(q),a)\ ]$.
\end{enumerate}
Ще казваме, че $f$ задава изоморфизъм на $\A'$ върху $\A''$ и ще означаваме $\A' \cong_f \A''$ или $f:\A' \cong \A''$.

\begin{proposition}
  Нека $\A' \cong_f \A''$. Тогава за всяка дума $\alpha$ и произволно състояние $q \in Q'$ е изпъленено, че
  \begin{equation}
    \label{eq:3}
    f(\delta^\star_{\A'}(q,\alpha)) = \delta^\star_{\A''}(f(q), \alpha).
  \end{equation}
\end{proposition}
\begin{proof}
  Както винаги, ще докажем Свойство (\ref{eq:3}) с индукция по дължината на думата $\alpha$.
  \begin{itemize}
  \item 
    Нека $|\alpha| = 0$, т.е. $\alpha = \varepsilon$. Тогава:
    \begin{align*}
      f(\delta^\star_{\A'}(q,\varepsilon)) & = f(q) & \comment{\text{от деф. на }\delta^\star_{\A'}}\\
                                           & = \delta^\star_{\A''}(f(q), \varepsilon). & \comment{\text{от деф. на }\delta^\star_{\A''}}\\
    \end{align*}
  \item
    Да приемем, че (\ref{eq:3}) е изпълнено за думи с дължина $n$.
  \item
    Да разгледаме произволна дума $\alpha$ с дължина $n+1$, т.е. $\alpha = \beta c$ и $|\beta| = n$. Тогава:
    \begin{align*}
      f(\delta^\star_{\A'}(q,\beta c)) & = f(\delta_{\A'}(\delta^\star_{\A'}(q,\beta), c)) & \comment{\text{от деф. на }\delta^\star_{\A'}}\\
                                       & = \delta_{\A''}( f(\delta^\star_{\A'}(q,\beta)), c) & \comment{f\text{ е изоморфизъм}}\\
                                       & = \delta_{\A''}( \delta^\star_{\A''}(f(q),\beta), c) & \comment{\text{от И.П. за }\beta}\\
                                       & = \delta^\star_{\A''}(f(q), \beta c) & \comment{\text{от деф. на }\delta^\star_{\A''}}.
    \end{align*}
  \end{itemize}
\end{proof}

\begin{framed}
  \begin{proposition}
    Ако $\A' \cong \A''$, то $\L(\A') = \L(\A'')$.
  \end{proposition}  
\end{framed}
\begin{hint}
  \marginpar{Лесно можем да съобразим, че в общия случай нямаме обратната посока на това твърдение.}
  Нека $\A' \cong_f \A''$. Тогава имаме следните еквивалентности:
  \begin{align*}
    \alpha \in \L(\A') & \iff \delta^\star_{\A'}(\qstart',\alpha) \in F' & \comment\text{деф. на }\L(\A')\\
                       & \iff f(\delta^\star_{\A'}(\qstart',\alpha)) \in F'' & \comment{f\text{ е изоморфизъм}}\\
                       & \iff \delta^\star_{\A''}(f(\qstart'),\alpha) \in F'' & \comment{\text{от (\ref{eq:3})}}\\
                       & \iff \delta^\star_{\A''}(\qstart'',\alpha) \in F'' & \comment{f(\qstart') \df \qstart''}\\
                       & \iff \alpha \in \L(\A''). & \comment\text{деф. на }\L(\A'')
  \end{align*}
\end{hint}

\begin{framed}
  \begin{thm}
    \label{th:regular:isomorphic:minimal}
    Нека регулярния език $L$ се разпознава от детерминирания краен автомат 
    $\A = \FA$, за който $\abs{Q} = \abs{\Sigma^\star/_{\approx_L}}$.
    Тогава $\A \cong \M$, където $\M$ е автоматът построен според \hyperref[th:myhill-nerode]{Теоремата на Майхил-Нероуд} за езика $L$.
  \end{thm}  
\end{framed}
\begin{proof}
  \marginpar{Съобразете, че $\A$ е {\em свързан}, т.е. всяко състояние на $\A$ е достижимо от началното.}
  Нека положим $q_\alpha \df \delta^\star_\A(\qstart, \alpha)$.
  Да дефинираме функцията $h:\Sigma^\star/_{\approx_L} \to Q$ по следния начин:
  \[h([\alpha]_L) \df q_\alpha.\]
  Ще докажем, че $h$ задава изоморфизъм на $\A$ върху $\M$. 
  \begin{enumerate}[(1)]
  \item
    Понеже $|\Sigma^\star/_{\approx_L}| = |Q|$, от \Prop{bijection-classes} знаем, че $h$ е биекция.
  \item
    Понеже $q_\varepsilon = \qstart$,
    то е ясно, че
    \[h([\varepsilon]_L) \df \qstart.\]
  \item
    Също лесно се съобразява, че
    \begin{align*}
      [\alpha]_L \in F^\M & \iff [\alpha]_L \subseteq L & \comment\text{деф. на }F^\M\\
                          & \iff \alpha \in L & \comment\text{свойство на }\approx_L\\
                          & \iff \alpha \in \L(\A) & \comment{L = \L(\A)}\\
                          & \iff q_\alpha \in F^\A & \comment\text{деф. на }\L(\A)\\
                          & \iff h([\alpha]_L) \in F^\A. & \comment\text{деф. на }h
    \end{align*}
  \item
    За последно оставихме проверката, че $h$ наистина е {\bf изоморфизъм}:
    \begin{align*}
      h(\delta_\M([\alpha]_L,b)) & = h([\alpha b]_L) & \comment\text{деф. на }\delta_\M\\
                                 & = q_{\alpha b} & \comment\text{деф. на }h\\
                                 & = \delta_\A(\delta^\star_\A(\qstart,\alpha), b) & \comment\text{деф. на }\delta^\star_\A\\
                                 & = \delta_\A( h([\alpha]_L), b). & \comment\text{деф. на }h
    \end{align*}
  \end{enumerate}
\end{proof}

\begin{framed}
  \begin{corollary}\label{cor:regular:minimal-isomorphic}
    За регулярния език $L$, нека имаме детерминираните крайни автомати $\A_1$ и $\A_2$, които разпознават $L$.
    Ако $|Q_1| = |Q_2| = |\Sigma^\star/_{\approx_L}|$, то $\A_1 \cong \A_2$.
  \end{corollary}
\end{framed}
\begin{hint}
  $\A_1 \cong \M\ \&\ \A_2 \cong \M \implies \A_1 \cong \A_2$.
\end{hint}

\begin{corollary}\index{Бжозовски}
  За регулярния език $L$, нека $\B$ е детерминираният краен автомат построен за $L$ според метода на Бжозовски.
  Тогава $\B$ е минимален автомат и $\M \cong \B$.
\end{corollary}
\begin{hint}
  Достатъчно е да се провери, че $|Q^\B| = |\Sigma^\star/_{\approx_L}|$.
  За целта, нека
  Нека да дефинираме $f:\Sigma/_{\approx_L}  \to Q^\B$ по следния начин:
  \[f([\alpha]_L) \df \alpha^{-1}(L).\] 
  Лесно се вижда, че $f$ е биекция.
  \marginpar{Оттук следва и $\M \cong_f \B$.}
  Понеже $L = \L(\B)$ и $|Q^\B| = |\Sigma^\star/_{\approx_L}|$, то според \Cor{regular:minimal-isomorphic} следва, че $\M \cong \B$.
\end{hint}


%%% Local Variables:
%%% mode: latex
%%% TeX-master: "../eai"
%%% End:
