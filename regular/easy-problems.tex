\section{Допълнителни задачи}

\subsection{Лесни задачи}

\ifcode
\begin{problem}
  \marginnote{Библиотеката {\bf regex} е част от си++11 стандарта на езика.}
  Да разгледаме следната програма на езика си++
  \begin{minted}[]{cpp}
    #include <iostream>
    #include <regex>
    #include <string>
    
    using namespace std;
    
    int main() {
      string input;
      regex reg("your regular expression");
      while (true) {
        cout << "Input:" << endl;
        cin >> input;
        if (!cin || input=="q") break;
        if (regex_match(input, reg)) {
          cout << "Valid input" << endl;
        }  
        else {
          cout << "Invalid input" << endl;
        }
      }
    }
  \end{minted}
  Разучете как се създават обекти от тип {\bf regex} и попълнете дефиницията на регулярния израз {\bf reg} в горната програма, така че програмата приема за валиден вход:
  \begin{enumerate}[a)]
  \item 
    само реални числа;
  \item
    факултетни номера във ФМИ;
  \item
    низове, които са съставени от поне 8 символа, измежду които се включват малки букви, големи букви,
    и цифри.    
  \end{enumerate}
\end{problem}
\fi

% \begin{problem}
%   Да фиксираме една дума $\alpha$ над дадена азбука $\Sigma$.
%   Опишете алгоритъм, който за вход произволен текстов файл, чието съдържание означаваме с $\beta$,
%   отговаря дали думата $\alpha$ се среща в $\beta$.
%   Каква е сложността на този алгоритъм относно дължините на $\alpha$ и $\beta$ ?
% \end{problem}

% \begin{problem}
%   Опишете алгоритъм, който при вход два регулярни израза $\mathbf{r}$ и $\mathbf{s}$,
%   проверява дали $\L(\mathbf{r}) \subseteq \L(\mathbf{s})$.
% \end{problem}

\ExtraMaterial{
\begin{problem}
  За всеки от следните езици $L$, постройте минимален краен детерминиран автомат $\A$, който разпознава езика $L$, където:
  \marginnote{$\card{\omega}{a} \df $ броят на срещанията на буквата $a$ в думата $\omega$, $\abs{\omega} \df $ дължината на $\omega$.}
  \begin{enumerate}[a)]
  \item 
    $L = \{a^nb\mid n \geq 0\}$;
  \item
    $L = \{a,b\}^\star\setminus\{\varepsilon\}$;
  \item
    $L = \{\omega \in \{a,b\}^\star \mid \card{\omega}{a}\text{ и }\card{\omega}{b}\text{ са четни}\}$;
  \item
    $L = \{\omega \in \{a,b\}^\star \mid \card{\omega}{a}\text{ е четно}\ \&\ \card{\omega}{b}\text{ е нечетно}\}$;
  \item
    $L = \{a^nb^m\mid n,m \geq 0\}$;
  \item
    $L = \{a^nb^m\mid n,m \geq 1\}$;
  \item
    $L = \{a,b\}^\star \setminus \{a\}$;
  \item
    $L = \{\omega \in \{a,b\}^\star \mid \card{\omega}{a} \geq 2\ \lor\ \card{\omega}{b} \leq 3 \}$;
  \item
    $L = \{\omega \in \{a,b\}^\star \mid \card{\omega}{a} \geq 2\ \&\ \card{\omega}{b} \geq 1\}$;
  \item
    $L = \{\omega \in \{a,b\}^\star \mid \text{на всяка нечетна позиция на }\omega\text{ е буквата }a\}$;
  \item
    $L = \{\omega \in \{a,b\}^\star \mid \card{\omega}{a}\text{ е четно }\&\ \card{\omega}{b} \leq 1\}$;
  \item
    $L = \{\omega \in \{a,b\}^\star \mid \abs{\omega} \leq 3\}$;
  \item
    $L = \{\omega \in \{a,b\}^\star \mid \omega \text{ не започва с }ab\}$;
  \item
    $L = \{\omega \in \{a,b\}^\star \mid \omega \text{ завършва с }ab\text{ или ba}\}$;
  \item
    $L = \{\omega \in \{a,b\}^\star \mid \omega\text{ започва или завършва с } a\}$;
  \item
    $L = \{\omega \in \{a,b\}^\star \mid \omega\text{ започва с }a \iff \omega\text{ завършва с }b\}$;
  \item
    $L = \{\omega \in \{a,b\}^\star \mid \abs{\omega} \equiv 0\ (\bmod\ 2)\ \&\ \card{\omega}{a} = 1\}$;
  \item
    $L = \{\omega \in \{a,b\}^\star \mid \text{ всяко }a\text{ в }\omega\text{ се следва от поне едно }b\}$;
  \item
    $L = \{\omega \in \{a,b\}^\star \mid \abs{\omega} \equiv 0 \bmod 3\}$;
  \item
    $L = \{\omega \in \{a,b\}^\star \mid \card{\omega}{a} \equiv 1 \bmod 3\}$;
  \item
    $L = \{\omega \in \{a,b\}^\star \mid \card{\omega}{a} \equiv 0 \bmod 3\ \&\ \card{\omega}{b} \equiv 1 \bmod 2\}$;
  \item
    $L = \{\omega \in \{a,b\}^\star \mid \card{\omega}{a} \equiv 0 \bmod 2\ \vee\ \card{\omega}{b} = 2\}$;
  \item
    $L = \{\omega \in \{a,b\}^\star \mid \omega \text{ съдържа равен брой срещания на }ab\text{ и на }ba\}$;
  \item
    $L = \{\omega_1 \sharp \omega_2 \sharp \omega_3 \mid |\omega_1| \geq 2\ \&\ |\omega_2| \geq 3\ \&\ |\omega_3| \geq 4\ \&\ \omega_i \in \{a,b\}^\star\text{
      за }i = 1,2,3\}$.
  \end{enumerate}
\end{problem}

\begin{problem}
  Нека $\Sigma = \{a,b\}$.  Проверете дали $L$ е регулярен, където
  \begin{multicols}{2}
  \begin{enumerate}[a)]
  % \item
  %   $L = \{\alpha^{\texttt{rev}} \mid \alpha \in L_0\}$, където $L_0$ е регулярен;
  \item
    % \marginnote{$\alpha = a^pb^p$}
    $L = \{a^ib^i\ \mid\ i\in\Nat\}$;
  \item
    $L = \{a^ib^i\ \mid\ i,j\in\Nat\ \&\ i\neq j\}$;
  \item
    % \marginnote{$\alpha = a^{p+1}b^p$.}
    $L = \{a^ib^j\ \mid\ i > j\}$;
  \item
    $L = \{a^nb^m \mid n\text{ дели }m\}$.
  \item
    $L = \{a^{2n}\ \mid\ n\geq 1\}$;
  \item
    $L = \{a^mb^na^{m+n}\ \mid\ m\geq 1\ \&\ n\geq 1\}$;
  \item
    $L = \{a^{n.m}\mid n,m\text{ са прости числа}\}$;
  \item
    $L = \{\omega\in\{a,b\}^\star \mid \card{\omega}{a} = \card{\omega}{b}\}$;
  \item
    % \marginnote{$\alpha = a^pba^pb$}
    $L = \{\omega\omega\mid \omega\in\{a,b\}^\star\}$;
  \item
    $L = \{\omega\omega^\rev\mid \omega\in\{a,b\}^\star\}$;
  \item
    $L = \{\alpha\beta\beta \in \{a,b\}^\star\mid \beta \neq \varepsilon\}$;
  \item
    $L = \{a^nb^nc^n\mid n\geq 0\}$;
  \item
    $L = \{\omega\omega\omega\mid \omega\in \Sigma^\star\}$;
  \item
    $L = \{a^{2^n}\mid n\geq 0\}$;
  \item
    $L = \{a^mb^n\mid n\neq m\}$;
  \item
    $L = \{a^{n!}b^{n!}\mid n\neq 1\}$;
  \item
    $L = \{a^{f_n} \mid f_0 = f_1 = 1\ \&\ f_{n+2} = f_{n+1} + f_{n}\}$;
  \item
    $L = \{\alpha \in \Sigma^\star \mid \abs{\ \card{\alpha}{a} - \card{\alpha}{b}\ } \leq 2\}$;
  \item
    $L = \{\alpha\beta\alpha \mid \alpha,\beta \in \Sigma^\star\ \&\ \abs{\beta} \leq \abs{\alpha}\}$;
  \item
    $L = \{\beta\gamma\gamma^\rev\mid \beta, \gamma \in \Sigma^\star\ \&\ \abs{\beta} \leq \abs{\gamma}\}$;
  \item
    $L = \{c^ka^nb^m \mid k,m,n > 0\ \&\ n \neq m\}$;
  \item
    $L = \{c^ka^nb^n \mid k > 0\ \&\ n \geq 0\}\cup\{a,b\}^\star$;
  \item
    $L = \{\omega \in \{a,b\}^\star \mid \card{\omega}{a}\text{ не дели }\card{\omega}{b}\}$;
  \item
    $L = \{\omega \in \{a,b\}^\star \mid \card{\omega}{a} < \card{\omega}{b}\}$;
  \item
    $L = \{\omega \in \{a,b\}^\star \mid \card{\omega}{a} = 2\card{\omega}{b}\}$;
  \item
    $L = \{\omega \in \{a,b\}^\star \mid \abs{\ \card{\omega}{a} - \card{\omega}{b}\ } \leq 3\}$.
  \end{enumerate}    
  \end{multicols}
\end{problem}
}

\begin{problem}
  Докажете, че следните езици са регулярни:
  \begin{enumerate}[a)]
  \item
    $L = \{\alpha \in \{a,b\}^\star \mid \abs{\ \card{\omega}{a} - \card{\omega}{b}\ } \leq 2 \text{ за всяка представка $\omega$ на $\alpha$}\}$;
  \item
    $L = \{\alpha \in \{a,b\}^\star \mid \abs{\ \card{\omega}{a} - \card{\omega}{b}\ } > 2 \text{ за някоя представка $\omega$ на $\alpha$}\}$;
  \item
    $L = \{\alpha \in \{a,b\}^\star \mid \abs{\ \card{\omega}{a} - \card{\omega}{b}\ } > 2 \text{ за някоя наставка $\omega$ на $\alpha$}\}$.
  \end{enumerate}
\end{problem}



% \begin{problem}
%   Нека $L$ е регулярен език над азбуката $\Sigma$. Докажете, че 
%   \[\text{Infix}(L) = \{\alpha \mid (\exists \beta,\gamma \in \Sigma^\star)[\beta\alpha\gamma \in L]\}\]
%   също е регулярен език.
% \end{problem}
% \begin{hint}
%   Най-лесно е да се построи автомат за $\text{Infix}(L)$ като се използва автомата за $L$.
% \end{hint}

\begin{problem}
  Нека $\Sigma = \{a,b,c,d\}$.
  Да се докаже, че езикът
  \[L = \{a_1a_2\cdots a_{2n} \in \Sigma^\star \mid (\forall j \in [1,n])[a_{2j-1} = a_{2j}]\ \&\ d\text{ се среща $\leq 3$ пъти}\}\]
  е регулярен.
\end{problem}

\begin{problem}
  Нека $L_1$ и $L_2$ са регулярни езици. Докажете, че $L$ също е регулярен език, където
  \[L = \{\alpha \mid (\exists \beta,\gamma)[\beta\alpha\gamma \in L_1]\ \&\ \alpha \in L_2 \vee \alpha^{\texttt{rev}} \in L_2\}.\]
\end{problem}

\begin{definition}
  \index{хомоморфизъм}
  Да фиксираме две азбуки $\Sigma_1$ и $\Sigma_2$.
  Хомоморфизъм е изображение $h:\Sigma^\star_1 \to \Sigma^\star_2$ със свойството, че
  за всеки две думи $\alpha,\beta\in\Sigma^\star_1$,
  \[h(\alpha\beta) = h(\alpha)\cdot h(\beta).\]
\end{definition}

Лесно се съобразява, че за всеки хомоморфизъм $h$, $h(\varepsilon) = \varepsilon$.

\begin{problem}
  Нека $L \subseteq \Sigma^\star_1$ е регулярен език и $h:\Sigma^\star_1\to\Sigma^\star_2$ е хомоморфизъм.
  Тогава
  $h(L) = \{h(\alpha) \in \Sigma^\star_2 \mid \alpha \in L\}$ е регулярен.
\end{problem}
\begin{hint}
  Индукция по построението на регулярни езици.
  % \begin{itemize}[-]
  % \item 
  %   За $L = \{a\}$, $h(L) = \{h(a)\}$.
  % \item
  %   $h(\emptyset) = \emptyset$.
  % \item
  %   Нека $L_1 = \L(r_1)$ и $L_2 = \L(r_2)$.
  %   Ще докажем, че $h(\L(r_1\cdot r_2))$ е регулярен.
  %   \begin{align*}
  %     h(\L(r_1\cdot r_2)) & = h(L_1\cdot L_2) & (\text{деф. на конкатенация})\\
  %     & = \{h(\gamma) \mid \gamma \in L_1 \cdot L_2\}\\
  %     & = \{h(\alpha\beta) \mid \alpha\in L_1\ \&\ \beta\in L_2\}\\
  %     & = \{h(\alpha)\cdot h(\beta) \mid \alpha \in L_1\ \&\ \beta \in L_2\} & (h\text{ е хомоморфизъм})\\
  %     & = \{\omega\gamma \mid \omega \in h(L_1)\ \&\ \gamma \in h(L_2)\}\\
  %     & = h(L_1)\cdot h(L_2).
  %   \end{align*}
  %   По И.П. имаме, че $h(L_1)$ и $h(L_2)$ са регулярни езици.
  %   Следователно, 
  %   \[h(\L(r_1\cdot r_2)) = h(L_1)\cdot h(L_2)\]
  %   е регулярен език.
  % \item
  %   От горното свойство имаме също, че за всяко $n$, $h(L^n) = h(L)^n$.
  % \item
  %   Освен това, 
  %   \begin{align*}
  %     h(\bigcup_n L_n) & = \{h(\alpha) \mid (\exists n)[\alpha \in L_n]\}\\
  %     & = \bigcup \{h(\alpha) \mid \alpha \in L_n\}\\
  %     & = \bigcup_n h(L_n).
  %   \end{align*}
  % \item
  %   Нека $L = \L(r^\star)$.
  %   Ще докажем, че $h(L^\star)$ е регулярен език.
  %   \begin{align*}
  %     h(L^\star) & = h(\bigcup_n L^n) & (\text{деф. на звезда на Клини})\\
  %     & = \bigcup_n h(L^n) & (\text{от горното свойство})\\
  %     & = \bigcup_n h(L)^n & (\text{от по-горното свойство})\\
  %     & = h(L)^\star & (\text{по деф.}).
  %   \end{align*}
  % \end{itemize}
\end{hint}

\begin{problem}
  Нека $L\subseteq \Sigma^\star_2$ е регулярен език и $h:\Sigma^\star_1\to\Sigma^\star_2$ е хомоморфизъм.
  Тогава езикът
  $h^{-1}(L) = \{\alpha \in \Sigma^\star_1 \mid h(\alpha) \in L\}$ е регулярен.  
\end{problem}
\begin{hint}
  Конструкция на автомат за $h^{-1}(L)$ при даден автомат за $L$.
  % Нека $\A$ е КДА разпознаващ езика $L$.
  % Ще построим $\A' = \pair{Q,\Sigma_1, \delta', s, F}$,
  % където дефинираме функцията на преходите $\delta'$ като $\delta'(q,a) = \delta^\star(q,h(a))$.
  % Понеже $h$ е хомоморфизъм, лесно се доказва с индукция по дължината на думата $\alpha \in \Sigma^\star_1$,
  % че $\delta'^\star(q,\alpha) = \delta^\star(q,h(\alpha))$.
  % Сега лесно се вижда, че $h^{-1}(\L(\A)) = \L(\A')$, защото:
  % \begin{align*}
  %   \alpha \in \L(\A') & \iff \delta'^\star(s,\alpha) \in F\\
  %   & \iff \delta^\star(s,h(\alpha)) \in F\\
  %   & \iff h(\alpha) \in \L(\A)\\
  %   & \iff \alpha \in h^{-1}(\L(\A)).
  % \end{align*}
\end{hint}

\begin{problem}
  Нека $\Sigma_1$ и $\Sigma_2$ са непресичащи се азбуки, а $L_1$ и $L_2$ са езици съответно над $\Sigma_1$ и $\Sigma_2$.
  За една дума $\omega \in (\Sigma_1 \cup \Sigma_2)^\star$, нека с $\omega_i \in \Sigma^\star_i$ да означим редицата от букви от $\Sigma_i$
  в реда, в който се срещат в $\omega$. Да разгледаме следния език
  \[L_1 \oplus L_2 = \{\omega \in (\Sigma_1 \cup \Sigma_2)^\star \mid \omega_1 \in L_1\ \&\ \omega_2 \in L_2\ \&\ |\omega_1| = |\omega_2|\}.\]
  \begin{enumerate}[a)]
  \item
    \marginnote{Да}
    Вярно ли е, че ако $L_1$ е краен, то $L_1 \oplus L_2$ е регулярен език?
  \item
    \marginnote{Не}
    Вярно ли е, че ако $L_1$ и $L_2$ са регулярни езици, то $L_1 \oplus L_2$ е регулярен език?
  \end{enumerate}
\end{problem}

\begin{problem}
  Нека $\Sigma_1$ и $\Sigma_2$ са непресичащи се азбуки, а $L_1$ и $L_2$ са езици съответно над $\Sigma_1$ и $\Sigma_2$.
  За една дума $\omega \in (\Sigma_1 \cup \Sigma_2)^\star$, нека с $\omega_i \in \Sigma^\star_i$ да означим редицата от букви от $\Sigma_i$
  в реда, в който се срещат в $\omega$. Да разгледаме следния език
  \[L_1 \oplus L_2 = \{\omega \in (\Sigma_1 \cup \Sigma_2)^\star \mid \omega_1 \in L_1\ \&\ \omega_2 \in L_2\}.\]
  \marginnote{Да}
  Вярно ли е, че ако $L_1$ и $L_2$ са регулярни езици, то $L_1 \oplus L_2$ е регулярен език?
\end{problem}

%%% Local Variables:
%%% mode: latex
%%% TeX-master: "../eai"
%%% End:
