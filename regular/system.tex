\section{Системи от регулярни изрази}

\mynote{\cite[стр. 100]{sakarovitch-book}.}
\begin{lemma}[Ардън]
  \label{lem:regular:system:arden}
  Нека $\mathbf{r}$  и $\mathbf{p}$ са регулярни изрази.
  Тогава регулярният израз $\mathbf{r}^\star \mathbf{p}$ е най-малкото решение на уравнението $X = \mathbf{r} \cdot X + \mathbf{p}$.
  Ако $\varepsilon  \not \in \L(\mathbf{r})$, то това решение е единствено.
\end{lemma}

Да разгледаме отново автомата от \Figure{a2}.

\begin{figure}[H]
  \begin{center}
    \begin{tikzpicture}[framed,->,>=stealth,thick,node distance=45pt,initial text=начало]
      \tikzstyle{every state}=[circle,minimum size=15pt,auto]
      
      \node[initial,state]      (1) {$q_0$};
      \node[state]              (2) [right of=1]{$q_1$};
      \node[accepting, state]   (3) [right of=2]{$q_2$};
      
      \path 
      (2) edge [loop above]    node [above] {$a$} (2)
      (1) edge [bend left=15]  node [above] {$a$} (2)
      (2) edge [bend left=15]  node [above] {$b$} (3)
      (1) edge [bend right=45] node [below] {$b$} (3)
      (3) edge [loop above]    node [above] {$a,b$} (3);
    \end{tikzpicture}
  \end{center}
  \caption{$\L(\A) \stackrel{?}{=} \L(\mathbf{a^\star b(a+b)^\star})$.}
\end{figure}

На него съответства следната система:
\begin{align*}
  & X_0 = a \cdot X_1 + b \cdot X_2 + \emptyset\\
  & X_1 = a \cdot X_1 + b \cdot X_2 + \emptyset\\
  & X_2 = a \cdot X_2 + b \cdot X_2 + \varepsilon.
\end{align*}

Чрез прилагане на \hyperref[lem:regular:system:arden]{лемата на Ардън} и заместване получаваме следното:

\begin{align*}
  & X_0 = a \cdot X_1 + b \cdot (a+b)^\star\\
  & X_1 = a \cdot X_1 + b \cdot (a+b)^\star\\
  & X_2 = (a+b)^\star.
\end{align*}

След това,

\begin{align*}
  & X_0 = a \cdot a^\star \cdot b \cdot (a+b)^\star + b \cdot (a+b)^\star\\
  & X_1 = a^\star \cdot b \cdot (a+b)^\star\\
  & X_2 = (a+b)^\star.
\end{align*}

Заключаваме, че езикът на автомата е
\[(a \cdot a^\star + \varepsilon) \cdot b \cdot (a+b)^\star = a^\star \cdot b \cdot (a+b)^\star.\]


%%% Local Variables:
%%% mode: latex
%%% TeX-master: "../eai"
%%% End:
