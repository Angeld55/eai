\section{Регулярни изрази и езици}

Да фиксираме една непразна азбука $\Sigma$.
\index{регулярен израз}
{\bf Регулярните изрази} $\mathbf{r}$ могат да се опишат със следната абстрактна граматика
\[\mathbf{r} ::= \mathbf{\emptyset}\ |\ \mathbf{\varepsilon}\ |\ \mathbf{a}\ |\ \mathbf{ r_1 \cdot r_2}\ |\ \mathbf{r_1 + r_2}\ |\ \mathbf{r^\star_1},\]
където $\varepsilon$ означава празната дума и $a$ е произволна буква от азбуката $\Sigma$.

Друг начин да се опишат регулярните изрази е по следния начин:
\marginpar{Това е пример за индуктивна дефиниция}
\begin{itemize}
\item 
  Символите $\emptyset$, $\varepsilon$ и $\mathbf{a}$, всяка буква $a \in \Sigma$, са регулярен изрази;
\item
  Ако $\mathbf{r_1}$ и $\mathbf{r_2}$ са регулярни изрази, то $\mathbf{r_1 \cdot r_2}$, $\mathbf{r_1 + r_2}$ и $\mathbf{r^\star_1}$
  също са регулярни изрази;
\item
  Всеки регулярен израз е получен по някое от горните правила.
\end{itemize}

\index{език!регулярен}
\marginpar{Това е друг пример за индуктивна (рекурсивна) дефиниция.}
Сега ще дефинираме езиците, които се описват с регулярни изрази.
Тези езици се наричат {\bf регулярни}.
Това ще направим следвайки индуктивната дефиниция на регулярните изрази,
т.е. за всеки регулярен израз $\mathbf{r}$ ще определим език $\L(\mathbf{r})$.
\begin{itemize}
\item
  $\{\varepsilon\}$ е регулярен език,
  който се разпознава от регулярния израз $\varepsilon$.
  Означаваме $\L(\varepsilon) = \{\varepsilon\}$;
\item
  за всяка буква $a \in \Sigma$, $\{a\}$ е регулярен език,
  който се разпознава от регулярния израз $\mathbf{a}$.
  Означаваме $\L(\mathbf{a}) = \{a\}$;
\item
  $\emptyset$ е регулярен език,
  който се разпознава от регулярния израз $\emptyset$. Означаваме $\L(\emptyset) = \emptyset$;
\item
  Нека $L_1$ и $L_2$ са регулярни езици, т.е. съществуват регулярни изрази $\mathbf{r_1}$
  и $\mathbf{r_2}$, за които $\L(\mathbf{r}_1) = L_1$ и $\L(\mathbf{r_2}) = L_2$.
  Тогава:
  \begin{itemize}
  \item 
    \index{обединение}
    $L_1 \cup L_2$ е регулярен език, който се описва с регулярния израз $\mathbf{r_1 + r_2}$.
    Това означава, че $\L(\mathbf{r_1}) \cup \L(\mathbf{r_2}) = \L(\mathbf{r_1+r_2})$.
  \item
    \index{конкатенация}
    \marginpar{Тази операция се наричка конкатенация. Обикновено изпускаме знака $\cdot$}
    $L_1 \cdot L_2$ е регулярен език, който се описва с регулярния израз $\mathbf{r_1 \cdot r_2}$.
    Това означава, че $\L(\mathbf{r_1}) \cdot \L(\mathbf{r_2}) = \L(\mathbf{r_1 \cdot r_2})$.
  \item
    \marginpar{Звезда на Клини}
    \index{звезда на Клини}
    $L^\star_1$ е регуларен език, който се описва с регулярния израз $\mathbf{r^\star_1}$.
    Това означава, че $L^\star_1 = \L(\mathbf{r^\star_1})$.
  \end{itemize}
\end{itemize}

\begin{remark}
  Ние знаем, че:
  \begin{itemize}
  \item
    Всеки регулярен израз представлява крайна дума над крайна азбука.
    Това означава, че множеството от всички регулярни изрази е изброимо безкрайно.
    Оттук следва, че всички регулярни езици образуват изброимо безкрайно множество.
  \item 
    Понеже $\Sigma$ е крайна азбука, то $\Sigma^\star$ е изброимо безкрайно множество;
  \item
    Един език над азбуката $\Sigma$ представлява елемент на $\Ps(\Sigma^\star)$.
    Това означава, че всички езици над азбуката $\Sigma$ представляват неизброимо безкрайно множество.
  \end{itemize}
  От всичко това следва, че има езици, които не са регулярни.
  По-нататък ще видим примери за такива езици.
\end{remark}

\begin{example}
  Нека да разгледаме няколко примера какво точно представлява прилагането
  на операцията звезда на Клини върху един език.
  \begin{itemize}
  \item 
    Нека $L = \{0,11\}$. Тогава:
    \begin{itemize}
    \item 
      $L^0 = \{\varepsilon\}$, $L^1 = L$,
    \item
      $L^2 = L^1\cdot L^1 = \{00,011,110,1111\}$,
    \item
      $L^3 = L^1\cdot L^2 = \{000,0011,0110,01111,1100,11011,11110,111111\}$.
    \end{itemize}
  \item
    Нека $L = \emptyset$.
    Тогава:
    \begin{itemize}
    \item 
      $L^0 = \{\varepsilon\}$,
    \item
      $L^1 = \emptyset$,
    \item
      $L^2 = L^1 \cdot L^1 = \emptyset$.
    \end{itemize}    
    Получаваме, че $L^\star = \{\varepsilon\}$, т.е. {\em краен} език
  \item
    Нека $L = \{0^i\mid i \in \Nat\} = \{\varepsilon, 0, 00, 000, \dots\}$.
    Тогава лесно може да се види, че $L = L^\star$.
  \end{itemize}
\end{example}

\begin{example}
  \marginpar{В \cite[стр. 73]{sipser1} е показан алгоритъм, за който по един автомат може да се получи регулярен израз описващ езика на автомата. Ние няма да разглеждаме този алгоритъм. }
  Нека да построим регулярни изрази за всеки от езиците от \Ex{automata-pictures}.
  \begin{enumerate}[a)]
  \item 
    Нека $\mathbf{r} = \mathbf{(a+b)^\star bab(a+b)^\star}$. Тогава
    \[\L(\mathbf{r}) = \{\omega \in \{a,b\}^\star \mid \omega \text{ съдържа } bab\}.\]
  \item
    Нека $\mathbf{r} = \mathbf{b^\star ab^\star a(a+b)^\star}$. Тогава
    \[\L(\mathbf{r}) = \{\omega \in \{a,b\}^\star \mid N_a(\omega) \geq 2\}.\]
  \item
    Нека $\mathbf{r} = \mathbf{(abb^\star)^\star}$. Тогава
    \[\L(\mathbf{r}) = \{\omega \in \{a,b\}^\star \mid \text{ всяко $a$ в $\omega$ се следва от поне едно $b$}\}.\]
  \item
    Нека $\mathbf{r} = \mathbf{(b^\star ab^\star ab^\star ab^\star)^\star}$. Тогава
    \[\L(\mathbf{r}) = \{\omega \in \{a,b\}^\star \mid N_a(\omega) \equiv 0 \bmod 3\}.\]
  \end{enumerate}
\end{example}


\begin{problem}
  За произволни регулярни изрази $r$ и $s$, 
  проверете:
  \begin{enumerate}[a)]
  \item 
    $r+s = s + r$;
  \item
    $(\varepsilon + r)^\star = r^\star$;
  \item
    $\emptyset^\star = \varepsilon$;
  \item
    $(r^\star s^\star) = (r+s)^\star$;
  \item
    $(r^\star)^\star = r^\star$;
  \item
    $(rs + r)^\star r = r(sr+r)^\star$;
  \item
    $s(rs+s)^\star r = rr^\star s(rr^\star s)^\star$;
  \item
    $(r+s)^\star = r^\star + s^\star$;
  \item
    $\emptyset^\star = \varepsilon^\star$;
  \end{enumerate}
\end{problem}

\begin{framed}
\begin{thm}[Клини]
  \label{th:regular-kleene}
  \index{Клини}
  Всеки автоматен език се описва с регулярен израз.
\end{thm}
\end{framed}
\begin{proof}
  \marginpar{\cite[стр. 79]{papadimitriou}; \cite[стр. 33]{hopcroft1}}
  Нека  $L = \L(\A)$, за някой краен детерминиран автомат $\A$.
  Да фиксираме едно изброяване на състоянията $Q = \{q_1,\dots,q_n\}$,
  като началното състояние е $q_1$.
  Ще означаваме с $L(i,j,k)$ множеството от тези думи, които
  могат да се разпознаят от автомата по път, който започва от $q_i$,
  завършва в $q_j$, и междинните състояния имат индекси $\leq k$.
  Например, за думата $\alpha = a_1a_2\cdots a_n$ имаме, че $\alpha \in L(i,j,k)$
  точно тогава, когато съществуват състояния $q_{l_1},\dots,q_{l_{n-1}}$, като $l_1,\dots,l_{n-1} \leq k$ и
  \[q_i\stackrel{a_1}{\rightarrow} q_{l_1} \stackrel{a_2}{\rightarrow} q_{l_2} \stackrel{a_3}{\rightarrow} \dots \stackrel{a_{n-1}}{\rightarrow} q_{l_{n-1}}\stackrel{a_n}{\rightarrow} q_j.\]
  Тогава за $n = \abs{Q}$, 
  \[L(i,j,n) = \{\alpha\in\Sigma^\star\mid \delta^\star(q_i,\alpha) = q_j\}.\]
  Така получаваме, че 
  \[\L(\A) = \bigcup\{L(1,j,n)\mid q_j \in F\} = \bigcup_{q_j\in F}L(1,j,n).\]
  Ще докажем с {\em индукция по $k$}, че за всяко $i,j,k$, множествата от думи $L(i,j,k)$
  се описват с регулярен израз $\mathbf{r^k_{i,j}}$
  \begin{enumerate}[a)]
  \item
    Нека $k = 0$. Ще докажем, че за всяко $i,j$, $L(i,j,0)$ се описва с регулярен израз.
    Имаме да разгледаме два случая.
    
    Ако $i = j$, то 
    \begin{equation}
      \label{eq:kleene-equal}
      L(i, j, 0) = \{\varepsilon\}\cup\{a\in\Sigma \mid \delta(q_i,a) = q_j\}.
    \end{equation}
    Ако $i \neq j$, то
    \[L(i, j, 0) = \{a\in\Sigma \mid \delta(q_i, a) = q_j\}.\]
    И в двата случая, понеже $L(i,j,0)$ е краен език, то е ясно, че той се описва с регулярен израз.
  \item
    Да предположим, че $k > 0$ и за всяко $i$, $j$, можем да намерим регулярните изрази
    съответстващи на $L(i,j,k-1)$. Тогава
    \[L(i,j,k) = L(i,j,k-1)\ \cup\ L(i,k,k-1)\cdot (L(k,k,k-1)^\star) \cdot L(k,j,k-1).\]
    Тогава по {\bf И.П.} следва, че $L(i,j,k)$ може да се опише с регулярен израз, който е
    \begin{equation}
      \label{eq:kleene}
      \mathbf{r^k_{i,j}} = \mathbf{r^{k-1}_{i,j} + r^{k-1}_{i,k}\cdot (r^{k-1}_{k,k})^\star\cdot r^{k-1}_{k,j}}.
    \end{equation}
  \end{enumerate}
  Заключаваме, че за всяко $i,j,k$, $L(i,j,k)$ може да се опише с регулярен израз $\mathbf{r^{k}_{i,j}}$.
  Тогава ако $F = \{q_{i_1},\dots,q_{i_k}\}$, то $\L(\A)$ се описва с регулярния израз
  \[\mathbf{r^n_{1,i_1} + r^n_{1,i_2} + \dots + r^n_{1,i_k}}.\]
\end{proof}

\begin{example}
  Да разгледаме следния автомат:

  \begin{framed}
  \begin{figure}[H]
    \begin{center}
      \begin{tikzpicture}[->,>=stealth,thick,node distance=45pt]
        \tikzstyle{every state}=[circle,minimum size=15pt,auto]
        
        \node[initial,state]      (1) {$q_1$};
        \node[accepting, state]   (2) [right of=1]{$q_2$};
        
        \path 
        (1) edge [loop above]  node [above] {$1$} (1)
        (1) edge  node [above] {$0$} (2)
        (2) edge [loop above] node [above] {$0,1$} (2);
      \end{tikzpicture}
      \end{center}
      \label{fig:a1}
      \caption{Автомат разпознаващ $\L(\mathbf{1^\star 0 (0 + 1)^\star)}$}
 \end{figure}
\end{framed}

Лесно се съобразява, че езикът на автомата от Фигура \ref{fig:a1} се описва с регулярния израз $\mathbf{1^\star 0 (0 + 1)^\star}$.
 % За да намерим регулярния език за автомата от Пример \ref{fig:a1}, 
 Следвайки конструкцията от доказателството на \Th{regular-kleene},
 езикът на този автомат се описва с регулярния израз $\mathbf{r^2_{1,2}}$, защото началното състояние е $q_1$, финалното е $q_2$ и 
 броят на състоянията в автомата е $2$.
 \begin{align*}
   \mathbf{r^2_{1,2}} & = \underbrace{\mathbf{r^{1}_{1,2}}}_{\mathbf{1^\star 0}} + \underbrace{\mathbf{r^{1}_{1,2}}}_{\mathbf{1^\star 0}}\cdot \underbrace{\mathbf{(r^1_{2,2})^\star}}_{\mathbf{(\varepsilon+0+1)^\star}} \cdot \underbrace{\mathbf{r^1_{2,2}}}_{\mathbf{\varepsilon+0+1}} & (\text{според (\ref{eq:kleene}})) \\
   &  = \mathbf{1^\star0 + 1^\star 0 (\varepsilon + 0 + 1)^\star (\varepsilon + 0 + 1)}\\
   & =  \mathbf{1^\star0 + 1^\star 0 (\varepsilon + 0 + 1)^+} & (\mathbf{r^+ = r^\star r})\\
   & =  \mathbf{1^\star0 + 1^\star 0 (0 + 1)^\star} & (\mathbf{r^\star = (\varepsilon + r)^+})\\
   & = \mathbf{1^\star 0 (\varepsilon + (0 + 1)^\star)} & (\mathbf{r + rq = r(\varepsilon + q)})\\
   & = \mathbf{1^\star 0 (0 + 1)^\star} & (\mathbf{r^\star = \varepsilon + r^\star})
  \end{align*}
  
  Тук използвахме, че:
  \begin{align*}
    \mathbf{r^1_{1,2}} & = \underbrace{\mathbf{r^0_{1,2}}}_{\mathbf{0}} + \underbrace{\mathbf{r^0_{1,1}}}_{\mathbf{\varepsilon + 1}}\cdot\underbrace{\mathbf{(r^0_{1,1})^\star}}_{\mathbf{(\varepsilon+1)^\star}} \cdot \underbrace{\mathbf{r^0_{1,2}}}_{\mathbf{0}}\\
    & = \mathbf{0 + (\varepsilon + 1)(\varepsilon + 1)^\star0} \\
    & = \mathbf{0 + 1^\star 0}\\
    & = \mathbf{1^\star0},\\
    \mathbf{r^1_{2,2}} & = \underbrace{\mathbf{r^0_{2,2}}}_{\mathbf{\varepsilon+0+1}} + \underbrace{\mathbf{r^0_{2,1}}}_{\mathbf{\emptyset}} \cdot \underbrace{\mathbf{(r^0_{1,1})^\star}}_{\mathbf{\varepsilon+1}}\cdot \underbrace{\mathbf{r^0_{1,2}}}_{\mathbf{0}}\\
    & = \mathbf{\varepsilon + 0 + 1 + \emptyset(\varepsilon + 1)^\star0}\\
    & = \varepsilon + 0 + 1 & (\text{защото }\mathbf{\emptyset \cdot r = \emptyset})
  \end{align*}
\end{example}

Следващата ни цел е да видим, че имаме и обратната посока на горната лема.
Ще докажем, че всеки регулярен език е автоматен. За тази цел първо ще 
въведем едно обобщение на понятието краен детерминиран автомат.


%%% Local Variables: 
%%% mode: latex
%%% TeX-master: "../eai"
%%% End: 
