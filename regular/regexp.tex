\section{Регулярни изрази и езици}

Да фиксираме една непразна азбука $\Sigma$.
\index{регулярен израз}
{\bf Регулярните изрази} $\mathbf{r}$ могат да се опишат със следната абстрактна граматика
\[\mathbf{r} ::= \mathbf{\emptyset}\ |\ \mathbf{\varepsilon}\ |\ \mathbf{a}\ |\ \mathbf{ r_1 \cdot r_2}\ |\ \mathbf{r_1 + r_2}\ |\ \mathbf{r^\star_1},\]
където $\varepsilon$ означава празната дума и $a$ е произволна буква от азбуката $\Sigma$.

Друг начин да се опишат регулярните изрази е по следния начин:
\marginpar{Това е пример за индуктивна дефиниция}
\begin{itemize}
\item 
  Символите $\emptyset$, $\varepsilon$ и $\mathbf{a}$, всяка буква $a \in \Sigma$, са регулярен изрази;
\item
  Ако $\mathbf{r_1}$ и $\mathbf{r_2}$ са регулярни изрази, то $\mathbf{r_1 \cdot r_2}$, $\mathbf{r_1 + r_2}$ и $\mathbf{r^\star_1}$
  също са регулярни изрази;
\item
  Всеки регулярен израз е получен по някое от горните правила.
\end{itemize}

\index{език!регулярен}
\marginpar{Това е друг пример за индуктивна (рекурсивна) дефиниция.}
Сега ще дефинираме езиците, които се описват с регулярни изрази.
Тези езици се наричат {\bf регулярни}.
Това ще направим следвайки индуктивната дефиниция на регулярните изрази,
т.е. за всеки регулярен израз $\mathbf{r}$ ще определим език $\L(\mathbf{r})$.
\begin{itemize}
\item
  $\{\varepsilon\}$ е регулярен език,
  който се разпознава от регулярния израз $\varepsilon$.
  Означаваме $\L(\varepsilon) = \{\varepsilon\}$;
\item
  за всяка буква $a \in \Sigma$, $\{a\}$ е регулярен език,
  който се разпознава от регулярния израз $\mathbf{a}$.
  Означаваме $\L(\mathbf{a}) = \{a\}$;
\item
  $\emptyset$ е регулярен език,
  който се разпознава от регулярния израз $\emptyset$. Означаваме $\L(\emptyset) = \emptyset$;
\item
  Нека $L_1$ и $L_2$ са регулярни езици, т.е. съществуват регулярни изрази $\mathbf{r_1}$
  и $\mathbf{r_2}$, за които $\L(\mathbf{r}_1) = L_1$ и $\L(\mathbf{r_2}) = L_2$.
  Тогава:
  \begin{itemize}
  \item 
    \index{регулярни операции!обединение}
    $L_1 \cup L_2$ е регулярен език, който се описва с регулярния израз $\mathbf{r_1 + r_2}$.
    Това означава, че $\L(\mathbf{r_1}) \cup \L(\mathbf{r_2}) = \L(\mathbf{r_1+r_2})$.
  \item
    \index{регулярни операции!конкатенация}
    \marginpar{Тази операция се наричка конкатенация. Обикновено изпускаме знака $\cdot$}
    $L_1 \cdot L_2$ е регулярен език, който се описва с регулярния израз $\mathbf{r_1 \cdot r_2}$.
    Това означава, че $\L(\mathbf{r_1}) \cdot \L(\mathbf{r_2}) = \L(\mathbf{r_1 \cdot r_2})$.
  \item
    \marginpar{Звезда на Клини}
    \index{регулярни операции!звезда на Клини}
    $L^\star_1$ е регуларен език, който се описва с регулярния израз $\mathbf{r^\star_1}$.
    Това означава, че $L^\star_1 = \L(\mathbf{r^\star_1})$.
  \end{itemize}
\end{itemize}

\begin{remark}
  Ние знаем, че:
  \begin{itemize}
  \item
    Всеки регулярен израз представлява крайна дума над крайна азбука.
    Това означава, че множеството от всички регулярни изрази е изброимо безкрайно.
    Оттук следва, че всички регулярни езици образуват изброимо безкрайно множество.
  \item 
    Понеже $\Sigma$ е крайна азбука, то $\Sigma^\star$ е изброимо безкрайно множество;
  \item
    Един език над азбуката $\Sigma$ представлява елемент на $\Ps(\Sigma^\star)$.
    Това означава, че всички езици над азбуката $\Sigma$ представляват неизброимо безкрайно множество.
  \end{itemize}
  От всичко това следва, че има езици, които не са регулярни.
  По-нататък ще видим примери за такива езици.
\end{remark}

\begin{example}
  Нека да разгледаме няколко примера какво точно представлява прилагането
  на операцията звезда на Клини върху един език.
  \begin{itemize}
  \item 
    Нека $L = \{0,11\}$. Тогава:
    \begin{itemize}
    \item 
      $L^0 = \{\varepsilon\}$, $L^1 = L$,
    \item
      $L^2 = L^1\cdot L^1 = \{00,011,110,1111\}$,
    \item
      $L^3 = L^1\cdot L^2 = \{000,0011,0110,01111,1100,11011,11110,111111\}$.
    \end{itemize}
  \item
    Нека $L = \emptyset$.
    Тогава:
    \begin{itemize}
    \item 
      $L^0 = \{\varepsilon\}$,
    \item
      $L^1 = \emptyset$,
    \item
      $L^2 = L^1 \cdot L^1 = \emptyset$.
    \end{itemize}    
    Получаваме, че $L^\star = \{\varepsilon\}$, т.е. {\em краен} език
  \item
    Нека $L = \{0^i\mid i \in \Nat\} = \{\varepsilon, 0, 00, 000, \dots\}$.
    Тогава лесно може да се види, че $L = L^\star$.
  \end{itemize}
\end{example}

\begin{example}
  \marginpar{В \cite[стр. 73]{sipser1} е показан алгоритъм, за който по един автомат може да се получи регулярен израз описващ езика на автомата. Ние няма да разглеждаме този алгоритъм. }
  Нека да построим регулярни изрази за всеки от езиците от \Ex{automata-pictures}.
  \begin{enumerate}[a)]
  \item 
    Нека $\mathbf{r} = \mathbf{(a+b)^\star bab(a+b)^\star}$. Тогава
    \[\L(\mathbf{r}) = \{\omega \in \{a,b\}^\star \mid \omega \text{ съдържа } bab\}.\]
  \item
    Нека $\mathbf{r} = \mathbf{b^\star ab^\star a(a+b)^\star}$. Тогава
    \[\L(\mathbf{r}) = \{\omega \in \{a,b\}^\star \mid N_a(\omega) \geq 2\}.\]
  \item
    Нека $\mathbf{r} = \mathbf{(b^\star abb^\star)^\star}$. Тогава
    \[\L(\mathbf{r}) = \{\omega \in \{a,b\}^\star \mid \text{ всяко $a$ в $\omega$ се следва от поне едно $b$}\}.\]
  \item
    Нека $\mathbf{r} = \mathbf{(b^\star ab^\star ab^\star ab^\star)^\star}$. Тогава
    \[\L(\mathbf{r}) = \{\omega \in \{a,b\}^\star \mid N_a(\omega) \equiv 0 \bmod 3\}.\]
  \end{enumerate}
\end{example}


\begin{problem}
  За произволни регулярни изрази $r$ и $s$, 
  проверете:
  \begin{enumerate}[a)]
  \item 
    $r+s = s + r$;
  \item
    $(\varepsilon + r)^\star = r^\star$;
  \item
    $\emptyset^\star = \varepsilon$;
  \item
    $(r^\star s^\star) = (r+s)^\star$;
  \item
    $(r^\star)^\star = r^\star$;
  \item
    $(rs + r)^\star r = r(sr+r)^\star$;
  \item
    $s(rs+s)^\star r = rr^\star s(rr^\star s)^\star$;
  \item
    $(r+s)^\star = r^\star + s^\star$;
  \item
    $\emptyset^\star = \varepsilon^\star$;
  \end{enumerate}
\end{problem}



%%% Local Variables: 
%%% mode: latex
%%% TeX-master: "../eai"
%%% End: 
