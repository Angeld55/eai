\section{Автомат на Бжозовски}\label{sect:regular:brzozowski}
\index{Бжозовски}

Имаме следната операция за произволна буква $a$,
\[a^{-1}(L) \df \{\omega \in \Sigma^\star \mid a\cdot\omega \in L\}.\]

\mynote{Бжозовски \cite{brzozowski-derivatives} описва алгоритъм за строене на автомат по регулярен израз \cite[стр. 112]{sakarovitch-book}.}

Да видим как се държи тази операция върху регулярните езици.
За целта следваме дефиницията на регулярните езици.
\begin{problem}
  Докажете, че за произволна буква $a$ и език $L$ са изпълнени равенствата:
  \begin{enumerate}[(1)]
  \item
    $a^{-1}(\emptyset) = \emptyset$;
  \item
    $a^{-1}(\{\varepsilon\}) = \emptyset$;
  \item
    $a^{-1}(\{b\}) =
    \begin{cases}
      \{\varepsilon\}, & \text{ ако }a = b\\
      \emptyset, & \text{ ако }a \neq b
    \end{cases}$
  \item
    $a^{-1}(L_1 \cup L_2) = a^{-1}(L_1) \cup a^{-1}(L_2)$;
  \item
    $a^{-1}(L_1 \cdot L_2) =
    \begin{cases}
      a^{-1}(L_1) \cdot L_2, & \text{ ако }\varepsilon\not\in L_1\\
      a^{-1}(L_1) \cdot L_2 \cup a^{-1}(L_2), & \text{ ако }\varepsilon\in L_1
    \end{cases}$
  \item
    $a^{-1}(L^\star) = a^{-1}(L) \cdot L^\star$.
  \end{enumerate}
\end{problem}
Аналогично, за произволна дума $\alpha$, нека
\[\alpha^{-1}(L) \df \{\omega \in \Sigma^\star \mid \alpha\omega \in L\}.\]

\begin{proposition}
  \label{pr:pullback}
  За всеки две думи $\alpha$ и $\beta$ е изпълнено, че:
  \[(\alpha\cdot\beta)^{-1}(L) = \beta^{-1}(\alpha^{-1}(L)).\]
\end{proposition}
\begin{proof}
  За прозиволна дума $\gamma$ имаме следните еквивалентности:
  \begin{align*}
    \gamma \in \beta^{-1}(\alpha^{-1}(L)) & \iff \beta\gamma \in \alpha^{-1}(L)\\
                                          & \iff \alpha\beta\gamma \in L\\
                                          & \iff \gamma \in (\alpha\beta)^{-1}(L).
  \end{align*}
\end{proof}

\begin{problem}
  Докажете, че ако $L$ е регулярен език, то $\alpha^{-1}(L)$ е регулярен език.
\end{problem}

\begin{problem}
  \label{prob:language-pullback}
  Докажете, че $L = \{\omega \in \Sigma^\star \mid \varepsilon \in \omega^{-1}(L)\}$.
\end{problem}


Нека е даден произволен език $L$. Ще покажем конструкция на детерминиран автомат
\[\B = \FA,\]
който разпознава $L$. Този автомат ще наречем \emph{автомат на Бжозовски}.
Ако $L$ е регулярен, то $\B$ ще бъде детерминиран краен автомат,
но ако $L$ не е регулярен, то $\B$ ще бъде детерминиран \emph{безкраен} автомат.
Конструкцията на автомата $\B$ е следната:
% \mynote{Да напомним, че имаме свойството
%   \[\alpha \in L \iff \varepsilon \in \alpha^{-1}(L).\]
%   Все още не ясно, че ако $L$ е регулярен, то $Q$ е крайно множество. Това ще видим след малко.
% }
\begin{itemize}
\item
  Състоянията $Q$ на автомата на Бжозовски ще бъдат езици над азбуката $\Sigma$, където:
  \[Q^\B \df \{\alpha^{-1}(L) \mid \alpha\in\Sigma^\star\}.\]
\item
  Началното състояние $\qstart$ на автомата на Бжозовски е дефинирано като:
  \[\qstart^\B \df L\hspace*{20pt} \comment{L = \varepsilon^{-1}(L)}.\]
\item
  За произволно състояние (език) $M$ и буква $a$, дефинираме функцията на преходите ето така:
  \[\delta_\B(M,a) \df a^{-1}(M).\]
\item
  Финалните състояния на автомата на Бжозовски са следните:
  \[F^\B \df \{ M \in Q^\B\mid \varepsilon \in M\}.\]
\end{itemize}

\begin{proposition}\label{pr:regular:brzozowski:delta}
  Нека $\B$ е автоматът на Бжозовски за езика $L$.
  За всяка дума $\alpha$ и всяко състояние $M \in Q^\B$ е изпъленена еквивалентността:
  \begin{equation}
    \label{eq:13}
    \delta^\star(M,\alpha) = \alpha^{-1}(M).
  \end{equation}
\end{proposition}
\begin{hint}
  \mynote{\todo Опитайте се да докажете това твърдение сами!}
  Индукция по дължината на думата $\alpha$, като използвате, че
  \[(\alpha b)^{-1}(M) = b^{-1}(\alpha^{-1}(M)).\]
\end{hint}
\begin{proof}
  \begin{itemize}
  \item
    Твърдението очевидно е изпълнено за $\alpha = \varepsilon$, защото
    \[\delta^\star(M,\varepsilon) \df M = \varepsilon^{-1}(M).\]
  \item
    Да примем, че за думи $\alpha$ с дължина $n$ е изпълнено Свойство (\ref{eq:13}).    
  \item
    Сега да разгледаме дума $\alpha$ с дължина $n+1$, т.е. $\alpha = \beta a$ и $|\beta| = n$. Тогава
    \begin{align*}
      \delta^\star(M,\beta a) & = \delta(\delta^\star(M,\beta),a) & \comment\text{от деф. на }\delta^\star\\
                              & =\delta(\beta^{-1}(M),a) & \comment\text{от \IndHyp}\\
                              & = a^{-1}(\beta^{-1}(M)) & \comment\text{от деф. на }\delta\\
                              & = (\beta a)^{-1}(M). & \comment\text{от \Proposition{pullback}}
    \end{align*}
  \end{itemize}
\end{proof}


\begin{important}
  \begin{proposition}
    \label{pr:brzozowski:language}
    За даден език $L$, нека $\B$ е детерминираният автомат построен по метода на Бжозовски.
    Тогава $L = \L(\B)$.
  \end{proposition}  
\end{important}
\mynote{Тук е важно да отбележим, че все още не знаем дали $\B$ има крайно много състояния. В \Example{regular:brzozowski:an-bn} ще видим един детерминиран безкраен автомат за език, който не е регулярен.}
\begin{hint}
  Съобразете, че имаме следните еквивалентности:
  \begin{align*}
    \alpha \in L & \iff \varepsilon\in\alpha^{-1}(L) & \comment\text{от \Problem{language-pullback}}\\
                 & \iff \varepsilon \in \delta^\star_\B(L,\alpha) & \comment\text{ от \Proposition{regular:brzozowski:delta}}\\
                 & \iff \delta^\star_\B(\qstart^\B,\alpha) \in F^\B & \comment \varepsilon \in \alpha^{-1}(L)\\
                 & \iff \alpha \in \L(\B). & \comment\text{деф. на език на автомат}
  \end{align*}
\end{hint}


%%% Local Variables:
%%% mode: latex
%%% TeX-master: "../eai"
%%% End:
