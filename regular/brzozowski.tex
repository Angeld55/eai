\section{Метод на Бжозовски}\label{sect:regular:brzozowski}
\index{Бжозовски}

Имаме следната операция за произволна буква $a$,
\[a^{-1}(L) \df \{\omega \in \Sigma^\star \mid a\omega \in L\}.\]
Аналогично, за произволна дума $\alpha$,
\[\alpha^{-1}(L) \df \{\omega \in \Sigma^\star \mid \alpha\omega \in L\}.\]

\begin{problem}
  Докажете, че:
  \begin{enumerate}[(1)]
  \item
    $a^{-1}(L_1 \cup L_2) = a^{-1}(L_1) \cup a^{-1}(L_2)$;
  \item
    $a^{-1}(L_1 \cap L_2) = a^{-1}(L_1) \cap a^{-1}(L_2)$;
  \item
    $a^{-1}(L_1 \setminus L_2) = a^{-1}(L_1) \setminus a^{-1}(L_2)$;
  \item
    $a^{-1}(L_1 \cdot L_2) =
    \begin{cases}
      a^{-1}(L_1) \cdot L_2, & \text{ ако }\varepsilon\not\in L_1\\
      a^{-1}(L_1) \cdot L_2 \cup L_2, & \text{ ако }\varepsilon\in L_1
    \end{cases}$
  \item
    $a^{-1}(L^\star) = a^{-1}(L) \cdot L^\star$.
  \end{enumerate}
\end{problem}

\begin{problem}\label{prob:pullback}
  За всеки две думи $\alpha, \beta \in \Sigma^\star$, докажете, че
  \[(\alpha\cdot\beta)^{-1}(L) = \beta^{-1}(\alpha^{-1}(L)).\]
\end{problem}

\begin{problem}
  Докажете, че ако $L$ е регулярен език, то $\alpha^{-1}(L)$ е регулярен език.
\end{problem}

\begin{problem}\label{prob:language-pullback}
  Докажете, че 
  \[L = \{\omega \in \Sigma^\star \mid \varepsilon \in \omega^{-1}(L)\}.\]
\end{problem}

\mynote{Бжозовски \cite{brzozowski-derivatives} описва алгоритъм за строене на автомат по регулярен израз.}

Нека е даден езикът $L$. Ще покажем конструкция на детерминиран автомат $\B = \FA$,
който разпознава $L$. Ако $L$ е регулярен, то $\B$ ще бъде детерминиран краен автомат,
но ако $L$ не е регулярен, то $\B$ ще бъде детерминиран \emph{безкраен} автомат.
Конструкцията на автомата $\B$ е следната:
\mynote{Да напомним, че имаме свойството
  \[\alpha \in L \iff \varepsilon \in \alpha^{-1}(L).\]
  Все още не ясно, че ако $L$ е регулярен, то $Q$ е крайно множество. Това ще видим след малко.
  В \Example{regular:brzozowski:an-bn} ще видим един детерминиран безкраен автомат за език, който не е регулярен.}
\begin{itemize}
\item
  Състоянията $Q$ ще бъдат от вида $\hat{M}$, за $M \subseteq \Sigma^\star$, където:
  \[Q \df \{\hat{M} \mid (\exists \alpha\in\Sigma^\star)[M = \alpha^{-1}(L)].\]
\item
  $\qstart \df \hat{L}$.
\item
  За произволни състояния $\hat{M}$ и $\hat{N}$ и буква $a$,
  \[\delta(\hat{M},a) = \hat{N} \stackrel{\text{деф}}{\iff} N = a^{-1}(M).\]
\item
  $F \df \{ \hat{M} \in Q\mid \varepsilon \in M\}$.
\end{itemize}

\begin{proposition}\label{pr:regular:brzozowski:delta}
  За всяка дума $\alpha$ е изпълнено, че:
  \[N = \alpha^{-1}(L) \iff \delta^\star(\hat{L},\alpha) = \hat{N}.\]
\end{proposition}
\begin{hint}
  Индукция по дължината на думата $\alpha$, като използвате, че
  \[(\alpha b)^{-1}(L) = b^{-1}(\alpha^{-1}(L)).\]
  Твърдението очевидно е изпълнено за $\alpha = \varepsilon$.
  Да примем, че за думи $\alpha$ с дължина $n$ е изпълено, че
  \begin{equation}
    \label{eq:13}
    N = \alpha^{-1}(L) \iff \delta^\star(\hat{L},\alpha) = \hat{N}.
  \end{equation}
  Да разгледаме дума $\alpha$ с дължина $n+1$, т.е. $\alpha = \beta a$ и $|\beta| = n$. Тогава
  \begin{align*}
    \delta^\star(\hat{L},\beta a) & = \delta(\delta^\star(\hat{L},\beta),a) & \comment{\text{от деф.}}\\
                                  & =\delta(\hat{M},a) & \comment{\text{нека }\hat{M} = \delta^\star(\hat{L},\beta)}\\
                                  & = \hat{N}.
  \end{align*}
  Свойство (\ref{eq:13}) приложено за $\beta$ ни казва, че
  \[\delta^\star(\hat{L},\beta) = \hat{M} \iff \beta^{-1}(L) = N.\]
  От дефиницията на $\delta$ имаме, че
  \[\delta(\hat{M},a) = \hat{N} \iff N = a^{-1}(M).\]
  Накрая заключаваме, че
  \begin{align*}
    \delta^\star(\hat{L},\beta a) = \hat{N} \iff N = a^{-1}(\beta^{-1}(L)) = \alpha^{-1}(L).
  \end{align*}
  
\end{hint}

\mynote{Тук е важно да отбележим, че все още не знаем, че $\B$ е краен.}
\begin{proposition}
  За даден език $L$, нека $\B$ е детерменираният автомат построен по метода на Бжозовски.
  Тогава $L = \L(\B)$.
\end{proposition}
\begin{hint}
  Съобразете, че имаме следните еквивалентности:
  \begin{align*}
    \alpha \in L & \iff \varepsilon\in\alpha^{-1}(L) & \comment\text{от \Problem{language-pullback}}\\
    & \iff \varepsilon\in M\ \&\ \hat{M} = \delta^\star(\hat{L},\alpha) & \comment{M \df \alpha^{-1}(L)\text{ от \Proposition{regular:brzozowski:delta}}}\\
                 & \iff \delta^\star(\qstart,\alpha) \in F. & \comment \qstart \df \hat{L} \text{ и }\varepsilon \in \alpha^{-1}(L)
  \end{align*}
\end{hint}

%%% Local Variables:
%%% mode: latex
%%% TeX-master: "../eai"
%%% End:
