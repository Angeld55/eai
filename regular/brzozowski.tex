\section{Метод на Бжозовски}\label{sect:regular:brzozowski}
\index{Бжозовски}

Имаме следната операция за произволна буква $a$,
\[a^{-1}(L) \df \{\omega \in \Sigma^\star \mid a\omega \in L\}.\]
Аналогично, за произволна дума $\alpha$,
\[\alpha^{-1}(L) \df \{\omega \in \Sigma^\star \mid \alpha\omega \in L\}.\]

\begin{problem}
  Докажете, че:
  \begin{enumerate}[(1)]
  \item
    $a^{-1}(L_1 \cup L_2) = a^{-1}(L_1) \cup a^{-1}(L_2)$;
  \item
    $a^{-1}(L_1 \cap L_2) = a^{-1}(L_1) \cap a^{-1}(L_2)$;
  \item
    $a^{-1}(L_1 \setminus L_2) = a^{-1}(L_1) \setminus a^{-1}(L_2)$;
  \item
    $a^{-1}(L_1 \cdot L_2) =
    \begin{cases}
      a^{-1}(L_1) \cdot L_2, & \text{ ако }\varepsilon\not\in L_1\\
      a^{-1}(L_1) \cdot L_2 \cup L_2, & \text{ ако }\varepsilon\in L_1
    \end{cases}$
  \item
    $a^{-1}(L^\star) = a^{-1}(L) \cdot L^\star$.
  \end{enumerate}
\end{problem}


\marginpar{Бжозовски \cite{brzozowski-derivatives} описва алгоритъм за строене на автомат по регулярен израз.}

Нека е даден езикът $L$. Ще покажем конструкция на детерминиран автомат $\B = \FA$,
който разпознава $L$. Ако $L$ е регулярен, то $\B$ ще бъде детерминиран краен автомат,
но ако $L$ не е регулярен, то $\B$ ще бъде детерминиран \emph{безкраен} автомат.
Конструкцията на автомата $\B$ е следната:
\marginpar{Да напомним, че имаме свойството
  \[\alpha \in L \iff \varepsilon \in \alpha^{-1}(L).\]
  Все още не ясно, че ако $L$ е регулярен, то $Q$ е крайно множество. Това ще видим след малко.
  В \Example{regular:brzozowski:an-bn} ще видим един детерминиран безкраен автомат за език, който не е регулярен.}
\begin{itemize}
\item
  Състоянията $Q$ ще бъдат от вида $q_M$, за $M \subseteq \Sigma^\star$, където:
  \[Q \df \{q_M \mid (\exists \alpha\in\Sigma^\star)[M = \alpha^{-1}(L)].\]
\item
  $\qstart \df q_L$.
\item
  За произволни езици $M$ и $N$ и буква $a$,
  \[\delta(q_M,a) \df q_N \stackrel{\text{деф}}{\iff} N = a^{-1}(M).\]
\item
  $F \df \{ q_M \in Q\mid \varepsilon \in M\}$.
\end{itemize}

\begin{proposition}\label{pr:regular:brzozowski:delta}
  За всяка дума $\alpha$ е изпълнено, че:
  \[N = \alpha^{-1}(L) \iff \delta^\star(q_L,\alpha) = q_N.\]
\end{proposition}
\begin{hint}
  Индукция по дължината на думата $\alpha$, като използвате, че
  \[(\alpha b)^{-1}(L) = b^{-1}(\alpha^{-1}(L)).\]
\end{hint}

\begin{proposition}
  За даден език $L$, нека $\B$ е детерменираният автомат построен по метода на Бжозовски.
  Тогава $L = \L(\B)$.
\end{proposition}
\begin{hint}
  Съобразете, че имаме следните еквивалентности:
  \begin{align*}
    \alpha \in L & \iff \varepsilon\in\alpha^{-1}(L) & \comment\text{нека }M \df \alpha^{-1}(L)\\
                 & \iff \varepsilon\in M\ \&\ q_M = \delta^\star(q_L,\alpha) & \comment\text{от \Proposition{regular:brzozowski:delta}}\\
                 & \iff \delta^\star(\qstart,\alpha) \in F. & \comment \qstart \df q_L
  \end{align*}
\end{hint}

%%% Local Variables:
%%% mode: latex
%%% TeX-master: "../eai"
%%% End:
