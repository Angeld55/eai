\section{Допълнителни задачи}

\section{Допълнителни задачи}

\subsection{Лесни задачи}

\ifcode
\begin{problem}
  \marginnote{Библиотеката {\bf regex} е част от си++11 стандарта на езика.}
  Да разгледаме следната програма на езика си++
  \begin{minted}[]{cpp}
    #include <iostream>
    #include <regex>
    #include <string>
    
    using namespace std;
    
    int main() {
      string input;
      regex reg("your regular expression");
      while (true) {
        cout << "Input:" << endl;
        cin >> input;
        if (!cin || input=="q") break;
        if (regex_match(input, reg)) {
          cout << "Valid input" << endl;
        }  
        else {
          cout << "Invalid input" << endl;
        }
      }
    }
  \end{minted}
  Разучете как се създават обекти от тип {\bf regex} и попълнете дефиницията на регулярния израз {\bf reg} в горната програма, така че програмата приема за валиден вход:
  \begin{enumerate}[a)]
  \item 
    само реални числа;
  \item
    факултетни номера във ФМИ;
  \item
    низове, които са съставени от поне 8 символа, измежду които се включват малки букви, големи букви,
    и цифри.    
  \end{enumerate}
\end{problem}
\fi

% \begin{problem}
%   Да фиксираме една дума $\alpha$ над дадена азбука $\Sigma$.
%   Опишете алгоритъм, който за вход произволен текстов файл, чието съдържание означаваме с $\beta$,
%   отговаря дали думата $\alpha$ се среща в $\beta$.
%   Каква е сложността на този алгоритъм относно дължините на $\alpha$ и $\beta$ ?
% \end{problem}

% \begin{problem}
%   Опишете алгоритъм, който при вход два регулярни израза $\mathbf{r}$ и $\mathbf{s}$,
%   проверява дали $\L(\mathbf{r}) \subseteq \L(\mathbf{s})$.
% \end{problem}

% \setlength{\marginparsep}{18pt}
% \setlength{\oddsidemargin}{5pt}
% \setlength{\evensidemargin}{100pt}
% \setlength{\hoffset}{-30pt}
% \setlength{\voffset}{-30pt}

% \newgeometry{marginparwidth=20pt, evensidemargin=20pt}
% \newgeometry{textwidth=600pt}

% \layout

\ExtraMaterial{
\begin{problem}
  За всеки от следните езици $L$, постройте минимален краен детерминиран автомат $\A$, който разпознава езика $L$, където:
  \marginnote{$\card{\omega}{a} \df $ броят на срещанията на буквата $a$ в думата $\omega$, $\abs{\omega} \df $ дължината на $\omega$.}
  \begin{enumerate}[a)]
  \item 
    $L = \{a^nb\mid n \geq 0\}$;
  \item
    $L = \{a,b\}^\star\setminus\{\varepsilon\}$;
  \item
    $L = \{\omega \in \{a,b\}^\star \mid \card{\omega}{a}\text{ и }\card{\omega}{b}\text{ са четни}\}$;
  \item
    $L = \{\omega \in \{a,b\}^\star \mid \card{\omega}{a}\text{ е четно}\ \&\ \card{\omega}{b}\text{ е нечетно}\}$;
  \item
    $L = \{a^nb^m\mid n,m \geq 0\}$;
  \item
    $L = \{a^nb^m\mid n,m \geq 1\}$;
  \item
    $L = \{a,b\}^\star \setminus \{a\}$;
  \item
    $L = \{\omega \in \{a,b\}^\star \mid \card{\omega}{a} \geq 2\ \lor\ \card{\omega}{b} \leq 3 \}$;
  \item
    $L = \{\omega \in \{a,b\}^\star \mid \card{\omega}{a} \geq 2\ \&\ \card{\omega}{b} \geq 1\}$;
  \item
    $L = \{\omega \in \{a,b\}^\star \mid \text{на всяка нечетна позиция на }\omega\text{ е буквата }a\}$;
  \item
    $L = \{\omega \in \{a,b\}^\star \mid \card{\omega}{a}\text{ е четно }\&\ \card{\omega}{b} \leq 1\}$;
  \item
    $L = \{\omega \in \{a,b\}^\star \mid \abs{\omega} \leq 3\}$;
  \item
    $L = \{\omega \in \{a,b\}^\star \mid \omega \text{ не започва с }ab\}$;
  \item
    $L = \{\omega \in \{a,b\}^\star \mid \omega \text{ завършва с }ab\text{ или ba}\}$;
  \item
    $L = \{\omega \in \{a,b\}^\star \mid \omega\text{ започва или завършва с } a\}$;
  \item
    $L = \{\omega \in \{a,b\}^\star \mid \omega\text{ започва с }a \iff \omega\text{ завършва с }b\}$;
  \item
    $L = \{\omega \in \{a,b\}^\star \mid \abs{\omega} \equiv 0\ (\bmod\ 2)\ \&\ \card{\omega}{a} = 1\}$;
  \item
    $L = \{\omega \in \{a,b\}^\star \mid \text{ всяко }a\text{ в }\omega\text{ се следва веднага от поне едно }b\}$;
  \item
    $L = \{\omega \in \{a,b\}^\star \mid \abs{\omega} \equiv 0 \bmod 3\}$;
  \item
    $L = \{\omega \in \{a,b\}^\star \mid \card{\omega}{a} \equiv 1 \bmod 3\}$;
  \item
    $L = \{\omega \in \{a,b\}^\star \mid \card{\omega}{a} \equiv 0 \bmod 3\ \&\ \card{\omega}{b} \equiv 1 \bmod 2\}$;
  \item
    $L = \{\omega \in \{a,b\}^\star \mid \card{\omega}{a} \equiv 0 \bmod 2\ \vee\ \card{\omega}{b} = 2\}$;
  \item
    $L = \{\omega \in \{a,b\}^\star \mid \omega \text{ съдържа равен брой срещания на }ab\text{ и на }ba\}$;
  \item
    $L = \{\omega_1 \sharp \omega_2 \sharp \omega_3 \mid |\omega_1| \geq 2\ \&\ |\omega_2| \geq 3\ \&\ |\omega_3| \geq 4\ \&\ \omega_i \in \{a,b\}^\star\text{
      за }i = 1,2,3\}$.
  \end{enumerate}
\end{problem}
}

\begin{extra2}
\begin{problem}
  Нека $\Sigma = \{a,b\}$.  Проверете дали $L$ е регулярен, където
  % \begin{multicols}{2}
  \begin{enumerate}[a)]
  % \item
  %   $L = \{\alpha^{\texttt{rev}} \mid \alpha \in L_0\}$, където $L_0$ е регулярен;
  \item
    % \marginnote{$\alpha = a^pb^p$}
    $L = \{a^ib^i\ \mid\ i\in\Nat\}$;
  \item
    $L = \{a^ib^i\ \mid\ i,j\in\Nat\ \&\ i\neq j\}$;
  \item
    % \marginnote{$\alpha = a^{p+1}b^p$.}
    $L = \{a^ib^j\ \mid\ i > j\}$;
  \item
    $L = \{a^nb^m \mid n\text{ дели }m\}$.
  \item
    $L = \{a^{2n}\ \mid\ n\geq 1\}$;
  \item
    $L = \{a^mb^na^{m+n}\ \mid\ m\geq 1\ \&\ n\geq 1\}$;
  \item
    $L = \{a^{n.m}\mid n,m\text{ са прости числа}\}$;
  \item
    $L = \{\omega\in\{a,b\}^\star \mid \card{\omega}{a} = \card{\omega}{b}\}$;
  \item
    % \marginnote{$\alpha = a^pba^pb$}
    $L = \{\omega\omega\mid \omega\in\{a,b\}^\star\}$;
  \item
    $L = \{\omega\omega^\rev\mid \omega\in\{a,b\}^\star\}$;
  \item
    $L = \{\alpha\beta\beta \in \{a,b\}^\star\mid \beta \neq \varepsilon\}$;
  \item
    $L = \{a^nb^nc^n\mid n\geq 0\}$;
  \item
    $L = \{\omega\omega\omega\mid \omega\in \Sigma^\star\}$;
  \item
    $L = \{a^{2^n}\mid n\geq 0\}$;
  \item
    $L = \{a^mb^n\mid n\neq m\}$;
  \item
    $L = \{a^{n!}b^{n!}\mid n\neq 1\}$;
  \item
    $L = \{a^{f_n} \mid f_0 = f_1 = 1\ \&\ f_{n+2} = f_{n+1} + f_{n}\}$;
  \item
    $L = \{\alpha \in \Sigma^\star \mid \abs{\ \card{\alpha}{a} - \card{\alpha}{b}\ } \leq 2\}$;
  \item
    $L = \{\alpha\beta\alpha \mid \alpha,\beta \in \Sigma^\star\ \&\ \abs{\beta} \leq \abs{\alpha}\}$;
  \item
    $L = \{\beta\gamma\gamma^\rev\mid \beta, \gamma \in \Sigma^\star\ \&\ \abs{\beta} \leq \abs{\gamma}\}$;
  \item
    $L = \{c^ka^nb^m \mid k,m,n > 0\ \&\ n \neq m\}$;
  \item
    $L = \{c^ka^nb^n \mid k > 0\ \&\ n \geq 0\}\cup\{a,b\}^\star$;
  \item
    $L = \{\omega \in \{a,b\}^\star \mid \card{\omega}{a}\text{ не дели }\card{\omega}{b}\}$;
  \item
    $L = \{\omega \in \{a,b\}^\star \mid \card{\omega}{a} < \card{\omega}{b}\}$;
  \item
    $L = \{\omega \in \{a,b\}^\star \mid \card{\omega}{a} = 2\card{\omega}{b}\}$;
  \item
    $L = \{\omega \in \{a,b\}^\star \mid \abs{\ \card{\omega}{a} - \card{\omega}{b}\ } \leq 3\}$.
  \end{enumerate}    
\end{problem}
\end{extra2}

% \restoregeometry

\begin{problem}
  Докажете, че следните езици са регулярни:
  \begin{enumerate}[a)]
  \item
    $L = \{\alpha \in \{a,b\}^\star \mid \text{ за всяка представка $\omega$ на $\alpha$ имаме }\abs{\ \card{\omega}{a} - \card{\omega}{b}\ } \leq 2 \}$;
  \item
    $L = \{\alpha \in \{a,b\}^\star \mid \text{ за някоя представка $\omega$ на $\alpha$ имаме }\abs{\ \card{\omega}{a} - \card{\omega}{b}\ } > 2 \}$;
  \item
    $L = \{\alpha \in \{a,b\}^\star \mid \text{ за някоя наставка $\omega$ на $\alpha$ имаме }\abs{\ \card{\omega}{a} - \card{\omega}{b}\ } > 2 \}$.
  \end{enumerate}
\end{problem}



% \begin{problem}
%   Нека $L$ е регулярен език над азбуката $\Sigma$. Докажете, че 
%   \[\text{Infix}(L) = \{\alpha \mid (\exists \beta,\gamma \in \Sigma^\star)[\beta\alpha\gamma \in L]\}\]
%   също е регулярен език.
% \end{problem}
% \begin{hint}
%   Най-лесно е да се построи автомат за $\text{Infix}(L)$ като се използва автомата за $L$.
% \end{hint}

\begin{problem}
  Нека $\Sigma = \{a,b,c,d\}$.
  Да се докаже, че езикът
  \[L = \{a_1a_2\cdots a_{2n} \in \Sigma^\star \mid (\forall j \in [1,n])[a_{2j-1} = a_{2j}]\ \&\ d\text{ се среща $\leq 3$ пъти}\}\]
  е регулярен.
\end{problem}

\begin{problem}
  Нека $L_1$ и $L_2$ са регулярни езици. Докажете, че $L$ също е регулярен език, където
  \[L = \{\alpha \mid (\exists \beta,\gamma)[\beta\alpha\gamma \in L_1]\ \&\ \alpha \in L_2 \vee \alpha^{\texttt{rev}} \in L_2\}.\]
\end{problem}

\begin{definition}
  \index{хомоморфизъм}
  Да фиксираме две азбуки $\Sigma_1$ и $\Sigma_2$.
  Хомоморфизъм е изображение $h:\Sigma^\star_1 \to \Sigma^\star_2$ със свойството, че
  за всеки две думи $\alpha,\beta\in\Sigma^\star_1$,
  \[h(\alpha\beta) = h(\alpha)\cdot h(\beta).\]
\end{definition}

Лесно се съобразява, че за всеки хомоморфизъм $h$, $h(\varepsilon) = \varepsilon$.

\begin{problem}
  Нека $L \subseteq \Sigma^\star_1$ е регулярен език и $h:\Sigma^\star_1\to\Sigma^\star_2$ е хомоморфизъм.
  Тогава
  $h(L) = \{h(\alpha) \in \Sigma^\star_2 \mid \alpha \in L\}$ е регулярен.
\end{problem}
\begin{hint}
  Индукция по построението на регулярни езици.
  % \begin{itemize}[-]
  % \item 
  %   За $L = \{a\}$, $h(L) = \{h(a)\}$.
  % \item
  %   $h(\emptyset) = \emptyset$.
  % \item
  %   Нека $L_1 = \L(r_1)$ и $L_2 = \L(r_2)$.
  %   Ще докажем, че $h(\L(r_1\cdot r_2))$ е регулярен.
  %   \begin{align*}
  %     h(\L(r_1\cdot r_2)) & = h(L_1\cdot L_2) & (\text{деф. на конкатенация})\\
  %     & = \{h(\gamma) \mid \gamma \in L_1 \cdot L_2\}\\
  %     & = \{h(\alpha\beta) \mid \alpha\in L_1\ \&\ \beta\in L_2\}\\
  %     & = \{h(\alpha)\cdot h(\beta) \mid \alpha \in L_1\ \&\ \beta \in L_2\} & (h\text{ е хомоморфизъм})\\
  %     & = \{\omega\gamma \mid \omega \in h(L_1)\ \&\ \gamma \in h(L_2)\}\\
  %     & = h(L_1)\cdot h(L_2).
  %   \end{align*}
  %   По И.П. имаме, че $h(L_1)$ и $h(L_2)$ са регулярни езици.
  %   Следователно, 
  %   \[h(\L(r_1\cdot r_2)) = h(L_1)\cdot h(L_2)\]
  %   е регулярен език.
  % \item
  %   От горното свойство имаме също, че за всяко $n$, $h(L^n) = h(L)^n$.
  % \item
  %   Освен това, 
  %   \begin{align*}
  %     h(\bigcup_n L_n) & = \{h(\alpha) \mid (\exists n)[\alpha \in L_n]\}\\
  %     & = \bigcup \{h(\alpha) \mid \alpha \in L_n\}\\
  %     & = \bigcup_n h(L_n).
  %   \end{align*}
  % \item
  %   Нека $L = \L(r^\star)$.
  %   Ще докажем, че $h(L^\star)$ е регулярен език.
  %   \begin{align*}
  %     h(L^\star) & = h(\bigcup_n L^n) & (\text{деф. на звезда на Клини})\\
  %     & = \bigcup_n h(L^n) & (\text{от горното свойство})\\
  %     & = \bigcup_n h(L)^n & (\text{от по-горното свойство})\\
  %     & = h(L)^\star & (\text{по деф.}).
  %   \end{align*}
  % \end{itemize}
\end{hint}

\begin{problem}
  Нека $L\subseteq \Sigma^\star_2$ е регулярен език и $h:\Sigma^\star_1\to\Sigma^\star_2$ е хомоморфизъм.
  Тогава езикът
  $h^{-1}(L) = \{\alpha \in \Sigma^\star_1 \mid h(\alpha) \in L\}$ е регулярен.  
\end{problem}
\begin{hint}
  Конструкция на автомат за $h^{-1}(L)$ при даден автомат за $L$.
  % Нека $\A$ е КДА разпознаващ езика $L$.
  % Ще построим $\A' = \pair{Q,\Sigma_1, \delta', s, F}$,
  % където дефинираме функцията на преходите $\delta'$ като $\delta'(q,a) = \delta^\star(q,h(a))$.
  % Понеже $h$ е хомоморфизъм, лесно се доказва с индукция по дължината на думата $\alpha \in \Sigma^\star_1$,
  % че $\delta'^\star(q,\alpha) = \delta^\star(q,h(\alpha))$.
  % Сега лесно се вижда, че $h^{-1}(\L(\A)) = \L(\A')$, защото:
  % \begin{align*}
  %   \alpha \in \L(\A') & \iff \delta'^\star(s,\alpha) \in F\\
  %   & \iff \delta^\star(s,h(\alpha)) \in F\\
  %   & \iff h(\alpha) \in \L(\A)\\
  %   & \iff \alpha \in h^{-1}(\L(\A)).
  % \end{align*}
\end{hint}

\begin{problem}
  Нека $\Sigma_1$ и $\Sigma_2$ са непресичащи се азбуки, а $L_1$ и $L_2$ са езици съответно над $\Sigma_1$ и $\Sigma_2$.
  За една дума $\omega \in (\Sigma_1 \cup \Sigma_2)^\star$, нека с $\omega_i \in \Sigma^\star_i$ да означим редицата от букви от $\Sigma_i$
  в реда, в който се срещат в $\omega$. Да разгледаме следния език
  \[L_1 \oplus L_2 = \{\omega \in (\Sigma_1 \cup \Sigma_2)^\star \mid \omega_1 \in L_1\ \&\ \omega_2 \in L_2\ \&\ |\omega_1| = |\omega_2|\}.\]
  \begin{enumerate}[a)]
  \item
    \marginnote{Да}
    Вярно ли е, че ако $L_1$ е краен, то $L_1 \oplus L_2$ е регулярен език?
  \item
    \marginnote{Не}
    Вярно ли е, че ако $L_1$ и $L_2$ са регулярни езици, то $L_1 \oplus L_2$ е регулярен език?
  \end{enumerate}
\end{problem}

\begin{problem}
  Нека $\Sigma_1$ и $\Sigma_2$ са непресичащи се азбуки, а $L_1$ и $L_2$ са езици съответно над $\Sigma_1$ и $\Sigma_2$.
  За една дума $\omega \in (\Sigma_1 \cup \Sigma_2)^\star$, нека с $\omega_i \in \Sigma^\star_i$ да означим редицата от букви от $\Sigma_i$
  в реда, в който се срещат в $\omega$. Да разгледаме следния език
  \[L_1 \oplus L_2 = \{\omega \in (\Sigma_1 \cup \Sigma_2)^\star \mid \omega_1 \in L_1\ \&\ \omega_2 \in L_2\}.\]
  \marginnote{Да}
  Вярно ли е, че ако $L_1$ и $L_2$ са регулярни езици, то $L_1 \oplus L_2$ е регулярен език?
\end{problem}

%%% Local Variables:
%%% mode: latex
%%% TeX-master: "../eai"
%%% End:



\subsection{Не толкова лесни задачи}

{\bf Това вече е добре да се махне}
\begin{problem}
  Докажете, че няма полиномиален алгоритъм за детерминизация на краен недетерминиран автомат.
\end{problem}
\begin{hint}
  За произволно $n$, разгледайте недетерминирания автомат $\A_n$, за който
  \[(\forall \alpha,\beta \in \{0,1\}^\star)[|\alpha| = |\beta| = n \implies (\alpha\beta \in \L(\A_n) \iff \alpha \neq \beta)].\]
  Този автомат ще има $2n+2$ състояния.

  Допуснете, че за него съществува детерминиран автомат $\D_n$ с $< 2^n$ на брой състояния.
  Разгледайте всички думи с дължина $n$, $\omega_1,\omega_2,\dots,\omega_{2^n}$.
  Приложете принципа на Дирихле и достигнете до противоречие.
\end{hint}


\begin{problem}
  \marginpar{\cite[стр. 84]{papadimitriou}}
  При дадени езици $L$, $L'$ над азбуката $\Sigma$, да разгледаме:
  \begin{enumerate}[a)]
  \item
    $\texttt{Pref}(L) = \{\alpha \in \Sigma^\star \mid (\exists \beta \in \Sigma^\star)[\alpha\beta \in L]\}$;
  \item
    $\texttt{NoPref}(L) = \{\alpha \in L \mid \text{ не съществува префикс на $\alpha$ в $L$}\}$;
  \item
    $\texttt{NoExtend}(L) = \{\alpha \in L \mid \text{ $\alpha$ не е префикс на никоя дума от $L$}\}$;
  \item
    $\mbox{Suf}(L) = \{\beta \in \Sigma^\star \mid (\exists \alpha \in \Sigma^\star)[\alpha\beta \in L]\}$;
  \item
    $\text{Infix}(L) = \{\alpha \mid (\exists \beta,\gamma \in \Sigma^\star)[\beta\alpha\gamma \in L]\}$;
  \item 
    $\frac{1}{2}(L) = \{\omega \in \Sigma^\star \mid (\exists \alpha \in \Sigma^\star)[\omega\alpha \in L\ \&\ \abs{\omega} = \abs{\alpha}]\}$;
  \item
    \marginpar{right quotient of $L$ by $L'$}
    $L/L' = \{\alpha \in \Sigma^\star \mid (\exists \beta \in L')[\alpha\beta \in L ] \}$;
  \item
    $L^{-1}(L') = \{ \beta \mid (\exists \alpha \in L)[ \alpha\beta \in L']\}$;
  \item
    $\mbox{Max}(L) = \{\alpha \in \Sigma^\star \mid (\forall \beta\in\Sigma^\star)[\beta \neq \varepsilon\implies \alpha\beta \not\in L]\}$.
  \end{enumerate}
  За всички тези езици, докажете, че са регулярни при условие, че $L$ и $L'$ са регулярни.
  \marginpar{Тази конструкция няма да бъде ефективна}
  Освен това, докажете, че $L/L'$ е регулярен и при условието, че $L$ е регулярен, но $L'$ е произволен език над азбуката $\Sigma$.
\end{problem}
\begin{hint}
  \begin{enumerate}[a)]
  \item 
    Индукция по дефиницията на регулярен израз.
  \item[в)]
    Най-лесно е да се построи автомат за $\text{Infix}(L)$ като се използва автомата за $L$.
  \item[г)]
    Конструкция с автомат за $L$ и автомат за $L^{\texttt{rev}}$.
  \end{enumerate}
\end{hint}

\begin{problem}
  \marginpar{\cite[стр. 75]{kozen}; \cite[стр. 89]{papadimitriou}}
  Да фиксираме азбука само с един символ $\Sigma = \{a\}$.
  Да положим за всяко $p,q\in\Nat$, 
  \[L(p,q) = \{a^k \mid (\exists n\in\Nat)[k = p+q\cdot n]\}.\]
  Ако за един език $L$ съществуват константи $p_1,\dots,p_k$ и $q_1,\dots,q_k$, такива че 
  \[L = \bigcup_{1\leq i \leq k} L(p_i,q_i),\]
  то казваме, че $L$ е {\em породен от аритметични прогресии}.
  \begin{enumerate}[a)]
  \item 
    Докажете, че $L \subseteq \{a\}^\star$ е регулярен език точно тогава, когато $L$ е породен от аритметична прогресия.
  \item
    За произволна азбука $\Sigma$, докажете, че ако $L \subseteq \Sigma^\star$ е регулярен език,
    то езикът $\{a^{\abs{\omega}} \mid \omega \in L\}$  е породен от аритметична прогресия.    
  \end{enumerate}
\end{problem}
\begin{hint}
  \begin{enumerate}[a)]
  \item 
    За едната посока, разгледайте ДКА за $L$.
  \item
    За втората част, разгледайте $h:\Sigma\to\{a\}$ деф. като $(\forall b\in\Sigma)[h(b) = a]$.
    Докажете, че $h$ е поражда хомоморфизъм между $\Sigma^\star$ и $\{a\}^\star$.
    Тогава $h(L) = \{a^{\abs{\omega}} \mid \omega \in L\}$, а
    ние знаем, че регулярните езици са затворени относно хомоморфни образи.  
  \end{enumerate}
\end{hint}

\begin{problem}
  Вярно ли е, че:
  \begin{itemize}
  \item 
    $\{a^m \mid a^{m^2} \in L(p,q)\}$ е регулярен език ?
  \item
    $\{a^m \mid a^{2^m} \in L(p,q)\}$ е регулярен език ?
  \end{itemize}
\end{problem}


\begin{problem}
  За даден език $L$ над азбуката $\Sigma$, да разгледаме езиците:
  \begin{enumerate}[a)]
  \item
    $L' = \{\alpha \in \Sigma^\star \mid (\exists \beta\in\Sigma^\star)[\abs{\alpha} = 2\abs{\beta}\ \&\ \alpha\beta \in L]\}$;
  \item 
    $L'' = \{\alpha \in \Sigma^\star \mid (\exists \beta\in\Sigma^\star)[2\abs{\alpha} = \abs{\beta}\ \&\ \alpha\beta \in L]\}$;
  \item 
    $\frac{1}{3}(L) = \{\alpha \in \Sigma^\star \mid (\exists \beta,\gamma \in \Sigma^\star)[\abs{\alpha} = \abs{\beta} = \abs{\gamma}\ \&\ \alpha\beta\gamma \in L]\}$;
  \item
    $\frac{2}{3}(L) = \{\beta \in \Sigma^\star \mid (\exists \beta,\gamma \in \Sigma^\star)[\abs{\alpha} = \abs{\beta} = \abs{\gamma}\ \&\ \alpha\beta\gamma \in L]\}$;
  \item
    $\frac{3}{3}(L) = \{\gamma \in \Sigma^\star \mid (\exists \beta,\gamma \in \Sigma^\star)[\abs{\alpha} = \abs{\beta} = \abs{\gamma}\ \&\ \alpha\beta\gamma \in L]\}$;
  \item
    $\hat{L} = \{\alpha\gamma \in \Sigma^\star \mid (\exists \beta,\gamma \in \Sigma^\star)[\abs{\alpha} = \abs{\beta} = \abs{\gamma}\ \&\ \alpha\beta\gamma \in L]\}$;
  \item
    $\sqrt{L} = \{\alpha \mid (\exists \beta \in \Sigma^\star)[\abs{\beta} = \abs{\alpha}^2\ \&\ \alpha\beta \in L]\}$;
  \item
    $\log(L) = \{\alpha \mid (\exists \beta \in \Sigma^\star)[\abs{\beta} = 2^{\abs{\alpha}}\ \&\ \alpha\beta \in L]\}$;
  \end{enumerate}
  Проверете ако $L$ е регулярен, то кои от горните езици също са регулярни.
\end{problem}

\begin{problem}
  Да разгледаме езика
  \[L = \{\omega \in \{0,1\}^\star \mid \omega\text{ съдържа равен брой поднизове }01\text{ и }10\}.\]
  Например, $101 \in L$, защото съдържа по веднъж $10$ и $01$.
  $1010 \not\in  L$, защото съдържа два пъти $10$ и само веднъж $01$.
  Докажете, че $L$ е регулярен.
\end{problem}

\begin{problem}
  Нека $L$ е регулярен език над азбуката $\{a,b\}$. Докажете, че следните езици са регулярни:
  \begin{enumerate}[a)]
  \item 
    $\texttt{Diff}_1(L) \df \{\alpha \in L \mid (\exists \beta \in L)[|\alpha| = |\beta|\ \&\ \alpha \text{ се различава от $\beta$ в една позиция}]\}$;
  \item
    $\texttt{Diff}_n(L) \df \{\alpha \in L \mid (\exists \beta \in L)[n \leq |\alpha| = |\beta|\ \&\ \alpha \text{ се различава от $\beta$ в $n$ позиции}]\}$;
  \item
    $\texttt{Diff}(L) \df \{\alpha \in L \mid (\exists \beta \in L)[|\alpha| = |\beta|\ \&\ \alpha \text{ се различава от $\beta$ във всяка  позиция}]\}$;
  \end{enumerate}
\end{problem}
\begin{hint}
  Ако $L = \L(\A)$, то правим декартово произведение на $\A$ плюс флаг дали сме направили грешка.

  Не е ли по-лесно с индукция по построението на регулярните езици ?
\end{hint}


% \begin{problem}
%   \marginpar{(\cite{kozen}, стр. 75)}
%   Да фиксираме азбука само с един символ $\Sigma = \{a\}$.
%   Множеството $U$ е {\em породен от аритметична прогресия}, ако съществуват числа $q \geq 0$ и $p > 0$,
%   такива че $(\forall n \geq q)[n \in U\ \iff\ n+p \in U]$.
%   Докажете, че $L \subseteq \{a\}^\star$ е регулярен точно тогава, когато множеството $\{m \mid a^m \in L\}$
%   е породено от аритметична прогресия.
% \end{problem}
% \begin{hint}
%   Разгледайте КДА за $L$.
% \end{hint}

% \begin{hint}
%   \begin{itemize}
%   \item 
%     Докажете, че за всяко $p,q \in \Nat$, $L(p,q)$ е регулярен език.
%   \item
%     Докажете, че за крайно много $p_0,\dots,p_k$, $q_0,\dots,q_k$,
%     $\bigcup_{i \leq k}L(p_i,q_i)$ е регулярен език.
%   \item
%     С индукция по построението на регулярните езици, 
%     докажете, че ако $L$ е регулярен, то $L$ може да се представи
%     като крайно обединение на езици породени от аритметични прогресии.
%     Съществената част от доказателството се състои в следното:
%     \begin{itemize}
%     \item 
%       \marginpar{$L(p_1,q_1)\cdot L(p_2,q_2) = L(p_1+p_2,\mbox{НОД}(q_1,q_2))\setminus F$, където $F$ е крайно м-во, и ако $q_1 = q_2$, то $F = \emptyset$}
%       езикът $L(p_1,q_1) \cdot L(p_2,q_2)$ може да се представи като крайно обединение 
%       на езици породени от артиметични прогресии.
%     \item
%       езикът $L(p,q)^\star$ може да се представи като крайно обединение 
%       на езици породени от артиметични прогресии.
%     \end{itemize}
%   \end{itemize}
% \end{hint}



\begin{problem}
  Да разгледаме азбуката:
  \[\Sigma_3 = \left\{\begin{bmatrix} 0\\0\\0\end{bmatrix},\begin{bmatrix} 0\\0\\1\end{bmatrix},\begin{bmatrix} 0\\1\\0\end{bmatrix},\begin{bmatrix} 0\\1\\1\end{bmatrix},\dots,\begin{bmatrix} 1\\1\\1\end{bmatrix}\right\}.\]
  Докажете, че 
  $L = \left\{\begin{bmatrix} \alpha\\\beta\\\gamma\end{bmatrix} \in \Sigma^\star_3 \mid \ov{\alpha}_{(2)}+\ov{\beta}_{(2)} = \ov{\gamma}_{(2)}\right\}$
  е автоматен език.
\end{problem}
\ifhints
\begin{hint}
  По-удобно е да построим автомат $\A$, $\L(\A) = L^{\texttt{rev}}$.
  Да започнем с състоянието $q_{\scriptscriptstyle{=}}$, за което искаме да имаме свойството, че за произволно състояние $q$,
  \[\delta^\star(q, \tiny{ \begin{bmatrix} \alpha\\ \beta \\ \gamma\end{bmatrix} }) = q_{\scriptscriptstyle{=}}  \iff \ov{\alpha^{\texttt{rev}}}_{(2)} + \ov{\beta^{\texttt{rev}}}_{(2)} = \ov{\gamma^{\texttt{rev}}}_{(2)}.\]
  Понеже за $\ov{\varepsilon}_{(2)} + \ov{\varepsilon}_{(2)} = \ov{\varepsilon}_{(2)}$, състоянието $q_{\scriptscriptstyle{=}}$ ще бъде начално и финално за $\A$.
  \begin{itemize}
  \item 
    Нека $\ov{\alpha}_{(2)}+\ov{\beta}_{(2)} = \ov{\gamma}_{(2)}$. Тогава:
    \begin{itemize}
    \item 
      $\ov{0\alpha}_{(2)} + \ov{0\beta}_{(2)} = \ov{0\gamma}_{(2)}$. Следователно,
      \[\delta(q_{\scriptscriptstyle{=}},\tiny{ \begin{bmatrix} 0\\ 0 \\ 0\end{bmatrix} }) = q_{\scriptscriptstyle{=}}.\]
    \item
      $\ov{0\alpha}_{(2)} + \ov{1\beta}_{(2)} = \ov{1\gamma}_{(2)}$. Следователно,
      \[\delta(q_{\scriptscriptstyle{=}},\tiny{ \begin{bmatrix} 0\\ 0 \\ 0\end{bmatrix} }) = q_{\scriptscriptstyle{=}}.\]
    \item
      $\ov{1\alpha}_{(2)} + \ov{0\beta}_{(2)} = \ov{1\gamma}_{(2)}$. Следователно,
      \[\delta(q_{\scriptscriptstyle{=}},\tiny{ \begin{bmatrix} 1\\ 0 \\ 1\end{bmatrix} }) = q_{\scriptscriptstyle{=}}.\]
    \item
      $\ov{1\alpha}_{(2)} + \ov{1\beta}_{(2)} = \ov{10\gamma}_{(2)}$. Този случай е по-специален и трябва да бъде разгледан отделно.
      Трябва да отидем в състояние $q_1$, в което ще помним, че третия ред трябва да започва с $1$-ца. Затова имаме следния преход:
      \[\delta(q_{\scriptscriptstyle{=}},\tiny{ \begin{bmatrix} 1\\ 1 \\ 0\end{bmatrix} }) = q_1.\]
    \item
      За останалите $x \in \Sigma_3$ имаме, че
      \[\delta(q_{\scriptscriptstyle{=}},x) = q_{\texttt{err}},\]
      където $q_{\texttt{err}}$ е състоянието, от което не можем да излезем.
    \end{itemize}
  \item
    Горните разглеждания ни подсказват, че ще ни трябва и състояние $q_1$, за което искаме да е изпълнено свойството,
    че за произволно $q$,
    \[\delta^\star(q, \tiny{ \begin{bmatrix} \alpha\\ \beta \\ \gamma\end{bmatrix} }) = q_{\scriptscriptstyle{1}}  \iff \alpha^{\texttt{rev}}_{(2)} + \beta^{\texttt{rev}}_{(2)} = 1\gamma^{\texttt{rev}}_{(2)}.\]
    Да разгледаме сега случая $\alpha + \beta = 1\gamma$. Тогава:
    \begin{itemize}
    \item 
      Очевидно е, че $0\alpha + 0\beta = 1\gamma$. Следователно,
      \[\delta(q_1,\tiny{ \begin{bmatrix} 0\\ 0 \\ 1\end{bmatrix} }) = q_{\scriptscriptstyle{=}}.\]
    \item
      $1\alpha + 1\beta = 11\gamma$. Следователно,
      \[\delta(q_1,\tiny{ \begin{bmatrix} 1\\ 1 \\ 1\end{bmatrix} }) = q_{1}.\]
    \item
      $1\alpha + 0\beta = 10\gamma$ и $0\alpha + 1\beta = 10\gamma$. Следователно,
      \[\delta(q_1,\tiny{ \begin{bmatrix} 1\\ 0 \\ 0\end{bmatrix} }) = \delta(q_1,\tiny{ \begin{bmatrix} 0\\ 1 \\ 0\end{bmatrix} }) = q_{1}.\]
    \item
      За останалите $x \in \Sigma_3$ имаме, че
      \[\delta(q_{1},x) = q_{\texttt{err}}.\]
    \end{itemize}    
  \end{itemize}
\end{hint}
\fi

\begin{problem}
  Да разгледаме азбуката:
  \[\Sigma_2 = \left\{\begin{bmatrix} 0\\0\end{bmatrix},\begin{bmatrix} 0\\1\end{bmatrix},\begin{bmatrix} 1\\0\end{bmatrix},\begin{bmatrix} 1\\1\end{bmatrix}\right\}.\]
  Една дума над азбуката $\Sigma_2$ ни дава два реда от $0$-ли и $1$-ци, които ще разглеждаме като числа в двоична бройна система.
  Да разгледаме езиците:
  \begin{itemize}
  \item 
    $L_1 = \left\{\begin{bmatrix} \alpha\\ \beta \end{bmatrix} \in \Sigma^\star_2 \mid \ov{\alpha}_{(2)} < \ov{\beta}_{(2)}\right\}$;
  \item
    $L_2 = \left\{\begin{bmatrix} \alpha\\ \beta \end{bmatrix} \in \Sigma^\star_2 \mid 3(\ov{\alpha}_{(2)}) = \ov{\beta}_{(2)}\right\}$;
  \item
    $L_3 = \left\{\begin{bmatrix} \alpha\\ \beta \end{bmatrix} \in \Sigma^\star_2 \mid \alpha = \beta^{\rev}\right\}$;
  \end{itemize}
  Докажете, че  $L_1$ и $L_2$ са автоматни, а $L_3$ не е автоматен.
\end{problem}
\ifhints
\begin{hint}
  Ще построим автомат $\A = \FA$ за езика $L^{\rev}_1$.
  За улеснение, в рамките на тази задача ще пишем:
  \begin{itemize}
  \item 
    $\alpha \equiv \beta$, ако $\ov{\alpha^{\rev}}_{(2)} = \ov{\beta^{\rev}}_{(2)}$,
  \item
    $\alpha \prec \beta$, ако $\ov{\alpha^{\rev}}_{(2)} < \ov{\beta^{\rev}}_{(2)}$,
  \item
    $\alpha \succ \beta$, ако $\ov{\alpha^{\rev}}_{(2)} > \ov{\beta^{\rev}}_{(2)}$.
  \end{itemize}
  Нека състоянията на автомата са $Q = \{q_{\scriptscriptstyle{=}},q_{\scriptscriptstyle{<}},q_{\scriptscriptstyle{>}}\}$.
  Искаме да е изпълнено свойствата:
  \begin{itemize}
  \item 
    $\delta^\star(q_{\scriptscriptstyle{=}}, \scriptsize{\begin{bmatrix} \alpha\\ \beta\end{bmatrix}}) = q_{\scriptscriptstyle{=}}$ точно тогава, когато $\alpha \equiv \beta$;
  \item 
    $\delta^\star(q_{\scriptscriptstyle{=}}, \scriptsize{\begin{bmatrix} \alpha\\ \beta\end{bmatrix}}) = q_{\scriptscriptstyle{<}}$ точно тогава, когато $\alpha \prec \beta$;
  \item 
    $\delta^\star(q_{\scriptscriptstyle{=}}, \scriptsize{\begin{bmatrix} \alpha\\ \beta\end{bmatrix}}) = q_{\scriptscriptstyle{>}}$ точно тогава, когато $\alpha \succ \beta$.
  \end{itemize}
  Множеството от финални състояния ще бъде $F = \{q_{\scriptscriptstyle{<}}\}$, а началното състояние $\qstart = q_{\scriptscriptstyle{=}}$.
  За да дефинираме функцията на преходите, трябва да разгледа няколко случая, в зависимост от това какво е отношението между $\alpha$ и $\beta$.
  \begin{itemize}
  \item
    Нека $\alpha \equiv \beta$. Тогава:  
    \begin{itemize}
    \item 
      $\alpha 0 \equiv \beta 0$ и $\alpha 1 \equiv \beta 1$. Следователно,
      \[\delta(q_{\scriptscriptstyle{=}},\scriptsize{\begin{bmatrix} 0\\0\end{bmatrix}}) = \delta(q_{\scriptscriptstyle{=}},\scriptsize{\begin{bmatrix} 1\\1\end{bmatrix}}) = q_{\scriptscriptstyle{=}}.\]
    \item
      $\alpha 0 \prec \beta 1$. Следователно,
      \[\delta(q_{\scriptscriptstyle{=}},\scriptsize{\begin{bmatrix} 0\\1\end{bmatrix}}) = q_{\scriptscriptstyle{>}}.\]
    \item
      $\alpha 1 \succ \beta 0$. Следователно,
      \[\delta(q_{\scriptscriptstyle{=}},\scriptsize{\begin{bmatrix} 1\\0\end{bmatrix}}) = q_{\scriptscriptstyle{<}}.\]
    \end{itemize}
  \item 
    Нека $\alpha \prec \beta$. Тогава:
    \begin{itemize}
    \item 
      $\alpha 0 \prec \beta 0$, $\alpha 1 \prec \beta 1$, $\alpha 0 \prec \beta 1$. Следователно,
      \[\delta(q_{\scriptscriptstyle{<}},\scriptsize{\begin{bmatrix} 0\\0\end{bmatrix}}) = \delta(q_{\scriptscriptstyle{<}},\scriptsize{\begin{bmatrix} 1\\1\end{bmatrix}}) = \delta(q_{\scriptscriptstyle{<}},\scriptsize{\begin{bmatrix} 0\\1\end{bmatrix}}) = q_{\scriptscriptstyle{<}}.\]
    \item
      $\alpha 1 \succ \beta 0$. Следователно,
      \[\delta(q_{\scriptscriptstyle{<}},\scriptsize{\begin{bmatrix} 1\\0\end{bmatrix}}) = q_{\scriptscriptstyle{>}}.\]
    \end{itemize}
  \item
    Нека $\alpha \succ \beta$. Тогава:
    \begin{itemize}
    \item 
      $\alpha 0 \succ \beta 0$, $\alpha 1 \succ \beta 1$, $\alpha 1 \succ \beta 0$. Следователно,
      \[\delta(q_{\scriptscriptstyle{>}},\scriptsize{\begin{bmatrix} 0\\0\end{bmatrix}}) = \delta(q_{\scriptscriptstyle{>}},\scriptsize{\begin{bmatrix} 1\\1\end{bmatrix}}) = \delta(q_{\scriptscriptstyle{>}},\scriptsize{\begin{bmatrix} 1\\0\end{bmatrix}}) = q_{\scriptscriptstyle{>}}.\]
    \item
      $\alpha 0 \prec \beta 1$. Следователно,
      \[\delta(q_{\scriptscriptstyle{>}},\scriptsize{\begin{bmatrix} 0\\1\end{bmatrix}}) = q_{\scriptscriptstyle{<}}.\]
    \end{itemize}
  \end{itemize}
  Докажете, че за така дефинирания автомат $\A$, $\L(\A) = L^{\texttt{rev}}_1$.
\end{hint}
\fi


\newpage

За дума $\alpha = a_1 \cdot a_2 \cdots a_n$, нека да означим
\begin{itemize}
\item 
  $\alpha[i,j] = a_i \cdot a_{i+1} \cdots a_j$,
\item
  $\alpha[i..] = a_i \cdot a_{i+1} \cdots a_n$.
\end{itemize}

Нека също така да означим $\Sigma^{\geq k} = \{ \alpha \in \Sigma^\star \mid |\alpha| \geq k\}$.

\begin{problem}
  Нека $L$ е регулярен език.
  Тогава езиците
  \begin{itemize}
  \item
    $L_1 = \{ \alpha \in \Sigma^\star \mid (\exists i)(\exists j)[\ \alpha[i,j] \in L\ ]\}$;
  \item
    $L_2 = \{ \alpha \in \Sigma^\star \mid (\forall i)(\forall j)[\ \alpha[i,j] \in L\ ]\}$;
  \item 
    $L_3 = \{ \alpha \in \Sigma^\star \mid (\forall i)(\exists j)[\ \alpha[i,j] \in L\ ]\}$;
  \item
    $L_4 = \{ \alpha \in \Sigma^\star \mid (\exists i)(\forall j)[\ \alpha[i,j] \in L\ ]\}$.
  \end{itemize}
  също са регулярни.
\end{problem}
\begin{hint}
  Лесно се вижда, че
  \[L_1 = \Sigma^\star \cdot L \cdot \Sigma^\star,\]
  както и следното
  \[L_2 = \ov{\Sigma^\star \cdot \ov{L} \cdot \Sigma^\star}.\]
  Да обърнем внимание, че
  \begin{align*}
    \alpha \in \Sigma^\star \cdot L & \iff (\exists \beta \in \Sigma^\star)(\exists \gamma \in L)[ \alpha = \beta \cdot \gamma]\\
                                    & \iff (\exists j)[\ \alpha[j..] \in L\ ].
  \end{align*}
  Аналогично получаваме, че
    \begin{align*}
      \alpha \in L\cdot \Sigma^\star & \iff (\exists \gamma \in L)(\exists \beta \in \Sigma^\star)[ \alpha = \gamma \cdot \beta ]\\
                                     & \iff (\exists j)[\ \alpha[1,j] \in L\ ].
  \end{align*}
  Нека да разгледаме по-подробно следния език:
  \begin{align*}
    \ov{L}_3 & = \{\alpha \in \Sigma^\star \mid (\exists i)(\forall j)[\ \alpha[i,j] \not\in L\ ]\}\\
             & = \{\alpha \in \Sigma^\star \mid (\exists i)(\forall j)[\ \alpha[i,j] \in \ov{L}\ ]\}\\
             & = \{\alpha \in \Sigma^\star \mid (\forall j)[\ \alpha[1,j] \in \Sigma^\star \cdot \ov{L}\ ]\}\\
  \end{align*}
  Така получаваме, че:
  \begin{align*}
    L_3 & = \{\alpha \in \Sigma^\star \mid (\exists j)[\ \alpha[1,j] \not\in \Sigma^\star \cdot \ov{L}\ ]\}\\
        & = \{\alpha \in \Sigma^\star \mid (\exists j)[\ \alpha[1,j] \in \ov{\Sigma^\star \cdot \ov{L}}\ ]\}\\
        & = \{\alpha \in \Sigma^\star \mid \alpha \in (\ov{\Sigma^\star \cdot \ov{L}}) \cdot \Sigma^\star \ ]\}.
  \end{align*}
  Оттук заключваме, че
  \[L_3 = (\ov{\Sigma^\star \cdot \ov{L}}) \cdot \Sigma^\star.\]
  Сега лесно можем да съобразим, че
  \[ L_4 = \ov{ \ov{(\Sigma^\star \cdot L)} \cdot \Sigma^\star}.\]
\end{hint}

\begin{problem}
  Нека $L$ е регулярен език над азбуката $\Sigma$.
  За произволно естествено число $k$, докажете, че езикът
  \[L_k = \{ \alpha \in \Sigma^\star \mid (\forall i)(\exists j)[\ |\alpha[i..]| \geq k \implies \alpha[i,j] \in L\ ]\}\]
  е регулярен.
\end{problem}
\begin{hint}
  \begin{align*}
    L_k & = \{ \alpha \in \Sigma^\star \mid (\forall i)(\exists j)[\ |\alpha[i..]| \geq k \implies \alpha[i,j] \in L\ ]\}\\
        & = \{ \alpha \in \Sigma^\star \mid (\forall i)[\ |\alpha[i..]| \geq k \implies (\exists j)[\alpha[i,j] \in L]\ ]\}\\
        & = \{ \alpha \in \Sigma^\star \mid (\forall i)[\ \alpha[i..] \in \Sigma^{\geq k} \implies \alpha[i..] \in L \cdot \Sigma^\star\ ]\}.
  \end{align*}
  Тогава
  \begin{align*}
    \ov{L}_k & = \{ \alpha \in \Sigma^\star \mid (\exists i)[\ \alpha[i..] \in \Sigma^{\geq k}\ \&\ \alpha[i..] \not\in L \cdot \Sigma^\star\ ]\}\\
             & = \{ \alpha \in \Sigma^\star \mid (\exists i)[\ \alpha[i..] \in \Sigma^{\geq k}\ \&\ \alpha[i..] \in \ov{L \cdot \Sigma^\star}\ ]\}\\
             & = \{ \alpha \in \Sigma^\star \mid (\exists i)[\ \alpha[i..] \in \Sigma^{\geq k} \cap \ov{L \cdot \Sigma^\star} \ ]\}\\
             & = \{ \alpha \in \Sigma^\star \mid \alpha \in \Sigma^\star \cdot (\Sigma^{\geq k} \cap \ov{L \cdot \Sigma^\star}) \}.
  \end{align*}
  Сега е ясно, че
  \[L_k = \ov{\Sigma^\star \cdot (\Sigma^{\geq k} \cap \ov{L \cdot \Sigma^\star})}.\]
\end{hint}



% \begin{problem}
%   За думите от вида $\alpha = a_1 \cdot a_2 \cdots a_n$, да означим
%   \[\alpha[i,j,k] = a_i \cdot a_{i+1} \cdots a_j \cdot a_k \cdot a_{k+1} \cdots a_n.\]
  
%   Тогава езикът
%   \[L_5 = \{ \alpha \in \Sigma^\star \mid (\exists i)(\forall j)(\exists k)[\ \alpha[i,j,k] \in L\ ]\}\]
%   също е регулярен.
% \end{problem}


%%% Local Variables:
%%% mode: latex
%%% TeX-master: "../eai"
%%% End:
