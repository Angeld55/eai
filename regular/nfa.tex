\section{Недетерминирани крайни автомати}
\index{автомат!недетерминиран}
\begin{dfn}
  \marginpar{Въведени от Рабин и Скот \cite{rabin-scott}}
  \marginpar{За по-голяма яснота, често ще означаваме с $\N$ недетерминирани автомати}
  Недетерминиран краен автомат представлява
  \[\N = \NFA,\]
  \begin{itemize}
  \item
    $Q$ е крайно множество от състояния;
  \item
    $\Sigma$ е крайна азбука;
  \item
    $\Delta: Q\times\Sigma \to \Ps(Q)$ е функцията на преходите.
    \marginpar{Да напомним, че $\Ps(Q) = \{R\mid R\subseteq Q\}$, $\abs{\Ps(Q)} = 2^{\abs{Q}}$}
    Да обърнем внимание, че е възможно за някоя двойка $(q,a)$ да няма нито един преход в автомата.
    Това е възможно, когато $\Delta(q,a) = \emptyset$;
  \item
    $\qstart \in Q$ е началното състояние;
  \item
    $F\subseteq Q$ е множеството от финални състояния.
  \end{itemize}
\end{dfn}

Удобно е да разширим функцията на преходите $\Delta: Q\times\Sigma \to \Ps(Q)$ 
до функцията $\Delta^\star: \Ps(Q)\times\Sigma^\star \to \Ps(Q)$,
която дефинираме по следния начин:
\marginpar{Обърнете внимание, че $\Delta^\star(R,a) = \bigcup_{r\in R}\Delta(r,a)$.}
\begin{itemize}
\item 
  $\Delta^\star(R, \varepsilon) = R$, за произволно $R \subseteq Q$;
\item
  $\Delta^\star(R, \alpha x) = \bigcup_{p \in \Delta^\star(R,\alpha)} \Delta(p, x)$, за произволни $x \in \Sigma$, $\alpha \in \Sigma^\star$, $q\in Q$.
\end{itemize}

\begin{framed}
  \[\L(\N) \df \{\omega \in \Sigma^\star \mid \Delta^\star(\{\qstart\},\omega) \cap F \neq \emptyset \}.\]
\end{framed}

\begin{prop}
  За всеки две думи $\alpha,\beta \in \Sigma^\star$ и всяко $R \subseteq Q$,
  \[ \Delta^\star(R, \alpha\beta) = \Delta^\star( \Delta^\star(R,\alpha),\beta).\]
\end{prop}
\begin{proof}
  Отново индукция по дължината на $\beta$.
  \begin{itemize}
  \item
    Нека $\beta = \varepsilon$. Тогава:
    \begin{align*}
      \Delta^\star(R,\alpha\varepsilon) & = \Delta^\star(R,\alpha) \\
                                        & = \Delta^\star( \Delta^\star(R,\alpha), \varepsilon). & \comment\text{деф. на }\Delta^\star
    \end{align*}
  \item
    Да приемем, че твърдението е вярно за думи с дължина $n$.
  \item
    Нека $\beta = \gamma b$, където $|\gamma| = n$.
    \begin{align*}
      \Delta^\star(R, \alpha\gamma b) & = \bigcup_{q \in \Delta^\star(R,\alpha\gamma)} \Delta(q, b) & \comment\text{от деф. на }\Delta^\star\\
                                      & = \bigcup_{q \in \Delta^\star(\Delta^\star(R,\alpha),\gamma)} \Delta(q,b) & \comment\text{от И.П.}\\
                                      & = \Delta^\star( \Delta^\star(R,\alpha), \gamma b) & \comment\text{от деф. на }\Delta^\star.
    \end{align*}
    
  \end{itemize}
\end{proof}

\begin{problem}
  Докажете, че за произволни $R_i \subseteq Q$, където $i < k$, е изпълнено, че:
  \[\Delta^\star( \bigcup_{i<k} R_i, \alpha) = \bigcup_{i<k} \Delta^\star( R_i, \alpha).\]
\end{problem}

И тук е удобно да въведем бинарната релация $\vdash_\N$ над $Q\times\Sigma^\star$,
която ще ни казва как моментното описание на автомата $\N$ се променя след изпълнение на една стъпка:
\[(q,x\alpha) \vdash_\N (p,\alpha), \text{ ако } p \in \Delta(q,x).\]
\marginpar{Рефл. и транз. затваряне на една релация е разгледано в Глава \ref{ch:intro}}
Рефлексивното и транзитивно затваряне на $\vdash_\N$ ще означаваме с $\vdash^\star_\N$.
За да дадем точна дефиниция на $\vdash^\star_\N$, първо ще дефинираме релацията $\vdash^n_\N$, която
определя работата на автомата $\N$ за $n$ стъпки.
\begin{itemize}
\item 
  $(q,\alpha) \vdash^0_\N (q,\alpha)$, защото за $0$ стъпки се случва нищо.
\item
  Нека $\Delta(q,x) \ni q'$ и $(q',\alpha) \vdash^n_\N (p, \beta)$. Тогава
  $(q,x\alpha) \vdash^{n+1}_\N (p,\beta)$, защото за $n+1$ стъпки първо правим една стъпка 
  и отиваме в моментното описание $(q',\alpha)$ и след това правим още $n$ стъпки.
\end{itemize}
Сега можем да дефинираме $\vdash^\star_\N$ като:
\[(q,\alpha) \vdash^\star_\N (p,\beta) \dff (\exists n\in\Nat)[(q,\alpha) \vdash^n_\N (p,\beta)].\]
Друг начин да дефинираме релацията $\vdash^\star_\N$ е следния:
\[(q,\alpha\beta) \vdash^\star_\N (p, \beta) \iff p \in \Delta^\star(\{q\},\alpha).\]
Получаваме, че 
\[\L(\N) = \{\alpha\in\Sigma^\star \mid (\exists q \in F)[(\qstart,\alpha) \vdash^\star_\N (q,\varepsilon)]\}.\]


\begin{framed}
\begin{thm}[Рабин-Скот \cite{rabin-scott}]
  За всеки недетерминиран краен автомат $\N$ съществува еквивалентен на него детерминиран краен автомат $\D$, т.е. $\L(\N) = \L(\D)$.
\end{thm}
\end{framed}
\begin{hint}
  Нека $\N = \NFA$. Ще построим детерминиран автомат
  \[\D = (Q',\Sigma,\delta,\qstart',F'),\]
  за който $\L(\N) = \L(\D)$.
  Конструкцията е следната:
  \marginpar{Да отбележим, че детерминираният автомат $\D$ има не повече от $2^{\abs{Q}}$ на брой състояния $Q'$}
  \begin{itemize}
  \item
    $Q' = \Ps(Q)$. Някои от тези състояния може да са недостижими и следователно да са излишни, но в общия случай трябва да имаме
    всички подмножества на $Q$.
  \item
    За произволна буква $a\in\Sigma$ и произволно $R \subseteq Q$,
    \begin{align*}
      \delta(R,a) & = \{q\in Q\mid (\exists r\in R)[q\in\Delta(r,a)]\}\\
                  & = \bigcup_{r\in R}\Delta(r,a)\\
                  & = \Delta^\star(R,a).
    \end{align*}
  \item
    $\qstart' = \{\qstart\}$;
  \item
    $F' = \{R \subseteq Q \mid R\cap F \neq \emptyset\}$.
  \end{itemize}
  Ще докажем с индукция по дължината на думата $\alpha$, че
  \begin{equation}
    \label{eq:6}
    (\forall \alpha\in\Sigma^\star)(\forall R \subseteq Q)[\ \Delta^\star(R,\alpha) = \delta^\star(R,\alpha)\ ].
  \end{equation}

  \begin{itemize}
  \item
    Ако $|\alpha| = 0$, т.е. $\alpha = \varepsilon$, то е ясно от дефиницията на $\Delta^\star$ и $\delta^\star$.
  \item
    Да приемем, че (\ref{eq:6}) е изпълнено за думи $\alpha$ с дължина $n$.
  \item
    Нека сега $\alpha$ има дължина $n+1$, т.е. $\alpha = \beta a$, където $|\beta| =n$ и $a \in \Sigma$.
    Тогава:
    \begin{align*}
      \Delta^\star(R, \beta a) & = \Delta^\star(\Delta^\star(R,\beta), a) & \comment\text{деф. на }\Delta^\star\\
                               & = \Delta^\star(\delta^\star(R,\beta), a) & \comment\text{от И.П.}\\
                               & = \delta( \delta^\star(R, \beta), a) & \comment\text{деф. на }\delta\\
                               & = \delta^\star( R, \beta a) & \comment\text{деф. на }\delta^\star
    \end{align*}
  \end{itemize}
  Сега вече е лесно да съобразим, че
  \begin{align*}
    \omega \in \L(\D) & \iff \delta^\star(\{\qstart\},\omega) \in F' & \comment\text{деф. на }\L(\D)\\
                      & \iff \delta^\star(\{\qstart\},\omega) \cap F \neq \emptyset & \comment\text{деф. на }F'\\
                      & \iff \Delta^\star(\{\qstart\},\omega) \cap F \neq \emptyset & \comment\text{от (\ref{eq:6})}\\
                      & \iff \omega \in \L(\N) & \comment\text{деф. на }\L(\N).
  \end{align*}
\end{hint}

\begin{problem}
  За всеки НКА $\N$ съществува НКА $\N'$ с едно финално състояние, 
  за който $\L(\N) = \L(\N')$.
\end{problem}
\begin{hint}
  Вместо формална конструкция, да разгледаме един пример, който илюстрира идеята.
  \begin{figure}[H]
    \begin{subfigure}[b]{0.3\textwidth}
      \begin{tikzpicture}[framed,->,>=stealth,thick,node distance=45pt]
        \tikzstyle{every state}=[circle,minimum size=20pt,auto]
        \node[initial below,state]      (1) {$q_0$};
        \node[state,accepting]     [above right of=1] (2) {$q_1$};
        \node[state,accepting]     [below right of=1] (3) {$q_2$};
        \path
        (1) edge [bend left=15] node  [above] {$a$} (2)
        (2) edge [bend left=15] node  [right] {$b$} (1)
        % (2) edge [loop above] node  [above] {$a$} (2)
        (2) edge [bend left=15] node  [right] {$a$} (3)
        (3) edge [bend left=15] node  [below] {$a$} (1)
        (3) edge [loop below] node  [right] {$b$} (3);
        % (1) edge [bend right=15] node [below] {$b$} (3);
      \end{tikzpicture}
      \caption{автомат $\N$}
    \end{subfigure}
    \begin{subfigure}[b]{0.4\textwidth}
      \begin{tikzpicture}[framed,->,>=stealth,thick,node distance=45pt]
        \tikzstyle{every state}=[circle,minimum size=20pt,auto]
        \node[initial below,state]                        (1) {$q_0$};
        \node[state]     [above right of=1]         (2) {$q_1$};
        \node[state]     [below right of=1]         (3) {$q_2$};
        \node[state,accepting]     [right=3cm of 1] (4) {$f$};
        \path
        (1) edge [bend left=15] node  [above] {$a$} (2)
        % (2) edge [loop above] node  [above] {$a$} (2)
        (2) edge [bend left=15] node  [right] {$b$} (1)
        (2) edge [bend left=15] node  [right] {$a$} (3)
        (3) edge [loop below] node  [right] {$b$} (3)
        (3) edge [bend left=15] node  [below] {$a$} (1)
        (1) edge [dashed,bend left=15] node  [above] {$a$} (4)
        (2) edge [dashed,bend left=15] node  [above] {$a$} (4)
        (3) edge [dashed,bend right=15] node  [below] {$b$} (4);
        % (1) edge [bend right=15] node [below] {$b$} (3);
      \end{tikzpicture}
    \caption{автомат $\N'$, $\L(\N') = \L(\N)$}
  \end{subfigure}
\end{figure}  
За произволен автомат $\N$, формулирайте точно конструкцията на $\N'$ с едно финално състояние и докажете, че наистина $\L(\N) = \L(\N')$.
Обърнете внимание, че примера показва, че е възможно $\N$ да е детерминиран автомат, но полученият $\N'$ да бъде недетерминиран.
\end{hint}

\begin{problem}
  \marginpar{Това означава, че автоматните езици са затворени относно операцията $\texttt{rev}$. По-късно ще видим, че можем да дадем и друго доказателство на това твърдение, като направим индукция по построението на регулярните езици.}
  Докажете, че ако $L$ е автоматен език, то $L^{rev} = \{\omega^{rev} \mid \omega \in L\}$
  също е автоматен.
\end{problem}

\begin{lemma}
  \label{lem:automata-basic}
  Съществува детерминистичен автомат $\A = \FA$, който разпознава езика $L$, където
  \begin{itemize}
  \item
    $L = \emptyset$,
  \item
    $L = \{\varepsilon\}$, или
  \item
    $L = \{a\}$, за произволна буква $a\in\Sigma$.
  \end{itemize}
\end{lemma}
\begin{hint}
  \begin{figure}[H]
    \begin{subfigure}[b]{0.2\textwidth}
      \label{subf:a1}
      \begin{tikzpicture}[->,>=stealth,thick,node distance=35pt]
        \tikzstyle{every state}=[circle,minimum size=15pt,auto]
        \node[initial below,state]      (1) {$q_0$};
      \end{tikzpicture}
      \caption{$L(\A) = \emptyset$}
    \end{subfigure}
    \qquad
    \begin{subfigure}[b]{0.2\textwidth}
      \begin{tikzpicture}[->,>=stealth,thick,node distance=35pt]
        \tikzstyle{every state}=[circle,minimum size=15pt,auto]
        \node[initial below,state,accepting]      (1) {$q_0$};
      \end{tikzpicture}
      \caption{$\L(\A) = \{\varepsilon\}$}
    \end{subfigure}
    \qquad
    \begin{subfigure}[b]{0.3\textwidth}
      \begin{tikzpicture}[->,>=stealth,thick,node distance=45pt]
        \tikzstyle{every state}=[circle,minimum size=15pt,auto]
        \node[initial below,state]      (1)              {$q_0$};
        \node[accepting,state]    (2) [right of=1] {$q_1$};
        \path 
        (1) edge  node [above] {$a$} (2);
      \end{tikzpicture}
      \caption{$\L(\A) = \{a\}$}
    \end{subfigure}
  \end{figure}
\end{hint}

\begin{lemma}
  \label{lem:concat}
  Класът на автоматните езици е затворен относно операцията {\em конкатенация}.
  Това означава, че ако $L_1$ и $L_2$ са два произволни автоматни езика, то $L_1\cdot L_2$
  също е автоматен език.
\end{lemma}
\begin{proof}
  Нека са дадени детерминистичните автомати:
  \begin{itemize}
  \item
    $\A_1 = \pair{\Sigma,Q_1,\delta_1,\qstart',F_1}$, където $\L(\A_1) = L_1$;
  \item
    $\A_2 = \pair{\Sigma,Q_2,\delta_2,\qstart'', F_2}$, където $\L(\A_2) = L_2$.
  \end{itemize}
  Ще дефинираме автомата $\N = \NFA$ по такъв начин, че
  \[\L(\N) = L_1\cdot L_2 = \L(\A_1)\cdot\L(\A_2).\]
  \begin{itemize}
  \item
    $Q = Q_1 \cup Q_2$;
  \item
    $\qstart = \qstart''$;
  \item
    $F = \begin{cases}
      F_1 \cup F_2, & \text{ ако } \qstart'' \in F_2\\
      F_2,          & \text{ иначе}.
    \end{cases}$
  \item 
    $\Delta(q,a) = 
    \begin{cases}
      \{\delta_1(q,a)\},                      & \text{ ако }q\in Q_1\setminus F_1\ \&\ a\in\Sigma\\
      \{\delta_1(q,a), \delta_2(\qstart'',a)\}, & \text{ ако }q \in F_1\ \&\ a\in\Sigma\\
      \{\delta_2(q,a)\},                      & \text{ ако }q\in Q_2\ \&\ a\in\Sigma.
    \end{cases}$
  \end{itemize}
  
  Първо ще докажем, че $\L(\A_1)\cdot\L(\A_2) \subseteq \L(\N)$.
  За целта, нека разгледаме думата $\alpha \in \L(\A_1)$, където $\alpha = a_0a_1\cdots a_{l-1}$, и нека редицата $(q_i)^n_{i=0}$ описва приемащото изчисление на $\A_1$ върху думата $\alpha$.
  Това означава, че:
  \begin{itemize}
  \item
    $q_0 = \qstart'$;
  \item
    $q_{i+1} = \delta_1(q_i,a_i)$ за $i < n$;
  \item
    $q_n \in F_1$.
  \end{itemize}  
  Също така, нека разгледаме думата $\beta \in \L(\A_2)$, където $\beta = b_0b_1\cdots b_{m-1}$, и нека редицата $(p_i)^m_{i=0}$ описва приемащото изчисление на $\A_2$ върху думата $\beta$.
  Това означава, че:
  \begin{itemize}
  \item
    $p_0 = \qstart''$;
  \item
    $p_{i+1} = \delta_2(p_i,b_i)$ за $i < m$;
  \item
    $p_m \in F_2$.
  \end{itemize}  
  От конструкцията на $\N$ се вижда лесно, че $(q_i)^l_{i=0}$ описва изчисление на $\N$ върху $\alpha$ и
  $(p_i)^{m}_{i=0}$ описва изчисление на $\N$ върху $\beta$.
  Това означава, че:
  \begin{itemize}
  \item
    $q_{i+1} \in \Delta(q_i,a_i)$ за $i < n$;
  \item
    $p_{i+1} \in \Delta(p_i,b_i)$ за $i < m$.
  \end{itemize}
  Тогава:
  \[(q_0,\alpha\beta) \vdash^\star_\N (q_l,\beta)\text{ и } (p_0, \underbrace{b_0b_1\cdots b_{m-1}}_{\beta}) \vdash_\N (p_1,b_1\cdots b_{m-1}) \vdash^\star_{\N} (p_m,\varepsilon).\]
  Понеже $q_l \in F_1$, а $p_0 = \qstart''$, то от конструкцията на $\N$ следва, че
  \[(q_l, \underbrace{b_0b_1\cdots b_{m-1}}_{\beta}) \vdash_\N (p_1,b_1\cdots b_{m-1}),\]
  защото $\delta_2(\qstart'',b_0) \in \Delta(q_l,b_0)$.
  
  Обединявайки всичко това, получаваме, че:
  \begin{align*}
    (\qstart, \alpha\beta) & \vdash^\star_\N (q_l,\beta)\\
                           & \vdash_\N (p_1,b_1\cdots b_{m-1})\\
                           & \vdash^\star_\N(p_m,\varepsilon).
  \end{align*}
  Понеже $p_m \in F_2$, то $\alpha\beta \in \L(\N)$.
  

  Сега ще докажем, че $\L(\N) \subseteq \L(\A_1) \cdot \L(\A_2)$.
  За целта, нека разгледаме думата $\omega \in \L(\N)$, където $\omega = a_0a_1\cdots a_{n-1}$.
  Да разгледаме редицата от състояния $(q_i)^{n}_{i=0}$, която описва едно приемащо изчисление на $\N$ върху $\omega$.
  \marginpar{Възможно е да има и други редици от състояния $(p_i)^{n}_{i=0}$, които да описват приемащи изчисления на $\N$ върху $\omega$.}
  Това означава, че:
  \begin{itemize}
  \item
    $q_0 = \qstart$;
  \item
    $q_{i+1} \in \Delta(q_i,a_i)$ за $i < n$;
  \item
    $q_n \in F$.
  \end{itemize}

  \marginpar{Ако $q_n \in F_1$, то според конструкцията на $\N$, $\varepsilon \in \L(\A_2)$ и всяко състояние от $(q_i)^{n}_{i=0}$ принадлежи на $Q_1$ и оттам $\omega \in \L(\A_1)$.}
  Интересният случай е когато $q_n \in F_2$.
  Според конструкцията на $\N$, можем да разбием редицата от състояния $(q_i)^n_{i=0}$ на две непразни подредици:
  \begin{itemize}
  \item
    $(q_{i})^{l}_{i=0}$ - тези които са от $Q_1$,
  \item
    $(q_i)^{n}_{i=l+1}$ - тези, които са от $Q_2$.
  \end{itemize}
  Нека $\alpha = a_0a_1\cdots a_{l-1}$ и $\beta = a_la_{l+1}\cdots a_{n-1}$.
  Ясно е, че:
  \[(q_0,\alpha\beta) \vdash^\star_\N (q_{l}, a_{l}a_{l+1}\cdots a_{n-1}) \vdash_\N (q_{l+1},a_{l+1}\cdots a_{n-1}) \vdash^\star_\N (q_n,\varepsilon).\]
  От конструкцията на $\N$ следва, че редицата от състояния $(q_i)^{l}_{i=0}$ описва изчислението на $\A_1$ върху $\alpha$.
  \marginpar{Това е единственият начин да направим преход от състояние на $Q_1$ към състояние на $Q_2$.}
  Също така от конструкцията следва, че щом $q_{l+1} \in \Delta(q_l,a_l)$, то $q_l \in F_1$ и $\delta_2(\qstart'',a_l) = q_{l+1}$. Заключаваме, че:
  \begin{itemize}
  \item
    $(q_0, \alpha) \vdash^\star_{\A_1} (q_l,\varepsilon)$.
    Понеже $q_0 = \qstart'$ и $q_l \in F_1$, то $\alpha \in \L(\A_1)$.
  \item
    $(\qstart'', \beta) \vdash^\star_{\A_2} (q_n,\varepsilon)$.
    Понеже $q_n \in F_2$, то $\beta \in \L(\A_2)$.
  \end{itemize}
\end{proof}

\begin{figure}[H]
  \center
  \begin{subfigure}[b]{0.3\textwidth}
    \label{subf:a1}
    \begin{tikzpicture}[framed,->,>=stealth,thick,node distance=45pt]
      \tikzstyle{every state}=[circle,minimum size=15pt,auto]
      \node[initial,state,accepting]      (1) {$q'_0$};
      \node[state]                        (2) [right of=1] {$q_1$};
      \node[state]                        (3) [above right of=2] {$q_2$};
      \node[state,accepting]              (4) [below right of=2] {$q_3$};
      \path
      (1) edge [loop above] node [above] {$b$} (1)
      (1) edge node [above] {$a$} (2)
      (2) edge node [above] {$a$} (3)
      (3) edge [loop right] node [right] {$b$} (3)
      (2) edge node [below] {$b$} (4)
      (3) edge [bend right=30] node [above] {$a$} (1)
      (4) edge [bend right=15] node [right] {$a$} (3)
      (4) edge [bend left=30] node [below] {$b$} (1);
    \end{tikzpicture}
    \caption{автомат $\A_1$}
  \end{subfigure}
  \qquad
  \qquad
  \qquad
  \begin{subfigure}[b]{0.3\textwidth}
    \begin{tikzpicture}[framed,->,>=stealth,thick,node distance=45pt]
      \tikzstyle{every state}=[circle,minimum size=15pt,auto]
      \node[initial,state]                (1) {$q''_0$};
      \node[state]     [above right of=1] (2) {$q_4$};
      \node[state,accepting]     [below right of=1] (3) {$q_5$};
      \path
      (1) edge [bend left=15] node  [above] {$a$} (2)
      (2) edge [loop above] node [above] {$b$} (2)
      (2) edge [bend left=15] node  [right] {$a$} (3)
      (3) edge [loop right]  node [right] {$a,b$} (3)
      (1) edge [bend right=15] node [below] {$b$} (3);
    \end{tikzpicture}
    \caption{автомат $\A_2$}
  \end{subfigure}
\end{figure}

\begin{example}
    За да построим автомат, който разпознава конкатенацията на $\L(\N_1)$ и $\L(\N_2)$,
    трябва да свържем финалните състояния на $\N_1$ с изходящите от $s_2$ състояния на $\N_2$.
    
    \begin{figure}[H]
      \center
      % \begin{subfigure}[b]{0.3\textwidth}
      \begin{tikzpicture}[framed,->,>=stealth,thick,node distance=2cm]
        \tikzstyle{every state}=[circle,minimum size=15pt,auto]
        \node[initial,state]                      (1) {$q'_0$};
        \node[state] [right of=1]                 (2) {$q_1$};
        \node[state] [above right of=2]           (3) {$q_2$};
        \node[state] [below right of=2]           (4) {$q_3$};
        \node[state] [right=4cm of 1]             (5) {$q''_0$};
        \node[state] [above right of=5]           (6) {$q_4$};
        \node[state,accepting] [below right of=5] (7) {$q_5$};
        \path
        (1) edge [loop above] node [above] {$b$} (1)
        (1) edge node [above]                         {$a$} (2)
        (2) edge node [above]                         {$a$} (3)
        (2) edge node [below]                         {$b$} (4)
        (3) edge [loop right] node [right]            {$b$} (3)
        (6) edge [loop above] node [above]            {$b$} (6)
        (7) edge [loop right] node [right]            {$a,b$} (7)
        (3) edge [bend right=15] node [above]         {$a$} (1)
        (4) edge [bend left=15] node [below]          {$b$} (1)
        (4) edge [bend left=15] node [left]          {$a$} (3)
        (5) edge [bend left=15] node [below]          {$a$} (6)
        (6) edge [bend left=15] node [right]          {$a$} (7)
        (5) edge [bend right=15] node [above]         {$b$} (7)
        (1) edge [dashed, bend left=45] node [above]  {$a$} (6)
        (1) edge [dashed, bend right=45] node [below] {$b$} (7)
        (4) edge [dashed, bend left=30] node [above]  {$a$} (6)
        (4) edge [dashed, bend left=10] node [above]  {$b$} (7);
      \end{tikzpicture}
      \caption{$\L(\N) = \L(\A_1)\cdot\L(\A_2)$}
  \end{figure}  
  Обърнете внимание, че $\A_1$ и $\A_2$ са детерминирани автомати, но $\N$ е недетерминиран.
  Също така, в този пример се оказва, че вече $q''_0$ е недостижимо състояние, но в общия случай не можем да 
  го премахнем, защото може да има преходи влизащи в $q''_0$.
\end{example}

\begin{lemma}
  \label{lem:union}
  \marginpar{Второ доказателство на това твърдение.}
  Класът от автоматните езици е затворен относно операцията {\em обединение}.
\end{lemma}
\begin{hint}
  Нека са дадени детерминистичните автомати:
  \begin{itemize}
  \item 
    $\A_1 = \pair{\Sigma,Q_1,\delta_1,\qstart',F_1}$, като $L(\A_1) = L_1$;
  \item
    $\A_2=\pair{\Sigma,Q_2,\delta_2,\qstart'',F_2}$, като $L(\A_2) = L_2$.
  \end{itemize}
  Ще дефинираме автомата $\N=\NFA$, така че
  \[L(\N) = L(\A_1) \cup L(\A_2).\]
  \begin{itemize}
  \item 
    $Q = Q_1 \cup Q_2 \cup \{\qstart\}$, където $\qstart\not\in Q_1\cup Q_2$;
  \item
    $F = 
    \begin{cases}
      F_1 \cup F_2 \cup \{\qstart\}, & \text{ ако } \qstart' \in F_1 \vee \qstart'' \in F_2\\
      F_1 \cup F_2,            & \text{ иначе } 
    \end{cases}$
  \item
    $\Delta(q,a) = 
    \begin{cases}
      \{\delta_1(q,a)\},                       & \text{ ако } q\in Q_1\ \&\ a\in\Sigma\\
      \{\delta_2(q,a)\},                       & \text{ ако } q\in Q_2\ \&\  a\in\Sigma\\
      \{\delta_1(\qstart',a), \delta_2(\qstart'',a)\}, & \text{ ако } q = \qstart\ \&\ a \in\Sigma.
    \end{cases}$
  \end{itemize}
\end{hint}

\begin{example}
    За да построим автомат, който разпознава обединението на $\L(\A_1)$ и $\L(\A_2)$,
    трябва да добавим ново начално състояние, което да свържем с наследниците на началните състояния на $\A_1$ и $\A_2$.
    
    \begin{figure}[H]
      \center
      % \begin{subfigure}[b]{0.3\textwidth}
      \begin{tikzpicture}[framed,->,>=stealth,thick,node distance=2cm]
        \tikzstyle{every state}=[circle,minimum size=20pt,auto]
        \node[initial,state,accepting]      (0) {$q_0$};
        \node[state,accepting] [above right of=0]        (1) {$q'_0$};
        \node[state]    [right of=1]        (2) {$q_1$};
        \node[state]                        (3) [above right of=2] {$q_2$};
        \node[state,accepting]                        (4) [below right of=2] {$q_3$};
        \node[state]    [below right=2cm of 0] (5) {$q''_0$};
        \node[state]     [above right of=5] (6) {$q_4$};
        \node[state,accepting]     [below right of=5] (7) {$q_5$};
        \path
        (1) edge [loop above] node [above] {$b$} (1)
        (1) edge node [above]                  {$a$} (2)
        (2) edge node [above]                  {$a$} (3)
        (2) edge node [below]                  {$b$} (4)
        (3) edge [bend right=15] node [above]  {$a,b$} (1)
        (4) edge [bend left=15]  node [below]  {$b$} (1)
        (4) edge [bend right=15]  node [right]  {$a$} (3)
        (5) edge [bend left=15] node [below]   {$a$} (6)
        (6) edge [bend left=15] node  [right] {$a,b$} (7)
        (5) edge [bend right=15] node [above]  {$b$} (7)
        (7) edge [loop right] node [right] {$a,b$} (7)
        (0) edge [dashed, bend right=15] node [below]  {$a$} (2)
        (0) edge [dashed, bend left=15] node [above]  {$b$} (1)
        (0) edge [dashed, bend right=15] node [below]  {$a$} (6)
        (0) edge [dashed, bend right=45] node [below]  {$b$} (7);
      \end{tikzpicture}
      \caption{$\L(\N) = \L(\A_1)\cup\L(\A_2)$}
  \end{figure}  
  Обърнете внимание, че $\A_1$ и $\A_2$ са детерминирани автомати, но $\N$ е недетерминиран.
  Освен това, новото състояние $q_0$ трябва да бъде маркирано като финално, защото $q'_0$ е финално.
\end{example}

\begin{lemma}
  \label{lem:kleene-star}
  Класът от автоматните езици е затворен относно операцията {\em звезда на Клини}, т.е.
  ако $\L(\A) = L$, то съществува $\N$, за който $\L(\N) = L^\star$.
\end{lemma}
\begin{proof}
  Да разгледаме автомата $\A = \pair{\Sigma,Q,\qstart,\delta,F}$.
  Първо ще построим недетерминистичен автомат $\N = \pair{\Sigma, Q, \qstart, \Delta, F}$, такъв че
  \[\L(\N) = (\L(\A))^+.\]
  После ще построим недетерминистичен автомат $\N'$, за който
  \[\L(\N') = \L(\N) \cup \{\varepsilon\} = (\L(\A))^\star.\]

  Дефинираме функцията на преходите $\Delta$ на $\N$ по следния начин:
  За $q \in Q$ и $a \in \Sigma$, определяме функцията на преходите $\Delta$ по следния начин:
  \begin{align*}
    \Delta(q,a) \df
    \begin{cases}
      \{\delta(q,a)\}, & \text{ако } q \not\in F\\
      \{\delta(q,a), \delta(\qstart,a) \}, & \text{ако } q \in F.
    \end{cases}
  \end{align*}
    
  Нека $\alpha = a_0a_1\cdots a_{n-1} \in \L(\N)$.
  Това означава, че $(\qstart,\alpha) \vdash^\star_\N (f,\varepsilon)$ за някое $f \in F$.
  Нека редицата от състояния $(q_i)^n_{i=0}$ описва едно приемащо изчисление на $\N$ върху $\alpha$, т.е.
  \begin{itemize}
  \item
    $q_0 = \qstart$;
  \item
    $q_{i+1} \in \Delta(q_i,a_i)$;
  \item
    $q_n \in F$.
  \end{itemize}
  \marginpar{Възможно е $l = 0$ и съответно тази подредица да е празна.}
  Да разгледаме подредицата $(q_{i_j})^{l-1}_{j = 0}$ на $(q_i)^{n}_{i=0}$ съставена от тези състояния, за които
  \[ q_{i_j} \in F\ \&\ \delta(\qstart,a_{i_j}) = q_{i_j+1}.\]

  Да положим:
  \begin{align*}
    & \alpha_{i_0} \df a_0\cdots a_{i_0-1};\\
    & \alpha_{i_{j+1}} \df a_{i_j}\cdots a_{i_{j+1}-1}\text{, за }0\leq j \leq l-2\\
    & \alpha_{i_l} \df a_{i_{l-1}}\cdots a_{n-1}.
  \end{align*}
  Ясно е, че $\alpha = \alpha_{i_0}\alpha_{i_1}\cdots\alpha_{i_l}$.

  Сега можем да разбием изчислението на $\N$ върху $\alpha$ по следния начин:
  \begin{align*}
    (q_0,\alpha_{i_0}\alpha_{i_1}\cdots \alpha_{i_l}) & \vdash^\star_\N (q_{i_0}, \alpha_{i_1}\cdots \alpha_{i_l})\\
                                                           & \vdash^\star_\N (q_{i_1},\alpha_{i_2}\cdots \alpha_{i_l})\\
                                                           & \cdots\\
                                                           & \vdash^\star_\N (q_{i_{l-1}}, \alpha_{i_l})\\
                                                           & \vdash^\star_\N(q_n,\varepsilon).
  \end{align*}
  Да разгледаме само първата част на изчислението:
  \[\underbrace{(q_0,\alpha_{i_0}) \vdash^\star_\N (q_{i_0}, \varepsilon)}_{\text{само преходи от }\A}.\]
  Понеже $q_0 = \qstart$ и $q_{i_0} \in F$, то е ясно, че $\alpha_{i_0} \in \L(\A)$.
  
  За $j < l$, изчислението    
  \[(q_{i_j},\alpha_{i_{j+1}}) \vdash^\star_\N (q_{i_{j+1}},\varepsilon)\]
  може по-подробно да се запише и така:
  \[(q_{i_j},a_{i_j}a_{i_j+1}\cdots a_{i_{j+1}-1}) \vdash_\N \underbrace{(q_{i_j+1}, a_{i_j+1}\cdots a_{i_{j+1}-1})  \vdash^\star_\N (q_{i_{j+1}},\varepsilon)}_{\text{само преходи от }\A}.\]
  Понеже имаме, че $\delta(\qstart,a_{i_j}) = q_{i_j+1}$, то оттук следва, че:
  \[\underbrace{(\qstart,a_{i_j}a_{i_j+1}\cdots a_{i_{j+1}-1}) \vdash_\N (q_{i_j+1}, a_{i_j+1}\cdots a_{i_{j+1}-1})}_{\text{преход от }\A}  \vdash^\star_\A (q_{i_{j+1}},\varepsilon).\]
  Заключаваме, че
  \[(\qstart,\alpha_{i_{j+1}}) \vdash^\star_\A (q_{i_{j+1}},\varepsilon).\]
  Понеже $q_{i_{j+1}} \in F$, ведната следва, че $\alpha_{i_{j+1}} \in \L(\A)$.
  
  За последната част от изчислението, имаме следното:
  \[(q_{i_l-1}, \alpha_{i_l}) \vdash^\star_\N (q_n, \varepsilon),\]
  което може по-подробно да се запише така:
  \[(q_{i_{l-1}},a_{i_l}a_{i_l+1}\cdots a_{n-1}) \vdash_\N \underbrace{(q_{i_{l-1}+1}, a_{i_l+1}\cdots a_{n-1})  \vdash^\star_\N (q_n,\varepsilon)}_{\text{само преходи от }\A}.\]
  Понеже имаме, че $\delta(\qstart,a_{i_{l-1}}) = q_{i_{l-1}+1}$, то оттук следва, че:
  \[\underbrace{(\qstart,a_{i_l}a_{i_l+1}\cdots a_{n-1}) \vdash_\N (q_{i_{l-1}+1}, a_{i_l+1}\cdots a_{n-1})}_{\text{преход от }\A}  \vdash^\star_\A (q_n,\varepsilon).\]
  Заключаваме, че
  \[(\qstart,\alpha_{i_l}) \vdash^\star_\A (q_n,\varepsilon).\]
  Понеже $q_n \in F$, ведната следва, че $\alpha_{i_l} \in \L(\A)$.
  От всичко дотук заключваме, че $\alpha \in (\L(\A))^+$.


  За другата посока, нека $\alpha = a_0\cdots a_{n-1} \in (\L(\A))^+$.
  Нека дефинираме
  \begin{align*}
    & \alpha_{i_0} \df a_0\cdots a_{i_0-1}\\
    & \alpha_{i_{j+1}} \df a_{i_j}\cdots a_{i_{j+1}-1}\text{ и } q_{i_{j}} \df \delta(\qstart, a_{i_j})\text{, за }0\leq j \leq l-2\\
    & \alpha_{i_l} \df a_{i_{l-1}}\cdots a_{n-1} \text{ и } q_{i_l-1} \df \delta(\qstart, a_{i_{l-1}}),
  \end{align*}
  за които $\alpha_{i_j} \in \L(\A)$, за $j = 0,\dots,l$.
  
  Оттук получаваме следните изчисления:
  \begin{align*}
    &(\qstart, \alpha_{i_0}) \vdash^\star_\A (f_{0}, \varepsilon) & \comment\text{за някое }f_0 \in F\\
    &(\qstart, \alpha_{i_1}) \vdash_\A (q_{i_0}, a_{i_0+1}\cdots a_{i_1-1}) \vdash^\star_\A (f_{1}, \varepsilon) & \comment\text{за някое }f_1 \in F\\
    &(\qstart, \alpha_{i_2}) \vdash_\A (q_{i_1}, a_{i_1+1}\cdots a_{i_2-1}) \vdash^\star_\A (f_{2}, \varepsilon) & \comment\text{за някое }f_2 \in F\\
    &\dots\\
    &(\qstart, \alpha_{i_l}) \vdash_\A (q_{i_{l-1}}, a_{i_{l-1}+1}\cdots a_n) \vdash^\star_\A (f_{l}, \varepsilon). & \comment\text{за някое }f_l \in F
  \end{align*}

  За $1 \leq j < l$,
  понеже $\delta(\qstart,a_{i_j}) = q_{i_j}$ и $f_{j-1} \in F$, то според конструкцията на $\N$,
  \[q_{i_j} \in \Delta(f_{j-1}, a_{i_j}).\]
  Оттук следва, че имаме следното изчисление на $\N$ върху $\alpha$:
  \begin{align*}
    &(\qstart, \alpha_{i_0}) \vdash^\star_\N (f_{0}, \varepsilon) & \comment\text{за някое }f_0 \in F\\
    &(f_0, \alpha_{i_1}) \vdash_\N (q_{i_0}, a_{i_0+1}\cdots a_{i_1-1}) \vdash^\star_\N (f_{1}, \varepsilon) & \comment\text{за някое }f_1 \in F\\
    &(f_1, \alpha_{i_2}) \vdash_\N (q_{i_1}, a_{i_1+1}\cdots a_{i_2-1}) \vdash^\star_\N (f_{2}, \varepsilon) & \comment\text{за някое }f_2 \in F\\
    &\dots\\
    &(f_{l-1}, \alpha_{i_l}) \vdash_\N (q_{i_{l-1}}, a_{i_{l-1}+1}\cdots a_n) \vdash^\star_\N (f_{l}, \varepsilon). & \comment\text{за някое }f_l \in F
  \end{align*}
  Заключаваме, че $\alpha \in \L(\N)$.

  Така доказахме, че $\L(\N) = (\L(\A))^+$.
  Сега ще построим недетерминистичен автомат
  \[\N' = \pair{\Sigma,Q',\qstart',\Delta',F'},\]
  такъв че
  \[\L(\N') = \L(\N)\cup\{\varepsilon\} = (\L(\A))^+ \cup \{\varepsilon\} = (\L(\A))^\star.\]
  \begin{itemize}
  \item
    $Q' = Q \cup \{\qstart'\}$;
  \item
    $F' = F \cup \{\qstart'\}$;
  \item
    $\Delta'(\qstart',a) = \Delta(\qstart,a)$, за всяко $a \in \Sigma$;
  \item
    $\Delta'(q,a) = \Delta(q,a)$, за всяко $a \in \Sigma$ и $q\in Q$.
  \end{itemize}
  Лесно се съобразява, че $\L(\N') = \L(\N)\cup\{\varepsilon\}$.
\end{proof}

\begin{example}
  Нека да приложим конструкцията за да намерим автомат разпознаващ $\L(\A)^\star$.
  
  \begin{figure}[H]
    % \center
    \begin{subfigure}[b]{0.3\textwidth}
      \begin{tikzpicture}[framed,->,>=stealth,thick,node distance=55pt]
        \tikzstyle{every state}=[circle,minimum size=20pt,auto]
        \node[initial below,state] (1) {$q_0$};
        \node[state]               (2) [right of=1] {$q_1$};
        \node[state,accepting]     (3) [right of=2] {$q_2$};
        \node[state,accepting]     (4) [above of=2] {$q_3$};
        \node[state]               (5) [above of=3] {$q_4$};
        \path
        (1) edge [bend right=15] node [below] {$a$} (2)
        (1) edge [bend left=15]  node [above] {$b$} (4)
        (2) edge [bend left=15]  node [left] {$a$} (4)
        (2) edge [bend right=15] node [below] {$b$} (3)
        (5) edge [loop above]    node [above] {$a,b$} (5)
        (4) edge [bend left=15]  node [above] {$a,b$} (5)
        (3) edge [bend right=15] node [right] {$b$} (5)
        (3) edge [bend left=45]  node [below] {$a$} (1);
      \end{tikzpicture}
      \caption{автомат $\A$}
    \end{subfigure}
    \hspace{2cm}
    \begin{subfigure}[b]{0.5\textwidth}
      \begin{tikzpicture}[framed,->,>=stealth,thick,node distance=55pt]
        \tikzstyle{every state}=[circle,minimum size=20pt,auto]
        
        \node[initial below, state] (1) [below right of=0] {$q_0$};
        \node[state]                (2) [right of=1] {$q_1$};
        \node[state,accepting]      (3) [right of=2] {$q_2$};
        \node[state,accepting]      (4) [above of=2] {$q_3$};
        \node[state]                (5) [above of=3] {$q_4$};
        \path
        (1) edge [bend left=15] node [above] {$b$} (4)
        (1) edge [bend right=15] node [below] {$a$} (2)
        (2) edge [bend right=15] node [below] {$b$} (3)
        (2) edge [bend left=15]  node [left] {$a$} (4)
        (3) edge [bend left=45]  node [below] {$a$} (1)
        (3) edge [dashed, bend right=20] node [above] {$b$} (4)
        (4) edge [dashed,bend left=15] node [right] {$a$} (2)
        (4) edge [dashed, loop above] node {$b$} (4)
        (5) edge [loop above] node [above] {$a,b$} (5)
        (4) edge [bend left=15] node [above] {$a,b$} (5)
        (3) edge [bend right=15] node [right] {$b$} (5)
        (3) edge [dashed, bend right=15] node [above] {$a$} (2);        
      \end{tikzpicture}
      \caption{$\L(\N) = \L(\A)^+$}
    \end{subfigure}
  \end{figure}
    
  \marginpar{Лесно се вижда, че $\L(\A) = \{b\} \cup \{ab\}\cdot\{aba\}^\star$}
  След като построим автомат за езика $\L(\A)^+$, трябва да приложим
  конструкцията за обединение на автомата за езика $\L(\A)^+$ с автомата за езика $\{\varepsilon\}$.
  Защо трябва да добавим ново начално състояние $q_0'$?
  Да допуснем, че вместо това сме направили $q_0$ финално.
  Тогава има опасност да разпознаем повече думи. Например, думата $aba$ би се разпознала от този автомат,
  но $aba \not\in\L(\A)^\star$.
  
  \begin{figure}[H]
    \centering
      \begin{tikzpicture}[framed,->,>=stealth,thick,node distance=55pt]
        \tikzstyle{every state}=[circle,minimum size=20pt,auto]
        \node[initial, state, accepting] (0) [above left of=1] {$q'_0$};
        \node[state] (1) [below right of=0] {$q_0$};
        \node[state]                (2) [right of=1] {$q_1$};
        \node[state,accepting]      (3) [right of=2] {$q_2$};
        \node[state,accepting]      (4) [above of=2] {$q_3$};
        \node[state]                (5) [above of=3] {$q_4$};
        \path
        (0) edge [dashed, bend left=15] node [above] {$b$} (4)
        (0) edge [dashed, bend left=15] node [above] {$a$} (2)
        (1) edge [bend left=15] node [above] {$b$} (4)
        (1) edge [bend right=15] node [below] {$a$} (2)
        (2) edge [bend right=15] node [below] {$b$} (3)
        (2) edge [bend left=15]  node [left] {$a$} (4)
        (3) edge [bend left=45]  node [below] {$a$} (1)
        (3) edge [dashed, bend right=20] node [above] {$b$} (4)
        (4) edge [dashed,bend left=15] node [right] {$a$} (2)
        (4) edge [dashed, loop above] node {$b$} (4)
        (5) edge [loop above] node [above] {$a,b$} (5)
        (4) edge [bend left=15] node [above] {$a,b$} (5)
        (3) edge [bend right=15] node [right] {$b$} (5)
        (3) edge [dashed, bend right=15] node [above] {$a$} (2);        
      \end{tikzpicture}
      \caption{$\L(\N') = \L(\N) \cup \{\varepsilon\} = \L(\A)^\star$}    
  \end{figure}

\end{example}



%%% Local Variables:
%%% mode: latex
%%% TeX-master: "../eai"
%%% End:
