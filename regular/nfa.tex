\section{Недетерминирани крайни автомати}
\index{автомат!недетерминиран}
\begin{dfn}
  \marginpar{Въведени от Рабин и Скот \cite{rabin-scott}}
  \marginpar{За яснота, често ще означаваме с $\N$ недетерминирани автомати}
  Недетерминиран краен автомат представлява
  \[\N = \NFA,\]
  \begin{itemize}
  \item
    $Q$ е крайно множество от състояния;
  \item
    $\Sigma$ е крайна азбука;
  \item
    $\Delta: Q\times\Sigma \to \Ps(Q)$ е функцията на преходите.
    \marginpar{Да напомним, че $\Ps(Q) = \{R\mid R\subseteq Q\}$, $\abs{\Ps(Q)} = 2^{\abs{Q}}$}
    \marginpar{В \cite{sipser1} се позволяват $\epsilon$-преходи}
    Обърнете внимание, че тя е тотална.
    Ако за някоя двойка $(q,a)$ няма нито един преход в автомата, то 
    $\Delta(q,a) = \emptyset$;
  \item
    $\qstart \in Q$ е началното състояние;
  \item
    $F\subseteq Q$ е множеството от финални състояния.
  \end{itemize}
\end{dfn}

Удобно е да разширим функцията на преходите $\Delta: Q\times\Sigma \to \Ps(Q)$ 
до функцията $\Delta^\star: Q\times\Sigma^\star \to \Ps(Q)$ по следния начин:
\begin{itemize}
\item 
  $\Delta^\star(q, \varepsilon) = \{q\}$;
\item
  \marginpar{Съобразете, че $\Delta^\star(q, \alpha b) = \bigcup_{p \in \Delta^\star(q,\alpha)} \Delta(p,b)$}
  $\Delta^\star(q, b\alpha) = \bigcup_{p \in \Delta(q,b)} \Delta^\star(p, \alpha)$;
\end{itemize}

\begin{framed}
\begin{thm}[Рабин-Скот \cite{rabin-scott}]
  За всеки НKА $\N$ съществува еквивалентен на него ДКА $\D$, т.е. $\L(\N) = \L(\D)$.
\end{thm}
\end{framed}
\begin{hint}
  Нека $\N = \NFA$. Ще построим детерминиран автомат
  \[\D = (Q',\Sigma,\delta,\qstart',F'),\]
  за който $\L(\N) = \L(\D)$.
  Конструкцията е следната:
  \marginpar{Да отбележим, че детерминираният автомат $\D$ има не повече от $2^{\abs{Q}}$ на брой състояния $Q'$}
  \begin{itemize}
  \item
    $Q' = \Ps(Q)$;
  \item
    $\delta(R,a) = \{q\in Q\mid (\exists r\in R)[q\in\Delta(r,a)]\} = \bigcup_{r\in R}\Delta(r,a)$;
  \item
    $\qstart' = \{\qstart\}$;
  \item
    $F' = \{R \subseteq Q \mid R\cap F \neq \emptyset\}$.
  \end{itemize}
\end{hint}

% \begin{problem}
%   За дума $\alpha = a_1a_2\cdots a_n$, дефинираме $\alpha^R = a_na_{n-1}\cdots a_1$.
%   \marginpar{Индукция по $\abs{\beta}$.}
%   Докажете, че
%   \[(\forall \alpha,\beta\in\Sigma^\star)[(\alpha\beta)^R = \beta^R\alpha^R].\]
% \end{problem}

\begin{problem}
  За всеки НКА $\N$ съществува НКА $\N'$ с едно финално състояние, 
  за който $\L(\N) = \L(\N')$.
\end{problem}
\begin{hint}
  Вместо формална конструкция, да разгледаме един пример, който илюстрира идеята.
  \begin{figure}[H]
    \begin{subfigure}[b]{0.3\textwidth}
      \begin{tikzpicture}[framed,->,>=stealth,thick,node distance=45pt]
        \tikzstyle{every state}=[circle,minimum size=20pt,auto]
        \node[initial below,state]      (1) {$s$};
        \node[state,accepting]     [above right of=1] (2) {$q_1$};
        \node[state,accepting]     [below right of=1] (3) {$q_2$};
        \path
        (1) edge [bend left=15] node  [above] {$a$} (2)
        (2) edge [bend left=15] node  [right] {$b$} (1)
        % (2) edge [loop above] node  [above] {$a$} (2)
        (2) edge [bend left=15] node  [right] {$a$} (3)
        (3) edge [bend left=15] node  [below] {$a$} (1)
        (3) edge [loop below] node  [right] {$b$} (3);
        % (1) edge [bend right=15] node [below] {$b$} (3);
      \end{tikzpicture}
      \caption{автомат $\N$}
    \end{subfigure}
    \begin{subfigure}[b]{0.4\textwidth}
      \begin{tikzpicture}[framed,->,>=stealth,thick,node distance=45pt]
        \tikzstyle{every state}=[circle,minimum size=20pt,auto]
        \node[initial,state]      (1) {$s$};
        \node[state]     [above right of=1] (2) {$q_1$};
        \node[state]     [below right of=1] (3) {$q_2$};
        \node[state,accepting]     [right=3cm of 1] (4) {$f$};
        \path
        (1) edge [bend left=15] node  [above] {$a$} (2)
        % (2) edge [loop above] node  [above] {$a$} (2)
        (2) edge [bend left=15] node  [right] {$b$} (1)
        (2) edge [bend left=15] node  [right] {$a$} (3)
        (3) edge [loop below] node  [right] {$b$} (3)
        (3) edge [bend left=15] node  [below] {$a$} (1)
        (1) edge [dashed,bend left=15] node  [above] {$a$} (4)
        (2) edge [dashed,bend left=15] node  [above] {$a$} (4)
        (3) edge [dashed,bend right=15] node  [below] {$b$} (4);
        % (1) edge [bend right=15] node [below] {$b$} (3);
      \end{tikzpicture}
    \caption{автомат $\N'$, $\L(\N') = \L(\N)$}
  \end{subfigure}
\end{figure}  
За произволен автомат $\N$, формулирайте точно конструкцията на $\N'$ с едно финално състояние и докажете, че наистина $\L(\N) = \L(\N')$.
Обърнете внимание, че примера показва, че е възможно $\N$ да е детерминиран автомат, но полученият $\N'$ да бъде недетерминиран.
\end{hint}

\begin{problem}
  \marginpar{Нека $\A$, $L = \L(\A)$, е само с едно финално състояние. }
  Докажете, че ако $L$ е автоматен език, то $L^{rev} = \{\omega^{rev} \mid \omega \in L\}$
  също е автоматен.
\end{problem}

\begin{lemma}
  \label{lem:automata-basic}
  Съществува НКА $\N = \NFA$, който разпознава езика $L(r)$, 
  където $r = \emptyset$, $r = \varepsilon$ или $r = a$, за $a\in \Sigma$.
\end{lemma}
\begin{hint}
  \begin{figure}[H]
    \begin{subfigure}[b]{0.2\textwidth}
      \label{subf:a1}
      \begin{tikzpicture}[->,>=stealth,thick,node distance=35pt]
        \tikzstyle{every state}=[circle,minimum size=15pt,auto]
        \node[initial,state]      (1) {$s$};
      \end{tikzpicture}
      \caption{$L(\emptyset)$}
    \end{subfigure}
    \qquad
    \begin{subfigure}[b]{0.2\textwidth}
      \begin{tikzpicture}[->,>=stealth,thick,node distance=35pt]
        \tikzstyle{every state}=[circle,minimum size=15pt,auto]
        \node[initial,state,accepting]      (1) {$s$};
      \end{tikzpicture}
      \caption{$L(\varepsilon)$}
    \end{subfigure}
    \qquad
    \begin{subfigure}[b]{0.3\textwidth}
      \begin{tikzpicture}[->,>=stealth,thick,node distance=35pt]
        \tikzstyle{every state}=[circle,minimum size=15pt,auto]
        \node[initial,state]      (1)              {$s$};
        \node[accepting,state]    (2) [right of=1] {$q$};
        \path 
        (1) edge  node [above] {$a$} (2);
      \end{tikzpicture}
      \caption{$L(a)$}
    \end{subfigure}
  \end{figure}
\end{hint}

\begin{lemma}
  \label{lem:concat}
  Класът на автоматните езици е затворен относно операцията {\em конкатенация}.
  Това означава, че ако $L_1$ и $L_2$ са два произволни автоматни езика, то $L_1\cdot L_2$
  също е автоматен език.
\end{lemma}
\begin{proof}
  Нека са дадени автоматите:
  \begin{itemize}
  \item
    $\N_1 = \NFAn{1}$, като $\L(\N_1) = L_1$;
  \item
    $\N_2 = \NFAn{2}$, като $\L(\N_2) = L_2$.
  \end{itemize}
  Ще дефинираме автомата $\N = \NFA$ като
  \[\L(\N) = L_1\cdot L_2 = \L(\N_1)\cdot\L(\N_2).\]
  \begin{itemize}
  \item
    $Q = Q_1 \cup Q_2$;
  \item
    $s = s_1$;
  \item
    $F = 
    \begin{cases}
      F_1 \cup F_2, & \text{ ако } s_2 \in F_2\\
      F_2,          & \text{ иначе}.
    \end{cases}$
  \item 
    $\Delta(q,a) = 
    \begin{cases}
      \Delta_1(q,a),                      & \text{ ако }q\in Q_1\setminus F_1\ \&\ a\in\Sigma\\
      \Delta_2(q,a),                      & \text{ ако }q\in Q_2\ \&\ a\in\Sigma\\
      \Delta_1(q,a) \cup \Delta_2(s_2,a), & \text{ ако }q \in F_1\ \&\ a\in\Sigma.
    \end{cases}$
  \end{itemize}
\end{proof}

\begin{figure}[H]
  \center
  \begin{subfigure}[b]{0.3\textwidth}
    \label{subf:a1}
    \begin{tikzpicture}[framed,->,>=stealth,thick,node distance=45pt]
      \tikzstyle{every state}=[circle,minimum size=15pt,auto]
      \node[initial,state,accepting]      (1) {$s_1$};
      \node[state]                        (2) [right of=1] {$q_1$};
      \node[state]                        (3) [above right of=2] {$q_2$};
      \node[state,accepting]              (4) [below right of=2] {$q_3$};
      \path
      (1) edge node [above] {$a$} (2)
      (2) edge node [above] {$a$} (3)
      (2) edge node [below] {$b$} (4)
      (3) edge [bend right=30] node [above] {$a$} (1)
      (4) edge [bend left=30] node [below] {$b$} (1);
    \end{tikzpicture}
    \caption{автомат $\N_1$}
  \end{subfigure}
  \qquad
  \qquad
  \qquad
  \begin{subfigure}[b]{0.3\textwidth}
    \begin{tikzpicture}[framed,->,>=stealth,thick,node distance=45pt]
      \tikzstyle{every state}=[circle,minimum size=15pt,auto]
      \node[initial,state]      (1) {$s_2$};
      \node[state]     [above right of=1] (2) {$q_4$};
      \node[state,accepting]     [below right of=1] (3) {$q_5$};
      \path
      (1) edge [bend left=15] node  [above] {$a$} (2)
      (2) edge [bend left=15] node  [right] {$a$} (3)
      (1) edge [bend right=15] node [below] {$b$} (3);
    \end{tikzpicture}
    \caption{автомат $\N_2$}
  \end{subfigure}
\end{figure}

\begin{example}
    За да построим автомат, който разпознава конкатенацията на $\L(\N_1)$ и $\L(\N_2)$,
    трябва да свържем финалните състояния на $\N_1$ с изходящите от $s_2$ състояния на $\N_2$.
    
    \begin{figure}[H]
      \center
      % \begin{subfigure}[b]{0.3\textwidth}
      \begin{tikzpicture}[framed,->,>=stealth,thick,node distance=2cm]
        \tikzstyle{every state}=[circle,minimum size=15pt,auto]
        \node[initial,state]                      (1) {$s_1$};
        \node[state] [right of=1]                 (2) {$q_1$};
        \node[state] [above right of=2]           (3) {$q_2$};
        \node[state] [below right of=2]           (4) {$q_3$};
        \node[state] [right=4cm of 1]             (5) {$s_2$};
        \node[state] [above right of=5]           (6) {$q_4$};
        \node[state,accepting] [below right of=5] (7) {$q_5$};
        \path
        (1) edge node [above]                         {$a$} (2)
        (2) edge node [above]                         {$a$} (3)
        (2) edge node [below]                         {$b$} (4)
        (3) edge [bend right=15] node [above]         {$a$} (1)
        (4) edge [bend left=15] node [below]          {$b$} (1)
        (5) edge [bend left=15] node [below]          {$a$} (6)
        (6) edge [bend left=15] node [right]          {$a$} (7)
        (5) edge [bend right=15] node [above]         {$b$} (7)
        (1) edge [dashed, bend left=45] node [above]  {$a$} (6)
        (1) edge [dashed, bend right=45] node [below] {$b$} (7)
        (4) edge [dashed, bend left=45] node [above]  {$a$} (6)
        (4) edge [dashed, bend left=10] node [above]  {$b$} (7);
      \end{tikzpicture}
      \caption{$\L(\N) = \L(\N_1)\cdot\L(\N_2)$}
  \end{figure}  
  Обърнете внимание, че $\N_1$ и $\N_2$ са детерминирани автомати, но $\N$ е недетерминиран.
  Също така, в този пример се оказва, че вече $s_2$ е недостижимо състояние, но в общия случай не можем да 
  го премахнем, защото може да има преходи влизащи в $s_2$.
\end{example}


\begin{lemma}
  \label{lem:union}
  Класът от автоматните езици е затворен относно операцията {\em обединение}.
\end{lemma}
\begin{hint}
  Нека са дадени автоматите:
  \begin{itemize}
  \item 
    $\N_1 = \NFAn{1}$, като $L(\N_1) = L_1$;
  \item
    $\N_2=\NFAn{2}$, като $L(\N_2) = L_2$.
  \end{itemize}
  Ще дефинираме автомата $\N=\NFA$, така че
  \[L(\N) = L(\N_1) \cup L(\N_2).\]
  \begin{itemize}
  \item 
    $Q = Q_1 \cup Q_2 \cup \{s\}$;
  \item
    $F = 
    \begin{cases}
      F_1 \cup F_2 \cup \{s\}, & \text{ ако } s_1 \in F_1 \vee s_2 \in F_2\\
      F_1 \cup F_2,            & \text{ иначе } 
    \end{cases}$
  \item
    $
    \Delta(q,a) = 
    \begin{cases}
      \Delta_1(q,a),                       & \text{ ако } q\in Q_1\ \&\ a\in\Sigma\\
      \Delta_2(q,a),                       & \text{ ако } q\in Q_2\ \&\  a\in\Sigma\\
      \Delta_1(s_1,a) \cup \Delta_2(s_2,a), & \text{ ако } q = s\ \&\  a \in\Sigma.
    \end{cases}
    $
  \end{itemize}
\end{hint}
\begin{remark}
  В началното състояние на новопостроения автомат $\N$ не влизат ребра.
\end{remark}


\begin{example}
    За да построим автомат, който разпознава обединението на $\L(\N_1)$ и $\L(\N_2)$,
    трябва да добавим ново начално състояние, което да свържем с наследниците на началните състояния на $\N_1$ и $\N_2$.
    
    \begin{figure}[H]
      \center
      % \begin{subfigure}[b]{0.3\textwidth}
      \begin{tikzpicture}[framed,->,>=stealth,thick,node distance=2cm]
        \tikzstyle{every state}=[circle,minimum size=20pt,auto]
        \node[initial,state,accepting]      (0) {$s$};
        \node[state,accepting]    [above right of=0]        (1) {$s_1$};
        \node[state]    [right of=1]        (2) {$q_1$};
        \node[state]                        (3) [above right of=2] {$q_2$};
        \node[state,accepting]                        (4) [below right of=2] {$q_3$};
        \node[state]    [below right=2cm of 0] (5) {$s_2$};
        \node[state]     [above right of=5] (6) {$q_4$};
        \node[state,accepting]     [below right of=5] (7) {$q_5$};
        \path
        (1) edge node [above]                  {$a$} (2)
        (2) edge node [above]                  {$a$} (3)
        (2) edge node [below]                  {$b$} (4)
        (3) edge [bend right=15] node [above]  {$a$} (1)
        (4) edge [bend left=15]  node [below]  {$b$} (1)
        (5) edge [bend left=15] node [below]   {$a$} (6)
        (6) edge [bend left=15] node  [right] {$a$} (7)
        (5) edge [bend right=15] node [above]  {$b$} (7)
        (0) edge [dashed, bend right=15] node [below]  {$a$} (2)
        (0) edge [dashed, bend right=15] node [below]  {$a$} (6)
        (0) edge [dashed, bend right=45] node [below]  {$b$} (7);
      \end{tikzpicture}
      \caption{$\L(\N) = \L(\N_1)\cup\L(\N_2)$}
  \end{figure}  
  Обърнете внимание, че $\N_1$ и $\N_2$ са детерминирани автомати, но $\N$ е недетерминиран.
  Освен това, новото състояние $s$ трябва да бъде маркирано като финално, защото $s_1$ е финално.
\end{example}

\begin{lemma}
  \label{lem:kleene-star}
  Класът от автоматните езици е затворен относно операцията {\em звезда на Клини}, т.е.
  ако $\L(\N) = L$, то съществува $\hat{\N}$, за който $\L(\hat{\N}) = L^\star$.
\end{lemma}
\begin{proof}
  Нека е даден автомата $\N = \NFA$, за който е изпънено, че
  $\L(\N) = \L(\mathbf{r})$.
  Първата стъпка е да построим $\N_1 = \NFAn{1}$, такъв че 
  \[\L(\N_1) = \bigcup_{n\geq 1} (\L(\N))^n = \bigcup_{n\geq 1} (\L(\mathbf{r}))^n = \L(\mathbf{r^+}).\]
  Това можем да направим по следния начин:
  \begin{itemize}
  \item
    $Q_1 = Q$;
  \item
    $s_1 = s$;
  \item
    $F_1 = F$;
  \item
    Функцията на преходите $\Delta_1$ включва всичко от $\Delta$ като добавя нови преходи към наследниците на началното състояние $s$ на $\N$. Дефиницията на $\Delta_1$ може да се запише така:
    \[\Delta_1(q,a) = 
    \begin{cases}
      \Delta(q,a), & \text{ ако } q\in Q\setminus F, a \in \Sigma\\
      \Delta(q,a) \cup \Delta(s,a), & \text{ ако } q\in F, a\in\Sigma.
    \end{cases}\]
  \end{itemize}
  Накрая строим желания автомат $\hat\N$, така че $\L(\hat\N) = \{\varepsilon\} \cup \L(\N_1)$.
  Ние знаем как да построим $\hat\N$, защото можем да посторим автомат $\N_0$ за езика $\{\varepsilon\}$
  (\Lem{automata-basic}) и тогава $\hat\N$ се получава от конструкцията за обединение от \Lem{union} приложена върху $\N_0$ и $\N_1$.
\end{proof}


\begin{example}
  Нека да приложим конструкцията за да намерим автомат разпознаващ $\L(\N)^\star$.
  
  \begin{figure}[H]
    % \center
    \begin{subfigure}[b]{0.3\textwidth}
      \begin{tikzpicture}[framed,->,>=stealth,thick,node distance=45pt]
        \tikzstyle{every state}=[circle,minimum size=20pt,auto]
        \node[initial,state]      (1) {$s$};
        \node[state]              (2) [right of=1] {$q_1$};
        \node[state,accepting]    (3) [right of=2] {$q_2$};
        \node[state,accepting]    (4) [above of=2] {$q_3$};
        \path
        (1) edge node [above] {$a$} (2)
        (1) edge [bend left=15] node [above] {$b$} (4)
        (2) edge node [above] {$b$} (3)
        (3) edge [bend left=45] node [below] {$a$} (1);
      \end{tikzpicture}
      \caption{автомат $\N$}
    \end{subfigure}
    \hspace{2cm}
    \begin{subfigure}[b]{0.5\textwidth}
      \begin{tikzpicture}[framed,->,>=stealth,thick,node distance=45pt]
        \tikzstyle{every state}=[circle,minimum size=20pt,auto]
        % \node[initial above,state,accepting] (0) {$s'$};
        \node[initial, state]                (1) [below right of=0] {$s$};
        \node[state]                         (2) [right of=1] {$q_1$};
        \node[state,accepting]               (3) [right of=2] {$q_2$};
        \node[state,accepting]               (4) [above of=2] {$q_3$};
        \path
        % (0) edge [dashed, bend left=15] node [above] {$a$} (2)
        % (0) edge [dashed, bend left=15] node [above] {$b$} (4)
        (1) edge [bend left=15] node [above] {$b$} (4)
        (1) edge node [above] {$a$} (2)
        (2) edge node [below] {$b$} (3)
        (3) edge [bend left=45] node [below] {$a$} (1)
        (3) edge [dashed, bend right=45] node [above] {$b$} (4)
        (4) edge [dashed] node [left] {$a$} (2)
        (4) edge [dashed, loop above] node {$b$} (4)
        (3) edge [dashed, bend right=45] node [above] {$a$} (2);        
      \end{tikzpicture}
      \caption{$\L(\N_1) = \L(\N)^+$}
    \end{subfigure}
  \end{figure}
    
  \marginpar{Лесно се вижда, че $\L(\N) = \{b\} \cup \{aba\}^\star \cdot ab$}
  След като построим автомат за езика $\L(\N)^+$, трябва да приложим
  конструкцията за обединение на автомата за езика $\L(\N)^+$ с автомата за езика $\{\varepsilon\}$.
  Защо трябва да добавим ново начално състояние $s'$?
  Да допуснем, че вместо това сме направили $s$ финално.
  Тогава има опасност да разпознаем повече думи. Например, думата $aba$ би се разпознала от този автомат,
  но $aba \not\in\L(\N)^\star$.
  
  \begin{figure}[H]
    \centering
    % \begin{subfigure}[b]{0.5\textwidth}
      \begin{tikzpicture}[framed,->,>=stealth,thick,node distance=45pt]
        \tikzstyle{every state}=[circle,minimum size=20pt,auto]
        \node[initial, state, accepting]     (0) {$s'$};
        \node[state]                         (1) [below right of=0] {$s$};
        \node[state]                         (2) [right of=1] {$q_1$};
        \node[state, accepting]              (3) [right of=2] {$q_2$};
        \node[state, accepting]              (4) [above of=2] {$q_3$};
        \path
        (0) edge [dashed, bend left=15] node [above] {$a$} (2)
        (0) edge [dashed, bend left=15] node [above] {$b$} (4)
        (1) edge [bend left=15] node [above] {$b$} (4)
        (1) edge node [above] {$a$} (2)
        (2) edge node [below] {$b$} (3)
        (3) edge [bend left=45] node [below] {$a$} (1)
        (3) edge [dashed, bend right=45] node [above] {$b$} (4)
        (4) edge [dashed] node [left] {$a$} (2)
        (4) edge [dashed, loop above] node {$b$} (4)
        (3) edge [dashed, bend right=45] node [above] {$a$} (2);        
      \end{tikzpicture}
      \caption{$\L(\hat\N) = \L(\N)^\star = \L(\N)^+ \cup \{\varepsilon\}$}    
  \end{figure}

\end{example}



%%% Local Variables:
%%% mode: latex
%%% TeX-master: "../eai"
%%% End:
