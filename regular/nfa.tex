\section{Недетерминирани крайни автомати}
\index{автомат!недетерминиран (НКА)}
\begin{definition}
  \mynote{Въведени от Рабин и Скот \cite{rabin-scott}. За по-голяма яснота, често ще означаваме с $\N$ недетерминирани автомати, като ще запазим $\A$ за детерминираните.}
  \index{Рабин}
  \index{Скот}
  Недетерминиран краен автомат представлява петорка
  \[\N = \NFA,\]
  \begin{itemize}
  \item
    $Q$ е крайно множество от състояния;
  \item
    $\Sigma$ е крайна азбука;
  \item
    $\Delta: Q\times\Sigma \to \Ps(Q)$ е функцията на преходите.
    \mynote{Да напомним, че $\Ps(Q) \df \{R\mid R\subseteq Q\}$, $\abs{\Ps(Q)} = 2^{\abs{Q}}$}
    Да обърнем внимание, че е възможно за някоя двойка $(q,a)$ да няма нито един преход в автомата.
    Това е възможно, когато $\Delta(q,a) = \emptyset$;
  \item
    $Q_{\texttt{start}} \subseteq Q$ е множество от начални състояния;
  \item
    $F\subseteq Q$ е множеството от финални състояния.
  \end{itemize}
\end{definition}
\mynote{В \cite{papadimitriou} $\Delta$ е релация и се позволяват $\varepsilon$-преходи. В \cite{sipser3} пък е функция, но пак се позволяват
  $\varepsilon$-преходи. В \cite{hopcroft1} е функция без $\varepsilon$-преходи. Навсякъде има само едно начално.}

Удобно е да разширим функцията на преходите $\Delta: Q\times\Sigma \to \Ps(Q)$ 
до функцията $\Delta^\star: \Ps(Q)\times\Sigma^\star \to \Ps(Q)$,
която дефинираме за произволно $R \subseteq Q$ и $\alpha \in \Sigma^\star$ по следния начин:
\begin{itemize}
\item
  Ако $\alpha = \varepsilon$, то $\Delta^\star(R, \varepsilon) \df R$;
\item
  Ако $\alpha = \beta a$, то
  $\Delta^\star(R, \alpha b) \df \bigcup\{\Delta(p, b) \mid p \in \Delta^\star(R,\alpha)\}$.
\end{itemize}

\mynote{Да напомним, че $\bigcup\{\{0,1\}, \{1,2,3\}\} = \{0,1\} \cup \{1,2,3\}$.}

\begin{framed}
  \[\L(\N) \df \{\omega \in \Sigma^\star \mid \Delta^\star(Q_{\texttt{start}},\omega) \cap F \neq \emptyset \}.\]
\end{framed}

Обърнете внимание, че според нашата дефиниция е изпълнено следното:
\begin{equation}
  \label{eq:12}
  \Delta^\star(R,a) = \bigcup\{\Delta(r,a) \mid r \in R\}.
\end{equation}

Тогава за произволна дума $\alpha$ и буква $b$ имаме, че
\begin{equation}
  \label{eq:14}
  \Delta^\star(R,\alpha b) = \Delta^\star(\Delta^\star(R,\alpha),b),
\end{equation}
защото
\begin{align*}
  \Delta^\star(R,\alpha b) & = \bigcup\{\Delta(r,b) \mid r \in \Delta^\star(R,\alpha)\} & \comment\text{деф.}\\
                           & = \bigcup\{\Delta(r,b) \mid r \in U \} & \comment{U \df \Delta^\star(R,\alpha)}\\
                           & = \Delta^\star(U,b) & \comment\text{от (\ref{eq:12})}\\
                           & = \Delta^\star(\Delta^\star(R,\alpha),b). & \comment{U \df \Delta^\star(R,\alpha)}
\end{align*}

Ще видим, че това свойство е изпълнено и за произволни думи $\beta$ вместо само за букви $b$.

\begin{proposition}
  \label{pr:nfa:delta-star}
  \mynote{Сравенете с \Proposition{dfa:delta-star}.}
  За всеки две думи $\alpha,\beta \in \Sigma^\star$ и всяко $R \subseteq Q$,
  \[ \Delta^\star(R, \alpha\beta) = \Delta^\star( \Delta^\star(R,\alpha),\beta).\]
\end{proposition}
\begin{proof}
  Отново индукция по дължината на $\beta$.

  \begin{itemize}
  \item
    Нека $|\beta| = 0$, т.е. $\beta = \varepsilon$. Тогава:
    \begin{align*}
      \Delta^\star(R,\alpha\varepsilon) & = \Delta^\star(R,\alpha) & \comment\alpha\varepsilon = \alpha\\
                                        & = \Delta^\star( \Delta^\star(R,\alpha), \varepsilon). & \comment\text{деф. на }\Delta^\star
    \end{align*}
  \item
    Да приемем, че твърдението е вярно за думи $\beta$ с дължина $n$.
  \item
    Нека $|\beta| = n+1$, т.е. $\beta = \gamma b$, където $|\gamma| = n$.
    \begin{align*}
      \Delta^\star(R, \alpha\gamma b)  & = \Delta^\star(\Delta^\star(R,\alpha\gamma), b) & \comment{\text{от (\ref{eq:14})}}\\% \bigcup \{\Delta(p,b)  \mid p \in \Delta^\star(R,\alpha\gamma)\}& \comment\text{от деф. на }\Delta^\star\\
                                       & = \Delta^\star(\Delta^\star(\Delta^\star(R,\alpha),\gamma), b) & \comment\text{от И.П. за }\gamma\\
                                       & = \Delta^\star(\Delta^\star(U,\gamma), b) & \comment{\text{нека }U \df \Delta^\star(R,\alpha)}\\
                                       & = \Delta^\star(U,\gamma b) & \comment\text{от (\ref{eq:14})}\\
                                       & = \Delta^\star(U,\beta) & \comment\text{защото }\beta = \gamma b\\
                                       & = \Delta^\star( \Delta^\star(R,\alpha), \beta). & \comment{\text{защото }U = \Delta^\star(R,\alpha)}
    \end{align*}
  \end{itemize}
\end{proof}

% \begin{problem}
%   Докажете, че за произволни $R_i \subseteq Q$, където $i < k$, е изпълнено, че:
%   \[\Delta^\star( \bigcup_{i<k} R_i, \alpha) = \bigcup_{i<k} \Delta^\star( R_i, \alpha).\]
% \end{problem}

И тук е удобно да въведем бинарната релация $\vdash_\N$ над $Q\times\Sigma^\star$,
която ще ни казва как моментното описание на автомата $\N$ се променя след изпълнение на една стъпка:
\[(q,a\beta) \vdash_\N (p,\beta), \text{ ако } p \in \Delta(q,a).\]
\mynote{Рефл. и транз. затваряне на една релация е разгледано в Глава~\ref{ch:intro}}
Рефлексивното и транзитивно затваряне на $\vdash_\N$ ще означаваме с $\vdash^\star_\N$.
За да дадем по-ясна дефиниция на $\vdash^\star_\N$, първо ще дефинираме релацията $\vdash^n_\N$, която
определя работата на автомата $\N$ за $n$ на брой стъпки.

\begin{prooftree}
  \AxiomC{}
  \UnaryInfC{$(q,\alpha) \vdash^0_\N (q,\alpha)$}
\end{prooftree}
\mynote{Добре е дефиницията на $\vdash^n_\N$ да е съгласувана с дефиницията на $\Delta^\star$. Алтернативна дефиниция в този случай е следната:
  \begin{prooftree}
    \def\defaultHypSeparation{\hskip .0in}
  \AxiomC{$(r,\alpha) \vdash^n_\N (p, \beta)$}
  \AxiomC{$r \in \Delta(q,b)$}
  \BinaryInfC{$(q,b\alpha) \vdash^{n+1}_\N (p,\beta)$}
\end{prooftree}
}
\begin{prooftree}
  \AxiomC{$(q,\alpha) \vdash^n_\N (r, b \beta)$}
  \AxiomC{$p \in \Delta(r,b)$}
  \BinaryInfC{$(q,\alpha) \vdash^{n+1}_\N (p,\beta)$}
\end{prooftree}


% \begin{itemize}
% \item 
%   $(q,\alpha) \vdash^0_\N (q,\alpha)$, защото за $0$ стъпки се случва нищо.
% \item
%   Нека $\Delta(q,x) \ni q'$ и $(q',\alpha) \vdash^n_\N (p, \beta)$. Тогава
%   $(q,x\alpha) \vdash^{n+1}_\N (p,\beta)$, защото за $n+1$ на брой стъпки първо правим една стъпка 
%   и отиваме в моментното описание $(q',\alpha)$ и след това правим още $n$ на брой стъпки.
% \end{itemize}
Сега можем да дефинираме $\vdash^\star_\N$ като:
\[(q,\alpha) \vdash^\star_\N (p,\beta) \dff (\exists n\in\Nat)[(q,\alpha) \vdash^n_\N (p,\beta)].\]
Друг начин да дефинираме релацията $\vdash^\star_\N$ е следния:
\[(q,\alpha\beta) \vdash^\star_\N (p, \beta) \iff p \in \Delta^\star(\{q\},\alpha).\]
Получаваме, че 
\[\L(\N) = \{\alpha\in\Sigma^\star \mid (\exists q\in Q_{\texttt{start}})(\exists f \in F)[(q,\alpha) \vdash^\star_\N (f,\varepsilon)]\}.\]

\begin{framed}
\begin{thm}[Рабин-Скот \cite{rabin-scott}]
  За всеки недетерминиран краен автомат $\N$ съществува еквивалентен на него детерминиран краен автомат $\D$, т.е.
  \[\L(\N) = \L(\D).\]
\end{thm}
\end{framed}
\begin{hint}
  Нека $\N = \NFA$. Ще построим детерминиран автомат
  \[\D = (Q',\Sigma,\delta,\qstart,F'),\]
  за който $\L(\N) = \L(\D)$.
  Конструкцията е следната:
  \begin{itemize}
  \item
    \mynote{Да отбележим, че детерминираният автомат $\D$ има не повече от $2^{\abs{Q}}$ на брой състояния. Реално на нас ни трябват само тези множества
      $R$, за които съществува дума $\alpha$ и $\Delta^\star(Q_{\texttt{start}},\alpha) = R$.}
    $Q' = \{\code{R} \mid R \subseteq Q\}$;
  \item
    За произволна буква $a\in\Sigma$ и произволно $R \subseteq Q$,
    \begin{align*}
      \delta(\code{R},a) & \df \Delta^\star(R,a).
    \end{align*}
  \item
    $\qstart = \code{Q_{\texttt{start}}}$;
  \item
    $F' \df \{\code{R} \in Q' \mid R\cap F \neq \emptyset\}$.
  \end{itemize}
  Ще докажем, че за произволна дума $\alpha$ и произволно множество $R \subseteq Q$
  е изпълнено, че:
  \begin{equation}
    \label{eq:6}
    \code{\Delta^\star(R,\alpha)} = \delta^\star(\code{R},\alpha).
  \end{equation}
  Това ще направим с индукция по дължината на думата $\alpha$.
  \begin{itemize}
  \item
    Ако $|\alpha| = 0$, т.е. $\alpha = \varepsilon$, то е ясно от дефиницията на $\Delta^\star$ и $\delta^\star$, т.е.
    за всяко $R \subseteq Q$ е изпълнено, че:
    \[\code{\Delta^\star(R,\varepsilon)} = \code{R} = \delta^\star(\code{R},\varepsilon).\]
  \item
    Да приемем, че (\ref{eq:6}) е изпълнено за думи $\alpha$ с дължина $n$, т.е.
    \[(\forall \alpha\in\Sigma^{n})(\forall R \subseteq Q)[\ \code{\Delta^\star(R,\alpha)} = \delta^\star(\code{R},\alpha)\ ].\]
  \item
    Нека сега $\alpha$ има дължина $n+1$, т.е. $\alpha = \beta a$, където $|\beta| = n$ и $a \in \Sigma$.
    \begin{align*}
      \delta^\star( \code{R}, \beta a) & = \delta(\delta^\star(\code{R}, \beta a)) & \comment\text{деф. на }\delta^\star\\
                                       & = \delta(\code{\Delta^\star(R,\beta)},a) & \comment\text{от И.П. за }\beta\\
                                       & = \code{\Delta^\star(\Delta^\star(R,\beta),a)} & \comment\text{от деф. на }\delta\\
                                       & = \code{\Delta^\star(R,\beta a)}. & \comment\text{от \Proposition{nfa:delta-star}}
    \end{align*}
    Така доказахме, че
    \[(\forall \alpha\in\Sigma^{n+1})(\forall R \subseteq Q)[\ \code{\Delta^\star(R,\alpha)} = \delta^\star(\code{R},\alpha)\ ].\]
    Това означава, че според принципа на математическата индукция имаме Свойство~(\ref{eq:6}).
  \end{itemize}
  Сега вече е лесно да съобразим, че
  \begin{align*}
    \omega \in \L(\D) & \iff \delta^\star(\qstart, \omega) \in F' & \comment\text{деф. на }\L(\D)\\
                      & \iff \Delta^\star( Q_{\texttt{start}},\omega) \cap F \neq \emptyset & \comment\text{от (\ref{eq:6})}\\
                      & \iff \omega \in \L(\N) & \comment\text{деф. на }\L(\N).
  \end{align*}
\end{hint}

\mynote{Хубаво е да има един пример за детерминизация.}

% \begin{problem}
%   За всеки недетерминиран краен автомат $\N$ съществува недетерминиран краен автомат $\N'$ с едно финално състояние, 
%   за който $\L(\N) = \L(\N')$.
% \end{problem}
% \begin{hint}
%   Вместо формална конструкция, да разгледаме един пример, който илюстрира идеята.
%   \begin{figure}[H]
%     \begin{subfigure}[b]{0.2\textwidth}
%       \begin{tikzpicture}[framed,->,>=stealth,thick,node distance=45pt]
%         \tikzstyle{every state}=[circle,minimum size=20pt,auto]
%         \node[initial below,state]      (1) {$q_0$};
%         \node[state,accepting]     [above right of=1] (2) {$q_1$};
%         \node[state,accepting]     [below right of=1] (3) {$q_2$};
%         \path
%         (1) edge [bend left=15] node  [above] {$a$} (2)
%         (2) edge [bend left=15] node  [right] {$b$} (1)
%         (2) edge [bend left=15] node  [right] {$a$} (3)
%         (3) edge [bend left=15] node  [below] {$a$} (1)
%         (3) edge [loop below] node  [right] {$b$} (3);
%       \end{tikzpicture}
%       \caption{автомат $\N$}
%     \end{subfigure}
%     \hspace*{1.4cm}
%     \begin{subfigure}[b]{0.5\textwidth}
%       \begin{tikzpicture}[framed,->,>=stealth,thick,node distance=45pt]
%         \tikzstyle{every state}=[circle,minimum size=20pt,auto]
%         \node[initial below,state]                    (1) {$q_0$};
%         \node[state]               [above right of=1] (2) {$q_1$};
%         \node[state]               [below right of=1] (3) {$q_2$};
%         \node[state,accepting]     [right=5cm of 1]   (4) {$f$};

%         \path
%         (1) edge [bend left=15] node  [above] {$a$} (2)
%         (2) edge [bend left=15] node  [right] {$b$} (1)
%         (2) edge [bend left=15] node  [right] {$a$} (3)
%         (3) edge [loop below]   node  [right] {$b$} (3)
%         (3) edge [bend left=15] node  [below] {$a$} (1)
%         (1) edge [dashed,bend left=15] node  [above] {$a$} (4)
%         (2) edge [dashed,bend left=15] node  [above] {$a$} (4)
%         (3) edge [dashed,bend right=15] node  [below] {$b$} (4);
%       \end{tikzpicture}
%     \caption{автомат $\N'$, за който $\L(\N') = \L(\N)$}
%   \end{subfigure}
% \end{figure}  
% За произволен автомат $\N$, формулирайте точно конструкцията на $\N'$ с едно финално състояние и докажете, че наистина $\L(\N) = \L(\N')$.
% Обърнете внимание, че примерът показва, че е възможно $\N$ да е детерминиран автомат, но полученият $\N'$ да бъде недетерминиран.
% \end{hint}

\begin{problem}
  \mynote{По-късно ще видим, че можем да дадем и друго доказателство на това твърдение, като направим индукция по построението на регулярните езици.}
  Докажете, че автоматните езици са затворени относно операцията $\rev$.
  С други думи, докажете, че ако $L$ е автоматен език, то
  \[L^{\rev} = \{\omega^{\rev} \mid \omega \in L\}\]
  също е автоматен.
\end{problem}

\begin{lemma}
  \label{lem:automata-basic}
  Съществува краен детерминиран автомат $\A = \FA$, който разпознава езика $L$, където
  \begin{itemize}
  \item
    $L = \emptyset$,
  \item
    $L = \{\varepsilon\}$, или
  \item
    $L = \{a\}$, за произволна буква $a\in\Sigma$.
  \end{itemize}
\end{lemma}
\begin{hint}
  \begin{figure}[H]
    \begin{subfigure}[b]{0.2\textwidth}
      \label{subf:a1}
      \begin{tikzpicture}[framed,->,>=stealth,thick,node distance=35pt]
        \tikzstyle{every state}=[circle,minimum size=15pt,auto]
        \node[initial below,state]      (1) {$q_0$};
      \end{tikzpicture}
      \caption{$L(\A) = \emptyset$}
    \end{subfigure}
    \qquad
    \begin{subfigure}[b]{0.2\textwidth}
      \begin{tikzpicture}[framed,->,>=stealth,thick,node distance=35pt]
        \tikzstyle{every state}=[circle,minimum size=15pt,auto]
        \node[initial below,state,accepting]      (1) {$q_0$};
      \end{tikzpicture}
      \caption{$\L(\A) = \{\varepsilon\}$}
    \end{subfigure}
    \qquad
    \begin{subfigure}[b]{0.2\textwidth}
      \begin{tikzpicture}[framed,->,>=stealth,thick,node distance=45pt]
        \tikzstyle{every state}=[circle,minimum size=15pt,auto]
        \node[initial below,state]      (1)              {$q_0$};
        \node[accepting,state]    (2) [right of=1] {$q_1$};
        \path 
        (1) edge  node [above] {$a$} (2);
      \end{tikzpicture}
      \caption{$\L(\A) = \{a\}$}
    \end{subfigure}
  \end{figure}
\end{hint}



\begin{framed}
  \begin{lemma}
    \label{lem:concat}
    Класът на автоматните езици е затворен относно операцията {\em конкатенация}.
    Това означава, че ако $L_1$ и $L_2$ са два произволни автоматни езика, то $L_1\cdot L_2$
    също е автоматен език.
  \end{lemma}  
\end{framed}
\begin{proof}
  \mynote{Тук отново приемаме, че $Q_1 \cap Q_2 = \emptyset$.}
  Нека за по-просто да вземем два детерминирани крайни автомата:
  \begin{itemize}
  \item
    $\A_1 = \pair{\Sigma,Q_1,\delta_1,\qstart',F_1}$, където $\L(\A_1) = L_1$;
  \item
    $\A_2 = \pair{\Sigma,Q_2,\delta_2,\qstart'', F_2}$, където $\L(\A_2) = L_2$.
  \end{itemize}
  Ще дефинираме автомата $\N = \NFA$ по такъв начин, че
  \[\L(\N) = L_1\cdot L_2 = \L(\A_1)\cdot\L(\A_2).\]
  \begin{itemize}
  \item
    $Q \df Q_1 \cup Q_2$;
  \item
    $Q_{\texttt{start}} \df \{\qstart'\}$;
  \item
    $F \df \begin{cases}
      F_1 \cup F_2, & \text{ ако } \qstart'' \in F_2\\
      F_2,          & \text{ иначе}.
    \end{cases}$
  \item 
    $\Delta(q,a) \df
    \begin{cases}
      \{\delta_1(q,a)\},                      & \text{ ако }q\in Q_1\setminus F_1\ \&\ a\in\Sigma\\
      \{\delta_1(q,a), \delta_2(\qstart'',a)\}, & \text{ ако }q \in F_1\ \&\ a\in\Sigma\\
      \{\delta_2(q,a)\},                      & \text{ ако }q\in Q_2\ \&\ a\in\Sigma.
    \end{cases}$
  \end{itemize}
  Първо ще докажем, че
  \[\L(\A_1)\cdot\L(\A_2) \subseteq \L(\N).\]
  За целта, нека разгледаме думата $\alpha \in \L(\A_1)$ и $\beta \in \L(\A_2)$. Това означава, че имаме следните изчисления:
  \begin{align*}
    & (\qstart',\alpha) \vdash^\star_{\A_1} (q_1,\varepsilon), \text{за някое }q_1 \in F_1\\
    & (\qstart'',\beta) \vdash^\star_{\A_2} (q_2,\varepsilon), \text{за някое }q_2 \in F_2.
  \end{align*}
  Според дефиницията на недетерминирания краен автомат $\N$ е ясно, че:
  \[(\qstart',\alpha) \vdash^\star_{\N} (q_1,\varepsilon), \text{за някое }q_1 \in F_1.\]
  
  Ако $\beta = \varepsilon$, то това означава, че $\qstart'' \in F_2$ и следователно $F_1 \subseteq F$.
  Тогава получаваме, че $\alpha \cdot \beta = \alpha \in \L(\N)$, защото
  \[(\qstart',\alpha) \vdash^\star_{\N} (q_1,\varepsilon), \text{за някое }q_1 \in F_1 \subseteq F.\]

  Ако $\beta = b\gamma$, за някоя дума $\gamma \in \Sigma^\star$, то тогава можем да разбием изчислението на $\beta$ в $\A_2$ по следния начин:
  \[(\qstart'',b\gamma) \vdash_{\A_2} (q,\gamma) \vdash^\star_{\A_2} (q_2,\varepsilon), \text{за някое }q_2 \in F_2,\]
  където $q = \delta_2(\qstart'',b)$.
  
  Според дефиницията на недетерминирния краен автомат $\N$ е ясно, че:
  \[(q,\gamma) \vdash^\star_{\N} (q_2,\varepsilon), \text{за някое }q_2 \in F_2.\]
  Освен това, имаме, че $q \in \Delta(q_1,b)$, защото $q_1 \in F_1$. Това означава, че
  \[(q_1,b\gamma) \vdash_\N (q,\gamma).\]
  Съединявайки последните две изчисления, получаваме, че:
  \[(q_1,\beta) \vdash^\star_\N (q_2,\varepsilon),\text{ за някое }q_2 \in F_2.\]
  Сега съединяваме и изчислението за $\alpha$ и получаваме, че:
  \[(\qstart',\alpha\beta) \vdash^\star_\N (q_1,\beta) \vdash^\star_\N (q_2,\varepsilon),\text{ за някое }q_2 \in F_2.\]
  Оттук заключаваме, че във всички случаи за $\beta$, $\alpha \cdot \beta \in \L(\N)$.

  Сега ще докажем, че
  \[\L(\N) \subseteq \L(\A_1) \cdot \L(\A_2).\]
  За целта, нека разгледаме думата $\omega \in \L(\N)$, където $\omega = a_0a_1\cdots a_{n-1}$.
  Да разгледаме една редица от състояния $(q_i)^{n}_{i=0}$, която описва приемащо изчисление на $\N$ върху $\omega$.
  \mynote{Възможно е да има и други редици от състояния $(p_i)^{n}_{i=0}$, които да описват приемащи изчисления на $\N$ върху $\omega$.}
  Това означава, че:
  \begin{itemize}
  \item
    $q_0 = \qstart$;
  \item
    $q_{i+1} \in \Delta(q_i, \omega\slice{i})$ за $i < n$;
  \item
    $q_n \in F$.
  \end{itemize}
  
  Ако $q_n \in F_1$, то според конструкцията на $\N$, $\varepsilon \in \L(\A_2)$ и всяко състояние от $(q_i)^{n}_{i=0}$ принадлежи на $Q_1$ и оттам $\omega \in \L(\A_1)$.
  Интересният случай е когато $q_n \in F_2$.
  Според конструкцията на $\N$, не можем да преминем от състояние от $Q_2$ в състояние от $Q_1$.
  Това означава, че можем да разбием редицата от състояния $(q_i)^n_{i=0}$ на две непразни подредици:
  \begin{itemize}
  \item
    $(q_{i})^{\ell}_{i=0}$ - тези които са от $Q_1$,
  \item
    $(q_i)^{n}_{i=\ell+1}$ - тези, които са от $Q_2$.
  \end{itemize}
  Нека $\alpha = \omega\slice{:\ell}$ и $\beta = \omega\slice{\ell:}$.
  Ясно е, че:
  \[(q_0,\omega) \vdash^\star_\N (q_{\ell}, \omega\slice{\ell:}) \vdash_\N (q_{\ell+1},\omega\slice{\ell+1:}) \vdash^\star_\N (q_n,\varepsilon).\]
  От конструкцията на $\N$ следва, че редицата от състояния $(q_i)^{\ell}_{i=0}$ описва приемащо изчисление на $\A_1$ върху $\alpha$.
  \mynote{Това е единственият начин да направим преход от състояние на $Q_1$ към състояние на $Q_2$.}
  Също така от конструкцията следва, че щом $q_{\ell+1} \in \Delta(q_\ell,a_\ell)$, то $q_{\ell} \in F_1$ и $\delta_2(\qstart'',a_\ell) = q_{\ell+1}$. Заключаваме, че:
  \begin{itemize}
  \item
    $(q_0, \alpha) \vdash^\star_{\A_1} (q_\ell,\varepsilon)$.
    Понеже $q_0 = \qstart'$ и $q_\ell \in F_1$, то $\alpha \in \L(\A_1)$.
  \item
    $(\qstart'', \beta) \vdash^\star_{\A_2} (q_n,\varepsilon)$.
    Понеже $q_n \in F_2$, то $\beta \in \L(\A_2)$.
  \end{itemize}
\end{proof}

\begin{figure}[H]
  \center
  \begin{subfigure}[b]{0.3\textwidth}
    \label{subf:a1}
    \begin{tikzpicture}[framed,->,>=stealth,thick,node distance=45pt,initial text=начало]
      \tikzstyle{every state}=[circle,minimum size=15pt,auto]
      \node[initial below,state,accepting]      (1) {$q'_0$};
      \node[state]                        (2) [right of=1] {$q_1$};
      \node[state]                        (3) [above right of=2] {$q_2$};
      \node[state,accepting]              (4) [below right of=2] {$q_3$};
      \path
      (1) edge [loop above] node [above] {$b$} (1)
      (1) edge node [above] {$a$} (2)
      (2) edge node [above] {$a$} (3)
      (3) edge [loop right] node [right] {$b$} (3)
      (2) edge node [below] {$b$} (4)
      (3) edge [bend right=30] node [above] {$a$} (1)
      (4) edge [bend right=15] node [right] {$a$} (3)
      (4) edge [bend left=30] node [below] {$b$} (1);
    \end{tikzpicture}
    \caption{автомат $\A_1$}
  \end{subfigure}
  \qquad
  \qquad
  \qquad
  \begin{subfigure}[b]{0.3\textwidth}
    \begin{tikzpicture}[framed,->,>=stealth,thick,node distance=45pt,initial text=начало]
      \tikzstyle{every state}=[circle,minimum size=15pt,auto]
      \node[initial,state]                (1) {$q''_0$};
      \node[state]     [above right of=1] (2) {$q_4$};
      \node[state,accepting]     [below right of=1] (3) {$q_5$};
      \path
      (1) edge [bend left=15] node  [above] {$a$} (2)
      (2) edge [loop above] node [above] {$b$} (2)
      (2) edge [bend left=15] node  [right] {$a$} (3)
      (3) edge [loop right]  node [right] {$a,b$} (3)
      (1) edge [bend right=15] node [below] {$b$} (3);
    \end{tikzpicture}
    \caption{автомат $\A_2$}
  \end{subfigure}
\end{figure}

\begin{example}
    За да построим автомат, който разпознава конкатенацията на $\L(\A_1)$ и $\L(\A_2)$,
    трябва да свържем финалните състояния на $\A_1$ с изходящите от $s_2$ състояния на $\A_2$.
    
    \begin{figure}[H]
      \center
      % \begin{subfigure}[b]{0.3\textwidth}
      \begin{tikzpicture}[framed,->,>=stealth,thick,node distance=2cm,initial text=начало]
        \tikzstyle{every state}=[circle,minimum size=15pt,auto]
        \node[initial,state]                      (1) {$q'_0$};
        \node[state] [right of=1]                 (2) {$q_1$};
        \node[state] [above right of=2]           (3) {$q_2$};
        \node[state] [below right of=2]           (4) {$q_3$};
        \node[state] [right=4cm of 1]             (5) {$q''_0$};
        \node[state] [above right of=5]           (6) {$q_4$};
        \node[state,accepting] [below right of=5] (7) {$q_5$};
        \path
        (1) edge [loop above] node [above] {$b$} (1)
        (1) edge node [above]                         {$a$} (2)
        (2) edge node [above]                         {$a$} (3)
        (2) edge node [below]                         {$b$} (4)
        (3) edge [loop right] node [right]            {$b$} (3)
        (6) edge [loop above] node [above]            {$b$} (6)
        (7) edge [loop right] node [right]            {$a,b$} (7)
        (3) edge [bend right=15] node [above]         {$a$} (1)
        (4) edge [bend left=15] node [below]          {$b$} (1)
        (4) edge [bend left=15] node [left]          {$a$} (3)
        (5) edge [bend left=15] node [below]          {$a$} (6)
        (6) edge [bend left=15] node [right]          {$a$} (7)
        (5) edge [bend right=15] node [above]         {$b$} (7)
        (1) edge [dashed, bend left=45] node [above]  {$a$} (6)
        (1) edge [dashed, bend right=45] node [below] {$b$} (7)
        (4) edge [dashed, bend left=30] node [above]  {$a$} (6)
        (4) edge [dashed, bend left=10] node [above]  {$b$} (7);
      \end{tikzpicture}
      \caption{$\L(\N) = \L(\A_1)\cdot\L(\A_2)$}
  \end{figure}  
  Обърнете внимание, че $\A_1$ и $\A_2$ са детерминирани автомати, но $\N$ е недетерминиран.
  Също така, в този пример се оказва, че вече $q''_0$ е недостижимо състояние, но в общия случай не можем да 
  го премахнем, защото може да има преходи влизащи в $q''_0$.
\end{example}

%%% Local Variables:
%%% mode: latex
%%% TeX-master: "../eai"
%%% End:

% \subsection{Затвореност относно обединение}

\begin{important}
  \begin{lemma}
    \label{lem:union}
    Класът от автоматните езици е затворен относно операцията {\em обединение}.
  \end{lemma}  
\end{important}
\begin{hint}
  \mynote{Да напомним, че вече знаем, че автоматните езици са затворени относно операцията обединие. Това видяхме в \Proposition{automata-union}. Сега ще дадем втора конструкция.}
  Нека са дадени детерминистичните автомати:
  \begin{itemize}
  \item 
    $\A_1 = \pair{\Sigma,Q_1,\delta_1,\qstart',F_1}$, като $L(\A_1) = L_1$;
  \item
    $\A_2=\pair{\Sigma,Q_2,\delta_2,\qstart'',F_2}$, като $L(\A_2) = L_2$.
  \end{itemize}
  Ще дефинираме автомата $\N=\NFA$, така че
  \[L(\N) = L(\A_1) \cup L(\A_2).\]
  \begin{itemize}
  \item
    $Q_{\texttt{start}} = \{\qstart',\qstart''\}$;
  \item 
    $Q \df Q_1 \cup Q_2$;
  \item
    $F \df F_1 \cup F_2$;
    % $F \df 
    % \begin{cases}
    %   F_1 \cup F_2 \cup \{\qstart\}, & \text{ ако } \qstart' \in F_1 \vee \qstart'' \in F_2\\
    %   F_1 \cup F_2,            & \text{ иначе } 
    % \end{cases}$
  \item
    $\Delta(q,a) \df
    \begin{cases}
      \{\delta_1(q,a)\},                       & \text{ ако } q\in Q_1\text{ и }a\in\Sigma\\
      \{\delta_2(q,a)\},                       & \text{ ако } q\in Q_2\text{ и }a\in\Sigma.
    \end{cases}$
  \end{itemize}
\end{hint}

\begin{example}
    За да построим автомат, който разпознава обединението на $\L(\A_1)$ и $\L(\A_2)$,
    трябва да добавим ново начално състояние, което да свържем с наследниците на началните състояния на $\A_1$ и $\A_2$.
    
    \begin{figure}[H]
      \center
      % \begin{subfigure}[b]{0.3\textwidth}
      \begin{tikzpicture}[framed,->,>=stealth,thick,node distance=2cm]
        \tikzstyle{every state}=[circle,minimum size=20pt,auto]
        % \node[initial,state,accepting]      (0) {$q_0$};
        \node[initial,state,accepting] [above right of=0]        (1) {$q'_0$};
        \node[state]    [right of=1]        (2) {$q_1$};
        \node[state]                        (3) [above right of=2] {$q_2$};
        \node[state,accepting]                        (4) [below right of=2] {$q_3$};
        \node[initial,state]    [below right=2cm of 0] (5) {$q''_0$};
        \node[state]     [above right of=5] (6) {$q_4$};
        \node[state,accepting]     [below right of=5] (7) {$q_5$};
        \path
        (1) edge [loop above] node [above] {$b$} (1)
        (1) edge node [above]                  {$a$} (2)
        (2) edge node [above]                  {$a$} (3)
        (2) edge node [below]                  {$b$} (4)
        (3) edge [bend right=15] node [above]  {$a,b$} (1)
        (4) edge [bend left=15]  node [below]  {$b$} (1)
        (4) edge [bend right=15]  node [right]  {$a$} (3)
        (5) edge [bend left=15] node [below]   {$a$} (6)
        (6) edge [bend left=15] node  [right] {$a,b$} (7)
        (5) edge [bend right=15] node [above]  {$b$} (7)
        (7) edge [loop right] node [right] {$a,b$} (7);
        % (0) edge [dashed, bend right=15] node [below]  {$a$} (2)
        % (0) edge [dashed, bend left=15] node [above]  {$b$} (1)
        % (0) edge [dashed, bend right=15] node [below]  {$a$} (6)
        % (0) edge [dashed, bend right=45] node [below]  {$b$} (7);
      \end{tikzpicture}
      \caption{$\L(\N) = \L(\A_1)\cup\L(\A_2)$.}
  \end{figure}  
  Обърнете внимание, че $\A_1$ и $\A_2$ са детерминирани автомати, но $\N$ е недетерминиран.
  Освен това, новото състояние $q_0$ трябва да бъде маркирано като финално, защото $q'_0$ е финално.
\end{example}

%%% Local Variables:
%%% mode: latex
%%% TeX-master: "../eai"
%%% End:

\subsection{Затвореност относно звезда}

\begin{framed}
  \begin{lemma}
    \label{lem:kleene-star}
    Класът от автоматните езици е затворен относно операцията {\em звезда на Клини}, т.е.
    за всеки автоматен език $L$, езикът $L^\star$ също е автоматен.
  \end{lemma}  
\end{framed}
\begin{proof}
  Да разгледаме детерминирния краен автомат
  \[\A = \FA.\]% pair{\Sigma,Q,\qstart,\delta,F}.\]
  Първо ще построим недетерминиран краен автомат
  \[\N = \NFA,\] такъв че
  \[\L(\N) = (\L(\A))^+.\]
  После ще построим недетерминиран краен автомат $\N'$, за който
  \[\L(\N') = \L(\N) \cup \{\varepsilon\} = (\L(\A))^\star.\]

  Дефинираме функцията на преходите $\Delta$ на $\N$ като за $q \in Q$ и $a \in \Sigma$ определяме функцията на преходите $\Delta$ по следния начин:
  \begin{align*}
    \Delta(q,a) \df
    \begin{cases}
      \{\delta(q,a)\}, & \text{ако } q \not\in F\\
      \{\delta(q,a), \delta(\qstart,a) \}, & \text{ако } q \in F.
    \end{cases}
  \end{align*}
    
  Нека $\alpha = a_0a_1\cdots a_{n-1} \in \L(\N)$.
  Това означава, че $(\qstart,\alpha) \vdash^\star_\N (f,\varepsilon)$ за някое $f \in F$.
  Нека редицата от състояния $(q_i)^n_{i=0}$ описва едно приемащо изчисление на $\N$ върху $\alpha$, т.е.
  \begin{itemize}
  \item
    $q_0 = \qstart$;
  \item
    $q_{i+1} \in \Delta(q_i,a_i)$;
  \item
    $q_n \in F$.
  \end{itemize}
  Да разгледаме максималната подпоследователност от състояния $(q_{i_j})^{\ell+1}_{j = 0}$ на $(q_i)^{n}_{i=0}$ съставена от тези състояния, за които
  \mynote{Тук е малко по-сложно, защото правим разбиване на изчислението не на база всички финални състояния, а на тези финални състояния, от които изчислението продължава в насленик на $\qstart$,
  защото това са местата, където можем да разцепим думата на части.}
  \begin{itemize}
  \item
    $q_{i_0} = \qstart$;
  \item
    $q_{i_j} \in F\ \&\ \delta(\qstart,a_{i_j}) = q_{i_j+1}$, за $j = 1,\dots,\ell$;
  \item
    $q_{i_{\ell+1}} = q_n$.
  \end{itemize}
  Да разбием думата $\alpha$ като $\alpha = \alpha_0\alpha_1\cdots\alpha_l$, където:
  \begin{align*}
    & \alpha_0 \df a_{i_0} \alpha'_0\\
    & \alpha_1 \df a_{i_1}\alpha'_1\\
    & \cdots\\
    & \alpha_\ell \df a_{i_\ell}\alpha'_\ell.
  \end{align*}
  Сега можем да разбием изчислението на $\N$ върху $\alpha$ по следния начин:
  \begin{align*}
    (q_{i_0},\alpha_{0}\alpha_{1}\cdots \alpha_{\ell}) & \vdash^\star_\N (q_{i_1}, \alpha_{1}\cdots \alpha_{\ell}) & \comment{ q_{i_0} = \qstart}\\
                                                    & \vdash^\star_\N (q_{i_2},\alpha_{2}\cdots \alpha_{\ell})\\
                                                    & \cdots\\
                                                    & \vdash^\star_\N (q_{i_{\ell}}, \alpha_{\ell})\\
                                                    & \vdash^\star_\N(q_{i_{\ell+1}},\varepsilon). & \comment{ q_{i_{\ell+1}} = q_n \in F}
  \end{align*}
  Да разгледаме само първата част на изчислението:
  \[\underbrace{(q_{i_0},\alpha_0) \vdash^\star_\N (q_{i_1}, \varepsilon)}_{\text{само преходи от }\A}.\]
  Понеже $q_{i_0} = \qstart$ и $q_{i_1} \in F$, то е ясно, че $\alpha_0 \in \L(\A)$.
  
  За $j = 0,\dots,\ell$, изчислението
  \[(q_{i_j},\alpha_j) \vdash^\star_\N (q_{i_{j+1}},\varepsilon)\]
  може по-подробно да се запише и така:
  \[(q_{i_j}, a_{i_{j}}\alpha'_j) \vdash_\N \underbrace{(q_{i_j+1}, \alpha'_j)  \vdash^\star_\N (q_{i_{j+1}},\varepsilon)}_{\text{само преходи от }\A}.\]
  Понеже имаме, че $\delta(\qstart,a_{i_j}) = q_{i_j+1}$, то оттук следва, че:
  \[\underbrace{(\qstart,a_{i_j}\alpha'_{j}) \vdash_\N (q_{i_j+1}, \alpha'_{j})}_{\text{преход от }\A}  \vdash^\star_\A (q_{i_{j+1}},\varepsilon).\]
  Заключаваме, че
  \[(\qstart,\alpha_{j}) \vdash^\star_\A (q_{i_{j+1}},\varepsilon).\]
  Понеже $q_{i_{j+1}} \in F$, веднага следва, че $\alpha_{j} \in \L(\A)$.
  От всичко дотук заключваме, че $\alpha \in (\L(\A))^+$.

  За другата посока, нека $\alpha \in (\L(\A))^+$.
  Това означава, че думата $\alpha$ може да се представи като
  $\alpha = \alpha_0 \cdot \alpha_1 \cdots \alpha_\ell$, където $\alpha_j \in \L(\A)$ и $\alpha_j \neq \varepsilon$, за $j = 0,\dots, \ell$.
  Нека за $j=0,\dots,\ell$ да положим
  \[\alpha_j \df a_j \cdot \alpha'_j\text{ и } q_{j} \df \delta(\qstart, a_j).\]
  Оттук получаваме за $j = 0,\dots, \ell$ следните изчисления:
  \[(\qstart, \alpha_{j}) \vdash_\A (q_{j}, \alpha'_j) \vdash^\star_\A (f_{j}, \varepsilon), \text{ за някои }f_j \in F.\]
  Понеже функцията на преходите на $\N$ разширява функцията на преходите на $\A$, то е ясно е, че имаме също така и следното:
  \[(\qstart, \alpha_{j}) \vdash_\N (q_{j}, \alpha'_j) \vdash^\star_\N (f_{j}, \varepsilon), \text{ за някои }f_j \in F.\]

  За $1 \leq j < \ell$,
  понеже $\delta(\qstart,a_{j}) = q_{j}$ и $f_{j-1} \in F$, то според конструкцията на недетерминирания краен автомат $\N$,
  \[q_{j} \in \Delta(f_{j-1}, a_{j}).\]
  Оттук следва, че имаме следното изчисление на $\N$ върху $\alpha$:
  \begin{align*}
    &(\qstart, \alpha_{0}) \vdash^\star_\N (f_{0}, \varepsilon) & \comment\text{за някое }f_0 \in F\\
    &(f_0, \alpha_{1}) \vdash_\N (q_{1}, \alpha'_1) \vdash^\star_\N (f_{1}, \varepsilon) & \comment\text{за някое }f_1 \in F\\
    &(f_1, \alpha_{2}) \vdash_\N (q_{2}, \alpha'_2) \vdash^\star_\N (f_{2}, \varepsilon) & \comment\text{за някое }f_2 \in F\\
    &\dots\\
    &(f_{\ell-1}, \alpha_{\ell}) \vdash_\N (q_{\ell}, \alpha'_{\ell}) \vdash^\star_\N (f_{\ell}, \varepsilon). & \comment\text{за някое }f_{\ell} \in F
  \end{align*}
  Обединявайки всичко това, заключаваме, че  $\alpha \in \L(\N)$.

  Така доказахме, че $\L(\N) = (\L(\A))^+$.
  Сега ще построим недетерминиран краен автомат
  \[\N' = \pair{\Sigma,Q',\qstart',\Delta',F'},\]
  такъв че
  \[\L(\N') = \L(\N)\cup\{\varepsilon\} = (\L(\A))^+ \cup \{\varepsilon\} = (\L(\A))^\star.\]
  \begin{itemize}
  \item
    $Q' = Q \cup \{\qstart'\}$;
  \item
    $F' = F \cup \{\qstart'\}$;
  \item
    $\Delta'(\qstart',a) = \Delta(\qstart,a)$, за всяко $a \in \Sigma$;
  \item
    $\Delta'(q,a) = \Delta(q,a)$, за всяко $a \in \Sigma$ и $q\in Q$.
  \end{itemize}
  Лесно се съобразява, че $\L(\N') = \L(\N)\cup\{\varepsilon\}$.
\end{proof}

\begin{example}
  Нека да приложим конструкцията за да намерим автомат разпознаващ $\L(\A)^\star$.
  
  \begin{figure}[H]
    % \center
    \begin{subfigure}[b]{0.3\textwidth}
      \begin{tikzpicture}[framed,->,>=stealth,thick,node distance=55pt]
        \tikzstyle{every state}=[circle,minimum size=20pt,auto]
        \node[initial below,state] (1) {$q_0$};
        \node[state]               (2) [right of=1] {$q_1$};
        \node[state,accepting]     (3) [right of=2] {$q_2$};
        \node[state,accepting]     (4) [above of=2] {$q_3$};
        \node[state]               (5) [above of=3] {$q_4$};
        \path
        (1) edge [bend right=15] node [below] {$a$} (2)
        (1) edge [bend left=15]  node [above] {$b$} (4)
        (2) edge [bend left=15]  node [left] {$a$} (4)
        (2) edge [bend right=15] node [below] {$b$} (3)
        (5) edge [loop above]    node [above] {$a,b$} (5)
        (4) edge [bend left=15]  node [above] {$a,b$} (5)
        (3) edge [bend right=15] node [right] {$b$} (5)
        (3) edge [bend left=45]  node [below] {$a$} (1);
      \end{tikzpicture}
      \caption{автомат $\A$}
    \end{subfigure}
    \hspace{2cm}
    \begin{subfigure}[b]{0.5\textwidth}
      \begin{tikzpicture}[framed,->,>=stealth,thick,node distance=55pt]
        \tikzstyle{every state}=[circle,minimum size=20pt,auto]
        
        \node[initial below, state] (1) [below right of=0] {$q_0$};
        \node[state]                (2) [right of=1] {$q_1$};
        \node[state,accepting]      (3) [right of=2] {$q_2$};
        \node[state,accepting]      (4) [above of=2] {$q_3$};
        \node[state]                (5) [above of=3] {$q_4$};
        \path
        (1) edge [bend left=15] node [above] {$b$} (4)
        (1) edge [bend right=15] node [below] {$a$} (2)
        (2) edge [bend right=15] node [below] {$b$} (3)
        (2) edge [bend left=15]  node [left] {$a$} (4)
        (3) edge [bend left=45]  node [below] {$a$} (1)
        (3) edge [dashed, bend right=20] node [above] {$b$} (4)
        (4) edge [dashed,bend left=15] node [right] {$a$} (2)
        (4) edge [dashed, loop above] node {$b$} (4)
        (5) edge [loop above] node [above] {$a,b$} (5)
        (4) edge [bend left=15] node [above] {$a,b$} (5)
        (3) edge [bend right=15] node [right] {$b$} (5)
        (3) edge [dashed, bend right=15] node [above] {$a$} (2);        
      \end{tikzpicture}
      \caption{$\L(\N) = \L(\A)^+$}
    \end{subfigure}
  \end{figure}
    
  \mynote{Лесно се вижда, че $\L(\A) = \{b\} \cup \{ab\}\cdot\{aba\}^\star$}
  След като построим автомат за езика $\L(\A)^+$, трябва да приложим
  конструкцията за обединение на автомата за езика $\L(\A)^+$ с автомата за езика $\{\varepsilon\}$.
  Защо трябва да добавим ново начално състояние $q_0'$?
  Да допуснем, че вместо това сме направили $q_0$ финално.
  Тогава има опасност да разпознаем повече думи. Например, думата $aba$ би се разпознала от този автомат,
  но $aba \not\in\L(\A)^\star$.
  
  \begin{figure}[H]
    \centering
      \begin{tikzpicture}[framed,->,>=stealth,thick,node distance=55pt]
        \tikzstyle{every state}=[circle,minimum size=20pt,auto]
        \node[initial, state, accepting] (0) [above of=1] {$q'_0$};
        \node[initial,state] (1) [below of=0] {$q_0$};
        \node[state]                (2) [right of=1] {$q_1$};
        \node[state,accepting]      (3) [right of=2] {$q_2$};
        \node[state,accepting]      (4) [above of=2] {$q_3$};
        \node[state]                (5) [above of=3] {$q_4$};
        \path
        % (0) edge [dashed, bend left=15] node [above] {$b$} (4)
        % (0) edge [dashed, bend left=15] node [above] {$a$} (2)
        (1) edge [bend left=15] node [above] {$b$} (4)
        (1) edge [bend right=15] node [below] {$a$} (2)
        (2) edge [bend right=15] node [below] {$b$} (3)
        (2) edge [bend left=15]  node [left] {$a$} (4)
        (3) edge [bend left=45]  node [below] {$a$} (1)
        (3) edge [dashed, bend right=20] node [above] {$b$} (4)
        (4) edge [dashed,bend left=15] node [right] {$a$} (2)
        (4) edge [dashed, loop above] node {$b$} (4)
        (5) edge [loop above] node [above] {$a,b$} (5)
        (4) edge [bend left=15] node [above] {$a,b$} (5)
        (3) edge [bend right=15] node [right] {$b$} (5)
        (3) edge [dashed, bend right=15] node [above] {$a$} (2);        
      \end{tikzpicture}
      \caption{$\L(\N') = \L(\N) \cup \{\varepsilon\} = \L(\A)^\star$}    
  \end{figure}

\end{example}

%%% Local Variables:
%%% mode: latex
%%% TeX-master: "../eai"
%%% End:






%%% Local Variables:
%%% mode: latex
%%% TeX-master: "../eai"
%%% End:
