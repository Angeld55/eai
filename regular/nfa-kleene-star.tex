% \subsection{Затвореност относно звезда}

\begin{framed}
  \begin{lemma}
    \label{lem:kleene-star}
    Класът от автоматните езици е затворен относно операцията {\em звезда на Клини}, т.е.
    за всеки автоматен език $L$, езикът $L^\star$ също е автоматен.
  \end{lemma}  
\end{framed}
\begin{proof}
  Да разгледаме детерминирния краен автомат
  \[\A = \FA.\]% pair{\Sigma,Q,\qstart,\delta,F}.\]
  Първо ще построим недетерминиран краен автомат
  \[\N = \NFA,\] такъв че
  \[\L(\N) = (\L(\A))^+.\]
  После ще построим недетерминиран краен автомат $\N'$, за който
  \[\L(\N') = \L(\N) \cup \{\varepsilon\} = (\L(\A))^\star.\]

  Дефинираме функцията на преходите $\Delta$ на $\N$ като за $q \in Q$ и $a \in \Sigma$ определяме функцията на преходите $\Delta$ по следния начин:
  \begin{align*}
    \Delta(q,a) \df
    \begin{cases}
      \{\delta(q,a)\}, & \text{ако } q \not\in F\\
      \{\delta(q,a), \delta(\qstart,a) \}, & \text{ако } q \in F.
    \end{cases}
  \end{align*}
    
  Нека $\alpha = a_0a_1\cdots a_{n-1} \in \L(\N)$.
  Това означава, че $(\qstart,\alpha) \vdash^\star_\N (f,\varepsilon)$ за някое $f \in F$.
  Нека редицата от състояния $(q_i)^n_{i=0}$ описва едно приемащо изчисление на $\N$ върху $\alpha$, т.е.
  \begin{itemize}
  \item
    $q_0 = \qstart$;
  \item
    $q_{i+1} \in \Delta(q_i,a_i)$;
  \item
    $q_n \in F$.
  \end{itemize}
  Да разгледаме максималната подпоследователност от състояния $(q_{i_j})^{\ell+1}_{j = 0}$ на $(q_i)^{n}_{i=0}$ съставена от тези състояния, за които
  \mynote{Тук е малко по-сложно, защото правим разбиване на изчислението не на база всички финални състояния, а на тези финални състояния, от които изчислението продължава в насленик на $\qstart$,
  защото това са местата, където можем да разцепим думата на части.}
  \begin{itemize}
  \item
    $q_{i_0} = \qstart$;
  \item
    $q_{i_j} \in F\ \&\ \delta(\qstart,a_{i_j}) = q_{i_j+1}$, за $j = 1,\dots,\ell$;
  \item
    $q_{i_{\ell+1}} = q_n$.
  \end{itemize}
  Да разбием думата $\alpha$ като $\alpha = \alpha_0\alpha_1\cdots\alpha_l$, където:
  \begin{align*}
    & \alpha_0 \df a_{i_0} \alpha'_0\\
    & \alpha_1 \df a_{i_1}\alpha'_1\\
    & \cdots\\
    & \alpha_\ell \df a_{i_\ell}\alpha'_\ell.
  \end{align*}
  Сега можем да разбием изчислението на $\N$ върху $\alpha$ по следния начин:
  \begin{align*}
    (q_{i_0},\alpha_{0}\alpha_{1}\cdots \alpha_{\ell}) & \vdash^\star_\N (q_{i_1}, \alpha_{1}\cdots \alpha_{\ell}) & \comment{ q_{i_0} = \qstart}\\
                                                    & \vdash^\star_\N (q_{i_2},\alpha_{2}\cdots \alpha_{\ell})\\
                                                    & \cdots\\
                                                    & \vdash^\star_\N (q_{i_{\ell}}, \alpha_{\ell})\\
                                                    & \vdash^\star_\N(q_{i_{\ell+1}},\varepsilon). & \comment{ q_{i_{\ell+1}} = q_n \in F}
  \end{align*}
  Да разгледаме само първата част на изчислението:
  \[\underbrace{(q_{i_0},\alpha_0) \vdash^\star_\N (q_{i_1}, \varepsilon)}_{\text{само преходи от }\A}.\]
  Понеже $q_{i_0} = \qstart$ и $q_{i_1} \in F$, то е ясно, че $\alpha_0 \in \L(\A)$.
  
  За $j = 0,\dots,\ell$, изчислението
  \[(q_{i_j},\alpha_j) \vdash^\star_\N (q_{i_{j+1}},\varepsilon)\]
  може по-подробно да се запише и така:
  \[(q_{i_j}, a_{i_{j}}\alpha'_j) \vdash_\N \underbrace{(q_{i_j+1}, \alpha'_j)  \vdash^\star_\N (q_{i_{j+1}},\varepsilon)}_{\text{само преходи от }\A}.\]
  Понеже имаме, че $\delta(\qstart,a_{i_j}) = q_{i_j+1}$, то оттук следва, че:
  \[\underbrace{(\qstart,a_{i_j}\alpha'_{j}) \vdash_\N (q_{i_j+1}, \alpha'_{j})}_{\text{преход от }\A}  \vdash^\star_\A (q_{i_{j+1}},\varepsilon).\]
  Заключаваме, че
  \[(\qstart,\alpha_{j}) \vdash^\star_\A (q_{i_{j+1}},\varepsilon).\]
  Понеже $q_{i_{j+1}} \in F$, веднага следва, че $\alpha_{j} \in \L(\A)$.
  От всичко дотук заключваме, че $\alpha \in (\L(\A))^+$.

  За другата посока, нека $\alpha \in (\L(\A))^+$.
  Това означава, че думата $\alpha$ може да се представи като
  $\alpha = \alpha_0 \cdot \alpha_1 \cdots \alpha_\ell$, където $\alpha_j \in \L(\A)$ и $\alpha_j \neq \varepsilon$, за $j = 0,\dots, \ell$.
  Нека за $j=0,\dots,\ell$ да положим
  \[\alpha_j \df a_j \cdot \alpha'_j\text{ и } q_{j} \df \delta(\qstart, a_j).\]
  Оттук получаваме за $j = 0,\dots, \ell$ следните изчисления:
  \[(\qstart, \alpha_{j}) \vdash_\A (q_{j}, \alpha'_j) \vdash^\star_\A (f_{j}, \varepsilon), \text{ за някои }f_j \in F.\]
  Понеже функцията на преходите на $\N$ разширява функцията на преходите на $\A$, то е ясно е, че имаме също така и следното:
  \[(\qstart, \alpha_{j}) \vdash_\N (q_{j}, \alpha'_j) \vdash^\star_\N (f_{j}, \varepsilon), \text{ за някои }f_j \in F.\]

  За $1 \leq j < \ell$,
  понеже $\delta(\qstart,a_{j}) = q_{j}$ и $f_{j-1} \in F$, то според конструкцията на недетерминирания краен автомат $\N$,
  \[q_{j} \in \Delta(f_{j-1}, a_{j}).\]
  Оттук следва, че имаме следното изчисление на $\N$ върху $\alpha$:
  \begin{align*}
    &(\qstart, \alpha_{0}) \vdash^\star_\N (f_{0}, \varepsilon) & \comment\text{за някое }f_0 \in F\\
    &(f_0, \alpha_{1}) \vdash_\N (q_{1}, \alpha'_1) \vdash^\star_\N (f_{1}, \varepsilon) & \comment\text{за някое }f_1 \in F\\
    &(f_1, \alpha_{2}) \vdash_\N (q_{2}, \alpha'_2) \vdash^\star_\N (f_{2}, \varepsilon) & \comment\text{за някое }f_2 \in F\\
    &\dots\\
    &(f_{\ell-1}, \alpha_{\ell}) \vdash_\N (q_{\ell}, \alpha'_{\ell}) \vdash^\star_\N (f_{\ell}, \varepsilon). & \comment\text{за някое }f_{\ell} \in F
  \end{align*}
  Обединявайки всичко това, заключаваме, че  $\alpha \in \L(\N)$.

  Така доказахме, че $\L(\N) = (\L(\A))^+$.
  Сега ще построим недетерминиран краен автомат $\N' = \pair{\Sigma,Q',\qstart',\Delta',F'}$, такъв че:
  \[\L(\N') = \L(\N)\cup\{\varepsilon\} = (\L(\A))^+ \cup \{\varepsilon\} = (\L(\A))^\star.\]
  \begin{itemize}
  \item
    $Q' \df Q \cup \{\qstart'\}$;
  \item
    $F' \df F \cup \{\qstart'\}$;
  \item
    $\Delta'(\qstart',a) \df \Delta(\qstart,a)$, за всяко $a \in \Sigma$;
  \item
    $\Delta'(q,a) \df \Delta(q,a)$, за всяко $a \in \Sigma$ и $q\in Q$.
  \end{itemize}
  Лесно се съобразява, че $\L(\N') = \L(\N)\cup\{\varepsilon\}$.
\end{proof}

\begin{example}
  Нека да приложим конструкцията за да намерим автомат разпознаващ $\L(\A)^\star$.
  
  \begin{figure}[H]
    % \center
    \begin{subfigure}[b]{0.3\textwidth}
      \begin{tikzpicture}[framed,->,>=stealth,thick,node distance=55pt,initial text=начало]
        \tikzstyle{every state}=[circle,minimum size=20pt,auto]
        \node[initial below,state] (1) {$q_0$};
        \node[state]               (2) [right of=1] {$q_1$};
        \node[state,accepting]     (3) [right of=2] {$q_2$};
        \node[state,accepting]     (4) [above of=2] {$q_3$};
        \node[state]               (5) [above of=3] {$q_4$};
        \path
        (1) edge [bend right=15] node [below] {$a$} (2)
        (1) edge [bend left=15]  node [above] {$b$} (4)
        (2) edge [bend left=15]  node [left] {$a$} (4)
        (2) edge [bend right=15] node [below] {$b$} (3)
        (5) edge [loop above]    node [above] {$a,b$} (5)
        (4) edge [bend left=15]  node [above] {$a,b$} (5)
        (3) edge [bend right=15] node [right] {$b$} (5)
        (3) edge [bend left=45]  node [below] {$a$} (1);
      \end{tikzpicture}
      \caption{автомат $\A$}
    \end{subfigure}
    \hspace{2cm}
    \begin{subfigure}[b]{0.5\textwidth}
      \begin{tikzpicture}[framed,->,>=stealth,thick,node distance=55pt,initial text=начало]
        \tikzstyle{every state}=[circle,minimum size=20pt,auto]
        
        \node[initial below, state] (1) [below right of=0] {$q_0$};
        \node[state]                (2) [right of=1] {$q_1$};
        \node[state,accepting]      (3) [right of=2] {$q_2$};
        \node[state,accepting]      (4) [above of=2] {$q_3$};
        \node[state]                (5) [above of=3] {$q_4$};
        \path
        (1) edge [bend left=15] node [above] {$b$} (4)
        (1) edge [bend right=15] node [below] {$a$} (2)
        (2) edge [bend right=15] node [below] {$b$} (3)
        (2) edge [bend left=15]  node [left] {$a$} (4)
        (3) edge [bend left=45]  node [below] {$a$} (1)
        (3) edge [dashed, bend right=20] node [above] {$b$} (4)
        (4) edge [dashed,bend left=15] node [right] {$a$} (2)
        (4) edge [dashed, loop above] node {$b$} (4)
        (5) edge [loop above] node [above] {$a,b$} (5)
        (4) edge [bend left=15] node [above] {$a,b$} (5)
        (3) edge [bend right=15] node [right] {$b$} (5)
        (3) edge [dashed, bend right=15] node [above] {$a$} (2);        
      \end{tikzpicture}
      \caption{$\L(\N) = \L(\A)^+$}
    \end{subfigure}
  \end{figure}
    
  \mynote{Лесно се вижда, че $\L(\A) = \{b\} \cup \{ab\}\cdot\{aba\}^\star$}
  След като построим автомат за езика $\L(\A)^+$, трябва да приложим
  конструкцията за обединение на автомата за езика $\L(\A)^+$ с автомата за езика $\{\varepsilon\}$.
  Защо трябва да добавим ново начално състояние $q_0'$?
  Да допуснем, че вместо това сме направили $q_0$ финално.
  Тогава има опасност да разпознаем повече думи. Например, думата $aba$ би се разпознала от този автомат,
  но $aba \not\in\L(\A)^\star$.
  
  \begin{figure}[H]
    \centering
      \begin{tikzpicture}[framed,->,>=stealth,thick,node distance=55pt,initial text=начало]
        \tikzstyle{every state}=[circle,minimum size=20pt,auto]
        \node[initial, state, accepting] (0) [above of=1] {$q'_0$};
        \node[initial,state] (1) [below of=0] {$q_0$};
        \node[state]                (2) [right of=1] {$q_1$};
        \node[state,accepting]      (3) [right of=2] {$q_2$};
        \node[state,accepting]      (4) [above of=2] {$q_3$};
        \node[state]                (5) [above of=3] {$q_4$};
        \path
        % (0) edge [dashed, bend left=15] node [above] {$b$} (4)
        % (0) edge [dashed, bend left=15] node [above] {$a$} (2)
        (1) edge [bend left=15] node [above] {$b$} (4)
        (1) edge [bend right=15] node [below] {$a$} (2)
        (2) edge [bend right=15] node [below] {$b$} (3)
        (2) edge [bend left=15]  node [left] {$a$} (4)
        (3) edge [bend left=45]  node [below] {$a$} (1)
        (3) edge [dashed, bend right=20] node [above] {$b$} (4)
        (4) edge [dashed,bend left=15] node [right] {$a$} (2)
        (4) edge [dashed, loop above] node {$b$} (4)
        (5) edge [loop above] node [above] {$a,b$} (5)
        (4) edge [bend left=15] node [above] {$a,b$} (5)
        (3) edge [bend right=15] node [right] {$b$} (5)
        (3) edge [dashed, bend right=15] node [above] {$a$} (2);        
      \end{tikzpicture}
      \caption{$\L(\N') = \L(\N) \cup \{\varepsilon\} = \L(\A)^\star$}    
  \end{figure}

\end{example}

%%% Local Variables:
%%% mode: latex
%%% TeX-master: "../eai"
%%% End:
