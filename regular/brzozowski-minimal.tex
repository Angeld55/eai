\section{Минимален автомат}

Сега ще видим, че автоматът на Бжозовски $\B$ за даден регулярен език $L$ е в известен смисъл най-добрият възможен.
Накратко, $\B$ има най-малкия възможен брой състояния измежду всички детерминирани крайни автомати, които разпознават $L$. За да успеем да видим това, първо трябва да се подготвим.

Нека $\A = \FA$ е ДКА. За всяко състояние $q$ на $\A$ да разгледаме езика
\[\L_\A(q) \df \{\omega \in \Sigma^\star \mid \delta^\star(q,\alpha) \in F\}.\]
В частност имаме, че $\L(\A) = \L_\A(\qstart)$.
Без ограничение на общността, нека приемем, че винаги разглеждаме само свързани ДКА $\A$, т.е.
всяко състояние е достижимо от началното.
\mynote{Тук използваме, че $\delta$ е тотална функция. За някои състояния $p$ може да съществуват безкрайно много думи $\alpha$, за които $q_\alpha = p$.}
Нека за всяка дума $\alpha$ да положим $q_\alpha \df \delta^\star(\qstart,\alpha)$.
Понеже $\A$ е свързан, то всяко състояние на $\A$ може да се разглежда като $q_\alpha$ за някоя дума $\alpha$.

\begin{proposition}\label{pr:well-defined-pullback}
  Нека $L = \L(\A)$. Тогава за всяка дума $\alpha$ е изпълнено, че:
  \[\L_\A(q_\alpha) = \alpha^{-1}(L).\]
\end{proposition}
\begin{proof}
  За произволна дума $\gamma$ имаме следните еквивалентности:
  \begin{align*}
    \gamma \in \L_\A(q_\alpha) & \iff \delta^\star(q_\alpha,\gamma) \in F & \comment \text{от деф. на }\L_\A(q_\alpha)\\
                               & \iff \delta^\star(\delta^\star(\qstart,\alpha),\gamma) \in F & \comment q_\alpha \df \delta^\star(\qstart,\alpha)\\
                               & \iff \delta^\star(\qstart,\alpha\gamma) \in F & \comment\text{\Proposition{dfa:delta-star}}\\
                               & \iff \alpha\gamma \in \L(\A) & \comment \text{деф. на }\L(\A)\\
                               & \iff \gamma \in \alpha^{-1}(L). & \comment{L = \L(\A)}
  \end{align*}
\end{proof}

\begin{proposition}
  Нека $\B$ е автомат на Бжозовски за езика $L$. Тогава за произволно състояние $\hat{M}$ на $\B$,
  \[\L_\B(\hat{M}) = M.\]
\end{proposition}
\begin{proof}
  Понеже за произволна дума $\alpha$ имаме еквивалентностите
  \begin{align*}
    \alpha \in \L_\B(\hat{M}) & \iff \delta^\star_\B(\hat{M},\alpha) \in F^\B\\
                              & \iff \varepsilon \in \alpha^{-1}(M)\\
                              & \iff \alpha \in M,
  \end{align*}
  заключаваме, че $\L_\B(\hat{M}) = M$.
\end{proof}


\begin{proposition}\label{pr:surjective-cardinality}
  Нека $A$ и $B$ са множества, като $A$ е крайно, за които съществува \emph{сюрективна} функция $f: A \to B$.
  Тогава $|B| \leq |A|$.
\end{proposition}
\begin{hint}
  Понеже $A$ е крайно множество, можем да изброим елементите му в редица.
  Нека $A = \{a_0,a_1, \dots, a_{n-1}\}$. Разгледайте $g:B \to A$, където
  \mynote{Можем да вземем и $\max$ вместо $\min$.} 
  \[g(b) \df a_m\text{ за }m = \min\{i < n \mid f(a_i) = b\}.\]
  Докажете, че $g$ е инективна.
\end{hint}

\begin{lemma}\label{lem:brzozowski:surjective}
  Нека $L$ е регулярен език и $\A$ е ДКА, който разпознава $L$,
  а $\B$ е автоматът на Бжозовски за $L$. Тогава $|Q^\B| \leq |Q^\A|$.
\end{lemma}
\mynote{Тази лема ни казва, че автоматът на Бжозовски има възможно най-малкия брой състояния измежду всички ДКА за $L$.}
\begin{proof}
  Да разгледаме функцията $f:Q^\A \to Q^\B$ зададена по следния начин:
  \[f(q) \df \hat{M}, \text{ където } M = \L_\A(q).\]
  Нека първо да съобразим защо $f$ е добре дефинирана функция.
  Да напомним, че приехме, че $\A$ е свързан автомат. Тогава за произволно състояние $p$,
  нека $\alpha$ е една дума, за която $p = \delta^\star(\qstart,\alpha)$, т.е. $p = q_\alpha$.  
  Тогава от \Proposition{well-defined-pullback} следва, че $f(q_\alpha) = \hat{M}$, където $M = \alpha^{-1}(L)$.
  
  Освен това, $f$ е сюрективна, защото за произволно $\hat{M} \in Q^\B$ има дума $\alpha$,
  за която $M = \alpha^{-1}(L)$. Тогава $f(q_\alpha) = \hat{M}$.
  Сега от \Proposition{surjective-cardinality} можем да заключим, че
  \[|Q^\B| \leq |Q^\A|.\]
\end{proof}

Така получаваме следния критерий за проверка дали даден език е регулярен.
\begin{framed}
  \begin{corollary}\label{cor:brzozowski:finite}
    Един език $L$ е регулярен точно тогава, когато автоматът $\B$ на Бжозовски за $L$, има крайно много състояния.
  \end{corollary}
\end{framed}
\begin{proof}
  Нека $L$ е регулярен. Тогава $L = \L(\A)$ за някой ДКА $\A$. Да напомним, че от \Proposition{brzozowski:language} също имаме и $\L(\B) = L$.
  Понеже $|Q^\B| \leq |Q^\A|$, то $\B$ има краен брой състояния.
  Обратно, ако $\B$ има крайно много състояния, то тогава $\B$ е ДКА. Понеже $\L(\B) = L$, то $L$ е автоматен и следователно регулярен.
\end{proof}


\begin{corollary}
  Нека $L$ е регулярен език. Тогава автоматът $\B$, построен по метода на Бжозовски за $L$, има минималния възможен брой състояния
  измежду всички детерминирани крайни автомати разпознаващи $L$.
\end{corollary}  

\begin{proposition}\label{pr:surjection-bijection}
  Нека $A$ и $B$ са крайни равномощни множества.
  Докажете, че ако $g:A \to B$ е сюрекция, то $g$ е биекция.
\end{proposition}
\mynote{Ясно е, че щом $A$ и $B$ са равномощни, то има биекция между тях. Тук доказваме, че всяка сюрекция между тях е също така и биекция.}
\begin{proof}
  Нека $A = \{a_0,\dots,a_{n-1}\}$ и $B = \{b_0,\dots,b_{n-1}\}$.
  Нека, за всеки индекс $i < n$,
  \[A_i \df \{a_j \in A \mid g(a_j) = b_i\}.\]
  Щом $g$ е сюрекция, то $A_i \neq \emptyset$ за всеки индекс $i < n$.
  Понеже $g$ е функция, то $A_i \cap A_j = \emptyset$ за всеки два различни индекса $i$ и $j$.
  \mynote{Да напомним формулата $|A\cup B| = |A| + |B| + |A \cap B|$. Просто в нашия случай $|A \cap B| = 0$.}
  Това означава, че
  \[n = |A| = |\bigcup_{i<n} A_i| = \sum_{i<n}|A_i|.\]
  Оттук следва, че щом за всяко $i$ имаме, че $|A_i| \neq 0$, то $|A_i| = 1$.
  Заключаваме, че $g$ е инекция, защото в противен случай щяхме да имаме някое $i$, за което $|A_i| > 1$.
\end{proof}

\begin{framed}
  \begin{theorem}\label{th:brzozowski-minimal:unique}
    За всеки регулярен език $L$ съществува единствен минимален ДКА с точност до изоморфизъм.
  \end{theorem}  
\end{framed}
\mynote{С други думи, ако имаме два минимални ДКА за $L$, то можем да получим единия автомат от другия чрез внимателно преименуване на състоянията.}
\begin{proof}
  Вече знаем, че автоматът $\B$, построен по метода на Бжозовски за $L$, има минималния възможен брой състояния.
  Нека $\A$ е друг ДКА разпознаващ $L$ и $|Q^\A| = |Q^\B|$. Трябва да докажем, че $\A \cong \B$.
  Ясно е, че $\A$ е свързан автомат, т.е. всяко състояние $p$ на $\A$ е от вида $p = q_\alpha$.
  В противен случай, $\A$ нямаше да бъде минимален.
  Да разгледаме функцията $f:Q^\A \to Q^\B$, където
  \[f(p) = \hat{M}\ \dff\ M = \L_\A(p).\]
  От \Proposition{well-defined-pullback} знаем, че $f(q_\alpha) = \L_\A(q_\alpha) = \alpha^{-1}(L)$ следва, че $f$ е сюрективна.
  Понеже $|Q^\A| = |Q^\B|$, то от \Proposition{surjection-bijection} имаме, че $f$ е всъщност биекция.
  Остава да видим защо $\A \cong_f \B$.
  Да напомним, че състоянията на $Q^\A$ можем да ги разглеждаме като $q_\alpha$ и тогава $\L_\A(q_\alpha) = \alpha^{-1}(L)$ от \Proposition{well-defined-pullback}.
  \begin{itemize}
  \item
    За началното състояние имаме, че:
    \begin{align*}
      f(\qstart^\A) & = f(q_\varepsilon) & \comment{\qstart^\A = q_\varepsilon}\\
                 & = \hat{L}. & \comment{\varepsilon^{-1}(L) = L}
    \end{align*}    
  \item
    За финалните състояния имаме, че:
    \begin{align*}
      q_\alpha \in F^\A & \iff \delta^\star_\A(\qstart,\alpha) \in F^\A & \comment q_\alpha = \delta^\star_\A(\qstart,\alpha)\\
                        & \iff \alpha \in \L(\A) & \comment\text{деф. на }\L(\A)\\
                        & \iff \varepsilon \in \alpha^{-1}(L) & \comment{L = \L(\A)}\\
                        & \iff \hat{M} \in F^\B. & \comment\text{ за }M = \alpha^{-1}(L)
    \end{align*}
  \item
    Остава да докажем, че за произволна буква $b$ и произволни състояния $q,p \in Q_1$ е изпълнено, че:
    \[\delta_\A(q,b) = p\ \iff\ \delta_\B(f(q),b) = f(p).\]
    Понеже всяко състояние на $\A$ може да се запише като $q_\alpha$ за някое $\alpha$, то можем да запишем горния ред и така:
    \[\delta_\A(q_\alpha,b) = p\ \iff\ \delta_\B(f(q_{\alpha}),b) = f(p).\]
    Нека положим
    \begin{align*}
      M & \df \alpha^{-1}(L)\\
      N & \df (\alpha b)^{-1}(L) = b^{-1}(M).    
    \end{align*}
    От \Proposition{well-defined-pullback} знаем, че $\hat{M} = f(q_\alpha)$, защото
    $\L_\A(q_\alpha) = \alpha^{-1}(L)$.
    
    За посоката $(\Rightarrow)$ имаме, че $p = q_{\alpha b}$. 
    Тогава $f(q_{\alpha b}) = \hat{N}$, защото $(\alpha b)^{-1}(L) = \L_\A(q_{\alpha b})$.
    Остава единствено да отбележим, че от конструкцията на автомата $\B$ знаем, че:
    \[\delta_\B(\underbrace{\hat{M}}_{f(q_{\alpha})},b) = \underbrace{\hat{N}}_{f(p)}.\]
    За посоката $(\Leftarrow)$, знаем, че $\delta_\B(\hat{M},b) = \hat{N} = f(p)$.
    Знаем, че $N = \L_\A(p) = (\alpha b)^{-1}(L) = \L_\A(q_{\alpha b})$.
    Понеже $f$ е биекция, то $p = q_{\alpha b}$.
    Заключаваме, че \[\delta_\A(q_\alpha,b) = p.\]

    \begin{figure}[H]
      \centering
     \begin{tikzpicture}[framed,->,>=stealth,thick,node distance=60pt]
        \tikzstyle{every state}=[circle,minimum size=20pt,auto]
        
        \node[state,initial]                  (1)              {$q_\varepsilon$};
        \node[state]                  (2) [right of=1] {$q_\alpha$};
        \node[state]                  (3) [right of=2] {$q_{\alpha b}$};

        \node[state,initial]                  (4) [below of=1] {$\hat{L}$};
        \node[state]                  (5) [below of=2] {$\hat{M}$};
        \node[state]                  (6) [below of=3] {$\hat{N}$};
        
        \path
        (2) edge [bend left=15]  node [above] {$b$} (3)
        (5) edge [bend right=15] node [below] {$b$} (6)
        (1) edge [dashed, bend right=15] node [left]  {$f$} (4)
        (2) edge [dashed, bend right=15]  node [right] {$f$} (5)
        (3) edge [dashed, bend left=15]  node [right] {$f$} (6);

        \draw [photon] (1) -- node [above] {$\alpha$} (2);
        \draw [photon] (4) -- node [below] {$\alpha$} (5);
      \end{tikzpicture}
      \caption{Визуализация на последното свойство за $\A \cong_f \B$.}
    \end{figure}
    
  \end{itemize}
\end{proof}

%%% Local Variables:
%%% mode: latex
%%% TeX-master: "../eai"
%%% End:
