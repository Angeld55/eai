\section{Минимален автомат}
\label{sect:regular:brzozowski-minimal}

Сега ще видим, че автоматът на Бжозовски $\B$ за даден регулярен език $L$ е в известен смисъл най-добрият възможен.
Накратко, $\B$ има най-малкия възможен брой състояния измежду всички детерминирани крайни автомати, които разпознават $L$. За да успеем да видим това, първо трябва да се подготвим.

Нека $\A = \FA$ е ДКА. За всяко състояние $q$ на $\A$ да разгледаме езика
\[\L_\A(q) \df \{\omega \in \Sigma^\star \mid \delta^\star(q,\alpha) \in F\}.\]
В частност имаме, че $\L(\A) = \L_\A(\qstart)$.
Без ограничение на общността, нека приемем, че винаги разглеждаме само свързани ДКА $\A$, т.е.
всяко състояние е достижимо от началното.
\mynote{Тук използваме, че $\delta$ е тотална функция. За някои състояния $p$ може да съществуват безкрайно много думи $\alpha$, за които $q_\alpha = p$.}
Нека за всяка дума $\alpha$ да положим $q_\alpha \df \delta^\star(\qstart,\alpha)$.
Понеже $\A$ е свързан, то всяко състояние на $\A$ може да се разглежда като $q_\alpha$ за някоя дума $\alpha$.

\begin{proposition}\label{pr:well-defined-pullback}
  Нека $L = \L(\A)$. Тогава за всяка дума $\alpha$ е изпълнено, че:
  \[\L_\A(q_\alpha) = \alpha^{-1}(L).\]
\end{proposition}
\begin{proof}
  За произволна дума $\omega$ имаме следните еквивалентности:
  \begin{align*}
    \omega \in \L_\A(q_\alpha) & \iff \delta^\star(q_\alpha,\omega) \in F & \comment \text{от деф. на }\L_\A(q_\alpha)\\
                               & \iff \delta^\star(\delta^\star(\qstart,\alpha),\omega) \in F & \comment q_\alpha \df \delta^\star(\qstart,\alpha)\\
                               & \iff \delta^\star(\qstart,\alpha\omega) \in F & \comment\text{\Proposition{dfa:delta-star}}\\
                               & \iff \alpha\omega \in \L(\A) & \comment \text{деф. на }\L(\A)\\
                               & \iff \alpha\omega \in L & \comment{L = \L(\A)}\\
                               & \iff \omega \in \alpha^{-1}(L). & \comment\text{деф. на }\alpha^{-1}(L)
  \end{align*}
\end{proof}

\begin{important}
  \begin{proposition}
    Нека $\B$ е автомат на Бжозовски за езика $L$. Тогава за произволно състояние $M$ на $\B$ имаме равенството:
    \[\L_\B(\underbrace{M}_{\text{състояние}}) = \underbrace{M}_{\text{език}}.\]
  \end{proposition}  
\end{important}
\begin{proof}
  Понеже за произволна дума $\alpha$ имаме еквивалентностите
  \begin{align*}
    \alpha \in \L_\B(M) & \iff \delta^\star_\B(M,\alpha) \in F^\B\\
                        & \iff \varepsilon \in \alpha^{-1}(M) & \comment\text{деф. на }F^\B\\
                        & \iff \alpha \in M, & \comment\text{от \Problem{language-pullback}}
  \end{align*}
  заключаваме, че $\L_\B(M) = M$.
\end{proof}


\begin{proposition}\label{pr:surjective-cardinality}
  Нека $A$ и $B$ са множества, като $A$ е крайно, за които съществува \emph{сюрективна} функция $f: A \to B$.
  Тогава $|B| \leq |A|$.
\end{proposition}
\begin{hint}
  Понеже $A$ е крайно множество, можем да изброим елементите му в редица.
  Нека $A = \{a_0,a_1, \dots, a_{n-1}\}$. Разгледайте $g:B \to A$, където
  \mynote{Вярно ли е това твърдение, ако $A$ е безкрайно ?}
  \[g(b) \df a_m\text{ за }m = \min\{i < n \mid f(a_i) = b\}.\]
  Да отбележим, че дефиницията на $g$ е коректна, защото множеството $\{i < n \mid f(a_i) = b\}$ е непразно, понеже $f$ е сюректвина.
  Докажете, че $g$ е инективна.
\end{hint}

\begin{lemma}
  \label{lem:brzozowski:surjective}
  Нека $L$ е регулярен език и $\A$ е ДКА, който разпознава $L$,
  а $\B$ е автоматът на Бжозовски за $L$. Тогава $|Q^\B| \leq |Q^\A|$.
\end{lemma}
\mynote{Тази лема ни казва, че автоматът на Бжозовски има възможно най-малкия брой състояния измежду всички ДКА разпознаващи $L$.}
\begin{proof}
  Да разгледаме функцията $f:Q^\A \to Q^\B$ зададена по следния начин:
  \begin{equation}
    \label{eq:reg:brzozowski-minimal:f}
    f(q) \df \L_\A(q).
  \end{equation}
  Първо, да видим защо за всяко състояние $q \in Q^\A$, то $f(q) \in Q^\B$.
  Да напомним, че приехме още в началото на раздела приехме, че $\A$ е свързан автомат.
  Това означава, че за всяко състояние $p$, съществува дума $\alpha$, за която $p = \delta^\star_\A(\qstart^\A,\alpha)$,
  т.е. $p = q_\alpha$ според означението, което въведохме в началото на \Section{regular:brzozowski-minimal}.
  Тогава от \Proposition{well-defined-pullback} следва, че $f(q_\alpha) = \alpha^{-1}(L) \in Q^\B$, защото $\alpha^{-1}(L) = \L_\A(q_\alpha)$.
  
  % нека първо да съобразим защо $f$ е добре дефинирана.% , т.е. защо дефиницията (\ref{eq:reg:brzozowski-minimal:f}) задава функция.
  % Да напомним, че приехме, че $\A$ е свързан автомат. Тогава за произволно състояние $p$,
  % нека $\alpha$ е една дума, за която $p = \delta^\star(\qstart,\alpha)$, т.е. $p = q_\alpha$
  % според означението, което въведохме в началото на \Section{regular:brzozowski-minimal}.
  % Тогава от \Proposition{well-defined-pullback} следва, че $f(q_\alpha) = \alpha^{-1}(L)$, защото $\alpha^{-1}(L) = \L_\A(q_\alpha)$.

  Второ, да видим, че $f$ е сюрективна функция. За тази цел, да разгледаме произволно състояние $M \in Q^\B$, което означава, че има дума $\alpha$, за която $M = \alpha^{-1}(L)$.
  Отново според \Proposition{well-defined-pullback}, $f(q_\alpha) = \L_\A(q_\alpha) = \alpha^{-1} = M$.
  Сега от \Proposition{surjective-cardinality} можем да заключим, че
  \[|Q^\B| \leq |Q^\A|.\]
\end{proof}

Така получаваме следния критерий за проверка дали даден език е регулярен.
\begin{framed}
  \begin{corollary}\label{cor:brzozowski:finite}
    Един език $L$ е регулярен точно тогава, когато автоматът $\B$ на Бжозовски за $L$, има крайно много състояния.
  \end{corollary}
\end{framed}
\begin{proof}
  Нека $L$ е регулярен. Тогава $L = \L(\A)$ за някой ДКА $\A$. Да напомним, че от \Proposition{brzozowski:language} също имаме и $\L(\B) = L$.
  Понеже $|Q^\B| \leq |Q^\A|$, то $\B$ има краен брой състояния.
  Обратно, ако $\B$ има крайно много състояния, то тогава $\B$ е ДКА. Понеже $\L(\B) = L$, то $L$ е автоматен и следователно регулярен.
\end{proof}


\begin{corollary}
  Нека $L$ е регулярен език. Тогава автоматът $\B$, построен по метода на Бжозовски за $L$, има минималния възможен брой състояния
  измежду всички детерминирани крайни автомати разпознаващи $L$.
\end{corollary}  

\begin{proposition}\label{pr:surjection-bijection}
  Нека $A$ и $B$ са крайни равномощни множества.
  Докажете, че ако $g:A \to B$ е сюрекция, то $g$ е биекция.
\end{proposition}
\mynote{Ясно е, че щом $A$ и $B$ са равномощни, то има биекция между тях. Тук доказваме, че всяка сюрекция между тях е също така и биекция.}
\begin{proof}
  Нека $A = \{a_0,\dots,a_{n-1}\}$ и $B = \{b_0,\dots,b_{n-1}\}$.
  Нека, за всеки индекс $i < n$ да положим
  \[A_i \df \{a_j \in A \mid g(a_j) = b_i\}.\]
  Щом $g$ е сюрекция, то $A_i \neq \emptyset$ за всеки индекс $i < n$.
  Понеже $g$ е функция, то $A_i \cap A_j = \emptyset$ за всеки два различни индекса $i$ и $j$.
  \mynote{Да напомним, че от курса по Дискретна математика имаме формулата $|X\cup Y| = |X| + |Y| + |X \cap Y|$. Просто в нашия случай $|X \cap Y| = 0$.}
  Това означава, че
  \[n = |A| = |\bigcup_{i<n} A_i| = \sum_{i<n}|A_i|.\]
  Оттук следва, че щом за всяко $i$ имаме, че $|A_i| \neq 0$, то $|A_i| = 1$.
  Заключаваме, че $g$ е инекция, защото в противен случай щяхме да имаме някое $i$, за което $|A_i| > 1$.
\end{proof}

\begin{framed}
  \begin{theorem}\label{th:brzozowski-minimal:unique}
    За всеки регулярен език $L$ съществува единствен минимален ДКА с точност до изоморфизъм.
  \end{theorem}  
\end{framed}
\mynote{С други думи, ако имаме два минимални ДКА за $L$, то можем да получим единия автомат от другия чрез внимателно преименуване на състоянията.}
\begin{proof}
  Вече знаем, че автоматът $\B$, построен по метода на Бжозовски за $L$, има минималния възможен брой състояния.
  Нека $\A$ е друг ДКА разпознаващ $L$ и $|Q^\A| = |Q^\B|$. Трябва да докажем, че $\A \cong \B$.
  Ясно е, че $\A$ е свързан автомат, т.е. всяко състояние $p$ на $\A$ е от вида $p = q_\alpha$.
  В противен случай, $\A$ нямаше да бъде минимален.
  Да разгледаме функцията $f:Q^\A \to Q^\B$, където
  \begin{equation}
    \label{eq:reg:brzozowski-minimal:f:2}
    f(p) \df \L_\A(p).
  \end{equation}
  От \Lemma{brzozowski:surjective} знаем, че $f$ е сюрективна.
  % От \Proposition{well-defined-pullback} знаем, че $f(q_\alpha) = \L_\A(q_\alpha) = \alpha^{-1}(L)$ следва, че $f$ е сюрективна.
  Понеже $|Q^\A| = |Q^\B|$, то от \Proposition{surjection-bijection} имаме, че $f$ е всъщност биекция.
  Остава да видим защо $\A \cong_f \B$.
  Да напомним, че състоянията на $Q^\A$ можем да ги разглеждаме като $q_\alpha$ и тогава $\L_\A(q_\alpha) = \alpha^{-1}(L)$ от \Proposition{well-defined-pullback}.
  \begin{itemize}
  \item
    За началното състояние имаме, че:
    \begin{align*}
      f(\qstart^\A) & = f(q_\varepsilon) & \comment{\qstart^\A = q_\varepsilon}\\
                    & = \L_\A(q_\varepsilon) & \comment\text{от деф. на }f\\
                    & = \varepsilon^{-1}(L) & \comment\text{\Proposition{well-defined-pullback}}\\
                    & = L.
    \end{align*}    
  \item
    За финалните състояния имаме, че:
    \begin{align*}
      q_\alpha \in F^\A & \iff \delta^\star_\A(\qstart,\alpha) \in F^\A & \comment q_\alpha = \delta^\star_\A(\qstart,\alpha)\\
                        & \iff \alpha \in \L(\A) & \comment\text{деф. на }\L(\A)\\
                        & \iff \varepsilon \in \alpha^{-1}(L) & \comment{L = \L(\A)}\\
                        & \iff \alpha^{-1}(L) \in F^\B. & \comment\text{деф. на }F^\B
    \end{align*}
  \item
    \mynote{
     \begin{tikzpicture}[->,>=stealth,thick,node distance=45pt,initial text=начало,scale=0.8,every node/.style={scale=0.8}]
        \tikzstyle{every state}=[circle,minimum size=20pt,auto]
        
        \node[state,initial above]          (1)              {$q_\varepsilon$};
        \node[state]                  (2) [right of=1] {$q_\alpha$};
        \node[state, inner sep=2pt]                  (3) [right of=2] {$q_{\alpha b}$};

        \node[state, initial below, inner sep=2pt]   (4) [below of=1] {$L$};
        \node[state, inner sep=2pt]                  (5) [below of=2] {$L_\alpha$};
        \node[state, inner sep=2pt]                  (6) [below of=3] {$L_{\alpha b}$};
        
        \path
        (2) edge [bend left=15]  node [above] {$b$} (3)
        (5) edge [bend right=15] node [below] {$b$} (6)
        (1) edge [dashed, bend right=15] node [left]  {$f$} (4)
        (2) edge [dashed, bend right=15]  node [left] {$f$} (5)
        (3) edge [dashed, bend left=15]  node [right] {$f$} (6);

        \draw [photon] (1) -- node [above] {$\alpha$} (2);
        \draw [photon] (4) -- node [above] {$\alpha$} (5);
      \end{tikzpicture}

      Тук сме означили
      \begin{align*}
        & L_{\alpha} = \alpha^{-1}(L)\\
        & L_{\alpha b} = (\alpha b)^{-1}(L).
      \end{align*}
    }
    Остава да докажем, че за произволна буква $b$ и произволни състояния $q,p \in Q_1$ е изпълнено, че:
    \begin{equation}
      \label{eq:brzozowski-minimal:eq}
      \delta_\A(q,b) = p\ \iff\ \delta_\B(f(q),b) = f(p).
    \end{equation}
    Понеже всяко състояние на $\A$ може да се запише като $q_\alpha$ за някоя дума $\alpha$, то можем да запишем горния ред и така:
    \[\delta_\A(q_\alpha,b) = p\ \iff\ \delta_\B(f(q_{\alpha}),b) = f(p).\]
    % Нека положим
    % \begin{align*}
    %   M & \df \alpha^{-1}(L)\\
    %   N & \df (\alpha b)^{-1}(L) = b^{-1}(\alpha^{-1}(L)) = b^{-1}(M).    
    % \end{align*}
    От \Proposition{well-defined-pullback} знаем, че $M = f(q_\alpha)$, защото
    $\L_\A(q_\alpha) = \alpha^{-1}(L)$. 

    За посоката $(\Rightarrow)$, нека $p$ е състояние, за което имаме лявата страна на (\ref{eq:brzozowski-minimal:eq}), т.е.
    \[\delta_\A(q_\alpha,b) = p.\]
    Оттук веднага е ясно, че според нашите означения, $p = q_{\alpha b}$.
    Според дефиницията на $f$ имаме следното:
    \begin{align*}
      & f(q_{\alpha}) = \L_\A(q_\alpha) = \alpha^{-1}(L)\\
      & f(q_{\alpha b}) = \L_\A(q_{\alpha b}) = (\alpha b)^{-1}(L) = b^{-1}(\alpha^{-1}(L)).
    \end{align*}

    % $f(q_{\alpha b}) = \L_\A(q_{\alpha b}) = (\alpha b)^{-1}(L)$



    % разгледаме състоянието $p$, за което $\delta_\A(q_\alpha,b) = p$.
    % Ясно е, че $p = q_{\alpha b}$, защото
    % \[p = \delta_\A(q_\alpha,b) = \delta_\A(\delta^\star_\A(\qstart^\A,\alpha),b) = \delta^\star_\A(\qstart^\A,\alpha b).\]
    % Тогава имаме, че $f(p) = f(q_{\alpha b}) = N$, защото $(\alpha b)^{-1}(L) = \L_\A(q_{\alpha b})$ отново според \Proposition{well-defined-pullback}.
    Остава единствено да отбележим, че от конструкцията на автомата $\B$ знаем, че:
    \[\delta_\B(\underbrace{\alpha^{-1}(L)}_{f(q_{\alpha})},b) = \underbrace{(\alpha b)^{-1}(L)}_{f(p)}.\]
    Така получихме дясната страна на еквивалентността в Свойство~\ref{eq:brzozowski-minimal:eq}.
    
    За посоката $(\Leftarrow)$, нека $p$ е състояние, за което имаме дясната страна на (\ref{eq:brzozowski-minimal:eq}), т.е.
    \[\delta_\B(f(q_{\alpha}),b) = f(p).\]
    Според конструкцията на автомата $\B$,
    \[\delta_\B(\underbrace{\alpha^{-1}(L)}_{f(q_\alpha)},b) = \underbrace{(\alpha b)^{-1}(L)}_{f(p)}.\]
    Сега на помощ ни идва фактът, че $f$ е биекция.
    Понеже $f(p) = (\alpha b)^{-1}(L)$ и $f(q_{\alpha b}) = (\alpha b)^{-1}(L)$ и $f$ е биекция, то
    $p = q_{\alpha b}$ и следователно, $\delta_\A(q_\alpha, b) = p$.
    Така получихме дясната страна на еквивалентността (\ref{eq:brzozowski-minimal:eq}).
  \end{itemize}
\end{proof}

%%% Local Variables:
%%% mode: latex
%%% TeX-master: "../eai"
%%% End:
