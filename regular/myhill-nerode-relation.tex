\section{Релация на Майхил-Нероуд}

\begin{itemize}
\item
  \index{Майхил-Нероуд!релация}
  \index{$\approx_L$}
  \mynote{$\approx_L$ е известна като релация на Майхил-Нероуд}
  Нека $L \subseteq \Sigma^\star$ е език и нека $\alpha,\beta \in \Sigma^\star$.
  Казваме, че $\alpha$ и $\beta$ са {\bf еквивалентни относно} $L$, което записваме 
  като $\alpha \approx_L \beta$, когато:
  \[\alpha \approx_L \beta\ \dff\ \alpha^{-1}(L) = \beta^{-1}(L).\]
  С други думи, 
  \[\alpha \approx_L \beta \iff (\forall \omega \in \Sigma^\star)[\ \alpha\omega \in L \iff \beta\omega \in L\ ].\]
\item
  \mynote{Трябва ли $\A$ да е тотален?}
  Нека $\A = \FA$ е краен детерминиран автомат.
  \index{$\sim_\A$}
  Казваме, че две думи $\alpha,\beta \in \Sigma^\star$ са {\bf еквивалентни относно $\A$},
  което означаваме с $\alpha \sim_\A \beta$, ако 
  \[\delta^\star(\qstart,\alpha) = \delta^\star(\qstart,\beta).\]
\item
  Проверете, че $\approx_L$ и $\sim_\A$ са {\bf релации на еквивалентност}, т.е.
  те са рефлексивни, транзитивни и симетрични.
\item
  Класът на еквивалентност на думата $\alpha$ относно релацията $\approx_L$ означаваме като
  \[[\alpha]_L \df \{\beta \in \Sigma^\star \mid \alpha \approx_L \beta\}.\]
  Означаваме 
  \[\Sigma^\star/_{\approx_L} \df \{[\alpha]_L \mid \alpha \in \Sigma^\star\}.\]
  Тогава с $\abs{\Sigma^\star/_{\approx_L}}$ ще означаваме броя на класовете на еквивалентност на релацията $\approx_L$.
\item
  Класът на еквивалентност на думата $\alpha$ относно релацията $\sim_\A$ означаваме като
  \[[\alpha]_\A \df \{\beta \in \Sigma^\star \mid \alpha \sim_\A \beta\}.\]
  Означаваме 
  \[\Sigma^\star/_{\sim_\A} \df \{[\alpha]_\A \mid \alpha \in \Sigma^\star\}.\]
  С $\abs{\Sigma^\star/_{\sim_\A}}$ ще означаваме броя на класовете на еквивалентност на релацията $\sim_\A$.
\item
  Съобразете, че всяко състояние на $\A$, което е достижимо от началното състояние, определя клас на еквивалентност относно 
  релацията $\sim_\A$. Това означава, че функцията $g:\Sigma^\star/_\A \to Q$, където
  \[g([\alpha]_\A) \df \delta^\star(\qstart,\alpha)\]
  е инекция. Следователно,
  \[|\Sigma^\star/_{\sim_\A}| \leq |Q|.\]
  Ако в автоматът $\A$ няма недостижими от $\qstart$ състояния, то $g$ е биекция и съответно
  \[|\Sigma^\star/_{\sim_\A}| = |Q|.\]
\item
  Релациите $\approx_\L$ и $\sim_\A$ са дясно-инвариантни, т.е. за всеки две думи $\alpha$ и $\beta$
  е изпълнено:
  \mynote{\writedown Проверете!}
  \begin{align*}
    \alpha \sim_\A \beta  &\implies (\forall \gamma\in\Sigma^\star)[\alpha\gamma \sim_\A \beta\gamma],\\
    \alpha \approx_\L \beta & \implies (\forall \gamma\in\Sigma^\star)[\alpha\gamma \approx_\L \beta\gamma].
  \end{align*}
\end{itemize}

\begin{problem}
  Докажете, че за всяка дума $\alpha \in \Sigma^\star$ е изпълнено, че:
  \[\alpha \in L \iff [\alpha]_L \subseteq L.\]
\end{problem}

\begin{proposition}\label{pr:rel-finer}
  Нека $\A = \FA$ е ДКА и $L = \L(\A)$. Тогава за произволни думи $\alpha$ и $\beta$ е изпълнено, че 
  \[\alpha\sim_\A\beta \implies \alpha\approx_{L}\beta.\]
  С други думи, за всяка дума $\alpha$ е изпълнено, че $[\alpha]_\A \subseteq [\alpha]_L$.
\end{proposition}
\begin{proof}
  Да вземем две произволни думи $\alpha$ и $\beta$, за които $\alpha \sim_\A \beta$, т.е. $\delta^\star(\qstart, \alpha) = \delta^\star(\qstart,\beta)$.
  Ще проверим, че  $\alpha \approx_{L} \beta$, т.е. $\alpha^{-1}(L) = \beta^{-1}(L)$.
  За произволна дума $\gamma$ имаме следното:
  \begin{align*}
    \gamma \in \alpha^{-1}(L) & \iff \alpha\gamma \in L\\
                              % & \iff \alpha\gamma \in \L(\A) & \comment{L = \L(\A)}\\
                              & \iff \delta^\star(\qstart,\alpha\gamma)\in F & \comment{L = \L(\A)}\\
                              & \iff \delta^\star(\delta^\star(\qstart,\alpha),\gamma) \in F & \comment\text{\Proposition{dfa:delta-star}}\\
                              & \iff \delta^\star(\delta^\star(\qstart,\beta),\gamma) \in F & \comment{\text{защото }\alpha \sim_\A \beta}\\
                              & \iff \delta^\star(\qstart,\beta\gamma) \in F & \comment\text{\Proposition{dfa:delta-star}}\\
                              & \iff \beta\gamma \in \L(\A) & \comment\text{деф. на }\L(\A)\\
                              % & \iff \beta\gamma \in L & \comment{L = \L(\A)} \\
                              & \iff \gamma \in \beta^{-1}(L).
  \end{align*}
  Заключаваме, че 
  \[(\forall \alpha \in \Sigma^\star)(\forall \beta \in \Sigma^\star)[\ \alpha\sim_\A\beta \implies \alpha\approx_{L}\beta\ ].\]
\end{proof}

\begin{problem}
  Докажете, че за произволен ДКА $\A$ и $L = \L(\A)$,
  за всяка дума $\alpha$,
  \[[\alpha]_{L} = \bigcup_{\beta \in [\alpha]_{L}}[\beta]_\A.\]
\end{problem}
% \begin{hint}
%   Първо, нека $\gamma \in \bigcup_{\beta \in [\alpha]_{L}}[\beta]_\A$, т.е. за някое $\beta \in [\alpha]_L$ имаме, че $\gamma \in [\beta]_\A$.
%   Но щом $\beta \in [\alpha]_L$, то тогава $[\beta]_L = [\alpha]_L$, а от $\gamma \in [\beta]_\A$, по \Prop{rel-finer} следва, че $\gamma \in [\beta]_L = [\alpha]_L$.
%   Заключаваме, че $\bigcup_{\beta \in [\alpha]_{L}}[\beta]_\A \subseteq [\alpha]_L$.

%   Нека сега $\gamma \in [\alpha]_L$. Но тогава е ясно, че $\gamma \in [\gamma]_\A$ и следователно директно получаваме, че
%   $[\alpha]_L \subseteq \bigcup_{\beta \in [\alpha]_{L}}[\beta]_\A$.
% \end{hint}

\begin{problem}\label{prob:surjective-cardinality}
  Нека $A$ и $B$ са множества, като $A$ е крайно, за които съществува \emph{сюрективна} функция $f: A \to B$.
  Тогава $|B| \leq |A|$.
\end{problem}
\begin{hint}
  Нека $A = \{a_0,a_1, \dots, a_{n-1}\}$. Разгледайте $g:B \to A$, където
  \[g(b) = a_m\text{ за }m = \min\{i < n \mid f(a_i) = b\}.\]
  Докажете, че $g$ е инективна.
\end{hint}

\begin{proposition}\label{pr:approx-less-sim}
  Нека $\A$ е детерминиран краен автомат и $L = \L(\A)$.
  Тогава 
  \[\abs{\Sigma^\star/_{\approx_{L}}} \leq \abs{\Sigma^\star/_{\sim_\A}}.\]
\end{proposition}
\begin{hint}
  Да разгледаме функцията $f: \Sigma^\star/_{\sim_\A} \to \Sigma^\star/_{\approx_L}$, където
  \[f([\alpha]_\A) \df [\alpha]_L.\]
  \mynote{$f$ да бъде добре дефинирана означава, че правилото, с което сме дефинирали $f$ задава функция.}
  Директно от \Proposition{rel-finer} се съобразява, че за произволни думи $\alpha$ и $\beta$,
  \[[\alpha]_\A = [\beta]_\A\ \implies f([\alpha]_\A) = f([\beta]_\A),\]
  откъдето следва, че $f$ е добре дефинирана.
  Ясно е, че $f$ е сюрективна функция.
  От \Problem{surjective-cardinality} следва, че
  \[\abs{\Sigma^\star/_{\approx_{L}}} \leq \abs{\Sigma^\star/_{\sim_\A}}.\]
\end{hint}

\begin{problem}\label{prob:surjection-bijection}
  Нека $A$ и $B$ са крайни равномощни множества.
  Докажете, че ако $g:A \to B$ е сюрекция, то $g$ е биекция.
\end{problem}
\begin{proof}
  Нека $A = \{a_0,\dots,a_{n-1}\}$ и $B = \{b_0,\dots,b_{n-1}\}$. Нека, за всяко $i < n$,
  \[A_i \df \{a_j \in A \mid g(a_j) = b_i\}.\]
  Щом $g$ е сюрекция, то $A_i \neq \emptyset$ за всеки индекс $i < n$.
  Понеже $g$ е функция, то $A_i \cap A_j = \emptyset$ за всеки два различни индекса $i$ и $j$.
  Това означава, че
  \[n = |A| = |\bigcup_{i<n} A_i| = \sum_{i<n}|A_i|.\]
  Оттук следва, че щом за всяко $i$ имаме, че $|A_i| \neq 0$, то $|A_i| = 1$.
  Заключаваме, че $g$ е инекция, защото в противен случай щяхме да имаме някое $i$, за което $|A_i| > 1$.
\end{proof}

\begin{framed}
  \begin{proposition}
    \label{pr:upper-bound}
    Нека $L$ е произволен регулярен език.
    Всеки ДКА $\A$, за който $L = \L(\A)$ има свойството
    \[\abs{\Sigma^\star/_{\approx_L}} \leq \abs{Q},\]
    т.е. броят на класовете на еквивалентност на релацията $\approx_L$
    не надвишава броя на състоянията на автомата $\A$.
  \end{proposition}  
\end{framed}
\begin{proof}
  Да изберем $\A$, който разпознава $L$. 
  Тъй като всяко достижимо състояние на $\A$ определя клас на еквивалентност относно $\sim_\A$,
  то получаваме, че $\abs{\Sigma^\star/_{\sim_\A}} \leq |Q|$.
  Комбинирайки със \Proposition{approx-less-sim},
  \[\abs{\Sigma^\star/_{\approx_L}} \leq \abs{\Sigma^\star/_{\sim_\A}} \leq \abs{Q}.\]
\end{proof}

Така получаваме {\em долна граница} за броя на състоянията в един ДКА разпознаващ езика $L$.
Този брой е не по-малък от броя на класовете на еквивалентност на $\approx_L$.
В следващия раздел ще видим, че тази долна граница може да бъде достигната.

\begin{framed}
  \begin{proposition}
    Езикът $L$ е регулярен точно тогава, когато релацията $\approx_L$ има {\em крайно много} класове на еквивалентност.
  \end{proposition}  
\end{framed}
\begin{proof}
  Ако $L$ е регулярен, то той се разпознава от някой ДКА $\A$, който има крайно много състояния 
  и следователно крайно много класове на еквивалентност относно $\sim_\A$.
  Релацията $\approx_L$ е по-груба от $\sim_\A$ и има по-малко класове на еквивалентност.
  Следователно, $\approx_L$ има крайно много класове на еквивалентност.
  
  За другата посока, ако $\approx_L$ има крайно много класове на еквивалентност, то можем да 
  построим ДКА $\A$ както в доказателството на \hyperref[th:myhill-nerode]{Теоремата на Майхил-Нероуд}, който разпознава $L$.
\end{proof}

Това следствие ни дава още един начин за проверка дали даден език е регулярен.
За разлика от \Lemma{pumping-reg}, сега имаме {\bf необходимо и достатъчно условие}.
При даден език $L$, ние разглеждаме неговата релация $\approx_L$.
Ако тя има крайно много класове, то езикът $L$ е регулярен.
В противен случай, езикът $L$ не е регулярен.

Ще завършим с едно твърдение, което ще ни бъде полезно по-нататък, когато искаме да докажем, че за всеки регулярен език съществува
единствен минимален краен детерминистичен автомат.
\begin{proposition}
  \label{pr:bijection-classes}
  Нека $L$ е произволен регулярен език и $\A$ е ДКА, за който $L = \L(\A)$.
  Ако $|Q| = |\Sigma^\star/_{\approx_L}|$, то функцията $h:\Sigma^\star/_{\approx_L} \to Q$, където
  \[h([\alpha]_L) \df \delta^\star(\qstart,\alpha),\]
  е биекция.
\end{proposition}
\begin{proof}
  Имаме, че:
  \[|Q| = |\Sigma^\star/_{\approx_L}| \leq |\Sigma^\star/_{\sim_\A}| \leq |Q|,\]
  откъдето следва, че
  \[|\Sigma^\star/_{\approx_L}| = |\Sigma^\star/_{\sim_\A}| = |Q|.\]

  Според \Proposition{approx-less-sim} и \Problem{surjection-bijection} следва, че функцията
  $f:\Sigma^\star/_{\sim_\A} \to \Sigma^\star/_{\approx_L}$ дефинирана като $f([\alpha]_\A) \df [\alpha]_L$ е биекция.
  Сега да проверим, че $h$ е наистина функция.
  \mynote{Понеже работим с класове на еквивалентност и $h$ е дефинирана посредством произволни представители на тези класове на еквивалентност, то е важно да
    проверяваме, че нашата дефиниция задава наистина функция.}
  Да разгледаме две произволни думи $\alpha$ и $\beta$, за които $\delta^\star(\qstart,\alpha) \neq \delta^\star(\qstart,\beta)$, 
  което означава, по дефиниция на $\approx_\A$, че $[\alpha]_\A \neq [\beta]_\A$.
  Понеже $f$ е биекция, веднага заключаваме, че $[\alpha]_L \neq [\beta]_L$.
  
  Да отбележим, че щом $|\Sigma^\star/_{\sim_\A}| = |Q|$, то автоматът $\A$
  е свързан, т.е. всяко негово състояние е достижимо от началното. Това означава, че $h$ е сюрекция, защото
  за произволно състояние $q$, съществува дума $\alpha$, за която $\delta^\star(\qstart, \alpha) = q$ и тогава $h([\alpha]_L) = q$.
  
  От \Problem{surjection-bijection} заключаваме, че $h$ е биекция.
\end{proof}

%%% Local Variables:
%%% mode: latex
%%% TeX-master: "../eai"
%%% End:
