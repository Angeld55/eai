\section{Минимален автомат}

\subsection{Проверка за регулярност на език}

\begin{prop}
  Езикът $L$ е регулярен точно тогава, когато релацията $\approx_L$ има {\em крайно много} класове на еквивалентност.
\end{prop}
\begin{proof}
  Ако $L$ е регулярен, то той се разпознава от някой ДКА $\A$, който има крайно много състояния 
  и следователно крайно много класове на еквивалентност относно $\sim_\A$.
  Релацията $\approx_L$ е по-груба от $\sim_\A$ и има по-малко класове на еквивалентност.
  Следователно, $\approx_L$ има крайно много класове на еквивалентност.
  
  За другата посока, ако $\approx_L$ има крайно много класове на еквивалентност, то можем да 
  построим ДКА $\A$ както в доказателството на \hyperref[th:myhill-nerode]{Теоремата на Майхил-Нероуд}, който разпознава $L$.
\end{proof}

Това следствие ни дава още един начин за проверка дали даден език е регулярен.
За разлика от \Lem{pumping-reg}, сега имаме {\bf необходимо и достатъчно условие}.
При даден език $L$, ние разглеждаме неговата релация $\approx_L$.
Ако тя има крайно много класове, то езикът $L$ е регулярен.
В противен случай, езикът $L$ не е регулярен.

\begin{example}
  За езика $L = \{a^nb^n\mid n \in \Nat\}$ имаме, че $\abs{\approx_L} = \infty$,
  защото \[(\forall k,j\in\Nat)[k \neq j \implies [a^kb]_L \neq [a^jb]_L].\]
  Проверете, че $[a^kb]_L = \{a^kb,a^{k+1}b^{2},\dots,a^{k+l}b^{l+1},\dots\}$.
  Така получаваме, че релацията $\approx_L$ има безкрайно много класове на еквивалентност.
  Заключаваме, че този език {\bf не} е регулярен.
\end{example}

\begin{example}
  За езика $L = \{a^{n^2} \mid n \in \Nat\}$ имаме, че $\abs{\approx_L} = \infty$,
  защото \[(\forall m,n\in\Nat)[m \neq n \implies [a^{n^2}]_L \neq [a^{m^2}]_L].\]
  
  Без ограничение на общността, да разгледаме $n < m$ и думата $\gamma = a^{2n+1}$.
  Тогава $a^{n^2}\gamma = a^{(n+1)^2} \in L$, но 
  $m^2 < m^2 + 2n + 1 < (m+1)^2$ и следователно $a^{m^2}\gamma = a^{m^2+2n+1}\not\in L$.
\end{example}

\begin{example}
  За езика $L = \{a^{n!} \mid n \in \Nat\}$ имаме, че $\abs{\approx_L} = \infty$,
  защото \[(\forall m,n\in\Nat)[m \neq n \implies [a^{n!}]_L \neq [a^{m!}]_L].\]
  
  Без ограничение на общността, да разгледаме $n < m$ и думата $\gamma = a^{(n!)n}$.
  Тогава $a^{n!}\gamma = a^{(n+1)!} \in L$, но 
  $m! < m! + (n!)n < m! + (m!)m = (m+1)!$ и следователно $a^{m!}\gamma = a^{m!+(n!)n}\not\in L$.
\end{example}

\begin{problem}
  Да разгледаме езика
  \[L = \{a^{f_n} \mid f_0 = f_1 = 1\ \&\ f_{n+2} = f_{n+1} + f_{n}\}.\]
  Докажете, че $\abs{\approx_L} = \infty$.
\end{problem}

%\marginpar{\href{http://en.wikipedia.org/wiki/DFA_minimization}{Уикипедия}}

\subsection{Релация на Майхил-Нероуд}

\begin{itemize}
\item
  \index{Майхил-Нероуд!релация}
  \marginpar{$\approx_L$ е известна като релация на Майхил-Нероуд}
  Нека $L \subseteq \Sigma^\star$ е език и нека $\alpha,\beta \in \Sigma^\star$.
  Казваме, че $\alpha$ и $\beta$ са {\bf еквивалентни относно} $L$, което записваме 
  като $\alpha \approx_L \beta$, когато:
  \[\alpha \approx_L \beta \dff \alpha^{-1}(L) = \beta^{-1}(L).\]
  С други думи, 
  \[\alpha \approx_L \beta \iff (\forall \omega \in \Sigma^\star)[\alpha\omega \in L \iff \beta\omega \in L].\]
\item
  \marginpar{Трябва ли $\A$ да е тотален?}
  Нека $\A = \FA$ е ДКА.
  Казваме, че две думи $\alpha,\beta \in \Sigma^\star$ са {\bf еквивалентни относно $\A$},
  което означаваме с $\alpha \sim_\A \beta$, ако 
  \[\delta^\star(\qstart,\alpha) = \delta^\star(\qstart,\beta).\]
\item
  Проверете, че $\approx_L$ и $\sim_\A$ са {\bf релации на еквивалентност}, т.е.
  те са рефлексивни, транзитивни и симетрични.
\item
  Класът на еквивалентност на думата $\alpha$ относно релацията $\approx_L$ означаваме като
  \[[\alpha]_L \df \{\beta \in \Sigma^\star \mid \alpha \approx_L \beta\}.\]
  Означаваме 
  \[\Sigma^\star/_{\approx_L} \df \{[\alpha]_L \mid \alpha \in \Sigma^\star\}.\]
  Тогава с $\abs{\Sigma^\star/_{\approx_L}}$ ще означаваме броя на класовете на еквивалентност на релацията $\approx_L$.
\item
  Класът на еквивалентност на думата $\alpha$ относно релацията $\sim_\A$ означаваме като
  \[[\alpha]_\A \df \{\beta \in \Sigma^\star \mid \alpha \sim_\A \beta\}.\]
  Означаваме 
  \[\Sigma^\star/_{\sim_\A} \df \{[\alpha]_\A \mid \alpha \in \Sigma^\star\}.\]
  С $\abs{\Sigma^\star/_{\sim_\A}}$ ще означаваме броя на класовете на еквивалентност на релацията $\sim_\A$.
\item
  Съобразете, че всяко състояние на $\A$, което е достижимо от началното състояние, определя клас на еквивалентност относно 
  релацията $\sim_\A$. Това означава, че ако за всяка дума означим  $q_\alpha = \delta^\star_\A(s,\alpha)$, то
  $\alpha \sim_\A \beta$ точно тогава, когато $q_\alpha = q_\beta$. Заключаваме, че броят на класовете на еквивалентност
  на $\sim_\A$ е равен на броя на достижимите от $s$ състояния. Следователно,
  \[|\Sigma^\star/_{\sim_\A}| \leq |Q^\A|.\]
\item
  Релациите $\approx_\L$ и $\sim_\A$ са дясно-инвариантни, т.е. за всеки две думи $\alpha$ и $\beta$
  е изпълнено:
  \begin{align*}
    \alpha \sim_\A \beta  &\implies (\forall \gamma\in\Sigma^\star)[\alpha\gamma \sim_\A \beta\gamma],\\
    \alpha \approx_\L \beta & \implies (\forall \gamma\in\Sigma^\star)[\alpha\gamma \approx_\L \beta\gamma].
  \end{align*}
\end{itemize}

\begin{prop}
  $\alpha \in L \iff [\alpha]_L \subseteq L$.
\end{prop}

\begin{prop}
  \label{pr:rel-finer}
  Нека $\A = \FA$ е ДКА и $L = \L(\A)$.
  Тогава
  \[(\forall \alpha,\beta \in \Sigma^\star)[\alpha\sim_\A\beta \implies \alpha\approx_{L}\beta].\]
  Това означава, че
  $[\alpha]_\A \subseteq [\alpha]_{L}$, за всяка дума $\alpha \in \Sigma^\star$.
\end{prop}
\begin{proof}
%  \marginpar{стр. 95 от \cite{papadimitriou}}
  % Да означим за всяка дума $\alpha$, $q_\alpha = \delta^\star_\A(\qstart, \alpha)$.
  % Лесно се съобразява, че за всеки две думи $\alpha$ и $\beta$ имаме 
  % \begin{align*}
  %   \alpha \sim_\A \beta & \iff \delta^\star(\qstart,\alpha) = \delta^\star(\qstart,\beta) & (\text{по деф. на }\sim_\A)\\
  %   & \iff q_\alpha = q_\beta.
  % \end{align*}
  Нека $\alpha \sim_\A \beta$, т.е. $\delta^\star_\A(\qstart, \alpha) = \delta^\star_\A(\qstart,\beta)$.
  
  Ще проверим, че  $\alpha \approx_{L} \beta$, т.е. $\alpha^{-1}(L) = \beta^{-1}(L)$.
  За произволно $\gamma \in \Sigma^\star$ имаме:
  \begin{align*}
    \gamma \in \alpha^{-1}(L) & \iff \alpha\gamma \in L\\
                              & \iff \delta^\star_\A(\qstart,\alpha\gamma)\in F & \comment{L = \L(\A)}\\
                              & \iff \delta^\star_\A(\delta^\star_\A(\qstart,\alpha),\gamma) \in F & \comment{\text{по деф. на }\delta^\star}\\
                              & \iff \delta^\star_\A(\delta^\star_\A(\qstart,\beta),\gamma) \in F & \comment{\text{защото }\alpha \sim_\A \beta}\\
                              & \iff \delta^\star_\A(\qstart,\beta\gamma) \in F & \comment{\text{свойство на }\delta^\star_\A}\\
                              & \iff \beta\gamma \in \L(\A)\\
                              & \iff \gamma \in \beta^{-1}(L).
  \end{align*}
  Заключаваме, че 
  \[(\forall \alpha,\beta \in \Sigma^\star)[\alpha\sim_\A\beta \implies \alpha\approx_{L}\beta].\]
\end{proof}

\begin{prop}
  Нека $\A$ е тотален ДКА и нека $L = \L(\A)$.
  Тогава за всяка дума $\alpha$,
  \[[\alpha]_{L} = \bigcup_{\beta \in [\alpha]_{L}}[\beta]_\A.\]
\end{prop}
% \begin{hint}
%   Понеже $[\alpha]_\A \subseteq [\alpha]_L$ и $\alpha \in [\alpha]_L$, то е очевидно, че
%   \[\bigcup_{\beta \in [\alpha]_{L}}[\beta]_\A \subseteq [\alpha]_{L}.\]
%   За другата посока,
%   ако $\gamma \in [\alpha]_L$, то е ясно, че $\gamma \in [\gamma]_\A$.
  
% \end{hint}

\begin{prop}
  \label{pr:approx-less-sim}
  Нека $\A$ е тотален ДКА и нека $L = \L(\A)$.
  Тогава 
  \[\abs{\Sigma^\star/_{\approx_{L}}} \leq \abs{\Sigma^\star/_{\sim_\A}}.\]
\end{prop}
\begin{hint}
  Да разгледаме изображението 
  \[f([\alpha]_L) \df \{[\gamma]_\A \mid \gamma \approx_L \alpha\}.\]
  Получаваме, че
  \[[\alpha]_L = \bigcup f([\alpha]_L).\]

  \begin{itemize}
  \item
    Ясно е, че за всяко $\alpha$, $f([\alpha]_L) \neq \emptyset$, защото $[\alpha]_\A \in f([\alpha]_L)$.
  \item 
    Съобразете, че $f$ е {\bf функция}, т.е. 
    \[(\forall\alpha,\beta\in\Sigma^\star)[\alpha \approx_L \beta \implies f([\alpha]_L) = f([\beta]_L)].\]
  \item
    Съобразете, че 
    \[(\forall\alpha,\beta\in\Sigma^\star)[\alpha \not\approx_L \beta \implies f([\alpha]_L) \cap f([\beta]_L) = \emptyset].\]
  \item
    Заключете, че \[\abs{\Sigma^\star/_{\approx_{L}}} \leq \abs{\Sigma^\star/_{\sim_\A}}.\]
  \end{itemize}
\end{hint}

\begin{cor}
  \label{cor:upper-bound}
  Нека $L$ е произволен регулярен език $L$.  
  Всеки тотален ДКА $\A$, който разпознава $L$ има свойството
  \[\abs{\Sigma^\star/_{\approx_L}} \leq \abs{Q},\]
  т.е. броят на класовете на еквивалентност на релацията $\approx_L$
  не надвишава броя на състоянията на автомата.
\end{cor}
\begin{proof}
  Да изберем $\A$, който разпознава $L$. % да бъде такъв, че да {\em няма недостижими състояния}.
  Тъй като всяко достижимо състояние на $\A$ определя клас на еквивалентност относно $\sim_\A$,
  то получаваме, че $\abs{\Sigma^\star/_{\sim_\A}} \leq |Q|$.
  Комбинирайки със \Prop{approx-less-sim},
  \[\abs{\Sigma^\star/_{\approx_L}} \leq \abs{\Sigma^\star/_{\sim_\A}} \leq \abs{Q}.\]
\end{proof}
Така получаваме {\em долна граница} за броя на състоянията в тотален автомат разпознаващ езика $L$.
Този брой е не по-малък от броя на класовете на еквивалентност на $\approx_L$.
В следващия раздел ще видим, че тази долна граница може да бъде достигната.



%%% Local Variables:
%%% mode: latex
%%% TeX-master: "../eai"
%%% End:
