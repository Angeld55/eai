\section{Релация на Майхил-Нероуд}

\begin{itemize}
\item
  \index{Майхил-Нероуд!релация}
  \marginpar{$\approx_L$ е известна като релация на Майхил-Нероуд}
  Нека $L \subseteq \Sigma^\star$ е език и нека $\alpha,\beta \in \Sigma^\star$.
  Казваме, че $\alpha$ и $\beta$ са {\bf еквивалентни относно} $L$, което записваме 
  като $\alpha \approx_L \beta$, когато:
  \[\alpha \approx_L \beta\ \dff\ \alpha^{-1}(L) = \beta^{-1}(L).\]
  С други думи, 
  \[\alpha \approx_L \beta \iff (\forall \omega \in \Sigma^\star)[\ \alpha\omega \in L \iff \beta\omega \in L\ ].\]
\item
  \marginpar{Трябва ли $\A$ да е тотален?}
  Нека $\A = \FA$ е ДКА.
  Казваме, че две думи $\alpha,\beta \in \Sigma^\star$ са {\bf еквивалентни относно $\A$},
  което означаваме с $\alpha \sim_\A \beta$, ако 
  \[\delta^\star(\qstart,\alpha) = \delta^\star(\qstart,\beta).\]
\item
  Проверете, че $\approx_L$ и $\sim_\A$ са {\bf релации на еквивалентност}, т.е.
  те са рефлексивни, транзитивни и симетрични.
\item
  Класът на еквивалентност на думата $\alpha$ относно релацията $\approx_L$ означаваме като
  \[[\alpha]_L \df \{\beta \in \Sigma^\star \mid \alpha \approx_L \beta\}.\]
  Означаваме 
  \[\Sigma^\star/_{\approx_L} \df \{[\alpha]_L \mid \alpha \in \Sigma^\star\}.\]
  Тогава с $\abs{\Sigma^\star/_{\approx_L}}$ ще означаваме броя на класовете на еквивалентност на релацията $\approx_L$.
\item
  Класът на еквивалентност на думата $\alpha$ относно релацията $\sim_\A$ означаваме като
  \[[\alpha]_\A \df \{\beta \in \Sigma^\star \mid \alpha \sim_\A \beta\}.\]
  Означаваме 
  \[\Sigma^\star/_{\sim_\A} \df \{[\alpha]_\A \mid \alpha \in \Sigma^\star\}.\]
  С $\abs{\Sigma^\star/_{\sim_\A}}$ ще означаваме броя на класовете на еквивалентност на релацията $\sim_\A$.
\item
  Съобразете, че всяко състояние на $\A$, което е достижимо от началното състояние, определя клас на еквивалентност относно 
  релацията $\sim_\A$. Това означава, че функцията $g:\Sigma^\star/_\A \to Q$, където
  \[g([\alpha]_\A) \df \delta^\star(\qstart,\alpha)\]
  е инекция. Следователно,
  \[|\Sigma^\star/_{\sim_\A}| \leq |Q|.\]
  Ако в автомата $\A$ няма недостижими от $\qstart$ състояния, то $g$ е биекция и съответно
  \[|\Sigma^\star/_{\sim_\A}| = |Q|.\]
\item
  Релациите $\approx_\L$ и $\sim_\A$ са дясно-инвариантни, т.е. за всеки две думи $\alpha$ и $\beta$
  е изпълнено:
  \begin{align*}
    \alpha \sim_\A \beta  &\implies (\forall \gamma\in\Sigma^\star)[\alpha\gamma \sim_\A \beta\gamma],\\
    \alpha \approx_\L \beta & \implies (\forall \gamma\in\Sigma^\star)[\alpha\gamma \approx_\L \beta\gamma].
  \end{align*}
\end{itemize}

\begin{problem}
  Докажете, че за всяка дума $\alpha \in \Sigma^\star$ е изпълнено, че:
  \[\alpha \in L \iff [\alpha]_L \subseteq L.\]
\end{problem}

% \begin{remark}
%   Нека $\A$ е детерминистичен краен автомат.
%   Тогава функцията $g:\Sigma^\star/_{\sim_\A} \to Q$, където
%   \[g([\alpha]_\A) \df \delta^\star(\qstart, \alpha)\]
%   е инективна.
%   Ако $\A$ няма недостожими състояния, то функцията $g$ е биективна.
%   Това означава, че тогава $|Q| = |\Sigma^\star/_{\sim_\A}|$.
% \end{remark}

\begin{prop}
  \label{pr:rel-finer}
  Нека $\A = \FA$ е ДКА и $L = \L(\A)$. Тогава
  \[(\forall \alpha,\beta \in \Sigma^\star)[\ \alpha\sim_\A\beta \implies \alpha\approx_{L}\beta\ ].\]
\end{prop}
\begin{proof}
  Нека $\alpha \sim_\A \beta$, т.е. $\delta^\star(\qstart, \alpha) = \delta^\star(\qstart,\beta)$.
  Ще проверим, че  $\alpha \approx_{L} \beta$, т.е. $\alpha^{-1}(L) = \beta^{-1}(L)$.
  За произволно $\gamma \in \Sigma^\star$ имаме:
  \begin{align*}
    \gamma \in \alpha^{-1}(L) & \iff \alpha\gamma \in L\\
                              & \iff \alpha\gamma \in \L(\A) & \comment{L = \L(\A)}\\
                              & \iff \delta^\star(\qstart,\alpha\gamma)\in F & \comment{L = \L(\A)}\\
                              & \iff \delta^\star(\delta^\star(\qstart,\alpha),\gamma) \in F & \comment{\text{деф. на }\delta^\star}\\
                              & \iff \delta^\star(\delta^\star(\qstart,\beta),\gamma) \in F & \comment{\text{защото }\alpha \sim_\A \beta}\\
                              & \iff \delta^\star(\qstart,\beta\gamma) \in F & \comment{\text{свойство на }\delta^\star}\\
                              & \iff \beta\gamma \in \L(\A) & \comment\text{деф. на }\L(\A)\\
                              & \iff \beta\gamma \in L & \comment{L = \L(\A)} \\
                              & \iff \gamma \in \beta^{-1}(L)
  \end{align*}
  Заключаваме, че 
  \[(\forall \alpha,\beta \in \Sigma^\star)[\ \alpha\sim_\A\beta \implies \alpha\approx_{L}\beta\ ].\]
\end{proof}

\begin{problem}
  \marginpar{Съобразете, че за всяка дума $\alpha \in \Sigma^\star$, $[\alpha]_\A \subseteq [\alpha]_{L}$.}
  Докажете, че за ДКА $\A$ и $L = \L(\A)$,
  за всяка дума $\alpha$,
  \[[\alpha]_{L} = \bigcup_{\beta \in [\alpha]_{L}}[\beta]_\A.\]
\end{problem}

\begin{problem}
  Нека $A$ и $B$ са крайни множества, за които съществува сюрективна функция $f: A \to B$.
  Тогава $|B| \leq |A|$.
\end{problem}

\begin{prop}
  \label{pr:approx-less-sim}
  Нека $\A$ е ДКА и нека $L = \L(\A)$.
  Тогава 
  \[\abs{\Sigma^\star/_{\approx_{L}}} \leq \abs{\Sigma^\star/_{\sim_\A}}.\]
\end{prop}
\begin{hint}
  Да разгледаме функцията $f: \Sigma^\star/_{\sim_\A} \to \Sigma^\star/_{\approx_L}$, където
  \[f([\alpha]_\A) \df [\alpha]_L.\]
  Директно от \Prop{rel-finer} се съобразява, че
  \[(\forall \alpha,\beta \in \Sigma^\star)[\ [\alpha]_\A = [\beta]_\A\ \implies f([\alpha]_\A) = f([\beta]_\A)\ ],\]
  откъдето следва, че $f$ е добре дефинирана.
  Ясно е, че $f$ е сюрективна функция.
  Оттук следва, че
  \[\abs{\Sigma^\star/_{\approx_{L}}} \leq \abs{\Sigma^\star/_{\sim_\A}}.\]
\end{hint}

\begin{framed}
  \begin{prop}
    \label{pr:upper-bound}
    Нека $L$ е произволен регулярен език.
    Всеки детерминистичен краен автомат $\A$, за който $L = \L(\A)$ има свойството
    \[\abs{\Sigma^\star/_{\approx_L}} \leq \abs{Q},\]
    т.е. броят на класовете на еквивалентност на релацията $\approx_L$
    не надвишава броя на състоянията на автомата.
  \end{prop}  
\end{framed}
\begin{proof}
  Да изберем $\A$, който разпознава $L$. % да бъде такъв, че да {\em няма недостижими състояния}.
  Тъй като всяко достижимо състояние на $\A$ определя клас на еквивалентност относно $\sim_\A$,
  то получаваме, че $\abs{\Sigma^\star/_{\sim_\A}} \leq |Q|$.
  Комбинирайки със \Prop{approx-less-sim},
  \[\abs{\Sigma^\star/_{\approx_L}} \leq \abs{\Sigma^\star/_{\sim_\A}} \leq \abs{Q}.\]
\end{proof}


Така получаваме {\em долна граница} за броя на състоянията в краен детерминистичен автомат разпознаващ езика $L$.
Този брой е не по-малък от броя на класовете на еквивалентност на $\approx_L$.
В следващия раздел ще видим, че тази долна граница може да бъде достигната.


Ще завършим с едно твърдение, което ще ни бъде полезно по-нататък, когато искаме да докажем, че за всеки регулярен език съществува
единствен минимален краен детерминистичен автомат.
\begin{prop}
  \label{pr:bijection-classes}
  Нека $L$ е произволен регулярен език и $\A$ е детерминистичен краен автомат, за който $L = \L(\A)$.
  Ако $|Q| = |\Sigma^\star/_{\approx_L}|$, то функцията $h:\Sigma^\star/_{\approx_L} \to Q$, където
  \[h([\alpha]_L) \df \delta^\star(\qstart,\alpha),\]
  е биекция.
\end{prop}
\begin{proof}
  Имаме, че:
  \[|Q| = |\Sigma^\star/_{\approx_L}| \leq |\Sigma^\star/_{\sim_\A}| \leq |Q|,\]
  откъдето следва, че
  \[|\Sigma^\star/_{\approx_L}| = |\Sigma^\star/_{\sim_\A}| = |Q|.\]
  Това означава, че функцията $g:\Sigma^\star/_{\sim_\A} \to Q$, където
  \[g([\alpha]_\A) \df \delta^\star(\qstart,\alpha)\] е биекция,
  защото $|\Sigma^\star/_{\sim_\A}| = |Q|$.
  От \Prop{approx-less-sim} имаме, че функцията $f:\Sigma^\star/_{\sim_\A} \to \Sigma^\star/_{\approx_L}$, където
  \[f([\alpha]_\A) \df [\alpha]_L\] е биекция,
  защото знаем, че $f$ е сюрекция и $|\Sigma^\star/_{\approx_L}| = |\Sigma^\star/_{\sim_\A}|$.
  Оттук заключаваме, че $h = g \circ f^{-1}$ е биекция, защото е композиция на две биекции.
\end{proof}

%%% Local Variables:
%%% mode: latex
%%% TeX-master: "../eai"
%%% End:
