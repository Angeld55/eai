\subsection{Затвореност относно сечиение, обединение, разлика}

\begin{framed}
  \begin{proposition}
    \index{автоматни езици!сечение}
    \label{pr:automata-cap}
    Класът на автоматните езици е затворен относно операцията {\em сечение}.
    Това означава, че ако $L_1$ и $L_2$ са два произволни автоматни езика над азбуката $\Sigma$, то $L_1\cap L_2$
    също е автоматен език.
  \end{proposition}  
\end{framed}
\begin{proof}
  Да разгледаме два автомата \[\A' = \langle{\Sigma,Q',\qstart',\delta',F'}\rangle\text{ и } \A'' = \langle{\Sigma, Q'', \qstart'', \delta'', F''}\rangle.\]
  Определяме автомата $\A = \FA$, по следния начин:
  \marginpar{Изчислението на автомата $\A$ върху думата $\alpha$ едновременно симулира изчислението на $\A'$ и $\A''$ върху $\alpha$.}
  \begin{itemize}
  \item
    $Q \df Q'\times Q''$;
  \item
    \marginpar{Съобразете, че $\delta$ е тотална функция.}
    За всяко $\pair{r_1,r_2} \in Q$ и всяко $a \in \Sigma$,
    \[\delta(\pair{r_1,r_2},a) \df \pair{\delta_1(r_1,a),\delta_2(r_2,a)};\]
  \item
    $\qstart \df \pair{\qstart',\qstart''}$;
  \item
    $F \df \{\pair{r_1,r_2}\mid r_1\in F''\ \&\ r_2 \in F''\} = F' \times F''$
  \end{itemize}
  Трябва да докажем, че за всяка дума $\alpha \in \Sigma^\star$ е изпълнено, че:
  \begin{equation}
    \label{eq:7}
    (\forall p\in Q')(\forall q\in Q'')[\ \delta^\star(\pair{p,q},\alpha) = \pair{ \delta^\star_1(p,\alpha), \delta^\star_2(q,\alpha) }\ ].
  \end{equation}
  Ще докажем \Property{eq:7} с индукция по дължината на думата $\alpha$.
  \begin{itemize}
  \item
    Нека $|\alpha| = 0$, т.е. $\alpha = \varepsilon$. Тогава всичко е ясно, защото
    \begin{align*}
      \delta^\star( \pair{p,q}, \varepsilon) & =\pair{p,q} & \comment\text{деф. на }\delta^\star\\
                                             & = \pair{\delta^\star_1(p,\varepsilon), \delta^\star_2(q,\varepsilon)}. & \comment\text{деф. на $\delta^\star_1$ и $\delta^\star_2$} 
    \end{align*}
  \item
    Да приемем, че \Property{eq:7} е изпълнено за думи $\alpha$ с дължина $n$.
  \item
    Нека $|\alpha| = n+1$, т.е. $\alpha = \beta a$ и $|\beta| = n$. Тогава:
    \begin{align*}
      \delta^\star(\pair{p, q},\beta a) & = \delta( \delta^\star( \pair{p, q}, \beta ), a) & \comment{\text{деф. на }\delta^\star}\\
                                        & = \delta( \pair{ \delta^\star_1(p, \beta), \delta^\star_2(q, \beta)}, a ) & \comment{\text{ от И.П.}}\\
                                        & = \pair{ \delta_1( \delta^\star_1(p, \beta), a), \delta_2(\delta^\star_2(q,\beta), a)} & \comment\text{деф. на }\delta\\
                                        & = \pair{ \delta^\star_1(p, \beta a), \delta^\star_2(q, \beta a) } & \comment{\text{деф. на $\delta^\star_1$ и $\delta^\star_2$}}.
    \end{align*}
  \end{itemize}
  Използвайки \Property{eq:7} лесно можем да докажем, че
  \[\L(\A) = \L(\A') \cap \L(\A'').\]
  Имаме следните еквивалентности:
  \begin{align*}
    \omega \in \L(\A) & \iff \delta^\star(\qstart, \omega) \in F & \comment\text{деф. на }\L(\A)\\
                      & \iff \delta^\star(\pair{\qstart',\qstart''}, \omega) \in F' \times F'' & \comment\text{деф. на }\A\\
                      & \iff \pair{\delta^\star_1(\qstart',\omega), \delta^\star_2(\qstart'',\omega)} \in F' \times F'' & \comment{\text{от (\ref{eq:7})}}\\
                      & \iff \delta^\star_1(\qstart',\omega) \in F'\ \&\ \delta^\star_2(\qstart'',\omega) \in F''\\
                      & \iff \omega \in \L(\A')\ \&\ \omega\in\L(\A'') \\
                      & \iff \omega \in \L(\A') \cap \L(\A'').
  \end{align*}
  
\end{proof}

\begin{framed}
  \begin{proposition}
    \index{автоматни езици!допълнение}
    \label{pr:automata-complement}
    Класът на автоматните езици са затворени относно операцията допълнение, т.е.
    ако $L$ е автоматен език, то $\Sigma^\star\setminus L$ също е автоматен език.
  \end{proposition}  
\end{framed}
\begin{hint}
  \marginpar{\writedown Проверете, че $\Sigma^\star\setminus L = \L(\A')$. Съобразете, че тук е важно, че $\delta$ е тотална функция на преходите.}
  Нека $L = L(\A)$, където $\A = \FA$.
  Да вземем автомата
  \[\A' = \pair{Q,\Sigma,\qstart,\delta,Q\setminus F},\]
  т.е. $\A'$ е същия като $\A$, с единствената разлика, че финалните състояния на $\A'$
  са тези състояния, които {\bf не} са финални в $\A$.
\end{hint}

\begin{framed}
  \begin{proposition}
    \index{автоматни езици!обединение}
    \label{pr:automata-union}
    Класът на автоматните езици е затворен относно операцията {\bf обединение}.
    Това означава, че ако $L_1$ и $L_2$ са два произволни автоматни езика, то $L_1\cup L_2$
    също е автоматен език.
  \end{proposition}  
\end{framed}
\begin{hint}
  \marginpar{\writedown Докажете, че така построения автомат $\A$ разпознава $L_1\cup L_2$. Тук отново е важно, че $\delta_1$ и $\delta_2$ са тотални функции на преходите.}
  Първият подход е да използвайте конструкцията на автомата $\A$ от \Proposition{automata-cap},
  с единствената разлика, че тук избираме финалните състояния да бъдат елементите на множеството
  \begin{align*}
    F & \df \{\pair{q',q''} \in Q' \times Q'' \mid q' \in F'\ \lor\ q'' \in F''\}\\
      & = F'\times Q'' \cup Q'\times F''.
  \end{align*}
  Друг подход е да се използва правилото на Де Морган, а именно:
  \[L_1 \cup L_2 = \ov{\ov{L_1} \cap \ov{L_2}}.\]
\end{hint}

%%% Local Variables:
%%% mode: latex
%%% TeX-master: "../eai"
%%% End:
