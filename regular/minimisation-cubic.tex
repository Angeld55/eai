\subsection{Кубичен алгоритъм за минимизация}

При даден език $L$ и детерминиран краен автомат $\A = \FA$, който го разпознава, целта ни е да построим нов детерминиран краен автомат $\A'$,
който има толкова състояния колкото са класовете на еквивалентност на релацията $\equiv_\A$.
Това ще направим като ,,слеем'' състоянията на $\A$, които са еквивалентни относно релацията $\equiv_\A$.
Проблемът с намирането на класовете на еквивалентност на релацията $\equiv_\A$ е кванторът $(\forall \omega \in \Sigma^\star)$
във нейната дефиниция (чрез Формула \ref{eq:1}), защото $\Sigma^\star$ е безкрайно множество от думи.
За да разрешим този проблем, ще разгледаме апроксимации на езиците $\L_\A(q)$.
За ествено число $n$, да означим 
\[\L^n_\A(p) \df \{\omega \in \Sigma^\star \mid \abs{\omega} \leq n\ \&\ \delta^\star(p,\omega) \in F\}.\]
Лесно се съобразява, че
\mynote{Можем ли да дадем горна граница на $n$?}
\[L(\A) = \bigcup_{n\geq 0} \L^n_\A(\qstart).\]

\index{$\equiv^n_\A$}
За всяко естествено число $n$, дефинираме бинарните релации $\equiv^n_\A$ върху $Q$ по следния начин:
\[p \equiv^n_\A q \dff \L^n_\A(p) = \L^n_\A(q).\]

Релациите $\equiv^n_\A$ представляват апроксимации на релацията $\equiv_\A$.
Обърнете внимание, че за всяко $n$, $\equiv^n_\A$ е {\em по-груба} релация от $\equiv^{n+1}_\A$, 
която на свой ред е по-груба от $\equiv_\A$.
Алгоритъмът строи $\equiv^n_\A$ докато не срещнем $n$, за което
\[\equiv^n_\A\ =\ \equiv^{n+1}_\A.\]
Тъй като броят на класовете на еквивалентност на $\equiv_\A$ е краен (той е $\leq \abs{Q}$), то 
със сигурност ще намерим такова $n$, за което $\equiv^n_\A\ =\ \equiv^{n+1}_\A$.
Тогава заключаваме, че за това $n$ имаме, че
\[\equiv_\A\ =\ \equiv^n_\A.\]

\mynote{Ако $q \in F$, то $\L^0_\A(q) = F$ и ако $q \not\in F$, то $\L^0_\A(q) = Q\setminus F$.}
Понеже единствената дума с дължина $0$ e $\varepsilon$ и по определение $\delta^\star(p,\varepsilon) = p$, 
лесно се съобразява, че $\equiv^0_\A$ има два класа на еквивалентност.
Единият е $F$, а другият е $Q\setminus F$.

Вече имаме базовия случай за $n=0$.
Да видим сега как можем да намерим $\equiv^{n+1}_\A$ при положение, че вече сме намерили $\equiv^n_\A$.
\begin{framed}
  \begin{proposition}\label{pr:one-letter-test}
    За всеки две състояния $p$ и $q$, и всяко естествено число $n$, $p \equiv^{n+1}_\A q$ точно тогава, когато
    \begin{enumerate}[a)]
    \item
      $p \equiv^n_\A q$ и
    \item
      $(\forall a \in \Sigma)[\delta(q,a) \equiv^n_\A \delta(p,a)]$.
    \end{enumerate}
  \end{proposition}  
\end{framed}
\begin{hint}
  \mynote{\cite[стр. 99]{papadimitriou}}
  \begin{align*}
    p \equiv^{n+1}_\A q & \iff \L^{n+1}_\A(p) = \L^{n+1}_\A(q)\\
                     & \iff \L^n_\A(p) = \L^n_\A(q)\ \&\ (\forall a \in \Sigma)[\ \L^n_\A(\delta(p,a)) = \L^n_\A(\delta(q,a))\ ]\\
                     & \iff p \equiv^n_\A q\ \&\ (\forall a \in \Sigma)[\ \delta(p,a) \equiv^n_\A \delta(q,a)\ ].
  \end{align*}
\end{hint}

\begin{proposition}\label{pr:minimisation-cubic:equiv-all}
  Ако $\equiv^n_\A\ =\ \equiv^{n+1}_\A$, то за всяко $m > n$ е изпълнено, че:
  \begin{equation}
    \label{eq:8}
    \equiv^n_\A\ =\ \equiv^m_\A.
  \end{equation}
\end{proposition}
\begin{proof}
  Ще докажем Свойство~(\ref{eq:8}) с индукция по $m > n$, като базата на индукцията е случаят $m = n + 1$, за който Свойство~(\ref{eq:8})
  е изпълнено по условие.
  Индукционното ни предположение е, че $\equiv^n_\A\ =\ \equiv^{m}_\A$ за някое $m > n+1$.
  Индукционната ни стъпка е да докажем, че $\equiv^n_\A\ =\ \equiv^{m+1}_\A$. За произволни състояния $p$ и $q$ имаме следните еквивалентности:
  \begin{align*}
    p \equiv^{m+1}_\A q & \iff p \equiv^m_\A q\ \&\ (\forall a \in \Sigma)[\delta(p,a) \equiv^m_\A \delta(q,a)] & \comment\text{от \Proposition{one-letter-test}}\\
                        & \iff p \equiv^n_\A q\ \&\ (\forall a \in \Sigma)[\delta(p,a) \equiv^n_\A \delta(q,a)] & \comment\text{от И.П.}\\
                        & \iff p \equiv^{n+1}_\A q & \comment\text{от \Proposition{one-letter-test}}\\
                        & \iff p \equiv^n_\A q. & \comment\text{от условието}
  \end{align*}
\end{proof}

\begin{proposition}\label{pr:minimisation-cubic:equiv-approx}
  Докажете, че за произволно естествено число $n$,
  \[\equiv^n_\A\ =\ \equiv^{n+1}_\A\ \implies\ \equiv^n_\A\ =\ \equiv_\A.\]
\end{proposition}
\begin{proof}
  Нека $\equiv^n_\A\ =\ \equiv^{n+1}_\A$.
  Ще докажем, че за произволни състояния $p$ и $q$ е изпълено, че:
  \[p \equiv_\A q \iff p \equiv^n_\A q.\]
  Ясно е, че $p \equiv_\A q \implies p \equiv^n_\A q$. 
  Да видим защо имаме и обратната посока, т.е. защо $p \equiv^n_\A q \implies p \equiv_\A q$.
  Ние ще докажем контрапозицията на импликацията, т.е. ще докажем следното:
  \[p \not\equiv_\A q \implies p \not\equiv^n_\A q.\]
  И така, нека $p \not\equiv_\A q$. Това означава, че същестува дума $\alpha$, за която:
  \[\neg(\delta^\star(p,\alpha) \in F \iff \delta^\star(q,\alpha) \in F).\]
  Това означава, че $p \not\equiv^{|\alpha|}_\A q$. Имаме два случая.
  \begin{itemize}
  \item
    Нека $|\alpha| \leq n$.
    Тогава $p \not\equiv^n_\A q$, защото от дефиницията следва, че
    \[p \not\equiv^{|\alpha|}_\A q \implies p \not\equiv^n_\A q.\]
  \item
    Ако $|\alpha| > n$, от \Proposition{minimisation-cubic:equiv-all} следва, че
    \[p \equiv^n_\A q \implies p \equiv^{|\alpha|}_\A q,\]
    чиято контрапозиция е:
    \[p \not\equiv^{|\alpha|}_\A q \implies p \not\equiv^{n}_\A q.\]
    Заключаваме, че $p \not\equiv^{n}_\A q$.
  \end{itemize}
\end{proof}

\begin{proposition}
  За всеки ДКА $\A$ е изпълнено, че 
  $\equiv^{|Q|}_\A\ =\ \equiv_\A$.
\end{proposition}
\begin{proof}
  Ако съществува $n < |Q|$, такова че $\equiv^n_\A\ =\ \equiv^{n+1}_\A$, то според предишните две твърдения,
  то $\equiv^n_\A\ =\ \equiv^{|Q|}_\A\ =\ \equiv_\A$.

  Нека сега приемем, че за всяко $n < |Q|$ е изпълнено, че $\equiv^n_\A\ \neq\ \equiv^{n+1}_\A$.
  Това означава, че всеки клас на еквивалентност на $\equiv^{|Q|}_\A$ съдържа точно едно състояние, защото
  всеки клас на еквивалентност съдържа поне едно състояние, а ние имаме точно $|Q|$ на брой състояния.
  Тогава със сигурност имаме, че $\equiv^{|Q|}_\A\ =\ \equiv^{|Q|+1}_\A$,
  защото
  \[p \equiv^{|Q|+1}_\A q \implies p \equiv^{|Q|}_\A q\]
  и няма как $\equiv^{|Q|+1}_\A$ да ,,раздроби'' някой клас на $\equiv^{|Q|}_\A$.
  Тогава от \Proposition{minimisation-cubic:equiv-approx} следва, че $\equiv^n_\A\ =\ \equiv_\A$.
\end{proof}

% \mynote{\cite{khoussainov-nerode}}



\begin{algorithm}[H]
  \caption{Кубичен алгоритъм за минимизация}
  \label{alg:minimisation-cube}
  \begin{algorithmic}[1]
    \ForAll{$p < |Q|$}\Comment{състоянията са индексирани от $0$ до $|Q|-1$}
    \ForAll{$q < p$} 
    \If{$(p \in F \iff q \not\in F)$}
    \State E[p][q] = false \Comment{Имаме, че $p \not\equiv q$}
    \Else
    \State E[p][q] = true \Comment{Имаме, че $p \equiv q$}
    \EndIf
    \EndFor
    \EndFor
    \Repeat
    \State ready = true
    \ForAll{$p < |Q|$}
    \ForAll{$q < p$}
    \If{E[p][q]}
    \ForAll{$a \in \Sigma$}
    \If{! E[$\delta(p,a)$][$\delta(q,a)$]}  \Comment{$p \equiv q$, но $\delta(p,a) \not\equiv \delta(q,a)$}
    \State{E[p][q] = false}
    \State{ready = false}
    \EndIf
    \EndFor
    \EndIf
    \EndFor
    \EndFor
    \Until{ready}
  \end{algorithmic}
\end{algorithm}

\begin{proposition}
  За всеки две състояния $q_i,q_j \in Q$, където $i > j$,
  \[q_i \equiv_\A q_j \iff \texttt{E[i][j] == true}.\]
\end{proposition}

\begin{problem}
  Съобразете, че \Algorithm{minimisation-cube} има времева сложност $\mathcal{O}(|\A|^3)$.
\end{problem}


%%% Local Variables:
%%% mode: latex
%%% TeX-master: "../eai"
%%% End:
