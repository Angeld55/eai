\section{Автоматни езици}

% Един от източниците е втора и трета глава от книгата на Сипсер, \cite{sipser}.
% Друг основен източник е книгата на Пападимитриу и Люис, \cite{papadimitriou}.
%По Сипсер, стр. 35
\begin{dfn}
  Краен автомат е петорка $\A = \FA$, където
  \begin{enumerate}[1)]
  \item
    $Q$ е крайно множество от състояния;
  \item
    $\Sigma$ е азбука;
  \item
    % \marginpar{Тук нямаме $\varepsilon$-преходи}
    % \marginpar{В \cite{sipser1} се разглеждат тотални $\delta$ функции}
    $\delta:Q\times\Sigma\to Q$ е (частична) функция на преходите;
  \item
    $s\in Q$ е начално състояние;
  \item
    $F\subseteq Q$ е множеството от финални състояния, $F \neq \emptyset$.
  \end{enumerate}
\end{dfn}

\index{автомат!детерминиран}\index{автомат!тотален детерминиран}
Ако функцията на преходите $\delta$ е тотална функция, то казваме, 
че автоматът $\A$ е {\bf тотален}. Това означава, че за всяка двойка $(a,q) \in \Sigma\times Q$,
същесествува $q' \in Q$, за което $\delta(a,q) = q'$.

Нека имаме една дума $\alpha \in \Sigma^\star$, $\alpha = a_1a_2\cdots a_n$.
Казваме, че $\alpha$ се {\bf разпознава} от автомата $\A$, ако
съществува редица от състояния $q_0,q_1,q_2,\dots,q_n$, такива че:
\begin{itemize}
\item
  $q_0 = s$, началното състояние на автомата;
\item
  $\delta(q_i,a_{i+1}) = q_{i+1}$, за всяко $i = 0, \dots, n-1$;
\item
  $q_n \in F$.
\end{itemize}

Казваме, че $\A$ {\bf разпознава} езика $L$, ако $\A$ разпознава точно думите от $L$, т.е.
$L = \{\alpha \in \Sigma^\star \mid \A\mbox{ разпознава }\alpha\}$.
Обикновено означаваме езика, който се разпознава от даден автомат $\A$ с $\L(\A)$.
\index{език!автоматен}
В такъв случай ще казваме, че езикът $L$ е {\bf автоматен}.

При дадена (частична) функция на преходите $\delta$,
често е удобно да разглеждаме (частичната) функция $\delta^\star:Q\times\Sigma^\star \to Q$, кято е дефинирана по следния начин:
% \marginpar{Това е пример за индуктивна (рекурсивна) дефиниция по дължината на думата $\alpha$}
\begin{itemize}
\item 
  $\delta^\star(q,\varepsilon) = q$, за всяко $q\in Q$;
\item
  $\delta^\star(q,a\beta) = \delta^\star(\delta(q,a),\beta)$, за всяко $q\in Q$, всяко $a\in\Sigma$ и $\beta\in\Sigma^\star$.
\end{itemize}
Тогава една дума $\alpha$ се {\em разпознава} от автомата $\A$ точно тогава, когато $\delta^\star(s,\alpha) \in F$.
Оттук следва, че
\[\L(\A) = \{\alpha\in\Sigma^\star \mid \delta^\star(s,\alpha) \in F\}.\]

\begin{prop}
  $(\forall q\in Q)(\forall\alpha,\beta\in\Sigma^\star)[\delta^\star(q,\alpha\beta) = \delta^\star(\delta^\star(q,\alpha),\beta)]$.
\end{prop}
\begin{proof}
  % \marginpar{\ding{45} Напише доказателството!}
  Индукция по дължината на $\alpha$.
\end{proof}

\index{моментно описание}
% \marginpar{(На англ. {\em instantaneous description})}
{\em Моментното описание} на изчисление с краен автомат представлява двойка от вида $(q,\alpha) \in Q\times\Sigma^\star$,
т.е. автоматът се намира в състояние $q$, а думата, която остава да се прочете е $\alpha$.
Удобно е да въведем бинарната релация $\vdash_\A$ над $Q\times\Sigma^\star$,
която ще ни казва как моментното описание на автомата $\A$ се променя след изпълнение на една стъпка:
\[(q,x\alpha) \vdash_\A (p,\alpha), \text{ ако } \delta(q,x) = p.\]
\marginpar{Рефл. и транз. затваряне на една релация е разгледано в Глава \ref{ch:intro}}
Рефлексивното и транзитивно затваряне на $\vdash_\A$ ще означаваме с $\vdash^\star_\A$.
Получаваме, че 
\[\L(\A) = \{\alpha\in\Sigma^\star \mid (s,\alpha) \vdash^\star_\A(p,\varepsilon)\ \&\ p \in F\}.\]

Нашата дефиниция на автомат позволява $\delta$ да бъде частична функция, т.е.
може да има $q\in Q$ и $a\in\Sigma$, за които $\delta(q,a)$ не е дефинирана.
Следващото твърдение ни казва, че ние съвсем спокойно можем да разглеждаме автомати
само с тотални функции на преходите  $\delta$.
\begin{prop}
  За всеки краен автомат $\A$, съществува {\em тотален} краен автомат $\A'$,
  за който $\L(\A) = \L(\A')$.
\end{prop}
\begin{proof}
  Нека $\A = \FA$.
  Дефинираме тоталния автомат 
  \[\A' = \pair{Q\cup\{q_e\}, \Sigma, \delta', s, F},\]
  като за всеки преход $(q,a)$, за който $\delta$ не е дефинирана, 
  дефинираме $\delta'$ да отива в новото състояние $q_e$.
  Ето и цялата дефиниция на новата функция на преходите $\delta'$:
  \begin{itemize}
  \item 
    \marginpar{$q_e$ - error състояние}
    $\delta'(q_e,a) = q_e$, за всяко $a\in\Sigma$;
  \item
    \marginpar{$\A'$ симулира $\A$}
    За всяко $q\in Q$, $a\in\Sigma$, ако $\delta(q,a) = p$, то
    $\delta'(q,a) = p$;
  \item
    За всяко $q\in Q$, $a\in\Sigma$, ако $\delta(q,a)$ не е дефинирано, то
    $\delta'(q,a) = q_e$.
  \end{itemize}
  \marginpar{\writedown Довършете доказателството!}
  Сега лесно може да се докаже, че $\L(\A) = \L(\A')$.
\end{proof}

\begin{prop}
  \label{pr:automata-union}
  Класът на автоматните езици е затворен относно операцията {\em обединение}.
  Това означава, че ако $L_1$ и $L_2$ са два произволни автоматни езика над азбуката $\Sigma$, то $L_1\cup L_2$
  също е автоматен език.
\end{prop}
\begin{proof}
  \marginpar{Защо изискваме $\A_1$ и $\A_2$ да са тотални?}
  Нека $L_1 = \L(\A_1)$ и $L_2 = \L(\A_2)$, 
  където \[\A_1 = \FAn{1},\ \A_2 = \FAn{2}\] са {\em тотални}.
  Определяме автомата $\A = \FA$, който разпознава $L_1\cup L_2$ по следния начин:
  \begin{itemize}
  \item
    $Q = Q_1\times Q_2$;
  \item
    \marginpar{Едновременно симулираме изчисление и по двата автомата}
    \marginpar{По-нататък ще дадем друга конструкция за обединение, която ще бъде по-ефективна относно броя на състоянията}
    \marginpar{\writedown Проверете, че $\L(\A) = \L(\A_1)\cup \L(\A_2)$}
    Определяме за всяко $\pair{r_1,r_2} \in Q$ и всяко $a \in \Sigma$,
    \[\delta(\pair{r_1,r_2},a) = \pair{\delta_1(r_1,a),\delta_2(r_2,a)};\]
  \item
    $s = \pair{s_1,s_2}$;
  \item
    $F = \{\pair{r_1,r_2}\mid r_1\in F_1\vee r_2 \in F_2\} = (F_1\times Q_2)\cup (Q_1\times F_2)$.
  \end{itemize}
\end{proof}

\begin{cor}
  Класът на автоматните езици е затворен относно операцията {\bf сечение}.
  Това означава, че ако $L_1$ и $L_2$ са два произволни автоматни езика над азбуката $\Sigma$, то $L_1\cap L_2$
  също е автоматен език.
\end{cor}
\begin{proof}
  \marginpar{\ding{45} Докажете, че така построения автомат $\A$ разпознава $L_1\cap L_2$!}
  Използвайте конструкцията на автомата $\A$ от \Prop{automata-union},
  с единствената разлика, че тук избираме финалните състояния да бъдат елементите на множеството
  \[F = \{\pair{q_1,q_2} \mid q_1 \in F_1\ \&\ q_2 \in F_2\} = F_1\times F_2.\]
\end{proof}

\begin{prop}
  \label{pr:automata-complement}
  Нека $L$ е автоматен език.
  Тогава $\Sigma^\star\setminus L$ също е автоматен език.
\end{prop}
\begin{proof}
  \marginpar{Защо искаме $\A$ да бъде тотален ?}
  Нека $L = L(\A)$, където $\A = \FA$ е {\bf тотален}.
  Да вземем автомата $\A' = \pair{Q,\Sigma,s,\delta,Q\setminus F}$,
  т.е. $\A'$ е същия като $\A$, с единствената разлика, че финалните състояния на $\A'$
  са тези състояния, които {\bf не} са финални в $\A$.
  \marginpar{\writedown Проверете, че $\Sigma^\star\setminus L = \L(\A')$}
\end{proof}


\begin{example}
  \label{ex:automata-pictures}
  Да разгледаме няколко примера за автомати и езиците, които разпознават.
  Дефинирайте функцията на преходите $\delta$ за всеки автомат.

  \begin{figure}[H]
    \begin{subfigure}[b]{0.5\textwidth}
      \begin{tikzpicture}[->,>=stealth,thick,node distance=45pt]
        \tikzstyle{every state}=[circle,minimum size=20pt,auto]
        
        \node[initial below, state]   (0) {$s$};
        \node[state]            (1) [right of=0]{$q_1$};
        \node[state]            (2) [right of=1]{$q_2$};
        \node[state,accepting]  (3) [right of=2]{$q_3$};
        
        \path 
        (0) edge [loop above]   node [above] {$a$}    (0)
        (0) edge [bend left=15] node [above] {$b$}    (1)
        (1) edge [loop above]   node [above] {$b$}    (1)
        (1) edge [bend left=15] node [above] {$a$}    (2)
        (2) edge [bend left=30] node [below] {$a$}    (0)
        (2) edge [bend left=15] node [above] {$b$}    (3)
        (3) edge [loop above]   node [above] {$a,b$}  (3);
      \end{tikzpicture}
      \caption{$\{\omega \in \{a,b\}^\star \mid \omega\mbox{ съдържа }bab\}$}
    \end{subfigure}
    \quad
    \begin{subfigure}[b]{0.4\textwidth}
      \begin{tikzpicture}[->,>=stealth,thick,node distance=45pt]
        \tikzstyle{every state}=[circle,minimum size=20pt,auto]
        
        \node[initial below, state]   (0) {$s$};
        \node[state]            (1) [right of=0]{$q_1$};
        \node[state,accepting]  (2) [right of=1]{$q_2$};
        
        \path 
        (0) edge [loop above]   node [above] {$b$}    (0)
        (0) edge [bend left=15] node [above] {$a$}    (1)
        (1) edge [loop above]   node [above] {$b$}    (1)
        (1) edge [bend left=15] node [above] {$a$}    (2)
        (2) edge [loop above]   node [above] {$a,b$}  (2);
      \end{tikzpicture}
      \caption{$\{\omega \in \{a,b\}^\star \mid N_a(\omega) \geq 2\}$}
    \end{subfigure}
    \begin{subfigure}[b]{0.5\textwidth}
      \begin{tikzpicture}[->,>=stealth,thick,node distance=45pt]
        \tikzstyle{every state}=[circle,minimum size=20pt,auto]
        
        \node[initial below, accepting, state] (0) {$s$};
        \node[state]                     (1) [right of=0]{$q_1$};
        
        \path 
        (0) edge [loop above]   node [above] {$b$}   (0)
        (0) edge [bend left=15] node [above] {$a$}   (1)
        (1) edge [bend left=15] node [below] {$b$}   (0);
      \end{tikzpicture}
      \caption{$\{\omega \in \{a,b\}^\star \mid $ всяко $a$ в $\omega$ се следва от поне едно $b\}$ }
    \end{subfigure}
    \qquad
    \begin{subfigure}[b]{0.5\textwidth}
      \begin{tikzpicture}[->,>=stealth,thick,node distance=45pt]
        \tikzstyle{every state}=[circle,minimum size=20pt,auto]
        
        \node[initial below, state, accepting]   (0) {$s$};
        \node[state]                       (1) [right of=0]{$q_1$};
        \node[state]                       (2) [right of=1]{$q_2$};
        
        \path 
        (0) edge [loop above]   node   [above] {$b$}    (0)
        (0) edge [bend left=15] node   [above] {$a$}    (1)
        (1) edge [loop above]   node   [above] {$b$}    (1)
        (1) edge [bend left=15] node   [above] {$a$}    (2)
        (2) edge [loop above]   node   [above] {$b$}    (2)
        (2) edge [bend left=30] node   [below] {$a$}    (0);
      \end{tikzpicture}
      \caption{$\{\omega \in \{a,b\}^\star \mid N_a(\omega) \equiv 0\ (\bmod\ 3)\}$}
    \end{subfigure}
  \end{figure}    
\end{example}

В повечето от горните примери може сравнително лесно да се съобрази, че построения автомат разпознава желания език.
При по-сложни задачи обаче, ще се наложи да дадем доказателство, като обикновено се прилага 
{\em метода на математическата индукция} върху дължината на думите.
Ще разгледаме няколко такива примера.

\begin{problem}
  Докажете, че езикът $L$ е автоматен, където
  \[L = \{\alpha \in \{a,b\}^\star\ \mid\ \alpha\mbox{ не съдържа две поредни срещания на }a\}.\]
\end{problem}
\begin{proof}
  \marginpar{Най-лесния начин да се сетим как да построим $\A$ е като първо построим тотален автомат за езика, който разпознава тези думи, в които се съдържат две поредни срещания на $a$ и вземем неговото допълнение съгласно Твърдение \ref{pr:automata-complement}}
  Да разгледаме $\A = \FA$ с функция на преходите
  \begin{figure}[H]
    \begin{center}
      \begin{tikzpicture}[->,>=stealth,thick,node distance=45pt]
        \tikzstyle{every state}=[circle,minimum size=20pt,auto]
        
        \node[initial, accepting, state] (0) {$s$};
        \node[accepting, state]   (1) [right of=0]{$q_1$};
        \node[state]   (2) [right of=1]{$q_2$};
        
        \path 
        (0) edge [loop above]   node [above] {$b$}   (0)
        (0) edge [bend left=15] node [above] {$a$}   (1)
        (1) edge [bend left=15] node [below] {$b$}   (0)
        (1) edge [bend left=15] node [above] {$a$}   (2)
        (2) edge [loop above]   node [above] {$a,b$} (2);
      \end{tikzpicture}
    \end{center}
 \end{figure}

 Ще докажем, че $L = \L(\A)$.
 Първо ще се концентрираме върху доказателството на $\L(\A) \subseteq L$.
 \marginpar{Озн. $\abs{\alpha}$ - дължината на думата $\alpha$}
 Ще докажем с индукция по дължината на думата $\alpha$, че:
 \begin{enumerate}[(1)]
 \item 
   ако $\delta^\star(s,\alpha) = s$, то
   $\alpha$ не съдържа две поредни срещания на $a$
   и ако $\abs{\alpha} > 0$, то $\alpha$ завършва на $b$;
 \item
   ако $\delta^\star(s,\alpha) = q_1$, то
   $\alpha$ не съдържа две поредни срещания на $a$
   и завършва на $a$.
 \end{enumerate}

 За $\abs{\alpha} = 0$, то твърденията (1) и (2) са ясни (Защо?).
 Да приемем, че твърденията $(1)$ и $(2)$ са верни за произволни думи $\alpha$ с дължина $n$.
 Нека $\abs{\alpha} = n+1$, т.е. $\alpha = \beta x$, където $\abs{\beta} = n$ и $x \in \Sigma$.
 Ще докажем (1) и (2) за $\alpha$.
 \begin{itemize}[-]
 \item 
   Нека $\delta^\star(s,\beta x) = s = \delta(\delta^\star(s,\beta),x)$.
   Според дефиницията на функцията $\delta$, $x = b$ и $\delta^\star(s,\beta) \in \{s,q_1\}$.
   Тогава по {\bf И.П.} за (1) и (2), $\beta$ не съдържа две поредни срещания на $a$.
   Тогава е очевидно, че $\beta x$ също не съдържа две поредни срещания на $a$.
 \item
   Нека $\delta^\star(s,\beta x) = q_1 = \delta(\delta^\star(s,\beta),x)$.
   Според дефиницията на $\delta$, $x = a$ и $\delta^\star(s,\beta) = s$.
   Тогава по {\bf И.П.} за (2), $\beta$ не съдържа две поредни срещания на $a$
   и завършва на $b$.
   Тогава е очевидно, че $\beta x$ също не съдържа две поредни срещания на $a$.
 \end{itemize}
 
 Така доказахме с индукция по дължината на думата, че за всяка дума $\alpha$
 са  изпълнени твърденията $(1)$ и $(2)$. По дефиниция, ако $\alpha \in \L(\A)$,
 то $\delta^\star(s,\alpha) \in \{s,q_1\}$ и от $(1)$ и $(2)$ следва, че и в двата случа
 $\alpha$ не съдържа две поредни срещания на буквата $a$, т.е. $\alpha \in L$.
 С други думи, доказахме, че 
 \[\L(\A) \subseteq L.\]

 Сега ще докажем другата посока, т.е. $L \subseteq \L(\A)$.
 Това означава да докажем, че
 \[(\forall \alpha \in \Sigma^\star)[\alpha \in L\ \Rightarrow\ \delta^\star(s,\alpha) \in F],\]
 \marginpar{Да напомним, че $p \Rightarrow q \equiv \neg q \Rightarrow \neg  p$}
 което е еквивалентно на
 \begin{equation}
   \label{eq:case2}
   (\forall \alpha \in \Sigma^\star)[\delta^\star(s,\alpha) \not\in F \ \Rightarrow\ \alpha\not\in L].
 \end{equation}
 Това е лесно да се съобрази.
 Щом $\delta^\star(s,\alpha) \not\in F$, то 
 $\delta^\star(s,\alpha) = q_2$ и думата $\alpha$ може да се представи по следния начин:
 \[\alpha = \beta a \gamma\ \&\ \delta^\star(s,\beta) = q_1.\]
 
 Използвайки свойство (2) от по-горе, понеже $\delta^\star(s,\beta) = q_1$, то
 $\beta$ не съдържа две поредни срещания на $a$, но завършва на $a$.
 Сега е очевидно, че $\beta a$ съдържа две поредни срещания на $a$ и 
 щом $\beta a$ е префикс на $\alpha$, то думата $\alpha \not\in L$.
 С това доказахме Свойство \ref{eq:case2}, а следователно и посоката $L\subseteq \L(\A)$.
\end{proof}

\begin{problem}
  Докажете, че следния език е автоматен:
  \[L = \{w \in \{a,b\}^\star \mid a \text{ се среща четен брой, докато $b$ нечетен брой пъти в }w\}.\]
\end{problem}
\begin{hint}
  Разгледайте автомата $\A$:
  \begin{figure}[H]
    \begin{center}
      \begin{tikzpicture}[->,>=stealth,thick,node distance=50pt]
        \tikzstyle{every state}=[circle,minimum size=15pt,auto]
        
        \node[initial,state]      (00) {$00$};
        \node[state]              (10) [above right of=00]{$10$};
        \node[accepting, state]   (01) [below right of=00]{$01$};
        \node[state]              (11) [below right of=10]{$11$};
        
        \path 
        (00) edge  [bend left=15]  node [above]  {$a$} (10)
        (10) edge  [bend left=15]  node [below]  {$a$} (00)
        (01) edge  [bend right=15]  node [above]  {$b$} (00)
        (00) edge  [bend right=15]  node [below]  {$b$} (01)
        (10) edge  [bend left=15]  node [above]  {$b$} (11)
        (11) edge  [bend left=15]  node [below]  {$b$} (10)
        (01) edge  [bend right=15]  node [below]  {$a$} (11)
        (11) edge  [bend right=15]  node [above]  {$a$} (01);
      \end{tikzpicture}
    \end{center}
    \caption{$\L(\A) \stackrel{?}{=} L$}
  \end{figure}
  Докажете с индукция по дължината на думата $\omega$, че:
  \begin{enumerate}[a)]
  \item 
    $\delta^\star(q_{00}, \omega) = q_{00} \implies (\exists k,n\in\Nat)[N_a(\omega) = 2k\ \&\ N_b(\omega) = 2n]$;
  \item 
    $\delta^\star(q_{00}, \omega) = q_{01} \implies (\exists k,n\in\Nat)[N_a(\omega) = 2k\ \&\ N_b(\omega) = 2n+1]$;
  \item 
    $\delta^\star(q_{00}, \omega) = q_{10} \implies (\exists k,n\in\Nat)[N_a(\omega) = 2k+1\ \&\ N_b(\omega) = 2n]$;
  \item 
    $\delta^\star(q_{00}, \omega) = q_{11} \implies (\exists k,n\in\Nat)[N_a(\omega) = 2k+1\ \&\ N_b(\omega) = 2n+1]$;
  \end{enumerate}
\end{hint}


\begin{problem}
  Докажете, че езикът $L = \{w \in \{0,1\}^\star \mid |w| \equiv 1 \bmod 3\}$ е автоматен.
\end{problem}
\begin{hint}
  \marginpar{$|w| = $ дължината на думата $w$}
  Разгледайте автомата $\A$:
  \begin{figure}[H]
    \begin{center}
      \begin{tikzpicture}[->,>=stealth,thick,node distance=45pt]
        \tikzstyle{every state}=[circle,minimum size=15pt,auto]
        
        \node[initial,state]      (0) {$q_0$};
        \node[accepting, state]              (1) [right of=0]{$q_1$};
        \node[state]   (2) [right of=1]{$q_2$};
        
        \path 
        (0) edge  [bend left=15]  node [above]  {$0,1$} (1)
        (1) edge  [bend left=15]  node [above]  {$0,1$} (2)
        (2) edge  [bend left=30]  node [below]  {$0,1$} (0);
      \end{tikzpicture}
      \end{center}
      \caption{$\L(\A) \stackrel{?}{=} \{w\in\{0,1\}^\star \mid |w| \equiv 1 \bmod 3\}$}
 \end{figure}  
 Докажете с индукция по дължината на думата $w$ едновременно следните три твърдения:
 \begin{enumerate}[(1)]
 \item
   $\delta^\star(q_0, w) = q_0 \implies |w| \equiv 0 \bmod 3$;
 \item 
   $\delta^\star(q_0, w) = q_1 \implies |w| \equiv 1 \bmod 3$;
 \item
   $\delta^\star(q_0, w) = q_2 \implies |w| \equiv 2 \bmod 3$.
 \end{enumerate}
\end{hint}

За една дума $\alpha \in \{0,1\}^\star$, 
нека с $\alpha_{(2)}$ да означим числото в десетична бройна система, което се представя в двоична бройна система като $\alpha$.
Например, $1101_{(2)} = 1 \cdot 2^3+1\cdot 2^2+0\cdot 2^1+1\cdot 2^0 = 13$.
Тогава имаме следните свойства:
\begin{itemize}
\item
  $\varepsilon_{(2)} = 0$,
\item
  $(\alpha0)_{(2)} = 2\cdot(\alpha)_{(2)}$,
\item
  $(\alpha1)_{(2)} = 2\cdot(\alpha)_{(2)} + 1$.
\end{itemize}
\marginpar{Да отбележим, че за всяко число $n$ има безкрайно много думи $\alpha$, за които $\alpha_{(2)} = n$. Например, $10_{(2)} = 010_{(2)} = 0010_{(2)} = \cdots$}

\begin{problem}
  Докажете, че $L = \{\omega \in \{0,1\}^\star \mid \omega_{(2)} \equiv 2\ (\bmod\ 3)\}$ е автоматен.
\end{problem}
\begin{proof}
  Нашият автомат ще има три състояния $\{q_0,q_1,q_2\}$, като началното състояние ще бъде $q_0$.
  Целта ни е да дефинираме така автомата, че да имаме следното свойство:
  \begin{equation}
    (\forall\alpha\in\Sigma^\star)(\forall i < 3)[\alpha_{(2)} \equiv i\ (\bmod\ 3)\ \Leftrightarrow\ \delta^\star(q_0,\alpha) = q_i],
  \end{equation}
  т.е. всяко състояние отговаря на определен остатък при деление на три.
  Понеже искаме нашия автомат да разпознава тези думи $\alpha$,
  за които $\alpha_{(2)} \equiv 2\mod 3$, финалното състояние ще бъде $q_2$.
  Дефинираме функцията $\delta$ следвайки свойствата:
  \begin{itemize}
  \item
    $\delta(q_0,0) = q_0$, защото ако $\alpha_{(2)} \equiv 0 \bmod 3$, то $(\alpha0)_{(2)} \equiv 0 \bmod 3$;
  \item 
    $\delta(q_0,1) = q_1$, защото ако $\alpha_{(2)} \equiv 0 \bmod 3$, то $(\alpha1)_{(2)} \equiv 1 \bmod 3$;
  \item
    $\delta(q_1,0) = q_2$, защото ако $\alpha_{(2)} \equiv 1 \bmod 3$, то $(\alpha0)_{(2)} \equiv 2 \bmod 3$;
  \item 
    $\delta(q_1,1) = q_0$, защото ако $\alpha_{(2)} \equiv 1 \bmod 3$, то $(\alpha1)_{(2)} \equiv 0 \bmod 3$;
  \item
    $\delta(q_2,0) = q_1$, защото ако $\alpha_{(2)} \equiv 2 \bmod 3$, то $(\alpha0)_{(2)} \equiv 1 \bmod 3$;
  \item 
    $\delta(q_2,1) = q_2$, защото ако $\alpha_{(2)} \equiv 2 \bmod 3$, то $(\alpha1)_{(2)} \equiv 2 \bmod 3$.
  \end{itemize}
  Ето и картинка на автомата $\A$:
  \begin{framed}
  \begin{figure}[H]
    \begin{center}
      \begin{tikzpicture}[->,>=stealth,thick,node distance=45pt]
        \tikzstyle{every state}=[circle,minimum size=15pt,auto]
        
        \node[initial,state]      (0) {$q_0$};
        \node[state]              (1) [right of=0]{$q_1$};
        \node[accepting, state]   (2) [right of=1]{$q_2$};
        
        \path 
        (0) edge  [loop above]    node [above]  {$0$} (0)
        (0) edge  [bend left=15]  node [above]  {$1$} (1)
        (2) edge  [bend left=15] node [below]  {$0$} (1)
        (1) edge  [bend left=15]  node [below]  {$1$} (0)
        (1) edge  [bend left=15] node [above]  {$0$} (2)
        (2) edge  [loop above]    node [above]  {$1$} (2);
      \end{tikzpicture}
      \end{center}
      \caption{$\L(\A) \stackrel{?}{=} \{\alpha\in\{0,1\}^\star \mid \alpha_{(2)} \equiv 2\ (\bmod\ 3)\}$}
 \end{figure}
 \end{framed}
 \noindent Да разгледаме твърденията:
 \begin{enumerate}[(1)]
  \item 
    $\delta^\star(q_0,\alpha) = q_0 \implies \alpha_{(2)} \equiv 0 \mod 3$;
  \item 
    $\delta^\star(q_0,\alpha) = q_1 \implies \alpha_{(2)} \equiv 1 \mod 3$;
  \item 
    $\delta^\star(q_0,\alpha) = q_2 \implies \alpha_{(2)} \equiv 2 \mod 3$.
  \end{enumerate}
  Ще докажем (1), (2) и (3) {\em едновременно} с индукция по дължината на думата $\alpha$.
  За $\abs{\alpha} = 0$, всички условия са изпълнени. (Защо?)
  Да приемем, че (1), (2) и (3) са изпълнени за думи с дължина $n$.
  Нека $\abs{\alpha} = n+1$, т.е. $\alpha = \beta x$, $\abs{\beta} = n$.
  За да приложим индукционното предположение, ще използваме следното свойство:
  \[\delta^\star(q_0,\beta x) = \delta(\delta^\star(q_0,\beta),x).\]
  
  Ще докажем подробно само (3) понеже другите твърдения се доказват по сходен начин.
  \marginpar{Обърнете внимание, че в доказателството на (3) използваме И.П. не само за (3), но и за (2). Поради тази причина трябва да докажем трите свойства едновременно}
  Нека $\delta^\star(q_0,\beta x) = q_2$. 
  Имаме два случая:
  \begin{itemize}
  \item 
    $x = 0$. 
    Тогава, по дефиницията на $\delta$, 
    $\delta(q_1,0) = q_2$ и следователно, $\delta^\star(q_0,\beta) = q_1$.
    По {\bf И.П.} за (2) с $\beta$,
    \[\delta^\star(q_0,\beta) = q_1\ \Rightarrow\ \beta_{(2)} \equiv 1 \bmod 3\]
    Тогава, $(\beta0)_{(2)} \equiv 2 \mod 3$. Така доказахме, че
    \[\delta^\star(q_0,\beta 0) = q_2\ \Rightarrow\ (\beta 0)_{(2)} \equiv 2 \bmod 3.\]
  \item
    $x = 1$.
    Тогава, по дефиницията на $\delta$, $\delta(q_2,1) = q_2$ и следователно,
    $\delta^\star(q_0,\beta) = q_2$.
    По {\bf И.П.} за (3) с $\beta$,
    \[\delta^\star(q_0,\beta) = q_2\ \Rightarrow\ \beta_{(2)} \equiv 2 \mod 3.\]
    Тогава, $(\beta1)_{(2)} \equiv 2 \mod 3$. Така доказахме, че
    \[\delta^\star(q_0,\beta 1) = q_2\ \Rightarrow\ (\beta 1)_{(2)} \equiv 2 \mod 3.\]
  \end{itemize}
  
  За да докажем (1), нека $\delta^\star(q_0,\beta x) = q_0$. 
  \begin{itemize}
  \item 
    $x = 0$. Разсъжденията са аналогични, като използваме {\bf И.П.} за (1).
  \item
    $x = 1$. Разсъжденията са аналогични, като използваме {\bf И.П.} за (2).
  \end{itemize}
  
  По същия начин доказваме и (2). Нека $\delta^\star(q_0,\beta x) = q_1$. 
  \begin{itemize}
  \item 
    При $x = 0$, използваме {\bf И.П.} за (3).
  \item
    При $x = 1$, използваме {\bf И.П.} за (1).
  \end{itemize}

  От (1), (2) и (3) следва директно, че $\L(\A) \subseteq L$.
  
  За другата посока, нека $\alpha \in L$, т.е. $\alpha_{(2)} \equiv 2 \bmod 3$.
  Ако допуснем, че $\alpha \not\in \L(\A)$, то това означава, че $\delta^\star(q_0,\alpha) \in \{q_0,q_1\}$.
  Но в тези случаи получаваме от твърдения (1) и (2), че $\alpha_{(2)} \equiv 0 \bmod 3$ или $\alpha_{(2)} \equiv 1 \bmod 3$.
  Това е противоречие с избора на $\alpha \in L$. Следователно, ако $\alpha \in L$, то $\delta(q_0,\alpha) = q_2$.
  Така доказахме и посоката $L \subseteq \L(\A)$.
\end{proof}


%%% Local Variables:
%%% mode: latex
%%% TeX-master: "../eai"
%%% End:
