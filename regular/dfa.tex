\section{Автоматни езици}

\begin{definition}
  \mynote{Често ще пишем съкратено ДКА вместо детерминиран краен автомат.}
  Детерминиран краен автомат е петорка $\A = \FA$, където
  \begin{itemize}
  \item
    $\Sigma$ е азбука;
  \item
    $Q$ е крайно множество от състояния;
  \item
    $\delta:Q\times\Sigma\to Q$ е тотална функция, която ще наричаме
    \emph{функция на преходите};
  \item
    $\qstart\in Q$ е начално състояние;
  \item
    $F\subseteq Q$ е множеството от финални състояния
  \end{itemize}
\end{definition}
\index{автомат!детерминиран}

Нека имаме една дума $\alpha \in \Sigma^\star$, $\alpha = a_0a_1\cdots a_{n-1}$.
Казваме, че $\alpha$ се {\bf разпознава} от автомата $\A$, ако
съществува редица от състояния $q_0,q_1,q_2,\dots,q_n$, такива че:
\begin{itemize}
\item
  $q_0 = \qstart$, началното състояние на автомата;
\item
  $\delta(q_i,a_{i}) = q_{i+1}$, за всяко $i = 0, \dots, n-1$;
\item
  $q_n \in F$.
\end{itemize}

Казваме, че $\A$ {\bf разпознава} езика $L$, ако $\A$ разпознава точно думите от $L$, т.е.
$L = \{\alpha \in \Sigma^\star \mid \A\mbox{ разпознава }\alpha\}$.
Обикновено означаваме езика, който се разпознава от даден автомат $\A$ с $\L(\A)$.
\index{език!автоматен}
В такъв случай ще казваме, че езикът $L$ е {\bf автоматен}.

При дадена функция на преходите $\delta$,
често е удобно да разглеждаме функция $\delta^\star:Q\times\Sigma^\star \to Q$, която е дефинирана за всяко $q\in Q$ и $\alpha \in \Sigma^\star$ по следния начин:
\index{$\delta^\star$}
% \mynote{Това е пример за индуктивна (рекурсивна) дефиниция по дължината на думата $\alpha$}
\begin{itemize}
\item
  Ако $\alpha = \varepsilon$, то $\delta^\star(q,\varepsilon) \df q$;
\item
  Ако $\alpha = \beta a$, то $\delta^\star(q,\beta a) \df \delta(\delta^\star(q,\beta), a)$.
\end{itemize}
Лесно се съобразява, че една дума $\alpha$ се {\em разпознава} от автомата $\A$ точно тогава, когато $\delta^\star(\qstart,\alpha) \in F$.
Оттук следва, че
\begin{framed}
\[\L(\A) \df \{\ \alpha\in\Sigma^\star \mid \delta^\star(\qstart,\alpha) \in F\ \}.\]
\end{framed}

\mynote{Обърнете внимание, че $\delta(q,a) = \delta^\star(q,a)$ за $a\in\Sigma$}

\begin{proposition}
  \label{pr:dfa:delta-star}
  Нека $\A$ е ДКА. Тогава за всяко състояние $q$ и произволни думи $\alpha$ и $\beta$ е изпълнено, че
  \[\delta^\star(q,\alpha\beta) = \delta^\star(\delta^\star(q,\alpha),\beta).\]
\end{proposition}
\begin{hint}
  Индукция по дължината на $\beta$.

  \begin{itemize}
  \item
    $|\beta| = 0$, т.е. $\beta = \varepsilon$. Тогава за всяко състояние $q$ и произволна дума $\alpha$ имаме:
    \[\delta^\star(q, \alpha\varepsilon) = \delta^\star( \underbrace{\delta^\star(q, \alpha)}_{q}, \varepsilon).\]
  \item
    Да приемем, че твърдението е изпълнено за думи $\beta$ с дължина $n$.
  \item
    Нека $|\beta| = n+1$, т.е. $\beta = \gamma b$, където $|\gamma| = n$. Тогава за всяко състояние $q$ и произволна дума $\alpha$ имаме:
    \begin{align*}
      \delta^\star(q, \alpha\beta) & = \delta^\star(q, \alpha \gamma b)\\
                                   & = \delta( \delta^\star(q, \alpha\gamma),b) & \comment\text{от деф. на }\delta^\star\\
                                   & = \delta(\delta^\star(\underbrace{\delta^\star(q,\alpha)}_{p}, \gamma), b) & \comment{\text{от И.П. приложено за }\gamma}\\
                                   & = \delta( \delta^\star(p, \gamma), b) \\
                                   & = \delta^\star( p, \gamma b) & \comment\text{от деф. на }\delta^\star\\
                                   & = \delta^\star(\delta^\star(q, \alpha), \beta). & \comment{p = \delta^\star(q,\alpha)}
    \end{align*}
  \end{itemize}
\end{hint}

\begin{remark}
Формално погледнато, доказателството на \Proposition{dfa:delta-star} протича по следната схема:
  \begin{prooftree}
    \AxiomC{$P(0)$}
    \AxiomC{$(\forall n\in\Nat)[P(n) \implies P(n+1)]$}
    \BinaryInfC{$(\forall n \in \Nat)[P(n)]$,}
  \end{prooftree}
  където свойството $P$ е дефинирано така:
  \[P(n) \df (\forall \beta\in\Sigma^n)(\forall \alpha\in\Sigma^\star)(\forall q\in Q)[\delta^\star(q,\alpha\beta) = \delta^\star(\delta^\star(q,\alpha),\beta)].\]
\end{remark}


\index{моментно описание}
\mynote{(На англ. {\em instantaneous description}). В случая въвеждането на това понятие не е напълно необходимо, но по-късно, когато въведем стекови автомати и машини на Тюринг, ще работим с такива моментни описания на изчисления и затова е добре още отначало да свикнем с него.}
{\bf Моментното описание} на изчисление с краен автомат представлява двойка от вида $(q,\alpha) \in Q\times\Sigma^\star$,
т.е. автоматът се намира в състояние $q$, а думата, която остава да се прочете е $\alpha$.
Удобно е да въведем бинарната релация $\vdash_\A$ над $Q\times\Sigma^\star$,
която ще ни казва как моментното описание на автомата $\A$ се променя след изпълнение на една стъпка:
\[(q,b\alpha) \vdash_\A (p,\alpha) \stackrel{\text{деф}}{\iff} \delta(q,b) = p.\]
Рефлексивното и транзитивно затваряне на $\vdash_\A$ ще означаваме с $\vdash^\star_\A$.
За да дадем по-удобна дефиниция на $\vdash^\star_\A$, първо ще дефинираме релацията $\vdash^n_\A$, която
определя работата на автомата $\A$ върху моментни описания за $n$ стъпки.
\mynote{Рефл. и транз. затваряне на една релация е разгледано в Глава \ref{ch:intro}.}

\begin{figure}[H]
\begin{subfigure}[b]{0.5\textwidth}
\begin{prooftree}
  \AxiomC{}
  \RightLabel{\scriptsize{(рефлексивност)}}
  \UnaryInfC{$(q,\alpha) \vdash^0 (q,\alpha)$}
\end{prooftree}
\end{subfigure}
~
\begin{subfigure}[b]{0.5\textwidth}
\begin{prooftree}
  \AxiomC{$(q,\alpha) \vdash^n (r, b\beta)$}
  \AxiomC{$\delta(r,b) = p$}
  \RightLabel{\scriptsize{(транзитивност)}}
  \BinaryInfC{$(q,\alpha) \vdash^{n+1} (p,\beta)$}
\end{prooftree}  
\end{subfigure}
\end{figure}

Дефинираме релацията $\vdash^\star_\A$ като рефлексивното и транзитивно затваряне на релацията $\vdash_A$, или с други думи:
\[(q,\alpha) \vdash^\star_\A (p,\beta) \dff (\exists \ell\in\Nat)[(q,\alpha) \vdash^\ell_\A (p,\beta)].\]

\begin{problem}
  Докажете, че за произволна дума $\beta \in \Sigma^\star$,
  \[(q,\alpha\beta) \vdash^\star_\A (p, \beta) \iff \delta^\star(q,\alpha) = p.\]  
\end{problem}
Можем да заключим, че имаме следното преставяне:
\[\L(\A) = \{\alpha\in\Sigma^\star \mid (\exists q \in F)[(\qstart,\alpha) \vdash^\star_\A(q,\varepsilon)]\}.\]


%%% Local Variables:
%%% mode: latex
%%% TeX-master: "../eai"
%%% End:
