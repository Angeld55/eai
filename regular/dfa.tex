\section{Автоматни езици}

\begin{dfn}
  Краен автомат е петорка $\A = \FA$, където
  \begin{itemize}
  \item
    $Q$ е крайно множество от състояния;
  \item
    $\Sigma$ е азбука;
  \item
    $\delta:Q\times\Sigma\to Q$ е {\em тотална} функция на преходите;
  \item
    $\qstart\in Q$ е начално състояние;
  \item
    $F\subseteq Q$ е множеството от финални състояния
  \end{itemize}
\end{dfn}
\index{автомат!детерминиран}

Нека имаме една дума $\alpha \in \Sigma^\star$, $\alpha = a_0a_1\cdots a_{n-1}$.
Казваме, че $\alpha$ се {\bf разпознава} от автомата $\A$, ако
съществува редица от състояния $q_0,q_1,q_2,\dots,q_n$, такива че:
\begin{itemize}
\item
  $q_0 = \qstart$, началното състояние на автомата;
\item
  $\delta(q_i,a_{i}) = q_{i+1}$, за всяко $i = 0, \dots, n-1$;
\item
  $q_n \in F$.
\end{itemize}

Казваме, че $\A$ {\bf разпознава} езика $L$, ако $\A$ разпознава точно думите от $L$, т.е.
$L = \{\alpha \in \Sigma^\star \mid \A\mbox{ разпознава }\alpha\}$.
Обикновено означаваме езика, който се разпознава от даден автомат $\A$ с $\L(\A)$.
\index{език!автоматен}
В такъв случай ще казваме, че езикът $L$ е {\bf автоматен}.

При дадена функция на преходите $\delta$,
често е удобно да разглеждаме функция $\delta^\star:Q\times\Sigma^\star \to Q$, която е дефинирана по следния начин:
% \marginpar{Това е пример за индуктивна (рекурсивна) дефиниция по дължината на думата $\alpha$}
\begin{itemize}
\item 
  $\delta^\star(q,\varepsilon) = q$, за всяко $q\in Q$;
\item
  $\delta^\star(q,\beta a) = \delta(\delta^\star(q,\beta), a)$, за всяко $q\in Q$, всяко $a\in\Sigma$ и $\beta\in\Sigma^\star$.
\end{itemize}
Лесно се съобразява, че една дума $\alpha$ се {\em разпознава} от автомата $\A$ точно тогава, когато $\delta^\star(\qstart,\alpha) \in F$.
Оттук следва, че
\begin{framed}
\[\L(\A) \df \{\ \alpha\in\Sigma^\star \mid \delta^\star(\qstart,\alpha) \in F\ \}.\]
\end{framed}

\marginpar{Обърнете внимание, че $\delta(q,a) = \delta^\star(q,a)$ за $a\in\Sigma$}

\begin{prop}
  \label{pr:delta-star}
  $(\forall q\in Q)(\forall\alpha,\beta\in\Sigma^\star)[\delta^\star(q,\alpha\beta) = \delta^\star(\delta^\star(q,\alpha),\beta)]$.
\end{prop}
\begin{hint}
  Индукция по дължината на $\beta$.

  \begin{itemize}
  \item
    $|\beta| = 0$, т.е. $\beta = \varepsilon$. Тогава за произволна дума $\alpha$ имаме:
    \[\delta^\star(q, \alpha\varepsilon) = \delta^\star( \underbrace{\delta^\star(q, \alpha)}_{q}, \varepsilon).\]
  \item
    Да приемем, че твърдението е изпълнено за думи $\beta$ с дължина $n$.
  \item
    Нека $|\beta| = n+1$, т.е. $\beta = \gamma b$, където $|\gamma| = n$. Тогава за произволна дума $\alpha$ имаме:
    \begin{align*}
      \delta^\star(q, \alpha\beta) & = \delta^\star(q, \alpha \gamma b)\\
                                   & = \delta( \delta^\star(q, \alpha\gamma),b) & \comment\text{от деф. на }\delta^\star\\
                                   & = \delta(\delta^\star(\underbrace{\delta^\star(q,\alpha)}_{p}, \gamma), b) & \comment{\text{от И.П. приложено за }\gamma}\\
                                   & = \delta( \delta^\star(p, \gamma), b) \\
                                   & = \delta^\star( p, \gamma b) & \comment\text{от деф. на }\delta^\star\\
                                   & = \delta^\star(\delta^\star(q, \alpha), \beta). & \comment{p = \delta^\star(q,\alpha)}
    \end{align*}
  \end{itemize}
\end{hint}

\index{моментно описание}
\marginpar{(На англ. {\em instantaneous description}). В случая въвеждането на това понятие не е напълно необходимо, но по-късно, когато въведем стекови автомати и машини на Тюринг, ще работим с такива моментни описания на изчисления и затова е добре още отначало да свикнем с него.}
{\bf Моментното описание} на изчисление с краен автомат представлява двойка от вида $(q,\alpha) \in Q\times\Sigma^\star$,
т.е. автоматът се намира в състояние $q$, а думата, която остава да се прочете е $\alpha$.
Удобно е да въведем бинарната релация $\vdash_\A$ над $Q\times\Sigma^\star$,
която ще ни казва как моментното описание на автомата $\A$ се променя след изпълнение на една стъпка:
\[(q,x\alpha) \vdash_\A (p,\alpha), \text{ ако } \delta(q,x) = p.\]
\marginpar{Рефл. и транз. затваряне на една релация е разгледано в Глава \ref{ch:intro}}
Рефлексивното и транзитивно затваряне на $\vdash_\A$ ще означаваме с $\vdash^\star_\A$.
За да дадем точна дефиниция на $\vdash^\star_\A$, първо ще дефинираме релацията $\vdash^n_\A$, която
определя работата на автомата $\A$ за $n$ стъпки.
\begin{itemize}
\item 
  $(q,\alpha) \vdash^0_\A (q,\alpha)$, защото за $0$ стъпки се случва нищо.
\item
  Нека $\delta(q,x) = q'$ и $(q',\alpha) \vdash^n_\A (p, \beta)$. Тогава
  $(q,x\alpha) \vdash^{n+1}_\A (p,\beta)$, защото за $n+1$ стъпки първо правим една стъпка 
  и отиваме в моментното описание $(q',\alpha)$ и след това правим още $n$ стъпки.
\end{itemize}
Сега можем да дефинираме $\vdash^\star_\A$ като:
\[(q,\alpha) \vdash^\star_\A (p,\beta) \dff (\exists n\in\Nat)[(q,\alpha) \vdash^n_\A (p,\beta)].\]
Друг начин да дефинираме релацията $\vdash^\star_\A$ е следния:
\[(q,\alpha\beta) \vdash^\star_\A (p, \beta) \iff \delta^\star(q,\alpha) = p.\]
Получаваме, че 
\[\L(\A) = \{\alpha\in\Sigma^\star \mid (\exists q \in F)[(\qstart,\alpha) \vdash^\star_\A(q,\varepsilon)]\}.\]

% Нашата дефиниция на автомат позволява $\delta$ да бъде частична функция, т.е.
% може да има $q\in Q$ и $a\in\Sigma$, за които $\delta(q,a)$ не е дефинирана.
% Следващото твърдение ни казва, че ние съвсем спокойно можем да разглеждаме автомати
% само с тотални функции на преходите  $\delta$.
% \begin{prop}
%   За всеки краен автомат $\A$, съществува {\em тотален} краен автомат $\A'$,
%   за който $\L(\A) = \L(\A')$.
% \end{prop}
% \begin{hint}
%   Нека $\A = \FA$.
%   Дефинираме тоталния автомат 
%   \[\A' = \pair{Q\cup\{q_e\}, \Sigma, \qstart, \delta', F},\]
%   като за всеки преход $(q,a)$, за който $\delta$ не е дефинирана, 
%   дефинираме $\delta'$ да отива в новото състояние $q_e$.
%   Ето и цялата дефиниция на новата функция на преходите $\delta'$:
%   \begin{itemize}
%   \item 
%     \marginpar{$q_e$ - error състояние}
%     $\delta'(q_e,a) = q_e$, за всяко $a\in\Sigma$;
%   \item
%     \marginpar{$\A'$ симулира $\A$}
%     За всяко $q\in Q$, $a\in\Sigma$, ако $\delta(q,a) = p$, то
%     $\delta'(q,a) = p$;
%   \item
%     За всяко $q\in Q$, $a\in\Sigma$, ако $\delta(q,a)$ не е дефинирано, то
%     $\delta'(q,a) = q_e$.
%   \end{itemize}
%   Сега лесно може да се докаже, че $\L(\A) = \L(\A')$.
% \end{hint}


\begin{example}
  \label{ex:automata-pictures}
  Да разгледаме няколко примера за автомати и езиците, които разпознават.
  Дефинирайте функцията на преходите $\delta$ за всеки автомат.

  \begin{figure}[H]
    \begin{subfigure}[b]{0.45\textwidth}
      \begin{tikzpicture}[->,>=stealth,thick,node distance=45pt]
        \tikzstyle{every state}=[circle,minimum size=20pt,auto]
        
        \node[initial below, state]   (0) {$q_0$};
        \node[state]                  (1) [right of=0]{$q_1$};
        \node[state]                  (2) [right of=1]{$q_2$};
        \node[state,accepting]        (3) [right of=2]{$q_3$};
        
        \path 
        (0) edge [loop above]   node [above] {$a$}    (0)
        (0) edge [bend left=15] node [above] {$b$}    (1)
        (1) edge [loop above]   node [above] {$b$}    (1)
        (1) edge [bend left=15] node [above] {$a$}    (2)
        (2) edge [bend left=30] node [below] {$a$}    (0)
        (2) edge [bend left=15] node [above] {$b$}    (3)
        (3) edge [loop above]   node [above] {$a,b$}  (3);
      \end{tikzpicture}
      \caption{$\{\omega \in \{a,b\}^\star \mid \omega\mbox{ съдържа }bab\}$}
    \end{subfigure}
    \qquad
    \begin{subfigure}[b]{0.45\textwidth}
      \begin{tikzpicture}[->,>=stealth,thick,node distance=45pt]
        \tikzstyle{every state}=[circle,minimum size=20pt,auto]
        
        \node[initial below, state]   (0) {$q_0$};
        \node[state]                  (1) [right of=0]{$q_1$};
        \node[state,accepting]        (2) [right of=1]{$q_2$};
        
        \path 
        (0) edge [loop above]   node [above] {$b$}    (0)
        (0) edge [bend left=15] node [above] {$a$}    (1)
        (1) edge [loop above]   node [above] {$b$}    (1)
        (1) edge [bend left=15] node [above] {$a$}    (2)
        (2) edge [loop above]   node [above] {$a,b$}  (2);
      \end{tikzpicture}
      \caption{$\{\omega \in \{a,b\}^\star \mid N_a(\omega) \geq 2\}$}
    \end{subfigure}
    \begin{subfigure}[b]{0.45\textwidth}
      \begin{tikzpicture}[->,>=stealth,thick,node distance=45pt]
        \tikzstyle{every state}=[circle,minimum size=20pt,auto]
        
        \node[initial below, accepting, state] (0) {$q_0$};
        \node[state]                           (1) [right of=0]{$q_1$};
        \node[state]                           (2) [right of=1]{$q_2$};
        
        \path 
        (0) edge [loop above]   node [above] {$b$}   (0)
        (0) edge [bend left=15] node [above] {$a$}   (1)
        (1) edge [bend left=15] node [below] {$b$}   (0)
        (1) edge [bend left=15] node [above] {$a$}   (2)
        (2) edge [loop above] node   [above] {$a,b$} (2);
      \end{tikzpicture}
      \caption{$\{\omega \in \{a,b\}^\star \mid $ всяко $a$ в $\omega$ се следва от поне едно $b\}$ }
    \end{subfigure}
    \qquad
    \begin{subfigure}[b]{0.45\textwidth}
      \begin{tikzpicture}[->,>=stealth,thick,node distance=45pt]
        \tikzstyle{every state}=[circle,minimum size=20pt,auto]
        
        \node[initial below, state, accepting]   (0) {$q_0$};
        \node[state]                             (1) [right of=0]{$q_1$};
        \node[state]                             (2) [right of=1]{$q_2$};
        
        \path 
        (0) edge [loop above]   node   [above] {$b$}    (0)
        (0) edge [bend left=15] node   [above] {$a$}    (1)
        (1) edge [loop above]   node   [above] {$b$}    (1)
        (1) edge [bend left=15] node   [above] {$a$}    (2)
        (2) edge [loop above]   node   [above] {$b$}    (2)
        (2) edge [bend left=30] node   [below] {$a$}    (0);
      \end{tikzpicture}
      \caption{$\{\omega \in \{a,b\}^\star \mid N_a(\omega) \equiv 0\ (\bmod\ 3)\}$}
    \end{subfigure}
  \end{figure}    
\end{example}

В повечето от горните примери може сравнително лесно да се съобрази, че построения автомат разпознава желания език.
При по-сложни задачи обаче, ще се наложи да дадем доказателство, като обикновено се прилага 
{\em метода на математическата индукция} върху дължината на думите.
Ще разгледаме няколко такива примера.

\begin{problem}
  Докажете, че езикът
  \[L = \{\alpha \in \{a,b\}^\star\ \mid\ \alpha\mbox{ не съдържа две поредни срещания на }a\}\]
  е автоматен.
\end{problem}
\begin{proof}
  \marginpar{Най-лесния начин да се сетим как да построим $\A$ е като първо построим автомат за езика, който разпознава тези думи, в които се съдържат две поредни срещания на $a$ и вземем неговото допълнение съгласно Твърдение \ref{pr:automata-complement}. По-късно ще разгледаме общ метод за строене на автомат по език.}
  Да разгледаме $\A = \FA$ с функция на преходите
  \begin{figure}[H]
    \begin{center}
      \begin{tikzpicture}[->,>=stealth,thick,node distance=55pt]
        \tikzstyle{every state}=[circle,minimum size=20pt,auto]
        
        \node[initial, accepting, state] (0) {$q_0$};
        \node[accepting, state]          (1) [right of=0]{$q_1$};
        \node[state]                     (2) [right of=1]{$q_2$};
        
        \path 
        (0) edge [loop above]   node [above] {$b$}   (0)
        (0) edge [bend left=15] node [above] {$a$}   (1)
        (1) edge [bend left=15] node [below] {$b$}   (0)
        (1) edge [bend left=15] node [above] {$a$}   (2)
        (2) edge [loop above]   node [above] {$a,b$} (2);
      \end{tikzpicture}
    \end{center}
 \end{figure}

 Ще докажем, че $L = \L(\A)$.
 Първо ще се концентрираме върху доказателството на $\L(\A) \subseteq L$.
 \marginpar{Озн. $\abs{\alpha}$ - дължината на думата $\alpha$}
 Ще докажем с индукция по дължината на думата $\alpha$, че:
 \begin{enumerate}[(1)]
 \item 
   ако $\delta^\star(q_0,\alpha) = q_0$, то
   $\alpha$ не съдържа две поредни срещания на $a$
   и ако $\abs{\alpha} > 0$, то $\alpha$ завършва на $b$;
 \item
   ако $\delta^\star(q_0,\alpha) = q_1$, то
   $\alpha$ не съдържа две поредни срещания на $a$
   и завършва на $a$.
 \end{enumerate}

 \begin{itemize}
 \item
   За $\abs{\alpha} = 0$, т.е. $\alpha = \varepsilon$, твърденията (1) и (2) са ясни (Защо?).
 \item
   Да приемем, че твърденията $(1)$ и $(2)$ са верни за произволни думи $\alpha$ с дължина $n$.
 \item
   Нека $\abs{\alpha} = n+1$, т.е. $\alpha = \beta x$, където $\abs{\beta} = n$ и $x \in \Sigma$.
   Ще докажем (1) и (2) за $\alpha$.
   \begin{itemize}[-]
   \item 
     Нека $\delta^\star(q_0,\beta x) = q_0 = \delta(\delta^\star(q_0,\beta),x)$.
     Според дефиницията на функцията $\delta$, $x = b$ и $\delta^\star(q_0,\beta) \in \{q_0,q_1\}$.
     Тогава по {\bf И.П.} за (1) и (2), $\beta$ не съдържа две поредни срещания на $a$.
     Тогава е очевидно, че $\beta x$ също не съдържа две поредни срещания на $a$.
   \item
     Нека $\delta^\star(q_0,\beta x) = q_1 = \delta(\delta^\star(q_0,\beta),x)$.
     Според дефиницията на $\delta$, $x = a$ и $\delta^\star(q_0,\beta) = q_0$.
     Тогава по {\bf И.П.} за (2), $\beta$ не съдържа две поредни срещания на $a$
     и завършва на $b$.
     Тогава е очевидно, че $\beta x$ също не съдържа две поредни срещания на $a$.
   \end{itemize}
 \end{itemize}
  
 Така доказахме с индукция по дължината на думата, че за всяка дума $\alpha$ са  изпълнени твърденията $(1)$ и $(2)$.
 По дефиниция, ако $\alpha \in \L(\A)$, то $\delta^\star(q_0,\alpha) \in F = \{q_0,q_1\}$ и от $(1)$ и $(2)$ следва, че и в двата случа
 $\alpha$ не съдържа две поредни срещания на буквата $a$, т.е. $\alpha \in L$.
 С други думи, доказахме, че 
 \[\L(\A) \subseteq L.\]

 Сега ще докажем другата посока, т.е. $L \subseteq \L(\A)$.
 Това означава да докажем, че
 \[(\forall \alpha \in \Sigma^\star)[\alpha \in L\ \Rightarrow\ \delta^\star(q_0,\alpha) \in F],\]
 \marginpar{Да напомним, че от съждителното смятане знаем, че $p \rightarrow q \equiv \neg q \rightarrow \neg  p$.}
 което е еквивалентно на
 \begin{equation}
   \label{eq:case2}
   (\forall \alpha \in \Sigma^\star)[\delta^\star(q_0,\alpha) \not\in F \ \Rightarrow\ \alpha\not\in L].
 \end{equation}
 Това е лесно да се съобрази.
 Щом $\delta^\star(q_0,\alpha) \not\in F$, то 
 $\delta^\star(q_0,\alpha) = q_2$ и думата $\alpha$ може да се представи по следния начин:
 \[\alpha = \beta a \gamma\ \&\ \delta^\star(q_0,\beta) = q_1.\]
 
 Използвайки свойство (2) от по-горе, понеже $\delta^\star(q_0,\beta) = q_1$, то
 $\beta$ не съдържа две поредни срещания на $a$, но завършва на $a$.
 Сега е очевидно, че $\beta a$ съдържа две поредни срещания на $a$ и 
 щом $\beta a$ е префикс на $\alpha$, то думата $\alpha \not\in L$.
 С това доказахме Свойство \ref{eq:case2}, а следователно и посоката $L\subseteq \L(\A)$.
\end{proof}


\begin{problem}
  Докажете, че езикът
  \[L = \{\omega \in \{a,b\}^\star \mid a \text{ се среща четен брой, докато $b$ нечетен брой пъти в }\omega\}\]
  е автоматен.
\end{problem}
\begin{hint}
  Разгледайте автомата $\A$:
\begin{framed}
  \begin{figure}[H]
    \begin{center}
      \begin{tikzpicture}[->,>=stealth,thick,node distance=55pt]
        \tikzstyle{every state}=[circle,minimum size=15pt,auto]
        
        \node[initial,state]      (00) {$q_{00}$};
        \node[state]              (10) [above right of=00]{$q_{10}$};
        \node[accepting, state]   (01) [below right of=00]{$q_{01}$};
        \node[state]              (11) [below right of=10]{$q_{11}$};
        
        \path 
        (00) edge  [bend left=15]  node [above]  {$a$} (10)
        (10) edge  [bend left=15]  node [below]  {$a$} (00)
        (01) edge  [bend right=15] node [above]  {$b$} (00)
        (00) edge  [bend right=15] node [below]  {$b$} (01)
        (10) edge  [bend left=15]  node [above]  {$b$} (11)
        (11) edge  [bend left=15]  node [below]  {$b$} (10)
        (01) edge  [bend right=15] node [below]  {$a$} (11)
        (11) edge  [bend right=15] node [above]  {$a$} (01);
      \end{tikzpicture}
    \end{center}
    \caption{$\L(\A) \stackrel{?}{=} L$}
  \end{figure}
\end{framed}
  Докажете с индукция по дължината на думата $\omega$, че:
  \marginpar{Озн. $N_a(\omega)$ - броят на срещанията на буквата $a$ в думата $\omega$}
  \begin{enumerate}[a)]
  \item 
    $\delta^\star(q_{00}, \omega) = q_{00} \implies (\exists k,n\in\Nat)[N_a(\omega) = 2k\ \&\ N_b(\omega) = 2n]$;
  \item 
    $\delta^\star(q_{00}, \omega) = q_{01} \implies (\exists k,n\in\Nat)[N_a(\omega) = 2k\ \&\ N_b(\omega) = 2n+1]$;
  \item 
    $\delta^\star(q_{00}, \omega) = q_{10} \implies (\exists k,n\in\Nat)[N_a(\omega) = 2k+1\ \&\ N_b(\omega) = 2n]$;
  \item 
    $\delta^\star(q_{00}, \omega) = q_{11} \implies (\exists k,n\in\Nat)[N_a(\omega) = 2k+1\ \&\ N_b(\omega) = 2n+1]$;
  \end{enumerate}
  Оттук направете извода, че за произволна дума $\omega$,
  \[(\forall i<2)(\forall j<2)[\ \delta^\star(q_{00},\omega) = q_{ij} \iff N_a(\omega) \equiv i \bmod 2\ \&\ N_b(\omega) \equiv j \bmod 2\ ].\]
\end{hint}


% \begin{problem}
%   Докажете, че езикът $L = \{\omega \in \{0,1\}^\star \mid |\omega| \equiv 1 \bmod 3\}$ е автоматен.
% \end{problem}
% \begin{hint}
%   Разгледайте автомата $\A$:
%   \begin{framed}
%     \begin{figure}[H]
%       \begin{center}
%         \begin{tikzpicture}[->,>=stealth,thick,node distance=55pt]
%           \tikzstyle{every state}=[circle,minimum size=15pt,auto]
          
%           \node[initial,state]    (0) {$q_0$};
%           \node[accepting, state] (1) [right of=0]{$q_1$};
%           \node[state]            (2) [right of=1]{$q_2$};
          
%           \path 
%           (0) edge  [bend left=15]  node [above]  {$0,1$} (1)
%           (1) edge  [bend left=15]  node [above]  {$0,1$} (2)
%           (2) edge  [bend left=30]  node [below]  {$0,1$} (0);
%         \end{tikzpicture}
%       \end{center}
%       \caption{$\L(\A) \stackrel{?}{=} \{\omega\in\{0,1\}^\star \mid |\omega| \equiv 1 \bmod 3\}$}
%     \end{figure}  
%   \end{framed}
%  Докажете с индукция по дължината на думата $\omega$ едновременно следните три твърдения:
%  \begin{enumerate}[(1)]
%  \item
%    $\delta^\star(q_0, \omega) = q_0 \implies |\omega| \equiv 0 \bmod 3$;
%  \item 
%    $\delta^\star(q_0, \omega) = q_1 \implies |\omega| \equiv 1 \bmod 3$;
%  \item
%    $\delta^\star(q_0, \omega) = q_2 \implies |\omega| \equiv 2 \bmod 3$.
%  \end{enumerate}
%  Оттук направете извода, че 
%  \[(\forall \omega \in \Sigma^\star)(\forall i < 3)[\ \delta^\star(q_0, \omega) = q_i\ \iff\ |\omega| \equiv i \bmod 3\ ].\]
% \end{hint}

За една дума $\alpha \in \{0,1\}^\star$, 
нека с $\ov{\alpha}_{(k)}$ да означим числото, което се представя в $k$-ична бройна система като $\alpha$.
Например, 
\[\ov{1101}_{(2)} = 1 \cdot 2^3+1\cdot 2^2+0\cdot 2^1+1\cdot 2^0 = \ov{13}_{(10)} = 1.10^1 + 3.10^0.\]
\marginpar{Да отбележим, че за всяко число $n$ има безкрайно много думи $\alpha$, за които $\ov{\alpha}_{(2)} = n$. Например, $\ov{10}_{(2)} = \ov{010}_{(2)} = \ov{0010}_{(2)} = \cdots$}
За $k = 2$, имаме следната дефиниция:
\begin{itemize}
\item
  $\ov{\varepsilon}_{(2)} = 0$,
\item
  $\ov{\alpha 0}_{(2)} = 2\cdot \ov{\alpha}_{(2)}$,
\item
  $\ov{\alpha 1}_{(2)} = 2\cdot \ov{\alpha}_{(2)} + 1$.
\end{itemize}


\begin{problem}
  Докажете, че езикът
  \[L = \{\alpha \in \{0,1\}^\star \mid \ov{\alpha}_{(2)} \equiv 2\ (\bmod\ 3)\}\]
  е автоматен.
\end{problem}
\begin{proof}
  Нашият автомат ще има три състояния $\{q_0,q_1,q_2\}$, като началното състояние ще бъде $q_0$.
  Целта ни е да дефинираме така автомата, че да имаме следното свойство:
  \begin{equation}
    (\forall\alpha\in\Sigma^\star)(\forall i < 3)[\ \ov{\alpha}_{(2)} \equiv i\ (\bmod\ 3)\ \Leftrightarrow\ \delta^\star(q_0,\alpha) = q_i\ ],
  \end{equation}
  т.е. всяко състояние отговаря на определен остатък при деление на три.
  Понеже искаме нашия автомат да разпознава тези думи $\alpha$,
  за които $\alpha_{(2)} \equiv 2\mod 3$, финалното състояние ще бъде $q_2$.
  Дефинираме функцията $\delta$ по следния начин:
  \begin{align*}
    & \delta(q_0,0) = q_0 & \comment{\ \ov{\alpha}_{(2)} \equiv 0 \bmod 3 \implies \ov{\alpha 0}_{(2)} \equiv 0 \bmod 3 }\\
    & \delta(q_0,1) = q_1 & \comment{\ \ov{\alpha}_{(2)} \equiv 0 \bmod 3 \implies \ov{\alpha 1}_{(2)} \equiv 1 \bmod 3 }\\
    & \delta(q_1,0) = q_2 & \comment{\ \ov{\alpha}_{(2)} \equiv 1 \bmod 3 \implies \ov{\alpha 0}_{(2)} \equiv 2 \bmod 3 }\\
    & \delta(q_1,1) = q_0 & \comment{\ \ov{\alpha}_{(2)} \equiv 1 \bmod 3 \implies \ov{\alpha 1}_{(2)} \equiv 0 \bmod 3 }\\
    & \delta(q_2,0) = q_1 & \comment{\ \ov{\alpha}_{(2)} \equiv 2 \bmod 3 \implies \ov{\alpha 0}_{(2)} \equiv 1 \bmod 3 }\\
    & \delta(q_2,1) = q_2 & \comment{\ \ov{\alpha}_{(2)} \equiv 2 \bmod 3 \implies \ov{\alpha 1}_{(2)} \equiv 2 \bmod 3 }
  \end{align*}
  Ето и картинка на автомата $\A$:
  \begin{framed}
  \begin{figure}[H]
    \begin{center}
      \begin{tikzpicture}[->,>=stealth,thick,node distance=55pt]
        \tikzstyle{every state}=[circle,minimum size=15pt,auto]
        
        \node[initial,state]      (0) {$q_0$};
        \node[state]              (1) [right of=0]{$q_1$};
        \node[accepting, state]   (2) [right of=1]{$q_2$};
        
        \path 
        (0) edge  [loop above]    node [above]  {$0$} (0)
        (0) edge  [bend left=15]  node [above]  {$1$} (1)
        (2) edge  [bend left=15]  node [below]  {$0$} (1)
        (1) edge  [bend left=15]  node [below]  {$1$} (0)
        (1) edge  [bend left=15]  node [above]  {$0$} (2)
        (2) edge  [loop above]    node [above]  {$1$} (2);
      \end{tikzpicture}
      \end{center}
      \caption{$\L(\A) \stackrel{?}{=} \{\alpha\in\{0,1\}^\star \mid \ov{\alpha}_{(2)} \equiv 2\ (\bmod\ 3)\}$}
 \end{figure}
 \end{framed}
 \noindent Да разгледаме твърденията:
 \begin{enumerate}[(1)]
  \item 
    $\delta^\star(q_0,\alpha) = q_0 \implies \ov{\alpha}_{(2)} \equiv 0 \mod 3$;
  \item 
    $\delta^\star(q_0,\alpha) = q_1 \implies \ov{\alpha}_{(2)} \equiv 1 \mod 3$;
  \item 
    $\delta^\star(q_0,\alpha) = q_2 \implies \ov{\alpha}_{(2)} \equiv 2 \mod 3$.
  \end{enumerate}
  \marginpar{Обърнете внимание, че в доказателството на (3) използваме И.П. не само за (3), но и за (2). Поради тази причина трябва да докажем трите свойства едновременно}
  Ще докажем (1), (2) и (3) {\em едновременно} с индукция по дължината на думата $\alpha$.
  За $\abs{\alpha} = 0$, всички условия са изпълнени. (Защо?)
  Да приемем, че (1), (2) и (3) са изпълнени за думи с дължина $n$.
  Нека $\abs{\alpha} = n+1$, т.е. $\alpha = \beta x$, $\abs{\beta} = n$.

  Ще докажем подробно само (3) понеже другите твърдения се доказват по сходен начин.
  Нека $\delta^\star(q_0,\beta x) = q_2$. 
  Имаме два случая:
  \begin{itemize}
  \item 
    $x = 0$. 
    Тогава, по дефиницията на $\delta$, 
    $\delta(q_1,0) = q_2$ и следователно, $\delta^\star(q_0,\beta) = q_1$.
    По {\bf И.П.} за (2) с $\beta$,
    \[\delta^\star(q_0,\beta) = q_1\ \Rightarrow\ \ov{\beta}_{(2)} \equiv 1 \bmod 3\]
    Тогава, $\ov{\beta0}_{(2)} \equiv 2 \mod 3$. Така доказахме, че
    \[\delta^\star(q_0,\beta 0) = q_2\ \Rightarrow\ \ov{\beta 0}_{(2)} \equiv 2 \bmod 3.\]
  \item
    $x = 1$.
    Тогава, по дефиницията на $\delta$, $\delta(q_2,1) = q_2$ и следователно,
    $\delta^\star(q_0,\beta) = q_2$.
    По {\bf И.П.} за (3) с $\beta$,
    \[\delta^\star(q_0,\beta) = q_2\ \Rightarrow\ \ov{\beta}_{(2)} \equiv 2 \bmod 3.\]
    Тогава, $\ov{\beta1}_{(2)} \equiv 2 \mod 3$. Така доказахме, че
    \[\delta^\star(q_0,\beta 1) = q_2\ \Rightarrow\ \ov{\beta 1}_{(2)} \equiv 2 \bmod 3.\]
  \end{itemize}
  \marginpar{\writedown Довършете доказателствата на (1) и (2)}
  За да докажем (1), нека $\delta^\star(q_0,\beta x) = q_0$. 
  \begin{itemize}
  \item 
    $x = 0$. Разсъжденията са аналогични, като използваме {\bf И.П.} за (1).
  \item
    $x = 1$. Разсъжденията са аналогични, като използваме {\bf И.П.} за (2).
  \end{itemize}
  
  По същия начин доказваме и (2). Нека $\delta^\star(q_0,\beta x) = q_1$. 
  \begin{itemize}
  \item 
    При $x = 0$, използваме {\bf И.П.} за (3).
  \item
    При $x = 1$, използваме {\bf И.П.} за (1).
  \end{itemize}
  \marginpar{\writedown Защо?}
  От (1), (2) и (3) следва директно, че
  \[(\forall\alpha\in\Sigma^\star)(\forall i < 3)[\ \ov{\alpha}_{(2)} \equiv i\ (\bmod\ 3)\ \Leftrightarrow\ \delta^\star(q_0,\alpha) = q_i\ ],\]
  откъдето получаваме, че $\L(\A) = L$.
\end{proof}


% Можем да дадем и следната дефиниция на $\ov{\alpha}_{(2)}$:
% \begin{itemize}
% \item
%   $\ov{\varepsilon}_{(2)} = 0$,
% \item
%   $\ov{0\alpha}_{(2)} = \ov{\alpha}_{(2)}$,
% \item
%   $\ov{1\alpha}_{(2)} = 2^{|\alpha|} + \ov{\alpha}_{(2)}$.
% \end{itemize}

% Тогава:
% \begin{align*}
%   0^{-1}(L) & = \{\alpha \in \{0,1\}^\star \mid 0\alpha \in L\}\\
%             & = \{\alpha \in \{0,1\}^\star \mid \ov{0\alpha}_{(2)} \equiv 2 \bmod 3\}\\
%             & = \{\alpha \in \{0,1\}^\star \mid \ov{\alpha}_{(2)} \equiv 2 \bmod 3\}\\
%             & = L\\
%   \\
%   1^{-1}(L) & = \{\alpha \in \{0,1\}^\star \mid 1\alpha \in L\}\\
%             & = \{\alpha \in \{0,1\}^\star \mid \ov{1\alpha}_{(2)} \equiv 2 \bmod 3\}\\
%             & = \{\alpha \in \{0,1\}^\star \mid 2^{|\alpha|} + \ov{\alpha}_{(2)} \equiv 2 \bmod 3\}\\
%             & \df L_1.
% \end{align*}

% Лесно се съобразява, че $L \neq L_1$, защото например за думата $\alpha = 10$
% имаме, че $\alpha \in L$, но $\alpha \not\in L_1$.
% Продължаваме нататък:
% \begin{align*}
%   0^{-1}(L_1) & = \{\alpha \in \{0,1\}^\star \mid 0\alpha \in L_1\}\\
%               & = \{\alpha \in \{0,1\}^\star \mid 2^{|0\alpha|} + \ov{0\alpha}_{(2)} \equiv 2 \bmod 3\}\\
%               & = \{\alpha \in \{0,1\}^\star \mid 2\cdot 2^{|\alpha|} + \ov{\alpha}_{(2)} \equiv 2 \bmod 3\}\\
%               & \df L_2\\
%   \\
%   1^{-1}(L_1) & = \{\alpha \in \{0,1\}^\star \mid 1\alpha \in L_1\}\\
%             & = \{\alpha \in \{0,1\}^\star \mid 2^{|1\alpha|} + \ov{1\alpha}_{(2)} \equiv 2 \bmod 3\}\\
%             & = \{\alpha \in \{0,1\}^\star \mid 2\cdot 2^{|\alpha|} + 2^{|\alpha|} + \ov{\alpha}_{(2)} \equiv 2 \bmod 3\}\\
%               & = \{\alpha \in \{0,1\}^\star \mid 3\cdot 2^{|\alpha|} + \ov{\alpha}_{(2)} \equiv 2 \bmod 3\}\\
%               & = \{\alpha \in \{0,1\}^\star \mid \ov{\alpha}_{(2)} \equiv 2 \bmod 3\}\\
%               & = L.
% \end{align*}
% Да проверим, че $L_2 \neq L$ и $L_2 \neq L_1$.
% Това отново е лесно. Нека например да разгледаме $\alpha = 11$.
% Непосредствено се проверява, че $\alpha \in L_2$, $\alpha \not\in L$, $\alpha \not\in L_1$.
% Продължаваме нататък:
% \begin{align*}
%   0^{-1}(L_2) & = \{\alpha \in \{0,1\}^\star \mid 0\alpha \in L_2\}\\
%               & = \{\alpha \in \{0,1\}^\star \mid 2\cdot 2^{|0\alpha|} + \ov{0\alpha}_{(2)} \equiv 2 \bmod 3\}\\
%               & = \{\alpha \in \{0,1\}^\star \mid 4\cdot 2^{|\alpha|} + \ov{\alpha}_{(2)} \equiv 2 \bmod 3\}\\
%               & = \{\alpha \in \{0,1\}^\star \mid 3\cdot 2^{|\alpha|} + 2^{|\alpha|} + \ov{\alpha}_{(2)} \equiv 2 \bmod 3\}\\
%               & = \{\alpha \in \{0,1\}^\star \mid 2^{|\alpha|} + \ov{\alpha}_{(2)} \equiv 2 \bmod 3\}\\
%               & = L_1
%   \\
%   1^{-1}(L_2) & = \{\alpha \in \{0,1\}^\star \mid 1\alpha \in L_2\}\\
%               & = \{\alpha \in \{0,1\}^\star \mid 2\cdot 2^{|1\alpha|} + \ov{1\alpha}_{(2)} \equiv 2 \bmod 3\}\\
%               & = \{\alpha \in \{0,1\}^\star \mid 4\cdot 2^{|\alpha|} + 2^{|\alpha|} + \ov{\alpha}_{(2)} \equiv 2 \bmod 3\}\\
%               & = \{\alpha \in \{0,1\}^\star \mid 3\cdot 2^{|\alpha|} + 2\cdot 2^{|\alpha|} + \ov{\alpha}_{(2)} \equiv 2 \bmod 3\}\\
%               & = \{\alpha \in \{0,1\}^\star \mid 2\cdot 2^{|\alpha|} + \ov{\alpha}_{(2)} \equiv 2 \bmod 3\}\\
%               & = L_2.
% \end{align*}

% Така можем да получим автомат за езика $L$, където всяко състояние е свързано с език.

% \begin{framed}
%   \begin{figure}[H]
%     \begin{center}
%       \begin{tikzpicture}[->,>=stealth,thick,node distance=55pt]
%         \tikzstyle{every state}=[circle,minimum size=15pt,auto]
        
%         \node[initial,state]      (0) {$L$};
%         \node[state]              (1) [right of=0]{$L_1$};
%         \node[accepting, state]   (2) [right of=1]{$L_2$};
        
%         \path 
%         (0) edge  [loop above]    node [above]  {$0$} (0)
%         (0) edge  [bend left=15]  node [above]  {$1$} (1)
%         (2) edge  [bend left=15]  node [below]  {$0$} (1)
%         (1) edge  [bend left=15]  node [below]  {$1$} (0)
%         (1) edge  [bend left=15]  node [above]  {$0$} (2)
%         (2) edge  [loop above]    node [above]  {$1$} (2);
%       \end{tikzpicture}
%     \end{center}
%     \caption{$\L(\A) = \{\alpha\in\{0,1\}^\star \mid \ov{\alpha}_{(2)} \equiv 2\ (\bmod\ 3)\}$}
%   \end{figure}
% \end{framed}

\subsection*{Затвореност на автоматните езици относно сечиение, обединение, разлика}

\begin{prop}
  \index{автоматни езици!сечение}
  \label{pr:automata-cap}
  Класът на автоматните езици е затворен относно операцията {\em сечение}.
  Това означава, че ако $L_1$ и $L_2$ са два произволни автоматни езика над азбуката $\Sigma$, то $L_1\cap L_2$
  също е автоматен език.
\end{prop}
\begin{proof}
  Да разгледаме два автомата \[\A' = (\Sigma,Q',\qstart',\delta',F'),\ \A'' = (\Sigma, Q'', \qstart'', \delta'', F'').\]
  Определяме автомата $\A = \FA$, по следния начин:
  \marginpar{Изчислението на автомата $\A$ върху думата $\alpha$ едновременно симулира изчислението на $\A'$ и $\A''$ върху $\alpha$.}
  \begin{itemize}
  \item
    $Q \df Q'\times Q''$;
  \item
    За всяко $\pair{r_1,r_2} \in Q$ и всяко $a \in \Sigma$,
    \[\delta(\pair{r_1,r_2},a) \df \pair{\delta_1(r_1,a),\delta_2(r_2,a)};\]
  \item
    $\qstart \df \pair{\qstart',\qstart''}$;
  \item
    $F \df \{\pair{r_1,r_2}\mid r_1\in F''\ \&\ r_2 \in F''\} = F' \times F''$
  \end{itemize}
  Ще докажем с индукция по дължината на думата $\alpha$, че за всяка дума $\alpha$:
  \begin{equation}
    \label{eq:7}
    (\forall p\in Q')(\forall q\in Q'')[\ \delta^\star(\pair{p,q},\alpha) = \pair{ \delta^\star_1(p,\alpha), \delta^\star_2(q,\alpha) }\ ].
  \end{equation}
  \begin{itemize}
  \item
    Нека $|\alpha| = 0$, т.е. $\alpha = \varepsilon$. Тогава всичко е ясно, защото
    \begin{align*}
      \delta^\star( \pair{p,q}, \varepsilon) & \df \pair{p,q}\\
                                             & = \pair{\delta^\star_1(p,\varepsilon), \delta^\star_2(q,\varepsilon) }.
    \end{align*}
  \item
    Да приемем, че (\ref{eq:7}) е изпълнено за думи $\alpha$ с дължина $n$.
  \item
    Нека $|\alpha| = n+1$, т.е. $\alpha = \beta a$ и $|\beta| = n$. Тогава:
    \begin{align*}
      \delta^\star(\pair{p, q},\beta a) & = \delta( \delta^\star( \pair{p, q}, \beta ), a) & \comment{\text{деф. на }\delta^\star}\\
                                        & = \delta( \pair{ \delta^\star_1(p, \beta), \delta^\star_2(q, \beta)}, a ) & \comment{\text{ от И.П.}}\\
                                        & = \pair{ \delta_1( \delta^\star_1(p, \beta), a), \delta_2(\delta^\star_2(q,\beta), a)} & \comment\text{деф. на }\delta\\
                                        & = \pair{ \delta^\star_1(p, \beta a), \delta^\star_2(q, \beta a) } & \comment{\text{деф. на $\delta^\star_1$ и $\delta^\star_2$}}.
    \end{align*}
  \end{itemize}
  Използвайки Свойство (\ref{eq:7}) лесно можем да докажем, че
  \[\L(\A) = \L(\A') \cap \L(\A'').\]
  Имаме следните еквивалентности:
  \begin{align*}
    \omega \in \L(\A) & \iff \delta^\star(\qstart, \omega) \in F \\
                      & \iff \delta^\star(\pair{\qstart',\qstart''}, \omega) \in F' \times F'' & \comment\text{деф. на }\A\\
                      & \iff \pair{\delta^\star_1(\qstart',\omega), \delta^\star_2(\qstart'',\omega)} \in F' \times F'' & \comment{\text{от (\ref{eq:7})}}\\
                      & \iff \delta^\star_1(\qstart',\omega) \in F'\ \&\ \delta^\star_2(\qstart'',\omega) \in F''\\
                      & \iff \omega \in \L(\A')\ \&\ \omega\in\L(\A'') \\
                      & \iff \omega \in \L(\A') \cap \L(\A'').
  \end{align*}
  
\end{proof}

\begin{prop}
  \index{автоматни езици!допълнение}
  \label{pr:automata-complement}
  \marginpar{Това означава, че автоматните езици са затворени относно операцията допълнение.}
  Нека $L$ е автоматен език.
  Тогава $\Sigma^\star\setminus L$ също е автоматен език.
\end{prop}
\begin{hint}
  \marginpar{\writedown Проверете, че $\Sigma^\star\setminus L = \L(\A')$}
  Нека $L = L(\A)$, където $\A = \FA$.
  Да вземем автомата
  \[\A' = \pair{Q,\Sigma,\qstart,\delta,Q\setminus F},\]
  т.е. $\A'$ е същия като $\A$, с единствената разлика, че финалните състояния на $\A'$
  са тези състояния, които {\bf не} са финални в $\A$.
\end{hint}

\begin{prop}
  \index{автоматни езици!обединение}
  \label{pr:automata-union}
  Класът на автоматните езици е затворен относно операцията {\bf обединение}.
  Това означава, че ако $L_1$ и $L_2$ са два произволни автоматни езика над азбуката $\Sigma$, то $L_1\cup L_2$
  също е автоматен език.
\end{prop}
\begin{hint}
  \marginpar{\ding{45} Докажете, че така построения автомат $\A$ разпознава $L_1\cup L_2$!}
  Първият подход е да използвайте конструкцията на автомата $\A$ от \Prop{automata-cap},
  с единствената разлика, че тук избираме финалните състояния да бъдат елементите на множеството
  \[F \df \{\pair{q',q''} \in Q' \times Q'' \mid q' \in F'\ \lor\ q'' \in F''\} = F'\times Q'' \cup Q'\times F''.\]
  Друг подход е да се използва правилото на Де Морган, а именно:
  \[L_1 \cup L_2 = \ov{\ov{L_1} \cap \ov{L_2}}.\]
\end{hint}


%%% Local Variables:
%%% mode: latex
%%% TeX-master: "../eai"
%%% End:
