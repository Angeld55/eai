\subsection{Примери}

\begin{example}
  Нека да разгледаме регулярния език
  \[L = \L(\mathbf{((a+b)^+\cdot a)^\star}).\]
  Да видим как ще построим автомата $\B$ за $L$.
  Едновременно ще строим състоянията на автомата $Q$ и функцията на преходите $\delta$.
\begin{itemize}
\item
  Ясно е, че ще започнем от началното състояние $q_L$.
  Удобно е да имаме предвид следното представяне
  \begin{align*}
    L & = \{\varepsilon\} \cup \{a,b\}^+ a L\\
      & = \{\varepsilon\} \cup \{a,b\} \cdot \{a,b\}^\star a L\\
      & = \{\varepsilon\} \cup a\{a,b\}^\star aL \cup b\{a,b\}^\star aL
  \end{align*}
\item
  Сега като имаме това представяне на $L$, лесно се съобразява, че
  \[a^{-1}(L) = b^{-1}(L) = \{a,b\}^\star aL.\]
  Нека положим $M \df \{a,b\}^\star aL$.
  Лесно се съобразява, че $M \neq L$, защото $\varepsilon \in L$, но $\varepsilon \not\in M$.
  Това означава, че имаме ново състояние $q_M$ и
  \begin{align*}
    & \delta(q_L,a) \df q_M & \comment\text{ защото }a^{-1}(L) = M\\
    & \delta(q_L,b) \df q_M. & \comment\text{ защото }b^{-1}(L) = M
  \end{align*}
\item
  Удобно е да представим езика $M$ по следния начин:
  \begin{align*}
    M & = \{a,b\}^\star aL\\
      & = aL \cup \{a,b\}^+aL\\
      & = aL \cup \{a,b\}\cdot \{a,b\}^\star aL\\
      & = aL \cup \{a,b\}\cdot M\\
      & = aL \cup aM \cup bM
  \end{align*}
  От това представяне на $M$ веднага се съобразява, че
  \[a^{-1}(M) = L \cup M.\]
  Нека да положим $N \df L \cup M$.
  Имаме, че $N \neq L$, защото $a\in N$, но $a \not\in L$.
  Освен това, $N \neq M$, защото $\varepsilon \in N$, но $\varepsilon \not\in M$.
  Това означава, че имаме ново състояние $q_N$ и тогава
  \begin{align*}
    & \delta(q_M,a) \df q_N & \comment\text{ защото }a^{-1}(M) = N\\
    & \delta(q_M,b) \df q_N & \comment\text{ защото }b^{-1}(M) = M.
  \end{align*}
\item
  Да разгледаме следното разлагане
  \begin{align*}
    N & = L \cup M \\
      & = \{\varepsilon\} \cup aM \cup bM \cup M\\
      & = \{\varepsilon\} \cup aM \cup bM \cup aL\\
      & = \{\varepsilon\} \cup aN \cup bM.
  \end{align*}
  Сега полагаме:
  \begin{align*}
    & \delta(q_{N},a) \df q_{N} & \comment\text{ защото }a^{-1}(N) = N\\
    & \delta(q_{N},b) \df q_{M} & \comment\text{ защото }b^{-1}(N) = M.
  \end{align*}
\item
  Нямаме повече нови състояния. Следователно,
  \[Q \df \{q_L, q_M,q_{N}\}.\]
\item
  Понеже $\varepsilon \in L$ и $\varepsilon \in N$ е ясно, че
  \[F \df \{q_L,q_N\}.\]
\end{itemize}

Сега вече сме готови да нарисуваме картинка на автомата.

% \begin{framed}
\begin{figure}[H]
  \begin{subfigure}[b]{.45\textwidth}
    \centering
    \begin{tikzpicture}[framed,->,>=stealth,thick,node distance=55pt]
      \tikzstyle{every state}=[circle,minimum size=20pt,font=\small]
      \node[initial above, state,accepting]   (L) {$q_L$};
      \node[state]                            (M) [right of=L]{$q_M$};
      \node[state,accepting]                  (N) [right of=M]{$q_{N}$};
      
      \path 
      (L) edge [bend left=15] node [above] {$a,b$} (M)
      (M) edge [loop above] node [above]   {$b$}   (M)
      (M) edge [bend left=15] node [above] {$a$}   (N)
      (N) edge [bend left=15] node [below] {$b$}   (M)
      (N) edge [loop above] node [above]   {$a$}   (N);
    \end{tikzpicture}
    \caption{Автомата $\B$ за езика $L$ по метода на Бжозовски.}
  \end{subfigure}
  \quad
  ~
  \quad
  \begin{subfigure}[b]{.45\textwidth}
    \centering
    \begin{tikzpicture}[framed,->,>=stealth,thick,node distance=55pt]
      \tikzstyle{every state}=[circle,minimum size=20pt,auto]
      
      \node[initial above, state,accepting]   (L) {$[\varepsilon]_L$};
      \node[state]                            (M) [right of=L]{$[a]_L$};
      \node[state,accepting]                  (N) [right of=M]{$[aa]_L$};
      
      \path 
      (L) edge [bend left=15] node [above] {$a,b$} (M)
      (M) edge [loop above] node [above]   {$b$}   (M)
      (M) edge [bend left=15] node [above] {$a$}   (N)
      (N) edge [bend left=15] node [below] {$b$}   (M)
      (N) edge [loop above] node [above]   {$a$}   (N);
    \end{tikzpicture}
    \caption{Минимален автомат $\M$ за езика $L$ по метода на Майхил-Нероуд.}
  \end{subfigure}
\end{figure}
\end{example}

\begin{example}
  Да разгледаме езика регулярния език
  \[L = \L(\mathbf{a\cdot(a+b)^\star\cdot b}).\]
  Ще построим автомат $\B$ по метода на Бжозовски, който рапознава езика $L$.
  \begin{itemize}
  \item 
    $a^{-1}(L) = \{a,b\}^\star b \df M$.
    Имаме, че $M \neq L$, защото $b \in M$, но $b \not\in L$.
    Тогава
    \begin{align*}
      & \delta(q_L, a) \df q_M & \comment\text{ защото }a^{-1}(L) = M\\
      & \delta(q_L,b) \df q_\emptyset & \comment\text{ защото }b^{-1}(L) = \emptyset\\
    \end{align*}
  \item    
    За по-нататък ще е удобно да представим $M$ по следния начин:
    \begin{align*}
      M & = a\cdot \{a,b\}^\star \cdot b \cup b\cdot \{a,b\}^\star \cdot b \cup \{b\}\\
        & = aM \cup bM \cup \{b\}.
    \end{align*}
    Сега е ясно, че $a^{-1}(M) = M$, а $b^{-1}(M) = \{\varepsilon\} \cup M$.
    Нека да положим $N \df \{\varepsilon\} \cup M$.
    Имаме, че $N \neq L$ и $N \neq M$, защото $\varepsilon \in N$, но $\varepsilon \not\in L$ и $\varepsilon \not\in M$.
    Тогава
    \begin{align*}
      & \delta(q_M,a) \df q_M & \comment\text{ защото }a^{-1}(M) = M\\
      & \delta(q_M,b) \df q_N & \comment\text{ защото }b^{-1}(M) = N
    \end{align*}
  \item
    Можем да представим езика $N$ по следния начин:
    \[N = \{\varepsilon\} \cup aM \cup bM \cup \{b\}.\]
    Тогава имаме, че:
    \begin{align*}
      & \delta(q_N,a) \df q_M & \comment\text{ защото } a^{-1}(N) = M\\
      & \delta(q_N,b) \df q_N & \comment\text{ защото } b^{-1}(N) = M.
    \end{align*}
  \item
    Завършваме с дефиницията на функцията на преходите като:
    \begin{align*}
      & \delta(q_\emptyset,a) \df q_\emptyset & \comment\text{ защото }a^{-1}(\emptyset) = \emptyset\\
      & \delta(q_\emptyset,b) \df q_\emptyset & \comment\text{ защото }b^{-1}(\emptyset) = \emptyset.
    \end{align*}
  \item
    Съобразете сами, че $F = \{q_N\}$.
  \end{itemize}
  \begin{figure}[H]
    \centering
    \begin{tikzpicture}[framed,->,>=stealth,thick,node distance=55pt]
      \tikzstyle{every state}=[circle]
      
      \node[initial, state]                   (L) {$q_L$};
      \node[state]                            (M) [above right of=L]{$q_M$};
      \node[state]                            (E) [below right of=L]{$q_\emptyset$};
      \node[state,accepting]                  (N) [right of=M]{$q_N$};
      
      \path 
      (L) edge [bend left=15]  node [left] {$a$} (M)
      (L) edge [bend right=15] node [left] {$b$} (E)
      (E) edge [loop right]    node [right] {$a,b$} (E) 
      (M) edge [bend right=15] node [below] {$b$} (N)
      (M) edge [loop above]    node [above] {$a$} (M)
      (N) edge [bend right=30] node [above] {$a$} (M)
      (N) edge [loop above]    node [above] {$b$} (N);
    \end{tikzpicture}
    \caption{Автомат за езика $\L(\mathbf{a\cdot (a+b)^\star\cdot b})$ по метода на Бжозовски.}
  \end{figure}
\end{example}

\begin{example}
  Да разгледаме езика 
  \[L = \{\omega \in \{a,b\}^\star \mid \omega \text{ съдържа четен брой $a$ и точно едно $b$}\}.\]
  Нека да видим дали можем да построим автомат за този език.
  \begin{itemize}
  \item 
    $a^{-1}(L) = \{\alpha \in \{a,b\}^\star \mid \alpha \text{ съдържа нечетен брой $a$ и точно едно $b$}\} \df M$;
  \item 
    $b^{-1}(L) = \{\alpha \in \{a,b\}^\star \mid \alpha \text{ съдържа четен брой $a$ и нито едно $b$}\} \df N$;
  \item
    $a^{-1}(M) = L$;
  \item
    $b^{-1}(M) = \{\alpha \in \{a,b\}^\star \mid \alpha \text{ съдържа нечетен брой $a$ и нито едно $b$}\} \df P$;
  \item
    $a^{-1}(N) = P$;
  \item
    $b^{-1}(N) = \emptyset$;
  \item
    $a^{-1}(P) = N$;
  \item
    $b^{-1}(P) = \emptyset$;
  \end{itemize}

  \begin{figure}[H]
    \centering
    \begin{tikzpicture}[framed,->,>=stealth,thick,node distance=65pt]
      \tikzstyle{every state}=[circle]
      
      \node[initial, state]        (L) {$q_L$};
      \node[state]                 (M) [above of=L]{$q_M$};
      \node[state]                 (P) [right of=M]{$q_P$};
      \node[state, accepting]      (N) [below of=P]{$q_N$};
      \node[state]                 (E) [above right of=N]{$q_\emptyset$};
            
      \path 
      (L) edge [bend right=15]  node [right] {$a$} (M)
      (M) edge [bend right=15]  node [left] {$a$} (L)
      (L) edge [bend right=15]  node [below] {$b$} (N)
      (M) edge [bend left=15]   node [above] {$b$} (P)
      (N) edge [bend left=15]   node [left] {$a$} (P)
      (P) edge [bend left=15]   node [right] {$a$} (N)
      (P) edge [bend left=15]   node [above] {$b$} (E)
      (N) edge [bend right=15]  node [below] {$b$} (E)
      (E) edge [loop right]    node [right] {$a,b$} (E);      
    \end{tikzpicture}
    \caption{Минимален автомат, който приема думи с четен брой $a$ и точно едно $b$}
  \end{figure}  
\end{example}


\begin{example}
  Да припомним, че в \Problem{regular:dfa:binary} се искаше да се докаже, че езикът 
  \[L = \{\alpha \in \{0,1\}^\star \mid \bin{\alpha} \equiv 2 \bmod 3\}\]
  е регулярен.
  Ние направихме това като построихме автомат за $L$ и доказахме, че той разпознава $L$.
  Нека сега пак да построим автомат за $L$, но този път чрез метода на Бжозовски.
  За целта ще ни трябва алтернативна дефиниция на $\bin{\alpha}$.
  За една дума $\alpha \in \{0,1\}^\star$, можем да дадем следната дефиниция на $\bin{\alpha}$:
  \begin{itemize}
  \item
    $\bin\varepsilon = 0$,
  \item
    $\bin{0\alpha} = \bin{\alpha}$,
  \item
    $\bin{1\alpha} = 2^{|\alpha|} + \bin{\alpha}$.
  \end{itemize}
  Тогава имаме, че $0^{-1}(L) = L$ и $1^{-1}(L) \df M$, защото:
  {\scriptsize
  \begin{multicols}{2}
    \begin{align*}
      0^{-1}(L) & = \{\alpha \in \{0,1\}^\star \mid 0\alpha \in L\}\\
                & = \{\alpha \in \{0,1\}^\star \mid \bin{0\alpha} \equiv 2 \bmod 3\}\\
                & = \{\alpha \in \{0,1\}^\star \mid \bin{\alpha} \equiv 2 \bmod 3\}\\
                & = L.
    \end{align*}    

    \begin{align*}
      1^{-1}(L) & = \{\alpha \in \{0,1\}^\star \mid 1\alpha \in L\}\\
                & = \{\alpha \in \{0,1\}^\star \mid \bin{1\alpha} \equiv 2 \bmod 3\}\\
                & = \{\alpha \in \{0,1\}^\star \mid 2^{|\alpha|} + \bin{\alpha} \equiv 2 \bmod 3\}\\
                & \df M.
    \end{align*}
  \end{multicols}
}
Лесно се съобразява, че $L \neq M$, защото например за думата $\alpha = 10$
имаме, че $\alpha \in L$, но $\alpha \not\in M$.
Продължаваме нататък:
{\scriptsize
\begin{multicols}{2}
\begin{align*}
  0^{-1}(M) & = \{\alpha \in \{0,1\}^\star \mid 0\alpha \in M\}\\
              & = \{\alpha \in \{0,1\}^\star \mid 2^{|0\alpha|} + \bin{0\alpha} \equiv 2 \bmod 3\}\\
              & = \{\alpha \in \{0,1\}^\star \mid 2\cdot 2^{|\alpha|} + \bin{\alpha} \equiv 2 \bmod 3\}\\
              & \df N.
\end{align*}
\begin{align*}
  1^{-1}(M) & = \{\alpha \in \{0,1\}^\star \mid 1\alpha \in M\}\\
            & = \{\alpha \in \{0,1\}^\star \mid 2^{|1\alpha|} + \bin{1\alpha} \equiv 2 \bmod 3\}\\
            & = \{\alpha \in \{0,1\}^\star \mid 2\cdot 2^{|\alpha|} + 2^{|\alpha|} + \bin{\alpha} \equiv 2 \bmod 3\}\\
            & = \{\alpha \in \{0,1\}^\star \mid 3\cdot 2^{|\alpha|} + \bin{\alpha} \equiv 2 \bmod 3\}\\
            & = \{\alpha \in \{0,1\}^\star \mid \bin{\alpha} \equiv 2 \bmod 3\}\\
            & = L.
\end{align*}
\end{multicols}
}
Да проверим, че $N \neq L$ и $N \neq M$.
Това отново е лесно. Нека например да разгледаме $\alpha = 11$.
Непосредствено се проверява, че $\alpha \in N$, $\alpha \not\in L$, $\alpha \not\in M$.
Продължаваме нататък:
{\scriptsize
\begin{multicols}{2}
\begin{align*}
  0^{-1}(N) & = \{\alpha \in \{0,1\}^\star \mid 0\alpha \in N\}\\
            & = \{\alpha \in \{0,1\}^\star \mid 2\cdot 2^{|0\alpha|} + \bin{0\alpha} \equiv 2 \bmod 3\}\\
            & = \{\alpha \in \{0,1\}^\star \mid 4\cdot 2^{|\alpha|} + \bin{\alpha} \equiv 2 \bmod 3\}\\
            & = \{\alpha \in \{0,1\}^\star \mid 3\cdot 2^{|\alpha|} + 2^{|\alpha|} + \bin{\alpha} \equiv 2 \bmod 3\}\\
            & = \{\alpha \in \{0,1\}^\star \mid 2^{|\alpha|} + \bin{\alpha} \equiv 2 \bmod 3\}\\
            & = M.
\end{align*}
\begin{align*}
  1^{-1}(N) & = \{\alpha \in \{0,1\}^\star \mid 1\alpha \in N\}\\
            & = \{\alpha \in \{0,1\}^\star \mid 2\cdot 2^{|1\alpha|} + \bin{1\alpha} \equiv 2 \bmod 3\}\\
            & = \{\alpha \in \{0,1\}^\star \mid 4\cdot 2^{|\alpha|} + 2^{|\alpha|} + \bin{\alpha} \equiv 2 \bmod 3\}\\
            & = \{\alpha \in \{0,1\}^\star \mid 3\cdot 2^{|\alpha|} + 2\cdot 2^{|\alpha|} + \bin{\alpha} \equiv 2 \bmod 3\}\\
            & = \{\alpha \in \{0,1\}^\star \mid 2\cdot 2^{|\alpha|} + \bin{\alpha} \equiv 2 \bmod 3\}\\
            & = N.
\end{align*}
\end{multicols}
}
% Така можем да получим автомат за езика $L$, където всяко състояние е свързано с език.

\begin{figure}[H]
  \begin{center}
    \begin{tikzpicture}[framed,->,>=stealth,thick,node distance=55pt]
      \tikzstyle{every state}=[circle]
      
      \node[initial,state]      (0) {$q_L$};
      \node[state]              (1) [right of=0]{$q_M$};
      \node[accepting, state]   (2) [right of=1]{$q_N$};
      
      \path 
      (0) edge  [loop above]    node [above]  {$0$} (0)
      (0) edge  [bend left=15]  node [above]  {$1$} (1)
      (2) edge  [bend left=15]  node [below]  {$0$} (1)
      (1) edge  [bend left=15]  node [below]  {$1$} (0)
      (1) edge  [bend left=15]  node [above]  {$0$} (2)
      (2) edge  [loop above]    node [above]  {$1$} (2);
    \end{tikzpicture}
  \end{center}
  \caption{$\L(\A) = \{\alpha\in\{0,1\}^\star \mid \ov{\alpha}_{(2)} \equiv 2\ (\bmod\ 3)\}$}
\end{figure}
\end{example}

\begin{example}\label{ex:regular:brzozowski:an-bn}
  Да разгледаме езика $L = \{a^nb^n\mid n \in \Nat\}$.
  Ние вече знаем от \Problem{regular:pumping:an-bn}, че $L$ не е регулярен език.
  Да се опитаме да построим автомат, който го разпознава.
  Нека
  \[L_k \df \{a^nb^{n+k}\mid n \in \Nat\}.\]
  Да видим какво се получава като приложим процедурата за строене 
  на минимален автомат.
  \begin{itemize}
  \item 
    $a^{-1}(L) = L_1$;
  \item
    $b^{-1}(L) = \emptyset$;
  \item
    $a^{-1}(L_1) = L_2$;
  \item
    $b^{-1}(L_1) = \{\varepsilon\}$;
  \item
    $a^{-1}(\{\varepsilon\}) = b^{-1}(\{\varepsilon\}) = \emptyset$;
  \item
    Вижда се, че $a^{-1}(L_k) = L_{k+1}$, за всяко $k$.
  \item
    Вижда се, че $b^{-1}(L_{k+1}) = \{b^k\}$, за всяко $k$.
    Освен това е ясно, че $b^{-1}(\{b^{k}\}) = \{b^{k-1}\}$, за всяко $k \geq 1$.
  \end{itemize}
  Получаваме, че езикът $L$ се разпознава от автомат с {\em безкрайно много състояния}.
  
% \begin{framed}  
  \begin{figure}[H]
    \centering
    \begin{tikzpicture}[framed,->,>=stealth,thick,node distance=70pt]
      \tikzstyle{every state}=[circle]% ,minimum size=15pt,auto]
      
      \node[state, initial above]             (0) {$L$};
      \node[state]                            (1) [right of=0]{$L_1$};
      \node[state]                            (2) [right of=1]{$L_2$};
      \node[state]                            (3) [right of=2]{$L_3$};
      \node[state,accepting]                  (A) [below of=1]{$\{\varepsilon\}$};
      \node[state]                            (B) [below right of=1]{$\{b\}$};
      \node[state]                            (BB) [below right of=2]{$\{bb\}$};
      \node[state]                            (E) [below of=A]{$\emptyset$};
      
      \coordinate[right of=3] (4);
      \coordinate[below right of=3] (BBB);
      \coordinate[below of=4] (BBBA);

      \path 
      (0) edge [bend left=15]   node [above] {$a$} (1)
      (1) edge [bend left=15]   node [above] {$a$} (2)
      (2) edge [bend left=15]   node [above] {$a$} (3)
      (0) edge [bend right=30]  node [left] {$b$} (E)
      (E) edge [loop left]      node [left] {$a,b$} (E)
      (1) edge [bend right=30]  node [left] {$b$} (A)
      (2) edge [bend right=15]  node [left] {$b$} (B)
      (3) edge [bend right=15]  node [left] {$b$} (BB)
      (B) edge [bend right=15]  node [above] {$b$} (A)
      (B) edge [bend left=15]  node [right] {$a$} (E)
      (A) edge [bend right=15]   node [right] {$a,b$} (E)
      (BB) edge [bend right=15] node [above] {$b$} (B)
      (BB) edge [bend left=15]  node [below] {$a$} (E);
      
      \draw [dashed,->,shorten >=0pt] (3) to[bend left=15] node[auto] {$a$} (4);
      \draw [dashed,->,shorten >=0pt] (BBB) to[bend right=15] node[above] {$b$} (BB);
      \draw [dashed,->,shorten >=0pt] (BBBA) to[bend left=30] node[below] {$a$} (E);
    \end{tikzpicture}
    \caption{{\em Безкраен} автомат за $\{a^nb^n \mid n \in \Nat\}$.}
  \end{figure}
% \end{framed}
\end{example}

%%% Local Variables:
%%% mode: latex
%%% TeX-master: "../eai"
%%% End:
