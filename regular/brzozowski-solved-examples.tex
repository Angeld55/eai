\subsection{Примерни задачи}

\mynote{Тук $\bm{r}^+ \df \bm{r} \cdot \bm{r}^\star$.}
\begin{extra}
\begin{problem}
  Постройте автомат $\B$ по метода на Бжозовски за регулярния език
  \[L = \L(\bm{((a+b)^+\cdot a)^\star}).\]
\end{problem}  
\begin{solution}
  \begin{multicols}{2}
    \begin{itemize}
      
\item
  Ясно е, че ще започнем от началното състояние $\hat{L}$.
  Удобно е да имаме предвид следното представяне
  \mynote{Очевидно е, че $\bm{r}^\star = \bm{\varepsilon} + \bm{r}^+$. Веднага се вижда, че $\hat{L}$ ще бъде и финално състояние, защото $\varepsilon \in L$.}
  \begin{align*}
    L & = (\{a,b\}^+ \cdot \{a\})^\star \\
      & = \{\varepsilon\} \cup (\{a,b\}^+ \cdot \{a\})^+\\
      & = \{\varepsilon\} \cup \{a,b\}^+ a L\\
      & = \{\varepsilon\} \cup \{a,b\} \cdot \{a,b\}^\star a L\\
      & = \{\varepsilon\} \cup a\{a,b\}^\star aL \cup b\{a,b\}^\star aL
  \end{align*}
\item
  Сега като имаме това представяне на $L$, лесно се съобразява, че
  \[a^{-1}(L) = b^{-1}(L) = \{a,b\}^\star aL.\]
  Нека положим $M \df \{a,b\}^\star aL$.
  Лесно се съобразява, че $M \neq L$, защото $\varepsilon \in L$, но $\varepsilon \not\in M$.
  Това означава, че имаме ново състояние $\hat{M}$ и
  \begin{align*}
    & \delta(\hat{L},a) \df \hat{M} & \comment\text{ защото }a^{-1}(L) = M\\
    & \delta(\hat{L},b) \df \hat{M}. & \comment\text{ защото }b^{-1}(L) = M
  \end{align*}
\item
  \mynote{Веднага се вижда, че $\hat{M}$ няма да бъде финално състояние.}
  За следващата стъпка е удобно да представим езика $M$ по следния начин:
  \begin{align*}
    M & = \{a,b\}^\star aL\\
      & = aL \cup \{a,b\}^+aL\\
      & = aL \cup \{a,b\}\cdot \{a,b\}^\star aL\\
      & = aL \cup \{a,b\}\cdot M\\
      & = aL \cup aM \cup bM
  \end{align*}
  От това представяне на $M$ веднага се съобразява, че
  \begin{align*}
    & a^{-1}(M) = L \cup M\\
    & b^{-1}(M) = M.
  \end{align*}
  \mynote{Обърнете внимание, че $L \subset N$, но $\hat{L}$ и $\hat{N}$ са различни състояния.}
  Нека да положим $N \df L \cup M$.
  Имаме, че $N \neq L$, защото $a\in N$, но $a \not\in L$.
  Освен това, $N \neq M$, защото $\varepsilon \in N$, но $\varepsilon \not\in M$.
  Това означава, че имаме ново състояние $\hat{N}$ и тогава
  \begin{align*}
    & \delta(\hat{M},a) \df \hat{N} & \comment\text{ защото }a^{-1}(M) = N\\
    & \delta(\hat{M},b) \df \hat{M} & \comment\text{ защото }b^{-1}(M) = M.
  \end{align*}
\item
  Да разгледаме следното представяне:
  \begin{align*}
    N & = L \cup M \\
      & = \{\varepsilon\} \cup aM \cup bM \cup M\\
      & = \{\varepsilon\} \cup aM \cup bM \cup aL\\
      & = \{\varepsilon\} \cup aN \cup bM.
  \end{align*}
  Веднага можем да съобразим, че
  \begin{align*}
    & a^{-1}(N) = N\\
    & b^{-1}(N) = M.
  \end{align*}
  Сега полагаме:
  \begin{align*}
    & \delta(\hat{N},a) \df \hat{N} & \comment\text{ защото }a^{-1}(N) = N\\
    & \delta(\hat{N},b) \df \hat{M} & \comment\text{ защото }b^{-1}(N) = M.
  \end{align*}
\item
  Нямаме повече нови състояния. Следователно,
  \[Q \df \{\hat{L}, \hat{M}, \hat{N}\}.\]
\item
  Понеже $\varepsilon \in L$ и $\varepsilon \in N$ е ясно, че
  \[F = \{\hat{L},\hat{N}\}.\]
  Сега вече сме готови да нарисуваме картинка на автомата.
\end{itemize}

\begin{figure}[H]
  \centering
    \begin{tikzpicture}[framed,->,>=stealth,thick,node distance=70pt,scale=0.8, every node/.style={scale=0.8},initial text=начало]
      \tikzstyle{every state}=[circle,font=\small]
      \node[initial above, state,accepting]   (L) {$\hat{L}$};
      \node[state]                            (M) [right of=L]{$\hat{M}$};
      \node[state,accepting]                  (N) [right of=M]{$\hat{N}$};
      
      \path 
      (L) edge [bend left=15] node [above] {$a,b$} (M)
      (M) edge [loop above] node [above]   {$b$}   (M)
      (M) edge [bend left=15] node [above] {$a$}   (N)
      (N) edge [bend left=15] node [below] {$b$}   (M)
      (N) edge [loop above] node [above]   {$a$}   (N);
    \end{tikzpicture}
    \caption{\scriptsize{Автомат за езика $L$ по метода на Бжозовски.}}
  \end{figure}
\end{multicols}
  \end{solution}
\end{extra}

\clearpage

\begin{extra}
  \begin{problem}\label{ex:brzozowski-solved-examples-2}
    Постройте автомат $\B$ по метода на Бжозовски, който рапознава регулярния език
    \[L = \L(\mathbf{a\cdot(a+b)^\star\cdot b}).\]
  \end{problem}  
\begin{solution}
  \begin{multicols}{2}
    \begin{itemize}
    \item
      Започваме с началното състояние $\hat{L}$, за което е ясно, че няма да бъде финално,
      защото $\varepsilon \not\in L$.
  \item 
    $a^{-1}(L) = \{a,b\}^\star b \df M$.
    Имаме, че $M \neq L$, защото $b \in M$, но $b \not\in L$.
    Тогава
    \begin{align*}
      & \delta(\hat{L}, a) \df \hat{M} & \comment\text{ защото }a^{-1}(L) = M\\
      & \delta(\hat{L},b) \df \hat{\emptyset} & \comment\text{ защото }b^{-1}(L) = \emptyset\\
    \end{align*}
    $\emptyset$ е съвсем нормален език и напълно нормално да имаме състояние $\hat{\emptyset}$.
  \item    
    За по-нататък ще е удобно да представим $M$ по следния начин:
    \begin{align*}
      M & = \{a,b\}^\star b \\
        & = \{a,b\}^+ b \cup \{b\}\\
        & = a\cdot \{a,b\}^\star \cdot b \cup b\cdot \{a,b\}^\star \cdot b \cup \{b\}\\
        & = aM \cup bM \cup \{b\}.
    \end{align*}
    Сега е ясно, че
    \begin{align*}
      & a^{-1}(M) = M\\
      & b^{-1}(M) = \{\varepsilon\} \cup M.
    \end{align*}
    Нека да положим $N \df \{\varepsilon\} \cup M$.
    Имаме, че $N \neq L$ и $N \neq M$, защото $\varepsilon \in N$, но $\varepsilon \not\in L$ и $\varepsilon \not\in M$.
    Тогава
    \begin{align*}
      & \delta(\hat{M},a) \df \hat{M} & \comment\text{ защото }a^{-1}(M) = M\\
      & \delta(\hat{M},b) \df \hat{N} & \comment\text{ защото }b^{-1}(M) = N
    \end{align*}
  \item
    Можем да представим езика $N$ по следния начин:
    \[N = \{\varepsilon\} \cup aM \cup bM \cup \{b\}.\]
    Тогава имаме, че:
    \begin{align*}
      & \delta(\hat{N},a) \df \hat{M} & \comment\text{ защото } a^{-1}(N) = M\\
      & \delta(\hat{N},b) \df \hat{N} & \comment\text{ защото } b^{-1}(N) = M.
    \end{align*}
  \item
    Завършваме с дефиницията на функцията на преходите като:
    \begin{align*}
      & \delta(\hat{\emptyset},a) \df \hat{\emptyset} & \comment\text{ защото }a^{-1}(\emptyset) = \emptyset\\
      & \delta(\hat{\emptyset},b) \df \hat{\emptyset} & \comment\text{ защото }b^{-1}(\emptyset) = \emptyset.
    \end{align*}
  \item
    Съобразете сами, че $F = \{\hat{N}\}$.
  \end{itemize}    
  \begin{figure}[H]
    \centering
    \begin{tikzpicture}[framed,->,>=stealth,thick,node distance=55pt,scale=0.8, every node/.style={scale=0.8},initial text=начало]
      \tikzstyle{every state}=[circle]
      
      \node[initial, state]                   (L) {$\hat{L}$};
      \node[state]                            (M) [above right of=L]{$\hat{M}$};
      \node[state]                            (E) [below right of=L]{$\hat{\emptyset}$};
      \node[state,accepting]                  (N) [right of=M]{$\hat{N}$};
      
      \path 
      (L) edge [bend left=15]  node [left] {$a$} (M)
      (L) edge [bend right=15] node [left] {$b$} (E)
      (E) edge [loop right]    node [right] {$a,b$} (E) 
      (M) edge [bend right=15] node [below] {$b$} (N)
      (M) edge [loop above]    node [above] {$a$} (M)
      (N) edge [bend right=30] node [above] {$a$} (M)
      (N) edge [loop above]    node [above] {$b$} (N);
    \end{tikzpicture}
    \caption{\scriptsize{Автомат за езика $\L(\mathbf{a\cdot (a+b)^\star\cdot b})$ по метода на Бжозовски.}}
  \end{figure}
  \end{multicols}
\end{solution}
\end{extra}

\newpage

\begin{extra}
  \begin{problem}
    Да припомним, че в \Problem{regular:dfa:binary} се искаше да се докаже, че езикът 
    \[L = \{\alpha \in \{0,1\}^\star \mid \bin{\alpha} \equiv 2 \bmod 3\}\]
    е регулярен.
    Ние направихме това като построихме автомат за $L$ и доказахме, че той разпознава $L$.
    Сега пък постройте автомат за $L$ по метода на Бжозовски.
  \end{problem}  
  \begin{solution}
    \begin{multicols}{2}
      За целта ще ни трябва алтернативна дефиниция на $\bin{\alpha}$.
      За една дума $\alpha \in \{0,1\}^\star$, можем да дадем следната дефиниция на $\bin{\alpha}$:
      \begin{itemize}
      \item
        $\bin\varepsilon = 0$,
      \item
        $\bin{0\alpha} = \bin{\alpha}$,
      \item
        $\bin{1\alpha} = 2^{|\alpha|} + \bin{\alpha}$.
      \end{itemize}
      Лесно се съобразява, че началното състояние $\hat{L}$ на автомата на Бжозовски не е финално, защото $\varepsilon \not\in L$.
      Да започнем с преходите от началното състояние:
      \begin{align*}
        0^{-1}(L) & = \{\alpha \in \{0,1\}^\star \mid 0\alpha \in L\}\\
                  & = \{\alpha \in \{0,1\}^\star \mid \bin{0\alpha} \equiv 2 \bmod 3\}\\
                  & = \{\alpha \in \{0,1\}^\star \mid \bin{\alpha} \equiv 2 \bmod 3\}\\
                  & = L.\\
        1^{-1}(L) & = \{\alpha \in \{0,1\}^\star \mid 1\alpha \in L\}\\
                  & = \{\alpha \in \{0,1\}^\star \mid \bin{1\alpha} \equiv 2 \bmod 3\}\\
                  & = \{\alpha \in \{0,1\}^\star \mid 2^{|\alpha|} + \bin{\alpha} \equiv 2 \bmod 3\}\\
                  & \df M.
      \end{align*}
      Лесно се съобразява, че езикът $M$ е различен от $L$,
      защото например думата $10 \in L$, но $10 \not\in M$.
      Също така, понеже $2^{|\varepsilon|} + \bin{\varepsilon} = 1$,
      то $\varepsilon \not \in M$ и следователно $\hat{M}$ няма да бъде
      финално състояние в автомата на Бжозовски.
      Продължаваме нататък:
      % \begin{multicols}{2}
      \begin{align*}
        0^{-1}(M) & = \{\alpha \mid 0\alpha \in M\}\\
                  & = \{\alpha \mid 2^{|0\alpha|} + \bin{0\alpha} \equiv 2 \bmod 3\}\\
                  & = \{\alpha \mid 2\cdot 2^{|\alpha|} + \bin{\alpha} \equiv 2 \bmod 3\}\\
                  & \df N.
      \end{align*}
      \begin{align*}
        1^{-1}(M) & = \{\alpha \mid 1\alpha \in M\}\\
                  & = \{\alpha \mid 2^{|1\alpha|} + \bin{1\alpha} \equiv 2 \bmod 3\}\\
                  & = \{\alpha \mid 2\cdot 2^{|\alpha|} + 2^{|\alpha|} + \bin{\alpha} \equiv 2 \bmod 3\}\\
                  & = \{\alpha \mid 3\cdot 2^{|\alpha|} + \bin{\alpha} \equiv 2 \bmod 3\}\\
                  & = \{\alpha \mid \bin{\alpha} \equiv 2 \bmod 3\}\\
                  & = L.
      \end{align*}
      Сега трябва да се уверим, че езикът $N$ е различен от $L$ и $M$.
      Това отново е лесно. Непосредствено се проверява, че думата $11 \in N$,
      но $11 \not\in L$ и $11 \not\in M$.
      Също така да отбележим, че понеже $2.2^{|\varepsilon|} + \bin{\varepsilon} = 2$,
      то $\hat{N}$ ще бъде финално състояние в автомата на Бжозовски, защото
      $\varepsilon \in N$.
      Продължаваме нататък с преходите от $\hat{N}$:
      \begin{align*}
        0^{-1}(N) & = \{\alpha \mid 0\alpha \in N\}\\
                  & = \{\alpha \mid 2\cdot 2^{|0\alpha|} + \bin{0\alpha} \equiv 2 \bmod 3\}\\
                  & = \{\alpha \mid 4\cdot 2^{|\alpha|} + \bin{\alpha} \equiv 2 \bmod 3\}\\
                  & = \{\alpha \mid 3\cdot 2^{|\alpha|} + 2^{|\alpha|} + \bin{\alpha} \equiv 2 \bmod 3\}\\
                  & = \{\alpha \mid 2^{|\alpha|} + \bin{\alpha} \equiv 2 \bmod 3\}\\
                  & = M\\
                  % \end{align*}
                  % \begin{align*}
        1^{-1}(N) & = \{\alpha \mid 1\alpha \in N\}\\
                  & = \{\alpha \mid 2\cdot 2^{|1\alpha|} + \bin{1\alpha} \equiv 2 \bmod 3\}\\
                  & = \{\alpha \mid 4\cdot 2^{|\alpha|} + 2^{|\alpha|} + \bin{\alpha} \equiv 2 \bmod 3\}\\
                  & = \{\alpha \mid 3\cdot 2^{|\alpha|} + 2\cdot 2^{|\alpha|} + \bin{\alpha} \equiv 2 \bmod 3\}\\
                  & = \{\alpha \mid 2\cdot 2^{|\alpha|} + \bin{\alpha} \equiv 2 \bmod 3\}\\
                  & = N.
      \end{align*}
      % Така можем да получим автомат за езика $L$, където всяко състояние е свързано с език.
\begin{figure}[H]
  \begin{center}
    \begin{tikzpicture}[framed,->,>=stealth,thick,node distance=55pt,scale=0.8, every node/.style={scale=0.8},initial text=начало]
      \tikzstyle{every state}=[circle]
      
      \node[initial below,state] (0) {$\hat{L}$};
      \node[state]               (1) [right of=0]{$\hat{M}$};
      \node[accepting, state]    (2) [right of=1]{$\hat{N}$};
      
      \path 
      (0) edge  [loop above]    node [above]  {$0$} (0)
      (0) edge  [bend left=15]  node [above]  {$1$} (1)
      (2) edge  [bend left=15]  node [below]  {$0$} (1)
      (1) edge  [bend left=15]  node [below]  {$1$} (0)
      (1) edge  [bend left=15]  node [above]  {$0$} (2)
      (2) edge  [loop above]    node [above]  {$1$} (2);
    \end{tikzpicture}
  \end{center}
  \caption{Автомат, получен чрез метода на Бжозовски за езика $L$.}
\end{figure}
\end{multicols}
\end{solution}
\end{extra}

% \newpage

\begin{extra}
\begin{problem}
  Постройте автомат по метода на Бжозовски за регулярния език
  \[L = \{\omega \in \{a,b\}^\star \mid \card{\omega}{a}\text{ е четно и }\card{\omega}{b} = 1\}.\]
\end{problem}
  \begin{solution}
    \begin{multicols}{2}
      Да започнем с преходите от началното състояние $\hat{L}$:
      \begin{align*}
        a^{-1}(L) & = \{\omega \in \{a,b\}^\star \mid \card{\omega}{a} \text{ е нечет. и }\card{\omega}{b} = 1\}\\
                  & \df M\\
        b^{-1}(L) & = \{\omega \in \{a,b\}^\star \mid \card{\omega}{a} \text{ е четно и } \card{\omega}{b} = 0\}\\
                  & \df N.
      \end{align*}
      Лесно се вижда, че $M$ и $N$ са различни езици от $L$.
      Например, $aab \in L$, но $aab \not\in M$;
      $aa \in N$, но $aa \not \in L$ и $aa \not\in M$.
      Сега да видим какви преходи имаме от състоянието $\hat{M}$:
      \begin{align*}
        a^{-1}(M) & = L;\\
        b^{-1}(M) & = \{\omega \in \{a,b\}^\star \mid \card{\omega}{a} \text{ е нечет. и } \card{\omega}{b} = 0\}\\
                  & \df P.
      \end{align*}
      Ясно е, че $P \neq M,L,N$, защото $a \in P$, но $a \notin M,L,N$.
      Завършваме с преходите от състоянията $\hat{N}$ и $\hat{P}$:
      \begin{align*}
        a^{-1}(N) & = P \quad\text{и}\quad b^{-1}(N) = \emptyset\\
        a^{-1}(P) & = N \quad\text{и}\quad b^{-1}(P) = \emptyset.
      \end{align*}

      \begin{figure}[H]
    \centering
    \begin{tikzpicture}[framed,->,>=stealth,thick,node distance=65pt,scale=0.8, every node/.style={scale=0.8},initial text=начало]
      \tikzstyle{every state}=[circle]
      
      \node[initial below, state]  (L) {$\hat{L}$};
      \node[state]                 (M) [above of=L]{$\hat{M}$};
      \node[state]                 (P) [right of=M]{$\hat{P}$};
      \node[state, accepting]      (N) [below of=P]{$\hat{N}$};
      \node[state]                 (E) [above right of=N]{$\hat{\emptyset}$};
            
      \path 
      (L) edge [bend right=15]  node [right] {$a$} (M)
      (M) edge [bend right=15]  node [left]  {$a$} (L)
      (L) edge [bend right=15]  node [below] {$b$} (N)
      (M) edge [bend left=15]   node [above] {$b$} (P)
      (N) edge [bend left=15]   node [left]  {$a$} (P)
      (P) edge [bend left=15]   node [right] {$a$} (N)
      (P) edge [bend left=15]   node [above] {$b$} (E)
      (N) edge [bend right=15]  node [below] {$b$} (E)
      (E) edge [loop above]     node [above] {$a,b$} (E);      
    \end{tikzpicture}
    \caption{Автомат, който приема думи с четен брой $a$ и точно едно $b$, получен чрез метода на Бжозовски.}
  \end{figure}
\end{multicols}
  \end{solution}

\end{extra}


\mynote{\cite[стр. 113]{sakarovitch-book}.}
\begin{extra}
\begin{example}\label{ex:regular:brzozowski:an-bn}
  Да разгледаме езика $L = \{a^nb^n\mid n \in \Nat\}$.
  Ние вече знаем от \Example{regular:pumping:an-bn}, че $L$ не е регулярен език.
  Да се опитаме да построим автомат, който го разпознава.
  \begin{multicols}{2}
    Нека за всяко $k$ да разгледаме езика
    \[L_k \df \{a^nb^{n+k}\mid n \in \Nat\}.\]
    Да видим какво се получава като приложим процедурата за строене 
    на минимален автомат.
    \begin{itemize}
    \item 
      $a^{-1}(L) = L_1$;
    \item
      $b^{-1}(L) = \emptyset$;
    \item
      $a^{-1}(L_1) = L_2$;
    \item
      $b^{-1}(L_1) = \{\varepsilon\}$;
    \item
      $a^{-1}(\{\varepsilon\}) = b^{-1}(\{\varepsilon\}) = \emptyset$;
    \item
      Лесно можем да докажем, че за всяко $k$ е изпълнено, че $a^{-1}(L_k) = L_{k+1}$.
    \item
      Лесно се вижда, че $b^{-1}(L_{k+1}) = \{b^k\}$, за всяко $k$.
    \item
      Ясно е, че $b^{-1}(\{b^{k}\}) = \{b^{k-1}\}$, за $k \geq 1$.
    \end{itemize}    
    Получаваме, че езикът $L$ се разпознава от автомат с {\em безкрайно много състояния}.
  
% \begin{framed}  
  \begin{figure}[H]
    \centering
    \begin{tikzpicture}[framed,->,>=stealth,thick,node distance=60pt,scale=0.8, every node/.style={scale=0.75},initial text=начало]
      % \pgftransformscale{.4}
      \tikzstyle{every state}=[circle]% ,minimum size=15pt,auto]
      
      \node[state, initial above, accepting]  (0) {$\hat{L}$};
      \node[state]                            (1) [right of=0]{$\hat{L}_1$};
      \node[state]                            (2) [right of=1]{$\hat{L}_2$};
      \node[state]                            (3) [right of=2]{$\hat{L}_3$};
      \node[state,accepting]                  (A) [below of=1]{$\hat{\{\varepsilon\}}$};
      \node[state]                            (B) [below right of=1]{$\hat{\{b\}}$};
      \node[state]                            (BB) [below right of=2]{$\hat{\{bb\}}$};
      \node[state]                            (E) [below of=A]{$\hat{\emptyset}$};
      
      \coordinate[right of=3] (4);
      \coordinate[below right of=3] (BBB);
      \coordinate[below of=4] (BBBA);

      \path 
      (0) edge [bend left=15]   node [above] {$a$} (1)
      (1) edge [bend left=15]   node [above] {$a$} (2)
      (2) edge [bend left=15]   node [above] {$a$} (3)
      (0) edge [bend right=30]  node [left] {$b$} (E)
      (E) edge [loop left]      node [left] {$a,b$} (E)
      (1) edge [bend right=30]  node [left] {$b$} (A)
      (2) edge [bend right=15]  node [left] {$b$} (B)
      (3) edge [bend right=15]  node [left] {$b$} (BB)
      (B) edge [bend right=15]  node [above] {$b$} (A)
      (B) edge [bend left=15]  node [right] {$a$} (E)
      (A) edge [bend right=15]   node [right] {$a,b$} (E)
      (BB) edge [bend right=15] node [above] {$b$} (B)
      (BB) edge [bend left=15]  node [below] {$a$} (E);
      
      \draw [dashed,->,shorten >=0pt] (3) to[bend left=15] node[auto] {$a$} (4);
      \draw [dashed,->,shorten >=0pt] (BBB) to[bend right=15] node[above] {$b$} (BB);
      \draw [dashed,->,shorten >=0pt] (BBBA) to[bend left=30] node[below] {$a$} (E);
    \end{tikzpicture}
    \caption{{\em Безкраен} автомат за езика $\{a^nb^n~\mid~n~\in~\Nat\}$ построен по метода на Бжозовски.}
  \end{figure}
    \end{multicols}  


% \end{framed}
\end{example}
\end{extra}

%%% Local Variables:
%%% mode: latex
%%% TeX-master: "../eai"
%%% End:
