\subsection{Примерни задачи}

\mynote{Винаги с удвоени окръжности ще отбелязваме финалните състояния.}

\begin{extra}
\begin{example}
  \label{ex:automata-pictures}
  Да разгледаме няколко примера за автомати и езиците, които разпознават.
  Дефинирайте функцията на преходите $\delta$ за всеки автомат.

  \begin{figure}[H]
    \begin{subfigure}[t]{0.45\textwidth}
      \centering
      \begin{tikzpicture}[->,>=stealth,thick,node distance=45pt]
        \tikzstyle{every state}=[circle,minimum size=20pt,auto]
        
        \node[initial below, state]   (0) {$q_0$};
        \node[state]                  (1) [right of=0]{$q_1$};
        \node[state]                  (2) [right of=1]{$q_2$};
        \node[state,accepting]        (3) [right of=2]{$q_3$};
        
        \path 
        (0) edge [loop above]   node [above] {$a$}    (0)
        (0) edge [bend left=15] node [above] {$b$}    (1)
        (1) edge [loop above]   node [above] {$b$}    (1)
        (1) edge [bend left=15] node [above] {$a$}    (2)
        (2) edge [bend left=45] node [below] {$a$}    (0)
        (2) edge [bend left=15] node [above] {$b$}    (3)
        (3) edge [loop above]   node [above] {$a,b$}  (3);
      \end{tikzpicture}
      \subcaption{$\{\omega \in \{a,b\}^\star \mid \omega\mbox{ съдържа }bab\}$}
    \end{subfigure}
    \qquad
    \begin{subfigure}[t]{0.45\textwidth}
      \centering
      \begin{tikzpicture}[->,>=stealth,thick,node distance=45pt]
        \tikzstyle{every state}=[circle,minimum size=20pt,auto]
        
        \node[initial below, state]   (0) {$q_0$};
        \node[state]                  (1) [right of=0]{$q_1$};
        \node[state,accepting]        (2) [right of=1]{$q_2$};
        
        \path 
        (0) edge [loop above]   node [above] {$b$}    (0)
        (0) edge [bend left=15] node [above] {$a$}    (1)
        (1) edge [loop above]   node [above] {$b$}    (1)
        (1) edge [bend left=15] node [above] {$a$}    (2)
        (2) edge [loop above]   node [above] {$a,b$}  (2);
      \end{tikzpicture}
      \subcaption{$\{\omega \in \{a,b\}^\star \mid \card{\omega}{a} \geq 2\}$}
    \end{subfigure}
    \begin{subfigure}[t]{0.45\textwidth}
      \centering
      \begin{tikzpicture}[->,>=stealth,thick,node distance=45pt]
        \tikzstyle{every state}=[circle,minimum size=20pt,auto]
        
        \node[initial below, accepting, state] (0) {$q_0$};
        \node[state]                           (1) [right of=0]{$q_1$};
        \node[state]                           (2) [right of=1]{$q_2$};
        
        \path 
        (0) edge [loop above]   node [above] {$b$}   (0)
        (0) edge [bend left=15] node [above] {$a$}   (1)
        (1) edge [bend left=15] node [below] {$b$}   (0)
        (1) edge [bend left=15] node [above] {$a$}   (2)
        (2) edge [loop above] node   [above] {$a,b$} (2);
      \end{tikzpicture}
      \subcaption{$\{\omega \in \{a,b\}^\star \mid $ всяко $a$ в $\omega$ веднага се следва от поне едно $b\}$ }
    \end{subfigure}
    \qquad
    \begin{subfigure}[t]{0.45\textwidth}
      \centering
      \begin{tikzpicture}[->,>=stealth,thick,node distance=45pt]
        \tikzstyle{every state}=[circle,minimum size=20pt,auto]
        
        \node[initial below, state, accepting]   (0) {$q_0$};
        \node[state]                             (1) [right of=0]{$q_1$};
        \node[state]                             (2) [right of=1]{$q_2$};
        
        \path 
        (0) edge [loop above]   node   [above] {$b$}    (0)
        (0) edge [bend left=15] node   [above] {$a$}    (1)
        (1) edge [loop above]   node   [above] {$b$}    (1)
        (1) edge [bend left=15] node   [above] {$a$}    (2)
        (2) edge [loop above]   node   [above] {$b$}    (2)
        (2) edge [bend left=45] node   [below] {$a$}    (0);
      \end{tikzpicture}
      \subcaption{$\{\omega \in \{a,b\}^\star \mid \card{\omega}{a} \equiv 0 \bmod\ 3\}$}
    \end{subfigure}
  \end{figure}    
\end{example}
\end{extra}

В повечето от горните примери може сравнително лесно да се съобрази, че построеният автомат разпознава желания език.
При по-сложни задачи обаче, ще се наложи да дадем доказателство, като обикновено се прилага 
{\em метода на математическата индукция} върху дължината на думите.
Ще разгледаме няколко такива примера.

\begin{problem}
  \mynote{Най-лесния начин да се сетим как да построим $\A$ е като първо построим автомат за езика, който разпознава тези думи, в които се съдържат две поредни срещания на $a$ и вземем неговото допълнение съгласно Твърдение \ref{pr:automata-complement}. По-късно ще разгледаме общ метод за строене на автомат по език.}
  Докажете, че езикът
  \[L = \{\alpha \in \{a,b\}^\star\ \mid\ \alpha\mbox{ не съдържа две поредни срещания на }a\}\]
  е автоматен.
\end{problem}
\begin{proof}
  Да разгледаме $\A = \FA$ с функция на преходите описана на \Figure{dfa:not-two-a}.
  \begin{figure}[H]
    \begin{center}
      \begin{tikzpicture}[framed,->,>=stealth,thick,node distance=55pt]
        \tikzstyle{every state}=[circle,minimum size=20pt,auto]
        
        \node[initial, accepting, state] (0) {$q_0$};
        \node[accepting, state]          (1) [right of=0]{$q_1$};
        \node[state]                     (2) [right of=1]{$q_2$};
        
        \path 
        (0) edge [loop above]   node [above] {$b$}   (0)
        (0) edge [bend left=15] node [above] {$a$}   (1)
        (1) edge [bend left=15] node [below] {$b$}   (0)
        (1) edge [bend left=15] node [above] {$a$}   (2)
        (2) edge [loop above]   node [above] {$a,b$} (2);
      \end{tikzpicture}
    \end{center}
    \caption{Автомат $\A$ разпознаващ думите, които не съдържат две поредни срещания на $a$}
    \label{fig:dfa:not-two-a}
  \end{figure}

  Ще докажем, че $L = \L(\A)$.
  Първо ще се концентрираме върху доказателството на $\L(\A) \subseteq L$.
  \mynote{$\abs{\alpha} \df $ дължината на $\alpha$.}
  Ще докажем, че за всяка дума $\alpha \in \Sigma^\star$ са изпълнени свойствата:
  \begin{enumerate}[(1)]
  \item 
    ако $\delta^\star(q_0,\alpha) = q_0$, то
    $\alpha$ не съдържа две поредни срещания на $a$
    и ако $\abs{\alpha} > 0$, то $\alpha$ завършва на $b$;
  \item
    ако $\delta^\star(q_0,\alpha) = q_1$, то
    $\alpha$ не съдържа две поредни срещания на $a$
    и завършва на $a$.
  \end{enumerate}
  Ще докажем (1) и (2) едновременно с индукция по дължината на думата $\alpha$.  
  \begin{itemize}
  \item
    За $\abs{\alpha} = 0$, т.е. $\alpha = \varepsilon$, твърденията (1) и (2) са ясни (Защо?).
  \item
    Да приемем, че твърденията $(1)$ и $(2)$ са верни за произволни думи $\alpha$ с дължина $n$.
  \item
    Нека $\abs{\alpha} = n+1$, т.е. $\alpha = \beta x$, където $\abs{\beta} = n$ и $x \in \Sigma$.
    Ще докажем (1) и (2) за $\alpha$.
    \begin{itemize}[-]
    \item 
      Нека $\delta^\star(q_0,\beta x) = q_0 = \delta(\delta^\star(q_0,\beta),x)$.
      Според дефиницията на функцията $\delta$, $x = b$ и $\delta^\star(q_0,\beta) \in \{q_0,q_1\}$.
      Тогава по {\bf И.П.} за (1) и (2), $\beta$ не съдържа две поредни срещания на $a$.
      Тогава е очевидно, че $\beta x$ също не съдържа две поредни срещания на $a$.
    \item
      Нека $\delta^\star(q_0,\beta x) = q_1 = \delta(\delta^\star(q_0,\beta),x)$.
      Според дефиницията на $\delta$, $x = a$ и $\delta^\star(q_0,\beta) = q_0$.
      Тогава по {\bf И.П.} за (1), $\beta$ не съдържа две поредни срещания на $a$ и ако $|\beta| > 0$, то $\beta$ завършва на $b$.
      Тогава е очевидно, че $\beta x$ също не съдържа две поредни срещания на $a$.
    \end{itemize}
  \end{itemize}
  
 Така доказахме с индукция по дължината на думата, че за всяка дума $\alpha$ са  изпълнени твърденията $(1)$ и $(2)$.
 По дефиниция, ако $\alpha \in \L(\A)$, то $\delta^\star(q_0,\alpha) \in F = \{q_0,q_1\}$ и от $(1)$ и $(2)$ следва, че и в двата случа
 $\alpha$ не съдържа две поредни срещания на буквата $a$, т.е. $\alpha \in L$.
 С други думи, доказахме, че 
 \[\L(\A) \subseteq L.\]
 Сега ще докажем другата посока, т.е. $L \subseteq \L(\A)$.
 Това означава да докажем, че
 \[(\forall \alpha \in \Sigma^\star)[\alpha \in L\ \Rightarrow\ \delta^\star(q_0,\alpha) \in F],\]
 \mynote{Да напомним, че от съждителното смятане знаем, че $p \rightarrow q \equiv \neg q \rightarrow \neg  p$.}
 което е еквивалентно на
 \begin{equation}
   \label{eq:case2}
   (\forall \alpha \in \Sigma^\star)[\delta^\star(q_0,\alpha) \not\in F \ \Rightarrow\ \alpha\not\in L].
 \end{equation}
 Това е лесно да се съобрази.
 Щом $\delta^\star(q_0,\alpha) \not\in F$, то 
 $\delta^\star(q_0,\alpha) = q_2$ и думата $\alpha$ може да се представи по следния начин:
 \[\alpha = \beta a \gamma\ \&\ \delta^\star(q_0,\beta) = q_1.\]
 
 Използвайки свойство (2) от по-горе, понеже $\delta^\star(q_0,\beta) = q_1$, то
 $\beta$ не съдържа две поредни срещания на $a$, но завършва на $a$.
 Сега е очевидно, че $\beta a$ съдържа две поредни срещания на $a$ и 
 щом $\beta a$ е префикс на $\alpha$, то думата $\alpha \not\in L$.
 С това доказахме Свойство \ref{eq:case2}, а следователно и посоката $L\subseteq \L(\A)$.
\end{proof}


\begin{problem}
  Докажете, че езикът
  \[L = \{\omega \in \{a,b\}^\star \mid a \text{ се среща четен брой, докато $b$ нечетен брой пъти в }\omega\}\]
  е автоматен.
\end{problem}
\begin{hint}
  Разгледайте автомата $\A$:
  \begin{figure}[H]
    \begin{center}
      \begin{tikzpicture}[->,>=stealth,thick,node distance=55pt]
        \tikzstyle{every state}=[circle,minimum size=15pt,auto]
        
        \node[initial,state]      (00) {$q_{00}$};
        \node[state]              (10) [above right of=00]{$q_{10}$};
        \node[accepting, state]   (01) [below right of=00]{$q_{01}$};
        \node[state]              (11) [below right of=10]{$q_{11}$};
        
        \path 
        (00) edge  [bend left=15]  node [above]  {$a$} (10)
        (10) edge  [bend left=15]  node [below]  {$a$} (00)
        (01) edge  [bend right=15] node [above]  {$b$} (00)
        (00) edge  [bend right=15] node [below]  {$b$} (01)
        (10) edge  [bend left=15]  node [above]  {$b$} (11)
        (11) edge  [bend left=15]  node [below]  {$b$} (10)
        (01) edge  [bend right=15] node [below]  {$a$} (11)
        (11) edge  [bend right=15] node [above]  {$a$} (01);
      \end{tikzpicture}
    \end{center}
    \caption{$\L(\A) \stackrel{?}{=} L$}
    \label{fig:dfa:modulus-2}
  \end{figure}

  Докажете с индукция по дължината на думата $\omega$, че:
  \mynote{$\card{\omega}{a} \df $ броят на срещанията на буквата $a$ в думата $\omega$.}
  \begin{enumerate}[a)]
  \item 
    $\delta^\star(q_{00}, \omega) = q_{00} \implies \card{\omega}{a} \equiv 0 \bmod 2\ \&\ \card{\omega}{b} \equiv 0 \bmod 2$;
  \item 
    $\delta^\star(q_{00}, \omega) = q_{01} \implies \card{\omega}{a} \equiv 0 \bmod 2\ \&\ \card{\omega}{b} \equiv 1 \bmod 2$;
  \item 
    $\delta^\star(q_{00}, \omega) = q_{10} \implies \card{\omega}{a} \equiv 1 \bmod 2\ \&\ \card{\omega}{b} \equiv 0 \bmod 2$;
  \item 
    $\delta^\star(q_{00}, \omega) = q_{11} \implies \card{\omega}{a} \equiv 1 \bmod 2\ \&\ \card{\omega}{b} \equiv 1 \bmod 2$;
  \end{enumerate}
  Оттук направете извода, че за произволна дума $\omega$,
  \[(\forall i<2)(\forall j<2)[\ \delta^\star(q_{00},\omega) = q_{ij} \iff \card{\omega}{a} \equiv i \bmod 2\ \&\ \card{\omega}{b} \equiv j \bmod 2\ ].\]
\end{hint}

За една дума $\alpha \in \{0,1\}^\star$, 
нека с $\ov{\alpha}_{(k)}$ да означим числото, което се представя в $k$-ична бройна система като $\alpha$.
Например, 
\[\bin{1101} = 1 \cdot 2^3+1\cdot 2^2+0\cdot 2^1+1\cdot 2^0 = \ov{13}_{(10)} = 1.10^1 + 3.10^0.\]
\mynote{Да отбележим, че за всяко число $n$ има безкрайно много думи $\alpha$, за които $\bin{\alpha} = n$.
  Например,
  \begin{align*}
    \bin{10} & = \bin{010}\\
             & = \bin{0010}\\
             & = \cdots
  \end{align*}}
За $k = 2$, имаме следната дефиниция:
\begin{itemize}
\item
  $\bin{\varepsilon} = 0$,
\item
  $\bin{\alpha 0} = 2\cdot \bin{\alpha}$,
\item
  $\bin{\alpha 1} = 2\cdot \bin{\alpha} + 1$.
\end{itemize}

\begin{problem}\label{prob:regular:dfa:binary}
  Докажете, че езикът $L = \{\alpha \in \{0,1\}^\star \mid \bin{\alpha} \equiv 2 \bmod\ 3\}$
  е автоматен.
\end{problem}
\begin{proof}
  Нашият автомат ще има три състояния $\{q_0,q_1,q_2\}$, като началното състояние ще бъде $q_0$.
  Целта ни е да дефинираме така автомата, че да имаме следното свойство:
  \begin{equation}
    (\forall\alpha\in\Sigma^\star)(\forall i < 3)[\ \bin{\alpha} \equiv i \bmod\ 3\ \Leftrightarrow\ \delta^\star(q_0,\alpha) = q_i\ ],
  \end{equation}
  т.е. всяко състояние отговаря на определен остатък при деление на три.
  Понеже искаме нашия автомат да разпознава тези думи $\alpha$,
  за които $\alpha_{(2)} \equiv 2\mod 3$, финалното състояние ще бъде $q_2$.
  Дефинираме функцията $\delta$ по следния начин:
  \begin{align*}
    & \delta(q_0,0) = q_0 & \comment{\ \bin{\alpha} \equiv 0 \bmod 3 \implies \bin{\alpha 0} \equiv 0 \bmod 3 }\\
    & \delta(q_0,1) = q_1 & \comment{\ \bin\alpha \equiv 0 \bmod 3 \implies \bin{\alpha 1} \equiv 1 \bmod 3 }\\
    & \delta(q_1,0) = q_2 & \comment{\ \bin{\alpha} \equiv 1 \bmod 3 \implies \bin{\alpha 0} \equiv 2 \bmod 3 }\\
    & \delta(q_1,1) = q_0 & \comment{\ \bin{\alpha} \equiv 1 \bmod 3 \implies \bin{\alpha 1} \equiv 0 \bmod 3 }\\
    & \delta(q_2,0) = q_1 & \comment{\ \bin{\alpha} \equiv 2 \bmod 3 \implies \bin{\alpha 0} \equiv 1 \bmod 3 }\\
    & \delta(q_2,1) = q_2 & \comment{\ \bin{\alpha} \equiv 2 \bmod 3 \implies \bin{\alpha 1} \equiv 2 \bmod 3 }
  \end{align*}
  Ето и картинка на автомата $\A$:
  \begin{figure}[H]
    \begin{center}
      \begin{tikzpicture}[->,>=stealth,thick,node distance=55pt]
        \tikzstyle{every state}=[circle,minimum size=15pt,auto]
        
        \node[initial,state]      (0) {$q_0$};
        \node[state]              (1) [right of=0]{$q_1$};
        \node[accepting, state]   (2) [right of=1]{$q_2$};
        
        \path 
        (0) edge  [loop above]    node [above]  {$0$} (0)
        (0) edge  [bend left=15]  node [above]  {$1$} (1)
        (2) edge  [bend left=15]  node [below]  {$0$} (1)
        (1) edge  [bend left=15]  node [below]  {$1$} (0)
        (1) edge  [bend left=15]  node [above]  {$0$} (2)
        (2) edge  [loop above]    node [above]  {$1$} (2);
      \end{tikzpicture}
      \end{center}
      \caption{$\L(\A) \stackrel{?}{=} \{\alpha\in\{0,1\}^\star \mid \bin{\alpha} \equiv 2\ (\bmod\ 3)\}$}
 \end{figure}
 \noindent Да разгледаме твърденията:
 \begin{enumerate}[(1)]
  \item 
    $\delta^\star(q_0,\alpha) = q_0 \implies \bin{\alpha} \equiv 0 \mod 3$;
  \item 
    $\delta^\star(q_0,\alpha) = q_1 \implies \bin{\alpha} \equiv 1 \mod 3$;
  \item 
    $\delta^\star(q_0,\alpha) = q_2 \implies \bin{\alpha} \equiv 2 \mod 3$.
  \end{enumerate}
  \mynote{Обърнете внимание, че в доказателството на (3) използваме И.П. не само за (3), но и за (2). Поради тази причина трябва да докажем трите свойства едновременно}
  Ще докажем (1), (2) и (3) {\em едновременно} с индукция по дължината на думата $\alpha$.
  За $\abs{\alpha} = 0$, всички условия са изпълнени. (Защо?)
  Да приемем, че (1), (2) и (3) са изпълнени за думи с дължина $n$.
  Нека $\abs{\alpha} = n+1$, т.е. $\alpha = \beta x$, $\abs{\beta} = n$.

  Ще докажем подробно само (3) понеже другите твърдения се доказват по сходен начин.
  Нека $\delta^\star(q_0,\beta x) = q_2$. 
  Имаме два случая:
  \begin{itemize}
  \item 
    $x = 0$. 
    Тогава, по дефиницията на $\delta$, 
    $\delta(q_1,0) = q_2$ и следователно, $\delta^\star(q_0,\beta) = q_1$.
    По {\bf И.П.} за (2) с $\beta$,
    \[\delta^\star(q_0,\beta) = q_1\ \Rightarrow\ \bin{\beta} \equiv 1 \bmod 3\]
    Тогава, $\bin{\beta0} \equiv 2 \mod 3$. Така доказахме, че
    \[\delta^\star(q_0,\beta 0) = q_2\ \Rightarrow\ \bin{\beta 0} \equiv 2 \bmod 3.\]
  \item
    $x = 1$.
    Тогава, по дефиницията на $\delta$, $\delta(q_2,1) = q_2$ и следователно,
    $\delta^\star(q_0,\beta) = q_2$.
    По {\bf И.П.} за (3) с $\beta$,
    \[\delta^\star(q_0,\beta) = q_2\ \Rightarrow\ \bin{\beta} \equiv 2 \bmod 3.\]
    Тогава, $\bin{\beta1} \equiv 2 \mod 3$. Така доказахме, че
    \[\delta^\star(q_0,\beta 1) = q_2\ \Rightarrow\ \bin{\beta 1} \equiv 2 \bmod 3.\]
  \end{itemize}
  \mynote{\writedown Довършете доказателствата на (1) и (2)}
  За да докажем (1), нека $\delta^\star(q_0,\beta x) = q_0$. 
  \begin{itemize}
  \item 
    $x = 0$. Разсъжденията са аналогични, като използваме {\bf И.П.} за (1).
  \item
    $x = 1$. Разсъжденията са аналогични, като използваме {\bf И.П.} за (2).
  \end{itemize}
  
  По същия начин доказваме и (2). Нека $\delta^\star(q_0,\beta x) = q_1$. 
  \begin{itemize}
  \item 
    При $x = 0$, използваме {\bf И.П.} за (3).
  \item
    При $x = 1$, използваме {\bf И.П.} за (1).
  \end{itemize}
  \mynote{\writedown Защо?}
  От (1), (2) и (3) следва директно, че
  \[(\forall\alpha\in\Sigma^\star)(\forall i < 3)[\ \bin{\alpha} \equiv i\ (\bmod\ 3)\ \Leftrightarrow\ \delta^\star(q_0,\alpha) = q_i\ ],\]
  откъдето получаваме, че $\L(\A) = L$.
\end{proof}



%%% Local Variables:
%%% mode: latex
%%% TeX-master: "../eai"
%%% End:
