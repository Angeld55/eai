\section{Минимален автомат}

Нека да започнем като въведем няколко означения.
Нека $\alpha$ е дума над азбуката $\Sigma$  и $L$ е език. Означаваме 
\[\alpha^{-1}L = \{\omega \in \Sigma^\star \mid \alpha\omega \in L\}.\]
Освен това, да означим 
\[\alpha L = \{\alpha\omega \in \Sigma^\star \mid \omega \in L\}.\]
Имаме свойството, че
\[L = \{\omega \in \Sigma^\star \mid \varepsilon \in \omega^{-1}L\}.\]

\begin{prop}
  За всеки две думи $\alpha, \beta \in \Sigma^\star$ е изпълнено, че 
  \[(\alpha\cdot\beta)^{-1}L = \beta^{-1}(\alpha^{-1}L).\]
\end{prop}

\subsection{Минимален автомат по даден регулярен език}
Да разгледаме езика $L = \L(\mathbf{((a+b)^+\cdot a)^\star})$.
\marginpar{
Удобно е да представим 
\begin{align*}
  L & = \{\varepsilon\} \cup \{a,b\}^+ a L\\
  & = \{\varepsilon\} \cup a\{a,b\}^\star aL \cup b\{a,b\}^\star aL
\end{align*}}

Да видим как можем директно да построим минимален тотален детерминиран автомат $\A$ разпознаващ $L$, където
\[\A = \pair{Q^\A,\Sigma,s,\delta_\A,F^\A}.\]
Състоянията на автомата ще бъдат от вида $q_B$, където $B \subseteq \Sigma^\star$, така че накрая искаме да имаме свойството
\[B = \{\omega \in \Sigma^\star \mid \delta^\star_\A(q_B,\omega) \in F^\A\}.\]
Тогава началното състояние ще бъде $q_L$, защото 
\[L = \L(\A) = \{\omega \in \Sigma^\star \mid \delta^\star_\A(q_L,\omega) \in F^\A\}.\]
Сега едновременно ще строим състоянията на автомата $Q^\A$ и функцията на преходите $\delta_\A$.
\begin{itemize}
\item 
\marginpar{
  Удобно е да представим
  \begin{align*}
    M & \df \{a,b\}^\star aL\\
    & = aL \cup \{a,b\}^+aL\\
    & = aL \cup aM \cup bM
  \end{align*}
  Тогава 
  \begin{align*}
    L & = \{\varepsilon\} \cup aM \cup bM
  \end{align*}
  Освен това,
  \begin{align*}
    N & \df L \cup M \\
    & = \{\varepsilon\} \cup aM \cup bM \cup M\\
    & = \{\varepsilon\} \cup aM \cup bM \cup aL\\
  \end{align*}}
  $a^{-1}L = \{a,b\}^\star aL \df M$.
  Имаме, че $M \neq L$, защото $\varepsilon \in L$, но $\varepsilon \not\in M$.
  Понеже $M \neq L$, имаме ново състояние $q_M$ в автомата и 
  $\delta_\A(q_L,a) = q_M$.
\item
  $b^{-1}L = \{a,b\}^\star aL = M$;
  Следователно, $\delta_\A(q_L,b) = q_M$.
\item
  $a^{-1}M = L \cup M \df N$. 
  Имаме, че $N \neq L$, защото $a\in N$, но $a \not\in L$.
  Освен това, $N \neq M$, защото $\varepsilon \in N$, но $\varepsilon \not\in M$.
  Понеже $N \neq L, M$, имаме ново състояние $q_N$ в автомата и 
  $\delta_\A(q_M,a) = q_N$.
\item
  $b^{-1}M = M$. Следователно, $\delta_\A(q_M,b) = q_M$.
\item
  $a^{-1}N = L \cup M = N$. Следователно, $\delta_\A(q_N,a) = q_N$.
\item
  $b^{-1}N = M$.
  Следователно, $\delta_\A(q_M,b) = q_M$.
\end{itemize}
От горните сметки следва, че 
\[(\forall \omega \in \Sigma^\star)[\omega^{-1}L \in \{L, N, M\}].\]
Така получихме, че 
\[Q^\A = \{q_L, q_N, q_M\}.\]
Нека сега да съобразим кои са финалните състояния.
Понеже имаме свойството 
\[L = \{\omega \in \Sigma^\star \mid \varepsilon \in \omega^{-1}L\},\]
то следва, че финалните състояния на автомата $\A$ са  $q_L$ и $q_M$,
защото $\varepsilon \in L,M$. 
Сега вече сме готови да нарисуваме картинка на автомата.

\begin{figure}[H]
  \begin{subfigure}[b]{.4\textwidth}
    \begin{tikzpicture}[->,>=stealth,thick,node distance=55pt]
      \tikzstyle{every state}=[circle,minimum size=20pt,auto]
      
      \node[initial above, state,accepting]   (L) {$q_L$};
      \node[state]                            (M) [right of=L]{$q_M$};
      \node[state,accepting]                  (N) [right of=M]{$q_N$};
      
      
      \path 
      (L) edge [bend left=15] node [above] {$a,b$}   (M)
      (M) edge [loop above] node [above] {$b$} (M)
      (M) edge [bend left=15] node [above] {$a$} (N)
      (N) edge [bend left=15] node [below] {$b$} (M)
      (N) edge [loop above] node [above] {$a$} (N);
    \end{tikzpicture}
    \caption{Минимален автомат за езика $\L(\mathbf{((a+b)^+a)^\star})$}
  \end{subfigure}
  \quad
  ~
  \quad
  \begin{subfigure}[b]{.4\textwidth}
    \begin{tikzpicture}[->,>=stealth,thick,node distance=55pt]
      \tikzstyle{every state}=[circle,minimum size=20pt,auto]
      
      \node[initial above, state,accepting]   (L) {$[\varepsilon]_L$};
      \node[state]                            (M) [right of=L]{$[a]_L$};
      \node[state,accepting]                  (N) [right of=M]{$[aa]_L$};
      
      
      \path 
      (L) edge [bend left=15] node [above] {$a,b$}   (M)
      (M) edge [loop above] node [above] {$b$} (M)
      (M) edge [bend left=15] node [above] {$a$} (N)
      (N) edge [bend left=15] node [below] {$b$} (M)
      (N) edge [loop above] node [above] {$a$} (N);
    \end{tikzpicture}
    \caption{Минимален автомат за езика $\L(\mathbf{((a+b)^+a)^\star})$}
  \end{subfigure}
\end{figure}

Остава да се уверим, че нашата конструкция е коректна, т.е. наистина автоматът $\A$ е минимален за езика $L$.
Според \hyperref[th:myhill-nerode]{Теоремата на Майхил-Нероуд}, 
състоянията на минималния автомат $\M$, разпознаващ $L$, са класовете на еквивалентност на релацията $\approx_L$.
Да разгледаме изображението $f$ с дефиниционна област $\Sigma^\star/_{\approx_L} = \{[\alpha]_L \mid \alpha \in \Sigma^\star\}$, дефинирано като:
\[f([\alpha]_L) = q_K, \text{ където }K = \alpha^{-1}L.\]
\begin{itemize}
\item 
  Ако $[\alpha]_L \neq [\beta]_L$, то $\alpha^{-1}L \neq \beta^{-1}L$ и оттук следва, че $f([\alpha]_L) \neq f([\beta]_L)$.
  Това означава, че $f$ е инективна.
\item
  Да разгледаме $q_K \in Q^\A$. От конструкцията на автомата $\A$ следва, че съществува дума $\alpha \in \Sigma^\star$,
  за която $K = \alpha^{-1}L$. Това означава, че $f([\alpha]_L) = q_K$.
  Следователно, $f$ е сюрективна.
\item
  Имаме и свойството:
  \begin{align*}
    f(\delta_\M([\alpha]_L,x)) & = f([\alpha\cdot x]_L) & (\text{деф. на }\delta_\M)\\
    & = q_N & (N = (\alpha\cdot x)^{-1}L = x^{-1}(\alpha^{-1}L))\\
    & = \delta_\A(q_M, x) & (M = \alpha^{-1}L)\\
    & = \delta_\A(f([\alpha]_L), x),
  \end{align*}
  от което следва, че $f$ е биекция.
\end{itemize}

Според горните разсъждения, 
\begin{align*}
  Q^\M & = \{f(q_L),f(q_M),f(q_N)\} = \{[\varepsilon]_L, [a]_L, [aa]_L\},\\
  F^\M & = \{f(q_L), f(q_N)\} = \{[\varepsilon]_L, [aa]_L\}.
\end{align*}

Проверете, че наистина $L = [\varepsilon]_L \cup [aa]_L$.

\begin{example}
  Да разгледаме езика $L = \L(\mathbf{a\cdot(a+b)^\star\cdot b})$.
  \begin{itemize}
  \item 
    $a^{-1}L = \{a,b\}^\star b = M$;
  \item
    $b^{-1}L = \emptyset$;
  \item
    $a^{-1}M = \{b\} \cup \{a,b\}^\star b = N$;
  \item
    $b^{-1}M = \{\varepsilon\} \cup \{b\} \cup \{a,b\}^\star b = P$;
  \item
    $a^{-1}N = \{b\} \cup \{a,b\}^\star b = M$;
  \item
    $b^{-1}N = \{\varepsilon\} \cup \{b\} \cup \{a,b\}^\star b = P$;
  \item
    $a^{-1}P = \{b\} \cup \{a,b\}^\star b = N$;
  \item
    $b^{-1}P = \{\varepsilon\} \cup \{b\} \cup \{a,b\}^\star b = P$.
  \end{itemize}

  \begin{figure}[H]
    \centering
    \begin{tikzpicture}[->,>=stealth,thick,node distance=55pt]
      \tikzstyle{every state}=[circle,minimum size=20pt,auto]
      
      \node[initial, state]                   (L) {$q_L$};
      \node[state]                            (M) [above right of=L]{$q_M$};
      \node[state]                            (E) [below right of=L]{$q_\emptyset$};
      \node[state]                            (N) [right of=M]{$q_N$};
      \node[state, accepting]                 (P) [below of=N]{$q_P$};
      
      
      \path 
      (L) edge [bend left=15]  node [above] {$a$} (M)
      (L) edge [bend right=15] node [above] {$b$} (E)
      (E) edge [loop right]    node [right] {$a,b$} (E) 
      (M) edge [bend right=15] node [below] {$a$} (N)
      (N) edge [bend right=30] node [above] {$a$} (M)
      (M) edge [bend right=15] node [below] {$b$} (P)
      (N) edge [bend left=15]  node [right] {$b$} (P)
      (P) edge [bend left=15]  node [left]  {$a$} (N)
      (P) edge [loop right]    node [right] {$b$} (P);      
    \end{tikzpicture}
    \caption{Минимален автомат за $\L(\mathbf{a\cdot (a+b)^\star\cdot b})$}
  \end{figure}
\end{example}

\begin{example}
  Да разгледаме езика 
  \[L = \{\omega \in \{a,b\}^\star \mid \omega \text{ съдържа четен брой $a$ и точно едно $b$}\}.\]
  Нека да видим дали можем да построим автомат за този език.
  \begin{itemize}
  \item 
    $a^{-1}L \df M$ е езика съставен от думите с нечетен брой $a$ и точно едно $b$;
  \item 
    $b^{-1}L \df N$ е езика съставен от думите с четен брой $a$ и нито едно $b$;
  \item
    $a^{-1}M = L$;
  \item
    $b^{-1}M \df P$ е езика съставен от думите с нечетен брой $a$ и нито едно $b$;
  \item
    $a^{-1}N = P$;
  \item
    $b^{-1}N = \emptyset$;
  \item
    $a^{-1}P = N$;
  \item
    $b^{-1}P = \emptyset$;
  \end{itemize}

  \begin{figure}[H]
    \centering
    \begin{tikzpicture}[->,>=stealth,thick,node distance=55pt]
      \tikzstyle{every state}=[circle,minimum size=20pt,auto]
      
      \node[initial, state]        (L) {$q_L$};
      \node[state]                 (M) [above right of=L]{$q_M$};
      \node[state, accepting]      (N) [below right of=M]{$q_N$};
      \node[state]                 (P) [right of=M]{$q_P$};
      \node[state]                 (E) [right of=N]{$q_\emptyset$};
            
      \path 
      (L) edge [bend right=15]  node [below] {$a$} (M)
      (M) edge [bend right=15]  node [above] {$a$} (L)
      (L) edge [bend right=15] node [above] {$b$} (N)
      (M) edge [bend left=15] node [above] {$b$} (P)
      (N) edge [bend left=15] node [left] {$a$} (P)
      (P) edge [bend left=15] node [right] {$a$} (N)
      (P) edge [bend left=15] node [right] {$b$} (E)
      (N) edge [bend right=15] node [below] {$b$} (E);
    \end{tikzpicture}
    \caption{Минимален автомат, който приема думи с четен брой $a$ и точно едно $b$}
  \end{figure}  
\end{example}


\begin{example}
  Да разгледаме езика $L = \{a^nb^n\mid n \in \Nat\}$. Да се опитаме да построим автомат, който го разпознава.
  Нека да означим $L_k = \{a^nb^{n+k}\mid n \in \Nat\}$. Да видим какво се получава като приложим процедурата за строене 
  на минимален автомат.
  \begin{itemize}
  \item 
    $a^{-1}L = L_1$;
  \item
    $b^{-1}L = \emptyset$;
  \item
    $a^{-1}L_1 = L_2$;
  \item
    $b^{-1}L_1 = \{\varepsilon\}$;
  \item
    $a^{-1}\{\varepsilon\} = b^{-1}\{\varepsilon\} = \emptyset$;
  \item
    Вижда се, че $a^{-1}L_k = L_{k+1}$, за всяко $k$.
  \item
    Вижда се, че $b^{-1}L_{k+1} = \{b^k\}$, за всяко $k$.
    Освен това е ясно, че $b^{-1}\{b^{k}\} = \{b^{k-1}\}$, за всяко $k \geq 1$.
  \end{itemize}
  Получаваме, че езикът $L$ се разпознава от автомат с {\em безкрайно много състояния}.
  
  \begin{figure}[H]
    \centering
    \begin{tikzpicture}[->,>=stealth,thick,node distance=55pt]
      \tikzstyle{every state}=[circle,minimum size=15pt,auto]
      
      \node[initial, state]                   (0) {$L$};
      \node[state]                            (1) [right of=0]{$L_1$};
      \node[state]                            (2) [right of=1]{$L_2$};
      \node[state]                            (3) [right of=2]{$L_3$};
      \node[state]                            (A) [below of=1]{$\emptyset$};
      \node[state]                            (B) [below right of=1]{$\{b\}$};
      \node[state]                            (BB) [below right of=2]{$\{bb\}$};
      \node[state, accepting]                 (E) [below of=A]{$\{\varepsilon\}$};
      
      \coordinate[right of=3] (4);
      \coordinate[below right of=3] (BBB);
      \coordinate[below of=4] (BBBA);

      \path 
      (0) edge [bend left=15]   node [above] {$a$} (1)
      (1) edge [bend left=15]   node [above] {$a$} (2)
      (2) edge [bend left=15]   node [above] {$a$} (3)
      (0) edge [bend right=30]  node [left] {$b$} (E)
      (E) edge [loop left]      node [left] {$a,b$} (E)
      (1) edge [bend right=30]  node [left] {$b$} (E)
      (2) edge [bend right=15]  node [left] {$b$} (B)
      (3) edge [bend right=15]  node [left] {$b$} (BB)
      (B) edge [bend right=15]  node [above] {$b$} (A)
      (B) edge [bend left=15]  node [right] {$a$} (E)
      (A) edge [bend right=15]   node [right] {$a,b$} (E)
      (BB) edge [bend right=15] node [above] {$b$} (B)
      (BB) edge [bend left=15]  node [below] {$a$} (E);
      
      \draw [dashed,->,shorten >=0pt] (3) to[bend left=15] node[auto] {$a$} (4);
      \draw [dashed,->,shorten >=0pt] (BBB) to[bend right=15] node[above] {$b$} (BB);
      \draw [dashed,->,shorten >=0pt] (BBBA) to[bend left=30] node[below] {$a$} (E);
    \end{tikzpicture}
    \caption{Получаваме {\em безкраен} автомат за $\{a^nb^n \mid n \in \Nat\}$}
  \end{figure}    
\end{example}

\subsection{Проверка за регулярност на език}

  \begin{prop}
    Езикът $L$ е регулярен точно тогава, когато релацията $\approx_L$ има {\em крайно много} класове на еквивалентност.
  \end{prop}
\begin{proof}
  Ако $L$ е регулярен, то той се разпознава от някой ДКА $\A$, който има крайно много състояния 
  и следователно крайно много класове на еквивалентност относно $\sim_\A$.
  Релацията $\approx_L$ е по-груба от $\sim_\A$ и има по-малко класове на еквивалентност.
  Следователно, $\approx_L$ има крайно много класове на еквивалентност.
  
  За другата посока, ако $\approx_L$ има крайно много класове на еквивалентност, то можем да 
  построим ДКА $\A$ както в доказателството на \hyperref[th:myhill-nerode]{Теоремата на Майхил-Нероуд}, който разпознава $L$.
\end{proof}

Това следствие ни дава още един начин за проверка дали даден език е регулярен.
За разлика от \Lem{pumping-reg}, сега имаме {\bf необходимо и достатъчно условие}.
При даден език $L$, ние разглеждаме неговата релация $\approx_L$.
Ако тя има крайно много класове, то езикът $L$ е регулярен.
В противен случай, езикът $L$ не е регулярен.

\begin{example}
  За езика $L = \{a^nb^n\mid n \in \Nat\}$ имаме, че $\abs{\approx_L} = \infty$,
  защото \[(\forall k,j\in\Nat)[k \neq j \implies [a^kb]_L \neq [a^jb]_L].\]
  Проверете, че $[a^kb]_L = \{a^kb,a^{k+1}b^{2},\dots,a^{k+l}b^{l+1},\dots\}$.
  Така получаваме, че релацията $\approx_L$ има безкрайно много класове на еквивалентност.
  Заключаваме, че този език {\bf не} е регулярен.
\end{example}

\begin{example}
  За езика $L = \{a^{n^2} \mid n \in \Nat\}$ имаме, че $\abs{\approx_L} = \infty$,
  защото \[(\forall m,n\in\Nat)[m \neq n \implies [a^{n^2}]_L \neq [a^{m^2}]_L].\]
  
  Без ограничение на общността, да разгледаме $n < m$ и думата $\gamma = a^{2n+1}$.
  Тогава $a^{n^2}\gamma = a^{(n+1)^2} \in L$, но 
  $m^2 < m^2 + 2n + 1 < (m+1)^2$ и следователно $a^{m^2}\gamma = a^{m^2+2n+1}\not\in L$.
\end{example}

\begin{example}
  За езика $L = \{a^{n!} \mid n \in \Nat\}$ имаме, че $\abs{\approx_L} = \infty$,
  защото \[(\forall m,n\in\Nat)[m \neq n \implies [a^{n!}]_L \neq [a^{m!}]_L].\]
  
  Без ограничение на общността, да разгледаме $n < m$ и думата $\gamma = a^{(n!)n}$.
  Тогава $a^{n!}\gamma = a^{(n+1)!} \in L$, но 
  $m! < m! + (n!)n < m! + (m!)m = (m+1)!$ и следователно $a^{m!}\gamma = a^{m!+(n!)n}\not\in L$.
\end{example}

\begin{problem}
  Да разгледаме езика
  \[L = \{a^{f_n} \mid f_0 = f_1 = 1\ \&\ f_{n+2} = f_{n+1} + f_{n}\}.\]
  Докажете, че $\abs{\approx_L} = \infty$.
\end{problem}

%\marginpar{\href{http://en.wikipedia.org/wiki/DFA_minimization}{Уикипедия}}

\subsection{Релация на Майхил-Нероуд}

\begin{itemize}
\item
  \index{Майхил-Нероуд!релация}
  \marginpar{$\approx_L$ е известна като релация на Майхил-Нероуд}
  Нека $L \subseteq \Sigma^\star$ е език и нека $\alpha,\beta \in \Sigma^\star$.
  Казваме, че $\alpha$ и $\beta$ са {\bf еквивалентни относно} $L$, което записваме 
  като $\alpha \approx_L \beta$, когато:
  \[\alpha \approx_L \beta \dff \alpha^{-1}L = \beta^{-1}L.\]
  С други думи, 
  \[\alpha \approx_L \beta \iff (\forall \omega \in \Sigma^\star)[\alpha\omega \in L \iff \beta\omega \in L].\]
\item
  \marginpar{Трябва ли $\A$ да е тотален?}
  Нека $\A = \FA$ е ДКА.
  Казваме, че две думи $\alpha,\beta \in \Sigma^\star$ са {\bf еквивалентни относно $\A$},
  което означаваме с $\alpha \sim_\A \beta$, ако 
  \[\delta^\star(s,\alpha) = \delta^\star(s,\beta).\]
\item
  Проверете, че $\approx_L$ и $\sim_\A$ са {\bf релации на еквивалентност}, т.е.
  те са рефлексивни, транзитивни и симетрични.
\item
  Класът на еквивалентност на думата $\alpha$ относно релацията $\approx_L$ означаваме като
  \[[\alpha]_L \df \{\beta \in \Sigma^\star \mid \alpha \approx_L \beta\}.\]
  Означаваме 
  \[\Sigma^\star/_{\approx_L} \df \{[\alpha]_L \mid \alpha \in \Sigma^\star\}.\]
  Тогава с $\abs{\Sigma^\star/_{\approx_L}}$ ще означаваме броя на класовете на еквивалентност на релацията $\approx_L$.
\item
  Класът на еквивалентност на думата $\alpha$ относно релацията $\sim_\A$ означаваме като
  \[[\alpha]_\A \df \{\beta \in \Sigma^\star \mid \alpha \sim_\A \beta\}.\]
  Означаваме 
  \[\Sigma^\star/_{\sim_\A} \df \{[\alpha]_\A \mid \alpha \in \Sigma^\star\}.\]
  С $\abs{\Sigma^\star/_{\sim_\A}}$ ще означаваме броя на класовете на еквивалентност на релацията $\sim_\A$.
\item
  Съобразете, че всяко състояние на $\A$, което е достижимо от началното състояние, определя клас на еквивалентност относно 
  релацията $\sim_\A$. Това означава, че ако за всяка дума означим  $q_\alpha = \delta^\star_\A(s,\alpha)$, то
  $\alpha \sim_\A \beta$ точно тогава, когато $q_\alpha = q_\beta$. Заключаваме, че броят на класовете на еквивалентност
  на $\sim_\A$ е равен на броя на достижимите от $s$ състояния. Следователно,
  \[|\Sigma^\star/_{\sim_\A}| \leq |Q^\A|.\]
\item
  Релациите $\approx_\L$ и $\sim_\A$ са дясно-инвариантни, т.е. за всеки две думи $\alpha$ и $\beta$
  е изпълнено:
  \begin{align*}
    \alpha \sim_\A \beta  &\implies (\forall \gamma\in\Sigma^\star)[\alpha\gamma \sim_\A \beta\gamma],\\
    \alpha \approx_\L \beta & \implies (\forall \gamma\in\Sigma^\star)[\alpha\gamma \approx_\L \beta\gamma].
  \end{align*}
\end{itemize}



\begin{prop}
  \label{pr:rel-finer}
  За всеки ДКА $\A = \FA$ е изпълнено:
  \[(\forall \alpha,\beta \in \Sigma^\star)[\alpha\sim_\A\beta \implies \alpha\approx_{\L(\A)}\beta].\]
  С други думи, 
  $[\alpha]_\A \subseteq [\alpha]_{\L(\A)}$, за всяка дума $\alpha \in \Sigma^\star$.
\end{prop}
\begin{proof}
%  \marginpar{стр. 95 от \cite{papadimitriou}}
  Да означим за всяка дума $\alpha$, $q_\alpha = \delta^\star_\A(s, \alpha)$.
  Лесно се съобразява, че за всеки две думи $\alpha$ и $\beta$ имаме 
  \begin{align*}
    \alpha \sim_\A \beta & \iff \delta^\star(s,\alpha) = \delta^\star(s,\beta) & (\text{по деф. на }\sim_\A)\\
    & \iff q_\alpha = q_\beta.
  \end{align*}
  Нека $\alpha \sim_\A \beta$. Ще проверим, че  $\alpha \approx_{\L(\A)} \beta$.
  За произволно $\gamma \in \Sigma^\star$ имаме:
  \begin{align*}
    \alpha\gamma \in \L(\A) & \iff \delta^\star(s,\alpha\gamma)\in F & (\text{по деф. на }\L(\A))\\
    & \iff \delta^\star(\delta^\star(s,\alpha),\gamma) \in F & (\text{по деф. на }\delta^\star)\\
    & \iff \delta^\star(q_\alpha, \gamma) \in F & (q_\alpha = \delta^\star(s,\alpha))\\
    & \iff \delta^\star(q_\beta, \gamma) \in F & (q_\alpha = q_\beta, \text{ защото }\alpha \sim_\A \beta)\\
    & \iff \delta^\star(\delta^\star(s,\beta),\gamma) \in F & (q_\beta = \delta^\star(s,\beta))\\
    & \iff \delta^\star(s,\beta\gamma) \in F & (\text{по деф. на }\delta^\star)\\
    & \iff \beta\gamma \in \L(\A) & (\text{по деф. на }\L(\A)).
  \end{align*}
  Заключаваме, че 
  \[(\forall \alpha,\beta \in \Sigma^\star)[\alpha\sim_\A\beta \implies \alpha\approx_{\L(\A)}\beta].\]
\end{proof}

\begin{problem}
  Докажете, че за всеки тотален ДКА $\A$ и всяка дума $\alpha$,
  \[[\alpha]_{\L(\A)} = \bigcup_{\beta \in [\alpha]_{\L(\A)}}[\beta]_\A.\]
\end{problem}


\begin{cor}
  \label{cor:approx-less-sim}
  За всеки тотален ДКА $\A$ е изпълнено, че
  \[\abs{\Sigma^\star/_{\approx_{\L(\A)}}} \leq \abs{\Sigma^\star/_{\sim_\A}}.\]
\end{cor}
\begin{hint}
  Да означим $L = \L(\A)$ и да разгледаме изображението 
  \[f([\alpha]_L) \df \{[\gamma]_\A \mid \gamma \approx_L \alpha\}.\]

  \begin{itemize}
  \item
    Ясно е, че за всяко $\alpha$, $f([\alpha]_L) \neq \emptyset$.
  \item 
    Съобразете, че $f$ е {\bf функция}, т.е. 
    \[(\forall\alpha,\beta\in\Sigma^\star)[\alpha \approx_L \beta \implies f([\alpha]_L) = f([\beta]_L)].\]
  \item
    Използвайте \Prop{rel-finer} за да съобразите, че 
    \[(\forall\alpha,\beta\in\Sigma^\star)[\alpha \not\approx_L \beta \implies f([\alpha]_L) \cap f([\beta]_L) = \emptyset].\]
  \item
    Заключете, че \[\abs{\Sigma^\star/_{\approx_{\L(\A)}}} \leq \abs{\Sigma^\star/_{\sim_\A}}.\]
  \end{itemize}
\end{hint}

\begin{cor}
  \label{cor:upper-bound}
  Нека $L$ е произволен регулярен език $L$.  
  Всеки тотален ДКА $\A$, който разпознава $L$ има свойството
  \[\abs{\Sigma^\star/_{\approx_L}} \leq \abs{Q},\]
  т.е. броят на класовете на еквивалентност на релацията $\approx_L$
  не надвишава броя на състоянията на автомата.
\end{cor}
\begin{proof}
  Да изберем $\A$, който разпознава $L$, бъде такъв, че да {\em няма недостижими състояния}.
  Тъй като всяко достижимо състояние определя клас на еквивалентност относно $\sim_\A$,
  то получаваме, че $\abs{Q} = \abs{\sim_\A}$.
  Комбинирайки със \Cor{approx-less-sim},
  \[\abs{Q} = \abs{\Sigma^\star/_{\sim_\A}} \geq \abs{\Sigma^\star/_{\approx_L}}.\]
\end{proof}
Така получаваме {\em долна граница} за броя на състоянията в тотален автомат разпознаващ езика $L$.
Този брой е не по-малък от броя на класовете на еквивалентност на $\approx_L$.
В следващия раздел ще видим, че тази долна граница може да бъде достигната.

\subsection{Теорема за съществуване на минимален автомат}

\begin{thm}[Майхил-Нероуд]
  \label{th:myhill-nerode}
  \index{Майхил-Нероуд!теорема}
  % \index{Майхил}
  % \index{Нероуд}
  \marginpar{на англ. Myhill-Nerode}
  Нека $L\subseteq \Sigma^\star$ е регулярен език.
  Тогава съществува ДКА $\A = \FA$, който разпознава $L$,
  с точно толкова състояния, колкото са класовете на еквивалентност на релацията $\approx_L$,
  т.е. $\abs{Q} = \abs{\Sigma^\star/_{\approx_L}}$.
\end{thm}
\begin{proof}
%  \marginpar{стр. 96 от \cite{papadimitriou}}
  Да фиксираме регулярния език $L$.
  Ще дефинираме тотален ДКА $\A = \FA$, разпознаващ $L$, като:
  \begin{itemize}
  \item
    $Q = \{[\alpha]_L\mid \alpha\in \Sigma^\star\}$;
  \item
    $s = [\varepsilon]_L$;
  \item
    $F = \{[\alpha]_L\mid \alpha\in L\} = \{[\alpha]_L \mid [\alpha]_L \cap L \neq \emptyset\}$;
  \item
    Определяме изображението $\delta$ като 
    за всяка буква $x \in \Sigma$ и всяко състояние $[\alpha]_L\in Q$, 
    \[\delta([\alpha]_L,x) = [\alpha x]_L.\]
  \end{itemize}
  
  Първо, трябва да се уверим, че множеството от състояния $Q$ е крайно, т.е.
  релацията $\approx_L$ има крайно много класове на еквивалентност.
  И така, тъй като $L$ е регулярен език, то той се разпознава от някой тотален ДКА $\A'$.
  От \Cor{upper-bound} имаме, че $\abs{Q^{\A'}} \geq \abs{\Sigma^\star/_{\approx_L}}$.
  Понеже $Q^{\A'}$ е крайно множество, то $\approx_L$ има крайно много класове и 
  следователно $Q$ също е крайно множество.

  Второ, трябва да се уверим, че изображението $\delta$ задава функция, т.е. 
  да проверим, че за всеки две думи $\alpha$, $\beta$ и всяка буква $x$,
  \[[\alpha]_L = [\beta]_L \implies \delta([\alpha]_L,x) = \delta([\beta]_L,x).\]
  Но това се вижда веднага, защото от определението на релацията $\approx_L$ следва, че
  ако $\alpha \approx_L \beta$, то за всяка буква $x$, $\alpha x \approx_L \beta x$,
  т.е. $[\alpha x]_L = [\beta x]_L$ и 
  \begin{align*}
    [\alpha]_L = [\beta]_L & \implies [\alpha x]_L = [\beta x]_L & (\text{свойство на }\approx_L)\\
    & \implies \delta([\alpha]_L,x) = [\alpha x]_L = [\beta x]_L = \delta([\beta]_L,x) & (\text{деф. на }\delta)
  \end{align*}
  
  Така вече сме показали, че $\A$ е коректно зададен тотален ДКА.
  Остава да покажем, че $\A$ разпознава езика $L$, т.е. $\L(\A) = L$.
  За целта, първо ще докажем едно помощно твърдение.
  \begin{prop}
    За всеки две думи $\alpha$ и $\beta$,
    $\delta^\star([\alpha]_L,\beta) = [\alpha\beta]_L$.
  \end{prop}
  \begin{proof}
    Ще докажем това свойство с индукция по дължината на $\beta$.
    \begin{itemize}
    \item
      За $\beta = \varepsilon$ свойството следва директно от дефиницията на $\delta^\star$ като рефлексивно и транзитивно затваряне на $\delta$,
      защото $\delta^\star([\alpha]_L,\varepsilon) = [\alpha]_L$.
    \item
      Нека $\abs{\beta} = n+1$ и да приемем, че сме доказали твърдението за думи с дължина $n$.
      Тогава $\beta = \gamma a$, където $\abs{\gamma} = n$. Свойството следва от следните равенства:
      \begin{align*}
        \delta^\star([\alpha]_L, \gamma a) & = \delta(\delta^\star([\alpha]_L,\gamma),a) & (\text{деф. на }\delta^\star)\\
                                          & = \delta([\alpha\gamma]_L,a) & (\text{от {\bf И.П.} за }\gamma)\\
                                          & = [\alpha\gamma a]_L & (\text{от деф. на }\delta)\\
                                          & = [\alpha\beta]_L & (\beta = \gamma a).
      \end{align*}
    \end{itemize}
  \end{proof}
  \noindent За да се убедим, че $L = \L(\A)$ е достатъчно да проследим еквивалентностите:
  \begin{align*}
    \alpha\in \L(\A) & \iff \delta^\star(s,\alpha) \in F & (\text{от деф. на }\L(\A))\\
                     & \iff \delta^\star([\varepsilon]_L,\alpha) \in F & (\text{по деф. }s = [\varepsilon]_L)\\
                     & \iff \delta^\star([\varepsilon]_L,\alpha) = [\alpha]_L\ \&\ \alpha\in L & (\text{от деф. на }F)\\
                     & \iff \alpha \in L & (\text{от последното твърдение}).
  \end{align*}
\end{proof}

\begin{dfn}
  \index{изоморфизъм}
  Нека $\A_1 = \FAn{1}$ и $\A_2 = \FAn{2}$.
  Казваме, че $\A_1$ и $\A_2$ са {\bf изоморфни}, което означаваме с $\A_1 \cong \A_2$, ако
  съществува биекция $f: Q_1\to Q_2$, за която:
  \begin{itemize}
  \item
    $f(s_1) = s_2$;
  \item
    $q \in F_1 \iff f(q) \in F_2$;
  \item
    $(\forall a\in\Sigma)(\forall q\in Q_1)[f(\delta_1(q,a)) = \delta_2(f(q),a)]$.
  \end{itemize}
  Ще казваме, че $f$ задава изоморфизъм на $\A_1$ върху $\A_2$.
\end{dfn}

Това означава, че два автомата $\A_1$ и $\A_2$ са изоморфни, ако можем да получим $\A_2$
като преименуваме състоянията на $\A_1$.

\begin{cor}
  Нека е даден регулярния език $L$.
  Всички минимални автомати за $L$ са изоморфни на $\A_0$, автоматът построен в \hyperref[th:myhill-nerode]{Теоремата на Майхил-Нероуд}.
\end{cor}
\begin{proof}
  Нека $\A = \FA$ е произволен тотален автомат, за който $\L(\A) = L$ и $\abs{Q} = \abs{\Sigma^\star/_{\approx_L}}$.
  Съобразете, че $\A$ е {\em свързан}, т.е. всяко състояние на $\A$ е достижимо от началното.
  Искаме да докажем, че $\A \cong \A_0$.
  Понеже $\A$ е свързан, за всяко състояние $q$ можем да намерим дума $\omega_q$,
  за която $\delta^\star(s,\omega_q) = q$.
  Да дефинираме изображението $f:Q\to \Sigma^\star/_{\approx_L}$ като $f(q) = [\omega_q]_L$.
  Ще докажем, че
  $f$ задава изоморфизъм на $\A$ върху $\A_0$. 
  \begin{itemize}
  \item
    Първо да съобразим, че ако $\delta^\star_\A(s,\alpha) = q$, то $[\omega_q]_L = [\alpha]_L$.
    Понеже $\delta^\star_\A(s,\alpha) = q = \delta^\star_\A(s,\omega_q)$, то $\omega_q \sim_\A \alpha$.
    От \Prop{rel-finer} имаме, че
    \[\omega_q \sim_\A \alpha \implies \omega_q \approx_L \alpha.\]
    Това означава, $[\omega_q]_L = [\alpha]_L$ и следователно $f$ е определена коректно, т.е. $f$ е {\bf функция}.
  \item
    Ще проверим, че $f$ е {\bf инективна}, т.е.
    \[(\forall q_1,q_2 \in Q)[q_1\neq q_2 \implies f(q_1) \neq f(q_2)].\]
    Да допуснем, че има състояния $q_1 \neq q_2$, за които 
    \[f(q_1) = [\omega_{q_1}]_L = [\omega_{q_2}]_L = f(q_2).\]
    Тогава $\omega_{q_1} \not\sim_\A \omega_{q_2}$ и $\omega_{q_1} \approx_L \omega_{q_2}$.
    \marginpar{\writedown Обяснете!}
    Но тогава от \Cor{upper-bound} получаваме, че $\abs{\Sigma^\star/_{\sim_\A}} > \abs{\Sigma^\star/_{\approx_L}}$,
    което противоречи с минималността на $\A$.
  \item
    За да бъде $f$ {\bf сюрективна} трябва за всеки клас $[\beta]_L$ да съществува състояние $q$, за което $f(q) = [\beta]_L$.
    Понеже $\A$ е свързан, съществува състояние $q$, за което $\delta^\star_\A(s,\beta) = q$.
    Вече се убедихме, че в този случай $\beta \approx_L \omega_q$, защото $\beta \sim_\A \omega_q$.
    Тогава $f(q) = [\omega_q]_L = [\beta]_L$.
  \item
    За последно оставихме проверката, че $f$ наистина е {\bf изоморфизъм}:
    \begin{align*}
      f(\delta_\A(q,a)) & = f(\delta_\A(\delta^\star_\A(s,\omega_q),a)) & (\text{от избора на }\omega_q)\\
      & = f(\delta^\star_\A(s,\omega_qa)) & (\text{от деф. на }\delta^\star_\A)\\
      & = [\omega_qa]_L & (\text{от деф. на }f)\\
      & = \delta^\star_{\A_0}([\varepsilon]_L, \omega_qa) & (\text{от деф. на }\A_0)\\ 
      & = \delta_{\A_0}(\delta^\star_{\A_0}([\varepsilon]_L, \omega_q),a) & (\text{от деф. на }\delta^\star_{\A_0})\\
      & = \delta_{\A_0}([\omega_q]_L, a) & (\text{свойство на }\delta^\star_{\A_0})\\
      & = \delta_{\A_0}(f(q), a) & ( f(q) = [\omega_q]_L).
    \end{align*}
  \end{itemize}
\end{proof}


%%% Local Variables: 
%%% mode: latex
%%% TeX-master: "EAI"
%%% End: 
