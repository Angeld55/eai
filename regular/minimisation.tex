\section{Алгоритъм за минимизация}
\label{sect:regular:minimisation}
\index{минимизация}

\begin{itemize}
\item
  Да фиксираме произволен детерминиран краен автомат
  \[\A = \FA.\]
\item
  За състояние $p$ в автомата $\A$, да означим с $\L_\A(p)$ езикът, който се разпознава от автомата $\A$,
  ако приемем, че $p$ е началното състояние на автомата, т.е.
  \[\L_\A(p) \df \{\omega \in \Sigma^\star \mid \delta^\star(p,\omega) \in F\}.\]
  В частност, $\L(\A) = \L_\A(\qstart)$.
\item
  \index{$\equiv_\A$}
  \marginpar{Може да е малко объркващо, но релацията $\equiv_\A$ е различна от релацията $\sim_\A$.
  Релацията $\sim_\A$ е върху думи, докато $\equiv_\A$ е върху състояния.}
  Сега дефинираме следната релация между състояния на автомата $\A$:
  \[p \equiv_\A q\ \dff\ \L_\A(p) = \L_\A(q).\]
  Това означава, че $p \equiv_\A q$ точно тогава, когато
  \begin{equation}
    \label{eq:1}
    (\forall \omega\in \Sigma^\star)[\ \delta^\star(p,\omega) \in F\ \iff\ \delta^\star(q,\omega) \in F\ ].
  \end{equation}
\item
  \marginpar{\writedown Защо?}
  Релацията $\equiv_\A$ между състояния на автомата $\A$ е релация на еквивалентност. 
\item
  Да означим
  \[Q/_{\equiv_\A} \df \{\ [q]_{\equiv_\A} \mid q \in Q\ \}.\]

\end{itemize}

\begin{problem}
  Докажете, че
  \[{[q]}_{\equiv_\A} \cap F \neq \emptyset \iff {[q]}_{\equiv_\A} \subseteq F.\]
\end{problem}

\begin{framed}
  \begin{proposition}
    \label{pr:equal-number}
    Нека $\A = \FA$ е детерминиран краен автомат {\em без недостижими от} $\qstart$ състояния.
    Тогава
    \[\abs{Q/_{\equiv_\A}} = \abs{\Sigma^\star/_{\approx_{\L(\A)}}}.\]  
  \end{proposition}  
\end{framed}
\begin{proof}
  Нека $L = \L(\A)$. Да положим $q_\alpha \df \delta^\star(\qstart,\alpha)$
  и да разгледаме $h:\Sigma^\star/_{\approx_L} \to Q/_{\equiv_\A}$, където
  \[h({[\alpha]}_L) \df [q_\alpha]_{\equiv_\A}.\]
  Ще докажем, че $h$ е биекция като използваме свойството, че
  \begin{equation}
    \label{eq:8}
    \alpha^{-1}(L) = \L_\A(q_\alpha),
  \end{equation}
  защото за произволна дума $\gamma$,
  \begin{align*}
    \gamma \in \L_\A(q_\alpha) & \iff \delta^\star(\delta^\star(\qstart,\alpha),\gamma) \in F\\
                                                   & \iff \delta^\star(\qstart,\alpha\gamma) \in F & \comment\text{свойство на }\delta^\star\\
                                                   & \iff \alpha \gamma \in \L(\A) & \comment\text{деф. на }\L(\A)\\
                                                   & \iff \gamma \in \alpha^{-1}(\L(\A))\\
                                                   & \iff \gamma \in \alpha^{-1}(L). & \comment\L(\A) = L
  \end{align*}
  Сега веднага получаваме, че
  \begin{equation}
    \label{eq:9}
    {[\alpha]}_L = {[\beta]}_L \iff h({[\alpha]}_L) = h({[\beta]}_L),
  \end{equation}
  защото
  \begin{align*}
    h({[\alpha]}_L) = h({[\beta]}_L) & \iff {[q_\alpha]}_{\equiv_\A} = {[q_\beta]}_{\equiv_\A} & \comment\text{деф. на }h\\ 
                                     & \iff \L_\A(q_\alpha) = \L_\A(q_\beta) & \comment\text{деф. на }\equiv_\A\\
                                     & \iff \alpha^{-1}(L) = \beta^{-1}(L) & \comment\text{от (\ref{eq:8})}\\
                                     & \iff {[\alpha]}_L = {[\beta]}_L. & \comment\text{деф. на }\approx_L
  \end{align*}
  
  \begin{itemize}
  \item
    Ясно е, че $h$ е добре дефинирана, т.е. $h$ наистина е функция.
    Това е така, защото от (\ref{eq:9}) имаме, че
    \[{[\alpha]}_L = {[\beta]}_L \implies h({[\alpha]}_L) = h({[\beta]}_L).\]
  \item
    \marginpar{Ако в $\A$ има недостижими от $\qstart$ състояния, то ще получим, че $|\Sigma^\star/_{\approx_L}| \leq |Q/_{\equiv_\A}|$.}
    В $\A$ няма недостижими състояния, откъдето следва, че за всеки клас на еквивалентност ${[q]}_{\equiv_\A}$,
    съществува дума $\alpha$, за която $\delta^\star(\qstart,\alpha) = q$.
    Това означава, че $h([\alpha]_L) = {[q]}_{\equiv_\A}$ и следователно $h$ е сюрекция.
  \item
    Остава да проверим, че $h$ е инективна, т.е.
    \[h({[\alpha]}_L) = h({[\beta]}_L) \implies {[\alpha]}_L = {[\beta]}_L.\]
    Но това следва веднага от (\ref{eq:9}).
  \end{itemize}
\end{proof}

\begin{problem}\label{prob:equiv-delta}
  Докажете, че за произволни $p,q \in Q$ и произволно $a\in\Sigma$,
  \[p \equiv_\A q \implies \delta(p,a) \equiv_\A \delta(q,a).\]
\end{problem}


Нека е даден автомата $A = \FA$.
След като сме намерили релацията $\equiv_\A$ за $\A$, 
строим автомата $\A' = (\Sigma,Q',\qstart',\delta_1,F')$, където:
\begin{itemize}
\item
  $Q' \df \{[q]_{\equiv_\A} \mid q\in Q\}$;
\item
  $\qstart' \df [\qstart]_{\equiv_\A}$;
\item
  $\delta_1([q]_{\equiv_\A}, a) \df [\delta(q,a)]_{\equiv_\A}$;
\item
  $F' \df \{[q]_{\equiv_\A}\mid [q]_{\equiv_\A} \cap F \neq \emptyset\}$;
\end{itemize}

\begin{problem}
  Докажете, че $\delta_1$ е добре дефинирана, т.е. $\delta_1$ е функция.
\end{problem}
\begin{hint}
  Използвайте Задача \ref{prob:equiv-delta}.
\end{hint}


\begin{proposition}
  \label{pr:minimisation-delta-1}
  За всяко състояние $q \in Q$ и дума $\alpha$ е изпълнено, че
  \[\delta^\star_1([q]_{\equiv_\A}, \alpha) = [\delta^\star(q,\alpha)]_{\equiv_\A}.\]
\end{proposition}
\begin{proof}
  Индукция по дължината на думата $\alpha$.
  \begin{itemize}
  \item
    Нека $|\alpha| = 0$, т.е. $\alpha = \varepsilon$. Тогава
    \[\delta^\star_1([q]_{\equiv_\A},\varepsilon) = [q]_{\equiv_\A} = [\delta^\star(q,\varepsilon)]_{\equiv_\A}.\]
  \item
    Да приемем, че твърдението е вярно за думи с дължина $n$.
  \item
    Нека $|\alpha| = n+1$, т.е. $\alpha = \beta a$ и $|\beta| = n$. Тогава
    \begin{align*}
      \delta^\star_1([q]_{\equiv_\A},\beta a) & = \delta_1(\delta^\star_1([q]_{\equiv_\A},\beta),a) & \comment\text{деф. на }\delta^\star_1\\
                                              & = \delta_1([\underbrace{\delta^\star(q,\beta)}_{p}]_{\equiv_\A},a) & \comment\text{И.П. за }\beta\\
                                              & = \delta_1([p]_{\equiv_\A},a) \\
                                              & = [\delta(p,a)]_{\equiv_\A} & \comment\text{деф. на }\delta_1\\
                                             & = [\delta(\delta^\star(q,\beta),a)]_{\equiv_\A} & \comment{p = \delta^\star_1(q,\beta)}\\
                                             & = [\delta^\star(q,\beta a)]_{\equiv_\A}. & \comment\text{деф. на }\delta^\star
    \end{align*}
  \end{itemize}
\end{proof}

\begin{framed}
  \begin{thm}
    $\A'$ е минимален автомат разпознаващ езика $\L(\A)$.
  \end{thm}
\end{framed}
\begin{hint}
  Лесно се съобразява, че $\L(\A) = \L(\A')$, защото
  \begin{align*}
    \alpha \in \L(\A) & \iff \delta^\star(\qstart,\alpha) \in F & \comment\text{деф. на }\L(\A)\\
                      & \iff [\delta^\star(\qstart,\alpha)]_{\equiv_\A} \in F' & \comment\text{деф. на }F'\\
                      & \iff \delta^\star_1([\qstart]_{\equiv_\A},\alpha) \in F' & \comment\text{\Prop{minimisation-delta-1}}\\
                      & \iff \alpha \in \L(\A'). & \comment\text{деф. на }\L(\A')
  \end{align*}
  От \Prop{equal-number} знаем, че $|Q'| = |\Sigma^\star/_{\approx_{\L(\A)}}|$.
  Тогава от \Prop{upper-bound} следва, че $\A'$ е минимален автомат.
\end{hint}

При даден език $L$ и детерминиран краен автомат $\A = \FA$, който го разпознава, целта ни е да построим нов детерминиран краен автомат $\M$,
който има толкова състояния колкото са класовете на еквивалентност на релацията $\approx_L$.
Това ще направим като ,,слеем'' състоянията на $\A$, които са еквивалентни относно релацията $\equiv_\A$.
Според \Prop{equal-number}, това означава, че всяко състояние на $\M$ ще отговаря на един клас на еквивалентност на релацията $\equiv_\A$.
Проблемът с намирането на класовете на еквивалентност на релацията $\equiv_\A$ е кванторът $\forall \omega \in \Sigma^\star$
във нейната дефиниция (чрез Формула \ref{eq:1}), защото $\Sigma^\star$ е безкрайно множество от думи.
За да разрешим този проблем, ще разгледаме апроксимации на езиците $\L_\A(q)$.
За ествено число $n$, да означим 
\[\L^n_\A(p) \df \{\omega \in \Sigma^\star \mid \abs{\omega} \leq n\ \&\ \delta^\star(p,\omega) \in F\}.\]
Лесно се съобразява, че
\marginpar{Можем ли да дадем горна граница на $n$?}
\[L(\A) = \bigcup_{n\geq 0} \L^n_\A(\qstart).\]

\index{$\equiv^n_\A$}
За всяко естествено число $n$, дефинираме бинарните релации $\equiv^n_\A$ върху $Q$ по следния начин:
\[p \equiv^n_\A q \dff \L^n_\A(p) = \L^n_\A(q).\]

Релациите $\equiv^n_\A$ представляват апроксимации на релацията $\equiv_\A$.
Обърнете внимание, че за всяко $n$, $\equiv^n_\A$ е {\em по-груба} релация от $\equiv^{n+1}_\A$, 
която на свой ред е по-груба от $\equiv_\A$.
Алгоритъмът строи $\equiv^n_\A$ докато не срещнем $n$, за което
\[\equiv^n_\A\ =\ \equiv^{n+1}_\A.\]
Тъй като броят на класовете на еквивалентност на $\equiv_\A$ е краен (той е $\leq \abs{Q}$), то 
със сигурност ще намерим такова $n$, за което $\equiv^n_\A\ =\ \equiv^{n+1}_\A$.
Тогава заключаваме, че за това $n$ имаме, че
\[\equiv_\A\ =\ \equiv^n_\A.\]

\marginpar{Ако $q \in F$, то $\L^0_\A(q) = F$ и ако $q \not\in F$, то $\L^0_\A(q) = Q\setminus F$.}
Понеже единствената дума с дължина $0$ e $\varepsilon$ и по определение $\delta^\star(p,\varepsilon) = p$, 
лесно се съобразява, че $\equiv^0_\A$ има два класа на еквивалентност.
Единият е $F$, а другият е $Q\setminus F$.

Вече имаме базовия случай за $n=0$.
Да видим сега как можем да намерим $\equiv^{n+1}_\A$ при положение, че вече сме намерили $\equiv^n_\A$.
\begin{framed}
  \begin{proposition}
    \label{pr:one-letter-test}
    За всеки две състояния $p,q \in Q$, и всяко естествено число $n$, $p \equiv^{n+1}_\A q$ точно тогава, когато
    \begin{enumerate}[a)]
    \item
      $p \equiv^n_\A q$ и
    \item
      $(\forall a \in \Sigma)[\delta(q,a) \equiv^n_\A \delta(p,a)]$.
    \end{enumerate}
  \end{proposition}  
\end{framed}
\begin{hint}
  \marginpar{\cite[стр. 99]{papadimitriou}}
  \begin{align*}
    p \equiv^{n+1}_\A q & \iff \L^{n+1}_\A(p) = \L^{n+1}_\A(q)\\
                     & \iff \L^n_\A(p) = \L^n_\A(q)\ \&\ (\forall a \in \Sigma)[\ \L^n_\A(\delta(p,a)) = \L^n_\A(\delta(q,a))\ ]\\
                     & \iff p \equiv^n_\A q\ \&\ (\forall a \in \Sigma)[\ \delta(p,a) \equiv^n_\A \delta(q,a)\ ].
  \end{align*}
\end{hint}

\begin{problem}
  Докажете, че ако $\equiv^n_\A\ =\ \equiv^{n+1}_\A$, то
  \[(\forall m \geq n)[\ \equiv^n_\A\ =\ \equiv_m\ ].\]
\end{problem}

\begin{problem}
  Докажете, че ако $\equiv^n_\A\ =\ \equiv^{n+1}_\A$, то
  \[\equiv^n_\A\ =\ \equiv_\A.\]
\end{problem}

\begin{problem}
  Докажете, че ако $n \geq |Q|$, то
  \[\equiv^n_\A\ =\ \equiv_\A.\]
\end{problem}


%%% Local Variables:
%%% mode: latex
%%% TeX-master: "../eai"
%%% End:
