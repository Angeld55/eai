\subsection{Алгоритъм за строене на минимален автомат}
\begin{itemize}
\item
  Да фиксираме произволен тотален ДКА $\A = \FA$.
\item
  За състояние $p$ в автомата $\A$, да означим с $\L_\A(p)$ езикът, който се разпознава от автомата $\A$,
  ако приемем, че $p$ е началното състояние на автомата, т.е.
  \[\L_\A(p) \df \{\omega \in \Sigma^\star \mid \delta^\star(p,\omega) \in F\}.\]
  В частност, $\L(\A) = \L_\A(s)$.
\item
  Сега дефинираме следната релация между състояния на автомата $\A$:
  \[p \equiv_\A q\ \dff\ \L_\A(p) = \L_\A(q).\]
  Това означава, че $p \equiv_\A q$ точно тогава, когато
  \begin{equation}
    \label{eq:1}
    (\forall \omega\in \Sigma^\star)[\delta^\star(p,\omega) \in F\ \iff\ \delta^\star(q,\omega) \in F].
  \end{equation}
\item
  Релацията $\equiv_\A$ между състояния на автомата $\A$ е релация на еквивалентност. 
\item
  Нека $q_\alpha$ е състоянието, което съответства на думата $\alpha$ в $\A$, т.е.
  $\delta^\star_\A(s,\alpha) = q_\alpha$. Тогава 
  \[\L_\A(q_\alpha) = \alpha^{-1}\L(\A).\]
  Оттук получаваме, че 
  \begin{align*}
    q_\alpha \equiv_\A q_\beta & \iff \L_\A(q_\alpha) = \L_\A(q_\beta)\\
    & \iff \alpha^{-1}\L(\A) = \beta^{-1}\L(\A)\\
    & \iff \alpha \sim_{\L(\A)} \beta.
  \end{align*}
  % \[q_\alpha \equiv_\A q_\beta\ \iff\ \alpha\approx_{\L(\A)} \beta.\]
  Това означава, че ако в $\A$ няма недостижими състояния от началното състояние $s$, то 
  \[\abs{\Sigma^\star/_{\equiv_\A}} = \abs{\Sigma^\star/_{\approx_{\L(\A)}}}.\]
\end{itemize}

При даден език $L$ и тотален ДКА $\A = \FA$, който го разпознава, нашата цел е да построим нов ДКА $\A_0$,
който има толкова състояния колкото са класовете на еквивалентност на релацията $\approx_\L$.
Това ще направим като ``слеем'' състоянията на $\A$, които са еквивалентни относно релацията $\equiv_\A$.
Това означава, че всяко състояние на $\A_0$ ще отговаря на един клас на еквивалентност на релацията $\equiv_\A$.
Проблемът с намирането на класовете на еквивалентност на релацията $\equiv_\A$ е кванторът $\forall \gamma \in \Sigma^\star$
във нейната дефиниция чрез (Формула \ref{eq:1}), защото $\Sigma^\star$ е безкрайно множество от думи.

Да фиксираме автомата $\A$ и $L = \L(\A)$.
Да означим 
\[\L^n_\A(p) \df \{\omega \in \Sigma^\star \mid \abs{\omega} \leq n\ \&\ \delta^\star(p,\omega) \in F\}.\]
Според тази дефиниция, $L = \bigcup_{n\geq 0} \L^n_\A(s)$.

За всяко естествено число $n$, дефинираме бинарните релации $\equiv_n$ върху $Q$ по следния начин:
\[p \equiv_n q \dff \L^n_\A(p) = \L^n_\A(q).\]

% Алгоритъмът представлява намирането на релации $\equiv_n$, където
% \[p\equiv_n q \iff (\forall\gamma\in\Sigma^\star)[\abs{\gamma}\leq n\ \rightarrow\ (\delta^\star(p,\gamma) \in F\ \iff\ \delta^\star(q,\gamma) \in F)].\]
Релациите $\equiv_n$ представляват апроксимации на релацията $\equiv_\A$.
Обърнете внимание, че за всяко $n$, $\equiv_n$ е {\em по-груба} релация от $\equiv_{n+1}$, 
която на свой ред е по-груба от $\equiv_\A$.
Алгоритъмът строи $\equiv_n$ докато не срещнем $n$, за което $\equiv_n\ =\ \equiv_{n+1}$.
Тъй като броят на класовете на еквивалентност на $\equiv_\A$ е краен (той е $\leq \abs{Q}$), то 
със сигурност ще намерим такова $n$, за което $\equiv_n\ =\ \equiv_{n+1}$.
Тогава заключаваме, че $\equiv_\A\ =\ \equiv_n$.

Понеже единствената дума с дължина $0$ e $\varepsilon$ и по определение $\delta^\star(p,\varepsilon) = p$, 
лесно се съобразява, че $\equiv_0$ има два класа на еквивалентност.
Единият е $F$, а другият е $Q\setminus F$.

\begin{prop}
  \label{pr:one-letter-test}
  За всеки две състояния $p,q \in Q$, и всяко $n$, $p \equiv_{n+1} q$ точно тогава, когато
  \begin{enumerate}[a)]
  \item
    $p \equiv_{n} q$ и
  \item
    $(\forall a \in \Sigma)[\delta(q,a) \equiv_{n} \delta(p,a)]$.
  \end{enumerate}
\end{prop}
\begin{hint}
  \marginpar{\cite[стр. 99]{papadimitriou}}
  \begin{align*}
    p \equiv_{n+1} q & \iff \L^{n+1}_\A(p) = \L^{n+1}_\A(q)\\
    & \iff \L^n_\A(p) = \L^n_\A(q)\ \&\ (\forall a \in \Sigma)[\L^n_\A(\delta(p,a)) = \L^n_\A(\delta(q,a))]\\
    & \iff p \equiv_n q\ \&\ (\forall a \in \Sigma)[\delta(p,a) \equiv_n \delta(q,a)].
  \end{align*}
\end{hint}

Нека е даден автомата $A = \FA$.
След като сме намерили релацията $\equiv_\A$ за $\A$, 
строим автомата $\A' = (Q',\Sigma,s',\delta',F')$, където:
\begin{itemize}
\item
  $Q' = \{[q]_{\equiv_\A} \mid q\in Q\}$;
\item
  $s' = [s]_{\equiv_\A}$;
\item
  $\delta'([q]_{\equiv_\A}, a) = [\delta(q,a)]_{\equiv_\A}$;
\item
  $F' = \{[q]_{\equiv_\A}\mid F\cap [q]_{\equiv_\A} \neq \emptyset\}$;
\end{itemize}

От всичко казано дотук знаем, че $\A'$ е минимален автомат разпознаващ езика $\L(\A)$.

\begin{example}
  Да разгледаме следния краен детерминиран автомат $\A$.
  \begin{figure}[H]
    \begin{subfigure}[b]{.4\textwidth}
      \begin{tikzpicture}[->,>=stealth,thick,node distance=55pt]
        \tikzstyle{every state}=[circle,minimum size=20pt,auto]
        
        \node[initial above, state]   (0) {$0$};
        \node[state]            (1) [above right of=0]{$1$};
        \node[state]            (2) [below right of=0]{$2$};
        \node[state,accepting]  (3) [right of=1]{$3$};
        \node[state,accepting]  (4) [right of=2]{$4$};
        \node[state,accepting]  (5) [below right of=3]{$5$};
        
        
        \path 
        (0) edge  node [above] {$a$}   (1)
        (0) edge  node [below] {$b$}   (2)
        (1) edge node [above] {$a$}    (3)
        (1) edge [bend left=15] node [below] {$b$}    (4)
        (2) edge [bend left=15] node [left] {$b$}    (3)
        (2) edge node [below] {$a$}   (4)
        (4) edge  node [below] {$a,b$} (5)
        (3) edge  node [left] {$a,b$}  (5)
        (5) edge [loop above]   node [above] {$a,b$}  (5);
      \end{tikzpicture}
      \caption{Ще построим минимален автомат разпознаващ $\L(\A)$}
    \end{subfigure}
    \qquad
    \qquad
    \begin{subfigure}[b]{0.5\textwidth}
      \begin{tikzpicture}[->,>=stealth,thick,node distance=45pt]
        \tikzstyle{every state}=[circle,minimum size=20pt,auto,scale=.9]
        
        \node[initial above, state]   (0) {$B_0$};
        \node[state]            (1) [right of=0]{$B_1$};
        \node[state,accepting]  (2) [right of=1]{$B_2$};
        
        \path 
        (0) edge [bend left=15] node [above] {$a,b$}   (1)
        (1) edge [bend left=15] node [above] {$a,b$}   (2)
        (2) edge [loop above] node [above] {$a,b$}   (2);
      \end{tikzpicture}
      \caption{Получаваме следния минимален автомат $\A_0$, $\L(\A_0) = \L(\A)$}
      \label{sub:min1}
    \end{subfigure}
  \end{figure}
  \marginpar{Съобразете, че $\L(\A) = \{\alpha \in \{a,b\}^\star \mid \abs{\alpha} \geq 2\}$.}

Ще приложим алгоритъма за минимизация за да получим минималния автомат за езика $L$.
За всяко $n = 0,1,2,\dots$, ще намерим класовете на еквивалентност на $\equiv_n$,
докато не намерим $n$, за което $\equiv_n\ =\ \equiv_{n+1}$.

\begin{itemize}
\item 
  Класовете на еквивалентност на $\equiv_0$ са два.
  Те са $A_0 = Q\setminus F = \{0,1,2\}$ и $A_1 = F = \{3,4,5\}$.
\item
  Сега да видим дали можем да разбием някои от класовете на еквивалентност на $\equiv_0$.
  
  \begin{tabular}{|c|c|c|c|c|c|c|}
    \hline
    $Q$ & $0$ & $1$ & $2$ & $3^\star$ & $4^\star$ & $5^\star$ \\
    \hline
    \hline
    $\equiv_0$ & $A_0$ & $A_0$ & $A_0$ & $A_1$ & $A_1$ & $A_1$\\
    \hline
    $a$ & $A_0$& $A_1$ & $A_1$ & $A_1$ & $A_1$ & $A_1$\\
    \hline
    $b$ & $A_0$& $A_1$ & $A_1$ & $A_1$ & $A_1$ & $A_1$\\
    \hline
  \end{tabular}

  Виждаме, че $0 \not\equiv_1 1$ и $1 \equiv_1 2$.
  Класовете на еквивалентност на $\equiv_1$ са 
  $B_0 = \{0\}$, $B_1 = \{1,2\}$, $B_2 = \{3,4,5\}$.
\item
  Сега да видим дали можем да разбием някои от класовете на еквивалентност на $\equiv_1$.
  
  \begin{tabular}{|c|c|c|c|c|c|c|}
    \hline
    $Q$ & $0$ & $1$ & $2$ & $3^\star$ & $4^\star$ & $5^\star$ \\
    \hline
    \hline
    $\equiv_1$ & $B_0$ & $B_1$ & $B_1$ & $B_2$ & $B_2$ & $B_2$\\
    \hline
    $a$ & $B_1$ & $B_2$ & $B_2$ & $B_2$ & $B_2$ & $B_2$\\
    \hline
    $b$ & $B_1$ & $B_2$ & $B_2$ & $B_2$ & $B_2$ & $B_2$\\
    \hline
  \end{tabular}

  Виждаме, че $\equiv_1\ =\ \equiv_2$.
  \marginpar{Получаваме, че $\equiv_\A\ =\ \equiv_1$}
  Следователно, минималният автомат има три състояния.
  Той е изобразен на Фигура \ref{sub:min1}.  
  Минималният автомат може да се представи и таблично:
  
  \begin{tabular}{|c|c|c|c|c|c|c|}
    % \hline
    % $Q$ & $0$ & $1$ & $2$ & $3^\star$ & $4^\star$ & $5^\star$ \\
    % \hline
    \hline
    $\delta$ & $B_0$ & $B_1$ & $B_2$ \\
    \hline
    $a$ & $B_1$ & $B_2$ & $B_2$ \\
    \hline
    $b$ & $B_1$ & $B_2$ & $B_2$ \\
    \hline
  \end{tabular}
\end{itemize}
\end{example}

\begin{example}
  Да разгледаме следния краен детерминиран автомат $\A$.
  \begin{figure}[H]
    % \begin{center}
    \begin{subfigure}[b]{0.4\textwidth}
      \begin{tikzpicture}[->,>=stealth,thick,node distance=55pt]
        \tikzstyle{every state}=[circle,minimum size=20pt,auto]
        
        \node[initial above, state]   (0) {$0$};
        \node[state,accepting]        (1) [above right of=0]{$1$};
        \node[state,accepting]        (2) [below right of=0]{$2$};
        \node[state]                  (3) [right of=1]{$3$};
        \node[state]                  (4) [right of=2]{$4$};
        \node[state,accepting]        (5) [below right of=3]{$5$};
        
        \path 
        (0) edge node [below] {$a$}   (1)
            edge node [below] {$b$}   (2)
        (1) edge node [above] {$a$}    (3)
            edge [bend left=15] node [below] {$b$}    (4)
        (2) edge [bend left=15] node [left] {$b$}    (3)
            edge node [below] {$a$}   (4)
        (4) edge node [below] {$a,b$} (5)
        (3) edge node [left] {$a,b$}  (5)
        (5) edge [loop above]   node [above] {$a,b$}  (5);
      \end{tikzpicture}
      \caption{Ще построим минимален автомат разпознаващ $\L(\A)$}
    \end{subfigure}
    \qquad
    \qquad
    \begin{subfigure}[b]{0.4\textwidth}
      \begin{tikzpicture}[->,>=stealth,thick,node distance=45pt]
        \tikzstyle{every state}=[circle,minimum size=20pt,auto,scale=.9]
        
        \node[initial above, state]   (0) {$C_0$};
        \node[state,accepting]  (1) [right of=0]{$C_1$};
        \node[state]            (2) [right of=1]{$C_2$};
        \node[state,accepting]  (3) [right of=2]{$C_3$};
                
        \path 
        (0) edge [bend left=15] node [above] {$a,b$}   (1)
        (1) edge [bend left=15] node [above] {$a,b$}   (2)
        (2) edge [bend left=15] node [above] {$a,b$}   (3)
        (3) edge [loop above]   node [above] {$a,b$}   (3);
      \end{tikzpicture}
      \caption{Получаваме следния минимален автомат $\A_0$, $\L(\A_0) = \L(\A)$}
      \label{sub:min2}
    \end{subfigure}
  \end{figure}

  \marginpar{Съобразете, че $\L(\A) = \{a,b\} \cup \{\alpha \in \{a,b\}^\star \mid \abs{\alpha} \geq 3\}$.}
  
  Отново следваме същата процедура за минимизация.
  Ще намерим класовете на еквивалентност на $\equiv_n$,
  докато не намерим $n$, за което $\equiv_n\ =\ \equiv_{n+1}$.
  \begin{itemize}
  \item
    Класовете на екиваленост на $\equiv_0$ са 
    $A_0 = Q\setminus F = \{0,3,4\}$ и $A_1 = F = \{1,2,5\}$.
  \item
    Разбиваме класовете на еквивалентност на $\equiv_0$ като използваме \Prop{one-letter-test}.
    
    \begin{tabular}{|c|c|c|c|c|c|c|}
      \hline
      $Q$ & 0 & $1^\star$ & $2^\star$ & 3 & 4 & $5^\star$ \\
      \hline
      \hline
      $\equiv_0$ & $A_0$ & $A_1$ & $A_1$ & $A_0$ & $A_0$ & $A_1$\\
      \hline
      $a$ & $A_1$& $A_0$ & $A_0$ & $A_1$ & $A_1$ & $A_1$\\
      \hline
      $b$ & $A_1$& $A_0$ & $A_0$ & $A_1$ & $A_1$ & $A_1$\\
      \hline
    \end{tabular}
    
    Виждаме, че $1 \not\equiv_1 5$ и $1 \equiv_0 5$.
    Следователно, $\equiv_0\ \neq\ \equiv_1$.
    Класовете на еквивалентност на $\equiv_1$ са 
    $B_0 = \{0,3,4\}$, $B_1 = \{1,2\}$, $B_2 = \{5\}$.
  \item
    Сега се опитваме да разбием класовете на еквивалентност на $\equiv_1$.

    \begin{tabular}{|c|c|c|c|c|c|c|}
      \hline
      $Q$ & 0 & $1^\star$ & $2^\star$ & 3 & 4 & $5^\star$ \\
      \hline
      \hline
      $\equiv_1$ & $B_0$ & $B_1$ & $B_1$ & $B_0$ & $B_0$ & $B_2$\\
      \hline
      $a$ & $B_1$ & $B_0$ & $B_0$ & $B_2$ & $B_2$ & $B_2$\\
      \hline
      $b$ & $B_1$ & $B_0$ & $B_0$ & $B_2$ & $B_2$ & $B_2$\\
      \hline
    \end{tabular}
    
    Имаме, че $0 \equiv_1 3$, но $0 \not\equiv_2 3$. Следователно $\equiv_1\ \neq\ \equiv_2$.
    Класовете на еквивалентност на $\equiv_2$ са 
    $C_0 = \{0\}$, $C_1 = \{1,2\}$, $C_2 = \{3,4\}$, $C_3 = \{5\}$.
  \item
    Отново опитваме да разбием класовете на $\equiv_2$.

      \begin{tabular}{|c|c|c|c|c|c|c|}
        \hline
        $Q$ & 0 & $1^\star$ & $2^\star$ & 3 & 4 & $5^\star$ \\
        \hline
        \hline
        $\equiv_2$ & $C_0$ & $C_1$ & $C_1$ & $C_2$ & $C_2$ & $C_3$\\
        \hline
        $a$ & $C_1$ & $C_2$ & $C_2$ & $C_3$ & $C_3$ & $C_3$\\
        \hline
        $b$ & $C_1$ & $C_2$ & $C_2$ & $C_3$ & $C_3$ & $C_3$\\
        \hline
      \end{tabular}
      
      Виждаме, че не можем да разбием $C_1$ или $C_2$.
      \marginpar{Получаваме, че $\equiv_\A\ =\ \equiv_2$}
      Следователно, $\equiv_2\ =\ \equiv_3$ и минималният автомат разпознаващ езика $L$
      има четири състояния. Вижте Фигура \ref{sub:min2} за преходите на минималния автомат.
      Минималният автомат може да се представи и таблично:

      \begin{tabular}{|c|c|c|c|c|}
        \hline
        $\delta$ & $C_0$ & $C_1$ & $C_2$ & $C_3$ \\
        \hline
        $a$ & $C_1$ & $C_2$ & $C_3$ & $C_3$ \\
        \hline
        $b$ & $C_1$ & $C_2$ & $C_3$ & $C_3$ \\
        \hline
      \end{tabular}
      
  \end{itemize}
\end{example}

%%% Local Variables: 
%%% mode: latex
%%% TeX-master: "../EAI"
%%% End: 
