\section{Регулярни граматики}
\index{граматика!неограничена}
\label{sect:regular-grammar}
\mynote{На англ. {\em unrestricted grammar}. Това е тип 0 граматики в йерархията на Чомски \cite[стр. 220]{hopcroft1}.}
{\bf Неограничена граматика} e наредена четворка от вида
\[G = (V, \Sigma, R, S),\]
където:
\begin{itemize}
\item
  $V$ е крайно множество от {\em променливи} (нетерминали);
\item
  $\Sigma$ е крайно множество от {\em букви} (терминали), $\Sigma \cap V = \emptyset$;
\item
  \mynote{В \cite{hopcroft1} правилата се наричат {\em productions} или {\em production rules}.}
  $R \subseteq (V\cup\Sigma)^+ \times (V \cup \Sigma)^\star$ е крайно множество от {\em правила}.
  За по-добра яснота, обикновено правилата $(\alpha, \beta) \in R$ ще означаваме като 
  $\alpha \to_G \beta$.
\item
  $S \in V$ е началната променлива (нетерминал). 
\end{itemize}

\index{граматика!извод}
Удобно е да дефинираме извод на думата $\beta$ от думата $\alpha$ в граматиката $G$ за $\ell$ стъпки, което ще означаваме като $\alpha \derive{\ell} \beta$,
с индукция по броя на стъпките $\ell$ по следния начин:
\begin{prooftree}
  \AxiomC{}
  \UnaryInfC{$\alpha \derive{0} \alpha$}
\end{prooftree}

\begin{prooftree}
  \AxiomC{$(\beta,\beta') \in R$}
  \UnaryInfC{$\alpha\beta\gamma \derive{1} \alpha\beta'\gamma$}
\end{prooftree}

\begin{prooftree}
  \AxiomC{$\alpha \derive{\ell} \gamma$}
  \AxiomC{$\gamma \derive{1} \beta$}
  \BinaryInfC{$\alpha \derive{\ell+1} \beta$}
\end{prooftree}

Нека $\derive{\star}$ е рефлексивното и транзитивно затваряне на релацията $\derive{1}$. С други думи,
\[ \alpha \derive{\star} \beta\ \iff\ (\exists \ell\in\Nat)[\ \alpha \derive{\ell} \beta\ ].\]
Езикът, който се поражда от граматиката $G$ е
\[\L(G) \df \{\omega \in \Sigma^\star \mid S \derive{\star} \omega\}.\]

Граматиките се разделят на няколко вида в зависимост от това какви {\em ограничения} налагаме върху правилата $R$.
В следващите няколко глави ще разгледаме различни ограничения. Сега ще разгледаме граматики с такъв вид правила,
които пораждат точно регулярните (или еквивалентно автоматни) езици.

\index{граматика!регулярна}
\index{граматика!тип 3}
\mynote{Също така се наричат граматики от тип 3 в йерархията на Чомски \cite[стр. 217]{hopcroft1}.}
Граматиката $G = (V, \Sigma, R, S)$ се нарича {\bf регулярна граматика},
ако всички правила са от вида 
\begin{align*}
  & A \to aB,\\
  & A \to a,\\
  & A \to \varepsilon,
\end{align*}
за произволни $A, B \in V$ и $a \in \Sigma$.
% Ако началната променлива $S$ не се среща в дясна част на правило, то позволяваме и правилото $S \to \varepsilon$,
% ако искаме $\varepsilon \in \L(G)$.

\begin{lemma}
  За всяка регулярна граматика $G$ съществува НКА $\N$, такъв че $\L(G) = \L(\N)$.
\end{lemma}
\begin{hint}
  Нека $G = \CFG$ и $V = \{A_0,\dots,A_k\}$, където $S = A_0$. Тогава дефинираме $\N$ по следния начин:
  \begin{itemize}
  \item
    $Q \df \{q_0,\dots,q_k,f\}$;
  \item
    $Q_{\texttt{start}} \df \{q_0\}$;
  \item
    $F \df \{q_i \mid A_i \to \varepsilon\} \cup \{f\}$;
  \item
    Релацията на преходите $\Delta$ е дефинирана по следния начин:
    \begin{align*}
      \Delta(q_i,a) \df & \{ q_j\ \mid\ A_i \to aA_j \text{ е правило в граматиката}\}\ \cup\\
                      & \{ f\ \mid\ A_i \to a \text{ е правило в граматиката}\}.
    \end{align*}
  \end{itemize}
  Докажете, че $\L(\N) = \L(G)$.
\end{hint}

\begin{lemma}
  За всеки ДКА $\A$ съществува регулярна граматика $G$, такава че $\L(\A)~=~\L(G)$.
\end{lemma}
\begin{hint}
  Нека $\A = \FA$ и $Q = \{q_0,\dots,q_k\}$, където $\qstart = q_0$. Тогава дефинираме $G = \CFG$ по следния начин:
  \begin{itemize}
  \item 
    $V \df \{A_0,\dots,A_k\}$;
  \item
    $S \df A_0$;
  \item
    $A_i \to aA_j\ \dff\ \delta(q_i,a) = q_j$;
  \item
    $A_{i} \to \varepsilon\ \dff\ q_{i} \in F$.
  \end{itemize}
  Докажете, че $\L(\A) = \L(G)$.
\end{hint}

\begin{framed}
  \begin{theorem}
    Един език е регулярен точно тогава, когато се поражда от регулярна граматика.
  \end{theorem}
\end{framed}


%%% Local Variables:
%%% mode: latex
%%% TeX-master: "../eai"
%%% End:
