\section{Регулярни граматики}
\index{граматика!неограничена}
\label{sect:regular-grammar}
{\bf Неограничена граматика} e наредена четворка от вида
\[G = (V, \Sigma, R, S),\]
където
\begin{itemize}
\item
  $V$ е крайно множество от {\em променливи} (нетерминали);
\item
  $\Sigma$ е крайно множество от {\em букви} (терминали), $\Sigma \cap V = \emptyset$;
\item
  \marginpar{В \cite{hopcroft1} правилата се наричат {\em productions}}
  $R \subseteq (V\cup\Sigma)^+ \times (V \cup \Sigma)^\star$ е крайно множество от {\em правила}.
  Обикновено правилата $(\alpha, \beta) \in R$ ще означаваме като 
  $\alpha \to_G \beta$, където $\alpha \in (V \cup \Sigma)^+, \beta \in (V \cup \Sigma)^\star$;
\item
  $S \in V$ е началната променлива (нетерминал). 
\end{itemize}

Казваме, че имаме {\bf извод} $\omega \to_G \omega'$, ако $\omega = \alpha\beta\gamma \in (V\cup\Sigma)^\star$,
$\omega' = \alpha\beta'\gamma \in (V\cup\Sigma)^\star$ и имаме правило $\beta \to \beta'$ в граматиката $G$.
Нека $\to^\star_G$ е рефлексивното и транзитивно затваряне на релацията $\to_G$, т.е.
\begin{align*}
  & \alpha \to^\star_G \alpha\\
  & \alpha \to^\star_G \alpha'\ \&\ \alpha' \to_G \alpha'' \implies \alpha \to^\star_G \alpha''.
\end{align*}

Езикът, който се поражда от граматиката $G$ е
\[\L(G) = \{\omega \in \Sigma^\star \mid S \to^\star_G \omega\}.\]

Граматиките се разделят на няколко вида в зависимост от това какви {\em ограничения} налагаме върху правилата $R$.
В следващите няколко глави ще разгледаме различни ограничения. Сега ще разгледаме граматики с такъв вид правила,
които пораждат точно регулярните (или еквивалентно автоматни) езици.

\index{граматика!регулярна}
Граматиката $G = (V, \Sigma, R, S)$ се нарича {\bf регулярна граматика},
ако правилата са от вида 
\begin{align*}
  & A \to \varepsilon,\\
  % & A \to a,\\
  & A \to aB,
\end{align*}
където $A, B \in V$ и $a \in \Sigma$.

\begin{prop}
  Нека $G = \CFG$ е регулярна граматика и $L = \L(G)$.
  Съществува краен автомат $\A$, такъв че $L = \L(\A)$.
\end{prop}
\begin{hint}
  Нека $V = \{A_1,\dots,A_k\}$. Тогава:
  \begin{itemize}
  \item 
    $Q = \{q_1,\dots,q_k\}$;
  \item
    $F = \{q_i \mid A_i \to \varepsilon\}$.
  \item
    $\delta(q_i,a) = q_j\ \iff\ A_i \to aA_j$.
  \end{itemize}
\end{hint}

\begin{prop}
  Нека $\A$ е краен автомат и $L = \L(\A)$.
  Съществува регулярна граматика $G$, такава че $L = \L(G)$.
\end{prop}
\begin{hint}
  Нека $Q = \{q_1,\dots,q_k\}$. Тогава:
  \begin{itemize}
  \item 
    $V = \{A_1,\dots,A_k\}$;
  \item
    $A_i \to aA_j\ \iff\ \delta(q_i,a) = q_j$;
  \item
    $A_{i} \to \varepsilon\ \iff\ q_{i} \in F$.
  \end{itemize}
\end{hint}


%%% Local Variables:
%%% mode: latex
%%% TeX-master: "../eai"
%%% End:
