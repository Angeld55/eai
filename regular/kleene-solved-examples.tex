\subsection{Примерни задачи}

\begin{extra}
\begin{problem}
  Приложете конструкцията от \hyperref[th:regular:kleenef]{теоремата на Клини} за да получите регулярен израз, който описва езика на автомата $\A$ изобразен на Фигура~\ref{fig:a1}.
    \begin{figure}[H]
      \begin{center}
        \begin{tikzpicture}[framed,->,>=stealth,thick,node distance=45pt,initial text=начало]
          \tikzstyle{every state}=[circle,minimum size=15pt,auto]
          
          \node[initial,state]      (1) {$q_0$};
          \node[accepting, state]   (2) [right of=1]{$q_1$};
          
          \path 
          (1) edge [loop above]  node [above] {$b$} (1)
          (1) edge  node [above] {$a$} (2)
          (2) edge [loop above] node [above] {$a,b$} (2);
        \end{tikzpicture}
      \end{center}
      \caption{$\L(\A) \stackrel{?}{=} \L(\mathbf{b^\star a (a + b)^\star)}$}
      \label{fig:a1}
    \end{figure}
\end{problem}
\begin{solution}
  Лесно се съобразява, че езикът на автомата от \Figure{a1} се описва с регулярния израз $\mathbf{b^\star a (a + b)^\star}$.
  Следвайки означенията и конструкцията от доказателството на \hyperref[th:regular:kleene]{теоремата на Клини},
  езикът на този автомат се описва с регулярния израз $\mathbf{r}^2_{0,1}$, защото началното състояние е $q_0$, финалното е $q_1$ и 
  броят на състоянията в автомата е $2$. Подробните сметки са следните:
  \begin{align*}
    \bm{r}^2_{0,1} & \equiv \underbrace{\bm{r}^{1}_{0,1}}_{\bm{b^\star a}} + \underbrace{\bm{r}^{1}_{0,1}}_{\bm{b^\star a}}\cdot \underbrace{\bm{(r^1_{1,1})^\star}}_{\mathbf{(\bm{\varepsilon}+a+b)^\star}} \cdot \underbrace{\mathbf{r}^1_{1,1}}_{\mathbf{\bm{\varepsilon}+a+b}} & \comment \text{според (\ref{eq:kleene}}) \\
                   & \equiv \mathbf{b^\star a + b^\star a (\bm{\varepsilon} + a + b)^\star (\bm{\varepsilon} + a + b)} & \comment{\text{просто заместваме}}\\
                   & \equiv \mathbf{b^\star a + b^\star a (\bm{\varepsilon} + a + b)^+} & \comment \mathbf{r^+ \df r^\star r}\\
                   & \equiv \mathbf{b^\star a + b^\star a (a + b)^\star} & \comment \mathbf{r^\star \equiv (\bm{\varepsilon} + r)^+}\\
                   & \equiv \mathbf{b^\star a (\bm{\varepsilon} + (a + b)^\star)} & \comment \mathbf{r + rq \equiv r(\bm{\varepsilon} + q)}\\
                   & \equiv \mathbf{b^\star a (a + b)^\star}. & \comment \mathbf{r^\star \equiv \bm{\varepsilon} + r^\star}
\end{align*}
Тук използвахме, че:
\begin{align*}
  \mathbf{r^1_{0,1}} & \equiv \underbrace{\mathbf{r}^0_{0,1}}_{\mathbf{a}} + \underbrace{\mathbf{r}^0_{0,0}}_{\bm{\varepsilon + b}}\cdot\underbrace{\mathbf{(r^0_{0,0})^\star}}_{\bm{(\varepsilon+b)^\star}} \cdot \underbrace{\mathbf{r^0_{0,1}}}_{\mathbf{b}} & \comment \text{според (\ref{eq:kleene}})\\
                     & \equiv \bm{a + (\varepsilon + b)(\varepsilon + b)^\star a} & \comment{\text{просто заместваме}}\\
                     & \equiv \mathbf{a + b^\star a}  & \comment \mathbf{r}^\star \equiv \bm{\varepsilon} + \mathbf{r}^\star \\
                     & \equiv \mathbf{b^\star a} \\
  \mathbf{r}^b_{b,b} & \equiv \underbrace{\mathbf{r}^a_{b,b}}_{\bm{\varepsilon+a+b}} + \underbrace{\mathbf{r}^a_{b,a}}_{\bm{\emptyset}} \cdot
                       \underbrace{(\mathbf{r}^a_{a,a})^\star}_{\bm{\varepsilon+b}} \cdot \underbrace{\mathbf{r}^a_{a,b}}_{\bm{a}}& \comment \text{отново според (\ref{eq:kleene}})\\
                     & \equiv \bm{\varepsilon + a + b + \emptyset(\varepsilon + b)^\star a} & \comment{\text{просто заместваме}}\\
                     & \equiv \bm{\varepsilon + a + b}. & \comment \text{защото }\bm{\emptyset \cdot r} \equiv \bm{\emptyset}
\end{align*}  
\end{solution}


\begin{problem}
  Приложете конструкцията от \hyperref[th:regular:kleenef]{теоремата на Клини} за да получите регулярен израз, който описва езика на автомата $\A$ изобразен на Фигура~\ref{fig:a2}.
  \begin{figure}[H]
      \begin{center}
        \begin{tikzpicture}[framed,->,>=stealth,thick,node distance=45pt,initial text=начало]
          \tikzstyle{every state}=[circle,minimum size=15pt,auto]
          
          \node[initial,state]      (1) {$q_0$};
          \node[state]              (2) [right of=1]{$q_1$};
          \node[accepting, state]   (3) [right of=2]{$q_2$};
          
          \path 
          (2) edge [loop above]    node [above] {$a$} (2)
          (1) edge [bend left=15]  node [above] {$a$} (2)
          (2) edge [bend left=15]  node [above] {$b$} (3)
          (1) edge [bend right=45] node [below] {$b$} (3)
          (3) edge [loop above]    node [above] {$a,b$} (3);
        \end{tikzpicture}
      \end{center}
      \caption{$\L(\A) \stackrel{?}{=} \L(\mathbf{a^\star b(a+b)^\star})$.}
      \label{fig:a2}
    \end{figure}
  \end{problem}
  \begin{solution}
    От \hyperref[th:regular:kleene]{теоремата на Клини} знаем, че $\L(\A) = \L(\mathbf{r}^3_{0,2})$, където:
  \begin{align*}
    \mathbf{r}^3_{0,2} & \equiv \underbrace{\mathbf{r}^2_{0,2}}_{\bm{a^\star b}} + \underbrace{\mathbf{r}^2_{0,2}}_{\bm{a^\star b}} \cdot \underbrace{(\mathbf{r}^2_{2,2})^\star}_{\bm{(\varepsilon+a+b)^\star}} \cdot \underbrace{\mathbf{r}^2_{2,2}}_{\bm{\varepsilon+a+b}} & \comment \text{според (\ref{eq:kleene}})\\
                       & \equiv \bm{a^\star b + a^\star b \cdot (a+b)^\star \cdot (\varepsilon+a+b)} & \comment{\text{просто заместваме}}\\
                       & \equiv \bm{a^\star b + a^\star b \cdot (a+b)^\star} & \comment \bm{r^\star} \equiv \bm{r^\star \cdot (\varepsilon + r)}\\
                       & \equiv \bm{a^\star b (\varepsilon + (a+b)^\star)} & \comment\bm{r_1 + r_1\cdot r_2 \equiv r_1 \cdot (\varepsilon+r_2)}\\
                       & \equiv \bm{a^\star b (a+b)^\star}. & \comment \bm{r^\star} \equiv \bm{\varepsilon + r^\star}
  \end{align*}
  Тук използвахме, че:
  \begin{align*}
    \mathbf{r}^2_{0,2} & \equiv \underbrace{\mathbf{r}^1_{0,2}}_{\bm{b}} + \underbrace{\mathbf{r}^1_{0,1}}_{\bm{a}} \cdot \underbrace{(\mathbf{r}^1_{1,1})^\star}_{\bm{a}} \cdot \underbrace{\mathbf{r}^1_{1,2}}_{\bm{b}} & \comment \text{според (\ref{eq:kleene}}) \\
                       & \equiv \bm{b + a \cdot a^\star \cdot b} & \comment{\text{просто заместваме}}\\
                       & \equiv \bm{(\varepsilon + a^+)\cdot b} & \comment \mathbf{r}^+ \df \mathbf{r} \cdot \mathbf{r}^\star\\
                       & \equiv \bm{a^\star b}. & \comment \mathbf{r}^\star \equiv \bm{\varepsilon + r^+.г}
  \end{align*}
  Заключаваме, че $\L(\A) = \L(\mathbf{a^\star b(a+b)^\star})$.
\end{solution}
\end{extra}

% \todo{Може да се сложи още един пример, но без да има решение.}


%%% Local Variables:
%%% mode: latex
%%% TeX-master: "../eai"
%%% End:
