\newpage
\subsection*{Примерни задачи}
\begin{extra2}
  \begin{example}
    \mynote{\cite[стр. 79]{kozen}}
    Да простроим минимален автомат $\A'$ разпознаващ езика на детерминирания краен автомат $\A$.
    \begin{figure}[H]
      \centering
      \begin{tikzpicture}[->,>=stealth,thick,node distance=55pt,initial text=начало]
        \tikzstyle{every state}=[circle,minimum size=20pt,scale=0.8, every node/.style={scale=0.8}]
        
        \node[initial, state]   (0) {$0$};
        \node[state]            (1) [above right of=0]{$1$};
        \node[state]            (2) [below right of=0]{$2$};
        \node[state,accepting]  (3) [right of=1]{$3$};
        \node[state,accepting]  (4) [right of=2]{$4$};
        \node[state,accepting]  (5) [below right of=3]{$5$};
        
        \path 
        (0) edge [bend left=15]  node [above left] {$a$}   (1)
        (0) edge [bend right=15] node [below left] {$b$}   (2)
        (1) edge [bend left=15]  node [above] {$a$}    (3)
        (1) edge [bend left=15]  node [below] {$b$}    (4)
        (2) edge [bend left=15]  node [left] {$b$}    (3)
        (2) edge [bend right=15] node [below] {$a$}   (4)
        (4) edge [bend right=15] node [below right] {$a,b$} (5)
        (3) edge [bend left=15]  node [above right] {$a,b$}  (5)
        (5) edge [loop right]    node [above] {$a,b$}  (5);
      \end{tikzpicture}
      \caption{Автоматът $\A$.}
    \end{figure}
    \mynote{В процедурата, която следваме тук, изобщо не се интересуваме какъв е езика на автомата $\A$.
      
      \writedown Взе пак съобразете, че езикът на автомата $\A$ е \[\{\omega \in \{a,b\}^\star \mid \abs{\omega} \geq 2\}.\]}
  Ще приложим алгоритъма за минимизация за да получим минималния автомат за езика $L$.
  За всяко $n = 0,1,2,\dots$, ще намерим класовете на еквивалентност на $\equiv^n_\A$,
  докато не намерим $n$, за което $\equiv^n_\A\ =\ \equiv^{n+1}_\A$.
  За първото такова $n$, според \Proposition{minimisation-cubic:equiv-approx}, знаем, че $\equiv^n_\A = \equiv_\A$.
  \begin{itemize}
  \item 
    Класовете на еквивалентност на $\equiv^0_\A$ са два. Те са
    \begin{align*}
      & A_0 = Q\setminus F = \{0,1,2\}\text{ и }\\
      & A_1 = F = \{3,4,5\}.
    \end{align*}
  \item
    Сега да видим дали можем да разбием някои от класовете на еквивалентност на $\equiv^0_\A$.
    \begin{tabular}{|c|c|c|c|c|c|c|}
      \hline
      $\equiv^0_\A$ & \multicolumn{3}{|c|}{$A_0$} & \multicolumn{3}{|c|}{$A_1$}\\
      \hline
      \hline
      $Q$ & $0$ & $1$ & $2$ & $3^\star$ & $4^\star$ & $5^\star$ \\
      \hline
      $a$ & $A_0$& $A_1$ & $A_1$ & $A_1$ & $A_1$ & $A_1$\\
      \hline
      $b$ & $A_0$& $A_1$ & $A_1$ & $A_1$ & $A_1$ & $A_1$\\
      \hline
    \end{tabular}
    
    Като използваме \Proposition{one-letter-test}, виждаме, че $0 \not\equiv^1_\A 1$, защото $\delta(0,a) \not\equiv^0_\A \delta(1,a)$.
    От друга страна, $1 \equiv^1_\A 2$.
    Класовете на еквивалентност на $\equiv^1_\A$ са следните множества:
    \begin{align*}
      & B_0 = \{0\},\\
      & B_1 = \{1,2\},\\
      & B_2 = \{3,4,5\}.
    \end{align*}
  \item
    Сега да видим дали можем да разбием някои от класовете на еквивалентност на $\equiv^1_\A$.
    
    \begin{tabular}{|c|c|c|c|c|c|c|}
      \hline
      $\equiv^1_\A$ & $B_0$ & \multicolumn{2}{|c|}{$B_1$} & \multicolumn{3}{|c|}{$B_2$}\\
      \hline
      \hline
      $Q$ & $0$ & $1$ & $2$ & $3^\star$ & $4^\star$ & $5^\star$ \\
      \hline
      $a$ & $B_1$ & $B_2$ & $B_2$ & $B_2$ & $B_2$ & $B_2$\\
      \hline
      $b$ & $B_1$ & $B_2$ & $B_2$ & $B_2$ & $B_2$ & $B_2$\\
      \hline
    \end{tabular}

    
    Виждаме, че $\equiv^1_\A\ =\ \equiv^2_\A$,
    което означава, че $\equiv_\A\ =\ \equiv^1_\A$.
    Следователно, минималният автомат $\M$ има три състояния.
    Той е изобразен на \Figure{min1}.  
    Минималният автомат може да се представи и таблично:

    \begin{tabular}{|c|c|c|c|c|c|c|}
      \hline
      $\delta'$ & $B_0$ & $B_1$ & $B_2$ \\
      \hline
      $a$ & $B_1$ & $B_2$ & $B_2$ \\
      \hline
      $b$ & $B_1$ & $B_2$ & $B_2$ \\
      \hline
    \end{tabular}
  \end{itemize}
  
  \begin{figure}[H]
    \centering
    \begin{tikzpicture}[->,>=stealth,thick,node distance=55pt,initial text=начало]
      \tikzstyle{every state}=[circle,minimum size=20pt,scale=0.8, every node/.style={scale=0.8}]
      
      \node[initial above, state]   (0) {$B_0$};
      \node[state]                  (1) [right of=0]{$B_1$};
      \node[state,accepting]        (2) [right of=1]{$B_2$};
      
      \path 
      (0) edge [bend left=15] node [above] {$a,b$}   (1)
      (1) edge [bend left=15] node [above] {$a,b$}   (2)
      (2) edge [loop above] node [above] {$a,b$}   (2);
    \end{tikzpicture}
  \caption{Минимален автомат $\A'$ за $\L(\A)$.}
  \label{fig:min1}
\end{figure}
\end{example}
\end{extra2}

\vspace*{1cm}

\begin{extra2}
\begin{example}
  Да построим минимален автомат $\A'$ разпознаващ $\L(\A)$.
  \begin{figure}[H]
    \centering
      \begin{tikzpicture}[->,>=stealth,thick,node distance=55pt,initial text=начало]
        \tikzstyle{every state}=[circle,minimum size=20pt,scale=0.8, every node/.style={scale=0.8}]
        
        \node[initial, state]         (0) {$0$};
        \node[state,accepting]        (1) [above right of=0]{$1$};
        \node[state,accepting]        (2) [below right of=0]{$2$};
        \node[state]                  (3) [right of=1]{$3$};
        \node[state]                  (4) [right of=2]{$4$};
        \node[state,accepting]        (5) [below right of=3]{$5$};
        
        \path 
        (0) edge [bend left=15]  node [above left] {$a$}   (1)
            edge [bend right=15] node [below left] {$b$}   (2)
        (1) edge [bend left=15]  node [above] {$a$}   (3)
            edge [bend left=15]  node [right] {$b$}   (4)
        (2) edge [bend left=15]  node [left] {$b$}    (3)
            edge [bend right=15] node [below] {$a$}   (4)
        (4) edge [bend right=15] node [below right] {$a,b$} (5)
        (3) edge [bend left=15]  node [above right] {$a,b$} (5)
        (5) edge [loop right]    node [above] {$a,b$} (5);
      \end{tikzpicture}
      \caption{Автоматът $\A$.}
    \end{figure}
  %   \qquad
  %   \qquad
  %   \begin{subfigure}[b]{0.4\textwidth}
  %     \begin{tikzpicture}[->,>=stealth,thick,node distance=45pt]
  %       \tikzstyle{every state}=[circle,minimum size=20pt,auto,scale=.9]
        
  %       \node[initial above, state]   (0) {$C_0$};
  %       \node[state,accepting]  (1) [right of=0]{$C_1$};
  %       \node[state]            (2) [right of=1]{$C_2$};
  %       \node[state,accepting]  (3) [right of=2]{$C_3$};
                
  %       \path 
  %       (0) edge [bend left=15] node [above] {$a,b$}   (1)
  %       (1) edge [bend left=15] node [above] {$a,b$}   (2)
  %       (2) edge [bend left=15] node [above] {$a,b$}   (3)
  %       (3) edge [loop above]   node [above] {$a,b$}   (3);
  %     \end{tikzpicture}
  %     \caption{Получаваме минималния автомат $\M$, $\L(\M) = \L(\A)$}
  %     \label{sub:min2}
  %   \end{subfigure}
  % \end{figure}
  % \end{framed}
% \end{problem}

  % \begin{solution}
    \mynote{\writedown Съобразете, че езикът на автомата $\A$ е $\{\omega \in \{a,b\}^\star \mid \abs{\omega} \neq 0,2\}$.}
    Отново следваме същата процедура за минимизация.
    Ще намерим класовете на еквивалентност на $\equiv^n_\A$,
    докато не намерим $n$, за което $\equiv^n_\A\ =\ \equiv^{n+1}_\A$.
    % \begin{multicols}{2}
  \begin{itemize}
  \item
    Класовете на екиваленост на $\equiv^0_\A$ са следните:
    \begin{align*}
      & A_0 \df Q\setminus F = \{0,3,4\} \text{ и }\\
      & A_1 \df F = \{1,2,5\}.
    \end{align*}
  \item
    Разбиваме класовете на еквивалентност на $\equiv^0_\A$ като използваме \Proposition{one-letter-test}.

    \begin{tabular}{|c|c|c|c|c|c|c|}
      \hline
      $\equiv^0_\A$ & \multicolumn{3}{|c|}{$A_0$} & \multicolumn{3}{|c|}{$A_1$}\\
      \hline
      \hline
      $Q$ & $0$ & $3$ & $4$ & $1^\star$ & $2^\star$ & $5^\star$ \\
      \hline
      $a$ & $A_1$& $A_1$ & $A_1$ & $A_0$ & $A_0$ & $A_1$\\
      \hline
      $b$ & $A_1$& $A_1$ & $A_1$ & $A_0$ & $A_0$ & $A_1$\\
      \hline
    \end{tabular}

    
    Виждаме, че $1 \not\equiv^1_\A 5$ и $1 \equiv^0_\A 5$.
    Следователно, $\equiv^0_\A\ \neq\ \equiv^1_\A$.
    Класовете на еквивалентност на $\equiv^1_\A$ са следните:
    \begin{align*}
      & B_0 \df \{0,3,4\},\\
      & B_1 \df \{1,2\},\\
      & B_2 \df \{5\}.
    \end{align*}
  \item
    Сега се опитваме да разбием класовете на еквивалентност на $\equiv^1_\A$.


    \begin{tabular}{|c|c|c|c|c|c|c|}
      \hline
      $\equiv^1_\A$ & \multicolumn{3}{|c|}{$B_0$} & \multicolumn{2}{|c|}{$B_1$} & $B_2$\\
      \hline
      \hline
      $Q$ & $0$ & $3$ & $4$ & $1^\star$ & $2^\star$ & $5^\star$ \\
      \hline
      $a$ & $B_1$& $B_2$ & $B_2$ & $B_0$ & $B_0$ & $B_2$\\
      \hline
      $b$ & $B_1$& $B_2$ & $B_2$ & $B_0$ & $B_0$ & $B_2$\\
      \hline
    \end{tabular}
    
    Имаме, че $0 \equiv^1_\A 3$, но $0 \not\equiv^2_\A 3$. Следователно $\equiv^1_\A\ \neq\ \equiv^2_\A$.
    Класовете на еквивалентност на $\equiv^2_\A$ са следните:
    \begin{align*}
      & C_0 \df \{0\},\\
      & C_1 \df \{1,2\},\\
      & C_2 \df \{3,4\},\\
      & C_3 \df \{5\}.
    \end{align*}
  \item
    Отново опитваме да разбием класовете на релацията $\equiv^2_\A$.

      \begin{tabular}{|c|c|c|c|c|c|c|}
        \hline
        $\equiv^2_\A$ & $C_0$ & \multicolumn{2}{|c|}{$C_1$} & \multicolumn{2}{|c|}{$C_2$} & $C_3$\\
        \hline
        \hline
        $Q$ & $0$ & $1^\star$ & $2^\star$ & $3$ & $4$ & $5^\star$ \\
        \hline
        $a$ & $C_1$& $C_2$ & $C_2$ & $C_3$ & $C_3$ & $C_3$\\
        \hline
        $b$ & $C_1$& $C_2$ & $C_2$ & $C_3$ & $C_3$ & $C_3$\\
        \hline
      \end{tabular}
      
      
      Виждаме, че не можем да разбием $C_1$ или $C_2$.
      Следователно, $\equiv^2_\A\ =\ \equiv^3_\A$.
      Оттук следва, че $\equiv^2_\A\ =\ \equiv_\A$ и минималният автомат разпознаващ езика $L$
      има четири състояния. Вижте \Figure{min2} за преходите на минималния автомат,
      които могат да се представят и таблично чрез функцията на преходите:

      \begin{tabular}{|c|c|c|c|c|}
        \hline
        $\delta$ & $C_0$ & $C_1$ & $C_2$ & $C_3$ \\
        \hline
        $a$ & $C_1$ & $C_2$ & $C_3$ & $C_3$ \\
        \hline
        $b$ & $C_1$ & $C_2$ & $C_3$ & $C_3$ \\
        \hline
      \end{tabular}
    \end{itemize}

    \begin{figure}[H]
      \centering
      \begin{tikzpicture}[->,>=stealth,thick,node distance=55pt]
        \tikzstyle{every state}=[circle,minimum size=20pt,scale=0.8, every node/.style={scale=0.8}, initial text=начало]
        
        \node[initial above, state]   (0) {$C_0$};
        \node[state,accepting]  (1) [right of=0]{$C_1$};
        \node[state]            (2) [right of=1]{$C_2$};
        \node[state,accepting]  (3) [right of=2]{$C_3$};
        
        \path 
        (0) edge [bend left=15] node [above] {$a,b$}   (1)
        (1) edge [bend left=15] node [above] {$a,b$}   (2)
        (2) edge [bend left=15] node [above] {$a,b$}   (3)
        (3) edge [loop above]   node [above] {$a,b$}   (3);
      \end{tikzpicture}
      \caption{Получаваме минималния автомат $\A'$ за езика на $\A$.}
      \label{fig:min2}
    \end{figure}
  \end{example}
\end{extra2}


%%% Local Variables:
%%% mode: latex
%%% TeX-master: "../eai"
%%% End:
