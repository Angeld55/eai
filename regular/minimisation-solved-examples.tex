\subsection{Примерни задачи}

\begin{problem}
  Постройте минимален автомат $\M$ разпознаващ езика на следния детерминиран краен автомат $\A$.
  \mynote{\writedown Съобразете, че $\L(\A) = \{\omega \in \{a,b\}^\star \mid \abs{\omega} \geq 2\}$.}
  \begin{framed}
    \begin{figure}[H]
      \begin{subfigure}[b]{0.45\textwidth}
        \begin{tikzpicture}[->,>=stealth,thick,node distance=55pt]
          \tikzstyle{every state}=[circle,minimum size=20pt,auto]
          
          \node[initial above, state]   (0) {$0$};
          \node[state]            (1) [above right of=0]{$1$};
          \node[state]            (2) [below right of=0]{$2$};
          \node[state,accepting]  (3) [right of=1]{$3$};
          \node[state,accepting]  (4) [right of=2]{$4$};
          \node[state,accepting]  (5) [below right of=3]{$5$};
          
          \path 
          (0) edge  node [above] {$a$}   (1)
          (0) edge  node [below] {$b$}   (2)
          (1) edge node [above] {$a$}    (3)
          (1) edge [bend left=15] node [below] {$b$}    (4)
          (2) edge [bend left=15] node [left] {$b$}    (3)
          (2) edge node [below] {$a$}   (4)
          (4) edge [bend right=15] node [right] {$a,b$} (5)
          (3) edge [bend left=15] node [right] {$a,b$}  (5)
          (5) edge [loop right]   node [above] {$a,b$}  (5);
        \end{tikzpicture}
        \caption{Ще построим минимален автомат разпознаващ езика на $\A$.}
      \end{subfigure}
      \quad
      \quad
      \begin{subfigure}[b]{0.4\textwidth}
        \begin{tikzpicture}[->,>=stealth,thick,node distance=45pt]
        \tikzstyle{every state}=[circle,minimum size=20pt,auto,scale=.9]
        
        \node[initial above, state]   (0) {$B_0$};
        \node[state]                  (1) [right of=0]{$B_1$};
        \node[state,accepting]        (2) [right of=1]{$B_2$};
        
        \path 
        (0) edge [bend left=15] node [above] {$a,b$}   (1)
        (1) edge [bend left=15] node [above] {$a,b$}   (2)
        (2) edge [loop above] node [above] {$a,b$}   (2);
      \end{tikzpicture}
      \caption{Получаваме минималния автомат $\M$, $\L(\M) = \L(\A)$}
      \label{sub:min1}
    \end{subfigure}
  \end{figure}
\end{framed}
\end{problem}

\ExtraMaterial{
  \begin{solution}
Ще приложим алгоритъма за минимизация за да получим минималния автомат за езика $L$.
За всяко $n = 0,1,2,\dots$, ще намерим класовете на еквивалентност на $\equiv^n_\A$,
докато не намерим $n$, за което $\equiv^n_\A\ =\ \equiv^{n+1}_\A$.

\begin{multicols}{2}
\begin{itemize}
\item 
  Класовете на еквивалентност на $\equiv^0_\A$ са два. Те са
  \[A_0 = Q\setminus F = \{0,1,2\}\text{ и }A_1 = F = \{3,4,5\}.\]
\item
  Сега да видим дали можем да разбием някои от класовете на еквивалентност на $\equiv^0_\A$.
  
  \begin{tabular}{|c|c|c|c|c|c|c|}
    \hline
    $Q$ & $0$ & $1$ & $2$ & $3^\star$ & $4^\star$ & $5^\star$ \\
    \hline
    \hline
    $\equiv^0_\A$ & $A_0$ & $A_0$ & $A_0$ & $A_1$ & $A_1$ & $A_1$\\
    \hline
    $a$ & $A_0$& $A_1$ & $A_1$ & $A_1$ & $A_1$ & $A_1$\\
    \hline
    $b$ & $A_0$& $A_1$ & $A_1$ & $A_1$ & $A_1$ & $A_1$\\
    \hline
  \end{tabular}

  Виждаме, че $0 \not\equiv^1_\A 1$ и $1 \equiv^1_\A 2$.
  Класовете на еквивалентност на $\equiv^1_\A$ са 
  $B_0 = \{0\}$, $B_1 = \{1,2\}$, $B_2 = \{3,4,5\}$.
\item
  Сега да видим дали можем да разбием някои от класовете на еквивалентност на $\equiv^1_\A$.
  
  \begin{tabular}{|c|c|c|c|c|c|c|}
    \hline
    $Q$ & $0$ & $1$ & $2$ & $3^\star$ & $4^\star$ & $5^\star$ \\
    \hline
    \hline
    $\equiv^1_\A$ & $B_0$ & $B_1$ & $B_1$ & $B_2$ & $B_2$ & $B_2$\\
    \hline
    $a$ & $B_1$ & $B_2$ & $B_2$ & $B_2$ & $B_2$ & $B_2$\\
    \hline
    $b$ & $B_1$ & $B_2$ & $B_2$ & $B_2$ & $B_2$ & $B_2$\\
    \hline
  \end{tabular}

  Виждаме, че $\equiv^1_\A\ =\ \equiv^2_\A$,
  което означава, че $\equiv_\A\ =\ \equiv^1_\A$.
  Следователно, минималният автомат има три състояния.
  Той е изобразен на Фигура \ref{sub:min1}.  
  Минималният автомат може да се представи и таблично:
  
  \begin{tabular}{|c|c|c|c|c|c|c|}
    \hline
    $\delta$ & $B_0$ & $B_1$ & $B_2$ \\
    \hline
    $a$ & $B_1$ & $B_2$ & $B_2$ \\
    \hline
    $b$ & $B_1$ & $B_2$ & $B_2$ \\
    \hline
  \end{tabular}
\end{itemize}
\end{multicols}

\end{solution}
}

\begin{problem}
  Постройте минимален автомат $\M$ разпознаващ езика на следния детерминиран краен автомат $\A$.
  \begin{framed}
  \begin{figure}[H]
    % \begin{center}
    \begin{subfigure}[b]{0.4\textwidth}
      \begin{tikzpicture}[->,>=stealth,thick,node distance=55pt]
        \tikzstyle{every state}=[circle,minimum size=20pt,auto]
        
        \node[initial above, state]   (0) {$0$};
        \node[state,accepting]        (1) [above right of=0]{$1$};
        \node[state,accepting]        (2) [below right of=0]{$2$};
        \node[state]                  (3) [right of=1]{$3$};
        \node[state]                  (4) [right of=2]{$4$};
        \node[state,accepting]        (5) [below right of=3]{$5$};
        
        \path 
        (0) edge [bend left=15] node [below] {$a$}   (1)
            edge [bend right=15] node [below] {$b$}   (2)
        (1) edge node [above] {$a$}    (3)
            edge [bend left=15] node [below] {$b$}    (4)
        (2) edge [bend left=15] node [left] {$b$}    (3)
            edge node [below] {$a$}   (4)
        (4) edge [bend right=15] node [below] {$a,b$} (5)
        (3) edge [bend left=15] node [right] {$a,b$}  (5)
        (5) edge [loop right]   node [above] {$a,b$}  (5);
      \end{tikzpicture}
      \caption{Ще построим минимален автомат разпознаващ $\L(\A)$}
    \end{subfigure}
    \qquad
    \qquad
    \begin{subfigure}[b]{0.4\textwidth}
      \begin{tikzpicture}[->,>=stealth,thick,node distance=45pt]
        \tikzstyle{every state}=[circle,minimum size=20pt,auto,scale=.9]
        
        \node[initial above, state]   (0) {$C_0$};
        \node[state,accepting]  (1) [right of=0]{$C_1$};
        \node[state]            (2) [right of=1]{$C_2$};
        \node[state,accepting]  (3) [right of=2]{$C_3$};
                
        \path 
        (0) edge [bend left=15] node [above] {$a,b$}   (1)
        (1) edge [bend left=15] node [above] {$a,b$}   (2)
        (2) edge [bend left=15] node [above] {$a,b$}   (3)
        (3) edge [loop above]   node [above] {$a,b$}   (3);
      \end{tikzpicture}
      \caption{Получаваме минималния автомат $\M$, $\L(\M) = \L(\A)$}
      \label{sub:min2}
    \end{subfigure}
  \end{figure}
  \end{framed}
\end{problem}

\ExtraMaterial{
  \begin{solution}
    \mynote{\writedown Съобразете, че $\L(\A) = \{\omega \in \{a,b\}^\star \mid \abs{\omega} = 1 \lor \abs{\omega} \geq 3\}$.}
    Отново следваме същата процедура за минимизация.
    Ще намерим класовете на еквивалентност на $\equiv^n_\A$,
    докато не намерим $n$, за което $\equiv^n_\A\ =\ \equiv^{n+1}_\A$.
    \begin{multicols}{2}
  \begin{itemize}
  \item
    Класовете на екиваленост на $\equiv^0_\A$ са 
    \[A_0 = Q\setminus F = \{0,3,4\} \text{ и }A_1 = F = \{1,2,5\}.\]
  \item
    Разбиваме класовете на еквивалентност на $\equiv^0_\A$ като използваме \Proposition{one-letter-test}.

    \begin{tabular}{|c|c|c|c|c|c|c|}
      \hline
      $Q$ & 0 & $1^\star$ & $2^\star$ & 3 & 4 & $5^\star$ \\
      \hline
      \hline
      $\equiv^0_\A$ & $A_0$ & $A_1$ & $A_1$ & $A_0$ & $A_0$ & $A_1$\\
      \hline
      $a$ & $A_1$& $A_0$ & $A_0$ & $A_1$ & $A_1$ & $A_1$\\
      \hline
      $b$ & $A_1$& $A_0$ & $A_0$ & $A_1$ & $A_1$ & $A_1$\\
      \hline
    \end{tabular}
    
    Виждаме, че $1 \not\equiv^1_\A 5$ и $1 \equiv^0_\A 5$.
    Следователно, $\equiv^0_\A\ \neq\ \equiv^1_\A$.
    Класовете на еквивалентност на $\equiv^1_\A$ са 
    $B_0 = \{0,3,4\}$, $B_1 = \{1,2\}$, $B_2 = \{5\}$.
  \item
    Сега се опитваме да разбием класовете на еквивалентност на $\equiv^1_\A$.
    
    \begin{tabular}{|c|c|c|c|c|c|c|}
      \hline
      $Q$ & 0 & $1^\star$ & $2^\star$ & 3 & 4 & $5^\star$ \\
      \hline
      \hline
      $\equiv^1_\A$ & $B_0$ & $B_1$ & $B_1$ & $B_0$ & $B_0$ & $B_2$\\
      \hline
      $a$ & $B_1$ & $B_0$ & $B_0$ & $B_2$ & $B_2$ & $B_2$\\
      \hline
      $b$ & $B_1$ & $B_0$ & $B_0$ & $B_2$ & $B_2$ & $B_2$\\
      \hline
    \end{tabular}
    
    Имаме, че $0 \equiv^1_\A 3$, но $0 \not\equiv^2_\A 3$. Следователно $\equiv^1_\A\ \neq\ \equiv^2_\A$.
    Класовете на еквивалентност на $\equiv^2_\A$ са 
    $C_0 = \{0\}$, $C_1 = \{1,2\}$, $C_2 = \{3,4\}$, $C_3 = \{5\}$.
  \item
    Отново опитваме да разбием класовете на $\equiv^2_\A$.

      \begin{tabular}{|c|c|c|c|c|c|c|}
        \hline
        $Q$ & 0 & $1^\star$ & $2^\star$ & 3 & 4 & $5^\star$ \\
        \hline
        \hline
        $\equiv^2_\A$ & $C_0$ & $C_1$ & $C_1$ & $C_2$ & $C_2$ & $C_3$\\
        \hline
        $a$ & $C_1$ & $C_2$ & $C_2$ & $C_3$ & $C_3$ & $C_3$\\
        \hline
        $b$ & $C_1$ & $C_2$ & $C_2$ & $C_3$ & $C_3$ & $C_3$\\
        \hline
      \end{tabular}
      
      Виждаме, че не можем да разбием $C_1$ или $C_2$.
      Следователно, $\equiv^2_\A\ =\ \equiv^3_\A$.
      Оттук следва, че $\equiv^2_\A\ =\ \equiv_\A$ и минималният автомат разпознаващ езика $L$
      има четири състояния. Вижте Фигура \ref{sub:min2} за преходите на минималния автомат,
      които могат да се представят и таблично чрез функцията на преходите:

      \begin{tabular}{|c|c|c|c|c|}
        \hline
        $\delta$ & $C_0$ & $C_1$ & $C_2$ & $C_3$ \\
        \hline
        $a$ & $C_1$ & $C_2$ & $C_3$ & $C_3$ \\
        \hline
        $b$ & $C_1$ & $C_2$ & $C_3$ & $C_3$ \\
        \hline
      \end{tabular}
      
  \end{itemize}      
    \end{multicols}
\end{solution}
}

%%% Local Variables:
%%% mode: latex
%%% TeX-master: "../eai"
%%% End:
