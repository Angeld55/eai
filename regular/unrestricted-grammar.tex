\section{Неограничени граматики}
\index{граматика!неограничена}
\label{sect:regular-grammar}
\mynote{На англ. {\em unrestricted grammar}. Това е тип 0 граматики в йерархията на Чомски \cite[стр. 220]{hopcroft1}.}
{\bf Неограничена граматика} e наредена четворка от вида
\[G = (V, \Sigma, R, S),\]
където:
\begin{itemize}
\item
  $V$ е крайно множество от {\em променливи} (нетерминали);
\item
  $\Sigma$ е крайно множество от {\em букви} (терминали), $\Sigma \cap V = \emptyset$;
\item
  \mynote{В \cite{hopcroft1} правилата се наричат {\em productions} или {\em production rules}.}
  $R \subseteq (V\cup\Sigma)^+ \times (V \cup \Sigma)^\star$ е крайно множество от {\em правила}.
  За по-добра яснота, обикновено правилата $(\alpha, \beta) \in R$ ще означаваме като 
  $\alpha \to_G \beta$.
\item
  $S \in V$ е началната променлива (нетерминал). 
\end{itemize}

\index{граматика!извод}
Удобно е да дефинираме извод на думата $\beta$ от думата $\alpha$ в граматиката $G$ за $\ell$ стъпки, което ще означаваме като $\alpha \derive{\ell}_G \beta$,
с индукция по броя на стъпките $\ell$ по следния начин:
\mynote{Обърнете внимание, че имаме недетерминизъм в тази дефиниция на извод. Също така, понякога ,за удобство, ще пишем просто $\derive{\ell}$ вместо $\derive{\ell}_G$,
  когато се знае за коя граматика говорим.}
\begin{prooftree}
  \AxiomC{}
  \UnaryInfC{$\alpha \derive{0}_G \alpha$}
\end{prooftree}

\begin{prooftree}
  \AxiomC{$\alpha \to_G \gamma$}
  \AxiomC{$\gamma \derive{\ell}_G \beta$}
  \BinaryInfC{$\alpha \derive{\ell+1}_G \beta$}
\end{prooftree}

\begin{prooftree}
  \AxiomC{$\alpha_1 \derive{\ell_1}_G \beta_1$}
  \AxiomC{$\alpha_2 \derive{\ell_2}_G \beta_2$}
  \BinaryInfC{$\alpha_1\cdot\alpha_2 \derive{\ell_1+\ell_2}_G \beta_1\cdot \beta_2$}
\end{prooftree}

\mynote{В частност имаме следното:
\begin{prooftree}
  \AxiomC{$\alpha \to_G \beta$}
  \UnaryInfC{$\alpha \derive{1}_G \beta$}
\end{prooftree}
}

Нека $\derive{\star}_G$ е рефлексивното и транзитивно затваряне на релацията $\derive{1}_G$. С други думи,
\[ \alpha \derive{\star}_G \beta\ \iff\ (\exists \ell\in\Nat)[\ \alpha \derive{\ell}_G \beta\ ].\]
Езикът, който се поражда от граматиката $G$ е следния:
\[\L(G) \df \{\omega \in \Sigma^\star \mid S \derive{\star}_G \omega\}.\]

\begin{proposition}\label{pr:grammar:add}
  За произволни думи $\alpha$ и $\beta$ е изпълнено, че:
  \begin{prooftree}
    \AxiomC{$\alpha \derive{\ell}_G \beta$}
    \AxiomC{$\gamma,\rho \in (V \cup \Sigma)^\star$}
    \BinaryInfC{$\gamma\alpha\rho \derive{\ell} \gamma \beta \rho$}
  \end{prooftree}
\end{proposition}

\begin{proposition}\label{pr:grammar:concat}
  За всяко $k$ е изпълнено, че:
  \begin{prooftree}
    \AxiomC{$\alpha_1 \derive{\ell_1} \beta_1$}
    \AxiomC{$\dots$}
    \AxiomC{$\alpha_k \derive{\ell_k} \beta_k$}
    \RightLabel{\scriptsize{$(\ell = \sum^k_{i=1} \ell_i)$}}
    \TrinaryInfC{$\alpha_1\cdots\alpha_k \derive{\ell} \beta_1\cdots\beta_k$}
  \end{prooftree}
\end{proposition}

%%% Local Variables:
%%% mode: latex
%%% TeX-master: "../eai"
%%% End:
