\section{Автомат на Майхил-Нероуд}\label{sect:myhill-nerode-theorem}
\index{Майхил-Нероуд}
\mynote{на англ. Myhill-Nerode}

\index{Майхил-Нероуд!релация}
\index{$\approx_L$}

Нека $L$ е език и нека $\alpha$ и $\beta$ са думи.
Казваме, че $\alpha$ и $\beta$ са {\bf еквивалентни относно} $L$, което записваме 
като $\alpha \approx_L \beta$, когато:
\[\alpha \approx_L \beta\ \dff\ \alpha^{-1}(L) = \beta^{-1}(L).\]
С други думи, 
\[\alpha \approx_L \beta \iff (\forall \omega \in \Sigma^\star)[\ \alpha\omega \in L \iff \beta\omega \in L\ ].\]
\mynote{$\approx_L$ е известна като релация на Майхил-Нероуд}
\begin{problem}
  Докажете, че за всяка дума $\alpha$ е изпълнено, че:
  \[\alpha \in L \iff [\alpha]_L \subseteq L.\]
\end{problem}


\mynote{Ще наричаме $\M$ автомат на Майхил-Нероуд. На практика във всеки учебник се разглежда автомата на Майхил-Нероуд вместо автомата на Бжозовски. Например, \cite[стр. 98]{papadimitriou}, \cite[стр. 65]{hopcroft1}, \cite[стр. 91]{sipser3}, \cite[стр. 89]{kozen}}


Ще дефинираме детерминиран автомат $\M = \FA$ по следния начин:
\begin{itemize}
\item
  $Q \df \{\ [\alpha]_L\mid \alpha\in \Sigma^\star\ \}$;
\item
  $\qstart \df [\varepsilon]_L$;
\item
  $F \df \{\ [\alpha]_L\mid \alpha\in L\ \}$;
\item
  Определяме функцията на преходите $\delta$ като 
  за всяка буква $b$ и всяка дума $\alpha$,
  \[\delta([\alpha]_L,b) \df [\alpha b]_L.\]
\end{itemize}

\begin{problem}
  Докажете, че $\delta:Q^\M \times \Sigma \to Q^\M$ е функция.
\end{problem}
\begin{hint}
  Трябда да докажете, че
  \[[\alpha]_L = [\beta]_L \implies \delta([\alpha]_L,b) = \delta([\beta]_L,b).\]
\end{hint}

\begin{problem}
  Докажете, че ако $\M$ е автоматът на Майхил-Нероуд за езика $L$, то $\L(\M) = L$.
\end{problem}
\begin{hint}
  Докажете, че за всеки две думи $\alpha$ и $\beta$ е изпълнено, че:
  \[\delta^\star_\M([\alpha]_L,\beta) = [\alpha\beta]_L.\]
  Оттук заключете директно, че $\L(\M) = L$.
\end{hint}

\begin{problem}
  Да разгледаме един език $L$.
  Нека $\B$ е автоматът на Бжозовски за $L$ и $\M$ е автоматът на Майхил-Нероуд за $L$.
  Да разгледаме $f:Q^\M \to Q^\B$ като
  \[f([\alpha]_L) = \hat{K} \dff K = \alpha^{-1}(L).\]
  Докажете, че $f$ е биекция.
\end{problem}

\begin{problem}[Теорема на Майхил-Нероуд]
  Докажете, че $L$ е регулярен език точно тогава, когато автоматът на Майхил-Нероуд $\M$ e краен.
\end{problem}

\begin{problem}
  Нека $L$ е регулярен език и $\M$ е автоматът на Майхил-Нероуд за $L$.
  Докажете, че $\M$ е минимален ДКА за $L$.
\end{problem}



%%% Local Variables:
%%% mode: latex
%%% TeX-master: "../eai"
%%% End:
