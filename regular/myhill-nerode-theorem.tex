\section{Автомат на Майхил-Нероуд}\label{sect:myhill-nerode-theorem}
\index{Майхил-Нероуд}
\mynote{на англ. Myhill-Nerode}

\index{Майхил-Нероуд!релация}
\index{$\approx_L$}

Нека $L$ е език и нека $\alpha$ и $\beta$ са думи.
Казваме, че $\alpha$ и $\beta$ са {\bf еквивалентни относно} $L$, което записваме 
като $\alpha \approx_L \beta$, когато:
\[\alpha \approx_L \beta\ \dff\ \alpha^{-1}(L) = \beta^{-1}(L).\]
С други думи, 
\[\alpha \approx_L \beta \iff (\forall \omega \in \Sigma^\star)[\ \alpha\omega \in L \iff \beta\omega \in L\ ].\]
\mynote{$\approx_L$ е известна като релация на Майхил-Нероуд}
\begin{problem}\label{prob:myhill-nerode-theorem-final}
  Докажете, че за всяка дума $\alpha$ е изпълнено, че:
  \[\alpha \in L \iff [\alpha]_L \subseteq L.\]
\end{problem}


\mynote{Ще наричаме $\M$ автомат на Майхил-Нероуд. На практика във всеки учебник се разглежда автомата на Майхил-Нероуд вместо автомата на Бжозовски. Например, \cite[стр. 98]{papadimitriou}, \cite[стр. 65]{hopcroft1}, \cite[стр. 91]{sipser3}, \cite[стр. 89]{kozen}}


Ще дефинираме детерминиран автомат $\M = \FA$ по следния начин:
\begin{itemize}
\item
  $Q \df \{\ [\alpha]_L\mid \alpha\in \Sigma^\star\ \}$;
\item
  $\qstart \df [\varepsilon]_L$;
\item
  $F \df \{\ [\alpha]_L\mid \alpha\in L\ \}$;
\item
  Определяме функцията на преходите $\delta$ като 
  за всяка буква $b$ и всяка дума $\alpha$,
  \[\delta([\alpha]_L,b) \df [\alpha b]_L.\]
\end{itemize}

\begin{problem}\label{prob:myhill-nerode-theorem-delta}
  Докажете, че $\delta:Q^\M \times \Sigma \to Q^\M$ е добре дефинирана функция.
\end{problem}
\mynote{С други думи, дефиницията на $\delta$ не зависи от избора на представител от класа $[\alpha]_L$, който сме направили.}
\begin{hint}
  Трябва да докажете, че
  \[[\alpha]_L = [\beta]_L \implies \delta([\alpha]_L,b) = \delta([\beta]_L,b).\]
\end{hint}

\begin{problem}\label{prob:myhill-nerode-theorem:language}
  Докажете, че ако $\M$ е автоматът на Майхил-Нероуд за езика $L$, то $\L(\M) = L$.
\end{problem}
\mynote{Индукция по дължината на думата $\beta$.

\mynote{
  % \begin{figure}[H]
      \begin{tikzpicture}[->,>=stealth,thick,node distance=50pt,scale=0.8,every node/.style={scale=0.8},initial text=начало]
        \tikzstyle{every state}=[circle,minimum size=20pt,auto]
        
        \node[state, initial below]   (1)              {$[\varepsilon]_L$};
        \node[state]                  (2) [right of=1] {$[\alpha]_L$};
        \node[state, inner sep=1pt]   (3) [right of=2] {$[\alpha\beta]_L$};
        
        \draw [photon] (1) -- node [above] {$\alpha$} (2);
        \draw [photon] (2) -- node [above] {$\beta$} (3);
      \end{tikzpicture}
      % \caption{Свойство (3)}
      % \end{figure}
}
}
\begin{hint}
  Докажете, че за всеки две думи $\alpha$ и $\beta$ е изпълнено, че:
  \[\delta^\star_\M([\alpha]_L,\beta) = [\alpha\beta]_L.\]
  Оттук заключете директно, че $\L(\M) = L$.
\end{hint}

\begin{problem}\label{prob:myhill-nerode-theorem:bijection}
  \mynote{Обърнете внимание, че тук не изискваме $L$ да е регулярен.}
  Да разгледаме един език $L$.
  Нека $\B$ е автоматът на Бжозовски за $L$ и $\M$ е автоматът на Майхил-Нероуд за $L$.
  Да разгледаме $f:Q^\M \to Q^\B$, където:
  \[f([\alpha]_L) = \hat{K} \dff K = \alpha^{-1}(L).\]
  Докажете, че $f$ е биекция.
\end{problem}

\mynote{Теоремата е доказана независимо от Майхил \cite{myhill-min} и Нероуд \cite{nerode-min}.}
\begin{problem}[Теорема на Майхил-Нероуд]\label{prob:myhill-nerode-theorem}
  Докажете, че $L$ е регулярен език точно тогава, когато автоматът на Майхил-Нероуд $\M$ e краен.
  Освен това, докажете, че $\M$ е минимален ДКА за $L$.
\end{problem}


\begin{example}
  Да разгледаме езика $L = \L(\mathbf{a\cdot(a+b)^\star\cdot b})$.
  \mynote{Да напомним, че в \Example{brzozowski-solved-examples-2} вече видяхме автоматът на Бжозовски за $L$.}
  \begin{figure}[H]

    \begin{subfigure}[t]{0.45\textwidth}
      \begin{tikzpicture}[framed,->,>=stealth,thick,node distance=55pt,initial text=начало,inner sep=1pt]
        \tikzstyle{every state}=[circle,minimum size=20pt,auto]
        
        \node[initial, state]                   (L) {$\hat{L}$};
        \node[state]                            (M) [above right of=L]{$\hat{M}$};
        \node[state]                            (E) [below right of=L]{$\hat{\emptyset}$};
        \node[state,accepting]                  (N) [right of=M]{$\hat{N}$};
        
        \path 
        (L) edge [bend left=15]  node [above] {$a$} (M)
        (L) edge [bend right=15] node [below] {$b$} (E)
        (E) edge [loop right]    node [right] {$a,b$} (E) 
        (M) edge [bend right=15] node [below] {$b$} (N)
        (M) edge [loop above]    node [above] {$a$} (M)
        (N) edge [bend right=30] node [above] {$a$} (M)
        (N) edge [loop above]    node [above] {$b$} (N);
      \end{tikzpicture}
      \caption{Автомат за $L$ по метода на Бжозовски.}      
    \end{subfigure}
    ~
    \begin{subfigure}[t]{0.45\textwidth}
      \begin{tikzpicture}[framed,->,>=stealth,thick,node distance=55pt,initial text=начало,inner sep=1pt]
        \tikzstyle{every state}=[circle,minimum size=20pt, auto]
        
        \node[initial, state]                   (L) {$[\varepsilon]_L$};
        \node[state]                            (M) [above right of=L]{$[a]_L$};
        \node[state]                            (E) [below right of=L]{$[b]_L$};
        \node[state,accepting]                  (N) [right of=M]{$[ab]_L$};
        
        \path 
        (L) edge [bend left=15]  node [left] {$a$} (M)
        (L) edge [bend right=15] node [left] {$b$} (E)
        (E) edge [loop right]    node [right] {$a,b$} (E) 
        (M) edge [bend right=15] node [below] {$b$} (N)
        (M) edge [loop above]    node [above] {$a$} (M)
        (N) edge [bend right=30] node [above] {$a$} (M)
        (N) edge [loop above]    node [above] {$b$} (N);
      \end{tikzpicture}
      \caption{Автомат за $L$ по метода на Майхил-Нероуд.}      
    \end{subfigure}
    
  \end{figure}

  Вече знаем, че този автомат е минимален за езика $L$.
  Да видим сега как директно можем да построим автомата на Майхил-Нероуд $\M$ за езика $L$ като използваме, че $\B \cong \M$.
  
\end{example}


\begin{example}
  Нека сега да приемем, че не знаем минималния автомат за езика $L$, а имаме следния автомат $\A$ за $L$:
  
      \begin{figure}[H]
    \centering
    \begin{tikzpicture}[framed,->,>=stealth,thick,node distance=55pt,initial text=начало]
      \tikzstyle{every state}=[circle]
      
      \node[initial, state]                   (L) {$q_0$};
      \node[state]                            (M) [above right of=L]{$q_1$};
      \node[state]                            (MM) [above right of=M]{$q_2$};
      \node[state]                            (E) [below right of=L]{$q_3$};
      \node[state]                            (EE)[right of=E]{$q_4$};
      \node[state,accepting]                  (N) [below right of=MM]{$q_5$};
      
      \path 
      (L) edge [bend left=15]  node [left] {$a$} (M)
      (L) edge [bend right=15] node [left] {$b$} (E)
      (E) edge [loop below]    node [below] {$a$} (E) 
      (M) edge [bend right=15] node [below] {$b$} (N)
      (M) edge [bend left=15] node [above] {$a$} (MM)
      (MM) edge [loop above]    node [above] {$a$} (MM)
      (MM) edge [bend right=15] node [left] {$b$} (N)
      (N) edge [bend right=30] node [above] {$a$} (MM)
      (N) edge [loop right]    node [right] {$b$} (N)
      (E) edge [bend left=15] node [above] {$b$} (EE)
      (EE) edge [bend left=15] node [below] {$a$} (E)
      (EE) edge [loop right]    node [right] {$b$} (EE); 
    \end{tikzpicture}

  \end{figure}
  

  Веднага се вижда, че $\L_\A(q_3) = \L_\A(q_4) = \emptyset$.
  Друг начин да се види равенството е по следния начин:
  \begin{align*}
    \L_\A(q_3) & = \{a\} \cdot \L_\A(q_3) \cup \{b\} \cdot \L_\A(q_4);\\
    \L_\A(q_4) & = \{a\} \cdot \L_\A(q_3) \cup \{b\} \cdot \L_\A(q_4).
  \end{align*}
  Това означава, че в минималния автомат състоянията $q_3$ и $q_4$ ще бъдат
  заменени от едно състояние.
  
  Лесно се вижда също, че $\L_\A(q_1) = \L_\A(q_2)$.
  Това е така, защото
  \begin{align*}
    \L_\A(q_1) & = \{a\} \cdot \L_\A(q_2) \cup \{b\}\cdot\L_\A(q_5); \\
    \L_\A(q_2) & = \{a\} \cdot \L_\A(q_2) \cup \{b\}\cdot\L_\A(q_5).
  \end{align*}
\end{example}




%%% Local Variables:
%%% mode: latex
%%% TeX-master: "../eai"
%%% End:
