\subsection{Затвореност относно конкатенация}

\begin{framed}
  \begin{lemma}
    \label{lem:concat}
    Класът на автоматните езици е затворен относно операцията {\em конкатенация}.
    Това означава, че ако $L_1$ и $L_2$ са два произволни автоматни езика, то $L_1\cdot L_2$
    също е автоматен език.
  \end{lemma}  
\end{framed}
\begin{proof}
  \mynote{Тук отново приемаме, че $Q_1 \cap Q_2 = \emptyset$.}
  Нека са дадени крайните детерминирани автомати:
  \begin{itemize}
  \item
    $\A_1 = \pair{\Sigma,Q_1,\delta_1,\qstart',F_1}$, където $\L(\A_1) = L_1$;
  \item
    $\A_2 = \pair{\Sigma,Q_2,\delta_2,\qstart'', F_2}$, където $\L(\A_2) = L_2$.
  \end{itemize}
  Ще дефинираме автомата $\N = \NFA$ по такъв начин, че
  \[\L(\N) = L_1\cdot L_2 = \L(\A_1)\cdot\L(\A_2).\]
  \begin{itemize}
  \item
    $Q \df Q_1 \cup Q_2$;
  \item
    $\qstart \df \qstart'$;
  \item
    $F \df \begin{cases}
      F_1 \cup F_2, & \text{ ако } \qstart'' \in F_2\\
      F_2,          & \text{ иначе}.
    \end{cases}$
  \item 
    $\Delta(q,a) \df
    \begin{cases}
      \{\delta_1(q,a)\},                      & \text{ ако }q\in Q_1\setminus F_1\ \&\ a\in\Sigma\\
      \{\delta_1(q,a), \delta_2(\qstart'',a)\}, & \text{ ако }q \in F_1\ \&\ a\in\Sigma\\
      \{\delta_2(q,a)\},                      & \text{ ако }q\in Q_2\ \&\ a\in\Sigma.
    \end{cases}$
  \end{itemize}
  Първо ще докажем, че
  \[\L(\A_1)\cdot\L(\A_2) \subseteq \L(\N).\]
  За целта, нека разгледаме думата $\alpha \in \L(\A_1)$ и $\beta \in \L(\A_2)$. Това означава, че имаме следните изчисления:
  \begin{align*}
    & (\qstart',\alpha) \vdash^\star_{\A_1} (q_1,\varepsilon), \text{за някое }q_1 \in F_1\\
    & (\qstart'',\beta) \vdash^\star_{\A_2} (q_2,\varepsilon), \text{за някое }q_2 \in F_2.
  \end{align*}
  Според дефиницията на недетерминирания краен автомат $\N$ е ясно, че:
  \[(\qstart',\alpha) \vdash^\star_{\N} (q_1,\varepsilon), \text{за някое }q_1 \in F_1.\]
  
  Ако $\beta = \varepsilon$, то това означава, че $\qstart'' \in F_2$ и следователно $F_1 \subseteq F$.
  Тогава получаваме, че $\alpha \cdot \beta = \alpha \in \L(\N)$, защото
  \[(\qstart',\alpha) \vdash^\star_{\N} (q_1,\varepsilon), \text{за някое }q_1 \in F_1 \subseteq F.\]

  Ако $\beta = b\gamma$, за някоя дума $\gamma \in \Sigma^\star$, то тогава можем да разбием изчислението на $\beta$ в $\A_2$ по следния начин:
  \[(\qstart'',b\gamma) \vdash_{\A_2} (q,\gamma) \vdash^\star_{\A_2} (q_2,\varepsilon), \text{за някое }q_2 \in F_2,\]
  където $q = \delta_2(\qstart'',b)$.
  
  Според дефиницията на недетерминирния краен автомат $\N$ е ясно, че:
  \[(q,\gamma) \vdash^\star_{\N} (q_2,\varepsilon), \text{за някое }q_2 \in F_2.\]
  Освен това, имаме, че $q \in \Delta(q_1,b)$, защото $q_1 \in F_1$. Това означава, че
  \[(q_1,b\gamma) \vdash_\N (q,\gamma).\]
  Съединявайки последните две изчисления, получаваме, че:
  \[(q_1,\beta) \vdash^\star_\N (q_2,\varepsilon),\text{ за някое }q_2 \in F_2.\]
  Сега съединяваме и изчислението за $\alpha$ и получаваме, че:
  \[(\qstart',\alpha\beta) \vdash^\star_\N (q_1,\beta) \vdash^\star_\N (q_2,\varepsilon),\text{ за някое }q_2 \in F_2.\]
  Оттук заключаваме, че във всички случаи за $\beta$, $\alpha \cdot \beta \in \L(\N)$.

  Сега ще докажем, че
  \[\L(\N) \subseteq \L(\A_1) \cdot \L(\A_2).\]
  За целта, нека разгледаме думата $\omega \in \L(\N)$, където $\omega = a_0a_1\cdots a_{n-1}$.
  Да разгледаме една редица от състояния $(q_i)^{n}_{i=0}$, която описва приемащо изчисление на $\N$ върху $\omega$.
  \mynote{Възможно е да има и други редици от състояния $(p_i)^{n}_{i=0}$, които да описват приемащи изчисления на $\N$ върху $\omega$.}
  Това означава, че:
  \begin{itemize}
  \item
    $q_0 = \qstart$;
  \item
    $q_{i+1} \in \Delta(q_i,a_i)$ за $i < n$;
  \item
    $q_n \in F$.
  \end{itemize}
  
  Ако $q_n \in F_1$, то според конструкцията на $\N$, $\varepsilon \in \L(\A_2)$ и всяко състояние от $(q_i)^{n}_{i=0}$ принадлежи на $Q_1$ и оттам $\omega \in \L(\A_1)$.
  Интересният случай е когато $q_n \in F_2$.
  Според конструкцията на $\N$, не можем да преминем от състояние от $Q_2$ в състояние от $Q_1$.
  Това означава, че можем да разбием редицата от състояния $(q_i)^n_{i=0}$ на две непразни подредици:
  \begin{itemize}
  \item
    $(q_{i})^{\ell}_{i=0}$ - тези които са от $Q_1$,
  \item
    $(q_i)^{n}_{i=\ell+1}$ - тези, които са от $Q_2$.
  \end{itemize}
  Нека $\alpha = a_0a_1\cdots a_{\ell-1}$ и $\beta = a_{\ell}a_{\ell+1}\cdots a_{n-1}$.
  Ясно е, че:
  \[(q_0,\alpha\beta) \vdash^\star_\N (q_{\ell}, \underbrace{a_{\ell}a_{\ell+1}\cdots a_{n-1}}_{\beta}) \vdash_\N (q_{\ell+1},a_{\ell+1}\cdots a_{n-1}) \vdash^\star_\N (q_n,\varepsilon).\]
  От конструкцията на $\N$ следва, че редицата от състояния $(q_i)^{\ell}_{i=0}$ описва приемащо изчисление на $\A_1$ върху $\alpha$.
  \mynote{Това е единственият начин да направим преход от състояние на $Q_1$ към състояние на $Q_2$.}
  Също така от конструкцията следва, че щом $q_{\ell+1} \in \Delta(q_\ell,a_\ell)$, то $q_{\ell} \in F_1$ и $\delta_2(\qstart'',a_\ell) = q_{\ell+1}$. Заключаваме, че:
  \begin{itemize}
  \item
    $(q_0, \alpha) \vdash^\star_{\A_1} (q_\ell,\varepsilon)$.
    Понеже $q_0 = \qstart'$ и $q_\ell \in F_1$, то $\alpha \in \L(\A_1)$.
  \item
    $(\qstart'', \beta) \vdash^\star_{\A_2} (q_n,\varepsilon)$.
    Понеже $q_n \in F_2$, то $\beta \in \L(\A_2)$.
  \end{itemize}
\end{proof}

\begin{figure}[H]
  \center
  \begin{subfigure}[b]{0.3\textwidth}
    \label{subf:a1}
    \begin{tikzpicture}[framed,->,>=stealth,thick,node distance=45pt]
      \tikzstyle{every state}=[circle,minimum size=15pt,auto]
      \node[initial below,state,accepting]      (1) {$q'_0$};
      \node[state]                        (2) [right of=1] {$q_1$};
      \node[state]                        (3) [above right of=2] {$q_2$};
      \node[state,accepting]              (4) [below right of=2] {$q_3$};
      \path
      (1) edge [loop above] node [above] {$b$} (1)
      (1) edge node [above] {$a$} (2)
      (2) edge node [above] {$a$} (3)
      (3) edge [loop right] node [right] {$b$} (3)
      (2) edge node [below] {$b$} (4)
      (3) edge [bend right=30] node [above] {$a$} (1)
      (4) edge [bend right=15] node [right] {$a$} (3)
      (4) edge [bend left=30] node [below] {$b$} (1);
    \end{tikzpicture}
    \caption{автомат $\A_1$}
  \end{subfigure}
  \qquad
  \qquad
  \qquad
  \begin{subfigure}[b]{0.3\textwidth}
    \begin{tikzpicture}[framed,->,>=stealth,thick,node distance=45pt]
      \tikzstyle{every state}=[circle,minimum size=15pt,auto]
      \node[initial,state]                (1) {$q''_0$};
      \node[state]     [above right of=1] (2) {$q_4$};
      \node[state,accepting]     [below right of=1] (3) {$q_5$};
      \path
      (1) edge [bend left=15] node  [above] {$a$} (2)
      (2) edge [loop above] node [above] {$b$} (2)
      (2) edge [bend left=15] node  [right] {$a$} (3)
      (3) edge [loop right]  node [right] {$a,b$} (3)
      (1) edge [bend right=15] node [below] {$b$} (3);
    \end{tikzpicture}
    \caption{автомат $\A_2$}
  \end{subfigure}
\end{figure}

\begin{example}
    За да построим автомат, който разпознава конкатенацията на $\L(\A_1)$ и $\L(\A_2)$,
    трябва да свържем финалните състояния на $\A_1$ с изходящите от $s_2$ състояния на $\A_2$.
    
    \begin{figure}[H]
      \center
      % \begin{subfigure}[b]{0.3\textwidth}
      \begin{tikzpicture}[framed,->,>=stealth,thick,node distance=2cm]
        \tikzstyle{every state}=[circle,minimum size=15pt,auto]
        \node[initial,state]                      (1) {$q'_0$};
        \node[state] [right of=1]                 (2) {$q_1$};
        \node[state] [above right of=2]           (3) {$q_2$};
        \node[state] [below right of=2]           (4) {$q_3$};
        \node[state] [right=4cm of 1]             (5) {$q''_0$};
        \node[state] [above right of=5]           (6) {$q_4$};
        \node[state,accepting] [below right of=5] (7) {$q_5$};
        \path
        (1) edge [loop above] node [above] {$b$} (1)
        (1) edge node [above]                         {$a$} (2)
        (2) edge node [above]                         {$a$} (3)
        (2) edge node [below]                         {$b$} (4)
        (3) edge [loop right] node [right]            {$b$} (3)
        (6) edge [loop above] node [above]            {$b$} (6)
        (7) edge [loop right] node [right]            {$a,b$} (7)
        (3) edge [bend right=15] node [above]         {$a$} (1)
        (4) edge [bend left=15] node [below]          {$b$} (1)
        (4) edge [bend left=15] node [left]          {$a$} (3)
        (5) edge [bend left=15] node [below]          {$a$} (6)
        (6) edge [bend left=15] node [right]          {$a$} (7)
        (5) edge [bend right=15] node [above]         {$b$} (7)
        (1) edge [dashed, bend left=45] node [above]  {$a$} (6)
        (1) edge [dashed, bend right=45] node [below] {$b$} (7)
        (4) edge [dashed, bend left=30] node [above]  {$a$} (6)
        (4) edge [dashed, bend left=10] node [above]  {$b$} (7);
      \end{tikzpicture}
      \caption{$\L(\N) = \L(\A_1)\cdot\L(\A_2)$}
  \end{figure}  
  Обърнете внимание, че $\A_1$ и $\A_2$ са детерминирани автомати, но $\N$ е недетерминиран.
  Също така, в този пример се оказва, че вече $q''_0$ е недостижимо състояние, но в общия случай не можем да 
  го премахнем, защото може да има преходи влизащи в $q''_0$.
\end{example}

%%% Local Variables:
%%% mode: latex
%%% TeX-master: "../eai"
%%% End:
