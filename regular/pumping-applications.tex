\subsection*{Приложения}

В този раздел ще видим, че с помощта на \hyperref[lem:regular:pumping]{лемата за покачването} някои основни проблеми са алгоритмично разрешими за автоматни езици. По-нататък ще разгледаме тези проблеми и за други видове езици и ще видим, че не винаги те са алгоритмично разрешими.

\begin{important}
\begin{description}
\item[(Проблемът за празнота)]
  Същестува ли алгоритъм, който по вход краен автомат $\A$,
  определя дали $\L(\A) = \emptyset$ ?
\item[(Проблемът за включването)]
  Същестува ли алгоритъм, който по вход два крайни автомата $\A$ и $\B$,
  определя дали $\L(\A) \subseteq \L(\B)$ ?
\item[(Проблемът за равенство)]
  Същестува ли алгоритъм, който по вход два крайни автомата $\A$ и $\B$,
  определя дали $\L(\A) = \L(\B)$ ?
\item[(Проблемът за безкрайност)]
  Същестува ли алгоритъм, който по вход краен автомат $\A$,
  определя дали $|\L(\A)| = \infty$ ?
\item[(Проблемът за универсалност)]
  Същестува ли алгоритъм, който по вход краен автомат $\A$,
  определя дали $\L(\A) = \Sigma^\star$ ?
\end{description}
\end{important}

\begin{proposition}
  \label{pr:regular:pumping-applications:emptiness}
  Езикът на крайния детерминиран автомата $\A = \FA$ е {\em непразен} точно тогава, когато съдържа дума $\alpha$,
  за която $\abs{\alpha} < \abs{Q}$.
\end{proposition}
\begin{proof}
  Ще разгледаме двете посоки на твърдението.
  \begin{description}
  \item[$(\Rightarrow)$]
    Нека $L$ е непразен език и нека $m = \min\{\abs{\alpha} \mid \alpha \in L\}$.
    Ще докажем, че $m < \abs{Q}$.    
    За целта, да допуснем, че $m \geq \abs{Q}$ и да изберем $\alpha \in L$, за която $\abs{\alpha} = m$.
    Според доказателството на \Lemma{regular:pumping}, съществува разбиване $xyz = \alpha$, 
    такова че $xz \in L$.
    При положение, че $\abs{y} \geq 1$, то $\abs{xz} < m$, което 
    е противоречие с минималността на думата $\alpha$.
    Заключаваме, че нашето допускане е грешно. Тогава $m < \abs{Q}$, откъдето следва, че 
    съществува дума $\alpha \in L$ с $\abs{\alpha} < \abs{Q}$.
  \item[$(\Leftarrow)$]
    Тази посока е тривиална.
    Ако $L$ съдържа дума $\alpha$, за която $\abs{\alpha} < \abs{Q}$,
    то е очевидно, че $L$ е непразен език.
  \end{description}
\end{proof}

% \begin{corollary}
%   \mynote{\writedown Обяснете!}
%   Съществува алгоритъм, който разрешава проблема за празнота на автоматните езици.
% \end{corollary}

% \begin{corollary}
%   \label{cor:pumping-applications:equal-languages}
%   Съществува алгоритъм, който разрешава проблема за включването за автоматни езици.
% \end{corollary}
% \begin{hint}
%   Съобразете, че за всеки два езика $L_1$ и $L_2$ е изпълнено:
%   \[L_1 \subseteq L_2 \iff (L_1\setminus L_2) = \emptyset.\]
% \end{hint}

\begin{proposition}
  \label{pr:regular:pumping-applications:infinity}
  Регулярният език $L$, 
  разпознаван от КДА $\A$, е {\em безкраен} точно тогава, когато съдържа дума $\alpha, \abs{Q} \leq \abs{\alpha} < 2\abs{Q}$.
\end{proposition}
\begin{proof}
  Да разгледаме двете посоки на твърдението.
  \begin{description}
  \item[$(\Leftarrow)$]
    Нека $L$ е регулярен език, за който съществува дума $\alpha$, такава че $\abs{Q} \leq \abs{\alpha} < 2\abs{Q}$.
    Тогава от \Lemma{regular:pumping} следва, че съществува разбиване $\alpha = xyz$ със свойството, че
    за всяко $i \in \Nat$, $xy^iz \in L$. Следователно, $L$ е безкраен, защото $\abs{y} \geq 1$.
  \item[$(\Rightarrow)$]
    Нека $L$ е безкраен език и % да приемем, че няма думи $\alpha$ със
    % свойството $\abs{Q} \leq \abs{\alpha} <  2\abs{Q}$.
    да вземем {\em най-късата} дума $\alpha \in L$, за която $\abs{\alpha} \geq 2\abs{Q}$.
    Понеже $L$ е безкраен, знаем, че такава дума съществува.
    Тогава отново по \Lemma{regular:pumping}, имаме следното разбиване на $\alpha$:
    \[\alpha = xyz,\ \abs{xy} \leq \abs{Q},\ 1\leq \abs{y},\ xz \in L.\]
    Но понеже $\abs{xyz} \geq 2\abs{Q}$, а $1 \leq \abs{y} \leq \abs{Q}$, то $\abs{xyz} > \abs{xz} \geq \abs{Q}$ и понеже избрахме $\alpha = xyz$
    да бъде най-късата дума с дължина поне $2\abs{Q}$, заключаваме, че $\abs{Q} \leq \abs{xz} < 2\abs{Q}$ и $xz \in L$.
  \end{description}
\end{proof}

% \begin{corollary}
%   \mynote{\writedown Обяснете!}
%   Съществува алгоритъм, който разрешава проблема за безкрайност за автоматни езици.
% \end{corollary}


Сега вече много лесно можем да видим, че следните проблеми са алгоритмично разрешими за автоматните езици.
\begin{description}
\item[(Проблемът за празнота)]
  Според \Proposition{regular:pumping-applications:emptiness}, \Algorithm{regular:pumping-applications:emptiness} разрешава проблемът за празнота за автоматни езици.
  \begin{algorithm}[H]
    \caption{Проблемът за празнота за автоматни езици}
    \label{alg:regular:pumping-applications:emptiness}
    \begin{algorithmic}[1]
      \Procedure{\texttt{isEmpty}}{$\A$}
      \ForAll{$n < |Q|$}
      \ForAll{$\omega \in \Sigma^n$}
      \If{$\texttt{belong}(\A,\omega)$}
      \State \Return \texttt{True}
      \EndIf
      \EndFor
      \EndFor
      \State \Return \texttt{False}
      \EndProcedure
    \end{algorithmic}
  \end{algorithm}
\item[(Проблемът за включване)]
  Тук единствено трябва да съобразим, че за всеки два езика $L_1$ и $L_2$ е изпълнена еквивалентността:
  \[L_1 \subseteq L_2 \iff (L_1\setminus L_2) = \emptyset.\]
  
  \begin{algorithm}[H]
    \caption{Проблемът за включване за автоматни езици}
    \label{alg:regular:pumping-applications:inclusion}
    \begin{algorithmic}[1]
      \Procedure{\texttt{isIncluded}}{$\A,\B$}
      \State $\B_1 := \texttt{complement}(\B)$
      \State $\A_1 := \texttt{intersect}(\A,\B_1)$
      \State \Return $\texttt{isEmpty}(\A_1)$
      \EndProcedure
    \end{algorithmic}
  \end{algorithm}
\item[(Проблемът за равенство)]
  Понеже е изпълнена еквивалентността
  \[L_1 = L_2 \iff L_1 \subseteq L_2\ \&\ L_2 \subseteq L_1,\]
  то е ясно, че проблемът за равенство на два автоматни езика е алгоритмично разрешим,
  защото просто трябва да приложим два пъти \Algorithm{regular:pumping-applications:inclusion}.
\item[(Проблемът за безкрайност)]
  От \Proposition{regular:pumping-applications:infinity} директно получаваме следния прост алгоритъм,
  който разрешава проблемът за безкрайност на автоматен език.
    \begin{algorithm}[H]
    \caption{Проблемът за безкрайност за автоматни езици}
    \label{alg:regular:pumping-applications:infinity}
    \begin{algorithmic}[1]
      \Procedure{\texttt{isInfinite}}{$\A$}
      \ForAll{$n := |Q|,...,2|Q|-1$}
      \ForAll{$\omega \in \Sigma^n$}
      \If{$\texttt{belong}(\A,\omega)$}
      \State \Return \texttt{True}
      \EndIf
      \EndFor
      \EndFor
      \State \Return \texttt{False}
      \EndProcedure
    \end{algorithmic}
  \end{algorithm}
\item[(Проблемът за универсалност)]
  Този проблем отново е алгоритмично разрешим, защото имаме еквивалентността:
  \[L = \Sigma^\star \iff \overline{L} = \emptyset.\]
\end{description}


%%% Local Variables:
%%% mode: latex
%%% TeX-master: "../eai"
%%% End:
