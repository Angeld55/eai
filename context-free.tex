\chapter{Безконтекстни езици и стекови автомати}


\section{Безконтекстни граматики}
\index{граматика!безконтекстна}
% От Сипсер, същото е в слайдовете на Сашка
% Малко е тъпо, че в Пападимитриу дефиницията е различна. Там \Sigma \subseteq V

\begin{dfn}
  \marginpar{На англ. {\em context-free grammar}}
  \marginpar{Други срещани наименования на български са {\em контекстно-свободна}, {\em контекстно-независима}}
%  \marginpar{Според йерархията на Чомски, това са граматики тип }
  Безконтекстна граматика e четворка от вида
  \[G = (V,\Sigma,R,S),\]
  където
  \begin{itemize}
  \item
    \marginpar{Променливите се наричат също нетерминали}
    $V$ е крайно множество от {\em променливи};
  \item
    \marginpar{Буквите се наричат също терминали.}
    $\Sigma$ е крайно множество от {\em букви}, $\Sigma \cap V = \emptyset$;
  \item
    $R \subseteq V\times (V\cup\Sigma)^\star$, крайно множество от {\em правила};
  \item
    $S \in V$ е началната променлива. 
  \end{itemize}
\end{dfn}

При дадена граматика $G$, за правилата на граматиката обикновено ще пишем $A \rightarrow \alpha$ вместо $(A,\alpha) \in R$.
Ще въведем и релация между думи $\alpha,\beta\in (V \cup \Sigma)^\star$, която ще казва, че думата $\beta$
се получаава от $\alpha$ като приложим правло от граматиката.
За две думи $u,v\in (V\cup\Sigma)^\star$ ще пишем $u \rightarrow_G v$, ако съществуват думи $x,y\in (\Sigma\cup V)^\star$, $A\in V$,
правило $A\rightarrow \alpha$ и $u = xAy$, $v = x\alpha y$.
С $\rightarrow^\star_G$ ще означаваме рефлексивното и транзитивно затваряне на релацията $\rightarrow_G$.
% \marginpar{Да се дефинира $\rightarrow^\star_G$}

Езикът породен от граматиката $G$ е множеството от думи
\[\L(G) = \{\alpha\in\Sigma^\star\mid S \rightarrow^\star_G \alpha\}.\]
  
\begin{problem}
  Докажете, че езикът $L = \{a^mb^nc^k\mid m+n \geq k\}$ е безконтекстен.
\end{problem}
\begin{proof}
  Да разгледаме граматиката $G$ с правила
  \begin{align*}
    S& \rightarrow aSc\vert aS \vert B\\
    B& \rightarrow bBc\vert bB\vert\varepsilon.
  \end{align*}
  
  Лесно се вижда с индукция по $n$, че за всяко $n$ имаме свойствата:
  \marginpar{\ding{45} Докажете!}
  \begin{itemize}
  \item 
    $S \rightarrow^\star a^nSc^n$,
  \item
    $S \rightarrow^\star a^nS$,
  \item
    $B \rightarrow^\star a^nBc^n$,
  \item
    $B \rightarrow^\star b^nB$.
  \end{itemize}
  Комбинирайки горните свойства, можем да видим, че за всяко $n \geq k$,
  \begin{itemize}
  \item 
    $S \rightarrow^\star a^nSc^k$,
  \item
    $B \rightarrow^\star b^nBc^k$.
  \end{itemize}
  За да докажем, че $L \subseteq L(G)$, 
  да разгледаме една дума $\omega \in L$, т.е. $\omega = a^mb^nc^k$, където $m+n \geq k$.
  Имаме два случая:
  \begin{itemize}
  \item 
    $k \leq m$, т.е. $m = k+l$ и $m+n = k+l+n$.
    Тогава имаме изводите:
    \[S \rightarrow^\star a^kSc^k,\ S \rightarrow^\star a^lS,\ S \rightarrow B,\ B \rightarrow^\star b^nB,\ B \rightarrow \varepsilon.\]
    Обединявайки всичко това, получаваме:
    \[S \rightarrow^\star a^mb^nc^k.\]
  \item
    $k > m$, т.е. $k = m+l$, за някое $l > 0$, и $m+n = k+r = m+l+r$, за някое $r$.
    Тогава имаме изводите:
    \[S \rightarrow^\star a^mSc^m,\ S\rightarrow B,\ B\rightarrow^\star b^lBc^l,\ B\rightarrow b^rB,\ B\rightarrow\varepsilon,\]
    и отново получаваме $S \rightarrow^\star a^mb^nc^k$.
  \end{itemize}
  Така доказахме, че $\omega \in \L(G)$.
  
  Сега ще докажем, че $\L(G) \subseteq L$.
  С индукция по дължината на извода $l$,
  ще докажем, че ако $S \stackrel{l}{\rightarrow}\omega$, то $\omega \in M$, където
  \[M = \{a^nSc^k\mid n\geq k\}\cup\{a^nb^mBc^k\mid n+m\geq k\}\cup\{a^nb^mc^k\mid n+m\geq k\}.\]
  
  Ако $l = 0$, то е ясно, че $S \stackrel{0}{\rightarrow} S$ и $S \in M$.

  Нека $l > 0$ и $S \stackrel{l-1}{\rightarrow} \alpha \rightarrow \omega$.
  От {\bf И.П.} имаме, че $\alpha \in M$. Нека $\omega$ се получава от $\alpha$ с прилагане на правилото $C \rightarrow \gamma$.
  Разглеждаме всички варианти за думата $\alpha \in M$ и за правилото $C\rightarrow \gamma$ в граматиката $G$
  за да докажем, че  $\omega \in M$.
  Удобно е да представим всички случаи в таблица.
  \begin{center}
    \begin{tabular}{| c | c | c |}
      \hline
      $\alpha\in M$ & $C \rightarrow \gamma$ & $\omega \in M?$ \\ \hline
      $a^nSc^k$ & $S \rightarrow aSc$ & $a^{n+1}Sc^{k+1}$ \\ \hline
      $a^nSc^k$ & $S \rightarrow aS$ & $a^{n+1}Sc^{k}$ \\ \hline
      $a^nSc^k$ & $S \rightarrow B$ & $a^{n}Bc^{k}$ \\ \hline
      $a^nb^mBc^k$ & $B \rightarrow bBc$ & $a^nb^{m+1}Bc^{k+1}$\\ \hline
      $a^nb^mBc^k$ & $B \rightarrow bB$ & $a^nb^{m+1}Bc^{k}$\\ \hline
      $a^nb^mBc^k$ & $B \rightarrow \varepsilon$ & $a^nb^{m}c^{k}$\\ \hline
    \end{tabular}
  \end{center}
  Във всички случаи се установява, че $\omega \in M$.
  Сега, за всяка дума $\omega \in L(G)$ следва, че
  \[\omega \in \Sigma^\star \cap M = \{a^mb^nc^k\mid m+n \geq k\}.\]
\end{proof}


\begin{problem}
  \marginpar{
    $S \to aS \mid aSc \mid aB \mid bB$\\
    $B \to bB \mid bBc \mid \varepsilon$
}
  Докажете, че езикът $L = \{a^mb^nc^k\mid m+n \geq k + 1\}$ е безконтекстен.  
\end{problem}

\begin{problem}
  \label{pr:nanb}
  Нека $\omega$ е произволна дума над азбуката $\{a,b\}$. 
  Тогава:
  \begin{enumerate}[a)]
  \item 
    ако $n_a(\omega) = n_b(\omega) + 1$, то съществуват думи $\omega_1$, $\omega_2$, за които
    $\omega = \omega_1 a \omega_2$, $n_a(\omega_1) = n_b(\omega_1)$ и $n_a(\omega_2) = n_b(\omega_2)$.
  \item
    ако $n_b(\omega) = n_a(\omega) + 1$, то съществуват думи $\omega_1$, $\omega_2$, за които
    $\omega = \omega_1 b \omega_2$, $n_a(\omega_1) = n_b(\omega_1)$ и $n_a(\omega_2) = n_b(\omega_2)$.
  \end{enumerate}
\end{problem}
\begin{proof}
  Пълна индукция по дължината на думата $\omega$, за които $n_a(\omega) = n_b(\omega)+1$.
  \begin{itemize}
  \item 
    $\abs{\omega} = 1$. Тогава $\omega_1 = \omega_2 = \varepsilon$ и $\omega = a$.
  \item
    $\abs{\omega} = n+1$. Ще разгледаме два случая, в зависимост от първия символ на $\omega$.
    \begin{itemize}
    \item 
      Случаят $\omega = a\omega'$ е лесен. (Защо?)
    \item
      Интересният случай е $\omega = b\omega'$.    
      Тогава $\omega = b^{i+1}a\omega'$. Да разгледаме думата $\omega''$, която се получава от $\omega$
      като премахнем първото срещане на думата $ba$, т.е. 
      $\omega'' = b^i\omega'$ и $\abs{\omega''} = n-1$.
      Понеже от $\omega$ сме премахнали равен брой $a$-та и $b$-та, $n_a(\omega'') = n_b(\omega'')+1$.
      Според {\bf И.П.} за $\omega''$, можем да запишем думата като $\omega'' = \omega''_1a\omega''_2$
      и $n_a(\omega''_1) = n_b(\omega''_1)$, $n_a(\omega''_2) = n_b(\omega''_2)$.
      Понеже $b^i$ е префикс на $\omega''_1$, за да получим обратно $\omega$, трябва 
      да прибавим премахнатата част $ba$ веднага след $b^i$ в $\omega''_1$.
    \end{itemize}
  \end{itemize}
\end{proof}

\begin{problem}
  За произволна дума $\omega \in \{a,b\}^\star$, 
  докажете, че ако $n_a(\omega) > n_b(\omega)$, то съществуват думи $\omega_1$ и $\omega_2$,
  за които $\omega = \omega_1 a \omega_2$ и $n_a(\omega_1) \geq n_b(\omega_1)$, $n_a(\omega_2) \geq n_b(\omega_2)$.
\end{problem}

\begin{problem}
  Да се докаже, че езикът $L = \{\alpha \in \{a,b\}^\star\mid n_a(\alpha) = n_b(\alpha)\}$ 
  е безконтекстен.
\end{problem}
\begin{proof}
  \marginpar{  Алтернативна граматика за езика $L$ е
  \begin{align*}
    S& \rightarrow aB\vert bA\\
    A& \rightarrow a\vert aS\vert bAA\\
    B& \rightarrow b\vert bS\vert aBB
  \end{align*}}
  Една възможна граматика $G$ е следната: 
  \[S \rightarrow aSbS\vert bSaS \vert\varepsilon.\]
  Например, да разгледаме извода на думата $aabbba$ в тази граматика:
  \begin{align*}
    S & \to aSbS \to aaSbSbS \to aa\varepsilon bSbS \to aab\varepsilon bS \to aabbbSaS\\
    & \to aabbb\varepsilon a S \to aabbba.
  \end{align*}
  
  Като следствие от \Prob{nanb} може лесно да се изведе, че за думи $\omega$, за които $n_a(\omega) = n_b(\omega)$,
  е изпълнено следното:
  \begin{enumerate}[a)]
  \item 
    ако $\omega = a\omega'$, то
    $\omega = a\omega_1b\omega_2$ и $n_a(\omega_1) = n_b(\omega_1)$, $n_a(\omega_2) = n_b(\omega_2)$;
  \item
    ако $\omega = b\omega'$, то
    $\omega = b\omega_1a\omega_2$ и $n_a(\omega_1) = n_b(\omega_1)$, $n_a(\omega_2) = n_b(\omega_2)$.
  \end{enumerate}

  Сега първо ще проверим, че $L \subseteq L(G)$.
  За целта ще докажем с {\em пълна индукция} по дължината на думата $\omega$, че за всяка дума $\omega$ със свойството $n_a(\omega) = n_b(\omega)$ е изпълнено
  $S \rightarrow^\star \omega$.
  \begin{itemize}
  \item 
    Нека $\abs{\omega} = 0$. Тогава $S \rightarrow \varepsilon$.
  \item
    Нека $\abs{\omega} = k+1$. Имаме два случая.
    \begin{itemize}
    \item 
      $\omega = a\omega^\prime$, т.е. от свойство а), $\omega = a\omega_1b\omega_2$ и $n_a(\omega_1) = n_b(\omega_1)$, $n_a(\omega_2) = n_b(\omega_2)$.
      Тогава $\abs{\omega_1} \leq k$ и по И.П. $S \rightarrow^\star \omega_1$.
      Аналогично, $S \rightarrow^\star \omega_2$.
      Понеже имаме правило $S \rightarrow aSbS$, заключаваме че $S \rightarrow^\star a\omega_1b\omega_2$.
    \item
      $\omega = b\omega^\prime$, т.е. свойство б), $\omega = b\omega_1a\omega_2$ и $n_a(\omega_1) = n_b(\omega_1)$, $n_a(\omega_2) = n_b(\omega_2)$.
      Този случай се разглежда аналогично.
    \end{itemize}
  \end{itemize}
  
  Преминаваме към доказателството на другата посока, т.е. $L(G) \subseteq L$.
  Тук с индукция по дължината на извода $l$ ще докажем, че
  $S \stackrel{l}{\rightarrow} \omega$, то $\omega \in M$,
  където
  \[M = \{\omega \in \{a,b,S\}^\star \mid n_a(\omega) = n_b(\omega)\}.\]
  За $l = 0$  е ясно, че $S \stackrel{0}{\rightarrow^\star} S$.
  За $l = k+1$, то $S \stackrel{k}{\rightarrow^\star} \alpha \rightarrow \omega$.
  От {\bf И.П.} имаме, че $\alpha \in M$.
  Нека $\omega$ се получава от $\alpha$ с прилагане на правилото $C \rightarrow \gamma$.
  Разглеждаме всички варианти за думата $\alpha \in M$ и за правилото $C\rightarrow \gamma$ в граматиката $G$
  за да докажем, че  $\omega \in M$.
  Удобно е да представим всички случаи в таблица.
  \begin{center}
    \begin{tabular}{| c | c | c |}
      \hline
      $\alpha$ & $C \rightarrow \gamma$ & $\omega$ \\ \hline
      $\in M$ & $S \rightarrow aSbS$ & $\in M$ \\ \hline
      $\in M$ & $S \rightarrow bSaS$ & $\in M$ \\ \hline
      $\in M$ & $S \rightarrow \varepsilon$ & $\in M$ \\ \hline
    \end{tabular}
  \end{center}
  Във всички случаи лесно се установява, че $\omega \in M$.
  Така за всяка дума $\omega \in L(G)$ следва, че
  \[\omega \in \Sigma^\star \cap M = L.\]
\end{proof}

\begin{problem}
  Докажете, че следните езици са безконтекстни.
  \begin{enumerate}[a)]
  \item
    \marginpar{$S \rightarrow aSa\ \vert\ bSb\ \vert\ \varepsilon$}
    $L = \{ww^R \mid w \in \{a,b\}^\star\}$;
  \item
    \marginpar{$S \rightarrow aSa\ \vert\ bSb\ \vert\ a\vert\ b\ \vert\ \varepsilon$}
    $L = \{w \in \{a,b\}^\star \mid w = w^R\}$;
  \item
    $L = \{a^nb^{2m}c^{n} \mid m,n \in \Nat\}$;
  \item
    $L = \{a^nb^{m}c^{m}d^n \mid m,n \in \Nat\}$;
  \item
    $L = \{a^nb^{2k} \mid n,k \in \Nat\ \&\ n \neq k\}$;
  \item
    \marginpar{$S \rightarrow aSb | aS | a$}
    $L = \{a^nb^k \mid n > k\}$;
  \item
    $L = \{a^nb^k \mid n \geq 2k\}$;
  \item
    \marginpar{$S \rightarrow aSc | B,\ B \rightarrow bBc | \varepsilon$}
    $L = \{a^nb^mc^{n+m}\mid n,m \in \Nat\}$;
  \item
    \marginpar{$S \rightarrow aSc | aS | B$, $B\rightarrow bBc | bB | \varepsilon$}
    $L = \{a^nb^kc^m \mid n + k \geq m\}$;
  \item
    \marginpar{$S \rightarrow aSc | aS | aB | bB$,\\$B\rightarrow bBc | bB | \varepsilon$}
    $L = \{a^nb^kc^m \mid n + k \geq m+1\}$;
  \item
    $L = \{a^nb^kc^m \mid n + k \geq m+2\}$;
  \item
    \marginpar{$S \rightarrow aSc | aS | B | Bc$,\\$B\rightarrow bBc | bB | \varepsilon$}
    $L = \{a^nb^kc^m \mid n + k + 1 \geq m\}$;
  \item
    $L = \{a^nb^kc^m \mid n + k + 2 \geq m\}$;
  \item
    $L = \{a^nb^kc^m \mid n + k \leq m\}$;
  \item
    $L = \{a^nb^kc^m \mid n + k \leq m+1\}$;
  \item
    \marginpar{Обединение на три езика}
    $L = \{a^nb^mc^k \mid n, m, k \text{ не са страни на триъгълник}\}$.
  \item
    $L = \{a,b\}^\star \setminus \{a^{2n}b^n \mid n\in\Nat\}$;
  \item
    \marginpar{$S\to EaE$, $E \to aEbE | bEaE | \varepsilon$}
    $L = \{\alpha \in \{a,b\}^\star\mid n_a(\alpha) = n_b(\alpha) + 1\}$;
  \item
    \marginpar{$S\to E | SaS$, $E \to aEbE | bEaE | \varepsilon$}
    $L = \{\alpha \in \{a,b\}^\star\mid n_a(\alpha) \geq n_b(\alpha)\}$;
  \item
    $L = \{\alpha \in \{a,b\}^\star\mid n_a(\alpha) > n_b(\alpha)\}$;
  \item
    $L = \{\omega_1 a \omega_2 b \mid \omega_1,\omega_2 \in \{a,b\}^\star\ \&\ \abs{\omega_1} = \abs{\omega_2}\}$;
  \item
    $L = \{\alpha c \beta \mid \alpha,\beta \in \{a,b\}^\star\ \&\ \alpha^R\mbox{ е поддума на }\beta \}$.
  \end{enumerate}
\end{problem}

\begin{problem}
  \marginpar{от Владислав}
  Да разгледаме граматиката $G = \CFG$, където  $V = \{S,A,B\}$, $\Sigma = \{a,b\}$, а правилата $R$ са
  \[S \to AA | B, A \to B | bb, B \to aa | aB.\]
  Да се намери езика на тази граматика и да се докаже, че граматиката разпознава точно този език.
\end{problem}



\section{Езици, които не са безконтекстни}

\begin{lemma}[за покачването (безконтекстни езици)]
  \index{лема за покачването!безконтекстни езици}
  \label{lem:pumping-context} 
  \marginpar{(стр. 123 от \cite{sipser1}; стр. 125 от \cite{hopcroft1})}
  За всеки безконтекстен език $L$ съществува $p>0$, такова
  че ако $\alpha\in L, \abs{\alpha} \geq p$, то съществува разбиване на думата на пет части, $\alpha=xyuvw$,
  за което е изпълнено:
  \begin{enumerate}[1)]
  \item
    $\abs{yv}\geq 1$,
  \item
    $\abs{yuv}\leq p$, и
  \item
    $(\forall i\geq 0)[xy^iuv^iw\in L]$.
\end{enumerate}
\end{lemma}
\begin{proof}
  Нека $G$ е граматиката за езика $L$.
  \marginpar{За простота, можем да си мислим, че $G$ е в НФЧ. Тогава $b=2$.}
  Нека \[b = \max\{\abs{\beta} \mid A\rightarrow_G \beta\}.\]
  Можем да приемем, че $b \geq 2$.
  \marginpar{Възлите във вътрешността на дървото са променливи, а листата са букви или $\varepsilon$}
  Това означава, че във всяко дърво на извод, всеки възел има
  не повече от $b$ наследника.
  Нека $p = b^{\abs{V}}+1$. Ще покажем, че $p$ е константа на покачването за граматиката $G$.
  Това означава, че всяка дума с дължина поне $p$ в езика $L$ има дърво на извод с височина
  поне $\abs{V} + 1$.
  
  Нека $\abs{\alpha} \geq p$ и $T$ е дърво на извода за думата $\alpha$.
  Понеже думата $\alpha$ може да има много дървета на извод, нека $T$ също така да бъде {\em с минимален брой възли}. 
  От направените по-горе разсъждения е ясно, че височината на $T$ е поне $\abs{V} + 1$,
  Следователно, по най-дългия път $\pi$ в $T$ имаме поне $\abs{V}+2$ възела, от които
  поне $\abs{V}+1$ са променливи, защото само листата могат да не са променливи.
  Да разгледаме последните $\abs{V}+1$ променливи по пътя $\pi$.
  От принципа на Дирихле следва, че измежду тези $\abs{V}+1$ променливи има поне една повтаряща се.
  Нека $R$ да бъде една такава променлива.
  Последните две повтаряния на $R$ разделят думата $\alpha$ на пет части.
  Нека $\alpha = xyuvw$.
  \begin{enumerate}[1)]
  \item
    $\abs{yv}\geq 1$,
    защото ако допуснем, че $\abs{yv} = 0$,
    то ще достигнем до противоречие с минималността на $T$.
  \item
    $\abs{yuv} \leq p$, защото сме избрали най-долното $R$.
  \item
    $xy^iuv^iw \in L$, защото можем да заменим поддървото 
    с корен последното $R$ за поддървото с корен предпоследното $R$.
    В случая $i = 0$, правим обратното.
  \end{enumerate}
\end{proof}

\begin{cor}
  \label{cor:pumping-context-free}
  \marginpar{Ако $L$ е краен език, то е ясно, че $L$ е безконтекстен.}
  Нека $L$ е произволен {\bf безкраен} език. Нека също така е изпълнено, че за всяко естествено число $p \geq 1$ можем да намерим дума $\alpha \in L$, $\abs{\alpha}\geq p$, такава че за всяко разбиване на думата на пет части, $\alpha = xyuvw$,
  със свойствата $\abs{yv} \geq 1$ и $\abs{yuv} \leq p$, е изпълнено, че $(\exists i)[xy^iuv^iw \not\in L]$.
  \marginpar{\writedown Докажете! Аналогично е на \Cor{pumping-reg}}
  Тогава $L$ {\bf не} е безконтекстен език.
\end{cor}

\begin{cor}
  \marginpar{\writedown Докажете!}
  Нека $G$ е безконтекстна граматика и $p$ е константата на покачването за $G$, $L = \L(G)$.
  Тогава $\abs{L} = \infty$ точно тогава, когато съществува $\alpha \in L$, за която $p \leq \abs{\alpha} < 2p$.
\end{cor}
% \begin{proof}
%   Ако съществува дума $\alpha \in L$, за която $\abs{\alpha} \geq p$, то от \Lem{pumping-context} следва,
%   че $\abs{L} = \infty$, защото $\alpha = xyuvw$ и $xy^iuv^iw \in L$, за всяко $i\in\Nat$.

%   За другата посока, нека сега $\abs{L} = \infty$.
%   Да изберем най-късата дума $\alpha \in L$, за която $\abs{\alpha} \geq p$.
%   Ще докажем, че $p \leq \abs{\alpha} < 2p$. За целта да допуснем, че $\abs{\alpha} \geq 2p$.
%   Тогава от \Lem{pumping-context} следва, че $\alpha = xyuvw$, $\abs{yv} \geq 1$, $\abs{yuv} \leq p$, $xy^0uv^0w = xuw \in L$.
%   Ако $\abs{xuw} < p$, то $\abs{yv} > p$, защото $\abs{yv} + \abs{xuw} = \abs{\alpha} \geq 2p$, и следователно $\abs{yuv} > p$, което е противоречие.
%   Следва, че $\abs{\alpha} > \abs{xuw} \geq p$.
%   Получихме, че думата $xuw\in L$ и $\abs{xuw} \geq p$. Това е противоречие с минималността на $\alpha$.
% \end{proof}

% \begin{framed}
%   \Lem{pumping-context} е полезна, когато искаме да докажем, че даден език $L$ {\bf не} е безконтекстен.
%   За целта, доказваме отрицанието на свойствата от \Lem{pumping-context} за $L$, т.е.
%   за всяка константа $p$, ние намираме дума $\alpha \in L$, $\abs{\alpha}\geq p$, такава че за всяко разбиване на думата на пет части, $\alpha = xyuvw$,
%   със свойствата $\abs{yv} \geq 1$ и $\abs{yuv} \leq p$, е изпълнено, че $(\exists i)[xy^iuv^iw \not\in L]$.
% \end{framed}


\begin{example}
  \label{example:anbncn}
  Езикът $L = \{a^nb^nc^n\ \mid\ n\in\Nat\}$ не е безконтекстен.
\end{example}
\begin{proof}
  \begin{itemize}
  \item 
    Разглеждаме произволна константа $p \geq 1$.
  \item
    Избираме дума $\alpha \in L$, $\abs{\alpha} \geq p$.
    В случая, нека $\alpha = a^pb^pc^p$.
  \item
    Разглеждаме произволно разбиване $xyuvw = \alpha$, за което $\abs{xyv} \leq p$ и $1 \leq \abs{yv}$.
  \item
    Трябва да изберем $i$, за което $xy^iuv^iw \not\in L$.
    Знаем, че поне едно от $y$ и $v$ не е празната дума.
    Имаме няколко случая за $y$ и $v$.
    \begin{itemize}
    \item
      $y$ и $v$ са думи съставени от една буква.
      В този случай получаваме, че $xy^2uv^2w$ има различен брой букви $a$, $b$ и $c$.
    \item
      $y$ или $v$ е съставена от две букви.
      Тогава е възможно да се окаже, че $xy^2uv^2w$ да има равен брой $a$, $b$ и $c$,
      но тогава редът на буквите е нарушен.
    \item
      понеже $\abs{yuv} \leq p$, то не е възможно в $y$ или $v$ да се срещат и трите букви.
    \end{itemize}  
    Оказа се, че във всички възможни случаи за $y$ и $v$, 
    $xy^2uv^2w \not\in L$.
  \end{itemize}
  Така от \Cor{pumping-context-free} следва, че езикът $L$ не е безконтекстен.
\end{proof}

% \begin{problem}
%   Да се даде пример за език $L$, който {\bf не} е безконтекстен, но удовлетворява
%   лемата за разрастването.
% \end{problem}

\begin{example}
  Приложете лемата за покачването за да докажете, че
  езикът $L$ не е безконтекстен, където:
  \begin{enumerate}[a)]
  \item
    $L = \{a^ib^jc^k\ \mid\ 0 \leq i \leq j \leq k\}$;
  \item
    $L = \{\beta\beta\mid \beta\in \{a,b\}^\star\}$;
  \item
    $L = \{a^{n^2}\mid n\in\Nat\}$.
  \end{enumerate}
\end{example}
\begin{proof}
  \begin{enumerate}[a)]
  \item
    Да фиксираме думата $\alpha = a^pb^pc^p$ и да разгледаме
    едно произволно нейно разбиване, $\alpha = xyuvw$, за което
    $\abs{yuv} \leq p$ и $1 \leq \abs{yv}$.
    Знаем, че поне една от $y$ и $v$ не е празната дума.
    \begin{itemize}
    \item
      $y$ и $v$ са съставени от една буква.
      Имаме три случая.
      \begin{enumerate}[i)]
      \item
        $a$ не се среща в $y$ и $v$.
        Тогава $xy^0vu^0w$ съдържа повече $a$ от $b$ или $c$.
      \item
        $b$ не се среща в $y$ и $v$.
        Ако $a$ се среща в $y$ или $v$, тогава $xy^2uv^2w$ съдържа повече $a$ от $b$
        Ако $c$ се среща в $y$ или $v$, тогава $xy^0uv^0w$ съдържа по-малко $c$ от $b$.
      \item
        $c$ не се среща в $y$ и $v$.
        Тогава $xy^2uv^2w$ съдържа повече $a$ или $b$ от $c$.
      \end{enumerate}      
    \item
      $y$ или $v$ е съставена от две букви.
      Тук разглеждаме $xy^2uv^2w$ и съобразяваме, че редът на буквите е нарушен.
    \end{itemize}
  \item
    \marginpar{Защо $\alpha = a^pba^pb$ не е добър кандидат?}
    Разгледайте $\alpha = a^pb^pa^pb^p$, т.е. $\beta = a^pb^p$ и $\alpha = \beta\beta$.
    Нека $xyuvw = \alpha$ е произволно разбиване на $\alpha$, за което е изпълнено, че
    $\abs{yuv} \leq p$ и $1\leq \abs{yv}$.
    \begin{itemize}
    \item
      Ако $yuv$ е в първата част на думата, то 
      $xy^0uv^0w = a^ib^ja^pb^p \not\in L$.
      Аналогично ако $yuv$ е във втората част на думата.
    \item
      Ако $yuv$ е в двете части на думата, то 
      $xy^0uv^0w = a^pb^ia^jb^p \not\in L$.
    \end{itemize}    
  \item
    Решава се аналогично както за регулярни езици.
  \end{enumerate}
\end{proof}


\begin{thm}
  Безконтекстните езици {\bf не} са затворени относно сечение и допълнение.
\end{thm}
\begin{proof}
  Да разгледаме езика
  \[L_0 = \{a^nb^nc^n\mid n\in\Nat\},\] за който вече знаем от Пример \ref{example:anbncn}, че не е безконтекстен.
  Да вземем също така и безконтекстните езици 
  \marginpar{\writedown Защо са безконтекстни?}
  \[L_1 = \{a^nb^nc^m\mid n,m\in\Nat\},\ L_2 = \{a^mb^nc^n\mid n,m\in\Nat\},\]
  \begin{itemize}
  \item 
    Понеже $L_0 = L_1\cap L_2$, то заключаваме, че безконтекстните езици не са затворени 
    относно операцията сечение.
  \item
    \marginpar{Озн. $\ov{L} = \Sigma^\star \setminus L$}
    Да допуснем, че безконтекстните езици са затворени относно операцията допълнение.
    Тогава  $\ov{L}_1$ и $\ov{L}_2$ са безконтекстни.
    Знаем, че безконтекстните езици са затворени относно обединение. 
    Следователно, езикът $L_3 = \ov{L}_1 \cup \ov{L}_2$ също е безконтекстен.
    Ние допуснахме, че безконтекстните са затворени относно допълнение, следователно $\ov{L}_3$
    също е безконтекстен.
    Но тогава получаваме, че езикът
    \[L_0 = L_1 \cap L_2 = \ov{\ov{L}_1 \cup \ov{L}_2} = \ov{L}_3\]
    е безконтекстен, което е противоречие.
  \end{itemize}
\end{proof}


\section{Алгоритми}

\subsection{Опростяване на безконтекстни граматики}

\subsubsection*{Премахване на безполезните променливи}

Нека е дадена безконтекстната граматика $G = \CFG$.
\marginpar{\cite{hopcroft1} стр. 88}
Една променлива $A$ се нарича {\bf полезна}, ако съществува извод от следния вид:
\[S \to^\star \alpha A \beta \to^\star \gamma,\]
където $\gamma \in \Sigma^\star$, а $\alpha,\beta \in (V \cup \Sigma)^\star$.
Това означава, че една променлива е полезна, ако участва в извода на някоя дума в езика на граматиката.
Една променлива се нарича {\bf безполезна}, ако не е полезна.
Целта ни е да получим еквивалентна граматика $G'$ без безполезни променливи.
Ще решим задачата като разгледаме две леми.

\begin{lemma}
  \label{lem:useless1}
  Нека е дадена безконтекстната граматика $G = \CFG$ и $\L(G) \neq \emptyset$.
  Съществува алгоритъм, който намира граматика $G' = \pair{V',\Sigma,S,R'}$, за която 
  $\L(G) = \L(G')$, и за всяка променлива $A' \in V'$, съществува дума $\alpha \in \Sigma^\star$,
  за която $A' \to^\star \alpha$.
\end{lemma}
\begin{proof}
  Да разгледаме следната проста итеративна процедура.
  \begin{algorithm}[H]
    \caption{Намираме $V' = \{A \in V\mid (\exists \alpha \in \Sigma^\star)[A \to^\star \alpha]\}$}
    \label{alg:useless}
    \begin{algorithmic}[1]
      \State $V' := \emptyset$
      \State $V'' := \{A \in V \mid (\exists \alpha \in \Sigma^\star)[A \to \alpha]\}$
      \While{$V' \neq V''$}
      \State $V' := V''$
      \State $V'' := V' \cup \{A \in V \mid (\exists \alpha \in (\Sigma \cup V')^\star)[A \to \alpha]\}$
      \EndWhile
      \State \Return $V'$
    \end{algorithmic}
  \end{algorithm}
  Трябва да докажем, че във $V'$ са точно полезните променливи за $G$.
  Очевидно е, че ако $A \in V'$, то $A$ е полезна променлива.
  \marginpar{\writedown Докажете!}
  За другата посока, с индукция по дължината на извода се доказва, че ако $A \to^\star_G \omega$,
  то $A \in V'$.
  
  Правилата на $G'$ са всички правила на $G$, в които участват променливи от $V'$ и букви от $\Sigma$.
\end{proof}

\begin{lemma}
  \label{lem:useless2}
  Съществува алгоритъм, който по дадена безконтекстна граматика $G = \CFG$, намира $G' = \pair{V',\Sigma',S,R'}$, $\L(G') = \L(G)$,
  със свойството, че за всяко $x \in V' \cup \Sigma'$ съществуват $\alpha, \beta \in (V'\cup\Sigma')^\star$,
  за които $S \to^\star \alpha x \beta$,
  т.е. всяка променлива или буква в $G'$ е достижима от началната променлива $S$.
\end{lemma}
\begin{proof}
  Намираме $V'$ и $\Sigma'$ итеративно, като в началото $V' = \{S\}$, $\Sigma' = \emptyset$.
  Ако $A \in V'$ и имаме правила $A \to \alpha_0 | \alpha_1 | \dots | \alpha_n$ в $G$,
  то за всяко $i = 0,\dots,n$ добавяме всички променливи на $\alpha_i$ към $V'$ и всички нетерминали на $\alpha_i$ към $\Sigma'$.
\end{proof}

\begin{thm}
  Всеки непразен безконтекстен език $L$ се поражда от безконтекстна граматика $G$
  без безполезни правила.
\end{thm}
\begin{proof}
  \marginpar{Защо е важна последователността на прилагане?}
  Нека е дадена безконтекстна граматика $G$ пораждаща $L$.
  Прилагаме върху $G$ първо процедурата от \Lem{useless1} и след това върху резултата прилагаме процедурата от \Lem{useless2}.
\end{proof}

\subsubsection*{Премахване на $\varepsilon$-правила}
\index{$\varepsilon$-правила}
За да премахнем правилата от вида $A \to \varepsilon$, следваме процедурата:
\marginpar{Броят на правилата може да се увеличи експоненциално, защото в най-лошия случай извеждаме всички подмножества на дадено множество от променливи}
\begin{enumerate}[1)]
\item 
  Намираме множеството $E = \{A \in V \mid A \to^\star \varepsilon\}$ по следния начин.
  Първо, $E := \{A \in V \mid A \to \varepsilon\}$.
  След това, за всяко правило от вида $B \to X_1\cdots X_k$, 
  ако всяко $X_i \in E$, то добавяме $B$ към $E$.
\item
  Строим множеството от правила $R'$, в което няма правила $\varepsilon$-правила по следния начин.
  За всяко правило $A \to X_1\cdots X_k$ в $R$,
  добавяме към $R'$ всички правила от вида $A \to \alpha_1\cdots\alpha_k$, където:
  \begin{itemize}[-]
  \item 
    ако $X_i \not\in E$, то $\alpha_i = X_i$;
  \item
    ако $X_i \in E$, то $\alpha_i = X_i$ или $\alpha_i = \varepsilon$;
  \item
    не всички $\alpha_i$-та са $\varepsilon$.
  \end{itemize}
\end{enumerate}

\begin{example}
  Нека е дадена граматиката $G$ с правила
  \[S\rightarrow D,D\rightarrow AD|b,A\rightarrow AB|BC|a, B\rightarrow AA|EC,C\rightarrow \varepsilon|CA|a, E\rightarrow \varepsilon|aEb.\]
  Тогава $E = \{X \in V \mid X \rightarrow^\star_G \varepsilon\} = \{A,B,C,E\}$.
  Това означава, че $\varepsilon \not\in \L(G)$.
  Граматиката $G'$ без $\varepsilon$-правила, за която $\L(G') = \L(G)$ има следните правила
  $S \to D, D\to AD|D|b, A \to A|B|C|AB|BC|a,B\to A|E|C|AA|EC, C \to C|A|CA|a, E \to aEb|ab$.
\end{example}

\subsubsection*{Премахване на преименуващи правила}
\index{преименуващи правила}
Преименуващите правила са от вида $A \to B$.
Нека е дадена граматика $G = \CFG$, в която има преименуващи правила.
Ще построим еквивалентна граматика $G'$ без преименуващи правила.
В началото нека в $R'$ да добавим всички правила от $R$, които не са преименуващи.
След това, за всякa променлива $A$, за която $A \to^\star_G B$,
ако $B \to \alpha$ е правило в $R$, което не е преименуващо,
то добавяме към $R'$ правилото $A \to \alpha$.

\begin{example}
  Нека е дадена граматиката $G$ с правила  
  \[A\rightarrow B|S,B\rightarrow C|BC,C\rightarrow AB|a|b,S\rightarrow B|CC|b.\]
  Първо добавяме към $R'$ правилата $B \to BC, C \to AB|a|b, S \to CC|b$.
  \begin{itemize}
  \item 
    Лесно се съобразява, че $A \to^\star_G B,S,C$.
    Добавяме правилата $A \to BC|AB|a|b|CC$.
  \item
    Имаме $B \to^\star_G C$.
    Добавяме правилата $B \to AB|a|b$.
  \item
    Имаме $S \to^\star_G B,C$.
    Добавяме правилата $S \to BC|AB|a|b$.
  \end{itemize}
  Накрая получаваме, че граматиката $G'$ има правила
  $A \to BC|AB|a|b|CC, B \to AB|a|b|BC, C \to AB|a|b, S \to BC|AB|CC|a|b$.
\end{example}

\subsection{Нормална Форма на Чомски}

\begin{dfn}
%[стр. 99 от \cite{sipser}]
\index{Нормална форма на Чомски}
Една безконтекстна граматика е в {\em нормална форма на Чомски}, ако
всяко правило е от вида
\[A \rightarrow BC\mbox{ и }A \rightarrow a,\]
като $B, C$ {\em не могат} да бъдат променливата за начало $S$.
Освен това, позволяваме правилото $S\to\varepsilon$.
\footnote{На стр. 151 в \cite{papadimitriou} дефиницията е малко по-различна.
Там дефинират $G$ да бъде в нормална форма на Чомски ако $R \subseteq V\times(V\cup\Sigma)^2$.
В този случай губим езиците $\{\varepsilon\}$ и $\{a\}$, за $a\in\Sigma$.}
\end{dfn}

\begin{thm}
  Всеки безконтекстен език $L$ е генериран от контекстно-свободна
  граматика в нормална форма на Чомски.
\end{thm}
\begin{proof}
%  \marginpar{Броят на правилата може да се увеличи експоненциално.}
  Нека имаме контекстно-свободна граматика $G$, за която $L = L(G)$.
  Ще построим контекстно-свободна граматика $G^\prime$ в нормална форма на Чомски, $L = L(G^\prime)$.
  % [стр. 99 от \cite{sipser}]
  Следваме следната процедура:
  \begin{itemize}
  \item
    Добавяме нов начален символ $S_0$ и правило $S_0 \to S$.
  \item
    \marginpar{Време $O(n)$}
    Съкращаваме дължината на правилата.
    Заменяме правилата от вида $A\to u_1\dots u_n$, $n\geq 3$, $u_i \in V\cup\Sigma$, с
    правилата \[A\to u_1A_1,\ A_1\to u_2A_2,\ \dots,\ A_{n-2} \to u_{n-1}u_n.\]
    където $A_i$ са нови променливи.
  \item
    \marginpar{Време $O(n^2)$}
    За всяка променлива $A \neq S_0$ премахваме правилата от вида $A\to\varepsilon$.
    Това правим по следния начин.
    
    Ако имаме правило от вида $R \to Au$ или $R\to u A$, $u \in V \cup \Sigma$,
    то добавяме правилото $R\to u$.
    %Правим това за всяко срещане на променливата $A$ в дясната страна на правило.
    Например, 
    \begin{itemize}
    \item 
      ако имаме правило $R\to aA$, то добавяме правилото $R \to a$;
    \item
      ако имаме правило $R\to AA$, то добавяме правилото $R \to A$.
    \end{itemize}
    Ако имаме правило от вида $R\to A$, то добавяме правилото $R\to\varepsilon$
    само ако променливата $R$ още не е преминала през процедурата за премахване на $\varepsilon$.
  \item
    \marginpar{Време $O(n^2)$}
    \marginpar{Памет $O(n^2)$}
    Премахваме преименуващите правила, т.е. правила от вида $A\to B$.
    Заменяме всяко правило от вида $B \to \beta$ с $A\to \beta$,
    освен ако $A \to \beta$ е вече премахнато преименуващо правило.
  \item
    \marginpar{Време $O(n)$}
    За правила от вида $A\to u_1 u_2$, където $u_1, u_2 \in V \cup \Sigma$, 
    заменяме всяка буква $u_i$ с новата променлива $U_i$
    и добавяме правилото $U_i\to u_i$.
    Например, правилото $A \to aB$ се заменя с правилото $A \to XB$ и добавяме правилото $X \to a$,
    където $X$ е нова променлива.
  \end{itemize}
\end{proof}

\begin{thm}
  При дадена безконтекстна граматика $G$ с дължина $n$, можем да намерим еквивалентна
  на нея граматика $G'$ в нормална форма на Чомски за време $O(n^2)$,
  като получената граматика е с дължина $O(n^2)$.
\end{thm}


% \begin{problem}
%   Нека е дадена граматиката  $G = \pair{\{S,A,B,C,D,E\}, \{a,b\},S, R}$.
%   \begin{enumerate}[a)]
%   \item
%     Намерете множеството $\{X \in V \mid X \rightarrow^\star_G \varepsilon\}$.
%   \item
%     Вярно ли е, че $\varepsilon \in L(G)$?
%   \item
%     Постройте граматика $G_1$ без $\varepsilon$-правила, за която $L(G_1)=L(G)\setminus\{\varepsilon\}$.
%   \end{enumerate}
%   Множеството от правила $R$ на граматиката $G$ е зададено като:
%   \begin{enumerate}[a)]
%   \item
%     $R = \{S\rightarrow D,D\rightarrow AD|b,A\rightarrow ACB|BC|a, B\rightarrow ABCA|CEC,C\rightarrow \varepsilon|CA|a, E\rightarrow \varepsilon|aEb\}$;
%   \item
%     $R = \{S \rightarrow aD, D\rightarrow \varepsilon|ABBA|ADD,A\rightarrow DEB|a,B\rightarrow DDD|DC|b,C\rightarrow CCE|a, E\rightarrow \varepsilon|bEa\}$;
%   \item
%     $R = \{ S\rightarrow D,D\rightarrow AD|b,A\rightarrow AB|BC|a, B\rightarrow AB|CC, C\rightarrow \varepsilon|CA|a, E\rightarrow a|EB\}$;
%   \item
%     $R = \{ S \rightarrow AD|a, D\rightarrow \varepsilon|BB|AD,A\rightarrow DB|a,B\rightarrow DD|DC|b,C\rightarrow CE|a, E\rightarrow AB|b|EA\}$;
%   \item
%     $R =\{S\rightarrow AS|SB|SS,B\rightarrow CA|b, C\rightarrow AA|a|BA,A\rightarrow \varepsilon|BS\}$;
%   % \item
%   %   $R = \{S\rightarrow AB|AC,B\rightarrow \varepsilon |BC|b,A\rightarrow BB|CC|a,C\rightarrow CS|a\}$;
%   % \item
%   %   $R = \{S\rightarrow AS|SB|SS,B\rightarrow AC|b, C\rightarrow A|a|AB,A\rightarrow \varepsilon|BS\}$;
%   \item
%     $R = \{S\rightarrow BA|CA,B\rightarrow \varepsilon |BC|b,A\rightarrow BB|CC|a, C\rightarrow CS|a\}$;
%   \item
%     $R = \{S\rightarrow AS|b,A\rightarrow AC|BC|a, B\rightarrow BC|CC,C\rightarrow \varepsilon|CA|a\}$;
%   \item
%     $R = \{S\rightarrow \varepsilon|BA|AS,A\rightarrow SB|a,B\rightarrow SS|SC|b,
%     C\rightarrow CC|a\}$; 
%   \end{enumerate}
% \end{problem}

\begin{problem}
  Нека е дадена граматиката  $G = \pair{\{S,A,B,C\}, \{a,b\}, S, R}$.
  Използвайте обща конструкция, за да премахнете ,,дългите'' правила 
  (т.е. правила с дължина поне 2, които не са в н.ф. на Чомски) от $ G$ като при това получите 
  безконтестна граматика $G_1$ с език $L(G)=L(G_1)$, където:
  \begin{enumerate}[a)]
  \item
    $R = \{S \rightarrow \varepsilon|ab|aAba, A\rightarrow aBCb, B\rightarrow bbb, C\rightarrow aC\vert aCaC\}\rangle$;
  \item
    $R = \{S \rightarrow \varepsilon|ab|baAb, A\rightarrow BaBb,B\rightarrow b,C\rightarrow AbA\vert aCCa\}$;
  \item
    $R = \{A\rightarrow BSB|a,B\rightarrow ba|BC,C\rightarrow BaSA|a|b,S\rightarrow CC|b\}$;
  \item
    $R = \{A\rightarrow BAS,B\rightarrow CB,C\rightarrow ab|ABbS,S\rightarrow CC|b\}$;
  \end{enumerate}
\end{problem}


% \begin{problem}
%   Намерете безконтекстна граматика в нормална форма на Чомски за езиците от задача 6.
% \end{problem}


\subsection{Проблемът за принадлежност}

\begin{thm}
  Съществува {\em полиномиален} алгоритъм , който проверява дали дадена дума принадлежни на граматиката $G$.
  \marginpar{За дума $\alpha$, алгоритъмът работи за време $O(\abs{\alpha}^3)$}
\end{thm}
% \begin{proof}[стр. 154 от \cite{papadimitriou}]
Можем да приемем, че $G = \CFG$ е граматика в нормална форма на Чомски.
Нека $\alpha = a_1a_2\dots a_n$ е дума, за която искаме да проверим дали $\alpha \in L(G)$.
\marginpar{Това е алгоритъм на Cocke, Younger и Kasami (CYK), който е пример за динамично програмиране (стр. 195 от \cite{kozen})}
\begin{algorithm}[H]
  \caption{Проверка за $\alpha \in L(G)$}
  \label{alg:belongs-to-grammar}
  \begin{algorithmic}[1]
    \State $n := \abs{\alpha}$ \Comment{Вход дума $\alpha = a_1\cdots a_n$}
    \ForAll{$i\in [1,n]$}
    \State $V[i,i] = \{A \in V \mid A\rightarrow a_i\}$
    \EndFor
    \ForAll{$i,j \in [1,n]\ \&\ i \neq j$}
    \State $V[i,j] = \emptyset$
    \EndFor      
    \ForAll{$s \in [1, n)$} \Comment{Дължина на интервала}
    \ForAll{$i \in [1, n-s]$}\Comment{Начало на интервала}
    \ForAll{$k \in [i, i + s)$}\Comment{Разделяне на интервала}
    \If{$\exists A\to BC \in R\ \&\ B \in V[i,k]\ \&\ C\in V[k+1,i+s]$}
    \State $V[i,i+s] := V[i,i+s] \cup \{A\}$
    \EndIf
    \EndFor
    \EndFor
    \EndFor
    \If{$S \in V[1,n]$}
    \State \Return \texttt{True}\Comment{Има извод на думата от $S$}
    \Else
    \State \Return \texttt{False}
    \EndIf
  \end{algorithmic}
\end{algorithm}

\begin{lemma}
  За дадена граматика в нормална форма на Чомски и дума $\alpha$, 
  за всяко $0 \leq s < \abs{\alpha}$, след $s$-тата итерация на алгоритъма (редове 6 - 10), за всяка позиция $i = 1,\dots,n-s$,
  \[V[i,i+s] = \{A \in V \mid A \rightarrow^\star_G a_i\dots a_{i+s}\}.\]
\end{lemma}
\begin{proof}
  Пълна индукция по $s$.
  За $s = 0$  е ясно. (Защо?)

  Нека твърдението е вярно за $s < n$. Ще докажем твърдението за $s+1$, т.е. за всяко $i = 1,\dots,n-s-1$,
  \[V[i,i+s+1] = \{A \in V \mid A \rightarrow^\star_G a_i\dots a_{i+s+1}\}.\]
  % Да разгледаме $A \in V[i,i+s+1]$.
  За едната посока, да разгледаме първoто правило в извода $A \to^\star_G a_i\cdots a_{i+s+1}$.
  Понеже $G$ е в НФЧ, то е от вида $A \to BC$ и тогава съществува някое $t$, за което 
  $B \to^\star a_i\cdots a_{i+t}$ и $C \to^\star a_{i+t+1}\cdots a_{i+s+1}$.
  От И.П. получаваме, че $B \in V[i,i+t]$ и $C \in V[i+t+1,i+s+1]$.
  Тогава от ред 10 на алгоритъма е ясно, че $A \in V[i,i+s+1]$.
  
  За другата посока, нека $A \in V[i,i+s+1]$.
  Единствената стъпка на алгоритъма, при която може да сме добавили $A$ към множеството $V[i,i+s+1]$ е ред 10.
  Тогава имаме, че съществува $k$, за което $B \in V[i,k]$, $C \in V[k+1,i+s+1]$, и $A\to BC$ е правило в граматиката $G$.
  От И.П. имаме, че $B \to^\star_G a_i\cdots a_k$ и $C \to^\star_G a_{k+1}\cdots a_{i+s+1}$.
  Заключаваме веднага, че $A \to^\star_G a_i\cdots a_{i+s+1}$.
\end{proof}

\begin{example}
  Нека е дадена граматиката $G$ с правила 
  $S\rightarrow a|AB|AC, C\rightarrow SB|AS,A\rightarrow a, B\rightarrow b$.
  Ще приложим $CYK$ алгоритъма за да проверим дали думата $aaabb \in \L(G)$.
  \begin{itemize}
  \item 
    $V[1,1] = V[2,2] = V[3,3] = \{S,A\}$.
    $V[4,4] = V[5,5] = \{B\}$.
  \item
    $V[1,2] = V[2,3] = \{C\}$.
    $V[3,4] = \{S,C\}$.
    $V[4,5] = \emptyset$.
  \item
    $V[1,3] = \{S\} \cup \emptyset$.
    $V[2,4] = \{S,C\} \cup \emptyset$.
    $V[3,5] = \emptyset \cup \{C\}$.
  \item
    $V[1,4] = \{S,C\} \cup \emptyset \cup \emptyset = \{S,C\}$.
    $V[2,5] = \{S\} \cup \emptyset \cup \{C\} = \{S,C\}$
  \item
    $V[1,5] = \{S,C\} \cup \emptyset \cup \emptyset \cup \{C\}= \{S,C\}$.
  \end{itemize}
  Понеже $S \in V[1,5]$, то $aaabb \in \L(G)$.
\end{example}

\begin{thm}
  \marginpar{\cite{hopcroft1}, стр. 137}
  Съществуват алгоритми, които определят по дадена безконтекстна граматика $G$ дали:
  \begin{enumerate}[a)]
  \item 
    $\abs{\L(G)} = 0$;
  \item
    $\abs{\L(G)} < \infty$;
  \item
    $\abs{\L(G)} = \infty$.
  \end{enumerate}
\end{thm}
\begin{proof}
  Нека е дадена една безконтекстна граматика $G$.
  \begin{description}
  \item[($\L(G) = \emptyset?$)]
    Прилагаме алгоритъма за премахване на безполезните променливи.
    Ако открием, че $S$ е безполезна променлива, то $\L(G) = \emptyset$.
  \item[($\abs{\L(G)} < \infty?$ или $\abs{\L(G)} = \infty?$)]
    Нека да разгледаме граматиката $G'$ в НФЧ без безполезни променливи, за която $\L(G) = \L(G')$.
    От граматиката $G' = \pair{V',\Sigma,S,R'}$ строим граф с възли променливите от $V'$ като
    за $A,B \in V'$ имаме ребро $A \to B$ точно тогава, когато съществува $C \in V'$,
    за което $A \to BC$ или $A \to CB$ е правило в $R'$.
    
    Ако в получения граф имаме цикъл, то $\L(G') = \infty$.
  \end{description}
\end{proof}

\section{Недетерминирани стекови автомати}

\index{автомат!недетерминиран стеков}
\marginpar{На англ. {\bf Push-down automaton}}% (стр. 157 от \cite{kozen})}
%Sipser p.102
\begin{dfn}
  Недетерминиран стеков автомат е 7-орка от вида
  \[P = \PDA,\] където:
  \begin{itemize}
  \item
    $Q$ е крайно множество от състояния;
  \item  
    $\Sigma$ е крайна входна азбука;
  \item
    $\Gamma$ е крайна стекова азбука;
  \item
    $\# \in \Gamma$ е символ за дъно на стека;
  \item
    $s\in Q$ е начално състояние;
  \item
    \marginpar{Озн. $\Ps_{fin}(A)$ - крайните подмножества на $A$}
    $\Delta:Q\times(\Sigma \cup \{\varepsilon\})\times\Gamma\rightarrow \Ps_{fin}(Q\times\Gamma^\star)$ 
    е функция на преходите;    
  \item
    $F\subseteq Q$ е множество от заключителни състояния.
  \end{itemize}
\end{dfn}

\marginpar{Instanteneous description}
{\em Моментно описание} (или конфигурация) на изчислението със стеков автомат представлява тройка от вида $(q,\alpha,\gamma) \in Q\times\Sigma^\star\times\Gamma^\star$,
т.е. автоматът се намира в състояние $q$, думата, която остава да се прочете е $\alpha$,
а съдържанието на стека е думата $\gamma$.
Удобно е да въведем бинарната релация $\vdash_P$ над $Q\times\Sigma^\star\times\Gamma^\star$,
която ще ни казва как моментното описание на автомата $P$ се променя след изпълнение на една стъпка:
\[(q,x\alpha,Y\gamma) \vdash_P (p,\alpha,\beta\gamma), \text{ ако } \Delta(q,x,Y) \ni (p,\beta),\]
\[(q,\alpha,Y\gamma) \vdash_P (p,\alpha,\beta\gamma), \text{ ако } \Delta(q,\varepsilon,Y) \ni (p,\beta).\]
Рефлексивното и транзитивно затваряне на $\vdash_P$ ще означаваме с $\vdash^\star_P$.
Сега вече можем да дадем дефиниция на език, разпознаван от стеков автомат $P$.
\begin{itemize}
\item
  $\L_F(P)$ е езика, който се разпознава от $P$ {\bfс финално състояние},
  \[\L_F(P) = \{\omega \mid (q_0,\omega,\#) \vdash^\star_P (q,\varepsilon,\alpha)\ \&\ q \in F\}.\]    
\item
  $\L_S(P)$ е езика, който се разпознава от $P$  {\bf с празен стек},
  \[\L_S(P) = \{w\mid (q_0,w,\#) \vdash^\star_P (q,\varepsilon,\varepsilon)\}.\]    
\end{itemize}

\begin{example}
  \label{ex:anbn}
  За езика $L = \{a^nb^n\mid n\in\Nat\}$ съществува стеков автомат $P$, такъв че
  $L = \L_S(P)$.
  Да разгледаме $P = \PDA$, където
  \begin{itemize}
  \item
    $Q = \{q\}$;
  \item
    $\Sigma = \{a,b\}$;
  \item
    $\Gamma = \{\#,A\}$, където символът $\#$ служи за дъно на стека, а броят на $A$-тата в стека ще показват колко букви $a$ сме прочели от думата;
  \item
    $F = \emptyset$, защото разпознаваме с празен стек, а не с финално състояние;
  \item 
    $\Delta(q,a,\#) = \{(q, A\#)\}$;
  \item 
    $\Delta(q,\varepsilon,\#) = \{(q,\varepsilon)\}$;
  \item 
    $\Delta(q,b,A) = \{(q,\varepsilon)\}$.
  \end{itemize}
  Вместо доказтелство, да видим как думата $a^2b^2$ се разпознава от автомата с празен стек:
  \marginpar{\writedown Докажете, че $L = \L_S(P)$!}
  \begin{align*}
    (q,a^2b^2,\#) & \vdash_P (q,ab^2,A\#) \\
    & \vdash_P (q,b^2, AA\#)\\
    & \vdash_P (q,b,A\#)\\
    & \vdash_P (q,\varepsilon,\#)\\
    & \vdash_P (q,\varepsilon,\varepsilon).
  \end{align*}
\end{example}

\begin{example}
  За езика $L = \{\omega\omega^R \mid \omega \in \{a,b\}^\star\}$ съществува стеков автомат $P$, такъв че
  $L = \L_S(P)$.
  Нека $P = \PDA$, където:
  \begin{itemize}
  \item 
    $\Delta(q, a, \#) = \{(q, A\#)\}$;
  \item 
    $\Delta(q, b, \#) = \{(q, B\#)\}$;
  \item
    $\Delta(q, a, A) = \{(q, AA), (p, \varepsilon)\}$;
  \item
    $\Delta(q, a, B) = \{(q, AB)\}$;
  \item
    $\Delta(q, b, B) = \{(q, BB), (p, \varepsilon)\}$;
  \item
    $\Delta(q, b, A) = \{(q, BA)\}$;
  \item
    $\Delta(p, a, A) = \{(p,\varepsilon)\}$;
  \item
    $\Delta(p, b, B) = \{(p,\varepsilon)\}$;
  \item
    $\Delta(q, \varepsilon, \#) = \{(q,\varepsilon)\}$;
  \item
    $\Delta(p, \varepsilon, \#) = \{(p,\varepsilon)\}$;
  \end{itemize}
  Основното наблюдение, което трябва да направим за да разберем конструкцията на автомата е, че
  всяка дума от вида $\omega\omega^R$ може да се запише като $\omega_1aa\omega^R_1$ или $\omega_1bb\omega^R_1$.
  Да видим защо $P$ разпознава думата $abaaba$ с празен стек.
  Започваме по следния начин:
  \begin{align*}
    (q,abaaba,\#) & \vdash_P (q,baaba,A\#)\\
    & \vdash_P (q, aaba, BA\#) \\
    & \vdash_P (q, aba, ABA\#).
  \end{align*}
  Сега можем да направим два избора как да продължим. Състоянието $p$ служи за маркер, което ни казва, че вече сме започнали 
  да четем $\omega^R$. Поради тази причина, продължаваме така:
  \begin{align*}
    (q, aba, ABA\#) & \vdash_P (p, ba, BA\#)\\
    & \vdash_P (p, a, A\#)\\
    & \vdash_P (p, \varepsilon, \#) \\
    & \vdash_P (p,\varepsilon,\varepsilon).
  \end{align*}
  Да проиграем още един пример. Да видим защо думата $aba$ не се извежда от автомата.
  \begin{align*}
    (q,aba,\#) & \vdash_P (q, ba,A\#)\\
    & \vdash_P (q, a, BA\#)\\
    & \vdash_P (q, \varepsilon, ABA\#).
  \end{align*}
  \marginpar{\writedown Докажете, че $\L_S(P) = L$ !}
  От последното моментно описание на автомата нямаме нито един преход, следователно
  думата $aba$ не се разпознава от $P$ с празен стек.
\end{example}


\begin{thm}
  \marginpar{(\cite{hopcroft1}, стр. 114) }
  Нека $L$ е произволен език над азбука $\Sigma$.
  \begin{enumerate}[1)]
  \item 
    Ако съществува НСА $P$, за който $L = \L_F(P)$, то съществува НСА $P^\prime$, за който $L = \L_S(P^\prime)$.
  \item
    Ако съществува НСА $P$, за който $L = \L_S(P)$, то съществува НСА $P^\prime$, за който $L = \L_F(P^\prime)$.
  \end{enumerate}
  С други думи, езиците разпознавани от НСА с празен стек са точно езиците разпознавани от НСА с финално състояние.
\end{thm}
\begin{proof}
  \begin{enumerate}[1)]
  \item 
    Нека $L = \L_F(P)$, където $P = \PDA$.
    Ще построим $P^\prime$, така че да симулира $P$ и като отидем във финално състояние ще изпразним стека.
    Нека
    \[P^\prime = \langle{Q\cup\{q_e,s^\prime\},\Sigma,\Gamma \cup \{\$\},\$,s^\prime,\Delta^\prime,\emptyset}\rangle,\]
    където $\$ \not\in \Gamma$.
    Важно е $P^\prime$ да има собствен нов символ за дъно на стека, защото е възможно за някоя дума $\alpha \not\in \L_F(P)$
    стековият автомат $P$ да си изчисти стека и така да разпознаем повече думи.
    \begin{itemize}
    \item 
      \marginpar{- започваме симулацията}
      $\Delta'(s^\prime,\varepsilon,\$) = \{(s,\#\$)\}$;
    \item
      \marginpar{- симулираме $P$}
      $\Delta'(q,a,X)$ включва множеството $\Delta(q,a,X)$, за всяко $q\in Q$, $a\in\Sigma_\varepsilon$, $X\in\Gamma$;
    \item
      \marginpar{- ако сме във финално, започваме да чистим стека}
      $\Delta'(q,\varepsilon,X)$ съдържа също и елемента $(q_e,\varepsilon)$, за всяко $q\in F$, $X \in \Gamma \cup \{\$\}$;
    \item
      \marginpar{- изчистваме стека}
      $\Delta'(q_e,\varepsilon,X) = \{(q_e,\varepsilon)\}$, за всяко $X \in \Gamma \cup \{\$\}$;
    \item
      $\Delta'$ няма други правила.
    \end{itemize}
  \item
    Сега имаме $L = \L_S(P)$, където $P = \langle{Q,\Sigma,\Gamma,\#,s,\Delta,\emptyset}\rangle$. 
    Да положим
    \[P^\prime = \langle{Q\cup\{s^\prime,q_f\}, \Sigma, \Gamma \cup \{\$\}, \Delta^\prime, \$, \{q_f\}}\rangle.\]
    $P^\prime$ ще симулира $P$ като ще внимаваме кога $P$ изчиства символа $\#$. Тогава ще искаме да отидем във финалното състояние $q_f$.
    \begin{itemize}
    \item 
      \marginpar{- започваме симулацията}
      $\Delta'(s',\varepsilon,\$) = \{(s, \#\$)\}$;
    \item
      \marginpar{- симулираме $P$}
      $\Delta'(q,a,X) = \Delta(q,a,X)$, за всяко $q \in Q$, $a \in \Sigma_\varepsilon$, $X \in \Gamma$;
    \item
      \marginpar{- щом сме стигнали до $\$$, значи $P$ е изчистил стека си}
      $\Delta'(q,\varepsilon,\$) = \{(q_f,\varepsilon)\}$.
    \end{itemize}
  \end{enumerate}
\end{proof}

\begin{problem}
  Като използвате стековия автомат от Пример \ref{ex:anbn}, дефинирайте автомат $P'$, за който
  $\L_F(P') = \{a^nb^n \mid n\in\Nat\}$.
\end{problem}

\begin{framed}
\begin{thm}
  \label{th:push-down-context-free}
  Класът на езиците, които се разпознават от краен стеков автомат съвпада с
  класа на безконтекстните езици.
\end{thm}
\end{framed}
\begin{proof}
  \marginpar{(\cite{hopcroft1}, стр. 117)}
  Ще разгледаме двете посоки на твърдението поотделно.
  \begin{enumerate}[1)]
  \item 
    Нека е дадена безконтекстна граматика $G = \CFG$.
    Нашата цел е да построим стеков автомат $P$, така че $\L_S(P) = \L(G)$.
    Нека  \[P = \langle{\{q\},\Sigma,\Sigma\cup V,S,q,\Delta,\emptyset}\rangle,\]
    където функцията на преходите е:
    \begin{align*}
      & \Delta(q,\varepsilon,A) = \{(q,\alpha)\mid A\to\alpha\mbox{ е правило в граматиката }G\}\\
      & \Delta(q,a,a) = \{(q,\varepsilon)\}
    \end{align*}
  \item
    Нека имаме $P = \langle{Q, \Sigma, \Gamma, \Delta, s, \#, \emptyset}\rangle$.
    Ще дефинираме безконтекстна граматика $G$, за която $\L_S(P) = \L(G)$.
    Променливите на граматика са 
    \[V = \{[q,A,p] \mid q,p \in Q, A \in \Gamma\}.\]
    Правилата на $G$ са следните:
    \begin{itemize}
    \item
      $S \to [s,\#,q]$, за всяко $q \in Q$;
    \item
      $[q,A,q_{m+1}] \to a[q_1,B_1,q_2][q_2,B_2,q_3]\dots [q_m,B_m,q_{m+1}]$,
      където 
      \[(q_1,B_1\dots B_m) \in \Delta(q, a, A)\]
      и произволни $q,q_1,\dots,q_{m+1} \in Q$,
      $a \in \Sigma \cup \{\varepsilon\}$.

      Да обърнем внимание, че е възможно $m = 0$.
      Това означава, че $(q_1,\varepsilon) \in \Delta(q, a, A)$ и тогава имаме правилото $[q,A,q_{1}] \to a$, където $a \in \Sigma \cup \{\varepsilon\}$.
    \end{itemize}
    Трябва да докажем, че:
    \[[q,A,p] \rightarrow^\star_G \alpha\ \Leftrightarrow\ (q,\alpha,A) \vdash^\star_P (p,\varepsilon,\varepsilon).\]
    \begin{description}
    \item[$(\Rightarrow)$]
      С пълна индукция по $i$, ще докажем, че 
      \[(q,\alpha,A) \vdash^i_P (p,\varepsilon,\varepsilon)\ \implies\ [q,A,p] \Rightarrow^\star_G \alpha.\]
      Ако $i = 1$, то е лесно, защото $\alpha \in \Sigma \cup\{\varepsilon\}$ и $m = 0$.

      Ако $i > 1$, нека $\alpha = a\beta$. Тогава:
      \marginpar{Възможно е $a = \varepsilon$}
      \[(q,a\beta,A) \vdash_P (q_1,\beta,B_1\dots B_n) \vdash^{i-1}_P (p, \varepsilon, \varepsilon).\]
      Да разбием думата $\beta$ на $n$ части, $\beta = \beta_1\cdots \beta_n$, със свойството, че след като прочетем $\beta_i$ 
      сме премахнали променливата $B_i$ от върха на стека. Това означава, че :
      \begin{align*}
        & (q_j, \beta_j, B_j) \vdash^{l_j}_P (q_{j+1},\varepsilon,\varepsilon), \text{ за }j = 1,\dots,n-1,\\
        & (q_n, \beta_n, B_n) \vdash^{l_n}_P (p,\varepsilon,\varepsilon),
      \end{align*}
      където $l_1+l_2+\cdots+l_n = i-1$.
      Сега по {\bf И.П.} получаваме:
      \begin{align*}
        & (q_j, \beta_j, B_j) \vdash^{l_j}_P (q_{j+1},\varepsilon,\varepsilon) \implies [q_j,B_j, q_{j+1}] \rightarrow^\star_G \beta_j, \text{ за }за j = 1,\dots,n-1,\\
        & (q_n, \beta_n, B_n) \vdash^{l_n}_P (p,\varepsilon,\varepsilon) \implies [q_n,B_n, p] \rightarrow^\star_G \beta_n.
      \end{align*}
      Обединявайки тези изводи с правилото
      \[[q,A,p] \rightarrow_G a[q_1,B_1,q_2]\dots[q_n,B_n,p],\]
      получаваме извода
      \[[q,A,p] \rightarrow^\star_G a\beta.\]
    \item[$(\Leftarrow)$]
      Отново с пълна индукция по $i$ ще докажем, че
      \[[q,A,p] \rightarrow^i_G \alpha \implies (q,\alpha,A) \vdash^\star_P (p,\varepsilon,\varepsilon).\]
      Ако $i = 1$, то имаме $[q,A,p] \rightarrow \alpha$, където $\alpha = a$ или $\alpha = \varepsilon$.
      Ако $i > 1$, то имаме, че $\alpha = a\beta$ и за някое $n$, 
      \[[q,A,p] \rightarrow_G a[q_1,B_1,q_2][q_2,B_2,q_3]\dots[q_n,B_n,p] \rightarrow^{i-1}_G \beta.\]
      Отново нека $\beta = \beta_1\dots \beta_n$, където 
      \begin{align*}
        & [q_j,B_j,q_{j+1}] \rightarrow^{i_j}_G \beta_j, \text{ за } j = 1,\dots,n-1,\\
        & [q_{n},B_n,p ] \rightarrow^{i_n}_G \beta_n,
      \end{align*}
      където $i_1 + i_2 + \cdots + i_n = i-1$.
      От {\bf И.П.} получаваме, че 
      \begin{align*}
        & [q_j,B_j,q_{j+1}] \rightarrow^{i_j}_G \beta_j \implies (q_j,\beta_j,B_j) \vdash^\star_P (q_{j+1},\varepsilon,\varepsilon),\ j = 1,\dots,n-1\\
        & [q_n,B_n,p] \rightarrow^{i_n}_G \beta_n \implies (q_n,\beta_n,B_n) \vdash^\star_P (p,\varepsilon,\varepsilon),
      \end{align*}
      Обединявайки всичко, което знаем, получаваме:
      \begin{align*}
        (q, a\beta, A) & \vdash_P (q_1, \beta_1\cdots\beta_n, B_1\cdots B_n)\\
        & \vdash^\star_P (q_2, \beta_{2}\cdots\beta_n, B_2\cdots B_n)\\
        & \dots\\
        & \vdash^\star_P (q_n, \beta_n, B_n)\\
        & \vdash^\star_P (p, \varepsilon, \varepsilon)
      \end{align*}
    \end{description}
  \end{enumerate}
\end{proof}

% \begin{problem}
%   Нека е дадена граматиката $G = \pair{\{S,A,B\},\{a,b\},S,R\}}$.
%   Постройте стеков автомат $P = \PDA$, такъв че $\L_S(P) = \L(G)$, където правилата $R$ на граматиката $G$ са зададени като:
%   \begin{enumerate}[a)]
%     % За едно тези двете да се даде пример как става 
%   \item
%     $R = \{S\rightarrow ASB\vert \varepsilon, A\rightarrow aAa\vert a, B\rightarrow bBb\vert b\}$;
%   \item
%     $R = \{S\rightarrow ASB\vert \varepsilon, A\rightarrow aA\vert a, B\rightarrow Bb\vert b\}$;
%   \item
%     $R =\{S\rightarrow SA|\varepsilon,A\rightarrow BSa|B, B\rightarrow b|BS|ab\}$;
%   \item
%     $R = \{S\rightarrow AS|\varepsilon,A\rightarrow SaBB|A, B\rightarrow b|BBbS|AA\}$;
%   \end{enumerate}
% \end{problem}

\begin{example}
  Нека е дадена граматиката $G$ с правила 
  $S\rightarrow ASB\vert \varepsilon, A\rightarrow aAa\vert a, B\rightarrow bBb\vert b$.
  Ще построим стеков автомат $P = \PDA$, такъв че $\L_S(P) = \L(G)$.
  \begin{itemize}
  \item
    $\Sigma = \{a,b\}$;
  \item 
    $\Gamma = \{A,S,B,a,b\}$;
  \item
    $\# = S$;
  \item
    $Q = \{q\}$;
  \item
    $F = \emptyset$;
  \item
    Дефинираме релацията на преходите, следвайки конструкцията от \Th{push-down-context-free}:
    \begin{itemize}
    \item 
      $\Delta(q,\varepsilon, S) = \{\pair{q,ASB}, \pair{q,\varepsilon}\}$;
    \item
      $\Delta(q, \varepsilon, A) = \{\pair{q, aAa}, \pair{q, a}\}$;
    \item
      $\Delta(q, \varepsilon, B) = \{\pair{q, bBb}, \pair{q, b}\}$;
    \item
      $\Delta(q, a, a) = \{\pair{q,\varepsilon}\}$;
    \item
      $\Delta(q, b, b) = \{\pair{q,\varepsilon}\}$.
    \end{itemize}
  \end{itemize}
\end{example}


\begin{thm}
  \marginpar{(стр. 144 от \cite{papadimitriou})}
  Нека $L$ e безконтекстен език и $R$ е регулярен език.
  Тогава тяхното сечение $L \cap R$ е безконтекстен език.
\end{thm}
\begin{proof}
  Нека имаме стеков автомат
  \[\M_1 = \PDAn{1}, \text{ където } \L_F(\M_1) = L,\]
  \marginpar{всъщност няма нужда да е детерминиран}
  и краен детерминиран автомат 
  \[\M_2 = \FAn{2}, \text{ където } \L(\M_2) = R.\]
  Ще определим нов стеков автомат $\M = \PDA$, където
  \begin{itemize}
  \item 
    $Q = Q_1 \times Q_2$;
  \item
    $s = \pair{s_1,s_2}$;
  \item
    $F = F_1 \times F_2$;
  \item 
    Функцията на преходите $\Delta$ е дефинирана както следва:
    \begin{itemize}
    \item 
      \marginpar{симулираме едновременно изчислението и на двата автомата}
      Ако $\Delta_1(q_1, a, b) \ni \pair{r_1,c}$
      и $\delta_2(q_2,a) = r_2$, то
      \[\Delta(\pair{q_1,q_2},a,b) \ni \pair{\pair{r_1,r_2}, c}.\]
    \item
      \marginpar{празен ход на автомата $M_2$}
      Ако $\Delta_1(q_1,\varepsilon,b) \ni \pair{r_1,c}$,
      то за всяко $q_2 \in Q_2$,
      \[\Delta(\pair{q_1,q_2},\varepsilon,b) \ni \pair{\pair{r_1,q_2},c}.\]    
    \item
      \marginpar{\writedown Докажете, че $\L(\M) = \L(\M_1) \cap \L(\M_2)$ !}
      $\Delta$ не съдържа други преходи;
    \end{itemize}
  \end{itemize}
\end{proof}

\begin{example}
  Езикът $L = \{w \in \{a,b,c\}^\star \mid n_a(w) = n_b(w) = n_c(w)\}$ не е безконтекстен.
\end{example}
\begin{proof}
  Да допуснем, че $L$ е безконтекстен език.
  Тогава \[L^\prime = L \cap \L(a^\star b^\star c^\star)\] също е безконтекстен език.
  Но $L^\prime = \{a^nb^nc^n \mid n \in \Nat\}$, за който знаем от Пример \ref{example:anbncn}, че {\em не} е безконтекстен.
  Достигнахме до противоречие. Следователно, $L$ не е безконтекстни език.
\end{proof}



\section*{Библиография}

Основни източници в тази глава са:
\begin{itemize}
\item 
  глава 4 от \cite{hopcroft1}, глави 5, 6 и 7 от \cite{hopcroft2};
\item
  глава 2 от \cite{sipser1};
\item
  глава 3 от \cite{papadimitriou}.
\end{itemize}


% \section{Въпроси}
% Вярно ли е, че:
% \begin{itemize}
% \item
% %  \marginpar{Да} 
%   ако $L$ е безконтекстен език, то езикът $L \cap \{a^{2n}b^{2k}\mid n,k\in\Nat\}$ е безконтекстен ?
% \item
%  % \marginpar{Да}
%   ако $L$ е безкраен безконтекстен език, то съществува безкрайна редица от регулярни езици $L_1,L_2,\dots$,
%   за които $L = \bigcup_{i\in\Nat}L_i$ ?
% \item
%   \marginpar{Не}
%   за всяка безкрайна редица от регулярни езици $L_1,L_2,\dots$, то 
%   езикът $L = \bigcup_{i\in\Nat}L_i$ е безконтекстен ?
% \item
%   %\marginpar{Да}
%   за всеки регулярен език $R$ и всеки безконтекстен език $L$, то $L \cap R$ е безконтекстен ?
% \item
%   за всеки регулярен език $R$ и всеки безконтекстен език $L$, то $L \cup R$ е безконтекстен ?
% \item
%   за всеки регулярен език $R$ и всеки безконтекстен език $L$, то $L \setminus R$ е безконтекстен ?
% \item
%   за всеки регулярен език $R$ и всеки безконтекстен език $L$, то $R \setminus L$ е безконтекстен ?
% \item
%   съществува регулярен език $R$ и безконтекстен език $L$, за които $L \cap R$ не е безконтекстен ?
% \item
%   съществува регулярен език $R$ и нерегулярен, но безконтекстен език $L$, за които $L \cap R$ е регулярен ?
% \item
%   за всеки два нерегулярни, но контекстно-свободни езика $L_1,L_2$, то $L_1\cup L_2$ е регулярен ?
% \item
%   съществуват два нерегулярни, но безконтекстни езика $L_1,L_2$, за които $L_1\setminus L_2$ е регулярен ?
% \item
%   съществуват два нерегулярни, но безконтекстни езика $L_1,L_2$, за които $L_1\cap L_2$ е регулярен ?
% \item
%   съществуват два нерегулярни, но безконтекстни езика $L_1,L_2$, за които $L_1\cup L_2$ е регулярен ?
% \item
%   съществува регулярен език $R$, който може да се представи като $R = L_1 \cup L_2$, където
%   $L_1 \cap L_2 = \emptyset$, $L_1,L_2$ са нерегулярни, но контекстно-свободни ?
% \item
%   езикът $\{a,b\}^\star \setminus \{a^nb^n \mid n\in\Nat\}$ е регулярен ?
% \item
%   езикът $\{a,b\}^\star \setminus \{a^nb^n \mid n\in\Nat\}$ е безконтекстен ?
% \item
%   езикът $\{a,b\}^\star \setminus \{a^nb^{2k+1} \mid n,k\in\Nat\}$ е регулярен ?
% \item
%   езикът $\{a,b\}^\star \setminus \{a^nb^{k} \mid n > k\}$ е регулярен ?
% \item
%   езикът $\{a,b\}^\star \setminus \{a^nbba^{n} \mid n \in \Nat\}$ е регулярен ?
% \item
%   езикът $\{a,b\}^\star \setminus \{a^nb^n \mid n\in\Nat\}$ е безконтекстен ?
% \item
%   езикът $\{a,b,c\}^\star \setminus \{a^nb^mc^k \mid m < n\ \&\ m < k\}$ е безконтекстен ?
% \item
%   \marginpar{Не. $\alpha = b^pa^pbba^p$.}
%   езикът $L = \{uvv^R \mid u,v \in \{a,b\}^\star\ \&\ \abs{u} \leq \abs{v}\}$ е регулярен ?
% \item
%   \marginpar{Да.}
%   езикът $L = \{uvv^R \mid u,v \in \{a,b\}^\star\ \&\ \abs{u} \leq \abs{v}\}$ е безконтекстен ?
% \item
%   съществува алгоритъм, който за даден вход регулярен израз $r$ и безконтекстна граматика $G$
%   проверява дали $\L(r) = \L(G)$?
% \item
%   съществува алгоритъм, който за даден вход регулярен израз $r$ и безконтекстна граматика $G$
%   проверява дали $\L(r) \cap \L(G) = \emptyset$?
% \item
%   съществува алгоритъм, който за даден вход регулярен израз $r$ и безконтекстна граматика $G$
%   проверява дали $\abs{\L(r) \cap \L(G)} < \infty$?
% \item
%   съществува алгоритъм, който за даден вход регулярен израз $r$ и безконтекстна граматика $G$
%   проверява дали $\abs{\L(r) \cap \L(G)} = \infty$?
% \item
%   съществува алгоритъм, който за даден вход регулярен израз $r$, безконтекстна граматика $G$
%   и число $k$, проверява дали $\abs{\L(r) \cap \L(G)} = k$?
% \item
%   съществува алгоритъм, който за даден вход регулярен израз $r$ и безконтекстна граматика $G$
%   проверява дали $\L(r) \setminus \L(G) = \emptyset$?
% \item
%   съществува алгоритъм, който за даден вход регулярен израз $r$ и безконтекстна граматика $G$
%   проверява дали $\L(G) \setminus \L(r) = \emptyset$?
% \item
%   съществува алгоритъм, който за даден вход регулярен израз $r$ и безконтекстна граматика $G$
%   проверява дали $\abs{\L(r) \setminus \L(G)} < \infty$?
% \item
%   съществува алгоритъм, който за даден вход регулярен израз $r$ и безконтекстна граматика $G$
%   проверява дали $\abs{\L(G) \setminus \L(r)} < \infty$?
% \item
%   съществува алгоритъм, който за даден вход регулярен израз $r$ и безконтекстна граматика $G$
%   проверява дали $\abs{\L(r) \setminus \L(G)} = \infty$?
% \item
%   съществува алгоритъм, който за даден вход регулярен израз $r$ и безконтекстна граматика $G$
%   проверява дали $\abs{\L(G) \setminus \L(r)} = \infty$?
% \item
%   съществува алгоритъм, който за даден вход регулярен израз $r$, безконтекстна граматика $G$
%   и число $k$, проверява дали $\abs{\L(r) \setminus \L(G)} = k$?
% \item
%   съществува алгоритъм, който за даден вход регулярен израз $r$, безконтекстна граматика $G$
%   и число $k$, проверява дали $\abs{\L(G) \setminus \L(r)} = k$?
% \end{itemize}

% Нека е дадена безконтекстна граматика $G$ с правила \[S\rightarrow a\vert AB \vert AC, A \rightarrow a, B\rightarrow b, C\rightarrow SB.\]
% Вярно ли е, че ако приложим CYK алгоритъма върху думата $\alpha$, където
% \begin{itemize}
% \item 
%   $\alpha = aabb$, то $N[1,1] = \{S\}$.
% \item 
%   $\alpha = aabb$, то $N[3,3] = \{B\}$.
% \item 
%   $\alpha = aabb$, то $N[1,4] = \{\}$.
% \item
%   $\alpha = baab$, то $N[2,4] = \{\}$.
% \item
%   $\alpha = baab$, то $N[1,3] = \{\}$.
% \end{itemize}



%%% Local Variables: 
%%% mode: latex
%%% TeX-master: "EAI"
%%% End: 
